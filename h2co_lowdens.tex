\section{Non-star-forming, low column-density clouds in absorption}
In \citet{Ginsburg2011a}, we noted that the \formaldehyde densitometer revealed
volume densities much higher than expected given the cloud-average densities
from \thirteenco observations.  The densities were higher even than typical
turbulence.  However, this argument was made on the basis of a statistical argument;
here we attempt to demonstrate that the clumps in GMCs are of very high density
in individual clouds.

In order to detect low-column-density clouds, we must use bright background
illumination sources at 2 and 6 cm, i.e. HII regions.  There are a few dozen of
these within the inner Galactic plane, including the sources observed in
\citet{Ginsburg2011a} and the majority of the bright sources in the BGPS
\citep{Ginsburg2012}.

As an example case-study, we examine G43.17+0.01, also known as W49.  In the
large survey, we observed two lines of sight towards W49, the second at
G43.16-0.03.  Both are very bright continuum sources, and two GMCs are easily
detected in \formaldehyde absorption and \thirteenco emission.  Figure
\ref{fig:w49fullspec} shows the spectrum dominated by W49 itself, but with
clear absorption components.  The continuum level subtracted from the spectra
are 73 K at 6 cm and 11 K at 2 cm for the south component, and 194 K at 6 cm
and 28 K at 2 cm for the north component.

\FigureTwo{figures_chH2CO/G43.17+0.01_H2CO_overplot_gbt9x.png}
{figures_chH2CO/G43.16-0.03_H2CO_overplot_gbt9x.png}
{Spectra of the \formaldehyde \oneone (black), \twotwo (red), and \thirteenco
1-0 (green) lines towards G43.17+0.01 (left) and G43.16-0.03 (right).
The \formaldehyde spectra are shown continuum-subtracted, and the \thirteenco
spectrum is offset by 1 K for clarity.  The GBT \twotwo spectra are multiplied
by a factor of 9 so the smaller lines can be seen.
}{fig:w49fullspec}{1}

We focus on the ``foreground'' lines at $\sim40 \kms$ and $\sim65 \kms$, since
they are not associated with the extremely massive W49 region.  It is difficult
to assess the level of star formation within these clouds, since they lie
directly along the line of sight to W49, but additional \formaldehyde spectra
of the surrounding sources that are bright at 8-1100 \um show that they are all
at the velocity of W49.  

The 40 \kms cloud is observed in its outskirts, not at the peak of the
\thirteenco emission.  The cloud structure is vast, spanning $\sim0.6\degrees$, 
or $\sim60$ pc at $D=2.8$ kpc \citep{Roman-Duval2009}.  It is detected in \oneone
absorption at all 6 locations observed in \formaldehyde, but \twotwo is only
detected in front of the W49 HII region because of the higher signal-to-noise at
that location.  The detected \thirteenco and \formaldehyde lines are fairly
narrow, with \formaldehyde FWHM $\sim1.3$-$2.8$ \kms and \thirteenco widths
from 1.8-5.9 \kms.  The \thirteenco lines are 50\% wider than the \formaldehyde
lines.

The highest \thirteenco contours are observed as a modest IRDC, but no dust
emission peaks are observed at 500 \um or 1.1 mm.  This is an indication that
any star formation, if present, is weak - no clusters are presently forming
from this cloud.   It resembles, in that respect, the California molecular
cloud.  However, it is much smaller, with $M\approx8.3\ee{3}\pm3.2\ee{3} \msun$
compared to California's $\sim10^5$.

\Figure{figures_chH2CO/W49_RGB_40kms_aplpy.png}
{The G43 40 \kms cloud.  The background image shows Herschel SPIRE 70 \um (red),
Spitzer MIPS 24 \um (green), and Spitzer IRAC 8 \um (blue) in the background with
the \thirteenco integrated image from $v=36 \kms$ to $v=43 \kms$ at contour levels of
1, 2, and 3 K superposed in orange contours.  The red and black circles
show the locations of \formaldehyde pointings, and their labels indicate the LSR velocity
of the strongest line in the spectrum.  The W49 HII region is seen
behind some of the faintest \thirteenco emission that is readily associated
with this cloud.  The dark swath in the 8 and 24 \um emission going through the
peak of the \thirteenco emission in the lower half of the image is likely a low
optical depth infrared dark cloud associated with this GMC.}
{fig:40kmscloud}{0.5}{0}

The \formaldehyde densitometer measurements are shown in Figure \ref{fig:h2codensg43}.
The figures show optical depth spectra, given by the equation
$$\tau = -\log\left(\frac{S_\nu + 2.73}{\bar{S_\nu} + 2.73}\right)$$
where $S_\nu$ is the spectrum (with continuum included) and $\bar{S_\nu}$ is
the measured continuum.

\FigureTwo{figures_chH2CO/G43.16-0.03_40kms_h2codensfit.png}
{figures_chH2CO/G43.17+0.01_40kms_h2codensfit.png}
{Optical depth spectra of the \oneone and \twotwo lines towards the two W49 lines
of sight.  The fitted parameters, along with the statistical 1-$\sigma$ errors,
are shown in the legend.}
{fig:h2codensg43}{1}

The density measurements are very precise, with $n\approx2.11\times10^4 \pm
0.17\ee{4}$ \percc and $n\approx 1.98\times10^4 \pm 0.32\ee{4}$ \percc for
G43.17+0.01 and G43.16-0.03 respectively.  At this level of precision, the 
density measurements are dominated by systematics - especially gas temperature
and collision rate uncertainties - which limit the accuracy to $\sim50\%$
\citep{Zeiger2010}.  Nonetheless, the density is much higher than the
\thirteenco-measured cloud-average density $n\approx 400$ \percc
\citep[for cloud GRSMC\_G043.04-00.11;][]{Roman-Duval2010a}, with
$n_{\formaldehyde}/n_{\thirteenco} \approx 50$.  

Since the W49 line of sight is clearly on the outskirts of the cloud, not
through its core, such a high density is unlikely to be an indication that
this line of sight corresponds to a centrally condensed density peak.  Using
Figure 4 of \citet{Ginsburg2011a}, we can `turbulence-correct' the density
measurements, but even in the most extreme case with a turbulent density
distribution lognormal width $\sigma_s = 1.5$, the correction is only a factor
of 2.5, reducing the discrepancy to a factor of $\sim20$.

% We should then ask, if there is gas at high density, how much is at this density?
% To address this question, we'll assume that the densities in all of the \formaldehyde
% lines of sight in the cloud are the same, and compare the \thirteenco and
% \formaldehyde derived column densities.  The \oneone line robustly reflects the
% total \formaldehyde column, even though it does not constrain the density
% without a corresponding \twotwo detection.

Comparing the integrated \formaldehyde lines to the integrated \thirteenco
lines, the integrated \formaldehyde column densities are
$N_{\ortho} = 2.03\ee{12} $ and $1.56\ee{12}$ \persc for G43.16
and G43.17 respectively.
The \thirteenco integrated spectra have brightness $T_{MB} = 2.6$ K and $1.3$ K
for G43.16 and G43.17 respectively.  Using the cloud-averaged excitation
temperature for this cloud, $\tau_{13}=2.3$ and $0.6$ respectively, so
\citet{Roman-Duval2010a} equation 3 yields column densities $N_{13} = 6.2\ee{15}
$ and $1.6\ee{15}$ \percc respectively.  Assuming an abundance relative to \hh
$X_{13} = 1.8\ee{-6}$ \citep[consistent with ][]{Roman-Duval2010a}, the
resulting \hh column densities are 3.5\ee{21} and 9.0 \ee{20} \percc
respectively.  The abundances of \ortho relative to \thirteenco are 3.2\ee{-4}
and 9.8\ee{-4} respectively, or relative to \hh, 5.8\ee{-10} and 1.7\ee{-9},
which are entirely consistent with other measurements of $X_{\ortho}$.  These
are relatively modest column densities, with $A_V=17$ and 4.5.

These measurements for a specific cloud validate the statistical argument made
in \citet{Ginsburg2011a}.  However, upon closer inspection of the cloud
morphology, the real explanation may be simple: the filling factor of gas
within the GMC is small on large scales, not local scales.  The implied volume
filling factor from this analysis and the \citet{Ginsburg2011a} analysis is
$\sim10^{-2}$; the assumption of spherical symmetry is therefore extremely
poor.  

This low filling factor has major implications for the gas: if it is in
gravitational collapse, the free-fall times are shorter by an order of
magnitude than usually assumed.  The long lifetimes of GMCs therefore implies
that the cloud cannot be undergoing gravitational collapse, but instead
maintains a turbulent equilibrium.
