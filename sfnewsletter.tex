%%%%%%%%%%%%%%%%%%%%%%%%%%%%%%%%%%%%%%%%%%%%%%%%%%%%%%%%%%%%%%%%%%%%%%%%%%%%%
%%%
%%%               LaTeX MACRO FOR DISSERTATION ABSTRACTS
%%%
%%%    Please use the following macro for your thesis abstract. You
%%%    have one full page for everything, and you are very welcome to
%%%    go into detail with your results, so the readers get a
%%%    comprehensive overview of your work. Merely fill in the
%%%    brackets below and mail to reipurth@ifa.hawaii.edu. If you
%%%    have problems, let me know in an accompanying note and I will fix them.
%%%
%%%
%%%%%%%%%%%%%%%%%%%%%%%%%%%%%%%%%%%%%%%%%%%%%%%%%%%%%%%%%%%%%%%%%%%%%%%%%%%%%

\documentstyle{article}
\textwidth 18cm
\textheight 23cm
\oddsidemargin -1cm
\topmargin 0cm
\parskip 0.15cm
\parindent 0pt
\small

\begin{document}
\begin{center}

%% If you use any personal Latex commands in your abstract, please include
%% their definitions here.

%% Between these brackets you write the title of your thesis:
{\Large\bf{Surveying Star Formation in the Galaxy}}

\vspace*{0.5cm}

%% Here comes your name
{\bf{ Adam Ginsburg }}

%% Here you write the institute where your thesis work was conducted, e.g.:
{Thesis work conducted at: Center for Astrophysics and Space Astronomy, University of Colorado, Boulder, Colorado, USA}

%% Here comes your present postal address (if you are about to move and know
%% your coming address give it as well) e.g.:
{Current address: 
391 UCB
Boulder, CO, USA 80309}

%% (if you use this part, remove %%)
{Address as of October 2013:
Karl-Schwarzschild-Straße 2, 
85748 Garching bei München, Germany
}

%% Here comes your e-mail address:
{Electronic mail: adam.g.ginsburg@gmail.com}

%% Name of your adviser:
{Ph.D dissertation directed by: John Bally}

%% Month and Year of thesis:
{Ph.D degree awarded: April 2013}

\vspace*{0.8cm}

\end{center}

%% Within the following brackets you place your text:
{
    I studied the formation of massive stars and clusters via millimeter,
    radio, and infrared observations.  The Bolocam Galactic Plane Survey (BGPS)
    was the first millimeter-wave blind survey of the plane of our Galaxy.  I
    wrote the data reduction pipeline for this survey and produced the final
    publicly released data products.  I ran extensive tests of the pipeline,
    using simulations to probe its performance.

    The BGPS detected over 8000 1.1 mm sources, the largest sample at this
    wavelength ever detected.  As a single-wavelength continuum survey, the
    BGPS serves as a finder chart for millimeter and radio observations.  
    I therefore performed follow-up surveys of BGPS sources in CO 3-2 and
    \formaldehyde, and others did similar follow-ups to measure velocities
    and distances towards these sources.

    \formaldehyde observations of ultracompact HII regions and other
    millimeter-bright sources were used to measure the local molecular gas
    density.  These measurements hint that density within molecular clouds does
    not follow a simple lognormal distribution.  They also show that
    star-forming clouds all contain gas at density $\gtrsim10^4$ \percc.

    I used the BGPS source catalog to identify the most massive compact clumps
    within the galaxy, identifying 18 with masses $M>10^4$ \msun in the first
    quadrant of the Galactic plane.   As these objects are all actively
    star-forming, the starless timescale of massive proto-cluster clumps must
    be relatively short, with lifetimes $\lesssim0.6$ Myr.
}


\end{document}

