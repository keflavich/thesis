%\documentclass[defaultstyle,11pt]{thesis}
%\documentclass[]{report}
%\documentclass[]{article}
%\usepackage{aastex_hack}
%\usepackage{deluxetable}
\documentclass[preprint]{aastex}


%%%%%%%%%%%%%%%%%%%%%%%%%%%%%%%%%%%%%%%%%%%%%%%%%%%%%%%%%%%%%%%%
%%%%%%%%%%%  see documentation for information about  %%%%%%%%%%
%%%%%%%%%%%  the options (11pt, defaultstyle, etc.)   %%%%%%%%%%
%%%%%%%  http://www.colorado.edu/its/docs/latex/thesis/  %%%%%%%
%%%%%%%%%%%%%%%%%%%%%%%%%%%%%%%%%%%%%%%%%%%%%%%%%%%%%%%%%%%%%%%%
%		\documentclass[typewriterstyle]{thesis}
% 		\documentclass[modernstyle]{thesis}
% 		\documentclass[modernstyle,11pt]{thesis}
%	 	\documentclass[modernstyle,12pt]{thesis}

%%%%%%%%%%%%%%%%%%%%%%%%%%%%%%%%%%%%%%%%%%%%%%%%%%%%%%%%%%%%%%%%
%%%%%%%%%%%    load any packages which are needed    %%%%%%%%%%%
%%%%%%%%%%%%%%%%%%%%%%%%%%%%%%%%%%%%%%%%%%%%%%%%%%%%%%%%%%%%%%%%
\usepackage{latexsym}		% to get LASY symbols
\usepackage{graphicx}		% to insert PostScript figures
%\usepackage{deluxetable}
\usepackage{rotating}		% for sideways tables/figures
\usepackage{natbib}  % Requires natbib.sty, available from http://ads.harvard.edu/pubs/bibtex/astronat/
\usepackage{savesym}
\usepackage{amssymb}
%\savesymbol{singlespace}
\savesymbol{doublespace}
%\usepackage{wrapfig}
%\usepackage{setspace}
\usepackage{xspace}
\usepackage{color}
\usepackage{multicol}
\usepackage{mdframed}
\usepackage{url}
\usepackage{subfigure}
%\usepackage{emulateapj}
\usepackage{lscape}
\usepackage{grffile}
\usepackage{standalone}
\standalonetrue
\usepackage{import}
\usepackage[utf8]{inputenc}
\usepackage{longtable}
\usepackage{booktabs}



%%%%%%%%%%%%%%%%%%%%%%%%%%%%%%%%%%%%%%%%%%%%%%%%%%%%%%%%%%%%%%%%
%%%%%%%%%%%%       all the preamble material:       %%%%%%%%%%%%
%%%%%%%%%%%%%%%%%%%%%%%%%%%%%%%%%%%%%%%%%%%%%%%%%%%%%%%%%%%%%%%%

% \title{Star Formation in the Galaxy}
% 
% \author{Adam G.}{Ginsburg}
% 
% \otherdegrees{B.S., Rice University, 2007\\
% 	      M.S., University of Colorado, Boulder, 2009}
% 
% \degree{Doctor of Philosophy}		%  #1 {long descr.}
% 	{Ph.D., Rocket Science (ok, fine, astrophysics)}		%  #2 {short descr.}
% 
% \dept{Department of}			%  #1 {designation}
% 	{Astrophysical and Planetary Sciences}		%  #2 {name}
% 
% \advisor{Prof.}				%  #1 {title}
% 	{John Bally}			%  #2 {name}
% 
% \reader{Prof.~Jeremy Darling}		%  2nd person to sign thesis
% \readerThree{Prof.~Jason Glenn}		%  3rd person to sign thesis
% \readerFour{Prof.~Michael Shull}	%  4rd person to sign thesis
% \readerFour{Prof.~Neal Evans}	%  4rd person to sign thesis
% 
% \abstract{  \OnePageChapter	% one page only ??
% 
%     I discovered dust in space.  
% 
% 	}
% 
% 
% \dedication[Dedication]{	% NEVER use \OnePageChapter here.
% 	To 1, the second number in binary.
% 	}
% 
% \acknowledgements{	\OnePageChapter	% *MUST* BE ONLY ONE PAGE!
% 	All y'all.
% 	}
% 
% \ToCisShort	% a 1-page Table of Contents ??
% 
% \LoFisShort	% a 1-page List of Figures ??
% %	\emptyLoF	% no List of Figures at all ??
% 
% \LoTisShort	% a 1-page List of Tables ??
% %	\emptyLoT	% no List of Tables at all ??
% 
% 
% %%%%%%%%%%%%%%%%%%%%%%%%%%%%%%%%%%%%%%%%%%%%%%%%%%%%%%%%%%%%%%%%%
% %%%%%%%%%%%%%%%       BEGIN DOCUMENT...         %%%%%%%%%%%%%%%%%
% %%%%%%%%%%%%%%%%%%%%%%%%%%%%%%%%%%%%%%%%%%%%%%%%%%%%%%%%%%%%%%%%%
% 
% %%%%  footnote style; default=\arabic  (numbered 1,2,3...)
% %%%%  others:  \roman, \Roman, \alph, \Alph, \fnsymbol
% %	"\fnsymbol" uses asterisk, dagger, double-dagger, etc.
% %	\renewcommand{\thefootnote}{\fnsymbol{footnote}}
% %	\setcounter{footnote}{0}

\newcommand{\paa}{Pa\ensuremath{\alpha}}
\newcommand{\brg}{Br\ensuremath{\gamma}}
\newcommand{\msun}{\ensuremath{M_{\odot}}}			%  Msun
\newcommand{\mdot}{\ensuremath{\dot{M}}\xspace}
\newcommand{\lsun}{\ensuremath{L_{\odot}}}			%  Lsun
\newcommand{\lbol}{\ensuremath{L_{\mathrm{bol}}}}	%  Lbol
\newcommand{\ks}{K\ensuremath{_{\mathrm{s}}}}		%  Ks
\newcommand{\hh}{\ensuremath{\textrm{H}_{2}}\xspace}			%  H2
\newcommand{\formaldehyde}{\ensuremath{\textrm{H}_2\textrm{CO}}\xspace}
\newcommand{\formaldehydeIso}{\ensuremath{\textrm{H}_2~^{13}\textrm{CO}}\xspace}
\newcommand{\methanol}{\ensuremath{\textrm{CH}_3\textrm{OH}}\xspace}
\newcommand{\ortho}{\ensuremath{\textrm{o-H}_2\textrm{CO}}}
\newcommand{\oneone}{\ensuremath{1_{10}-1_{11}}\xspace}
\newcommand{\twotwo}{\ensuremath{2_{11}-2_{12}}\xspace}
\newcommand{\threethree}{\ensuremath{3_{12}-3_{13}}\xspace}
\newcommand{\threeohthree}{\ensuremath{3_{03}-2_{02}}\xspace}
\newcommand{\threetwotwo}{\ensuremath{3_{22}-2_{21}}\xspace}
\newcommand{\threetwoone}{\ensuremath{3_{21}-2_{20}}\xspace}
\newcommand{\JKaKc}{\ensuremath{J_{K_a K_c}}}
\newcommand{\water}{H$_{2}$O}		%  H2O
\newcommand{\feii}{\ion{Fe}{2}}		%  FeII
\newcommand{\uchii}{UC\ion{H}{2}\xspace}
\newcommand{\UCHII}{UC\ion{H}{2}\xspace}
\newcommand{\hii}{H~{\sc ii}\xspace}
\newcommand{\Hii}{H~{\sc ii}\xspace}
\newcommand{\HII}{H~{\sc ii}\xspace}
\newcommand{\kms}{\textrm{km~s}\ensuremath{^{-1}}\xspace}	%  km s-1
\newcommand{\nsample}{456\xspace}
\newcommand{\CFR}{5\xspace} % nMPC / 0.25 / 2 (6 for W51 once, 8 for W51 twice) REFEDIT: With f_observed=0.3, becomes 3/2./0.3 = 5
\newcommand{\permyr}{\ensuremath{\mathrm{Myr}^{-1}}\xspace}
\newcommand{\tsuplim}{0.5\xspace} % upper limit on starless timescale
\newcommand{\ncandidates}{18\xspace}
\newcommand{\mindist}{8.7\xspace}
\newcommand{\rcluster}{2.5\xspace}
\newcommand{\ncomplete}{13\xspace}
\newcommand{\middistcut}{13.0\xspace}
\newcommand{\nMPC}{3\xspace} % only count W51 once.  W51, W49, G010
\newcommand{\obsfrac}{30}
\newcommand{\nMPCtot}{10\xspace} % = nmpc / obsfrac
\newcommand{\nMPCtoterr}{6\xspace} % = sqrt(nmpc) / obsfrac
\newcommand{\plaw}{2.1\xspace}
\newcommand{\plawerr}{0.3\xspace}
\newcommand{\mmin}{\ensuremath{10^4~\msun}\xspace}
%\newcommand{\perkmspc}{\textrm{per~km~s}\ensuremath{^{-1}}\textrm{pc}\ensuremath{^{-1}}\xspace}	%  km s-1 pc-1
\newcommand{\kmspc}{\textrm{km~s}\ensuremath{^{-1}}\textrm{pc}\ensuremath{^{-1}}\xspace}	%  km s-1 pc-1
\newcommand{\sqcm}{cm$^{2}$\xspace}		%  cm^2
\newcommand{\percc}{\ensuremath{\textrm{cm}^{-3}}\xspace}
\newcommand{\persc}{\ensuremath{\textrm{cm}^{-2}}\xspace}
\newcommand{\persr}{\ensuremath{\textrm{sr}^{-1}}\xspace}
\newcommand{\peryr}{\ensuremath{\textrm{yr}^{-1}}\xspace}
\newcommand{\perkmspc}{\textrm{per~km~s}\ensuremath{^{-1}}\textrm{pc}\ensuremath{^{-1}}\xspace}	%  km s-1 pc-1
\newcommand{\perkms}{\textrm{per~km~s}\ensuremath{^{-1}}\xspace}	%  km s-1 
\newcommand{\um}{\ensuremath{\mu m}\xspace}    % micron
\newcommand{\mum}{$\mu$m}
\newcommand{\htwo}{\ensuremath{\textrm{H}_2}}    % micron
\newcommand{\Htwo}{\ensuremath{\textrm{H}_2}}    % micron
\newcommand{\HtwoO}{\ensuremath{\textrm{H}_2\textrm{O}}}    % micron
\newcommand{\htwoo}{\ensuremath{\textrm{H}_2\textrm{O}}}    % micron
\newcommand{\ha}{\ensuremath{\textrm{H}\alpha}}
\newcommand{\hb}{\ensuremath{\textrm{H}\beta}}
%\newcommand{\so}{ SO~(5~6)-(4~5) }
\newcommand{\regone}{Sh~2-201}
\newcommand{\regtwo}{AFGL~4029}
\newcommand{\regthree}{LW Cas Nebula}
\newcommand{\regfour}{IC 1848}
\newcommand{\regfive}{W5 NW}
\newcommand{\regsix}{SFO 11}
\newcommand{\so}{ SO~\ensuremath{5_6-4_5} }
\newcommand{\SO}{ SO~\ensuremath{1_2-1_1} }
\newcommand{\ammonia}{NH\ensuremath{_3}\xspace}
\newcommand{\twelveco}{\ensuremath{^{12}\textrm{CO}}}
\newcommand{\thirteenco}{\ensuremath{^{13}\textrm{CO}}}
\newcommand{\ceighteeno}{\ensuremath{\textrm{C}^{18}\textrm{O}}}
\def\ee#1{\ensuremath{\times10^{#1}}}
\newcommand{\degrees}{\ensuremath{^{\circ}}}
\newcommand{\lowirac}{800}
\newcommand{\highirac}{8000}
\newcommand{\lowmips}{600}
\newcommand{\highmips}{5000}
\newcommand{\perbeam}{\ensuremath{\textrm{beam}^{-1}}}
\newcommand{\ds}{\ensuremath{\textrm{d}s}}
\newcommand{\dnu}{\ensuremath{\textrm{d}\nu}}
\newcommand{\dv}{\ensuremath{\textrm{d}v}}
\def\secref#1{Section \ref{#1}}
\def\eqref#1{Equation \ref{#1}}
%\newcommand{\arcmin}{'}

\newcommand{\necluster}{Sh~2-233IR~NE}
\newcommand{\swcluster}{Sh~2-233IR~SW}
\newcommand{\region}{IRAS 05358}

\newcommand{\nwfive}{40}
\newcommand{\nouter}{15}

\newcommand{\vone}{{\rm v}1.0\xspace}
\newcommand{\vtwo}{{\rm v}2.0\xspace}
\newcommand\mjysr{\ensuremath{{\rm MJy~sr}^{-1}}}
\newcommand\jybm{\ensuremath{{\rm Jy~bm}^{-1}}}
\newcommand\nbolocat{8552\xspace}
\newcommand\nbolocatnew{548\xspace}
\newcommand\nbolocatnonew{8004\xspace} % = nbolocat-nbolocatnew
\renewcommand\arcdeg{\mbox{$^\circ$}\xspace} 
\renewcommand\arcmin{\mbox{$^\prime$}\xspace} 
\renewcommand\arcsec{\mbox{$^{\prime\prime}$}\xspace} 

\newcommand{\todo}[1]{\textcolor{red}{#1}}
\newcommand{\okinfinal}[1]{{#1}}
\newcommand{\keywords}[1]{}
\newcommand{\email}[1]{}
\newcommand{\affil}[1]{}


%aastex hack
%\newcommand\arcdeg{\mbox{$^\circ$}}%
%\newcommand\arcmin{\mbox{$^\prime$}\xspace}%
%\newcommand\arcsec{\mbox{$^{\prime\prime}$}\xspace}%

%\newcommand\epsscale[1]{\gdef\eps@scaling{#1}}
%
%\newcommand\plotone[1]{%
% \typeout{Plotone included the file #1}
% \centering
% \leavevmode
% \includegraphics[width={\eps@scaling\columnwidth}]{#1}%
%}%
%\newcommand\plottwo[2]{{%
% \typeout{Plottwo included the files #1 #2}
% \centering
% \leavevmode
% \columnwidth=.45\columnwidth
% \includegraphics[width={\eps@scaling\columnwidth}]{#1}%
% \hfil
% \includegraphics[width={\eps@scaling\columnwidth}]{#2}%
%}}%


%\newcommand\farcm{\mbox{$.\mkern-4mu^\prime$}}%
%\let\farcm\farcm
%\newcommand\farcs{\mbox{$.\!\!^{\prime\prime}$}}%
%\let\farcs\farcs
%\newcommand\fp{\mbox{$.\!\!^{\scriptscriptstyle\mathrm p}$}}%
%\newcommand\micron{\mbox{$\mu$m}}%
%\def\farcm{%
% \mbox{.\kern -0.7ex\raisebox{.9ex}{\scriptsize$\prime$}}%
%}%
%\def\farcs{%
% \mbox{%
%  \kern  0.13ex.%
%  \kern -0.95ex\raisebox{.9ex}{\scriptsize$\prime\prime$}%
%  \kern -0.1ex%
% }%
%}%

\def\Figure#1#2#3#4#5{
\begin{figure*}[htp]
\includegraphics[scale=#4,angle=#5]{#1}
\caption{#2}
\label{#3}
\end{figure*}
}

% originally intended to be included in a two-column paper
% this is in includegraphics: ,width=3in
% but, not for thesis
\def\OneColFigure#1#2#3#4#5{
\begin{figure}[htpb]
\epsscale{#4}
\includegraphics[scale=#4,angle=#5]{#1}
\caption{#2}
\label{#3}
\end{figure}
}

\def\SubFigure#1#2#3#4#5{
\begin{figure*}[htp]
\addtocounter{figure}{-1}
\epsscale{#4}
\includegraphics[angle=#5]{#1}
\caption{#2}
\label{#3}
\end{figure*}
}

\def\FigureTwo#1#2#3#4#5{
\begin{figure*}[htp]
\epsscale{#5}
\plottwo{#1}{#2}
\caption{#3}
\label{#4}
\end{figure*}
}

\def\TallFigureTwo#1#2#3#4#5#6{
    \FigureTwo{#1}{#2}{#3}{#4}{#5}
    }

\def\SubFigureTwo#1#2#3#4#5{
\begin{figure*}[htp]
\addtocounter{figure}{-1}
\epsscale{#5}
\plottwo{#1}{#2}
\caption{#3}
\label{#4}
\end{figure*}
}

\def\FigureFour#1#2#3#4#5#6{
\begin{figure*}[htp]
\subfigure[]{ \includegraphics[width=3in,type=png,ext=.png,read=.png]{#1} }
\subfigure[]{ \includegraphics[width=3in,type=png,ext=.png,read=.png]{#2} }
\subfigure[]{ \includegraphics[width=3in,type=png,ext=.png,read=.png]{#3} }
\subfigure[]{ \includegraphics[width=3in,type=png,ext=.png,read=.png]{#4} }
\caption{#5}
\label{#6}
\end{figure*}
}

\def\Table#1#2#3#4#5#6{
\begin{deluxetable}{#1}
\tablewidth{0pt}
\tabletypesize{\footnotesize}
\tablecaption{#2}
\tablehead{#3}
\startdata
\label{#4}
#5
\enddata
\bigskip
#6
\end{deluxetable}
}

		% file containing author's macro definitions

\begin{document}
% %\documentclass[defaultstyle,11pt]{thesis}
%\documentclass[]{report}
%\documentclass[]{article}
%\usepackage{aastex_hack}
%\usepackage{deluxetable}
\documentclass[preprint]{aastex}


%%%%%%%%%%%%%%%%%%%%%%%%%%%%%%%%%%%%%%%%%%%%%%%%%%%%%%%%%%%%%%%%
%%%%%%%%%%%  see documentation for information about  %%%%%%%%%%
%%%%%%%%%%%  the options (11pt, defaultstyle, etc.)   %%%%%%%%%%
%%%%%%%  http://www.colorado.edu/its/docs/latex/thesis/  %%%%%%%
%%%%%%%%%%%%%%%%%%%%%%%%%%%%%%%%%%%%%%%%%%%%%%%%%%%%%%%%%%%%%%%%
%		\documentclass[typewriterstyle]{thesis}
% 		\documentclass[modernstyle]{thesis}
% 		\documentclass[modernstyle,11pt]{thesis}
%	 	\documentclass[modernstyle,12pt]{thesis}

%%%%%%%%%%%%%%%%%%%%%%%%%%%%%%%%%%%%%%%%%%%%%%%%%%%%%%%%%%%%%%%%
%%%%%%%%%%%    load any packages which are needed    %%%%%%%%%%%
%%%%%%%%%%%%%%%%%%%%%%%%%%%%%%%%%%%%%%%%%%%%%%%%%%%%%%%%%%%%%%%%
\usepackage{latexsym}		% to get LASY symbols
\usepackage{graphicx}		% to insert PostScript figures
%\usepackage{deluxetable}
\usepackage{rotating}		% for sideways tables/figures
\usepackage{natbib}  % Requires natbib.sty, available from http://ads.harvard.edu/pubs/bibtex/astronat/
\usepackage{savesym}
\usepackage{amssymb}
%\savesymbol{singlespace}
\savesymbol{doublespace}
%\usepackage{wrapfig}
%\usepackage{setspace}
\usepackage{xspace}
\usepackage{color}
\usepackage{multicol}
\usepackage{mdframed}
\usepackage{url}
\usepackage{subfigure}
%\usepackage{emulateapj}
\usepackage{lscape}
\usepackage{grffile}
\usepackage{standalone}
\standalonetrue
\usepackage{import}
\usepackage[utf8]{inputenc}
\usepackage{longtable}
\usepackage{booktabs}



%%%%%%%%%%%%%%%%%%%%%%%%%%%%%%%%%%%%%%%%%%%%%%%%%%%%%%%%%%%%%%%%
%%%%%%%%%%%%       all the preamble material:       %%%%%%%%%%%%
%%%%%%%%%%%%%%%%%%%%%%%%%%%%%%%%%%%%%%%%%%%%%%%%%%%%%%%%%%%%%%%%

% \title{Star Formation in the Galaxy}
% 
% \author{Adam G.}{Ginsburg}
% 
% \otherdegrees{B.S., Rice University, 2007\\
% 	      M.S., University of Colorado, Boulder, 2009}
% 
% \degree{Doctor of Philosophy}		%  #1 {long descr.}
% 	{Ph.D., Rocket Science (ok, fine, astrophysics)}		%  #2 {short descr.}
% 
% \dept{Department of}			%  #1 {designation}
% 	{Astrophysical and Planetary Sciences}		%  #2 {name}
% 
% \advisor{Prof.}				%  #1 {title}
% 	{John Bally}			%  #2 {name}
% 
% \reader{Prof.~Jeremy Darling}		%  2nd person to sign thesis
% \readerThree{Prof.~Jason Glenn}		%  3rd person to sign thesis
% \readerFour{Prof.~Michael Shull}	%  4rd person to sign thesis
% \readerFour{Prof.~Neal Evans}	%  4rd person to sign thesis
% 
% \abstract{  \OnePageChapter	% one page only ??
% 
%     I discovered dust in space.  
% 
% 	}
% 
% 
% \dedication[Dedication]{	% NEVER use \OnePageChapter here.
% 	To 1, the second number in binary.
% 	}
% 
% \acknowledgements{	\OnePageChapter	% *MUST* BE ONLY ONE PAGE!
% 	All y'all.
% 	}
% 
% \ToCisShort	% a 1-page Table of Contents ??
% 
% \LoFisShort	% a 1-page List of Figures ??
% %	\emptyLoF	% no List of Figures at all ??
% 
% \LoTisShort	% a 1-page List of Tables ??
% %	\emptyLoT	% no List of Tables at all ??
% 
% 
% %%%%%%%%%%%%%%%%%%%%%%%%%%%%%%%%%%%%%%%%%%%%%%%%%%%%%%%%%%%%%%%%%
% %%%%%%%%%%%%%%%       BEGIN DOCUMENT...         %%%%%%%%%%%%%%%%%
% %%%%%%%%%%%%%%%%%%%%%%%%%%%%%%%%%%%%%%%%%%%%%%%%%%%%%%%%%%%%%%%%%
% 
% %%%%  footnote style; default=\arabic  (numbered 1,2,3...)
% %%%%  others:  \roman, \Roman, \alph, \Alph, \fnsymbol
% %	"\fnsymbol" uses asterisk, dagger, double-dagger, etc.
% %	\renewcommand{\thefootnote}{\fnsymbol{footnote}}
% %	\setcounter{footnote}{0}

\input{macros}		% file containing author's macro definitions

\begin{document}
% \input{introduction}
% 
% %\input{ch_iras05358}
% \input{ch_w5}
% \input{ch_h2co}
% \input{ch_h2colarge}
% \input{ch_boundhii}
% 
% %\input ch2.tex			% file with Chapter 2 contents
% 
% %%%%%%%%%%%%%%%%%%%%%%%%%%%%%%%%%%%%%%%%%%%%%%%%%%%%%%%%%%%%%%%%%%%
% %%%%%%%%%%%%%%%%%%%%%%%  Bibliography %%%%%%%%%%%%%%%%%%%%%%%%%%%%%
% %%%%%%%%%%%%%%%%%%%%%%%%%%%%%%%%%%%%%%%%%%%%%%%%%%%%%%%%%%%%%%%%%%%
% 
% \bibliographystyle{plain}	% or "siam", or "alpha", or "abbrv"
% 				% see other styles (.bst files) in
% 				% $TEXHOME/texmf/bibtex/bst
% 
% \nocite{*}		% list all refs in database, cited or not.
% 
% \bibliography{thesis}		% bib database file refs.bib
% 
% %%%%%%%%%%%%%%%%%%%%%%%%%%%%%%%%%%%%%%%%%%%%%%%%%%%%%%%%%%%%%%%%%%%
% %%%%%%%%%%%%%%%%%%%%%%%%  Appendices %%%%%%%%%%%%%%%%%%%%%%%%%%%%%%
% %%%%%%%%%%%%%%%%%%%%%%%%%%%%%%%%%%%%%%%%%%%%%%%%%%%%%%%%%%%%%%%%%%%
% 
% \appendix	% don't forget this line if you have appendices!
% 
% %\input appA.tex			% file with Appendix A contents
% %\input appB.tex			% file with Appendix B contents
% 
% %%%%%%%%%%%%%%%%%%%%%%%%%%%%%%%%%%%%%%%%%%%%%%%%%%%%%%%%%%%%%%%%%%%
% %%%%%%%%%%%%%%%%%%%%%%%%   THE END   %%%%%%%%%%%%%%%%%%%%%%%%%%%%%%
% %%%%%%%%%%%%%%%%%%%%%%%%%%%%%%%%%%%%%%%%%%%%%%%%%%%%%%%%%%%%%%%%%%%
% 
% \end{document}
% 
% 

\chapter{Introduction}
\section{Preface}
This thesis\footnote{Necessarily the first two words of a thesis?} describes
the research I have performed with a wide variety of collaborators, mostly
centered on the Bolocam Galactic Plane Survey team led by John Bally and Jason
Glenn.  The BGPS data reduction process, at the core of this work, was done in 
collaboration with James Aguirre and Erik Rosolowsky.

However, the work proceeded somewhat haphazardly: I came into the BGPS team as
the rare student enthusiastic about data reduction.  I never planned to take
over the BGPS data, but it happened a few years into my time at CU.  This
thesis is therefore somewhat scattered: some of the observations reported here
were taken as `follow-up' to the BGPS before it was completed.

This document primarily consists of a number of published papers centered
around a common theme of radio and millimeter observations of the Galaxy, but
without an obvious common driving question.  I have therefore added
thesis-specific introductions to each section to describe where they fit in to
the bigger picture of this document.  I've also included sections describing
work that is not yet published but (hopefully) soon will be.

\section{Star Formation in the Galaxy}
It has been known for at least half a century that stars form from the
gravitational collapse of clouds of cool material.  The gas that will
eventually form stars is typically observed as dark features obscuring
background stars.  The brighter nebulae, which have been studied for far longer
\citep{Messier1764}, contain hot and diffuse gas.  These nebulae, while
spectacular, are not the construction materials of new stars.  However,
they mark the locations where new stars have formed - nebulae are often
stellar nurseries.

To track down the cool star-forming material, it is necessary to observe at
longer wavelengths.  Infrared observations can pierce through obscuring
material, as dust becomes more transparent at longer wavelengths.  With near-
and mid-infrared observations, such as those enabled by HgCdTe detectors like
those in the NICFPS and TripleSpec instruments at Apache Point Observatory and
the InSb detectors used on the Spitzer Space Telescope, it is possible to
observe obscured young stars.  These objects have just ignited fusion in their
cores and represent the youngest generation of new stars.

But this material has already formed stars.  To see the truly cold stuff, that
which still has potential to form new stars, we need to examine gas that is not
heated at all by stars.  Assuming we want to look for gas that can form a star
like our sun and that the density of the gas to form is $\sim10^4$ particles
per cubic centimeter (an assumption left unjustified for now), the Jeans scale
requires a temperature $T\sim10$ K, which means we need to look at wavelengths
$\lambda \gtrsim 100 \um$ in order to observe this gas.

Gas at these densities turns out to be quite rare.  While there are thousands
of stars within 100 pc of the sun, the closest known star-forming globules are
at distances greater than 100 pc.  While this sparsity is explained in part by
our current position in the Galaxy (we're buzzing along its outskirts at 250
\kms), it reflects the reality that star formation in the present epoch is
dispersed and rare.

Even more rare are the massive stars that end their lives in supernovae.  While there
are hundreds of stellar nurseries within a few hundred parsecs, the nearest
region of massive star formation is the Orion Molecular Cloud at a distance of
400 pc.  Out to 1000 pc, though, there are still only a handful of massive star
forming regions, including Monoceros R2 and Cepheus A.

These massive stars in many ways are the most important to study in order to
understand the evolution of gas and dust in the universe and our own origins.
In their deaths, they produce the heavy elements required to form dust,
planets, and life.  Throughout their lives and deaths, massive stars dump
energy into the interstellar medium and effectively control the motions and
future of the gas around them.

The bigger the star, the shorter it lives, so massive stars are nearly as rare
as their birth regions.  They also tend to be found nearby or within these
birth regions.  Since they can be found near large globs of dust, finding these
globs can help us discover new groups of massive stars.

This thesis summarizes surveys within our Galaxy to discover and examine
regions forming.  The largest body of work described here is the Bolocam
Galactic Plane Survey, the first dust continuum survey of a significant
fraction of the Galactic Plane.

With that broad overview in place, the next sections describe a few of the
specific problems addressed in this thesis in greater detail.

\subsection{Turbulence}
Turbulence is one of the defining features of the interstellar medium.
Turbulence is thought to govern many properties of the ISM, rendering it
scale-free and defining the distribution of velocities, densities,
temperatures, and magnetic fields in the gas between stars.

Turbulence is most easily modeled by a Kolmogorov spectrum, in which $\Delta v
\propto \ell^{1/3}$, i.e. the typical velocity dispersion is greatest at the
largest size scales.  Kolmogorov turbulence strictly only describes
incompressible fluids without magnetic fields, while the ISM is compressible
and threaded by magnetic fields.  Nonetheless, Kolmogorov turbulence is nearly
consistent with some observed properties of the ISM.  The Larson size-linewidth
relation, in particular, is similar to that predicted by Kolmogorov turbulence.

Turbulence is often quoted as a source of \emph{pressure} based on the
Kolmogorov description.  At size scales much smaller than the driving scale of
the turbulence (the ``box size'' in a simulation), turbulence becomes isotropic
and can add support against gravitational collapse.  

However, turbulence decays rapidly.  The turbulent decay timescale
$\tau_{decay}\propto L / v$, where $L$ is the turbulent length scale and $v$ is
the velocity scale.  It therefore increases with size scale as
$\tau_{decay}\propto L^{2/3}$.  Turbulence decays most quickly on the smallest
timescales.

We are therefore left with two conditions: Turbulence must be driven at large
scales for turbulence to provide support against gravity, and it must be
constantly driven to resupply the turbulence that is transferred to heat on the
smallest scales.

Because the ISM is compressible, interacting flows within the turbulent medium
will result in density enhancements and voids.  Many simulation studies have
determined that the resulting density distribution, and correspondingly the
column-density distribution, should be approximately log-normal.  Observational
studies agree that in regions not yet significantly affected by gravity, the 
column-density distribution is log-normal.  In regions where stars are actively
forming, a high-density power-law tail forms.

One theory of star formation holds that the initial mass function of stars is
determined entirely by turbulence.  In this description, the highest
overdensities in the turbulent medium become gravitationally unstable and
separate from the turbulent flow as they collapse into proto-stellar cores.

\subsection{Mass Functions}
Perhaps the most fundamental goal of star formation studies is to determine the
Initial Mass Function (IMF) of stars and what, if anything, causes it to vary.
It is also one of the most challenging statistically and observationally.

The IMF defines the probability distribution function of stellar masses at
birth, and therefore differs greatly from the present-day stellar mass function
that is very strongly affected by stellar death at the highest masses.  In
order to determine the mass function for the most massive stars, it is
necessary to look at their birth places.  These birth places are dusty, dense,
and rare.  

It remains somewhat unclear whether the IMF is a universal function or is sampled
independently for individual clusters.  If it is universal, there is a possibility of
forming massive stars anywhere stars form.  If cluster-dependent, then a massive star
must form with a surrounding cluster.

Some groups now claim that the initial mass function is decided in the gas
phase by the formation of cores.  The Core Mass Function measures the
probability distribution function of core masses, where cores are generally
identified observationally as column-density peaks in millimeter/submillimeter
emission maps.  The CMF has a similar functional form to the IMF, but its mean
is higher by a factor $\sim3$ in local star forming regions, leading to the claim
that star formation proceeds from CMF->IMF with 30\% efficiency.

Gas clouds follow a mass function that extends up to the largest
possible coherent scales, giant molecular clouds with scales $\sim50-100$ pc
that are limited by the scale-height of the ISM in Galactic disks.
Between `cores' and GMCs, intermediate scale blobs are often called `clumps'. 
The mass function of these clumps has yet to be determined.  

The mass function of GMCs was determine from CO emission towards the Galactic
plane and in nearby galaxies (e.g., M33) where they can be resolved.  The CMF
was measured in nearby clouds where 30\arcsec\ beams easily resolve $\sim0.1$
pc cores.  However, clumps are only found in large numbers in the Galactic
plane, where distances are uncertain.  They cannot be resolved in other
galaxies (except by ALMA now).

To understand star formation on a galactic scale, it is necessary to understand
the transition from large-scale giant molecular clouds and proto-stellar cores.
Clouds follow a shallow mass function, with the largest clouds containing most
of the gas.  Cores and stars are both drawn from steep mass functions in which
most of the mass near some peak in the distribution.  Presumably there must be
some intermediate state of the gas that is drawn from an intermediate
distribution, shallower than `cores' but steeper than `clouds'.  

\subsubsection{Clusters}
Clusters are also drawn from a mass function comparable to stars, but
ironically their distribution is better measured than for stars.  Clusters are
easily visible - and resolvable - in other galaxies, and massive clusters are
less likely to be embedded than massive stars.  In normal galaxies, cluster
populations are consistent with a Schechter distribution: a power-law
$\alpha\sim2$ with an exponential cutoff at large masses.

Since clusters are not drawn from the same parent distribution as GMCs (which
have $\alpha\sim1$), it is plausible that their precursors are, instead, the
intermediate-scale `clumps' observed in the millimeter continuum.  However, the
clump mass function has yet to be measured, so even this first step of
determining plausibility is incomplete.

Clusters are an important observational tool in astrophysics.  For stellar
studies, they have been used to select populations of co-eval stars.  In
extragalactic studies, they are frequently the smallest observable individual
units.  However, many recent works have pointed out that clusters may be
short-lived, transient phenomena.  Any study of their populations must take
in to account their dissolution.  The most massive clusters, however, are both
the most easily observed and the longest lived, and therefore provide some of the
most useful tools for understanding stars.

As with massive stars, massive clusters are rare.  Only a handful of young
massive clusters are known within our Galaxy, including the most massive,  NGC
3603, the Arches cluster, and Westerlund 1 \citep{PortegiesZwart2010}.  These
are the only locations in the galaxy known to be forming multiple stars near
the (possible) upper stellar mass limit.  Despite their importance, though,
only a handful of these clusters are known and the population of such clusters
is effectively unconstrained.  The incomplete knowledge of clusters is due to
extinction  and confusion within the plane.

%\subsection{Galactic Plane Surveys}
%The idea to observe the plane of the Galaxy is not new.

\section{Outline}
This thesis includes 5 chapters.
Chapter 2 describes observations of the W5 star-forming region to identify outflows;
this chapter is somewhat tangential to the rest.
Chapter 3 describes the BGPS data reduction process and data pipeline.
Chapter 4 is the pilot study of \formaldehyde towards previously-known UCHII regions.
It includes the methodology and analysis of turbulent properties of Galactic GMCs.
Chapter 5 expands upon Chapter 4, detailing the expansion of the \formaldehyde survey
to BGPS-selected sources.
Chapter 6 is a Letter identifying massive proto-clusters in the BGPS.
Chapter 7 concludes.

\ifstandalone
\bibliographystyle{apj_w_etal}  % or "siam", or "alpha", or "abbrv"
%\bibliography{thesis}      % bib database file refs.bib
\bibliography{bibdesk}      % bib database file refs.bib
\fi

\end{document}

% 
% %%\documentclass[defaultstyle,11pt]{thesis}
%\documentclass[]{report}
%\documentclass[]{article}
%\usepackage{aastex_hack}
%\usepackage{deluxetable}
\documentclass[preprint]{aastex}


%%%%%%%%%%%%%%%%%%%%%%%%%%%%%%%%%%%%%%%%%%%%%%%%%%%%%%%%%%%%%%%%
%%%%%%%%%%%  see documentation for information about  %%%%%%%%%%
%%%%%%%%%%%  the options (11pt, defaultstyle, etc.)   %%%%%%%%%%
%%%%%%%  http://www.colorado.edu/its/docs/latex/thesis/  %%%%%%%
%%%%%%%%%%%%%%%%%%%%%%%%%%%%%%%%%%%%%%%%%%%%%%%%%%%%%%%%%%%%%%%%
%		\documentclass[typewriterstyle]{thesis}
% 		\documentclass[modernstyle]{thesis}
% 		\documentclass[modernstyle,11pt]{thesis}
%	 	\documentclass[modernstyle,12pt]{thesis}

%%%%%%%%%%%%%%%%%%%%%%%%%%%%%%%%%%%%%%%%%%%%%%%%%%%%%%%%%%%%%%%%
%%%%%%%%%%%    load any packages which are needed    %%%%%%%%%%%
%%%%%%%%%%%%%%%%%%%%%%%%%%%%%%%%%%%%%%%%%%%%%%%%%%%%%%%%%%%%%%%%
\usepackage{latexsym}		% to get LASY symbols
\usepackage{graphicx}		% to insert PostScript figures
%\usepackage{deluxetable}
\usepackage{rotating}		% for sideways tables/figures
\usepackage{natbib}  % Requires natbib.sty, available from http://ads.harvard.edu/pubs/bibtex/astronat/
\usepackage{savesym}
\usepackage{amssymb}
%\savesymbol{singlespace}
\savesymbol{doublespace}
%\usepackage{wrapfig}
%\usepackage{setspace}
\usepackage{xspace}
\usepackage{color}
\usepackage{multicol}
\usepackage{mdframed}
\usepackage{url}
\usepackage{subfigure}
%\usepackage{emulateapj}
\usepackage{lscape}
\usepackage{grffile}
\usepackage{standalone}
\standalonetrue
\usepackage{import}
\usepackage[utf8]{inputenc}
\usepackage{longtable}
\usepackage{booktabs}



%%%%%%%%%%%%%%%%%%%%%%%%%%%%%%%%%%%%%%%%%%%%%%%%%%%%%%%%%%%%%%%%
%%%%%%%%%%%%       all the preamble material:       %%%%%%%%%%%%
%%%%%%%%%%%%%%%%%%%%%%%%%%%%%%%%%%%%%%%%%%%%%%%%%%%%%%%%%%%%%%%%

% \title{Star Formation in the Galaxy}
% 
% \author{Adam G.}{Ginsburg}
% 
% \otherdegrees{B.S., Rice University, 2007\\
% 	      M.S., University of Colorado, Boulder, 2009}
% 
% \degree{Doctor of Philosophy}		%  #1 {long descr.}
% 	{Ph.D., Rocket Science (ok, fine, astrophysics)}		%  #2 {short descr.}
% 
% \dept{Department of}			%  #1 {designation}
% 	{Astrophysical and Planetary Sciences}		%  #2 {name}
% 
% \advisor{Prof.}				%  #1 {title}
% 	{John Bally}			%  #2 {name}
% 
% \reader{Prof.~Jeremy Darling}		%  2nd person to sign thesis
% \readerThree{Prof.~Jason Glenn}		%  3rd person to sign thesis
% \readerFour{Prof.~Michael Shull}	%  4rd person to sign thesis
% \readerFour{Prof.~Neal Evans}	%  4rd person to sign thesis
% 
% \abstract{  \OnePageChapter	% one page only ??
% 
%     I discovered dust in space.  
% 
% 	}
% 
% 
% \dedication[Dedication]{	% NEVER use \OnePageChapter here.
% 	To 1, the second number in binary.
% 	}
% 
% \acknowledgements{	\OnePageChapter	% *MUST* BE ONLY ONE PAGE!
% 	All y'all.
% 	}
% 
% \ToCisShort	% a 1-page Table of Contents ??
% 
% \LoFisShort	% a 1-page List of Figures ??
% %	\emptyLoF	% no List of Figures at all ??
% 
% \LoTisShort	% a 1-page List of Tables ??
% %	\emptyLoT	% no List of Tables at all ??
% 
% 
% %%%%%%%%%%%%%%%%%%%%%%%%%%%%%%%%%%%%%%%%%%%%%%%%%%%%%%%%%%%%%%%%%
% %%%%%%%%%%%%%%%       BEGIN DOCUMENT...         %%%%%%%%%%%%%%%%%
% %%%%%%%%%%%%%%%%%%%%%%%%%%%%%%%%%%%%%%%%%%%%%%%%%%%%%%%%%%%%%%%%%
% 
% %%%%  footnote style; default=\arabic  (numbered 1,2,3...)
% %%%%  others:  \roman, \Roman, \alph, \Alph, \fnsymbol
% %	"\fnsymbol" uses asterisk, dagger, double-dagger, etc.
% %	\renewcommand{\thefootnote}{\fnsymbol{footnote}}
% %	\setcounter{footnote}{0}

\input{macros}		% file containing author's macro definitions

\begin{document}
% \input{introduction}
% 
% %\input{ch_iras05358}
% \input{ch_w5}
% \input{ch_h2co}
% \input{ch_h2colarge}
% \input{ch_boundhii}
% 
% %\input ch2.tex			% file with Chapter 2 contents
% 
% %%%%%%%%%%%%%%%%%%%%%%%%%%%%%%%%%%%%%%%%%%%%%%%%%%%%%%%%%%%%%%%%%%%
% %%%%%%%%%%%%%%%%%%%%%%%  Bibliography %%%%%%%%%%%%%%%%%%%%%%%%%%%%%
% %%%%%%%%%%%%%%%%%%%%%%%%%%%%%%%%%%%%%%%%%%%%%%%%%%%%%%%%%%%%%%%%%%%
% 
% \bibliographystyle{plain}	% or "siam", or "alpha", or "abbrv"
% 				% see other styles (.bst files) in
% 				% $TEXHOME/texmf/bibtex/bst
% 
% \nocite{*}		% list all refs in database, cited or not.
% 
% \bibliography{thesis}		% bib database file refs.bib
% 
% %%%%%%%%%%%%%%%%%%%%%%%%%%%%%%%%%%%%%%%%%%%%%%%%%%%%%%%%%%%%%%%%%%%
% %%%%%%%%%%%%%%%%%%%%%%%%  Appendices %%%%%%%%%%%%%%%%%%%%%%%%%%%%%%
% %%%%%%%%%%%%%%%%%%%%%%%%%%%%%%%%%%%%%%%%%%%%%%%%%%%%%%%%%%%%%%%%%%%
% 
% \appendix	% don't forget this line if you have appendices!
% 
% %\input appA.tex			% file with Appendix A contents
% %\input appB.tex			% file with Appendix B contents
% 
% %%%%%%%%%%%%%%%%%%%%%%%%%%%%%%%%%%%%%%%%%%%%%%%%%%%%%%%%%%%%%%%%%%%
% %%%%%%%%%%%%%%%%%%%%%%%%   THE END   %%%%%%%%%%%%%%%%%%%%%%%%%%%%%%
% %%%%%%%%%%%%%%%%%%%%%%%%%%%%%%%%%%%%%%%%%%%%%%%%%%%%%%%%%%%%%%%%%%%
% 
% \end{document}
% 
% 

\chapter{Outflows and proto-OB stars in a small protocluster, IRAS 05358+3543}

\section{Introduction}
Collimated, bipolar outflows accompany the birth of young stars from the
earliest stages of star formation to the end of their accretion phase
\citep[e.g.][]{reipurth2001}.    While the birth of isolated low-mass stars is
becoming well understood, the formation of massive stars ($>10 \msun$) and clusters remains a
topic of intense study.    Observations show that moderate to high-mass stars
tend to form in dense clusters \citep{lada2003}.    In a clustered environment,  the dynamics of
the gas and stars can profoundly impact both accretion and mass-loss processes.
Feedback from these massive clusters may play a significant role in momentum
injection and turbulence driving in the interstellar medium.  

Outflows from massive stars are less studied than those from low mass stars
largely because massive stars accrete most of their mass while deeply embedded.
Therefore, unlike low mass young stars that are accessible in the optical,
massive stellar outflows can only be seen at infrared and longer wavelengths.
Direct evidence for jets from massive young stellar objects (YSOs) from \hh\ or
optical emission is generally lacking
\citep[e.g.][]{alvarez2005,kumar2002,wang2003}, although there is evidence that
massive stars are the sources of collimated molecular outflows from millimeter
observations \citep[e.g.][]{beuther2002b}.  Outflows from massive stars may
allow accretion to continue after their radiation pressure would
otherwise halt accretion in a spherically symmetric system
\citep{krumholz2009}.  They therefore represent a crucial component in
understanding how stars above $\sim$10 \msun\ can form.

\region\ is a double cluster of embedded infrared sources located at a distance
of 1.8 kpc in the Auriga molecular cloud complex \citep{heyer1996} associated
with the HII regions Sh-2 231 through 235 at Galactic coordinates around $l,~b$
= 173.48,+2.45 in the Perseus arm.   \necluster\ is the
collection of highly obscured and mm-bright sources slightly northeast of
\swcluster, which is the location of the IRAS 05358+3543 point source and the
optical emission nebula (see Figure \ref{fig:overview_ha}).  The IRAS source is
probably a blend of the three brightest infrared objects in the MSX A-band and
MIPS 24 \um\ images, which are located at \necluster, IR 41, and IR 6. For the
purpose of this paper,the whole complex including both sources is referred toas
\region, and otherwise refer to individual objects specifically.

Early observations revealed the presence of OH \citep{Wouterloot1993}, \HtwoO\ 
\citep{Scalise1989, Henning1992}, and methanol \citep{Menten1991} masers about
an arcminute northeast of the IRAS source, indicating that massive stars are 
likely present at that location.  Near infrared observations revealed
the presence of two embedded clusters  \citep{porras2000,jiang2001} labeled
\swcluster\ for the southwestern cluster associated with the IRAS source, and \necluster\ 
for the northeastern cluster located near the OH, \HtwoO, and CH$_3$OH masers.
Stars identified in \citet{porras2000} are referred to by the designation
``IR (number)'' corresponding to the catalog number in that paper.
\citet{porras2000} also included scanning Fabry-Perot velocity measurements of
the inner $\sim1$\arcmin.  CO observations revealed broad line wings indicative
of a molecular outflow \citep{casoli1986,shepherd1996}.  \citet{kumar2002} and
\citet{khanzadyan2004} presented narrow band images of 2.12 \um\ \htwo\
emission that reveled the presence of multiple outflows.  Interferometric
imaging of CO and SiO confirmed the presence of at least three flows emerging
from the northeast cluster centered on the masers \citep{beuther2002} having a
total mass of about 20 \msun .  \citet{beuther2002} also presented MAMBO 1.2 mm
maps and a mass estimate of 610 \msun\ for the whole region.
\citet{williams2004} presented SCUBA maps and mass estimates of the clusters of
195/126\msun\ for \necluster\ and 24/12 \msun\ for \swcluster\ (850 \um/450 \um).
\citet{Zinchenko1997} measured the dense gas properties using the \ammonia\
(1,1) and (2,2) lines.  They measure a mean density $n \approx 10^{3.60}$ \percc,
temperature 26.5K, and a mass of 600 \msun .  The total luminosity of the two
clusters is about 6300 \lsun , indicating that the region is giving birth to
massive stars \citep{porras2000}. 

Millimeter wavelength interferometry with arcsecond angular resolution has
revealed a compact cluster of deeply embedded sources centered on the \HtwoO\
and methanol maser position \citep{beuther2002,beuther2007,leurini2007}.
\citet{beuther2002} identified 3 mm continuum cores, labeled mm1-mm3 (shown in
Figure \ref{fig:outflowsh2}).  \citet{beuther2007} resolved these cores into
smaller objects.  Source mm1a is associated with a cm continuum point source
and will be discussed in detail below.

\region\ has previously been observed at low spatial resolution in the J=2-1 and J=3-2
transitions with the Kosma 3m telescope \citep{Mao2004}.  While the general presence
of outflows was recognized and a total mass estimated, the specific outflows were not 
resolved.  \citet{beuther2002} observed the CO J=6-5, J=2-1, and J=1-0
transitions at moderate resolution in the inner few arcminutes.
\citet{Thomas2008} observed C$^{17}$O in the J=2-1 and J=3-2 transitions with a
single pointing using the JCMT.



\section{Observations}

A collection of data acquired by the authors and from publicly
available archives is presented.  An overview of the data is presented in figure
\ref{fig:overview_ha}. The goal was to develop a complete picture of the outflows
in \region\ and their probable sources.  CO data were acquired to estimate the
total outflowing mass and to identify outflowing molecular material
unassociated with \hh\ shocks.  Archival Spitzer IRAC and MIPS 24 \um\ data
were used to identify probable YSOs as candidate outflow sources.
Near-infrared spectra were acquired primarily to determine \hh\ kinematics and
develop a 3D picture of the region.  Optical spectra were acquired to attempt
to identify stellar types in the unobscured \swcluster\ region.  Finally,
archival VLA data were used to acquire better constraints on the position and
physical properties of the known ultracompact HII (UCHII) region, and to detect
or set limits on other UCHIIs.

\subsection{Sub-millimeter Observations}

The 345 GHz J = 3-2 rotational transition of CO was observed with the James
Clerk Maxwell Telescope (JCMT) on 4 January,  2008 with the 16 element (14
functional) HARP-B heterodyne focal plane array.   Two  12\arcmin\ $\times$
10\arcmin\  raster scans in R.A.  and Dec.  were taken with orthogonal
orientations to assure complete coverage in the region of interest; this
resulted in a useable field 11.7\arcmin\ $\times$ 11.3\arcmin\ with higher noise
along the edges.  The beam size at 345 GHz is about 15\arcsec.

Observations were conducted during grade 3 conditions with the 225 GHz zenith
optical depth of the atmosphere $\tau\sim0.1$. A channel width of 488 kHz
corresponding to 0.423 \kms\ was used.    The maps required a total of 1 hour
to acquire and resulted in an effective integration time of 4.6 seconds per
pixel (there are 12,000 $6\times6\arcsec$\ pixels in the final grid), resulting
in a noise per pixel of 0.36 K \kms.

The optical depth and telescope efficiency corrections were applied by the JCMT
pipeline to convert the recorded antenna temperatures to the corrected antenna
%temperature T$_A^*$
%where $T_A^* = T_A / ( \eta_T ~ \tau_{345} ~ \textrm{sec}[z] )$ and
%$\tau_{345} = .05 + 2.5 \tau_{225}$
\footnote{See \\
\url{http://docs.jach.hawaii.edu/JCMT/OVERVIEW/tel\_overview/} for a discussion
of JCMT parameters}.  An additional
main-beam correction has been applied, $$T_{mb}=\frac{T_A*}{\eta_{mb}}$$ where
$\eta_{mb} $ was measured by observing Mars to be $\approx0.60$ at 345
GHz.  Emission in the sidelobes is expected to be small at the outflow velocities.

On September 25 and November 15, 2008 the CO, $^{13}$CO, and C$^{18}$O J=2-1
transitions were observed in the central 3\arcmin\ of \region.  The beamsize
at 220 GHz is about 23\arcsec.  The sideband configuration used also includes
the \linebreak \nolinebreak{\so} and $^{13}$CS 5-4 transitions.  Conditions
during these observations were grade 5 ($\tau \sim 0.24-0.28$) and therefore
too poor to use the HARP instrument, but acceptable for the A3 detector.  

Data reduction used the Starlink package following the standard routines
recommended by the JCMT support scientists \footnote{
\url{http://www.jach.hawaii.edu/JCMT/spectral\_line/data\_reduction/acsisdr/}}.
The CO 3-2 data cube was extracted over a velocity range from --50 to 10 \kms\ LSR and
spectral baselines were fit over the velocity range --50 to --40 and 0 to 10
\kms\ and subtracted.  The data were re-gridded into  6\arcsec\  pixels  and 2
pixel Gaussian smoothing was used to fill in the gaps left by the two bad
detectors in the 4 $\times$4 array.   The data cube was cropped to remove
undersampled edges which have high noise and bad baselines.  The beam efficiency
was 0.68 at 230 GHz.

The A3 data cubes were extracted over the velocity range --60 to 20 \kms\ and
baselines were calculated over --60 to --40 and 0 to 20 \kms.  The data
was gridded into 10\arcsec\ pixels with 2 pixel gaussian smoothing to reduce
sub-resolution noise variations.  

% SCUBA 450 and 850\um\ data were acquired from the JCMT archive via the 
% Canadian Astronomical Data Center.

\subsection{Spitzer}

Spitzer IRAC bands 1 to 4 and MIPS band 1 data were retrieved from the Spitzer
Science Center archive.  \citet{qiu2008} acquired the data as part of a study
of many high-mass star forming regions; they identified YSO candidates based on
IRAC colors.  The version 18 post-BCD data products were used to produce images
and photometric catalog from \citet{qiu2008}, which was made from a more
carefully-reduced data set, was used for SED analysis.

\subsection{Near-IR images}
Near-infrared data were acquired using the Wide-field Infrared Camera (WIRCam) on
the Canada-France-Hawaii Telescope (CFHT) on Mauna Kea. The field of view is 
20\arcmin$\times$20\arcmin\ ~and pixel scale 0.3\arcsec.  Data were acquired on 
November 18, 19 and December 20, 2005.  The seeing was 0.5-0.7\arcsec\ during the
observations.  A 0.032 \um\ wide filter centered at 2.122 \um\ was used to take images
of the \hh\ S(1) 1-0 rovibrational transition.  Each \hh\ exposure was 58
seconds, and dithered images were taken for a total exposure time of 1755
seconds.  The data were reduced with the WIRCam pipeline.

\subsection{Near-IR spectra}
Near-infrared spectra were acquired using the TripleSpec instrument at Apache
Point Observatory.  TripleSpec simultaneously acquires J, H, and K band spectra
over a 42\arcsec\ long slit.  A slit width of 1.1\arcsec\ with an
approximate spectral resolution $\lambda/\Delta\lambda=2700$ was used.  

Observations were taken on the nights of December 2, 2008 and January 7, 2009.
Data on December 2 were taken in an ABBA nod pattern, but because of the need
to observe extended structure across the slit a stare strategy was selected on
January 7.  

The data were reduced using the {\sc twodspec} package in IRAF.  
HD31135, an A0 star, was used as a flux calibrator.  Wavelength calibration was
performed using night sky lines.  Lines filling the slit were subtracted
to remove atmospheric emission lines.  Telluric absorption correction was {\emph
not} performed, but telluric absorption is considered in the analysis.

The transformations from the observed geocentric reference frame to $v_{LSR}$ 
were computed to be 0.78 \kms\ on Dec 2 and 19.74 \kms\ on Jan 8.
% and 24.35 \kms\ on Jan 17.

\subsection{Optical Spectra}
Optical spectra were acquired using the Double Imaging Spectrograph instrument
at APO.  The high-resolution red and blue gratings
were centered at 6564 \AA\ and 5007 \AA\ with a coverage of about 1200 angstroms and
resolution $\lambda/\Delta\lambda \approx 5000$.  Sets of three 900s exposures
and three 200s exposures were acquired on the targets and on the spectrophotometric
calibrator G191-b2b with a 1.5" slit.  Observations were taken on the night
of January 17, 2009 under clear conditions.

Optical spectra were also reduced using the {\sc twodspec} package in IRAF.
Wavelength calibration was done with HeNeAr lamps and night sky lines
in the red band, and HeNeAr lamps in the blue band.  Lines filling the slit
were subtracted to remove atmospheric lines, though some astrophysical lines
also filled the slit and these were measured before background subtraction.
The $v_{LSR}$ correction for this date was 24.4 \kms.

% We convert our observations to the local standard of rest using...
% vLSR = vBSR + 9 cos(l) cos(b) + 12 sin(l) cos(b) + 7 sin(b) 
% 
% So the $v_{LSR}$ corrections @midnight are (towards object?):
% Jan 17 2009: 24.35 \kms
% Jan 8  2009: 19.74
% Dec 2  2009: 0.78 \kms
% 
% Subtract these velocities to get the correct velocity.  < Dec we were moving
% towards the object, after Dec 1ish we were moving away.
% 
% using \url{http://www.jupiterspacestation.org/software/Vlsr.html}

\subsection{Optical imaging}

CCD images images were obtained on the nights of 14 and 15 September 2009
NOAO Mosaic 1 Camera at the f/3.1 prime focus 
of the 4 meter Mayal telescope atthe  Kitt Peak National Observatory (KPNO).  
The Mosaic 1 camera is a 8192$\times$8192 pixel array (consisting of eight 
2048$\times$4096 pixel CCD chips) with a pixel scale of 0.26$''$ pixel$^{-1}$ and 
a field of view 35.4$'$ on a side.  Narrow-band filters centered on 
6569\AA\ and 6730\AA\ both with a FWHM of 80\AA\ were use to obtain 
H$\alpha$ and [SII] images.  An SDSS i' filter which is centered on 
7732\AA\ with a FWHM of 1548\AA was used for continuum imaging.
A set of five dithered 600 second exposures were obtained in 
H$\alpha$ and [SII] using the standard MOSDITHER pattern 
to eliminate cosmic rays and the gaps between the
individual chips in Mosaic.  A dithered set of five 180 second 
exposures were obtained in the in the broad-band SDSS i-band filter to
discriminate between H$\alpha$, [SII], and continuum emission. 
Images were reduced in the standard manner by the NOAO Mosaic reduction
pipeline \citep{valdes2007}.   


\Figure{figures_ch05358/mosaicHA_fullfield_simbadlabels}{The CFHT \hh\ (bold), CO 3-2 HARP
(thin), and CO 2-1 A3 (dashed) fields overlaid on the KPNO \ha\ mosaic
with selected objects identified by their SIMBAD names.  \swcluster\ is coincident
with IRAS 05358+3543. % \necluster\ is coincident with the SIMBAD position of LBN 808.
% WARNING: SIMBAD reports S231 at the position of LBN 808, while the original Sharpless
% catalog reports it to be where it is identified in our map.  Similarly SIMBAD reports
% S233 at LBN 802 while we mark the original Sharpless position.
% IRAS 05358+3543 or S233IR is the point to the southwest of LBN 808.
}{fig:overview_ha}{1.0}{}

% VLA DATA SECTION ( VERY LARGE ARRAY ) 
\subsection{VLA data}
VLA archival data from projects AR482, AR513, AS831, and AM697 were re-reduced
to perform a deeper search for UCHII regions and aquire more data points on the
known UCHII's SED.  Data from AR482 were previously published in
\citet{beuther2007}, the other data are unpublished.  The data were reduced
using the VLA pipeline in AIPS ({\sc vlarun}).  The observations used and
sensitivities and beam sizes achieved are listed in Tables \ref{tab:vlatimes}
and \ref{tab:vla}.  There appeared to be calibration errors in the AR482
observations (the phase calibrator was 2-3 times brighter than in all other
observations) and this data were therefore not used in the final analysis, but
it produced consistent pointing results.  


\Table{cccccccc}
{VLA Observation Program Names, Dates, and Times}
{\colhead{VLA } &  \colhead{Observation} &  \colhead{Time } &  \colhead{Array} &
\colhead{Band} & \colhead{Fluxcal} &  \colhead{Phase cal} & \colhead{Phase cal } \\
Observation & Date &on&&&&& Percent \\
Name &&Source&&&&& Uncertainty \\}
{tab:vlatimes}
{
AR482 & August 2 2001    & 2580s & B & X &3c286 & 0555+398   & 22  \\
AR513 & June 21 2003     & 7770s & A & X &3c286 & 0555+398   & 0.8 \\
AS831 & February 26 2005 & 2640s & B & X &3c286 & 0555+398   & 0.7 \\
AS831 & August 5 2005    & 2660s & C & X &3c286 & 0555+398   & 0.3 \\
AS831 & May 11 2006      & 2610s & A & X &3c286 & 0555+398   & 3.0 \\
AL704 & August 7 2007    & 6423s & A & Q &3c273 & 0555+398   & 18  \\ 
AL704 & September 1 2007 & 6423s & A & Q &3c273 & 0555+398   & 13  \\ 
AM697 & November 26 2001 & 2880s & D & Q &3c286 & 0555+398   & 2.2 \\
AM697 & November 28 2001 & 1530s & D & K &3c286 & 0555+398   & 2.1 \\
AM697 & November 28 2001 & 1530s & D & U &3c286 & 0555+398   & 5.8 \\
}{}


\section{Results}
\subsection{Near Infrared Imaging: Outflows and Stars}
Eleven distinct outflows have been identified in \region\ in the images.
Outflows are identified from a combination of J=3-2 CO data, shock excited
\htwo\ emission, and published interferometric maps \citep{beuther2002}.
Suspected CO outflows were identified by the presence of wings on the CO J=3-2
emission lines that extended beyond the typical velocity range of emission
associated with the line core.    The single dish data were compared to the
interferometric maps of \citet{beuther2002}.  The CFHT \hh\ image was then used
to search for shock-excited emission associated with the outflow lobes.

\begin{figure*}[htpb]
    \center
    \epsscale{0.75}
    \hspace{-1.2in}
    \plotone{figures_ch05358/outflow_overview}
%    \includegraphics[width=2.25in]{outflow_overview}
%    \hspace{-0.1in}
%    \includegraphics[width=2.5in]{COcontours_on_H2}
%    \hspace{-0.5in}
%    \includegraphics[width=2.5in]{so5645_on_H2}
    \caption{The outflows described in section \ref{sec:outflows} overlaid on
    the CFHT \hh\ image.  Numbers followed by {\it r} and {\it b} (red and
    blue), {\it n} and {\it s} (north and south), or {\it e} and {\it w} (east
    and west) are thought to be counterflows.  Red and blue vectors indicate
    red and blue doppler shifts.  Green vectors indicate where the doppler
    shift is ambiguous or cannot be determined.  Magenta circles are Spitzer
    24\um\ sources.  Red squares are \citet{beuther2002} mm sources (from left
    to right, mm1, mm2, mm3).  The blue diamond is a YSO candidate detected
    only in IRAC bands.  The length of the vectors corresponds to the
    approximate length of the outflows.  Source 1 and 6 correspond to
    \citet{porras2000} IR 6 and IR 41 respectively, and they are discussed under
    these names in sections \ref{sec:outflows}.  The bows of Outflow 1n and 4n 
    are detected in \ha\ and [S II] emission and are therefore as identified
    as Herbig-Haro objects HH 993 and 994 respectively.
    \label{fig:outflowsh2}}
\end{figure*}
% \Figure{figures_ch05358/khanzadyan_outflow_labels}{The outflows as identified by \citet{khanzadyan2004}
% overlaid on our \hh\ image}{fig:klabels}{0.75}
% \Figure{figures_ch05358/spitzer_RGB}{A Spitzer color composite image of the region.  Red: 8\um\ 
% Green: 4.5\um\  Blue: 3.6\um.}{fig:spitzer_rgb}{1.0}

\Figure{figures_ch05358/COcontours_on_H2} {CO contours integrated from $v_{LSR}=$ -13 to -4
\kms\ (red) and -34 to -21 \kms\ (blue) at levels of 2,4,6,10,20,30,40,50 K
\kms\ overlaid on the \hh\ image.  Specific outflows are labeled in Figure
\ref{fig:outflowsh2} on the same scale.} {fig:COonH2}{0.75}{}

\begin{figure*}[htpb]
    \hspace{-0.6in}
  \includegraphics[scale=0.40,clip=true]{figures_ch05358/so5645_on_h2}
%    \hspace{-0.6in}
  \includegraphics[scale=0.40,clip=true]{figures_ch05358/SII_outflowvectors}
  \caption{(a) \hh\ image with \so\ peak flux contours at 0.5-1.4 K in intervals of 0.15 K
  overlaid.  With a critical density $\sim3.5\ee{6}$ \citep{leidendb}, this
  transition is a dense gas tracer.  (b) The [S II] image with outflow vectors overlaid.
  Diffuse emission can be seen at the north ends of Outflows 1, 4, and 6 and around the
  reflection nebula near source IR 41.} 
  \label{fig:so_on_h2}
\end{figure*}



%\subsection{\htwo\ results}
\label{sec:outflows}

% fix figure references
% This section refers to three figures.
Figure \ref{fig:outflowsh2} shows the \htwo\ S(1) 1-0 2.1218 \um\ (a rovibrational
transition in the electronic ground state from the $v=1$, $J=3$ to the $v=0$,
$J=1$ state) emission in the vicinity of \region\ with outflows and possible
outflow sources labeled.  The mm cores from \citet{beuther2002} are identified
by red squares.  
% Figure \ref{fig:klabels} shows the same image with labels
% from \citet{khanzadyan2004} superposed.  Figure \ref{fig:spitzer_rgb} shows the
% same region in a  3-color IRAC image.

The flow vectors in figure \ref{fig:outflowsh2} were chosen on the basis of the
\htwo\ bow shock morphologies and orientations of chains of \htwo\ features,
association with arcsecond-scale CO features on the \citet{beuther2002} Figure
8 CO map, and/or association with lobes of Doppler-shifted CO emission in the
CO 3-2 data (see figure \ref{fig:COonH2}).  The color of the vector indicates
the suspected Doppler shift; red and blue correspond to red and blueshifts and
green vectors indicate that the Doppler shift is uncertain.    
 
% no clear red/blue
% probable from mm1a by Beuther2002
 {\it \region\ outflow 1:} The most prominent flow in \htwo\ is associated with
 the bright bow-shocks N1 and N6 \citep{khanzadyan2004} located towards PA $\approx$ 345\arcdeg\
 and 170\arcdeg\ respectively from the sub-mm source mms1b \citep{beuther2002}.
 This flow, \citet{beuther2002} outflow A, is associated with redshifted and
 blueshifted CO emission.  The northern shock is seen in \ha\ and [S II]
 emission (figure \ref{fig:so_on_h2}b and \ref{fig:HA_with_CO}) and is given a Herbig-Haro designation
 HH 993.

 This flow is indicated by oppositely directed green vectors from the vicinity
 of smm1, 2, and 3.   It is listed as ``Jet 1'' in \citet{qiu2008}.  \citet{kumar2002}
 identified the knot immediately behind the bow shock as a Mach disk.  In the
 \citet{beuther2002} interferometric maps, the north flow contains redshifted features
 and the south flow contains primarily blueshifted features.  There are also blueshifted 
 CO features to the west of the \hh\ knots that are probably part of a different flow
 that is not seen in \hh\ emission.
 
 The velocity of the flow as measured from \hh\ emssion is blueshifted as much
 as 80 \kms (LSR), but one component is blueshifted only 14 \kms\, which is
 consistent with the cloud velocity.  A redshifted SiO lobe is present in the
 south counterflow.  The presence of \ha, [S II], and [O III] emission in the
 north shock and corresponding nondetections in the south shock suggest that
 there is substantially greater extinction towards the south knot.  While the
 velocities in three of the four apertures picked along the TripleSpec slit are
 blueshifted, there are also knots with velocities consistent with the cloud
 velocity.  \citet{porras2000} measure the velocity of the counterflow to be
 -17.3 \kms, which is consistent with the cloud velocity. Outflow 1 is
 propagating very nearly in the plane of the sky.

% Our optical [S II] data were used to measure a density in the bow shock $\sim
% 500-700$\percc, which is typical of Herbig-Haro bow shocks (e.g. [CITE SOMEONE -
% BALLY?]).

% sources are not clear...
A line connecting the two bow shocks in Outflow 1 goes directly through
\citet{beuther2007} source mm2a despite the clear association in the
\citet{beuther2002} interferometric CO map (their Figure 8) with mm1a.  The
currently available data do not clarify which is the source of the outflow:
while the bent CO outflow appears to trace Outflow 1 back to mm1a, there are
additional parallel CO outflows towards the confused central region that could
originate from either mm1a or mm2a.

A Spitzer 4.5 \um\ and 24 \um\ source is barely detected in \hh\ 2.5\arcmin\ to
the north of Outflow 1.  It is only apparent when the \hh\ image is smoothed
and would have been dismissed as noise except for the association with
a probably 4.5 \um\ extended source.  It is labeled 24\um\ source 7 in figure
\ref{fig:outflowsh2}.  It appears to be slightly resolved at 4.5\um, and is
therefore likely shocked emission.  The object may be a protostellar source 
with an associated outflow, but its proximity to the projected path of Outflow
1 suggests that it may be an older outflow knot.

% Outflow 2 comes from Minier disk
{\it \region\ Outflow 2:} The second brightest \htwo\ features trace a bipolar
flow emerging from the immediate vicinity of the sub-mm cluster at PA $\approx$
135\arcdeg\ (red lobe) and 315\arcdeg\ (blue lobe).  It is listed as ``Jet 2'' in
Figure 6 of \citet{qiu2008}.  
% The northwest lobe is suspected to be blueshifted
% because of the high brightness of the \htwo\ emission (bow shock N3A, B, and C
% in \citet{khanzadyan2004}, figure \ref{fig:klabels}) and the presence of blueshifted emission
% in the JCMT CO data.  However, none of these lines of evidence are conclusive.
The counterflow probably overlaps in the line of sight with the counterflow
from Outflow 3.  It is shorter on the counterflow side either because it has
already penetrated the cloud and is no longer impacting any ambient gas or,
more likely, it has slowly drilled its way out of the molecular cloud and has
not been able to propagate as quickly as the northwest flow.  
The \hh\ velocities measured for these knots are $\sim$ 30 \kms\ blueshifted, or
marginally blue of the cloud LSR velocity.  

The disk identified in \citet{Minier2000} is approximately perpendicular to the
measured angle of Outflow 2 assuming that mm1a is the source of this flow.  It
is therefore an excellent candidate for the outflow source.  A diagram of the
mm1a region is shown in figure \ref{fig:mm1adiagram}.  See Section
\ref{sec:vlaresults} for detailed discussion.



{\it \region\ outflow 3:} The \citet{beuther2002} CO and SiO maps reveal a
third flow, their outflow B at PA $\approx$ 135\arcdeg\  (red lobe) and
315\arcdeg\ (blue lobe).  A chain of  \htwo\ features, \citet{khanzadyan2004}
features N3D and N3E, are probably shocks in this flow.  It is listed as ``Jet
3'' in \citet{qiu2008}.  The two chains of \htwo\ emission indicate that
outflows 2 and 3 are distinct.  There also appears to be a counterflow at a
shorter distance from the mm cores similar to counterflow 2.  

Outflows 2 and 3 may be associated with either redshifted or blueshifted
features in the \citet{beuther2002} CO and SiO maps.  High velocity flows with both
parities are present near both the northwest (\citet{beuther2002} outflow C)
and southeast flow for these jets, but the resolution of the millimeter
observations is inadequate to determine which flow is in which direction.
\citet{porras2000}
measures $v_{LSR} = -7.5$ \kms\ for their knot 4A, which corresponds to the 
blended southeast counterflow of outflows 2 and 3. Their Figure 7 shows a wide 
line that is probably better represented by two or three blended lines, one
consistent with the cloud velocity and the other(s) redshifted.  Since Outflow 2
has a measured blueshift and outflow 3 is significantly fainter, the redshifted
counterflow emission is probably associated with Outflow 2 and the blueshifted
with outflow 3.

{\it \region\ outflow 4:} The JCMT CO data and \htwo\ images reveal a large
outflow lobe consisting of blue lobes 1 and 4 that form a tongue of blueshifted
emission propagating to the northeast at PA $\approx$  20\arcdeg\ (Figure
\ref{fig:outflowsh2}) from the cluster of sub-mm cores.   A faint chain of
\htwo\ features runs along the axis of the CO tongue and terminates in a bright
\htwo\ bow shock located at the northern edge of \ref{fig:outflowsh2}.   Several
\htwo\ knots lie along the expected counterflow direction, but that portion of
the field contains multiple outflows and is highly confused.  If the
counterflow is symmetric with the northeast knot, it extends 2.1 parsecs on the
sky.  

The bow shock of Outflow 4 is seen in the HII and [S II] images, implying that
the extinction is much lower than in the cluster.  Two apertures placed along
the bow shock reveal that it is blueshifted about 70\kms\ and may be extincted
by as little as $A_V\sim.5$.  It is designated HH 994.
 
{\it \region\ outflow 5:} Figure \ref{fig:outflowsh2} shows a bright chain of
\htwo\ knots and bow shocks starting about 10\arcsec\ west of mm3 and
propagating south at PA $\approx$ 190\arcdeg.  The SiO maps of
\citet{beuther2002} show a tongue of blueshifted emission along this chain
(their Outflow C).   The outflow projects back to H$^{13}$CO$^+$ source 3,
which is also a weak mm source.  A lack of obvious counterflow and the
possibility that the knots identified with Outflow 5 could be associated with a
number of different crossing flows makes this identification very tentative.
Higher spatial resolution observations will be required to determine the
association of this outflow.


% OK.  This is dependent on 24um data and spitzer data where it is clear that
% outflows point back to this source
 {\it \region\ outflow 6:} The fourth brightest source in the Spitzer 24\um\ 
 data is located at J(2000) = 05:39:08.5, +35:46:38 (source 5 in the \region\ 
 section of the \citet{qiu2008} catalog) in the middle of the
 molecular ridge that extends from \region\ towards the northwest (24\um\ 
 object 4 in figure \ref{fig:outflowsh2}).   The star is
 located at the northwest end of the tongue of 1.2mm emission mapped by
 \citep{beuther2002} with the MAMBO instrument on the IRAM telescope.  This
 part of the cloud is also seen in silhouette against brighter surrounding
 emission at 8\um.  At wavelengths below 2\um, it is fainter than 14-th
 magnitude and therefore is not listed in the 2MASS catalog, and it is not
 detected in \citet{yan2009} down to 19th magnitude in K.
 %; it is also not detected in our deeper K-band images with an upper
 %limit of [FIND UPPER LIMIT] or ask Chi-hung.   
 
 Spitzer data indicates very red colors between 3.6 and 70 \um, indicating that
 this object is likely to be a Class I protostar.  The SED is fit using the
 online tool provided by \citet{robitaille2007}.  Unfortunately, a wide variety
 of parameters all achieved equally good fits, so no conclusions are drawn
 about the stellar mass or other very uncertain parameters.  However, the top
 models all had $A_V > 20$ and many in the range 30-50, indicating that the
 line of sight is probably through a thick envelope or disk towards this
 source.  

 % note: Av=(0.56N_H+0.23)*10e21
 
 This source lies at the base of the tongue of blueshifted CO 3-2
 emission that extends northwest of \region\ at PA $\approx$ 345\arcdeg\  and
 has mass $\sim .5\msun$.  A pair of \htwo\ features,
 \citet{khanzadyan2004} N12A and N12B are located 30 and 55\arcsec\ from the
 suspected YSO, forming a chain along the axis of the blueshifted CO
 tongue.    \citet{khanzadyan2004} \htwo\ knot N3F lies along the flow axis in
 the redshifted direction.
 
{\it \region\ outflow 7:} The 20\arcsec\ long chain of \htwo\ knots labeled
\citet{khanzadyan2004} N11 appears to trace part of a jet at PA $\approx$
345\arcdeg\ that propagated parallel to outflow 6 about 20\arcsec\ to the east.
The northwest portion of Outflow C in the \citet{beuther2002} SiO map is in 
approximately the same direction as Outflow 7, and it may represent a redshifted
counterflow to the northwest-pointing \hh\ knots.
The jet axis passes within a few arc-seconds of a faint and red YSO located at
J(2000) = 05 39 10.0, +35 46 27 (blue diamond in figure \ref{fig:outflowsh2}
about 35\arcsec\ south of the southern end of the \htwo\ feature).  It may be a
24\um\ source but is lost in the PSF of the bright source at the center of
\necluster.  This object is also undetected down to 19th magnitude in the
\citet{yan2009} K-band image.
 
{\it \region\ outflow 8:}   A prominent jet-like \htwo\ feature protrudes from
the vicinity of \swcluster\ at PA $\approx$ 335\arcdeg\  and ends in
 bright knot %\citet{khanzadyan2004} \htwo\ knot 
N9.     The feature N5B is is
located just outside the ring of \htwo\ emission that surrounds the IRAS source
at the base of the jet.  Towards the southeast, %\citet{khanzadyan2004} \Htwo\
knot N6 is located opposite knot N9 with respect to the southwest cluster.  IR 41,
the \ha\ emission source, labeled 24\um\ source 6 in figure \ref{fig:outflowsh2},
is probably the source of this outflow.
 
{\it \region\ outflow 9:} In the Spitzer and K$_s$ images, an infrared
reflection nebula opens towards the southwest at PA $\approx$ 245\arcdeg\ and
points towards a blueshifted CO region.  The reflection nebula is also seen in
\ha.  It is likely that the CO emission in CO Region 1 (table \ref{tab:comeas})
traces a fossil cavity whose walls provide the scattering surface of the
reflection nebula.

{\it \region\ outflow 10 and IR 6:}   A bright \htwo\ filament protrudes at PA $\approx$
15\arcdeg\ towards the northeast of IR 6 (24\um\ source 1, \citet{qiu2008}
source 8).  The star is the third brightest 24\um\ source in the \region\
region.   Since it is visible at visual wavelengths, it is not heavily
embedded.     Its  H$\alpha$ emission and association with an outflow lobe and
\htwo\ emission suggest that it is a moderate mass Herbig AeBe star associated
with the \region\ complex.  The optical spectrum confirms this hypothesis: the
star has \ha\ absorption wings on either side of a very bright, asymmetric \ha\
emission profile (see section \ref{sec:dis}). 

IR 6 is seen to be the source of Outflow 10.  Data for this source is available
from $\sim$0.45-24\um, so the \citet{robitaille2007} spectral fitter puts
strong constraints on the star's mass and luminosity.  The measured mass and
luminosity are  $M=4.5\pm0.5$ \msun\ and $L = 10^{2.3\pm.25} L_\odot$, parameters
consistent with a B7V ($\pm 1$ spectral class) main sequence star.  The range
of ages in the models covers $10^4-10^7$ years but favors stars in the range
$10^5-10^6$ years.

While there is a small clump of redshifted CO emission to the northeast of the
object, the \htwo\ spectrum shows that the north flow is blueshifted $v_{LSR}\sim-40
$\kms, and the lack of a visible counterflow suggests that the counterflow may
be masked behind an additional extincting medium.  The counterflow drawn in
figure \ref{fig:outflowsh2} is not seen in emission but is identified as a
probable location for a counterflow because of the confident association of
outflow 10n with source IR 6.

{\it \region\ outflow 11:} A chain of \hh\ knots is seen at 2.12\um\ and in the
Spitzer 4.5\um\ image.  They trace back to either IR 78 or 24\um\ source 4.
There is a tongue of redshifted CO 3-2 emission in the same direction as this
flow that suggests it may be redshifted.



%\subsection{IR 41}
{\it IR 41}: 
There is an arc-like \hh\ emission feature surrounding the \ha\ emission line
star IR 41.  This implies that the star is probably a late B-type star with too
little Lyman continuum emission to generate a photon-dominated region (PDR) but
enough soft UV to excite \hh.  From the measured \ha\ and nondetection of \hb\
at the star's location down to a 5-$\sigma$ limit of 1\ee{-17} erg s$^{-1}$
\persc \AA$^{-1}$, a lower limit on the extinction column $A_V=15$ is derived.
The \citet{robitaille2007} fitter yields a mass estimate of 7.4$\pm 0.6$\msun
and luminosity $L=10^{2.97\pm.16}L_\odot$ among the 222 best fits out of a grid
of 200,000 model SEDs (fits with $\chi^2<5000$).  The luminosity is very well
constrained, varying only modestly to $L=10^{2.99\pm.15}L_\odot$ for the 904
best fits ($\chi^2 < 10000$),  while the mass shifts down to $6.5\pm1.0\msun$.
The mass estimate may be biased by the lower number of high-mass models
computed.  The star's mass is most compatible with a main sequence B4V star,
though its luminosity is closer to a B5V star.  The disk mass is constrained to
be $>10^3 \msun$.  The age is reasonably well constrained to be $T =
10^{5.78\pm.12}$ for the best 904 models, but is essentially unconstrained for
the best 222.  Similarly, the stellar temperature is entirely unconstrained by
the fitting process.

The very high values of $\chi^2$ would normally be worrisome, but the $\chi^2$
statistic only represents statistical error, while the data is dominated by
various systematic errors including calibration offsets in the optical/NIR and
poor resolution in the far-IR.  Therefore, it is not possible to find a perfect
model fit, but still possible to put constraints on the physical properties of
the source.

\Figure{figures_ch05358/HA_COcontours_COscale}{The \ha\ image with CO contours at
redshifted, blueshifted, and middle velocities in red, blue, and green
respectively.  Contours are at 2,4,8,12,20 K \kms\ for the red and blue, 
and 20,25,30,40,50,60,70 K \kms\ for the green.  Red is integrated from
-12 to -4 \kms, Blue from -31 to -21 \kms, and green from -21 to -12 \kms.
}{fig:HA_with_CO}{1.0}{}

% \Figure{figures_ch05358/HA_Spitzercontours_Spitzerscale}{\ha\ image with Spitzer 8\um\ contours
% that trace out the edge of the \ha\ emission}{fig:HA_with_8um}{1.0}


%\subsection{South of \region}
{\it South of \region}: 
There is a symmetric flow with one faint \hh\ knot and a bright central source
about 4\arcmin\ south of \region.  The \hh\ knot is at J(2000) = 05:39:15.63 +35:42:13.2.
The flow has a clear red and blue region as identified in figure
\ref{fig:cofig}; the red flow extends from -9 to -14 \kms\ and the blue from -19
to -23 \kms\ (the outflow is swamped by ambient emission in the intermediate
velocity range).  The outflow is $\sim 2\arcmin$ long, though the probable
source identified is not directly between the two lobes.  The ellipses used are
labeled in table \ref{tab:comeas} as Red S and Blue S.



\subsection{Imaging results: Optical}
Deep [S II] images show that some of the outflows have pierced through the
obscuring dust layers or excited extremely bright sulfur emission.
\citet{khanzadyan2004} knot N1 at the end of Outflow 1 is visible [S II] emission
%in figure \ref{fig:sii}.  
The bow of outflow 4 and the northwest end of outflow 6 are
detected in [S II].  Only the Outflow 1 and 4 bows are detected in \ha\ 
emission, indicating that the emission is most likely from shock heating, not
external photoionizing radiation.  If the shocks were externally irradiated, we
would expect the emission to be dominated by the recombination lines.  Because
they have been detected in the optical, these two flows can be classified
as Herbig-Haro objects.


% \begin{figure*}
%   \epsscale{1.0}
%     \plottwo{s233_SII_on_outflows}{s233_halpha_on_outflows}
%     \caption{a. [S II] and b. \ha\ contours overlaid on the \hh\ image.
%     [S II] emission is seen at the bows of outflows 1, 4, and 6 as labeled in figure
%     \ref{fig:outflowsh2}.  \ha\ emission is only seen from Outflow 1 and the
%     emission nebula near IRAS 05358+3543}
%     \label{fig:sii}
% \end{figure*}


\subsection{CO results}

\region\ is located at the center of the CO 3-2 integrated velocity maps
(Figure \ref{fig:cofig}).  The parent molecular cloud, centered at $v_{LSR} =
-17.5$ \kms ,  extends from the southeast towards the northwest with the
brightest emission coming from the core associated with \swcluster, while the
highest integrated emission is associated with \necluster.  \necluster\ has a central
velocity of $\sim -16.0$ \kms\ from the optically thin \ceighteeno\ 2-1 measurements.
Material that has been swept up and accelerated by jets and outflows can be
seen at velocities $v_{LSR} < -21$ \kms and $v_{LSR} > -12$ \kms\ (Figure
\ref{fig:cofig}).  The integrated CO 3-2 map peaks at J(2000) = 05:39:12.8
+35:45:55, while the highest observed brightness temperature is at J(2000) =
5:39:09.4 +35:45:12.  This offset is discussed in the context of CO isotopologues in
section \ref{sec:co21} and in section \ref{sec:discussion-outflows} and shown in
figures \ref{fig:scuba_co21} and \ref{fig:co21map}.

Regions with line wings relative to the ambient cloud within 5\arcmin\ of the
northeast cluster were assumed to be associated with outflows from the cluster.
Further than 5\arcmin, it is likely that the high velocity wings are accelerated by
neighboring HII regions (see section \ref{sec:surroundings}).  These line
wings were integrated over the velocity range -34 to -21 \kms\ (blue) and -12 to
1 \kms\ (red) to acquire estimates of the outflowing mass under the assumption
that outflowing gas is optically thin.  The extracted regions are displayed in
Figure \ref{fig:cofig}b and measurements in table \ref{tab:comeas}.  The line
wings in the central arcminute and central 5 arcminutes were measured for
comparison with lower resolution data and to compute a total outflow mass in
the central region, though they are most prominent in the inner 12\farcs\ (see
Figure \ref{fig:co21_all3}).
.  %This measurement is displayed in Figure
% \ref{fig:s233bco32}.

The objects in Table \ref{tab:comeas} labeled CO Region 1, 2, and 3 have
uncertain associations with outflows.  CO Region 1 is tentatively associated
with outflow 11. CO region 2 may be associated with Outflow 3 but is in a
highly confused region and may have many contributors.  CO region 3 is probably
associated with outflow 10.  In contrast, the associations with outflows 4 and
6/7 are more certain because they are further from the central region and less
confused.   Outflow 1 is seen at high velocities in \citet{beuther2002}
interferometer maps.  Outflow 9 is selected primarily based on CO emission.


\Figure{figures_ch05358/CO3-2_ellipses}
{
The JCMT HARP CO J=3-2 map integrated over all velocities with significant
emission (-34 \kms\ to -4 \kms) shown in gray log scale from 0 to 150 K \kms.
The elliptical regions over which line wings were integrated are shown with
blue and red circles corresponding to blue and red line wings.  The
measurements are presented in table \ref{tab:comeas}.}{fig:cofig}{1.0}{}



\Figure{figures_ch05358/scuba_with_CO21contours}{SCUBA 850\um\ image in linear grayscale from
-1 to +10 mJy/beam, with a saturated peak of 24 mJy/beam, with \twelveco\ 2-1
(orange solid, contours at 45,60,85,100,115,130,145 K \kms) and \thirteenco\
2-1 (green dashed, contours at 20,30,40,45 K \kms) integrated contours.  The
box shows the region plotted in Figure \ref{fig:co21map}.}{fig:scuba_co21}{1.0}{}

\Figure{figures_ch05358/co21_plotmap}{CO spectra of \necluster\ in \twelveco\ (blue),
\thirteenco\ (green), and \ceighteeno\ (red).  The top-left plot is the pixel
centered at J(2000) = 5:39:13.67 +35:46:26.0 and each pixel is 10\arcsec\ on a side. 
The region mapped here is shown with a box in Figure \ref{fig:scuba_co21}.
Redshifted self-absorption, a possible infall tracer, is evident in the \twelveco\
spectra in the outer pixels.  The inner pixels show self-absorption only at
central velocities: this may be an indication that emission from outflows
dominates any infall signature, or simply that there is no bulk infall towards
\necluster.}{fig:co21map}{0.75}{}

\Figure{figures_ch05358/co21_all3_12as}{CO spectra of inner
12\arcsec\ centered on \necluster\ for all observed CO lines.   The CO 3-2 and
2-1 beams are not matched, but in both cases the area integrated over is 1-2
resolution elements across.  The divisions demarcating the red and blue line
wings are shown with vertical dashed lines at $v_{LSR}=-21$ and -12 \kms.
}{fig:co21_all3}{1.0}{}
{}

\begin{deluxetable}{lccccc}
\tablewidth{0pt}
  \tabletypesize{\footnotesize}
    \centering
    \tablecaption{Measured properties of CO flows
      \label{tab:comeas}
    }
\tablehead{ \colhead{\tablenotemark{a}Region Name}  & \colhead{$\int T_{mb}*$}  & \colhead{$M (M_\odot)$}  & \colhead{p ($M_\odot$ \kms)}  & \colhead{N (\persc)} & \colhead{E ($10^{42}$ erg)}}
% & \colhead{redintegrated} & \colhead{redmass} & \colhead{redmomentum} & \colhead{redenergy} } 
\startdata
% blue1 - keep, blue only: associated with outflow 4
\tablenotemark{b}A. Outflow 4a    &      4.27  &     .022  &     .15  &  1.4\ee{19}  & 11 \\
% blue 4 - keep, associated with outflow 4, blue only
\tablenotemark{b}B. Outflow 4b   &      4.60 & .032 & .21 &   1.5\ee{19}  & 13 \\
% blue 9 - keep, blue only, probably outflow 1
\tablenotemark{b}C. Outflow 1n   &      14.5 & .088 & .71 & 4.8\ee{19} & 66 \\ 
% blue 2 - keep, associated with outflow 6/7, blue only
\tablenotemark{b}D. Outflow 6/7  &       4.45 & .045 & .30 & 1.5\ee{19} & 29 \\
%red7 - keep, no association but same direction as Outflow 2
\tablenotemark{r}E. CO Region 3      &    1.31 & .016 & .112 & 4.3\ee{18} & 8.5 \\
% blue3 - remove, not associated with cluster
% blue3     &       9.525039  &     0.659057  &      3.36178  &  2.12468e+44   &     5.01441  &     0.346957   &     1.80229  &  1.01071e+44  \\
% blue 6 - keep, red and blue, use for central outflow
\tablenotemark{b}F. \necluster\ &      41.8 & .464 & 3.72 & 1.4\ee{20} & 330 \\
\tablenotemark{m}F. \necluster\ &      132.9& 1.47 & -    & 4.4\ee{20} & -\\
\tablenotemark{r}F. \necluster\ &      30.0 & .333 & 2.03 & 9.9\ee{19} & 135 \\
% blue 7 - keep, may be outflow 1 south counterflow
\tablenotemark{b}G. Outflow1s   &      14.6 & .064 & .48 & 4.8\ee{19} & 40 \\
%red5 - clear red excess, no clear association
\tablenotemark{r}H. CO Region 2      &    4.54 & .012 & .074 & 1.5\ee{19} & 5 \\
% blue 5 - keep, associated with outflow 9, blue only
\tablenotemark{b}I. Outflow 9    &      6.33 & .039 & .39 & 2.1\ee{19} & 43\\
% blue 8 - keep, blue only, no clear association
\tablenotemark{b}J. CO Region 1  &     3.61 & .015 & .12 & 1.2\ee{19} & 11\\
\tablenotemark{r}K. Red S   &     5.26 & .051 & .34  & 1.7\ee{19} & 26 \\
\tablenotemark{b}L. Blue S  &     3.66 & .053 & .47  & 1.2\ee{19} & 47 \\
% blue 10 - get rid of it
%blue10    &       30.399808 &      0.669383 &       4.35128 &   3.40323e+44  &      46.7147 &       1.02863  &       6.0041 &   3.86866e+44  \\
% blue 11 - remove, no clear association, blue only
%blue11    &        2.641900 &     0.0865521 &      0.438263 &   3.18589e+43  &     0.984461 &     0.0322523  &     0.233822 &   1.88811e+43  \\
% red 1 - south red, keep
% actually, ignore - don't discuss these South Source 1      &    10.6623   &    0.696333    &    4.21276   &  2.9246e+44  \\
% red 2 - too faint / unclear.  rm
%red2      &      2.822263   &   0.0320534   &    0.221736   &  1.9821e+43    &    9.40664   &    0.106834    &   0.579233   & 3.61028e+43  \\
% red 3 - too ambiguous again, not obviously red
%red3      &      6.267010   &    0.104144   &    0.716924   &  5.6371e+43    &    9.23909   &    0.153533    &   0.733901   & 4.04048e+43  \\
%red4 - rm, already covered
%red4      &     38.240242   &     1.97369   &      11.732   & 8.29297e+44    &    64.5034   &      3.3292    &    20.2909   & 1.37858e+45  \\
%red6 - rm, already covered
%red6      &     16.685583   &     6.29804   &     36.7241   & 2.56084e+45    &    24.2383   &     9.14886    &      53.33   & 3.47148e+45  \\
\tablenotemark{b}1\arcmin\ aperture\tablenotemark{c}& 15.1 & .96 & 7.6 & 5.0\ee{19} & 670\\
\tablenotemark{b}3\arcmin\ aperture& 2.7  & 1.6 & 12 & 9.0\ee{18} & 1000\\
\tablenotemark{b}5\arcmin\ aperture& 1.7  & 2.7 & 20 & 5.6\ee{18} & 1600 \\
\tablenotemark{r}1\arcmin\ aperture & 11.8 & 0.75 & 4.7 & 3.9\ee{19} & 320\\
\tablenotemark{r}3\arcmin\ aperture & 1.9 & 1.1 & 6.8 & 6.2\ee{18} & 460\\
\tablenotemark{r}5\arcmin\ aperture & 0.96 & 1.5 & 10 & 3.2\ee{18} & 640\\
\tablenotemark{b}1\arcmin\ $^{12}$CO 2-1 & 10.4 & .94 & 7.1 & 4.9\ee{19} & 590 \\
\tablenotemark{m}1\arcmin\ $^{12}$CO 2-1 & 97.78 & 8.83 & - & 4.6\ee{20} & - \\
\tablenotemark{r}1\arcmin\ $^{12}$CO 2-1 & 9.17 & 0.83 & 5.52 & 4.3\ee{19} & 430 \\
\tablenotemark{m}1\arcmin\ $^{13}$CO 2-1 & 41.12 & 211  & - & 1.1\ee{22} & - \\
\tablenotemark{m}1\arcmin\ C$^{18}$O 2-1 & 5.31 & 271 & - & 1.4\ee{22} & - \\
\enddata
\tablenotetext{a}{Unless labeled otherwise, regions are extracted from CO 3-2 data as shown in figure \ref{fig:cofig}b }
\tablenotetext{b}{Blue integration over velocity range -34 to -21 \kms}
\tablenotetext{c}{Apertures are centered on J(2000) = 05:39:11.238 +35:45:41.80 in \necluster}
\tablenotetext{r}{Red integration over velocity range -13 to -4 \kms}
\tablenotetext{m}{Middle range integration over -21 \kms\ to -13 \kms.  Assumed not to be outflowing, so no momentum is computed}
\end{deluxetable}


\subsection{Near-infrared spectroscopy: Velocities}
\label{sec:tspecresults}

The slit positions used and apertures extracted from those slits are displayed
in Figure \ref{fig:tspecslits}.  Position-velocity diagrams of the 1-0 S(1)
line are displayed in Figure \ref{fig:outflows_h2_pv}.  Velocity measurements
are presented in Table \ref{tab:OutflowH2} and derived properties in Table
\ref{tab:table4}.
% and the velocity fits to the 1-0 S(1) line are shown in figure \ref{fig:h2fits}.

\Figure{figures_ch05358/tspec_slits_apertures_on_h2}{TripleSpec slits (blue) overlaid on the
\hh\ image.  The red boxes indicate the apertures extracted from those slits to
fit and measure \hh\ properties.  The apertures are also indicated in the
position-velocity diagrams.}{fig:tspecslits}{0.75}{}

The near-IR spectrum of Outflow 1 has the largest signal.  All of the K-band
\hh\ lines except the 2-1 S(0) 2.3556 \um\ (too weak) and 1-0 S(4) 1.8920 \um
(poor atmospheric transmission) lines were detected (see Table
\ref{tab:nirmeas}).  Velocities from gaussian fits to each line are reported.
In the central portion of \necluster, outflowing \hh\ emission at
$v_{LSR}\approx-30$ \kms\ is detected.  This material may be associated  with a
line-of-sight flow, or may originate from the base of the already identified
flows 1-3.  In source IR 58,  Br$\gamma$ and He I 2.05835\um are detected,
indicating that there is an embedded PDR in this source.  There is a hint of a
second, fainter star adjacent to IR 58.  IR 93 is observed to be a double
source in the TripleSpec spectrum, but the spectrum is too weak to do any
identification.  Br$\gamma$ and possibly He I are detected at fainter levels.

% 2.06036 HeI?
% 2.16593 BrG

Table \ref{tab:nirmeas} shows the measured line strengths (when detected) of
all \hh\ lines in each aperture.  The errors listed are statistical errors
that do not include the systematics errors introduced by a failure to correct
for narrow atmospheric absorption lines.

\Figure{figures_ch05358/fivepvdiagrams}{Position-velocity diagrams of the \hh\ 2.1218 \um\ line in Outflows
1,2, 4, IR 6, and IR93/IR58.  The velocity range is from -340 to 190 \kms.}{fig:outflows_h2_pv}{1.0}{}
  
\Table{cccccc}
{TripleSpec fitted \htwo\ outflow velocities}
{Outflow Number & Aperture Number &  \tablenotemark{a}v$_{LSR}$ (\kms) &\tablenotemark{b}v$_{LSR} (\kms)$ }
{tab:OutflowH2}
{
1    & 1 & -33.54 (0.15)    & -31.85 (0.32) \\ %-32 (5)  \\
1    & 2 & -13.60 (0.57)    & -13.56 (0.96) \\ %-14 (5)  \\
1    & 3 & -40.51 (0.41)    & -36.13 (0.81) \\ %-36 (5)  \\
1    & 4 & -88.7  (2.8)     & -83.7  (7.9)  \\ %-83 (10) \\
2    & 1 & -82.6  (7.6)     & -81    (21)   \\ %-81 (25) \\ % width 64 +/- 20 - wide!
2    & 2 & -30.41 (0.57)    & -28.9  (1.4)  \\ %-29 (5)  \\
2    & 3 & -33.89 (0.62)    & -35.2  (3.7)  \\ %-35 (10) \\
4    & 1 & -73.34 (0.48)    & -70.2  (1.1)  \\ %-70 (10) \\
4    & 2 & -64.08 (0.61)    & -67.8  (2.2)  \\ %-67 (10) \\
IR6  & 1 & -39.4  (1.6)     & -39.4  (4.2)  \\ %-39 (10) \\
IR93 & 2 & -26.07 (0.43)    & -26.85 (0.97) \\ %-27 (5)  \\  % wider than other lines
IR93 & 3 & -30.6  (1.5)     & -32.0  (2.5)  \\ %-32 (5)  \\  % wider than other lines
IR93 & 4 & -29.14 (0.77)    & -30.3  (2.1)  \\ %-30 (5)  \\   % wider than other lines
IR93 & 6 & -47.7  (7.9)     & -71    (37)   \\ %-45 (15) \\   % manual fit - can't be reproduced easily
}{
\tablenotetext{a}{Measured from \hh\ 1-0 S(1) 2.1218 \um\ line}% as shown in figure \ref{fig:h2fits}}
\tablenotetext{b}{Measured from all detected \hh\ lines fit with model described in section \ref{sec:tspecresults}}
}                



\clearpage
\begin{deluxetable}{ccccccccc}
    \centering
    \tablewidth{0pt}
  \tabletypesize{\scriptsize}
    \tablecaption{Measured properties of \hh\ flows}
%     \tablehead{
% 	\colhead{Outflow} & \colhead{Center}\tablenotemark{a} & \colhead{PA}\tablenotemark{b} 
% 	& \colhead{Length}\tablenotemark{c} & \colhead{Source} \tablenotemark{d} & \colhead{Pos Length} \tablenotemark{e}
% 	& \colhead{Neg Length} \tablenotemark{e} & \colhead{Time (50\kms)}\tablenotemark{f} & \colhead{LOS} \tablenotemark{g}} 
    \tablehead{
    Outflow  & \tablenotemark{a}Center  & \tablenotemark{b}PA  & \tablenotemark{c}Length  & 
    \tablenotemark{d}Source  & \tablenotemark{e}Flow  & \tablenotemark{e}Counterflow  & \tablenotemark{f}Age  & \tablenotemark{g}LOS  \\
    & & & & & Length & Length & (50 \kms) & Velocity \\
    }
\startdata
1 & 05:39:13.023 +35:45:38.66 & -13.3     &142.3"   &mm2?     &58         &84.2        &1.4e4         &-    \\
2 & 05:39:13.058 +35:45:51.28 & -47.0     &44.6"    &mm1a     &44.6       &-           &6.6e3         &Blue \\
3 & 05:39:12.48  +35:45:54.9  & -62       &44"      &mm3?     &44         &-           &6.5e3         &Red  \\
4 & ambiguous                 & 17.8-21.8 &141-144" &  ?      &141-144    &-           &2.1e4         &Blue \\
5 & 05:39:12     +35:45:51    & 170       &38-48    &mm3?     &38-48      &-           &6.5e3         &Blue \\
6 & 05:39:09.7   +35:45:17    & 14.5      &197      &Q5\tablenotemark{h} &197        &-           &2.9e4         &Blue \\
8 & 05:39:10.002 +35:45:10.87 & -154.6    &105.5"   &IR41?    &54.7       &52.9        &7.9e3         &-    \\         
\enddata
\tablenotetext{a}{Midpoint of bipolar outflow if symmetric, position of jet
source candidate if asymmetric}
\tablenotetext{b}{Position angle uncertainties are $\sim 5\arcdeg$ because they
are not perfectly collimated, causing an ambiguity in their true directions.
The exact angles used to draw vectors in figure \ref{fig:outflowsh2} are listed
for reproducibility.}
\tablenotetext{c}{Total length of outflow on the sky, including counterflow}
\tablenotetext{d}{Candidate jet source object.  Outflows 2 and 6 have clear
associations, the others are weaker candidates.}
\tablenotetext{e}{Flow length is the distance from the CENTER position to the
last \hh\ knot in the position angle direction as listed.  Counterflow length
is the distance from the CENTER position to the opposite far knot.}
\tablenotetext{f}{Timescale of jet assuming it is propagating at 50 \kms, an
effective lower limit to see \hh\ emission.  If two lengths are available, uses
the longer of the two.  These are lower limits to the true timescale
\citep{parker1991}.}
\tablenotetext{g}{The parity of the outflow along the line of sight.  Outflow 1
and 8 have counterflows with parities as indicated in figure
\ref{fig:outflowsh2}}
\tablenotetext{h}{\citet{qiu2008} \region source 5. }
\end{deluxetable}



\subsection{Spectroscopic Results: Optical}
\label{sec:dis}

IR 6 and IR 41 (objects 1 and 6 in Figure \ref{fig:outflowsh2}) both show
\ha\ in emission.  IR 41 is close to the reflection nebula in the southeast
portion of \region\ and is probably the reflected star.   The reflection
nebula's spectrum is very similar to IR 41's spectrum at \ha\ in both width
and brightness (see Figure \ref{fig:outflow10_pv}).
%, and \ref{fig:outflow10_spec}).  

\Figure{figures_ch05358/Outflow10_IR6_41_color}{A position-velocity diagram of IR 6 and 41
including the reflection nebula near IR 41 ($\approx 7.4\msun$).  IR 6 shows a
two-peaked \ha\ emission profile, but is the less massive ($\approx 4.5\msun$) of
the pair.  The separation between the two sources is 55\farcs3, and each pixel
is 0\farcs4.}{fig:outflow10_pv}{0.25}{}

% \FigureTwo{outflow10_redap02}{outflow10_redap01}{Red spectra of the two
% stars in the Outflow 10 slit.  Left: IR 6 spectrum.  \ha\ absorption wings
% are evident around the bright \ha\ emission.  Right: IR 41
% spectrum.  \ha\ is clearly detected.  Measurements are presented in
% Table \ref{tab:optical}. }{fig:outflow10_spec}{1.0}

There are three
components in the \ha\ profile of IR 6: a broad absorption feature seen far
($\sim400\kms$ from the line center) on the wings and two emission peaks.  The
peaks are separated by 190 \kms\ and the blueshifted peak is weaker than the
redshifted (Table \ref{tab:IR6}).  The H$\beta$ profile shows much deeper
absorption and weaker emission but with similar characteristics.   The presence of the \ha\ emission
makes identification of the stellar type from the \ha\ line profile uncertain.
The derived extinction to IR 6 is at least $A_V=7$ from an assumed \ha/\hb\
ratio of 2.87 \citep{agnsquared}.  The \hb\ flux was measured from zero to the
peaks of the emission profile and therefore probably overestimates the
\hb\ flux and underestimates the extinction.

Additional lines detected in the outflows, ambient medium, and stars
are presented in Table 8.  The association of [S II] and [O I] emission
with the star forming region is uncertain and may be foreground emission.

\Table{cccccccc}
{IR 6 Deblended Profiles}
{& \tablenotemark{a} Blue & \tablenotemark{b}Blue & \tablenotemark{a} Red & \tablenotemark{b}Red & Absorption &
Gaussian / & \tablenotemark{b}Absorption \\
&Emission & Wavelength & Emission & Wavelength & & Lorentzian FWHM & Wavelength \\
}
{tab:IR6}
{\ha\ & 4.4\ee{-14} & 6559.79 & 1.3\ee{-13} & 6564.23 & -2.6\ee{-14} & 1.5 / 0.19 & 6563.02 \\
\hb & \tablenotemark{c}2.4\ee{-14} & 4857.68 & \tablenotemark{c}1.8\ee{-14} & 
    4864.28 & \tablenotemark{d} -4.6\ee{-14} & 0.17 / 16.5 & 4861.91 \\}
{
\linebreak
\tablecomments{Measurements are made using a Voigt profile fit in the IRAF {\sc splot} task.}
\tablenotetext{a}{Flux measurements are in units of erg s$^{-1}$ \persc \AA$^{-1}$}
\tablenotetext{b}{Wavelengths are in Geocentric coordinates.  Subtract 0.53\AA\ from \ha\
and 0.39\AA\ from \hb\ to put in LSR coordinates.}
\tablenotetext{c}{\hb\ emission was measured assuming a continuum of zero and therefore represents
an upper limit in the \hb\ emission}
\tablenotetext{d}{\hb\ deblending may contain systematic errors
from a guessed subtraction of the \hb\ emission}
}

\Table{cccccccc}
{Lines observed in the optical spectra}
{Source & \ha\ & \hb\ & [S II] 6716\AA & [S II] 6731\AA & [O I] 6300\AA & [O I] 6363\AA &  [O I] 5577\AA }
{tab:optical}
{ 
Outflow1 ap1                   & 4.3\ee{-16} & - & 5.7\ee{-16} & 6.3\ee{-16} & 5.3\ee{-16} & 1.8\ee{-16} & - \\
                               & 6561.49     & - & 6715.3      & 6729.6      & 6299.7      & 6363.3      & - \\
Outflow1 ap2                   & 4.5\ee{-16} & - & 4.5\ee{-16} & 4.6\ee{-16} & 3.1\ee{-16} & 1.2\ee{-16} & - \\
                               & 6561.22     & - & 6714.9      & 6729.3      & 6299.4      & 6363.2      & - \\
Ambient Medium - slit 1        & 6.7\ee{-17} & 5.3\ee{-18}   & 1.0\ee{-17} & 7.9\ee{-18} & 3.5\ee{-16} & 1.2\ee{-16} & 4.8\ee{-17}   \\ % 459 pixels blue
                               & 6562.87     & 4861.7        & 6716.7      & 6731.2      & 6300.3      & 6363.8      & 5578.0  \\ % 178 pixels red 1, 314 red 2
IR 41 nebula                   & 2.6\ee{-15} & - & - & - & 4.4\ee{-16} & 1.9\ee{-16} & - \\
                               & 6562.85     & - & - & - & 6300.3      & 6363.9      & - \\
IR 41                          & 6.5\ee{-15} & - & - & - & 1.1\ee{-16} & 7\ee{-17}   & - \\
                               & 6562.9      & - & - & - & 6300.0      & 6363.3      & - \\
IR 6                           &  1.76\ee{-13} & \tablenotemark{a} 4.1\ee{-14}    &-&-&-&-   &  \\ %2.9\ee{-16} \\
                               & -             & -                                &-&-&-&-   &  \\ %5578.5      \\
}{
\linebreak
\tablecomments{
Wavelengths listed are in \AA\ and are geocentric.  To convert to LSR
velocities, subtract 24.35 \kms.  The ambient medium fluxes represent averages
across the slit.
Fluxes are in erg s$^{-1} \persc $\AA$^{-1}$. 
}
\tablenotetext{a}{\hb\ measurement in IR 6 is an upper limit}
}


% Use temden & OI (6300+6363)/5577 to get temp...
% 5577: 5577.99, 1.07e-14
% 6300: 6300.34, 6.25e-14
% 6363: 6363.81, 2.09e-14
% exp(-ccm_a_opt(.5577)*.47/1.086) = 0.65345832487876254
% exp(-ccm_a_opt(.6363)*.47/1.086) = 0.6926005572989089
% exp(-ccm_a_opt(.6300)*.47/1.086) = 0.68965907033077245
%
% measurements:
% HBeta = 3.57e-15, 4861.32 = -.617 \kms
% HAlpha = 1.19e-14, 6562.87 = 2.74 \kms
% Ratio = 3.33....
% A_halpha = 

% \FigureTwo{outflow1_redap00}{outflow1_redap01}{Outflow 1 spectra.  Left: Front
% Right: Rear.  The slit orientation and position used were the same as the
% TripleSpec slit position, though longer.  No emission was detected from the
% counterflow. [need figure to define apertures?] }{fig:outflow1_spec}{1.0}



\subsection{Radio Interferometry}
\label{sec:vlaresults}
A point source was detected in the X, U, K and Q band VLA maps with high
significance at the same location as the X-band point source reported in
\citet{beuther2007}.  Seven-parameter gaussians were fit to each image to measure
the beam sizes and positions and flux densities.  The measurements are listed
in Table \ref{tab:vla}. The locations of the point source and the shape of the
beams from the re-reduced X and Q band images are displayed in figure
\ref{fig:mm1adiagram}. 
A Class II 6.7 GHz methanol maser was detected in \region\ by \citet{Menten1991}.
It was observed with the European VLBI Network (EVN) by \citet{Minier2000} and 
seen to consist of a linear string of maser spots that trace a probable disk in
addition to maser spots scattered around a line perpendicular to the proposed
disk (see Figure \ref{fig:mm1adiagram}).
The VLA source is more than a VLA beam away from the VLBI CH$_3$OH maser disk
identified by \citet{Minier2000}.  It is to the southeast in the opposite
direction of Outflow 2.  Outflow 2 is at position angle -47$\degree$, while the
disk is at PA 25\degree.  The 8$\degree$ difference from being perpendicular is
well within the error associated with determining the angle of the outflow in
this confused region, so the VLBI disk is a strong candidate for the source of Outflow 2.


\begin{figure*}[htpb]
  \epsscale{0.75}
  \plotone{figures_ch05358/figure_mm1a_pythondiagram}
  \caption{A diagram of the region surrounding mm1a from \citet{beuther2007}.
    The ellipses are centered at the measured source centers and their sizes
    represent the beam sizes of the Plateau de Bure interferometer at 1.2mm
    \citep[blue,][]{beuther2007}, Gemini MICHELLE at 7.9\um\
    \citep[red,][]{Longmore2006}, the VLA at 3.6cm (green), and the VLA at 7mm
    (orange). The maser disk was measured with the European VLBI Network by
    \citet{Minier2000}, so the size and direction of the disk are very well
    constrained.  The black circle is centered on the pointing center of the
    VLBI observation and represents the absolute pointing uncertainty.  The
    arrow pointing in the direction of Outflow 2 traces clumps along the
    outflow back to the mm emission region.  The vector is not to scale -
    Outflow 2 is about 45\arcsec\ long.
\label{fig:mm1adiagram}}
\end{figure*}

The astrometric uncertainty in VLA measurements are typically
$\lesssim0.1$\arcsec. Different epochs of high-resolution X-band and Q-band
data confirmed that the pointing accuracy is substantially better than
0.1\arcsec\ in this case.  The VLBI absolute pointing uncertainty is reported
to have an upper limit of $\sim 0.03$\arcsec\ \citep{Minier2000}.  The
separation between the VLA Q-band center and the VLBI disk center is
0.22\arcsec, whereas the separation between the combined X and Q band pointing
centers is only 0.027\arcsec, which can be viewed as a characteristic
uncertainty. This is evidence for at least two distinct massive stars in a
binary separated by $\sim$ 400 AU.  While the statistical significance of the
binary separation is quite high using formal errors, the systematic errors
cannot be constrained nearly as well.  This object is a candidate binary system
but is not yet confirmed.

%, and a possible third much younger member of the system in the
% hot core mm1a associated with the \citet{Longmore2006} mid-infrared source.
% However, the possibility of a third source is much less certain because the
% \citet{Longmore2006} observations were referenced to the VLA pointing and the
% uncertainty in the mm source location is enough to encompass the VLA source.

\begin{deluxetable}{lllll}
    \tablewidth{0pt}
    \tablecolumns{7}
    \tabletypesize{\footnotesize} %DO NOT include this in the caption!!!
    \tablecaption{VLA measurements near IRAS 05358 mm1a
        \label{tab:vla}
    }
%    \rotate
%\footnotesize
\tablehead{
\colhead{Frequency } & \colhead{Beam major /} & \colhead{RA (error)}   & 
\colhead{Peak flux }  & \colhead{Map RMS }   \\
\colhead{Observed} & \colhead{minor / PA} & \colhead{Dec (error)} &  \colhead{(error)} & \colhead{(mJy/beam)} \\ }
 % \colhead{Deconvolved source major / minor / PA } &   % all sources are unresolved...
\startdata
43.3 GHz & 0.022\arcsec\  / 0.029\arcsec\ / -10.4 \degree & 05:39:13.065425 (0.000015) &  1.319 (0.027) & 0.179 \\
                                                         &&35:45:51.14732 (0.00031)    &  \\
22.5 GHz &  1.52\arcsec\  / 1.28\arcsec\  / 232 \degree    & 05:39:13.05521 (0.0029)    &  1.26 (0.04)   & 0.091 \\
&&35:45:51.378 (0.046)        &\\
15.0 GHz & 1.58\arcsec\   / 2.00\arcsec\  / 0 \degree      & 05:39:13.062 (0.005)       &  1.274 (0.065) & 0.124 \\
&&35:45:51.4 (0.1)            &\\
8.4 GHz  &  0.107\arcsec\ / 0.122\arcsec\ / 7.9 \degree  & 05:39:13.064548 (0.000036) &  0.506 (0.003) & 0.015 \\
&&35:45:51.170356 (0.000613)  &\\
\enddata
\tablenotetext{.}{Errors reported here are fit errors.  Absolute flux calibration
errors are negligible for the X-band data but are about equivalent to measurement errors
for the K, and U bands and dominant in the Q band}
\end{deluxetable}


\section{Analysis}

\subsection{Near-Infrared Spectroscopic Extinction Measurements} 
Extinction along a line of sight can be calculated using the 1-0 Q(3) / 1-0 S(1) line ratio. 
\begin{equation}
    A_\lambda = 1.09 \left[ -\textrm{ln}
    \frac{S_\nu(S)/S_\nu(Q)}{A_{ul}(S)\lambda_Q/A_{ul}(Q) \lambda_S} \right] \left[
    \left(\frac{\lambda_S}{\lambda_Q}\right)^{-1.8} -1 \right]^{-1}
\end{equation}
Because they are from the same upper state,
their intensity ratio should be set by their Einstein A values times the
relative energies of the transitions.  %We use the lower-signal 1-0 Q(4) / 1-0 S(2) and 1-0 Q(2) /
%1-0 S(0) ratios as `sanity checks' when detected.
However, as shown by
\citet{luhman1998}, narrow atmospheric absorption lines in the long wavelength
portion of the K band, where the Q branch lines lie, can create a significant
bias.  Because the lines have not been corrected for atmospheric absorption,
the Q branch fluxes should actually be lower limits.  Since the 1-0 S(1)
transition at 2.1818 microns is affected very little by atmospheric absorption,
and the exinction measured is proportional to the Q/S line ratio, the measured
extinction should be a lower limit.

%Interesting but requires more work
The [Fe II] 1.6435 and 1.2567 \um\ lines were detected in Outflow 1, allowing for
another direct measurement of the extinction.  The measured ratio $FR$ =
1.26\um/1.64\um\ in Outflow 1 was .8, while the true value is at least 1.24 but
may be as high as 1.49 \citep{Smith2006,luhman1998,Giannini2008}.  The
extinction measured from this ratio ranges from $A_V = 4.1$ ($FR=1.24$) to 5.8
($FR=1.49$).  The S(1)/Q(3) ratio uncorrected for telluric absorption is .91,
which yields an extinction lower limit of $A_V = 18.7$, is inconsistent with
the measurement from [Fe II].  The \ha\ detection and \hb\ upper limit give a
lower limit on the extinction of $A_V = 6.6$, which is consistent with both of
the other methods to within the calibration uncertainty.  

It is possible that the two measurements come from unresolved regions with
different levels of extinction, though a factor of at least 3 change over an
area $\sim 100 AU$ far from the millimeter cores seems unlikely.  A strong IR
radiation field could plausibly change the line ratio from the expected
Einstein A value.  The question is not resolved but may be possible to address
with near-IR observations of nearby bright HH flows with more careful
atmospheric calibration.


\subsection{Optical Spectra}
\subsubsection{Stellar Type}
IR 6 is suspected to be the source of the bright \hh\ finger at PA $\approx$
15\arcdeg.  %It is also an \ha\ emission source, having previously been
%identified in the IPHAS \ha\ survey \citep{Witham2008}.  
IR 6 is also
a 24\um\ source and was detected by MSX (designation G173.4956+02.4218).  We
identify this star as a Herbig Ae/Be star. 

\subsubsection{Density and Extinction Measurements}
The spectrum of knot N1 (the bow of Outflow 1) allowed a measurement of
electron density in the shocks from the [S II] 6716/6731  line ratio
%(figure \ref{fig:outflow1_spec})
.  Densities were determined to be $n=$ 700 \percc\  in the forward lump
and $n=$500 \percc\ in the second lump.  \ha, [N II] 6583, [O I] 6300, and [O
III] 6363 were also detected, but no lines were detected in the blue portion of
the spectrum presumably because of extinction.  The measured velocities from [S
II] are faster than the \hh\ velocity measurements at about $v_{LSR} = -68
\pm 5$ \kms. 

There is also an ambient ionized medium that uniformly fills the slit with a
[S II]-measured density $n_e=120\ \percc$.  Evidently, nearby massive stars are
ionizing the low-density ISM located in front of \region.   This material is
moving at velocity $v_{LSR} = -7 \pm 5$ \kms\ and is extincted by $A_V=1.5$ as
determined from \ha/H$_\beta=2.87$  assuming the gas is at 10$^4$ K.



\subsection{UCHII region measurement}
A uniform density, ideal HII region will have an intensity curve $I = I_0
( 1 - e^{-\tau_\nu})$ where 
\begin{equation} \tau = 8.235\times10^{-2}
\left(\frac{T_e}{K}\right)^{-1.35} \left(\frac{\nu}{\textrm{GHz}}\right)^{-2.1}
\left(\frac{\textrm{EM}}{\textrm{pc~cm}^{-6}}\right) a(\nu,T)
\end{equation}
following \citet{rohlfs2004} equation 9.35, where $a(\nu,T) \approx 1$ is a
correction factor.   By assuming an excitation temperature $T_{ex} = 8500$K, 
blackbody with a turnover to an optically thin thermal source was fit to the
centimeter SED.  The turnover frequency from this fit is $\tau=$15.5 GHz,
corresponding to an  emission measure $EM=7.4 \times 10^8$ pc cm$^{-6}$.  This
turnover frequency is lower than the $\sim35$ GHz reported by
\citet{beuther2007}.  The turnover is clearly visible in the U, K and Q data
points in figure \ref{fig:HIIregionfit}.  

By assuming the X-band emission is optically thick, a source size can be derived.
\begin{equation}
    \label{eqn:uchiirad}
    2 r = \left[\frac{S_\nu}{2 k_B T_{ex}}  \lambda^2  D^2\right]^{1/2}
\end{equation}
where D is the distance to the source.  Assuming a spherical UCHII region and a
distance of 1.8 kpc, the source has radius $r=$30 AU (for comparison, the
Q band beam minor axis is $\sim$90 AU, so the region could in principle be
resolved by the VLA + Pie Town configuration).  

The measured density is $n=(EM / r)^{1/2} = 2.2\times10^6$ \percc, with a
corresponding emitting mass $M = n \mu m_H 4/3 \pi r^3 = 1.0\times 10^{-6}$
\msun\ using $\mu=1.4$.  Using \citet{kurtz1994} equation 1,

\begin{equation}
  N_{Lyc} = (8.04\times10^{46} s^{-1}) T_e^{-0.85} \left(\frac{r}{pc} \right)^3 n_e^2
\end{equation}

the number of Lyman continuum photons per second required to maintain
ionization is estimated to be  $N_{Lyc} = 5.9\times10^{44}$ 
%\footnote{This equation assumes an optically thin HII region, but the source is optically
%thick at 3.6cm}
, a factor of $\sim4$ lower than measured by \citet{beuther2007} and
closer to a B2 ZAMS star ($\sim11\msun$) than B1 using Table 2 of
\citet{panagia1973}.  If the star has not yet reached the main sequence, it
could be significantly more massive \citep{hosokawa2009}, so our stellar mass
estimate is a lower limit.

% 2*6.67e-8*(11*2e33)/(1.5*1.38e-16*8500/1.67e-24) / 1.496e13
The gravitational binding radius of a 11 \msun\ star is $r_g = 2 G M / c_s^2 \approx
190$ AU (the HII region is assumed to be supported entirely by thermal
pressure, which provides an upper limit on the binding radius since turbulent
pressure can exceed thermal pressure).  The UCHII region radius of 30 AU is
much smaller, indicating that, under the assumption of spherical symmetry, the
HII region is bound.

\citet{leurini2007} noted that the CH$_3$CN line profile around this source
could be fit with a binary system with separation $<1100$AU and a total mass
of 7-22 \msun .  This is entirely consistent with our picture of a massive binary
system with a 11 \msun\ star in a UCHII region and another high mass star with
a maser disk.

There are no other sources in the \region\ region to a 5$\sigma$ limit of 0.075
mJy in the X-band, which provides the strictest upper limit.  From equation
\ref{eqn:uchiirad}, this corresponds to an optically thick source size of 24
AU.  The maser disk has a spatial extent of around 140 AU, so it is quite unlikely
that either an undetected UCHII region or the observed UCHII are associated with
the maser disk.

Assuming the same turnover point for undetected sources,  
an upper limit is set on $N_{Lyc}$ for undetected sources:
\begin{equation}
  N_{Lyc} = (8.04\times10^{46} s^{-1}) \left(\frac{S_\nu }{  2 k_B T_{ex}^{1.85} }  \lambda_{cm}^2  D_{pc}^2\right) EM
\end{equation}
Our 5$\sigma$ upper limit is $N_{Lyc} = 1.38\ee{44}$ s$^{-1}$, indicating that
any stars present must be a later class than B3, or lower than about 8 \msun .  
For an emission measure as much as 3 times higher, the corresponding stellar
mass would be less than 10 \msun .  It is likely that no other massive
stars have formed in \necluster.

After independently determining the best-fit UCHII model to the VLA data, we
included the PdBI data points from \citet{beuther2007} and fit a power-law to
both data sets.  If the emission measure was allowed to vary, the derived
parameters were $EM=6.3\ee{8}$ and $\beta=0.7$.  However, doing this visibly
worsened the UCHII region fit without significantly improving the power-law
fit, so the fit was repeated holding a fixed emission measure, yielding
$\beta=0.8$ (plotted in Figure \ref{fig:HIIregionfit}b).  This power-law is
much shallower than the $\beta=1.6$ measured by \citet{beuther2007} without
access to the 44 GHz data point, and suggests that there is a significant
population of large grains in source mm1a.  However, we caution that the fits
were performed only accounting for statistical errors, not the significant and
unknown systematic errors that are likely to be present in mm interferometric
data.  The PdBI beams are much larger than the VLA beams, so the larger beams
could be systematically shifted up by including additional emission, which
would reduce $\beta$.  Nonetheless, the new VLA data constrains the UCHII
emission to contribute no more than 10\% of the 3.1mm flux.

\begin{figure*}[htpb]
\epsscale{0.75}
\plotone{figures_ch05358/HIIregion_fit}
\plotone{figures_ch05358/HIIregion_plusdust_fit}
\caption{(a) The HII region fit to measured X, K, U, and Q band data.  Error bars
represent statistical error in the flux measurement.  The Q band error is
dominated by flux calibration uncertainty (see Table \ref{tab:vlatimes}).  The
measured turnover is at 9.5 GHz. (b) A fit to both the VLA data presented in this paper
and the (sub)mm points from \citet{beuther2007}.  The best fit spectral index for the 
dust emission is $\alpha=2.8$ ($\beta=0.8$), significantly lower than the $\alpha=3.6$
measured by \citet{beuther2007} without the 0.7 mm data point.}
\label{fig:HIIregionfit}
\end{figure*}


\subsection{Mass, Energy, and Momentum estimates from CO}

\subsubsection{Equations}
The column density for CO J=3-2 is estimated using the equation 
\begin{equation}
  \label{eqn:column}
N_{\hh} =
\frac{\hh}{\textrm{CO}}\frac{8\pi\nu^3k_B}{3c^3hB_eA_{ul}}(1-e^{h\nu/k_BT_{ex}})^{-1}
\frac{1}{\eta_{mb}} \int T_A^*(v) dv 
\end{equation}
where $A_{ul}=A_{32}=2.5\times10^{-6}\textrm{s}^{-1}$ and
$A_{21}=6.9\ee{-7}$s$^{-1}$\citep{turner1977}, the rotational constant $B_e =
57.64$, 55.10, and 55.89  GHz for \twelveco, \thirteenco, and \ceighteeno\ respectively,
$\eta_{mb} = .68$, and $T_{ex}$ is assumed to be 20K.  The partition function
is approximated as 
\begin{equation}
    Z=\sum_{J=1}^\infty (2J+1) exp
  \left(\frac{-J(J+1)hB_e}{k_B T_{ex}}\right) \approx \int_0^\infty (2J+1)exp
  \left(\frac{-J(J+1)hB}{k_bT_{ex}}\right) dJ
\end{equation}
which is valid when $T_{ex} >> hB_e/k_B \sim 2.8 $K.
Equation \ref{eqn:column} becomes 
\begin{equation} 
  N_{\hh} = ( 3.27\times10^{18} \persc) \frac{1}{\eta_{mb}} \int T_A^*(v) dv
\end{equation}
where the integrand is in units K \kms.  The mass is then 
\begin{equation}
  M = \mu\ m_{\hh}\ A\ N_{\hh} = 1.42\times10^{-5} A \frac{1}{\eta_{mb}} \int
  T_A^*(v) dv
\end{equation} 
where A is the area in cm$^2$, $\mu=1.4$ is a constant to account for the presence of
helium, and again velocity is in \kms.

\subsubsection{CO J = 2-1 Isotopologue Comparison}
\label{sec:co21}

\citet{Thomas2008} observed C$^{17}$O in the J=2-1 and 3-2 transitions each with a
single pointing using the JCMT centered at J(2000) = 05:39:10.8 +35:45:16 and measured a
column density $N_{\hh}=4.03\times10^{22}$ cm$^{-2}$.  The peak column
density is $1.7\ee{22}$\persc\ in \thirteenco\ and $2.2\ee{22}$ in \ceighteeno\ at J(2000) = 5:39:10.2
+35:45:26, which is reasonably consistent with the C$^{17}$O measurement
considering abundance uncertainties.  The peaks of the integrated
spectra for \ceighteeno\ and \thirteenco\ are coincident, but the \twelveco\
integrated peak is at J(2000) = 5:39:12.6 +35:45:46 (Figure \ref{fig:scuba_co21},
discussed more in section \ref{sec:discussion-outflows}).

Measurements of the column density, mass, momentum, and energy are performed as
in Equation \ref{eqn:column}.  Assuming a \twelveco/\thirteenco\ ratio of 60
\citep{lucas1998} and optically thin \thirteenco, the mean column density
across the region is $N_{\hh} = 4.5\times10^{21}$ \persc.  The resulting total
mass of the central $\sim3$\arcmin\ is about 320 \msun, which is substantially
smaller than the 600 \msun\ measured by \citet{beuther2002} and
\citet{Zinchenko1997}, but it is nearly consistent with 870\um\ and NH$_3$
estimates of 450 and 400 \msun\ from \citet{Mao2004} and is within the
systematic uncertainties of these measurements.  Assuming \ceighteeno\ is
optically thin and the \ceighteeno/\thirteenco\ ratio is 10, the 
column density is 5.2\ee{21} \persc\ and the mass is 360 \msun, which is
consistent with the \thirteenco\ measurements, indicating that optical depth
effects are probably not responsible for the discrepancy with the dust mass
estimate.

% We also measure a total mass in a 1 \arcmin\ radius of \necluster\ of 210
% \msun\ (\thirteenco) and 270 \msun (\ceighteeno), which is consistent with
% the SCUBA measurements of \citet{williams2004}.

% Presumably the 1.2mm and NH$_3$ observations were
% sensitive only to the dense gas, while the 13CO traces parts of the cloud that
% are still supported by turbulent pressure and are therefore at lower densities.

\subsubsection{CO Mass and Energy Measurements for Specific Outflows}

% \citet{beuther2002} measure a total mass of 610 \msun\ and an outflow mass of 
% 20 \msun .  Our CO measurements of the same region give a mass estimate of 4 (CO 3-2) or 2 (CO 2-1)
% \msun, in poor agreement with their results.  See section \ref{sec:discussion-outflows}.

Table \ref{tab:comeas} lists measurements of mass and momentum in
apertures shown in figure \ref{fig:cofig}.  Where red and blue masses are
listed, there is an outflow in the red and blue along the line of sight.  Where
only one is listed, an excess to one side of the cloud rest velocity was
detected and assumed to be accelerated gas from a protostellar outflow.  Blue
velocities are integrated from -33 to -21 \kms.  Red velocities are integrated
from -12 \kms\ to 1 \kms.  All masses are computed assuming CO is optically
thin in the outflow, which leads to a lower bound on the mass; \thirteenco\ 2-1 was
measured to have an optical depth of 0.1 in 7 very high velocity outflows in
\citet{choi1993}, so if a relative abundance \twelveco/\thirteenco =
60 is assumed\citep{lucas1998}, masses increase by a factor of 6.

It is not possible to completely distinguish outflowing matter from the ambient
medium.  While the outflowing matter is generally at higher velocities, the
outflow and ambient line profiles are blended.  A uniform selection of high
velocities was applied across the region but this may include some matter from
the cloud, biasing the mass measurements upward.  Outflows in the plane of the
sky and low-velocity components of outflows will be blended with the cloud
profile, which would lead to underestimates of the outflowing mass.  The
momentum measurements, however, should be more robust because they are weighted
by velocity, and higher velocity material is more certainly outflowing.  The
momentum measurements are referenced to the central velocity of \necluster,
$v_{LSR}=-16.0$ \kms.



\section{Discussion}

\subsection{Outflow Mass and Momentum}
\label{sec:discussion-outflows}
\citet{beuther2002} reported a total outflowing mass of 20 \msun\ in \necluster.  We
measure a significantly lower outflow mass of 2 \msun\ under the assumption
that the gas is optically thin, but this assumption is not valid: a lower limit
can be set from the weak \thirteenco\ 2-1 outflow detection  (lower limit
because not all of the outflowing material is detected) on the outflowing mass
of $\sim 4$ \msun .  \citet{choi1993} measure an optical depth of  \thirteenco\
2-1 $\tau \approx 0.1$ in 7 very high velocity outflows.  
%If \region\
%is assumed to be typical of their sample and the solar abundance of
%\thirteenco\ (\twelveco/\thirteenco=89) is used, our outflow mass is about
%18\msun (the solar abundance is used in order to compare directly to \citet{beuther2002},
%who follow the method of \citet{Cabrit1990}, and to set an upper limit on the mass).  
Our \thirteenco\ data suggests that the optical depth is somewhat lower,
 $\tau\approx0.07$. The abundance \twelveco/\thirteenco = 60 is used  \citep{lucas1998}  
to derive a total outflowing mass estimate $M\approx25$ \msun .  The total outflowing mass is
therefore $\sim 4\%$ of the total cloud mass, though most
of the outflowing material is coming from \necluster, so as much as 13\% of the
material in \necluster\ may be outflowing.

The most prominent outflow in \region, Outflow 1, is primarily along
the plane of the sky, so the high velocity CO is likely associated
with the other outflows that have significant components along the line of
sight.  As pointed out in \citet{beuther2002}, the integrated and peak CO are 
aligned with the main mm core.  High-velocity \hh\ near the mm cores and the
blueshifted outflows 2 and 4 all suggest that there are many distinct outflows
that together are responsible for the high velocity CO gas.

The offset between the integrated \thirteenco\ peak and \twelveco\ peak in the 
J=2-1 integrated maps, which corresponds with an offset in the peak of the
integrated CO 3-2 map and the peak temperature observed in CO 3-2, suggests
that the gas mass is largely associated with \swcluster, but the outflowing gas
is primarily associated with \necluster.  The integrated and maximum brightness
temperatures in \thirteenco\ and \ceighteeno\ are also centered near \swcluster, 
which rules out optical depth as the cause of this offset.  CO may be depleted
in the dense mm cores, which would help account for the lower mass estimate
from CO isotopologues relative to dust mass and NH$_3$.  Alternately, the gas
temperature in \swcluster\ may be significantly higher than in \necluster\ 
except in the outflows, which are probably warm.  In this case, the outflowing
\twelveco\ enhances the integrated intensity because of its high temperature
and reduced effective optical depth, but it does not set the peak brightness
because of the low filling-factor of the high-temperature gas.

Because the outflows are seen in \hh, which requires shock velocities $\sim30$
\kms\ to be excited \citep{bally2007}, and because the association between the
high-velocity CO and the plane-of-the-sky \hh\ is unclear,  a velocity of 30
\kms\ is used when estimating the dynamical age.
Assuming the outflow is about 0.5 pc long in one direction (e.g. Outflow 1), 
the dynamical age is 1.6\ee{4} years. Outflow 4, which is around 1 pc
long, is also seen at a velocity of -70 \kms\ LSR, or about -50 \kms\ with
respect to the cloud, and therefore has a dynamic age 2\ee{4} years, which is
consistent.

\subsection{Energy Injection / Ejection}
Using an assumed outflow lifetime of $5\times10^3$ years for $v=100\ \kms$ as a
lower limit (because the full extent of the flows is not necessarily observed)
and $1\times10^5$ as an upper limit (for the CO velocities $\sim10\ \kms$ and
the longest $\sim1$ pc flows), mechanical luminosities of the outflows $L=E/t$
are derived.  The summed mechanical luminosity of the outflows is compared to the
turbulent decay luminosity within a 12\arcsec, 1\arcmin, and 5\arcmin\ radius
centered on \necluster\ in Table \ref{tab:turb}.  \setlength\tabcolsep{3pt}  

\Table{ccccccc}{Comparison of turbulent decay and outflow injection}
{Radius (pc) & $t_{turb}$\tablenotemark{a} (yr) & $L_{turb} (\lsun)$ & $L_{outflows}$\tablenotemark{b} $(\lsun) $
& Binding Energy (ergs) \tablenotemark{c} & Outflow Energy (ergs) & Turbulent Energy (ergs) \tablenotemark{d}}
{tab:turb}
{
0.10  & 2\ee{4} &  20  & 0.03-0.6    & 3.4 \ee{46} & 3.5 \ee{44} & 5.0\ee{46}\\
0.52  & 1\ee{5} &  12  & 0.6 - 9.4   & 5.9 \ee{46} & 5.9 \ee{45} & 1.5\ee{47}\\
2.62  & 5\ee{5} &  2.3 & 1-22        & 1.2 \ee{46} & 1.4 \ee{46} & 1.5\ee{47}\\
}
{
\tablenotetext{a}{Masses are assumed to be 600 \msun\ for the 1\arcmin\ and 5\arcmin\ 
apertures, and 200\msun\ for the 12\arcsec\ aperture.}
\tablenotetext{b}{Outflow luminosities are given as a range with a lower limit
$L=E_{out} / 10^5 \textrm{yr}$ and upper limit $L=E_{out}/ 5\times10^3 \textrm{yr}$, 
where $E_{out}$ is from Table \ref{tab:comeas} multiplied by 6 to correct
for outflow opacity.  }
\tablenotetext{c}{Binding energy is the order-of-magnitude estimate GM$^2$/R}
\tablenotetext{d}{Turbulent energy is computed using the measured 5 \kms\ line
width as the turbulent velocity.}
}


The rate of turbulent decay can be estimated from the crossing-time of the
region,  $L / v$, where $L$ is the length scale and $v$ is the the typical
turbulent velocity.  On the largest ($\sim$ few pc) scales, the mechanical
luminosity from high-velocity outflowing material is approximately capable of
balancing turbulent decay and upholding the cloud against collapse.  However,
at the size scales of the \necluster\ clump ($\sim 0.1$ pc), turbulent decay
occurs on more than an order of magnitude faster timescales than outflow energy
injection.  On the smallest scales, outflow energy can be lost from the cluster
through collimated outflows, though wide-angle flows and wrapped up magnetic
fields will not propagate outside of the core region. Once collimated flows
impact the local interstellar medium in a bow shock, their energy and momentum
are distributed more isotropically and again contribute to turbulence.  The
imbalance on a small size scale is consistent with the observed infall
signature (Figure \ref{fig:co21_all3}) in the inner 12\arcsec\ around
\necluster\ and the lack of a similar profile elsewhere.

\subsection{Comparison to other clumps}
The classification scheme laid out in \citet{klein2005} is used to identify
\necluster\ as a Protocluster and \swcluster\ as a Young Cluster.  \citet{maury2009}
performed a similar analysis of the Early Protocluster NGC 2264-C.  They also
found that the outflow mechanical luminosity could provide the majority of the
turbulent energy $L_{turb}\sim1.2 \lsun$ within the protocluster in a radius of
0.7 pc with a mass 2300\msun .  \citet{williams2003} performed an outflow
study of the OMC 2/3 region with radius 1.2 pc and mass 1100 \msun, which is also an Early
Protocluster, and concluded that $L_{turb} \sim L_{flow} \sim 1.3 \lsun$.
While all three regions have nearly the same turbulent decay luminosities and
outflow mechanical luminosities, \necluster\ in \region\ is significantly more
compact and lower mass than the Early Clusters, and is the only one of the three
that contains signatures of massive star formation.  

% \Table{ccccc}{Table of other protoclusters}
% {name & mass & radius & most massive star & luminosity}
% {tab:protoclusters}
% {
% IRAS 05358+3543 & 600  & 0.5 & 12 & 6300 \\
% NGC 2264-C      & 2300 & 0.7 & <8 & 2300 \\ }
% {notemarks}

% In the collapsing region, the
% luminosity from gravitational infall $L =  E_{grav} v/r \sim 8 \lsun$ is 2-3
% orders of magnitude larger than the outflow luminosity in the same area.


% The CO mass measured in the blue wing of the Outflow 4 bow shock probably represents
% a lower limit on the ejected mass from this object.  Assuming a constant mass loss
% rate and that the CO we measure was ejected in 1/5 the dynamical time, we derive
% a mass loss rate of 6\ee{-6} \msun/yr.  

% Protostellar feedback is necessary to halt star formation, and it determines the
% star formation efficiency.  Outflows inject momentum into their parent molecular
% clouds, creating turbulence and holding the cloud up against collapse.
% Although the majority of the outflows seen in \region\ have penetrated the
% cloud and dragged only a small fraction of the total cloud mass with them, the
% central outflow from the forming massive stars is carrying $\sim.3-6\%$ of the
% total mass of the \region\ cloud away at low velocities.  

% In \region, it seems likely that feedback from the massive stars forming in
% mm1a will  halt collapse before outflows destroy the region.  Once enough
% material has accreted onto the UCHII region's star, the HII region will expand
% until it is no longer gravitationally bound and an ionization front will
% proceed outward, pushing gas ahead of it at a few \kms.  Isolated star
% formation in portions of the molecular cloud not associated with the \region\
% clump will continue until the ionization front from the mm1a massive star
% pushes away the gas.  

\subsection{Surrounding Regions}
\label{sec:surroundings}
%We briefly discuss other results apparent in the \ha\ and CO 3-2 images.
%The surrounding regions will be discussed in more detail in Yan et al (2009).

About 8\arcmin\ to the southeast of \region\ is another embedded star forming
region, G173.58+2.45.  Interferometric and stellar population studies have been
performed by \citet{shepherd1996} and \citet{shepherd2002}.  The bipolar
outflow detected in their interferometric maps is also cleanly resolved in our
figure \ref{fig:cofig}.  In our wide-field \hh\ maps, there is a complex of
outflows similar to that of \region, but fainter.

The large HII region Sharpless 231 to the northeast can be seen in the \ha\ 
image (figure \ref{fig:overview_ha}).  The expanding HII region is pushing against
the molecular ridge that includes \region\ and accelerating the CO gas in the
blue direction (e.g. the northern blueshifted clumps in figures \ref{fig:HA_with_CO}
and \ref{fig:cofig}).  It can be seen from the IRAC 8\um\ data 
%(contours in figure \ref{fig:HA_with_8um}) 
that UV radiation from the HII region reaches to
the \region\ clusters.  The expanding HII region's pressure on the molecular
ridge may be responsible for triggering the collapse of \region\ and G173.  The
size gradient from S232 ($\sim 30\arcmin$\ across) to S231 ($\sim 10\arcmin$)
to S233 ($\sim 2-3\arcmin$) is suggestive of an age gradient assuming uniform
HII region expansion velocities and a common distance.  Investigation of this
hypothesis will require detailed stellar population studies in the HII regions
with proper regard for eliminating foreground and background sources.

\subsection{Massive Star Binary}
Our identification of a probable massive star binary with an associated outflow
contributes to a very small sample of known maser disks with \hh\ emission
perpendicular to the disk.  \citet{debuizer2003} observed 28 methanol maser
sources with linear distributions of maser spots in the \hh\ 2.12 \um\ line,
and he identified only 2 sources with \hh\ emission perpendicular to the maser
lines.  None of the outflows identified in his survey were as collimated as
Outflow 2, so the methanol disk / outflow combination presented here may be the
most convincing association of a massive protostellar disk with a collimated
outflow.

The association of a massive star with an UCHII region and a methanol maser
disk and the very small size of the UCHII region both suggest that the massive
stellar system is very young.  \citet{walsh1998} suggested that the development
of a UCHII region leads to the destruction of maser emission regions.  Their
conclusion is consistent with our interpretation of mm1a as a binary system. 

\section{Summary \& Conclusion}

We have presented a multiwavelength study of the \region\ star forming region.
\region\ contains an embedded cluster of massive stars and is surrounded by
outflows.  The outflows were linked to probable sources and determined that at
least one outflow is probably associated with a massive ($\sim 10 \msun$) star.
Added kinematic information and a wide field view of the infrared outflows has
been used to develop a more complete picture of the region.

\begin{itemize}
  \item \necluster\ is a Protocluster and \swcluster\ is a Young Cluster
  \item Energy injection on the scales of \region\ can maintain turbulence, but on
    the small scales of the \necluster\ protocluster, is inadequate by $\sim 2$ orders
    of magnitude.  \necluster\ is collapsing.
  \item there are 11 candidate outflows, 7 of which have candidate counterflows, in the 
    \region\ complex
  \item there is a probable massive binary with one member of mass 12 \msun\ in
    mm1a, and the other which is the source of Outflow 2
  \item there are at least two moderate-mass ($\sim$5\msun) young stars in \region\ 
%\item the extinction towards the \hh\ bow shocks is variable, suggesting they are pushing
%  past the molecular cloud
\end{itemize}

We have identified additional middle- and high-mass young stars with outflows, and 
presented a case for a high-mass binary system within the millimeter core mm1a.

% \section{Acknowledgements}
% We thank the anonymous referee for very helpful suggestions particularly
% regarding the mm1a SED analysis.  We thank Vincent Minier for providing us with
% the positions of the VLBI maser spots and Steve Myers and George Moellenbrock
% for their assistance with VLA data reduction.
% 
% We also thank Cara Battersby, Devin Silvia, Mike Shull, and Jeremy Darling for
% helpful comments on early drafts. 
% 
% This work made use of SAOIMAGE DS9 (\url{http://hea-www.harvard.edu/RD/ds9/}),
% IRAF (\url{http://iraf.net/}, scipy (\url{http://www.scipy.org}), and APLpy
% (\url{http://aplpy.sourceforge.net/}).
% 
% J. P. W. thanks the NSF for support through NSF-AST08-08144.


%\bibliographystyle{apj_w_etal}
%\bibliography{iras05358}


\clearpage
\begin{landscape}
\begin{deluxetable}{ccccccccccccc}\setlength\tabcolsep{3pt}
  \scriptsize
  \tabletypesize{\tiny}
  \tablewidth{0pt}
    \centering
    \tablecaption{Measured \hh\ line strengths
      \label{tab:nirmeas}}
    \tablehead{
        &1-0 S(0)&1-0 S(1)&1-0 S(2)&1-0 S(3)&1-0 S(6)&1-0 S(7)&1-0 S(8)&1-0 S(9)&1-0 Q(1)&1-0 Q(2)&1-0 Q(3)&1-0 Q(4)  \\
aperture&2.2233&2.12183&2.03376&1.95756&1.78795&1.74803&1.71466&1.68772&2.40659&2.41344&2.42373&2.43749}
\startdata
outflow1ap1  &   3.60E-15&  9.80E-15&  5.50E-15&  1.20E-14&  4.70E-15&  3.10E-15&  8.60E-16&  1.10E-15&  9.20E-15&  6.10E-15&  1.10E-14&  6.90E-15 \\
             & ( 2.4e-17)&( 3.4e-17)&( 6.8e-17)&(   2e-16)&(   2e-16)&( 2.8e-17)&( 2.8e-17)&( 2.7e-17)&( 1.4e-16)&( 7.4e-17)&( 8.8e-17)&( 7.8e-17) \\
outflow1ap2  &   7.10E-16&  1.80E-15&  9.90E-16&  1.80E-15&         -&         -&         -&         -&  3.00E-15&  1.90E-15&  3.20E-15&  2.00E-15 \\
             & ( 2.1e-17)&( 2.7e-17)&( 6.8e-17)&( 1.7e-16)&         -&         -&         -&         -&( 1.3e-16)&(   4e-17)&( 7.8e-17)&( 3.8e-17) \\
outflow1ap3  &   1.60E-15&  4.10E-15&  2.20E-15&  4.70E-15&         -&  8.30E-16&         -&         -&  5.60E-15&  3.70E-15&  6.60E-15&  4.80E-15 \\
             & ( 2.4e-17)&( 3.4e-17)&( 6.3e-17)&( 1.8e-16)&         -&( 2.8e-17)&         -&         -&( 1.4e-16)&( 5.9e-17)&( 8.2e-17)&( 5.2e-17) \\
outflow1ap4  &          -&  9.00E-16&         -&         -&         -&         -&         -&         -&         -&         -&         -&         - \\
             &          -&(   3e-17)&         -&         -&         -&         -&         -&         -&         -&         -&         -&         - \\
outflow2ap1  &          -&  3.60E-16&         -&         -&         -&         -&         -&         -&         -&         -&         -&         - \\
             &          -&( 1.5e-17)&         -&         -&         -&         -&         -&         -&         -&         -&         -&         - \\
outflow2ap2  &   9.40E-16&  2.40E-15&  1.50E-15&  1.80E-15&         -&  4.00E-16&         -&         -&  3.00E-15&         -&  3.70E-15&         - \\
             & ( 1.7e-17)&( 2.2e-17)&( 5.8e-17)&( 1.1e-16)&         -&( 2.3e-17)&         -&         -&( 4.7e-17)&         -&( 7.9e-17)&         - \\
outflow2ap3  &   2.10E-15&  1.90E-15&  1.80E-15&  2.20E-15&         -&  6.70E-16&         -&         -&  5.70E-15&         -&  7.30E-15&         - \\
             & ( 1.7e-17)&( 2.2e-17)&( 5.1e-17)&( 1.3e-16)&         -&( 2.9e-17)&         -&         -& -8.00E-16&         -& -8.00E-16&         - \\
outflow4ap1  &   5.50E-16&  2.00E-15&  8.50E-16&  2.00E-15&         -&  9.40E-16&  1.90E-16&  3.40E-16&  1.40E-15&         -&  1.40E-15&         - \\
             & (   2e-17)&(   2e-17)&(   5e-17)&( 1.3e-16)&         -&( 2.8e-17)&( 1.8e-17)&( 2.3e-17)& -4.00E-16&         -&( 6.9e-17)&         - \\
outflow4ap2  &   5.60E-16&  2.00E-15&  5.30E-16&  2.10E-15&         -&  5.80E-16&         -&  1.10E-16&         -&         -&  2.00E-15&         - \\
             & (   2e-17)&( 2.2e-17)&( 2.4e-17)&( 1.2e-16)&         -&( 2.3e-17)&         -&( 1.8e-17)&         -&         -& -2.00E-16&         - \\
     IR6ap1  &          -&  1.10E-15&         -&  9.30E-16&         -&  4.30E-16&         -&         -&         -&         -&         -&         - \\
             &          -&(   3e-17)&         -&( 1.4e-16)&         -&( 3.2e-17)&         -&         -&         -&         -&         -&         - \\
    IR93ap1  &          -&  6.60E-15&         -&  2.70E-15&         -&         -&         -&         -&         -&         -&  5.80E-15&         - \\
             &          -&( 3.5e-17)&         -&(   1e-16)&         -&         -&         -&         -&         -&         -&( 7.4e-17)&         - \\
    IR93ap2  &   4.40E-15&  6.60E-15&  3.90E-15&  3.30E-15&         -&  1.10E-15&         -&         -&  7.60E-15&  5.10E-15&  6.90E-15&  5.50E-15 \\
             & ( 3.2e-17)&( 3.7e-17)&( 9.2e-17)&( 1.4e-16)&         -&( 2.7e-17)&         -&         -&(   8e-17)&( 5.2e-17)&( 7.4e-17)&( 6.1e-17) \\
    IR93ap3  &   1.00E-15&  1.70E-15&         -&  9.00E-16&         -&         -&         -&         -&  2.00E-15&  1.70E-15&  1.90E-15&         - \\
             & ( 2.3e-17)&( 3.6e-17)&         -&( 1.2e-16)&         -&         -&         -&         -&(   8e-17)&( 3.8e-17)&( 8.8e-17)&         - \\
    IR93ap4  &   2.60E-15&  3.70E-15&         -&         -&         -&         -&         -&         -&  4.30E-15&  3.50E-15&  4.70E-15&         - \\
             & ( 3.2e-17)&( 3.6e-17)&         -&         -&         -&         -&         -&         -&( 8.5e-17)&( 5.2e-17)&( 7.4e-17)&         - \\
    IR93ap5  &          -&  1.90E-15&         -&  1.00E-15&         -&         -&         -&         -&         -&         -&         -&         - \\
             &          -&(2.4e-17) &         -&(1.00e-16)&         -&         -&         -&         -&         -&         -&         -&         - \\
    IR93ap6  &          -&  4.10E-16&         -&         -&         -&         -&         -&         -&         -&         -&         -&         - \\
             &          -&(   3e-17)&         -&         -&         -&         -&         -&         -&         -&         -&         -&         - \\  
\enddata
   \tablecomments{Fluxes are in units erg s$^{-1} \persc $\AA$^{-1}$.  Errors are listed on
   the second row for each aperture.  Errors of (0) indicate that the line was detected, but that
   the fluxes should not be trusted because the background was probably oversubtracted.}
\end{deluxetable}\addtocounter{table}{-1}
\clearpage
\end{landscape}

\begin{deluxetable}{ccccccccccccccc}
  \scriptsize
  \tabletypesize{\tiny}
  %\rotate
  \tablewidth{0pt}
    \centering
    \tablecaption{Measured \hh\ line strengths (cont'd)
      \label{tab:nirmeas2}}
    \tablehead{
&2-1 S(1)&2-1 S(3)&3-2 S(3)&3-2 S(4)&4-3 S(5) & [Fe II] & [Fe II] \\
&2.24772&2.07351&2.2014&2.12797&2.20095       & 1.6435  & 1.2567  \\ }
\startdata
outflow1ap1  &   2.00E-15&  1.20E-15&  6.20E-16&  2.60E-16&  7.10E-16 & 4.4e-15    &3.5e-15    \\
             & ( 2.5e-17)&( 3.5e-17)&( 2.2e-17)&( 1.6e-17)&( 1.9e-17) & ( 7.9e-17) &(   4e-17) \\
outflow1ap2  &          -&         -&         -&         -&         - & 6.7e-16    &3.1e-16    \\
             &          -&         -&         -&         -&         - & ( 7.8e-17) &( 3.3e-17) \\
outflow1ap3  &   9.90E-16&         -&  5.70E-16&  2.50E-16&  6.40E-16 & 1.3e-15    &5.7e-16    \\
             & ( 2.6e-17)&         -&(       0)&( 1.2e-17)&(       0) & ( 8.9e-17) &(   4e-17) \\
outflow1ap4  &          -&         -&         -&         -&         - & -          & -         \\
             &          -&         -&         -&         -&         - & -          & -         \\
outflow2ap1  &          -&         -&         -&         -&         - & -          & -         \\
             &          -&         -&         -&         -&         - & -          & -         \\
outflow2ap2  &   6.40E-16&         -&         -&         -&         - & -          & -         \\
             & ( 1.9e-17)&         -&         -&         -&         - & -          & -         \\
outflow2ap3  &          -&         -&         -&         -&         - & -          & -         \\
             &          -&         -&         -&         -&         - & -          & -         \\
outflow4ap1  &   4.30E-16&  4.20E-16&         -&         -&  1.60E-16 & -          & -         \\
             & ( 1.9e-17)& (0)      &         -&         -& (0)       & -          & -         \\
outflow4ap2  &          -&         -&         -&         -&         - & -          & -         \\
             &          -&         -&         -&         -&         - & -          & -         \\
     IR6ap1  &          -&         -&         -&         -&         - & -          & -         \\
             &          -&         -&         -&         -&         - & -          & -         \\
    IR93ap1  &          -&         -&         -&         -&         - & -          & -         \\
             &          -&         -&         -&         -&         - & -          & -         \\
    IR93ap2  &   3.80E-15&         -&  3.10E-15&         -&         - & -          & -         \\
             & ( 2.2e-17)&         -& (0)      &         -&         - & -          & -         \\
    IR93ap3  &          -&         -&         -&         -&         - & -          & -         \\
             &          -&         -&         -&         -&         - & -          & -         \\
    IR93ap4  &          -&         -&         -&         -&         - & -          & -         \\
             &          -&         -&         -&         -&         - & -          & -         \\
    IR93ap5  &          -&         -&         -&         -&         - & -          & -         \\
             &          -&         -&         -&         -&         - & -          & -         \\
    IR93ap6  &          -&         -&         -&         -&         - & -          & -         \\
             &          -&         -&         -&         -&         - & -          & -         \\             
   \enddata
   \tablecomments{Fluxes are in units erg s$^{-1} \persc $\AA$^{-1}$.  Errors are listed on
   the second row for each aperture.  Errors of (0) indicate that the line was detected, but that
   the fluxes should not be trusted because the background was probably oversubtracted.}
\end{deluxetable}
\clearpage

\ifstandalone
\bibliographystyle{apj_w_etal}  % or "siam", or "alpha", or "abbrv"
%\bibliography{thesis}      % bib database file refs.bib
\bibliography{bibdesk}      % bib database file refs.bib
\fi

\end{document}

% %\documentclass[defaultstyle,11pt]{thesis}
%\documentclass[]{report}
%\documentclass[]{article}
%\usepackage{aastex_hack}
%\usepackage{deluxetable}
\documentclass[preprint]{aastex}


%%%%%%%%%%%%%%%%%%%%%%%%%%%%%%%%%%%%%%%%%%%%%%%%%%%%%%%%%%%%%%%%
%%%%%%%%%%%  see documentation for information about  %%%%%%%%%%
%%%%%%%%%%%  the options (11pt, defaultstyle, etc.)   %%%%%%%%%%
%%%%%%%  http://www.colorado.edu/its/docs/latex/thesis/  %%%%%%%
%%%%%%%%%%%%%%%%%%%%%%%%%%%%%%%%%%%%%%%%%%%%%%%%%%%%%%%%%%%%%%%%
%		\documentclass[typewriterstyle]{thesis}
% 		\documentclass[modernstyle]{thesis}
% 		\documentclass[modernstyle,11pt]{thesis}
%	 	\documentclass[modernstyle,12pt]{thesis}

%%%%%%%%%%%%%%%%%%%%%%%%%%%%%%%%%%%%%%%%%%%%%%%%%%%%%%%%%%%%%%%%
%%%%%%%%%%%    load any packages which are needed    %%%%%%%%%%%
%%%%%%%%%%%%%%%%%%%%%%%%%%%%%%%%%%%%%%%%%%%%%%%%%%%%%%%%%%%%%%%%
\usepackage{latexsym}		% to get LASY symbols
\usepackage{graphicx}		% to insert PostScript figures
%\usepackage{deluxetable}
\usepackage{rotating}		% for sideways tables/figures
\usepackage{natbib}  % Requires natbib.sty, available from http://ads.harvard.edu/pubs/bibtex/astronat/
\usepackage{savesym}
\usepackage{amssymb}
%\savesymbol{singlespace}
\savesymbol{doublespace}
%\usepackage{wrapfig}
%\usepackage{setspace}
\usepackage{xspace}
\usepackage{color}
\usepackage{multicol}
\usepackage{mdframed}
\usepackage{url}
\usepackage{subfigure}
%\usepackage{emulateapj}
\usepackage{lscape}
\usepackage{grffile}
\usepackage{standalone}
\standalonetrue
\usepackage{import}
\usepackage[utf8]{inputenc}
\usepackage{longtable}
\usepackage{booktabs}



%%%%%%%%%%%%%%%%%%%%%%%%%%%%%%%%%%%%%%%%%%%%%%%%%%%%%%%%%%%%%%%%
%%%%%%%%%%%%       all the preamble material:       %%%%%%%%%%%%
%%%%%%%%%%%%%%%%%%%%%%%%%%%%%%%%%%%%%%%%%%%%%%%%%%%%%%%%%%%%%%%%

% \title{Star Formation in the Galaxy}
% 
% \author{Adam G.}{Ginsburg}
% 
% \otherdegrees{B.S., Rice University, 2007\\
% 	      M.S., University of Colorado, Boulder, 2009}
% 
% \degree{Doctor of Philosophy}		%  #1 {long descr.}
% 	{Ph.D., Rocket Science (ok, fine, astrophysics)}		%  #2 {short descr.}
% 
% \dept{Department of}			%  #1 {designation}
% 	{Astrophysical and Planetary Sciences}		%  #2 {name}
% 
% \advisor{Prof.}				%  #1 {title}
% 	{John Bally}			%  #2 {name}
% 
% \reader{Prof.~Jeremy Darling}		%  2nd person to sign thesis
% \readerThree{Prof.~Jason Glenn}		%  3rd person to sign thesis
% \readerFour{Prof.~Michael Shull}	%  4rd person to sign thesis
% \readerFour{Prof.~Neal Evans}	%  4rd person to sign thesis
% 
% \abstract{  \OnePageChapter	% one page only ??
% 
%     I discovered dust in space.  
% 
% 	}
% 
% 
% \dedication[Dedication]{	% NEVER use \OnePageChapter here.
% 	To 1, the second number in binary.
% 	}
% 
% \acknowledgements{	\OnePageChapter	% *MUST* BE ONLY ONE PAGE!
% 	All y'all.
% 	}
% 
% \ToCisShort	% a 1-page Table of Contents ??
% 
% \LoFisShort	% a 1-page List of Figures ??
% %	\emptyLoF	% no List of Figures at all ??
% 
% \LoTisShort	% a 1-page List of Tables ??
% %	\emptyLoT	% no List of Tables at all ??
% 
% 
% %%%%%%%%%%%%%%%%%%%%%%%%%%%%%%%%%%%%%%%%%%%%%%%%%%%%%%%%%%%%%%%%%
% %%%%%%%%%%%%%%%       BEGIN DOCUMENT...         %%%%%%%%%%%%%%%%%
% %%%%%%%%%%%%%%%%%%%%%%%%%%%%%%%%%%%%%%%%%%%%%%%%%%%%%%%%%%%%%%%%%
% 
% %%%%  footnote style; default=\arabic  (numbered 1,2,3...)
% %%%%  others:  \roman, \Roman, \alph, \Alph, \fnsymbol
% %	"\fnsymbol" uses asterisk, dagger, double-dagger, etc.
% %	\renewcommand{\thefootnote}{\fnsymbol{footnote}}
% %	\setcounter{footnote}{0}

\input{macros}		% file containing author's macro definitions

\begin{document}
% \input{introduction}
% 
% %\input{ch_iras05358}
% \input{ch_w5}
% \input{ch_h2co}
% \input{ch_h2colarge}
% \input{ch_boundhii}
% 
% %\input ch2.tex			% file with Chapter 2 contents
% 
% %%%%%%%%%%%%%%%%%%%%%%%%%%%%%%%%%%%%%%%%%%%%%%%%%%%%%%%%%%%%%%%%%%%
% %%%%%%%%%%%%%%%%%%%%%%%  Bibliography %%%%%%%%%%%%%%%%%%%%%%%%%%%%%
% %%%%%%%%%%%%%%%%%%%%%%%%%%%%%%%%%%%%%%%%%%%%%%%%%%%%%%%%%%%%%%%%%%%
% 
% \bibliographystyle{plain}	% or "siam", or "alpha", or "abbrv"
% 				% see other styles (.bst files) in
% 				% $TEXHOME/texmf/bibtex/bst
% 
% \nocite{*}		% list all refs in database, cited or not.
% 
% \bibliography{thesis}		% bib database file refs.bib
% 
% %%%%%%%%%%%%%%%%%%%%%%%%%%%%%%%%%%%%%%%%%%%%%%%%%%%%%%%%%%%%%%%%%%%
% %%%%%%%%%%%%%%%%%%%%%%%%  Appendices %%%%%%%%%%%%%%%%%%%%%%%%%%%%%%
% %%%%%%%%%%%%%%%%%%%%%%%%%%%%%%%%%%%%%%%%%%%%%%%%%%%%%%%%%%%%%%%%%%%
% 
% \appendix	% don't forget this line if you have appendices!
% 
% %\input appA.tex			% file with Appendix A contents
% %\input appB.tex			% file with Appendix B contents
% 
% %%%%%%%%%%%%%%%%%%%%%%%%%%%%%%%%%%%%%%%%%%%%%%%%%%%%%%%%%%%%%%%%%%%
% %%%%%%%%%%%%%%%%%%%%%%%%   THE END   %%%%%%%%%%%%%%%%%%%%%%%%%%%%%%
% %%%%%%%%%%%%%%%%%%%%%%%%%%%%%%%%%%%%%%%%%%%%%%%%%%%%%%%%%%%%%%%%%%%
% 
% \end{document}
% 
% 

\chapter{Using outflows to track star formation in the W5 HII region complex}
\label{ch:w5}
\section{Preface}
Only a few months after arriving at CU, I was given the opportunity to visit
the peak of Mauna Kea to perform observations with the JCMT.  I spend about 3
weeks at the telescope over the course of two years primarily mapping the W5
complex.  A side-project done during these observations resulted in my Comps II
project on IRAS 05358+3543.  These data were taken using Jonathan Williams'
Hawaii time allocation with the HARP receiver.  The data were taken with
essentially no plan for how they would be used.  The paper may have diminished
our group's overall interest in the W5 region: it turns out that star formation
is probably at its end here, being quenched by massive-star feedback.  However,
there is a largely ignored cloud to the northwest of the well-studied W5 bubbles
that has significant potential to form new stars.  

The W5 study was originally intended to include a Bolocam census of cores, but
the data in this region turned out to be the most problematic and contained
little signal.  We acquired additional data in 2009, but never got around to
performing a joint analysis of the CO and continuum data.  In part, at least,
this is because W5 is so faint at millimeter wavelengths compared to many
Galactic Plane sources.

This work is essentially a very detailed study of a star-forming region with
minimal implications for star forming theories at the moment.

\section{Introduction}


Galactic-scale shocks such as spiral density waves promote the formation of
giant molecular clouds (GMCs) where massive stars, star clusters, and OB
associations form.  The massive stars in such groups can either disrupt the
surrounding medium or promote further star formation.  While ionizing and soft
UV radiation, stellar winds, and eventually supernova explosions destroy clouds
in the immediate vicinity of massive stars, as the resulting bubbles age and
decelerate, they can also trigger further star formation.  In the ``collect and
collapse'' scenario
\citep[e.g.][]{elmegreen:sequential:1977}, gas swept-up by expanding bubbles
can collapse into new star-forming clouds.  In the ``radiation-driven
implosion'' model \citep{bertoldi:cometary:1990,klein:implosion:1983},
pre-existing clouds may be compressed by photo-ablation pressure or by the
increased pressure as they are overrun by an expanding shell.  In some
circumstances, forming stars are simply exposed as low-density gas is removed
by winds and radiation from massive stars.  These processes may play significant roles in
determining the efficiency of star formation in clustered environments
\citep{elmegreen1998}.


% Shortly after igniting, young massive stars may either provide the additional
% pressure needed to collapse neighboring gas clumps into stars, or they may
% simply disperse and destroy their natal environments.  The former phenomenon,
% known as triggered star formation, may play a significant role in determining
% the efficiency of star formation in clustered environments
% \citet{elmegreen1998}.  In addition, triggering from external processes
% asynchronous with star formation by direct collapse (e.g.  supernovae, spiral
% density waves) are thought to be partly responsible for the evenutal collapse
% of parent molecular clouds, but it is not yet known on what scales triggering
% is important, nor how strongly each mechanism contributes.  

Feedback from low mass stars may also control the shape of the stellar initial
mass function in clusters \citep{adams1996,Peters2010}.  Low mass young stars
generate high velocity, collimated outflows that contribute to the turbulent
support of a gas clump, preventing the clump from forming stars long enough
that it is eventually blown away by massive star feedback.  It is therefore
important to understand the strength of low-mass protostellar feedback relative
to other feedback mechanisms.

Outflows are a ubiquitous indicator of the presence of ongoing star formation
\citep{reipurth2001}.  CO outflows are an indicator of ongoing embedded star
formation at a younger stage than optical outflows because shielding from the
interstellar radiation field is required for CO to survive.  Although Herbig-
Haro shocks and \hh\ knots reveal the locations of the highest-velocity
segments of these outflows, CO has typically been thought of as a
``calorimeter'' measuring the majority of the mass and momentum ejected from
protostars or swept up by the ejecta \citep{Bachiller1996}.

%\subsection{W5}

The W5 star forming complex in the outer galaxy is a prime location to study
massive star formation and triggering.  The bright-rimmed clouds in W5 have
been recognized as good candidates for ongoing triggering by a number of groups
\citep{lefloch:cometary:1997,thompson:searching:2004,karr:triggered:2003}.  The
clustering properties were analyzed by \citet{koenig:clustered:2008} using
Spitzer infrared data, and a number of significant clusters were discovered.
The whole W5 complex may be a product of triggering, as it is located on one
side of the W4 chimney thought to be created by multiple supernovae during the
last $\sim$10 MYr \citep[][Figure \ref{fig:color_overview}]
{oey:hierarchical:2005}.  

Following \citet{koenig:clustered:2008}, we adopt a distance to W5 of 2 kpc
based on the water-maser parallax distance to the neighboring W3(OH) region
\citep{Hachisuka2006}.  As with W3, the W5 cloud is substantially
($\approx1.5\times$) closer than its kinematic distance would suggest
($v_{LSR}(-40~\kms)\approx3$ kpc).  Given this distance,
\citet{koenig:clustered:2008} derived a total gas mass of 6.5\ee{4} \msun\ from
a 2 \um\ extinction map.

The W5 complex was mapped in the $^{12}$CO 1-0 emission line by the Five
College Radio Astronomy Observatory (FCRAO) using the SEQUOIA receiver array
\citep{heyer:ogs:1998}.  The same array was used to map W5 in the \thirteenco\
1-0 line (C. Brunt, private communication).  Some early work searched for
outflows in W5 \citep{bretherton:unbiased:2002}, but the low-resolution CO 1-0
data only revealed a few, and only one was published.  The higher resolution
and sensitivity observations presented here reveal many additional outflows.

\begin{figure*}
  % Generated by GIMP
  \includegraphics[angle=90,width=5in]{figures_chw5/w345mos_brightened_labeled}
  \caption{An overview of the W3/4/5 complex (also known as the ``Heart and
  Soul Nebula'') in false color. Orange shows 8 \um\ emission from the Spitzer
  and MSX satellites.  Purple shows 21 cm continuum emission from the DRAO CGPS
  \citep{Taylor2003:CGPS}; the DSS R image was used to set the display opacity
  of the 21 cm continuum as displayed (purely for aesthetic purposes).  The
  green shows JCMT \twelveco\ 3-2 along with FCRAO \twelveco\ 1-0 to fill in 
  gaps that were not observed with the JCMT.  The image spans
  $\sim7\arcdeg$ in galactic longitude.  This overview image shows the
  hypothesized interaction between the W4 superbubble and the W3 and W5
  star-forming regions \citep{oey:hierarchical:2005}.}
  \label{fig:color_overview}
\end{figure*}

%\subsection{Neighboring Regions}
While W5 is thought to be associated with the W3/4/5 complex, there are other
infrared sources in the same part of the sky that are not obviously associated
with W5.  Some of these have been noted to be in the outer arm (several kpc
behind W5) by \citet{Digel1996} and \citet{Snell2002}.

\par
\par In section 2, we present the new and archival data used in our study.  In
section 3, we discuss the outflow detection process and compare outflow
detectability in W5 to that in Perseus.  In section 4, we discuss the physical
properties of the outflows and their implications for star formation in the W5
complex.  In section 5, we briefly describe the outer-arm outflows discovered.

\section{OBSERVATIONS}
\subsection{JCMT HARP CO 3-2}
CO J=3-2 345.79599 GHz data were acquired at the 15 m James Clerk Maxwell
Telescope (JCMT) using the HARP array on a series of observing runs in 2008.
On 2-4 January, 2008, $\sim$ 800 square arcminutes were mapped.  During the
run, $\tau_{225}$, the zenith opacity at 225 GHz measured using the Caltech
Submillimeter Observator tipping radiometer, ranged from 0.1 to 0.4
($0.4<\tau_{345
GHz}<1.6$\footnote{\url{http://docs.jach.hawaii.edu/JCMT/SCD/SN/002.2/node5.html}}).
Additional areas were mapped on 4-7 August, 16-20 and 31 October, and 1 and
12-15 Nov 2008 in similar conditions.  A total of $\sim$ 3 square degrees (12000
arcmin$^2$) in the W5 complex were mapped (a velocity-integrated mosaic is
shown in Figure \ref{fig:outflows_on_co32}).

HARP is a 16 pixel SIS receiver array acting as a front-end to the ACSIS
digital auto-correlation spectrometer.  In January 2008, 14 of the 16 detectors
were functional.  In the 2nd half of 2008, 12 of 16 were functional,
necessitating longer scans to achieve similar S/N.

In January 2008, a single spectral window centered at 345.7959899 with bandwidth 1.0
GHz and channel width 488 kHz (0.42 \kms) was used.  In August 2008 and later, we used 250 MHz
bandwidth and 61 kHz (0.05 \kms) channel width.  At this frequency, the beam
FWHM is 14\arcsec\ (0.14 pc at a distance of 2 kpc)
\footnote{\url{http://docs.jach.hawaii.edu/JCMT/OVERVIEW/tel_overview/}}.

A raster mapping strategy was used.  In 2008, the array was shifted by 1/2 of
an array spacing (58.2\arcsec) between scans.  Data was sampled at a rate of
$0.6 s$ per integration.  Two perpendicular scans were used for each field
observed.  Most fields were 10$\times$10\arcmin\ and took $\sim45$ minutes.
When only 12 receptors were available, 1/4 array stepping (29.1\arcsec) was
used with a sample rate of $0.4 s$ per integration.

Data were reduced using the SMURF package within the STARLINK software distribution
\footnote{\url{http://starlink.jach.hawaii.edu/}}.  The SMURF command {\sc makecube} was used to
generate mosaics of contiguous sub-fields.  The data were gridded on to cubes
with 6\arcsec\ pixels and smoothed with a $\sigma=2$-pixel gaussian, resulting
in a map FWHM resolution of 18\arcsec (0.17 pc).  A linear fit was subtracted from each
spectrum over emission-free velocities (generally -60 to -50 and -20 to -10
\kms) to remove the baseline.  The final map RMS was $\sigma_{T_A^*}\sim
0.06-0.11 K$ in 0.42 \kms\ channels.

The sky reference position (off position) in January 2008 was J2000 2:31:04.069 +62:59:13.81.
In later epochs, off positions closer to the target fields were selected from blank sky regions
identified in January 2008 in order to increase observing efficiency.  A
main-beam efficiency $\eta_{mb}=0.60$ was used as per the JCMT website to
convert measurements to $T_{mb}$, though maps and spectra are presented in the
original $T_A^*$ units.

\subsection{FCRAO Outer Galaxy Survey}
The FCRAO Outer Galaxy Survey (OGS)
observed the W5 complex in \twelveco\  \citep{heyer:ogs:1998} and \thirteenco\
1-0 (C. Brunt, private communication).  The \thirteenco\ data cube achieved a
mean sensitivity of 0.35 K per 0.13 \kms\ channel, or 0.6 K \kms\ integrated.
The \thirteenco\ cube was integrated over all velocities and resampled to match
the BGPS map using the {\sc montage}\footnote{\url{http://montage.ipac.caltech.edu/}}
package.  The FWHM beam size was  $\theta_{B}=$50\farcs (0.48 pc).  The integrated
\twelveco\ data cube, with a sensitivity $\sigma= 1 K$ \kms, is displayed with 
region name identifications in Figure \ref{fig:regionboxes_on_CO}.

\subsection{Spitzer}
Spitzer IRAC and MIPS 24 \um\ images from \citet{koenig:clustered:2008} were
used for morphological comparison.  The reduction and extraction techniques are
detailed in their paper.

\begin{figure*}
  %\epsscale{1.0}
  %CODE: cutouts.py
  \includegraphics[angle=90,width=5in]{figures_chw5/outflows_ostars_on_CO32}
  \caption{A mosaic of the CO 3-2 data cube integrated from -20 to -60 \kms.
  The grayscale is linear from 0 to 150 K \kms.  The red and blue X's mark the
  locations of redshifted and blueshifted outflows.  Dark red and dark blue
  plus symbols mark outflows at outer arm velocities.  Green circles mark the
  location of all known B0 and earlier stars in the W5 region from SIMBAD.}
  \label{fig:outflows_on_co32}
\end{figure*}

\Figure{figures_chw5/regionboxes_on_CO}
{Individual region masks overlaid on the FCRAO \twelveco\ integrated image.
The named regions, S201, AFGL4029, LWCas, W5NW, W5W, W5SE, W5S, and W5SW, were all
selected based on the presence of outflows within the box.
The inactive regions were selected from regions with substantial CO emission
but without outflows.  The `empty' regions have essentially no CO emission within
them and are used to place limits on the molecular gas within the east and west
`bubbles'.  W5NWpc is compared directly to the Perseus molecular cloud in 
Section \ref{sec:percompare}
%Finally, W5All is used in Section \ref{sec:covsbgps} to compute
%global properties of the complex.  
}{fig:regionboxes_on_CO}{0.5}{0}


\section{Analysis}
\subsection{Outflow Detections}
Outflows were identified in the CO data cubes by manually searching through
position-velocity space for line wings using STARLINK's GAIA display software.  Outflow
candidates were identified by high velocity wings inconsistent with the local
cloud velocity distribution, which ranged from a width of 3 \kms\ to  7 \kms.
Once an outflow candidate was identified in the position-velocity diagrams, the
velocity range over which the wing showed emission in the position-velocity
diagram (down to $T_A^*=0$) was integrated over to create a map from which the
approximate outflow size and position was determined (e.g. Figures
\ref{fig:outflow1} and \ref{fig:pv2b}). 
%The integration limits for computing physical
%properties were also set from the position-velocity diagram limits, not the
%averaged spectra: in some cases, emission at higher velocities away from the
%outflow was included in the aperture that is not part of the outflow.

Unlike \citet{curtis2010} and \citet{hatchell2009}, we did not use an
`objective' outflow identification method because of the greater velocity complexity
and poorer spatial resolution of our observations.  The outflow selection
criteria in these papers requires the presence of a sub-mm clump in order to
identify a candidate driving source (and therefore a targeted region in which to search for
outflows), making a similar objective identification impossible for
our survey.
As discussed later in 
Section \ref{sec:subregions}, the regions associated with outflows have wide
lines and many are double-peaked.  Additionally, many smaller areas associated
with outflows have collections of gaussian-profiled clumps that are not connected to
the cloud in position-velocity diagrams but are not outflows.  In particular,
W5 is pockmarked by dozens of small cometary globules that are sometimes
spatially coincident with the clouds but slightly offset in velocity.

While \citet{arce2010} described the benefits of 3D visualization using
isosurface contours, we found that the varying signal-to-noise across
large-scale ($\sim500$ pixel$^2$) regions with significant extent in RA/Dec and
limited velocity dynamic range made this method diffult for W5.  There
were many low-intensity outflows that were detectable by careful searches
through position-velocity space that are not as apparent using isosurface
methods.  Out of the 55 outflows reported here, only 14 \footnote{
Outflows 15, 20, 24, the cluster of outflows 26-32, 47, 48, 52, and 53 could 
all have readily been detected by pointed single-dish measurements.} would be
considered obvious, high-intensity, high-velocity flows from their spectra
alone; the rest could not be unambiguously detected without a search through
position-velocity space.

In the majority of sources, the individual outflow lobes were
unresolved, although some showed hints of position-velocity gradients at low
significance and in many the red and blue flows are spatially separated.    Only
Outflow 1's lobes were clearly resolved (Figure \ref{fig:outflow1}).  Some of
most suggestive gradients occurred where the outflow merged with its host
molecular cloud in position-velocity space, making the gradient difficult to
distinguish (e.g., Outflow 12, Figure \ref{fig:pv12}).  Bipolar pairs were
selected when there were red and blue flows close to one another.  The
classification of a bipolar flow was either `yc' (yes - confident), `yu' (yes -
unconfident), or `n' (no) in Table \ref{tab:outflows}.  This identification is
discussed in the captions for each outflow figure in the online
supplement. The AFGL 4029 region has many red and blue lobes but confusion
prevented pairing.  

\FigureFour{figures_chw5/pvdiagrams/pvdiagram_rotate_of1}{figures_chw5/OutflowOverlay_4.5_1}{figures_chw5/OutflowSpectra1}{figures_chw5/OutflowOverlay_8um_1}
{Position-velocity diagrams (a), spectra (c), and contour overlays of Outflow 1 on Spitzer 4.5 \um\ (b)
and 8 \um\ (d) images.  
This outflow is clearly resolved and bipolar.
{\it (a)}: Position-velocity diagram of the blue flow displayed in arcsinh stretch
from $T_A^*=$0 to 3 K.  Locations of the red and blue flows are indicated by vertical dashed lines.
The location of the position-velocity cut is indicated by the orange dashed line in panels (b) and
(d), although the position-velocity cut is longer than those cut-out images.
{\it (b)} Spitzer 4.5 \um\ image displayed in logarithmic stretch from 30 to 500 MJy \persr.    
{\it (c)}: Spectrum of
the outflow integrated over the outflow aperture and the velocity range
specified with shading.  The velocity center (vertical dashed line) is
determined by fitting a gaussian to the \thirteenco\ spectrum in an aperture
including both outflow lobes.  In the few cases in which \thirteenco\ 1-0 was 
unavailable, a gaussian was fit to the \twelveco\ 3-2 spectrum.
{\it (d)}: Contours of the red and blue outflows superposed on the 
Spitzer 8 \um\ image displayed in logarithmic stretch.  The contours are
generated from a total intensity image integrated over the outflow
velocities indicated in panel (c).  The contours in both panels (b) and (d) are displayed at levels of
0.5,1,1.5,2,3,4,5,6 K \kms\ ($\sigma\approx0.25$ K \kms).
The contour levels and stretches specified in this caption apply to all of the
figures in the supplementary materials except where otherwise noted.
}
{fig:outflow1}


\FigureFour{figures_chw5/pvdiagrams/pvdiagram_rotate_of2}{figures_chw5/OutflowOverlay_4.5_2}{figures_chw5/OutflowSpectra2}{figures_chw5/OutflowOverlay_8um_2}
{Position-velocity diagram, spectra, and contour overlays of Outflow 2 (see
Figure \ref{fig:outflow1} for a complete description).  While the two lobes are
widely separated, there are no nearby lobes that could lead to confusion, so we
regard this pair as a reliable bipolar outflow identification.  
}
{fig:pv2b}

\FigureFour{figures_chw5/pvdiagrams/pvdiagram_rotate_of12}{figures_chw5/OutflowOverlay_4.5_12}{figures_chw5/OutflowSpectra12}{figures_chw5/OutflowOverlay_8um_12}
{Position-velocity diagram, spectra, and contour overlays of Outflow 12.  Much of the red outflow
is lost in the complex velocity profile of the molecular cloud(s), but it is high enough velocity
to still be distinguished. }
{fig:pv12}

In cases where only the red- or blue-shifted lobe was visible, the
surrounding pixels were searched for lower-significance and lower-velocity
counterparts.  For cases in which emission was detected, a
candidate counterflow was identified and incorporated into the catalog.
However, in 12 cases, the counterflow still evaded detection, either because of
confusion or because the counterflow is not present in CO.

The outflow positions are overlaid on the CO 3-2 image in Figure
\ref{fig:outflows_on_co32} to provide an overview of where star formation is
most active.  The figures in Section \ref{sec:sfactivity}  show outflow
locations overlaid on small-scale images.

Because our detection method involved searching for high-velocity outflows
by eye, there should be no false detections.  However, it is possible
that some of these outflows are generated by mechanisms other than
protostellar jets and winds since we have not identified their driving sources.

One possible alternative driving mechanism is a photoevaporation flow,
which could be accelerated up to the sound speed of the ionized medium,
$c_{II} \approx 10$~\kms.
Gas accelerating away from the cloud would not be detected as an outflow
because it would be rapidly ionized.
However, gas driven inward would be accelerated and remain molecular.  It
could exhibit red and / or blue flows depending on the line of
sight orientation.  While there are viable candidates for this form of outflow
impersonator, such flows can only have peak velocities $v\lesssim c_{II}/4
\approx 2.5$~\kms\ in the strong adiabatic shock limit, so that any gas
seen with higher velocity tails are unlikely to be radiation-driven.  

Another plausible outflow impostor is the high-velocity tail in a
turbulent distribution.  However, for a typical molecular cloud, the low
temperatures would require very high mach-number shocks ($\mathcal{M}\gtrsim10$
assuming $T_{cloud}\sim20 $ K and $v_{flow} \sim 3$ \kms) that in idealized
turbulence should be rare and short-lived.  It is not known how frequent such
high-velocity excursions will be in non-ideal turbulence with gravity (A.
Goodman, P. Padoan, private communication).  Finally, it is less likely for
turbulent intermittency to have nearly coincident red and blue lobes, so
intermittency can be morphologically excluded in most cases. 


\subsubsection{Comparison to Perseus CO 3-2 observations}  % may not need subsub...
\label{sec:percompare}
We used the HARP CO 3-2 cubes from \citet{hatchell2007} to evaluate our ability
to identify outflows.  We selected an outflow that was
well-resolved and unconfused, L1448, and evaluated it at both the native
sensitivity of the \citet{hatchell2007} observations and degraded in resolution
and sensitivity to match our own.  We focus on L1448 IRS2,
labeled Outflow 30 in \citet{hatchell2007}. Figure \ref{fig:l1448} shows a
comparison between the original quality and degraded data.  

Integrating over the outflow velocity range, we measure each lobe to be
about $1.6\arcmin\times0.8\arcmin$ ($0.14\times0.07$pc).  Assuming a distance
to Perseus of 250 pc \citep[e.g.][]{Enoch2006},  we smooth by a factor of 8 by convolving
the cube with a FWHM = 111\arcsec\ gaussian, then downsample by the same factor of 8
to achieve 6\arcsec\ square pixels at 2 kpc.  The resulting noise was reduced
because of the spatial and spectral smoothing and was measured to be $\approx 0.05$ K in
0.54 \kms\ channels, which is comparable to the sensitivity in our
survey.  It is still possible to distinguish the outflows
from the cloud in position-velocity space.  Each lobe is individually
unresolved (long axis $\sim12\arcsec$ compared to our beam FWHM of 18\arcsec),
but the two are separated by $\gtrsim 20\arcsec$ and therefore an overall
spatial separation can still be measured.  Because they are just barely
unresolved at this distance, the lobes' surface brightnesses are approximately
the same at 2 kpc as at 250 pc; if this outflow were seen at a greater distance
it would appear fainter.

\citet{hatchell2007} detected 4 outflows within this map, plus an additional
confused candidate.  We note an additional grouping of outflowing material in
the north-middle of the map(centered on coordinate 150$\times$150 in Figure
\ref{fig:l1448}).  In the smoothed version, only three outflows are detected in
the blue and two in the red, making flow-counterflow association difficult.
The north-central blueshifted component appears to be the counterpart of the
red flow when smoothed, although it is clearly the counterpart of the northwest
blue flow in the full-resolution image.

We are therefore able to detect any outflows comparable to L1448 (assuming a
favorable geometry), but are likely to see clustered outflows as single or
possibly extended lobes and will count fewer lobes than would be detected at
higher resolution.  Additionally, it is clear from this example that two
adjacent outflows with opposite polarity are not necessarily associated, and
therefore the outflows' source(s) may not be between the two lobes.  

\Figure{figures_chw5/hatchell_l1448_compare}
{Comparison of L1448 seen at a distance of 250 pc (left) versus 2 kpc (middle) with
sensitivity 0.5 K and 0.05K per 0.5 \kms\ channel respectively.  {\it Far
Left}: Position-velocity diagram (log scale) of the outflow L1448 IRS2 at its
native resolution and velocity.  L1448 IRS2 is the rightmost outflow in the contour 
plots.  The PV diagram is rotated 45\arcdeg\ from RA/Dec axes to go along the outflow
axis.
{\it Middle Left}: Position-velocity diagram (log scale) of the same outflow smoothed
and rebinned to be eight times more distant.
{\it Top Right}: The integrated map is displayed at its native resolution (linear scale).
The red contours are of the same data integrated from 6.5 to 16 \kms\ and the blue from 
-6 to 0 \kms.  Contours are at 1,3, and 5 K \kms\ ($\sim 6, 18, 30 \sigma$).  Axes
are offsets in arcseconds.  Because we are only examining the relative detectability of
outflows at two distances, we are not concerned with absolute coordinates.
{\it Bottom Right}: The same map as it would be observed at eight times greater
distance.  Axes are offsets in arcseconds assuming the greater distance.
Contours are integrated over the same velocity range as above, but are
displayed at levels 0.25,0.50,0.75,1.00 K \kms\ ($\sim 12, 24, 48, 60 \sigma$).
The entire region is detected at high significance, but dominated by confusion.
It is still evident that the red and blue lobes are distinct, but they are each
unresolved. 
}
{fig:l1448}{0.5}{0}

In order to determine overall detectability of outflows compared to Perseus, we
compare to \citet{curtis2010} in Figure \ref{fig:lengthhist}.  Out of 29
outflows in their survey with measured `lobe lengths', 22 (71\%) were smaller
than 128\arcsec\, which would be below our 18\arcsec\ resolution if 
observed at 2 kpc.  Even the largest lobes (HRF26R,
HRF28R, HRF44B) would only extend $\sim60\arcsec$ at 2 kpc.  Each lobe in the
largest outflow in our survey, Outflow 1, is $\sim80\arcsec$ (660\arcsec\ at
250pc), but no other individual outflow lobes in W5 are clearly resolved.
However, as seen in Figure \ref{fig:lengthhist}, many bipolar lobes are
\emph{separated} by more than the telescope resolution, and the overall lobe
separation distribution (as opposed to the lobe length, which is mostly unmeasured
in our sample) in W5 is quite similar to the separation distribution in
Perseus.  The 2-sample KS test gives a 25\% probability that they are drawn
from the same distribution (the null hypothesis that they are drawn from the
same distribution cannot be rejected).  
%However, \citet{Yu1999} discovered a CO 1-0 outflow in Barnard 5 that extended
%$\sim12\arcmin$, substantially larger than the largest detected in our survey 
%even at a distance of 2 kpc.

\Figure{figures_chw5/OutflowLengthHistogram}
{Histogram of the measured outflow lobe separations.  The grey hatched region shows
\citet{curtis2010} values.  The vertical dashed line represents the spatial
resolution of our survey.  The two distributions are similar.}
{fig:lengthhist}{0.5}{0}

On average, the \citet{curtis2010} outflow velocities are similar to ours
(Figure \ref{fig:widthhist}).   We detect lower velocity outflows because we do
not set a strict lower velocity limit criterion.  We do not detect the highest
velocity outflows most likely because of our poorer sensitivity to the faint
high-velocity tips of outflows, although it is also possible that no
high-velocity ($v>20$ \kms) flows exist in the W5 region.  Note that the
histogram compares quantities that are not directly equivalent: the outflows in
\citet{curtis2010} and our own data are measured out to the point at which the
outflow signal is lost, while the `region' velocities are full-width half-max
(FWHM) velocities.  

Finally, we use the detectability of outflows in Perseus to inform our
expectations in W5.  Since it appears that we can detect outflows from low-mass
protostars with sub-stellar to $\sim30L_\odot$ luminosities
at the distance of W5 and these objects should be the most numerous in a
standard initial mass function, the distribution of physical properties in W5
outflows should be similar to those in Perseus.  However, because W5 is a
somewhat more massive cloud ($M_{W5}\approx 5 M_{Perseus}$ \footnote{$M_{W5}$
is estimated from \thirteenco.  We also estimate the total molecular mass in W5 using
the X-factor and acquire $M_{W5}=5.0\ee{4}$ \msun, in agreement with
\citet{karr:triggered:2003}, who estimated a molecular mass of 4.4\ee{4} from
\twelveco\ using the same X-factor.  \citet{koenig:clustered:2008} estimated a
total gas mass of 6.5\ee{4} from a 2MASS extinction map.   The total molecular mass
in Perseus is $M_{Perseus} \sim 10^4$ \citep{bally-perseus2008}}), we might expect the
high-end of the distribution to extend to higher values of outflow mass,
momentum, and energy.  Since we will likely see clustered outflows confused
into a smaller number of distinct lobes, we expect a bias towards higher values
of the derived quantities but a lower detection rate.

\subsubsection{Velocity, Column Density, and Mass Measurements}
\label{sec:measurements}
Throughout this section, we assume that the CO
lines are optically thin and thermally excited.   The measured properties
are presented in Table \ref{tab:outflows}.  These assumptions are
likely to be invalid, so we also discuss the consequences of applying `typical'
optical depth corrections to the derived quantities.   Because we do
not measure optical depths and the optical depth correction for CO 3-2 is less
well quantified than for CO 1-0 \citep{curtis2010,Cabrit1990}\footnote{In
\citet{curtis2010}, this correction factor ranged from 1.8 to 14.3;
\citet{arce2010} did not enumerate the optical depth correction they
used but it is typically around 7 \citep{Cabrit1990}.  }, we only present the
uncorrected measurements in Table \ref{tab:outflowsderived}.

The outflow velocity ranges were measured by examining
both RA-velocity and Dec-velocity diagrams interactively using the STARLINK
GAIA data cube viewing tool.  The velocity limits are set to include
all outflow emission that is distinguishable from the cloud (i.e. the velocity
at which outflow lobes dominate over the gaussian wing of the cloud
emission) down to zero emission.  An outflow size \citep[or lobe size,
following ][]{curtis2010} was determined by integrating over the blue and red
velocity ranges and creating an elliptical aperture to include both peaks; the
position and size therefore have approximately beam-sized ($\approx18\arcsec$)
accuracy.  The integrated outflow maps are shown as red and blue contours in
Figure \ref{fig:pv2b}.  The velocity center was computed by fitting a
gaussian to the FCRAO \thirteenco\ spectrum averaged over the elliptical
aperture.

\Figure{figures_chw5/WidthHistogram}
{Histogram of the outflow line widths. {\it Black lines}: histogram of the measured
outflow widths (half-width zero-intensity, measured from the fitted central
velocity of the cloud to the highest velocity with non-zero emission).  {\it
Blue dashed lines}: outflow half-width zero-intensity (HWZI) for the outer arm (non-W5) sample.
{\it Solid red shaded}: The measured widths (HWHM) of the sub-regions as
tabulated in Table \ref{tab:regionspectra}.   
{\it Gray dotted}: Outflow $v_{max}$ (HWZI) values for Perseus
from \citet{curtis2010}. }
{fig:widthhist}{0.5}{0}


The column density is estimated from \twelveco\ J=3-2  assuming local thermal
equilibrium (LTE) and optically thin emission using the equation 
$ N(\hh) =
5.3\ee{18}\eta_{mb}^{-1} \int T_A^*(v) dv $ for $T_{ex}=20$ K. 
The derivation is given in the Appendix.
The column density in the lobes is likely to be dominated by low-velocity gas
and therefore our dominant uncertainty may be missing low-velocity emission
rather than poor assumptions about the optical depth.

The scalar momentum and energy were computed from
\begin{equation}
      p = M \frac{\sum T_A^*(v) (v-v_{c}) \Delta v}{ \sum T_A^*(v) \Delta v}
\end{equation}
\begin{equation}
      E = \frac{M}{2} \frac{\sum T_A^*(v) (v-v_{c})^2 \Delta v}{ \sum T_A^*(v) \Delta v}
\end{equation}
where $v_c$ is the \thirteenco\ 1-0 centroid velocity.  The same 
assumptions used in determining column density are applied here.

We
estimate an outflow lifetime by taking half the distance between the red and
blue outflow centroids divided by the maximum measured velocity difference
($\Delta v_{max} = (v_{max,red}-v_{max,blue})/2$), $\tau_{flow} = L_{flow} / ( 2 \Delta
v_{max})$, where $L_{flow}$ refers to the length of the flow.  This method
assumes that the outflow inclination is 45\arcdeg; if it is more parallel to
the plane of the sky, we overestimate the age, and vice-versa.  The momentum
flux is then $\dot{P} = p / \tau$.  Similarly, we compute a mass loss rate by
dividing the total outflow mass by the dynamical age, which yields what is
likely a lower limit on the mass loss rate (if the lifetime is underestimated,
the mass loss rate is overestimated, but the outflow mass is always a lower
limit because of optical depth and confusion effects).    

The dynamical ages are highly suspect since the red and blue lobes are often
unresolved or barely resolved, and diffuse emission averaged with the lobe
emission can shift the centroid position.  Additionally, it is not clear what
portion of the outflow corresponds to the centroid: the bow shock or the jet could both potentially
dominate the outflow emission.  \citet{curtis2010} discuss the many ways in
which the dynamical age can be in error.  
Our mass loss rates are similar to those in Perseus \emph{without} correcting our 
measurements for optical depth, while our outflow masses are an order of magnitude lower.
It therefore appears that our dynamical age estimates must be too low, since we have no 
reason to expect protostars in W5 to be undergoing mass loss at a greater rate than those in
Perseus.
However, given more reliable
dynamical age estimates from higher resolution observations of shock tracers,
the mass loss rates could be corrected and compared to other star-forming
regions.

Because the emission was assumed to be optically thin, the mass, column,
energy, and momentum measurements we present are strictly lower limits.  While
some authors have computed correction factors to \twelveco\ 1-0 optical depths
\citep[e.g.][]{Cabrit1990},  the corrections are different for the 3-2
transition \citep[1.8 to 14.3,][]{curtis2010}.  Additionally, CO 3-2 may
require a correction for sub-thermal excitation because of its higher critical
density (the CO 3-2 critical density is 27 times higher than CO 1-0; see
Appendix \ref{appendix:dipole} for modeling of this effect).

Additionally, most of the outflow mass is at the lowest distinguishable
velocities in typical outflows \citep[e.g.][]{arce2010}.  It is therefore
plausible that in the more turbulent W5 region, a greater fraction of the
outflow mass is blended (velocity confused) with the cloud and therefore not
included in mass, momentum, and energy measurements.  This omission could be
greater than the underestimate due to poor opacity assumptions.

The total mass of the W5 outflows is $M_{tot}\approx1.5 \msun$,
substantially lower, even with an optical depth correction of $10\times$, than
the 163 \msun\ reported in Perseus \citep{arce2010}.  \citet{arce2010} also include
a correction factor of 2.5 to account for higher temperatures in outflows and a
factor of 2 to account for emission blended with the cloud.  The temperature
correction is inappropriate for CO 3-2 (see Appendix \ref{appendix:dipole},
Figure \ref{fig:approx}), but the resulting total outflow mass in W5 with an
optical depth correction and a factor of 2 confusion correction is about 30
\msun.  In order to make our measurements consistent with a mass of 160 \msun\ , a density
upper limit in the outflowing gas of $n(\hh) < 10^{3.5} \percc$ is required,
since a lower gas density results in greater mass for a given intensity (see
Appendix \ref{appendix:dipole}, Figure \ref{fig:coradex}).  However, we
expect the total outflow mass in W5 to be greater than in Perseus because 
of the greater cloud mass, implying that the density in the flows must be even
lower, or additional corrections are needed.

The total outflow momentum is $p_{tot}\approx10.9 \msun$ \kms, versus a quoted
517 \msun \kms\ in Perseus \citep{arce2010}.  \citet{arce2010} included
inclination and dissociative shock corrections for the momentum measurements
on top of the correction factors already applied to the mass.  If these
corrections are removed from the Perseus momentum total (except for optical depth,
which is variable in their data and therefore cannot be removed), the uncorrected outflow
momentum in Perseus would be about 74 \msun \kms. The W5 outflow momentum, if
corrected with a `typical' optical depth in the range 7-14, would match or exceed
this value.  If an additional CO 3-2 excitation correction (in the range 1-20)
is applied, the W5 outflow momentum would significantly exceed that in Perseus.

Assuming a turbulent line width $\Delta v \sim 3$ \kms\ (approximately the
smallest FWHM line-width observed), the total turbulent momentum in the ambient
cloud is $p = M_{tot} \Delta v = 1.3\ee{5} \msun$ \kms, which is $\sim10^5$
times the measured outflow momentum - the outflows detected in our
survey cannot be the sole source of the observed turbulent line widths, even if
corrected for optical depth and missing mass.  

Table \ref{tab:regionspectra} presents the turbulent momentum for each
sub-region computed by multiplying the measured velocity width by the
integrated \thirteenco\ mass.  Even if the outflow measurements are 
orders of magnitude low because of optical depth, cloud blending, sub-thermal
excitation, and other missing-mass considerations, outflows contribute
negligibly to the total momentum of high velocity gas in W5.  This result is
unsurprising, as there are many other likely sources of energy in the region
such as stellar wind bubbles and shock fronts between the ionized and molecular
gas.  Additionally, in regions unaffected by feedback from the HII region (e.g.
W5NW), cloud-cloud collisions are a possible source of energy.

Figure \ref{fig:outflowhist} displays the distribution of measured properties
and compares them to those derived in the COMPLETE \citep{arce2010} and
\citet{curtis2010} HARP CO 3-2 surveys of Perseus.  Our derived masses are
substantially lower than those in \citet{arce2010} even if corrected for
optical depth, but our momenta are similar to the CPOC (COMPLETE Perseus
Outflow Candidate) sample and our energies are higher, indicating a bias
towards detecting mass at high velocities.  The bias is more heavily towards
high velocities than the CO 1-0 used in \citet{arce2010}.  The discrepancy
between our values and those of \citet{arce2010} and \citet{curtis2010} can be
partly accounted for by the optical depth correction applied in those works:
\thirteenco\ was used to correct for opacity at low velocities, where most of
the outflow mass is expected.  Those works may also have been less affected by
blending because of the smaller cloud line widths in Perseus.

%\Figure{OutflowHistograms}{Histograms of outflow properties.}{fig:outflowhist}{1.0}
\FigureFour{figures_chw5/OutflowMassHistogram_legend}{figures_chw5/OutflowEnergyHistogram}{figures_chw5/OutflowMomentumHistogram}{figures_chw5/OutflowColumnHistogram}
{Histograms of outflow physical properties.  
%All display a peak at low values in addition
%to a high excess.  None are obviously well-fit by power-law or gaussian distributions.
The solid unfilled lines are the W5 outflows (this paper), the forward-slash
hashed lines show \citet{arce2010} CPOCs , the dark gray
shaded region shows \citet{arce2010} values for known outflows in Perseus, and
the light gray, backslash-hashed regions show \citet{curtis2010} CO 3-2 outflow
properties.  The outflow masses measured in Perseus are systematically higher
partly because both surveys corrected for line optical depth using \thirteenco.
The medians of the distributions are 0.017, 0.044, 0.33, and 0.14 \msun\ for
W5, Curtis, Arce Known, and Arce CPOCs respectively, which implies that an
optical depth and excitation correction factor of 2.5-20 would be required to
make the distributions agree (although W5, being a more massive region, might
be expected to have more massive and powerful outflows).  It is likely that CO
3-2 is sub-thermally excited in outflows, and CO outflows may be destroyed by
UV radiation in the W5 complex while they easily survive in the lower-mass
Perseus region, which are other factors that could push the W5 mass
distribution lower.
}
{fig:outflowhist}

The momentum flux and mass loss rate are compared to the values derived in
Perseus by \citet{hatchell2007} and \citet{curtis2010} in Figures
\ref{fig:outflowPflux} and \ref{fig:outflowmdot}.  Both of our values are
computed using the dynamical timescale $\tau_d$ measured from outflow lobe
separation, while the \citet{hatchell2007} values are derived using a more
direct momentum-flux measurement in which the momentum flux contribution 
of each pixel in the resolved outflow map is considered.  
The derived
momentum fluxes (Figure \ref{fig:outflowPflux}) are approximately consistent
with the \citet{curtis2010} Perseus momentum fluxes; \citet{curtis2010} measure
momentum fluxes in a range $1\ee{-6}<\dot{P}<7\ee{-4}$ \msun \kms \peryr,
higher than our measured $6\ee{-7}<\dot{P}<1\ee{-4}$ \msun \kms \peryr\ by
approximately the opacity correction they applied.  As seen in Figure
\ref{fig:outflowPflux}, the \citet{hatchell2007} momentum flux measurements in
Perseus cover a much lower range $6\ee{-8}<\dot{P}<2\ee{-5}$ \msun \kms \peryr\
and are not consistent with our measurements.  This disagreement is most likely
because of the difference in method.  The W5 outflows exhibit
substantially higher mass-loss rates and momentum fluxes if we assume a factor
of 10 opacity correction, as expected from our bias toward higher-velocity,
higher-mass flows.

\Figure{figures_chw5/OutflowMomentumFluxHistogram}
{Histogram of the measured outflow momentum fluxes.  The black thick line shows
our data, the grey shaded region shows the \citet{hatchell2007} data, and the
hatched region shows \citet{curtis2010} values.  Our measurements peak squarely
between the two Perseus JCMT CO 3-2 data sets, although the \citet{curtis2010}
results include an opacity correction that our data do not, suggesting that our
results are likely consistent with \citet{curtis2010} but inconsistent with the
\citet{hatchell2007} direct measurement method.}
{fig:outflowPflux}{0.5}{0}

\Figure{figures_chw5/OutflowMassLossRateHistogram}
{Histogram of the measured mass loss rate.  The black thick line shows our
data, while the grey shaded region shows the \citet{hatchell2007} data, which
is simply computed by $\dot{M} = \dot{P} \times 10 / 5$ \kms, where the factor
of 10 is a correction for opacity.  Our mass loss rates are very comparable to
those of \citet{hatchell2007}, but different methods were used so the
comparison may not be physically meaningful.  \citet{curtis2010} (hatched) used
a dynamical time method similar to our own and also derived similar mass loss
rates, although their mass measurements have been opacity-corrected using the
\thirteenco\ 3-2 line.  Because our mass loss rates agree reasonably with
Perseus, but our outflow mass measurements are an order of magnitude low, we
believe our dynamical age estimates to be too small.
}
{fig:outflowmdot}{0.5}{0}


\subsection{Structure of the W5 molecular clouds: A thin sheet?}
The W5 complex extends $\sim 1.6\arcdeg \times 0.7 \arcdeg$ within 20\arcdeg\
of parallel with the galactic plane.  At the assumed distance of 2 kpc, it has
a projected length of $\sim60$ pc (Figure \ref{fig:outflows_on_co32}).
%% In the far-infrared (Figure\ref{fig:outflows_on_IRAS100}) and %%
In the 8 \um\ band (Figure \ref{fig:color_overview}), the region appears to
consist of two blown-out bubbles with $\sim 10-15$ pc radii centered on
$\ell=138.1, b=1.4$ and $\ell=137.5,b=0.9$.  While the bubbles are filled in
with low-level far-infrared emission, there is no CO detected down to a
$3-\sigma$ limit of 3.0 K \kms\ (\twelveco\ 1-0), 2.4 K \kms\ (\twelveco\ 3-2,
excepting a few isolated clumps), and 1.5 K \kms\ (\thirteenco\ 1-0).  Using the 
X-factor (the CO-to-\hh\ conversion factor) for \twelveco\, $N(\hh) =
3.6\ee{20} \persc / (\mathrm{K}~\kms)$, we derive an upper limit $N(\hh) <
1.1\ee{21}\persc$, or $A_V \lesssim 0.6$.  Individual `wisps' and `clumps' of
CO can sometimes be seen, particularly towards the cloud edges, but in general
the bubbles are absent of CO gas.

% The 
% strictest column limit comes from the \twelveco\ 3-2 observations (assuming
% LTE, $\tau<<1$, and T$_{ex}=20K$), with a $3-\sigma$ column limit $N_{\hh} <
% 7.8\ee{18}$ within an 18\arcsec\ beam, or $A_V \lesssim 0.1$ magnitudes
% \citep[using the][conversion $A_V=1.9\ee{21} \persc$]{Bohlin1978}\footnote{ The
% \twelveco\ 1-0 line gives a limit of 3.8\ee{19} \persc\ or $A_V=0.02$ in a
% 45\arcsec\ FWHM beam.  This limit may be more restrictive than the CO 3-2 limit
% if CO 3-2 is sub-thermally excited, though at this resolution the CO 3-2 limit
% is $N(\hh) < 3.1\ee{18}$.}.  

Given such low column limits,  the W5 cloud must be much smaller along the line
of sight than its $\sim50$ pc size projected on the sky.  Alternately, along
the line-of-sight, the columns of molecular gas are too low for CO to
self-shield, and it is therefore destroyed by the UV radiation of W5's O-stars.
In either case, there is a significant excess of molecular gas in the plane of
the sky compared to the line of sight, which makes W5 an excellent location to
perform unobscured observations of the star formation process.  The implied
thin geometry of the W5 molecular cloud may therefore be similar to the bubbles
observed by \citet{Beaumont2009}, but on a larger scale.

There is also morphological evidence supporting the face-on
hypothesis.  In the AFGL 4029 region (Section \ref{sec:afgl4029}) and all along
the south of W5, there are ridges with many individual cometary `heads'
pointing towards the O-stars that are unconfused along the line of sight.  This
sort of separation would not be expected if we were looking through the clouds
towards the O-stars.  W5W, however, presents a counterexample in which there
are two clouds along the line of sight that may well be masking a more complex
geometry.

% \subsection{\thirteenco\ 1-0}
% The \thirteenco\ integrated intensity was used to compute the \hh\ column using
% \begin{equation}
%     N_{\hh} = 1.694\ee{20} / (1-e^{(-5.3/T_{ex})}) \int T_{A}^* (v)
%     dv\ \persc,
% \end{equation} 
% where we have assumed the excitation temperature is the gas kinetic temperature
% $T_{ex} = T_K = 20$ K.  We do not include a main-beam efficiency correction
% because the factor was unknown; this omission may introduce a systematic
% underestimate of up to $\sim40\%$ (based on the \twelveco\ 1-0 $\eta_{MB}$).
% The 1-$\sigma$ column sensitivity limit is 4.4\ee{20} \persc\ or 0.16 \msun for
% an assumed 2 kpc distance in 22\farcs5 square pixels.

\section{Sub-regions}
\label{sec:subregions}

Individual regions were selected from the mosaic for comparison.  All regions
with multiple outflows and indicators of star formation activity were named and
included as regions for analysis.  Additionally, three ``inactive'' regions were
selected based on the presence of \thirteenco\ emission but the lack of
outflows in the \twelveco\ 3-2 data.  Finally, two regions devoid of CO emission
were selected as a baseline comparison and to place upper limits on the molecular
gas content of the east and west `bubbles'.  The regions are identified on the
integrated \thirteenco\ image in Figure \ref{fig:regionboxes_on_CO}. 

%Cloud velocity widths were measured in a range 3-7 \kms (full-width
%zero-intensity).  This range is inferred from by-eye inspection of
%position-velocity diagrams and 1d gaussian fits to a few (arbitrary) lines of
%sight. 
Average spectra were taken of each ``region'' within the indicated box.
Gaussians were fit to the spectrum to determine line-widths and centers (Figure
\ref{fig:regionspectra}, Table \ref{tab:regionspectra}).  Gaussian fits were
necessary because in many locations there are at least two velocity components,
so the second moment (the ``intensity-weighted dispersion'') is a poor
estimator of line width.  Widths ranged from $v_{FWHM} = 2.3$ to 6.2 \kms\
(Figure \ref{fig:widthhist}).  


%\Table{cccccccccc}{Gaussian fit parameters of sub-regions}
{{Region} & {Velocity 1} & {Width 1} & {Amplitude 1} & {Velocity 2} & {Width 2} & {Amplitude 2} &  &  & \\
 & {(\kms)} & {(FWHM, \kms)} & {(K)} & {(\kms)} & {(FWHM, \kms)} & {(K)} &  & \\}
{tab:regionspectra}
{
S201 & -38.04 & 3.149 & 2.35 & - & - & -\\
AFGL4029 & -38.91 & 3.3605 & 1.48 & - & - & -\\
LWCas & -38.83 & 3.478 & 2.33 & - & - & -\\
W5W & -41.37 & 3.8775 & 3.07 & -36.16 & 3.8305 & 1.90\\
W5NW & -36.37 & 3.854 & 1.6 & - & - & -\\
W5NWpc & -36.37 & 3.713 & 1.19 & -41.81 & 4.3475 & 0.47\\
W5SW & -42.78 & 4.136 & 0.6 & -36.34 & 4.183 & 0.22\\
W5S & -40.15 & 2.914 & 0.34 & -35.76 & 2.2795 & 0.40\\
Inactive1 & -42.91 & 2.6555 & 0.75 & -39.38 & 4.2065 & 0.42\\
Inactive2 & -38.94 & 3.7365 & 1.2 & - & - & -\\
empty & -37.81 & 5.217 & 0.04 & - & - & -\\
\hline \\
\thirteenco\ fits &&&&&&& \thirteenco      & \thirteenco          & \thirteenco       \\
                   &&&&&&&              mass &              momentum &              energy\\
                   &&&&&&&(\msun)           &(\msun \kms)         &(ergs)              \\
\hline \\
S201 & -37.97 & 2.5615 & 0.56 & - & - & - & 1300 & 3500 & 8.9\ee{46}\\
AFGL4029 & -38.66 & 2.35 & 0.35 & - & - & - & 2600 & 6100 & 1.4\ee{47}\\
LWCas & -38.75 & 2.679 & 0.51 & - & - & - & 3700 & 10000 & 2.7\ee{47}\\
W5W & -41.23 & 2.773 & 1.09 & -36.51 & 3.5485 & 0.47 & 4500 & 13000 & 3.5\ee{47}\\
W5NW & -36.1 & 3.431 & 0.7 & - & - & - & 5300 & 18000 & 6.3\ee{47}\\
W5NWpc & -36.18 & 3.3135 & 0.42 & -41.44 & 3.619 & 0.14 & 15000 & 50000 & 1.6\ee{48}\\
W5SW & -42.6 & 3.807 & 0.1 & -36.15 & 4.2535 & 0.05 & 790 & 3000 & 1.1\ee{47}\\
W5S & -39.9 & 2.444 & 0.07 & -35.48 & 2.209 & 0.08 & 320 & 790 & 1.9\ee{46}\\
Inactive1 & -42.58 & 2.5145 & 0.1 & -38.97 & 2.82 & 0.07 & 1400 & 3500 & 8.7\ee{46}\\
Inactive2 & -38.82 & 3.196 & 0.37 & - & - & - & 3100 & 9900 & 3.2\ee{47}\\
empty & -38.44 & 4.7705 & 0.02 & - & - & - & 340 & 1600 & 7.8\ee{46}\\
}{%empty comment
}
 %tab:regionspectra
%\include{regiontable_mnras} % tab:regiontable

\Figure{figures_chw5/regionspectra_grid}
{Spatially averaged spectra of the individual regions analyzed.  \twelveco\ 3-2
is shown by thick black lines and \thirteenco\ 1-0 is shown by thin red lines.
Gaussian fits are overplotted in blue and green dashed lines, respectively.
The fit properties are given in Table
\ref{tab:regionspectra}.}{fig:regionspectra}{0.5}{0}


\subsection{Sh 2-201}
Sh 2-201 is an HII region and is part of the same molecular cloud as the
bright-rimmed clouds in W5E, but it does not share a cometary shape with these
clouds (Figure \ref{fig:S201}).  Instead, it is internally heated and has its
own ionizing source \citep{Felli1987}.  The AFGL 4029 cloud edge is at a projected
distance of $\sim7$ pc from the nearest exposed O-star, and the closest
illuminated point in the Spitzer 8 and 24 \um\ maps is at a projected distance
of $\sim 5$ pc.  The star forming process must therefore have begun before
radiation driven shocks from the W5 O-stars could have impacted the cloud.  
%There are 4 bipolar CO outflows
%associated with this region.  The outflows come primarily from the 8 \um- faint
%part of the region, perhaps because the 8 \um\ emission traces the edges of a
%blown out gas-poor cavity.

\Figure{figures_chw5/S201_CO32on8UM} %{regionoverlays/S201_BGPSon24UM}
{Small scale map of the Sh 2-201 region plotted with CO 3-2 contours integrated
from -60 to -20 \kms\ at levels 3,7.2,17.3,41.6, and 100 K \kms.  
% (left) The MIPS 24 \um\ image is
% displayed inverted in log scale from \lowmips\ to \highmips\ MJy \persr.
The IRAC 8 \um\ image is displayed in inverted log scale from \lowirac\
to \highirac\ MJy \persr. Contours of the CO 3-2 cube integrated from
-60 to -20 \kms\ are overlaid at logarithmically spaced levels from 3 to 100 K
\kms\ (3.0,7.2,17.3,41.6,100; $\sigma\approx0.7$ K \kms).  The
ellipses represent the individual outflow lobe apertures mentioned in Section
\ref{sec:measurements}.
}{fig:S201}{0.5}{0}

\subsection{AFGL 4029}
\label{sec:afgl4029}
% do I actually add any interesting information about AFGL 4029?
AFGL 4029 is a young cluster embedded in a cometary cloud (Figure
\ref{fig:afgl4029}).  There is one clear bipolar outflow and 6 single-lobed
flows that cannot be unambiguously associated with an opposite direction
counterpart.  The cluster is mostly unresolved in the data presented here and
is clearly the most active CO clump in W5.  It contains a cluster of at least 30 B-stars
\citep{Deharveng1997}.  The outflows from this region have a full width $\Delta
v \approx 30$ \kms, which is entirely inconsistent with a radiation-driven
inflow or outflow since it is greater than the sound speed in the ionized
medium.

The northeast cometary cloud is strongly affected by
the W5 HII region.  It has an outflow in the head of the cloud (Figure
\ref{fig:necomet}), and the cloud shows a velocity gradient with distance from
the HII region.  The polarity of the gradient suggests that the cometary cloud
must be on the far side of the ionizing O-star along the line of sight assuming
that the HII region pressure is responsible for accelerating the cloud edge.

\Figure{figures_chw5/AFGL4029_CO32on8UM}%{regionoverlays/AFGL4029_BGPSon24UM}
{Small scale map of the AFGL 4029 region plotted with CO 3-2 contours integrated
from -60 to -20 \kms\ at levels 3,7.2,17.3,41.6, and 100 K \kms.  
The IRAC 8 \um\ image is displayed in inverted log scale from \lowirac\ to
\highirac\ MJy \persr. Contours of the CO 3-2 cube integrated from -60 to -20
\kms\ are overlaid at logarithmically spaced levels from 3 to 100 K \kms\
(3.0,7.2,17.3,41.6,100; $\sigma\approx0.7$ K \kms).  Outflows 26-32 are ejected
from a forming dense cluster.  A diagram displaying the kinematics of the
northern cometary cloud is shown in Figure \ref{fig:necomet}. }
{fig:afgl4029}{0.5}{0}


\Figure{figures_chw5/NECometContours}
{The northeast cometary cloud.  Contours are shown at 0.5,1,2, and 5 K \kms\
integrated over the ranges -44.0 to -41.9 \kms\ (blue) and -38.1 to -35.6 \kms\
(red).  There is a velocity gradient across the tail, suggesting that the front
edge is being pushed away along the line of sight.}
{fig:necomet}{0.5}{0}

\subsection{W5 Ridge}
\label{sec:w5ridge}
The W5 complex consists of two HII region bubbles separated by a ridge of
molecular gas (Figure \ref{fig:lwcas}).  This ridge contains the LW Cas optical
nebula, a reflection nebula around the variable star LW Cas, on its east side
and an X-shaped nebula on the west.  The east portion of LW Cas Nebula is
bright in both the continuum and CO J=3-2 but lacks outflows (see Figure
\ref{fig:lwcas}).  The east portion also has the highest average peak antenna
temperature, suggesting that the gas temperature in this region is
substantially higher than in the majority of the W5 complex (higher spatial
densities could also increase the observed $T_A$, but the presence of nearby heating
sources make a higher temperature more plausible).  It is possible
that high gas temperatures are suppressing star formation in the cloud.
Alternately, the radiation that is heating the gas may destroy any outflowing
CO, which is more likely assuming the two Class I objects identified in this
region by \citet{koenig:clustered:2008} are genuine protostars.

\Figure{figures_chw5/LWCas_CO32on24UM} %{regionoverlays/LWCas_BGPSon8UM}
{Small scale map of the LW Cas nebula plotted with CO 3-2 contours integrated
from -60 to -20 \kms\ at levels 3,7.2,17.3,41.6, and 100 K \kms.  The feature containing outflows 20 and
21 is the X-shaped ridge referenced in Section \ref{sec:w5ridge}.  This
sub-region is notable for having very few outflows associated with the most
significant patches of CO emission.   The gas
around it is heated on the left side by the O7V star HD 18326 ($D_{proj}=8.5$
pc), suggesting that this gas could be substantially warmer than the other
molecular clouds in W5.
}{fig:lwcas}{1.0}{0}

The ridge is surprisingly faint in HI 21 cm emission compared to the two HII regions
(Figure \ref{fig:HIridge}) considering its 24 \um\ surface brightness.  The
integrated HI intensity from -45 to -35 \kms\ is $\sim800$ K \kms, whereas in
the HII region bubble it is around 1000 K \kms.  The CO-bright regions show
lower levels of emission similar to the ridge at 700-800 K \kms.  However, the
ridge contains no CO gas and very few young stars \citep[Figure 7 in
][]{koenig:clustered:2008}.  It is possible that the ridge contains cool HI but
has very low column-densities along the direction pointing towards the O-stars,
in which case the self-shielding is too little to prevent CO dissociation.
This ridge may therefore be an excellent location to explore the transition
from molecular to atomic gas under the influence of ionizing radiation in
conditions different from high-density photodissociation (photon-dominated)
regions.

\FigureTwo{figures_chw5/HI_IRAS100_overlay}{figures_chw5/24micron_21cmcontOverlay}
{{\it Top:} The DRAO 21 cm HI map integrated from -45 to -35 \kms\ displayed in grayscale
from 700 (black) to 1050 (white) K \kms\ with IRAS 100 \um\ contours (red, 40 MJy sr$^{-1}$) and
\twelveco\ 1-0 contours integrated over the same range (white, 4 K \kms)
overlaid.  The ridge of IRAS 100 \um\ emission at $\ell=138.0$ coincides with a
relative lack of HI emission at these velocities, suggesting either that there
is less or colder gas along the ridge.  {\it Bottom:} The Spitzer 24 \um\ map
with 21 cm continuum contours at 6, 8, and 10 MJy sr$^{-1}$ overlaid.  The IRAS contours
are also overlaid to provide a reference for comparing the two figures and to demonstrate
that the HII region abuts the cold-HI area.  The moderate
excess of continuum emission implies a somewhat higher electron density along the 
line of sight through the ridge.}
{fig:HIridge}{1.0}

We examine Outflow 20 as a possible case for pressure-driven implosion
(radiation, RDI, or gas pressure, PDI) by examining the relative timescales of the outflow
driving source and the HII-region-driven compression front.  A typical
molecular outflow source (Class 0 or I) has a lifetime of $\sim5\ee{5}$ years
\citep{Evans2009}.   Given that there is an active outflow at the head of this
cloud, we use 0.5 MYr as an upper limit.  The approximate distance from this
source to the cloud front behind it is $\sim 3.3$ pc.  If we assume the cloud
front has been pushed at a constant speed $v\leq c_{II} \approx 10 \kms$,  we
derive a lower limit on its age of 0.3 MYr.  While these limits allow for the
protostar to be older than the compression front by up to 0.2 MYr, it is likely
that the compression front moved more slowly (e.g., 3 \kms\ if it was pushed
by a D-type shock front) and that the protostar is not yet at the end of its
lifetime - it is very plausible that this soure was born in a radiation-driven
implosion.


\subsection{Southern Pillars}
\label{sec:pillars}
There are 3 cometary clouds that resemble the ``elephant trunk'' nebula in IC
1396 (Figure \ref{fig:comets}).  Each of these pillars contains evidence of at
least one outflow in the head of the cloud (see the supplementary materials, outflows
16-19 and 38)
%Figures \ref{fig:pv16},\ref{fig:pv17}, \ref{fig:pv18}, \ref{fig:pv19}, and \ref{fig:pv38}).  
These pillars are low-mass and isolated; there is no other outflow activity in
southern W5.  However, because of the bright illumination on their northern
edges and robust star formation tracers, these objects present a reasonable 
case for triggered star formation by the RDI mechanism.

The kinematics of these cometary clouds suggest that they have been pushed in
different directions by the HII region (Figure \ref{fig:comets}).  The central
cometary cloud (Figure \ref{fig:comets}b) has two tails.  The southwest tail
emission peaks around -39.5 \kms\ and the southeast tail peaks at -41.5 \kms,
while the head is peaked at an intermediate -40.5 \kms.  These velocity shifts
suggest that the gas is being accelerated perpendicular to the head-tail axis
and that the southeast tail is on the back side of the cometary head, while the
southwest tail is on the front side.  The expanding HII region is crushing this
head-tail system.

The southeast cometary cloud (Figure \ref{fig:comets}a) peaks at -35.0 \kms.
There are no clearly-separated CO tails as in the central cloud, but there is a
velocity shift across the tail, in which the west (right) side is blueshifted
compared to the east (left) side, which is the opposite sense from the central
cometary cloud.

The southwest cometary cloud (Figure \ref{fig:comets}c) peaks at -40.3 \kms\
and has weakly defined tails similar to the central cloud.  Both of its tails
are at approximately the same velocity (-42.5 \kms).

The kinematics of these tails provide some hints of their 3D structure and
location in the cloud.  Future study to compare the many cometary flows in W5
to physical models and simulations is warranted.  Since these flows are likely
at different locations along the line of sight (as required for their different
velocities), analysis of their ionized edges may allow for more precise
determination of the full 3D structure of the clouds relative to their ionizing
sources.

\Figure{figures_chw5/SouthComet_VelocityContours}
{CO 3-2 contours overlaid on the Spitzer 8 \um\ image of the W5S cometary
clouds described in Section \ref{sec:pillars}.  Contours are color-coded by velocity
and shown for 0.84 \kms\ channels at levels of 1 K (a, b) and 0.5 K (c).  The
velocity ranges plotted are (a) -41.5 to -33.0 \kms (b) -44.7 to -36.7 \kms\
(c) -43.6 to -35.6 \kms.  The labels show the minimum, maxmimum, and middle
velocities to guide the eye.  The grey boxes indicate the regions selected for
CO contours; while there is CO emission associated with the southern 8 \um\
emission, we do not display it here. The velocity gradients are discussed in
Section \ref{sec:pillars}.
}
{fig:comets}{0.5}{0}

%\Figure{regionoverlays/W5S_CO32on24UM}%{regionoverlays/W5S_BGPSon8UM}
%{Small scale map of the W5 S region showing BGPS 1.1 mm contours overlaid on
%the Spitzer 24 \um\ map.  The 1.1 mm sources are significantly brighter in this
%region than in the W5SE region, suggesting that they are either very significantly
%more condensed (reducing the impact of Bolocam's effective filter function) or they
%are hotter.
%}{fig:w5s}{1.0}

\subsection{W5 Southeast}
The region identified as W5SE has very little star formation activity despite
having significant molecular gas (M$_{\thirteenco} \sim 800$\msun).  While
there are two outflows and two Class I objects \citep{koenig:clustered:2008} in the
southeast of the two clumps ($\ell=138.15,b=0.77$, Figure \ref{fig:w5SE}), the main clump
($\ell=138.0,b=0.8$) has no detected outflows.  The CO emission is particularly
clumpy in this region, with many independent, unresolved clumps both in
position and velocity.  In the 8 and 24 micron Spitzer images, it is clear that
these clouds are illuminated from the northwest.  This region represents a case
in which the expanding HII region has impacted molecular gas but has not
triggered additional star formation.  The high clump-to-clump velocity
dispersion observed in this region may be analogous to the W5S cometary clouds
(Section \ref{sec:pillars}) but without condensed clumps around which to form
cometary clouds.

\Figure{figures_chw5/W5SE_CO32on8UM}
{Small scale map of the W5 SE region showing the star-forming clump containing
outflows 39 and 40 and the non-star-forming clump at $\ell=138.0,b=0.8$. 
CO 3-2 contours integrated from -60 to -20 \kms\ are displayed at levels
3,7.2,17.3,41.6, and 100 K \kms.
}{fig:w5SE}{0.5}{0}

\subsection{W5 Southwest}
There is an isolated clump associated with outflows in the southwest part of W5
(Figure \ref{fig:w5SW}) at $v_{LSR} \sim -45~\kms$.  While this clump is likely
to be associated with the W5 region, it shows little evidence of interaction
with the HII region.  If it is eventually impacted by the expanding ionization
front (i.e. if it is within the W5 complex), this clump will be an example of
``revealed'', not triggered, star formation.  

The other source in W5SW is a cometary cloud with a blueshifted head and
redshifted tail (Figure \ref{fig:swcomet}; Outflow 13).  The head contains a
redshifted outflow; no blueshifted counterpart was detected (the velocity
gradient displayed in Figure \ref{fig:swcomet} is smaller than the outflow
velocity and is also redshifted away from the head).  The lack of a blueshifted
counterpart may be because the flow is blowing into ionized gas where the CO is
dissociated.

Because of its evident interaction with the HII region, this source is an
interesting candidate for a non-protostellar outflow impersonator.  However,
because the head is blueshifted relative to the tail, we can infer that the
head has been accelerated towards us by pressure from the HII region, implying
that it is in the foreground of the cloud.  Given this geometry, a
radiation-driven flow would appear blueshifted, not redshifted, as the detected
flow is.  Additionally, the outflow is seen as fast as 7.5 \kms\ redshifted
from the cloud, which is a factor of 2 too fast to be driven by radiation in a
standard D-type shock.  Finally, the outflow velocity is much greater than seen
in a simulation of a cometary cloud by \citet{Gritschneder2010}, while the
head-to-tail velocity gradient is comparable.


\Figure{figures_chw5/W5SW_CO32on8UM}%{regionoverlays/W5SW_BGPSon24UM}
{Small scale map of the W5 SW region plotted with CO 3-2 contours integrated
from -60 to -20 \kms\ at levels 3,7.2,17.3,41.6, and 100 K \kms. Outflow 13 is at the head of a 
cometary cloud (Figure \ref{fig:swcomet}) and therefore has clearly been
affected by the expanding HII region, but the region including bipolar Outflow
10 shows no evidence of interaction with the HII region. 
}{fig:w5SW}{0.5}{0}

\Figure{figures_chw5/SWComet_VelocityContours}
{The cometary cloud in the W5 Southwest region (Outflow 13).  Contours are
shown at 1 K for 0.84 \kms\ wide channels from -37.2 \kms\ (blue) to -30.5
\kms\ (red).  The head is clearly blueshifted relative to the tail and contains
a spatially unresolved redshifted outflow.}
{fig:swcomet}{0.5}{0}


\subsection{W5 West / IC 1848}
\label{sec:i02459}
There is a bright infrared source seen in the center of W5W (IRAS 02459+6029;
Figure \ref{fig:w5w}), but the nearest CO outflow lobe is $\approx1$ pc away.
The nondetection may be due to confusion in this area: there are two layers of
CO gas separated by $\sim$5 \kms, so low-velocity outflow detection is more
difficult. % the minimum detectable outflow velocity in this region is $\sim10$ \kms.
Unlike the rest of the W5 complex, this region appears to have multiple
independent confusing components along the line of sight (Figure
\ref{fig:regionspectra}), and therefore the CO data provide much less
useful physical information (multiple components are also observed in the
\thirteenco\ data, ruling out self-absorption as the cause of the multiple
components).

%The presense of two separate sheets confuses the physical association in this
%region.  A velocity gradient is present in both sheets, but they are less
%separated in velocity on the side nearest the W5 O-stars, so it is unlikely
%that the sheets have been blown away from each other.

%The BGPS source G136.828+01.064 ($M_{1.1 mm}\sim220\msun$) is associated with a
%deeply embedded cluster and 5 Class I objects from
%\citet{koenig:clustered:2008}.  This location marks the worst velocity
%confusion in CO but contains hints that there may be outflows.  This region is a
%good candidate for high-resolution follow-up.


\Figure{figures_chw5/W5W_8um_TwoVelocityOverlay}%{regionoverlays/W5W_BGPSon24UM}
{Small scale map of the W5 W region.  The IRAC 8 \um\ image is displayed in
inverted log scale from \lowirac\ to \highirac\ MJy \persr.  Contours of the CO
3-2 cube integrated from -50 to -38 \kms\ (blue) and -38 to -26 \kms\ (red) are
overlaid at levels 5,10,20,30,40,50,60 K \kms\ $\sigma\approx0.5$ K \kms.  The
lack of outflow detections is partly explained by the two spatially overlapping
clouds that are adjacent in velocity.
}{fig:w5w}{0.5}{0}


\subsection{W5 NW}
The northwest cluster  containing outflows 1-8 is at a
slightly different velocity ($\sim-35~\kms$) than the majority of the W5 cloud
complex ($\sim-38~\kms$; Figure \ref{fig:w5pv}), but it shares contiguous
emission with the neighboring W5W region.  % (see Figure \ref{fig:w5nwpercompare}).     
It contains many outflows and therefore is actively forming stars  (Figure
\ref{fig:w5nw}).  However, this cluster exhibits much lower CO brightness
temperatures and weaker Spitzer 8 \um\ emission than the ``bright-rimmed
clouds'' seen near the W5 O-stars. We therefore conclude that the region has
not been directly impacted by any photoionizing radiation from the W5 O-stars.

The lack of interaction with the W5 O-stars implies that the star formation in
this region, though quite vigorous, has not been directly triggered.  Therefore
not all of the current generation of star formation in W5 has been triggered on
small or intermediate scales (e.g., radiation-driven implosion).  Even the
``collect and collapse'' scenario seems unlikely here, as the region with the
most outflows also displays some of the smoothest morphology (Figures
\ref{fig:outflows_on_co32} and \ref{fig:w5nw}); in ``collect and collapse'' the
expansion of an HII region leads to clumping and fragmentation, and the spaces
between the clumps should be relatively cleared out.

\Figure{figures_chw5/pvdiagrams/w5_12co_latPV_outflows}
{Integrated longitude-velocity diagram of the W5 complex from $b=0.25$ to
$b=2.15$ in \twelveco\ 1-0 from the FCRAO OGS.  The W5NW region is seen at a
distinct average velocity around $\ell=136.5$, $v_{LSR}=-34$ \kms.  The red and
blue triangles mark the longitude-velocity locations of the detected outflows.
In all cases, they mark the low-velocity start of the outflow.}
{fig:w5pv}{0.5}{0}


\Figure{figures_chw5/W5NW_CO32on8UM}%{regionoverlays/W5NW_BGPSon24UM}
{Small scale map of the W5 NW region plotted with CO 3-2 contours integrated
from -60 to -20 \kms\ at levels 3,7.2,17.3,41.6, and 100 K \kms. Despite its
distance from the W5 O-stars, $D_{proj}\approx20$ pc, this cluster is the most
active site of star formation in the complex as measured by outflow activity.}
{fig:w5nw}{0.5}{0}

\section{Discussion}

\subsection{Comparison to other outflows}
\label{sec:comparison}
The outflow properties we derive are similar to those in the B0-star forming
clump IRAS 05358+3543 \citep[$M\approx600\msun$][]{Ginsburg2009}, in which CO
3-2 and 2-1 were used to derive outflow masses in the range 0.01-0.09 \msun.
However, some significantly larger outflows, up to 1.6 pc in one direction were
detected, while the largest resolved outflow in our survey was only 0.8 pc (one
direction).  
%With 54 detected outflows across a star-forming region that
%includes many young B-stars (and forming B-stars, like AFGL 4029) and contains
%an order of magnitude more mass than IRAS 05358+3543, the lack of long outflows
%is somewhat surprising.

As noted in Section \ref{sec:percompare}, 
the total molecular mass in W5 is larger than
Perseus, $M_{W5}\sim 4.5\ee{4} \msun$ while $M_{Perseus}\sim10^4\msun$
\citep{bally-perseus2008}.  The length distribution of outflows (Figure
\ref{fig:lengthhist}) is strikingly similar, while other physical properties
have substantially different mean values with or without correction factors
included.

The W5NW region is more directly comparable to Perseus, with a total mass of
$\sim1.5\ee{4}$ \msun\ (Table \ref{tab:regionspectra}) and a similar size.  In
Figure \ref{fig:regionboxes_on_CO}, we show both the W5NW region, which
contains all of the identified outflows, and the W5NWpc region, which is a
larger area intended to be directly comparable in both mass and spatial scale
to the Perseus molecular cloud.  The W5NWpc region contains more than an order of
magnitude more turbulent energy than the Perseus complex \citep[$E_{turb,Per} =
1.6\ee{46}$ ergs,][]{arce2010} despite its similar mass.  Even the smaller W5NW
region has $\sim5\times$ more turbulent energy than the Perseus complex,
largely because of the greater average line width ($\sigma_{FWHM,W5NW}\approx
3.5$ \kms).  As with the whole of W5, there is far too much turbulent energy in
W5NW to be provided by outflows alone, implying the presence of another driver
of turbulence.

Figure \ref{fig:w5nwpercompare} shows the W5NWpc region and Perseus molecular cloud
on the same scale, though in two different emission lines.  The Perseus cloud
contains many more outflows and candidates (70 in Perseus vs. 13 in W5NWpc)
despite a much larger physical area surveyed in W5.  While it is likely that
many of the W5W outflows will break apart into multiple flows at higher
resolution, it does not seem likely that each would break apart into 5 flows,
as would be required to bring the numbers into agreement.  Since the highest
density of outflows in Perseus is in the NGC 1333 cloud, it may be that there
is no equivalently evolved region in W5NWpc.  The W5W region may be comparably
massive, but it is also confused and strongly interacting with the W5 HII
region - either star formation is suppressed in this region, or outflows are
rendered undetectable.  In the latter case, the most likely mechanisms for
hiding outflows are molecular dissociation by ionizing radiation and velocity
confusion.

Another possibility highlighted in Figure \ref{fig:w5nwpercompare} is that the
W5NW region is interacting with the W4 bubble.  The cloud in the top right of
Figure \ref{fig:w5nwpercompare} is somewhat cometary, has higher peak
brightness temperature, and is at a slightly different velocity (-45 \kms) than
W5NW.  The velocity difference of $\sim8$ \kms could simply be two clouds
physically unassociated along the line of sight, or could indicate the presence
of another expanding bubble pushing two sheets of gas away from each other.
Either way, the northwest portion of the W5NW region is clumpier than the
area in which the outflows were detected, and it includes no outflow detections.

\FigureTwo{figures_chw5/W5NW_perseuscomparison}{figures_chw5/perseuscomparison_arceellipses}
{(a) An integrated CO 3-2 image of the W5W/NW region with ellipses overlaid
displaying the locations and sizes of outflows.  The dark red and blue ellipses
in the lower right are associated with outer-arm outflows.  W5W is the bottom-left,
CO-bright region.  W5NW is the top-center region containing the cluster of outflows.
(b) An integrated CO 1-0 image of the Perseus molecular cloud from the COMPLETE
survey \citep{arce2010}.  Note that the spatial scale is identical to that of
(a) assuming that W5 is 8 times more distant than Perseus.  The green ellipses
represent \citet{arce2010} CPOCs while the orange represent known outflows from
the same paper.}
{fig:w5nwpercompare}{1.0}


% However, the Perseus molecular cloud contains a total of 4800 \msun\
% \citep[from near-IR extinction][]{Evans2009}, while many independent clouds
% within W5 contain a similar mass (Table \ref{tab:regionspetra}), and the total
% gas mass in W5 is $\sim15\times$ larger \citep[also from near-IR extinction
% extinction][]{koenig:clustered:2008}.  Assuming a similar level of star
% formation activity (or star formation efficiency per gas mass), we should see
% substantially more outflows and more outflowing mass than is observed.  The
% simplest explanation for the lower outflowing mass in W5 is that the CO we use
% to trace it is destroyed by UV photons from the O-stars in the region.

\subsection{Star Formation Activity}
\label{sec:sfactivity}
%We have divided the W5 region into distinct sub-regions that all show signs
%of star formation (Section \ref{sec:subregions}).  The AFGL 4029, LW Cas
%Nebula, IC 1848, W5SE, and southern pillar sub-regions are all along the edge
%of the diffuse HII regions that shape W5.  In these sub-regions, the
%radiation-driven ionization front pressure from the O-stars could have led to
%triggered collapse.  However, there are also outflows in Sh 201 and W5 NW,
%which are not adjacent to the HII region and do not show signs of external
%pressure such as illuminated cloud edges in the CO and Spitzer infrared maps.  

CO outflows are an excellent tracer of ongoing embedded star formation
\citep[e.g.][]{Shu1987}.  We use the locations of newly discovered outflows to
qualitatively describe the star formation activity within the W5 complex and
evaluate the hypothesis that star formation has been triggered on small or
intermediate scales.

Class 0/I objects are nearly always associated with outflows in nearby
star-forming regions \citep[e.g. Perseus][]{curtis2010,hatchell2007}.  However,
\citet{koenig:clustered:2008} detected 171 Class I sources in W5 using Spitzer
photometry.  Since our detection threshold for outflow appears to be similar to
that in Perseus (Section \ref{sec:percompare}), the lower number of outflow
detections is surprising, especially considering that some of the detected
outflows are outside the Spitzer-MIPS field (MIPS detections are required for
Class I objects, and flows 1-4 are outside that range) or are in the outer arm
(flows 39-54).  Additionally, we should detect outflows from Class 0
objects that would not be identified by Spitzer colors.

There are a number of explanations for our detection deficiency.  The Class I
objects detected within the HII region ``bubble'' most likely have outflows in
which the CO is dissociated similar to jet systems in Orion \citep[e.g.
HH46/47, a pc-scale flow in which CO is only visible very near the
protostar;][]{Chernin1991,Stanke1999}.  This hypothesis can be tested by
searching for optical and infrared jets associated with these objects, which
presumably have lower mass envelopes and therefore less extinction than
typical Class I objects.  Additionally, there are many outflow systems that are
are likely to be associated with clusters of outflows rather than individual
outflows as demonstrated in Section \ref{sec:percompare}, where we were able to identify
fewer outflows when `observing' the Perseus objects at a greater distance.  There
are 24 sources in the \citet{koenig:clustered:2008} Class I catalog within
15\arcsec\ (one JCMT beam at 345 GHz) of another, and in many cases there are
multiple \citet{koenig:clustered:2008} Class I sources within the contours of a
single outflow system.



\subsection{Evaluating Triggering}
In the previous section, we discussed in detail the relationship between each
sub-region and the HII region.  Some regions are observed to be star-forming
but not interacting with the HII region (W5NW, Sh 2-201), while others are
interacting with the HII region but show no evidence or reduced evidence of
star formation (W5SE, W5W, LW Cas).  At the very least, there is significant
complexity in the triggering mechanisms, and no one mechanism or size scale is
dominant.  If we were to trust outflows as unbiased tracers of star formation,
we might conclude that the majority of star formation in W5 is untriggered
(spontaneous), but such a conclusion is unreliable because both radiatively
triggered star formation and ``revealed'' star formation may not exhibit
molecular outflows (although ionized atomic outflows should still be visible
around young stars formed through these scenarios).

In Section \ref{sec:w5ridge}, we analyzed a particular case in which the RDI
mechanism could plausibly have crushed a cloud to create the observed
protostar.  
It is not possible to determine whether interaction with the HII region was a 
necessary precondition for the star's formation, but it at least accelerated
the process.  The other cometary clouds share this property,
but in total there are only 5 cometary clouds with detected outflows at their
tips, indicating that this mechanism is not the dominant driver of star
formation in W5.

The `collect and collapse' scenario might naively be expected to produce an
excess of young stars at the interaction front between the HII region and the
molecular cloud.  However, because such interactions naturally tend to form
instabilities, this scenario produces cloud morphologies indistinguishable from
those of RDI.  There is not an obvious excess of sources associated with cloud
edges over those deep within the clouds (e.g., Figure
\ref{fig:outflows_on_co32}).  We therefore cannot provide any direct evidence
for this triggering scenario.

The overall picture of W5 is of two concurrent episodes of massive-star
formation that have lead to adjacent blown-out bubbles.  Despite the added
external pressure along the central ridge, it is relatively deficient in both
star formation activity and dense gas, perhaps because of heating by the strong
ionizing radiation field.  The lack of star formation along that central ridge
implies that much of the gas was squeezed and heated, but it was not crushed
into gravitationally unstable fragments.  While some star formation may have
been triggered in W5, there is strong evidence for pre-existing star formation
being at least a comparable, if not the dominant, mechanism of star formation
in the complex.


\section{Outflow systems beyond W5}
Fifteen outflows were detected at velocities inconsistent with the local W5
cloud velocities.  Of these, 8 are consistent with Perseus arm velocities
($v_{LSR} > -55$ \kms) and could be associated with different clouds within the
same spiral arm.  The other 7 have central velocities $v_{LSR} < -55$ \kms\ and
are associated with the outer arm identified in previous surveys
\citep[e.g.][]{Digel1996}.  The properties of these outflows are given in Tables
\ref{tab:outeroutflows} and \ref{tab:outeroutflowsderived}; the distances listed are kinematic distances assuming
$R_0=8.4$ kpc and $v_0=254$~\kms\ \citep{Reid2009}.

Of these outflows, all but one are within 2\arcmin\ of an IRAS point source.
Outflow 54 is the most distant in our survey at a kinematic distance $d=7.5$
kpc ($v_{lsr}=-75.6$ \kms) and galactocentric distance $D_G = 14.7$ kpc.  It
has no known associations in the literature.

Outflows 41 - 44 are associated with a cloud at $v_{LSR}\sim -62$ \kms\ known
in the literature as LDN 1375 and associated with IRAS 02413+6037.  Outflows 53
and 55 are at a similar velocity and associated with IRAS 02598+6008 and IRAS
02425+6000 respectively.  All of these sources lie roughly on the periphery of
the W5 complex.

Outflows 45 - 52 are associated with a string of IRAS sources and HII regions
to the north of W5 and have velocities in the range $-55 < v_{LSR} < -45$.
They therefore could be in the Perseus arm but are clearly unassociated with
the W5 complex.  Outflows 45 and 46 are associated with IRAS 02435+6144 and
they may also be associated with the Sharpless HII region Sh 2-194.  Outflows
47 and 48 are associated with IRAS 02461+6147, also known as AFGL 5085.
Outflows 49 and 50 are nearby but not necessarily associated with IRAS
02475+6156, and may be associated with Sh 2-196.  Outflows 51 and 52 are
associated with IRAS 02541+6208.  

% \subsection{Comparison to other studies}
% 
% XXXX:  Use otherpapers compare.txt.  Discuss what the most powerful outflows in each region
% would look like at 2kpc
 
% \Table{ccccc}{Comparison to other outflow surveys}
% {{Paper} & {Region} & {Tracer} &  {Distance (pc)} & {Area (arcmin$^2$)}  & {Area (pc$^2$)} 
% & {Resolution (\arcsec)} & {Resolution (pc)} & {Outflows detected} \\ }
% {tab:surveys}
% {
% \citet{arce2010}                       & CO 1-0 & Perseus  &    250   &   57600     &   305   & 46 & 0.056  &    96 \\
% \citet{dionatos2010}                   & CO 3-2 & Serpens  &    310   &   29        &   0.24  & 14 & 0.021  &    20 \\
% \citet{davis:jcmt:2010}                & CO 3-2 & Taurus   &    140   &   2705      &   4.5   & 14 & 0.0095 &    16 \\
% \citet{hatchell2007} \tablenotemark{a} & CO 3-2 & Perseus  &    250   &   204       &   1.1   & 14 & 0.017  &    37 \\
% This paper                             & CO 3-2 & W5       &   2000   &   7200      &   2440  & 14 & 0.14   &    38 \\
% }{
% \tablenotetext{a}{This survey targeted 51 mm cores, hence its much higher detection rate per unit area.}
% }

\section{Conclusions}

We have identified \nwfive\ molecular outflow candidates in the W5 star forming
region and an additional \nouter\ outflows spatially coincident but located in
the outer arm of the Galaxy.  

\begin{itemize}
%    \item The majority of the millimeter sources are associated with outflows,
%      though some of the brightest millimeter sources lack outflows.  These
%      sources may consist of warmer dust that has not yet become Jeans
%      unstable.  Millimeter emission is a good tracer of active embedded star
%      formation, while far-infrared brightness is not.
%    \item The majority of the gas seen in \thirteenco\ and 1.1 mm emission is
%      associated with star formation.  There is therefore only a small amount
%      of
    \item The majority of the CO clouds in the W5 complex are forming stars.
      Star formation is not limited to cloud edges around the HII region.
      Because star formation activity is observed outside of the region of
      influence of the W5 O-stars, it is apparent that direct triggering by
      massive star feedback is not responsible for all of the star formation in
      W5.
    \item The W5 complex is seen nearly face-on as evidenced by a strict upper
      limit on the CO column through the center of the HII-region bubbles.  It
      is therefore an excellent region to study massive star feedback and
      revealed and triggered star formation.
    \item Outflows contribute negligibly to the turbulent energy of molecular
      clouds in the W5 complex.  This result is unsurprising near an HII
      region, but supports the idea that massive star forming regions are
      qualitatively different from low-mass star-forming regions in which the
      observed turbulence could be driven by outflow feedback.  Even in regions
      far separated from the O-stars, there is more turbulence and less energy
      injection from outflows than in, e.g., Perseus.
    \item Despite detecting a significant number of powerful outflows, the
      total outflowing mass detected in this survey ($\sim 1.5$ \msun\ without
      optical depth correction, perhaps $10-20$ \msun\ when optical depth is
      considered) was somewhat smaller than in Perseus, a low to intermediate
      mass star forming region with $\sim 1/6$ the molecular mass of W5. 
    \item The low mass measured may be partly because the CO 3-2 line is
      sub-thermally excited in outflows.  Therefore, while CO 3-2 is an
      excellent tracer of outflows for detection, it does not serve as a
      `calorimeter' in the same capacity as CO 1-0.
    \item Even considering excitation and optical depth corrections, it is
      likely that the mass of outflows in W5 is less than would be expected
      from a simple extrapolation from Perseus based on cloud mass. CO is
      likely to be photodissociated in the outflows when they reach the HII
      region, accounting for the deficiency around the HII region edges.
      However, in areas unaffected by the W5 O-stars such as W5NW, the
      deficiency may be because the greater turbulence in the W5 clouds
      suppresses star formation or hides outflows.
    \item Velocity gradients across the tails of many cometary clouds have been
      observed, hinting at their geometry and confirming that the outflows seen
      from their heads must be generated by protostars within.
      %The data presented contain tens of cometary clouds with
      %precise kinematic information about their molecular gas.
    \item Outflows have been detected in the Outer Arm at galactocentric
      distances $\gtrsim12$ kpc.  These represent some of the highest
      galactocentric distance star forming regions discovered to date.
    %\item  Feedback in W5 is primarily destroying rather than
    %  triggering SF.
\end{itemize}

\section{Acknowledgements}
We thank the two anonymous referees for their assistance in refining this
document.  We thank Devin Silvia for a careful proofread of the text. This work
has made use of the APLpy plotting package
(\url{http://aplpy.sourceforge.net}), the pyregion package
(\url{http://leejjoon.github.com/pyregion/}), the agpy code package
(\url{http://code.google.com/p/agpy/}) , IPAC's Montage
(\url{http://montage.ipac.caltech.edu/}), the DS9 visualization tool
(\url{http://hea-www.harvard.edu/RD/ds9/}), the pyspeckit spectrosopic analysis
toolkit (\url{http://pyspeckit.bitbucket.org}), and the STARLINK package
(\url{http://starlink.jach.hawaii.edu/}).  IRAS data was acquired through IRSA
at IPAC (\url{http://irsa.ipac.caltech.edu/}).  DRAO 21 cm data was acquired
from the Canadian Astronomical Data Center
(\url{http://cadcwww.hia.nrc.ca/cgps/}).  The authors are supported by the
National Science Foundation through NSF grant AST-0708403.  This research has
made use of the SIMBAD database, operated at CDS, Strasbourg, France

%{\it Facilities:} JCMT, VLA

%\bibliography{w5outflows}

%{\tiny
%\clearpage
%\onecolumn
\LongTable{ccccccccccc}{CO 3-2 Outflow Measured Properties}
{{Outflow} & {Latitude} & {Longitude} & {Ellipse} & {Ellipse} & {Ellipse} & {Velocity} & {Velocity} &  & {$\int T_A^* dv$} & {Bipolar?  \tablenotemark{a}}\\
{Number} &  &  & {Major} & {Minor} & {PA} & {center} & {min} & {max} &  & \\
 &  &  & {\arcsec} & {\arcsec} & {\degrees} & {(\kms)} & {(\kms)} & {(\kms)} & {(K \kms)} & \\}
{tab:outflows}
{
1b & 136.4437 & 1.2622 & 60 & 27 & 342 & -36.1 & -47.6 & -40.3 & 1.0 & yc\\
1r & 136.4674 & 1.2705 & 49 & 24 & 346 & -36.1 & -31.9 & -23.4 & 1.5 & yc\\
2b & 136.4899 & 1.1904 & 30 & 23 & 299 & -35.7 & -48.0 & -39.7 & 0.7 & yc\\
2r & 136.4743 & 1.2042 & 31 & 28 & 332 & -35.7 & -31.7 & -23.0 & 1.3 & yc\\
3 & 136.475 & 1.2548 & 35 & 25 & 332 & -31.8 & -31.8 & -26.8 & 1.3 & n\\
4b & 136.5038 & 1.2623 & 26 & 22 & 35 & -36.2 & -44.1 & -40.1 & 0.8 & yu\\
4r & 136.5109 & 1.2751 & 25 & 22 & 332 & -36.2 & -32.4 & -28.6 & 0.9 & yu\\
5r & 136.5126 & 1.2453 & 24 & 22 & 10 & -35.3 & -31.4 & -28.8 & 0.8 & yu\\
5b & 136.5236 & 1.2524 & 39 & 22 & 3 & -35.3 & -45.0 & -39.2 & 1.4 & yu\\
6b & 136.532 & 1.228 & 28 & 25 & 332 & -35.3 & -44.8 & -40.0 & 0.4 & yc\\
6r & 136.5327 & 1.2333 & 28 & 20 & 318 & -35.3 & -30.6 & -24.0 & 1.0 & yc\\
7b & 136.5453 & 1.2318 & 24 & 19 & 332 & -34.9 & -47.5 & -39.9 & 1.7 & yc\\
7r & 136.5506 & 1.2383 & 27 & 23 & 314 & -34.9 & -29.9 & -22.7 & 1.3 & yc\\
8b & 136.5799 & 1.2755 & 18 & 14 & 332 & -34.5 & -41.5 & -39.3 & 0.6 & yc\\
8r & 136.581 & 1.2601 & 34 & 30 & 332 & -34.5 & -29.6 & -23.9 & 1.4 & yc\\
9b & 136.67 & 1.2123 & 30 & 27 & 332 & -35.0 & -44.5 & -38.5 & 1.4 & yc\\
9r & 136.6766 & 1.2059 & 40 & 31 & 332 & -35.0 & -31.6 & -26.7 & 0.3 & yc\\
10b & 136.7172 & 0.7859 & 39 & 24 & 353 & -42.8 & -52.6 & -47.5 & 3.3 & yc\\
10r & 136.7271 & 0.7797 & 31 & 26 & 332 & -42.8 & -38.1 & -33.1 & 4.1 & yc\\
11b & 136.8195 & 1.082 & 25 & 24 & 331 & -34.2 & -40.7 & -37.0 & 3.1 & yc\\
11r & 136.8173 & 1.0799 & 24 & 22 & 331 & -34.2 & -31.4 & -20.4 & 1.5 & yc\\
12b & 136.8414 & 1.1512 & 30 & 26 & 332 & -40.4 & -53.3 & -46.2 & 1.5 & yc\\
12r & 136.8479 & 1.1517 & 27 & 25 & 332 & -40.4 & -34.6 & -30.1 & 0.9 & yc\\
13 & 136.8461 & 0.8426 & 28 & 27 & 332 & -31.0 & -31.0 & -23.5 & 1.0 & n\\
14 & 136.8591 & 1.176 & 24 & 23 & 332 & -47.1 & -54.5 & -47.1 & 0.8 & n\\
15 & 136.9443 & 1.0841 & 28 & 18 & 348 & -45.0 & -55.0 & -45.0 & 3.1 & n\\
16b & 137.3929 & 0.5977 & 23 & 18 & 333 & -40.7 & -47.0 & -42.6 & 0.7 & yu\\
16r & 137.3981 & 0.6121 & 22 & 19 & 357 & -40.7 & -38.7 & -35.2 & 1.9 & yu\\
17b & 137.4084 & 0.6762 & 20 & 18 & 293 & -40.3 & -57.9 & -43.0 & 2.3 & yc\\
17r & 137.412 & 0.6775 & 20 & 18 & 308 & -40.3 & -37.6 & -30.4 & 1.1 & yc\\
18b & 137.4925 & 0.6289 & 16 & 15 & 333 & -35.5 & -39.2 & -37.6 & 1.1 & yc\\
18r & 137.4908 & 0.6292 & 18 & 17 & 307 & -35.5 & -33.4 & -31.0 & 2.0 & yc\\
19b & 137.4815 & 0.6409 & 20 & 17 & 1 & -36.0 & -41.9 & -38.9 & 1.3 & yc\\
19r & 137.4798 & 0.6404 & 20 & 16 & 301 & -36.0 & -33.1 & -25.9 & 0.7 & yc\\
20r & 137.5368 & 1.2792 & 24 & 21 & 332 & -37.4 & -33.0 & -22.5 & 5.2 & yc\\
20b & 137.539 & 1.279 & 27 & 23 & 17 & -37.4 & -52.0 & -41.8 & 3.4 & yc\\
21b & 137.6152 & 1.3543 & 31 & 28 & 322 & -39.5 & -52.0 & -43.7 & 4.5 & yc\\
21r & 137.6169 & 1.3585 & 31 & 18 & 4 & -39.5 & -35.2 & -30.0 & 1.2 & yc\\
22 & 137.6213 & 1.506 & 27 & 21 & 293 & -40.3 & -46.0 & -40.3 & 2.1 & n\\
23b & 137.6389 & 1.5251 & 21 & 14 & 331 & -38.5 & -42.5 & -40.5 & 1.6 & yc\\
23r & 137.6449 & 1.5194 & 19 & 12 & 331 & -38.5 & -36.5 & -32.0 & 1.9 & yc\\
24r & 137.7094 & 1.4824 & 20 & 20 & 331 & -38.2 & -33.8 & -25.4 & 4.2 & yc\\
24b & 137.7146 & 1.4809 & 25 & 19 & 292 & -38.2 & -50.0 & -42.7 & 4.4 & yc\\
25b & 138.1398 & 1.6858 & 39 & 26 & 282 & -38.8 & -49.5 & -43.2 & 0.6 & yc\\
25r & 138.142 & 1.6884 & 43 & 35 & 11 & -38.8 & -34.3 & -27.5 & 1.7 & yc\\
26b & 138.2913 & 1.5538 & 29 & 29 & 355 & -38.7 & -52.0 & -47.4 & 1.2 & yc\\
26r & 138.2966 & 1.5564 & 28 & 28 & 330 & -38.7 & -30.0 & -20.0 & 4.2 & yc\\
27 & 138.3017 & 1.5689 & 26 & 25 & 330 & -30.0 & -30.0 & -22.0 & 1.8 & n\\
28 & 138.3042 & 1.5437 & 20 & 19 & 330 & -43.3 & -46.1 & -43.3 & 1.4 & n\\
29 & 138.3053 & 1.5537 & 22 & 20 & 330 & -45.3 & -51.6 & -45.3 & 2.5 & n\\
30 & 138.3115 & 1.5443 & 26 & 26 & 330 & -33.0 & -33.0 & -29.2 & 1.2 & n\\
31 & 138.3184 & 1.5566 & 26 & 25 & 330 & -44.4 & -49.1 & -44.4 & 1.1 & n\\
32 & 138.3213 & 1.5658 & 27 & 27 & 330 & -31.7 & -31.7 & -27.0 & 1.4 & n\\
33b & 138.3618 & 1.5073 & 28 & 26 & 330 & -39.4 & -49.5 & -44.0 & 1.3 & yc\\
33r & 138.3642 & 1.4959 & 29 & 21 & 330 & -39.4 & -34.7 & -25.8 & 2.0 & yc\\
34r & 138.4779 & 1.6137 & 22 & 21 & 330 & -36.9 & -33.1 & -29.1 & 0.5 & yc\\
34b & 138.4768 & 1.6142 & 21 & 20 & 330 & -36.9 & -43.6 & -40.6 & 0.8 & yc\\
35r & 138.4998 & 1.6496 & 22 & 20 & 4 & -37.5 & -31.3 & -24.1 & 1.4 & yc\\
35b & 138.5021 & 1.6458 & 23 & 21 & 330 & -37.5 & -49.5 & -43.6 & 1.3 & yc\\
36b & 138.5034 & 1.6654 & 35 & 26 & 5 & -37.5 & -50.4 & -42.3 & 1.2 & yc\\
36r & 138.5061 & 1.6576 & 22 & 21 & 330 & -37.5 & -32.6 & -26.7 & 1.4 & yc\\
37r & 138.5208 & 1.6618 & 27 & 22 & 330 & -38.5 & -33.5 & -31.4 & 0.6 & yc\\
37b & 138.5241 & 1.6667 & 23 & 23 & 18 & -38.5 & -47.0 & -43.6 & 0.6 & yc\\
38b & 137.4983 & 0.6062 & 16 & 15 & 333 & -36.1 & -39.2 & -38.5 & 0.8 & yc\\
38r & 137.4977 & 0.6055 & 15 & 14 & 307 & -36.1 & -33.7 & -32.5 & 0.5 & yc\\
39b & 138.1506 & 0.7724 & 23 & 16 & 321 & -38.8 & -45.3 & -41.0 & 2.0 & yc\\
39r & 138.1591 & 0.7713 & 17 & 13 & 304 & -38.8 & -36.6 & -34.7 & 0.7 & yc\\
40 & 138.1356 & 0.7634 & 22 & 18 & 4 & -36.0 & -36.0 & -27.6 & 2.2 & n\\
}{\multicolumn{11}{l}{Measured properties of the outflows.} \\ 
\multicolumn{11}{l}{$^a$ Is the outflow part of a bipolar pair?  yc = yes, confident; yu = yes, uncertain; n = no} \\ }
{}
{11}

\clearpage

\LongTable{cccccc}{CO 3-2 Outflow Derived Properties}
{{Outflow} & {Mass} & {Momentum} & {Energy} & {Dynamical} & {Momentum}\\
{Number} &  &  &  & {Age} & {Flux}\\
 & {(\msun)} & {(\msun \kms)} & {(10$^{42}$ ergs)} & {(10$^4$ years)} & {$10^{-6}$ \msun}\\
 &  &  &  &  & {\kms yr$^{-1}$}\\}
{tab:outflowsderived}
{
1b & 0.034 & 0.26 & 21.1 & 7.0 & 7.2\\
1r & 0.04 & 0.24 & 17.3 & 7.0 & 7.2\\
2b & 0.011 & 0.07 & 4.9 & 5.4 & 4.4\\
2r & 0.025 & 0.17 & 13.0 & 5.4 & 4.4\\
3 & 0.025 & 0.12 & 5.8 & - & -\\
4b & 0.01 & 0.06 & 4.0 & 7.2 & 1.5\\
4r & 0.011 & 0.04 & 1.8 & 7.2 & 1.5\\
5r & 0.01 & 0.04 & 2.0 & 4.5 & 4.0\\
5b & 0.025 & 0.14 & 8.0 & 4.5 & 4.0\\
6b & 0.007 & 0.04 & 3.0 & 1.7 & 8.1\\
6r & 0.013 & 0.09 & 6.8 & 1.7 & 8.1\\
7b & 0.017 & 0.13 & 10.5 & 2.4 & 10.9\\
7r & 0.018 & 0.13 & 9.7 & 2.4 & 10.9\\
8b & 0.003 & 0.02 & 0.9 & 4.9 & 4.5\\
8r & 0.032 & 0.2 & 13.2 & 4.9 & 4.5\\
9b & 0.025 & 0.13 & 7.2 & 3.9 & 4.2\\
9r & 0.009 & 0.04 & 1.8 & 3.9 & 4.2\\
10b & 0.068 & 0.41 & 25.7 & 3.9 & 22.0\\
10r & 0.074 & 0.45 & 28.0 & 3.9 & 22.0\\
11b & 0.042 & 0.17 & 7.2 & 0.7 & 35.3\\
11r & 0.017 & 0.09 & 5.9 & 0.7 & 35.3\\
12b & 0.026 & 0.14 & 8.7 & 1.8 & 15.2\\
12r & 0.014 & 0.13 & 12.7 & 1.8 & 15.2\\
13 & 0.016 & 0.1 & 5.8 & - & -\\
14 & 0.01 & 0.06 & 4.1 & - & -\\
15 & 0.036 & 0.24 & 17.3 & - & -\\
16b & 0.006 & 0.03 & 1.2 & 11.1 & 0.7\\
16r & 0.018 & 0.05 & 1.3 & 11.1 & 0.7\\
17b & 0.019 & 0.12 & 9.4 & 1.4 & 10.6\\
17r & 0.009 & 0.03 & 0.7 & 1.4 & 10.6\\
18b & 0.006 & 0.02 & 0.5 & 1.4 & 4.0\\
18r & 0.013 & 0.04 & 1.0 & 1.4 & 4.0\\
19b & 0.011 & 0.05 & 2.4 & 0.7 & 9.6\\
19r & 0.005 & 0.01 & 0.2 & 0.7 & 9.6\\
20r & 0.059 & 0.5 & 46.3 & 0.5 & 156.0\\
20b & 0.047 & 0.33 & 26.6 & 0.5 & 156.0\\
21b & 0.086 & 0.58 & 41.4 & 1.7 & 39.1\\
21r & 0.014 & 0.08 & 4.3 & 1.7 & 39.1\\
22 & 0.027 & 0.1 & 4.3 & - & -\\
23b & 0.011 & 0.03 & 0.9 & 4.5 & 1.3\\
23r & 0.01 & 0.03 & 1.0 & 4.5 & 1.3\\
24r & 0.037 & 0.3 & 26.1 & 1.7 & 34.1\\
24b & 0.047 & 0.28 & 18.3 & 1.7 & 34.1\\
25b & 0.014 & 0.09 & 6.8 & 1.0 & 42.8\\
25r & 0.056 & 0.35 & 23.0 & 1.0 & 42.8\\
26b & 0.023 & 0.24 & 26.1 & 1.1 & 98.3\\
26r & 0.072 & 0.85 & 106.0 & 1.1 & 98.3\\
27 & 0.026 & 0.09 & 4.5 & - & -\\
28 & 0.012 & 0.07 & 4.6 & - & -\\
29 & 0.024 & 0.06 & 2.1 & - & -\\
30 & 0.018 & 0.12 & 8.0 & - & -\\
31 & 0.016 & 0.03 & 0.7 & - & -\\
32 & 0.023 & 0.18 & 14.5 & - & -\\
33b & 0.022 & 0.14 & 10.1 & 3.0 & 11.4\\
33r & 0.026 & 0.2 & 16.1 & 3.0 & 11.4\\
34r & 0.005 & 0.03 & 1.7 & 0.7 & 8.4\\
34b & 0.007 & 0.03 & 1.2 & 0.7 & 8.4\\
35r & 0.013 & 0.12 & 11.6 & 1.3 & 18.7\\
35b & 0.014 & 0.12 & 11.0 & 1.3 & 18.7\\
36b & 0.025 & 0.19 & 15.8 & 2.7 & 10.7\\
36r & 0.014 & 0.1 & 6.8 & 2.7 & 10.7\\
37r & 0.008 & 0.04 & 1.6 & 2.1 & 4.3\\
37b & 0.007 & 0.06 & 4.2 & 2.1 & 4.3\\
38b & 0.005 & 0.01 & 0.4 & 0.9 & 2.3\\
38r & 0.002 & 0.01 & 0.1 & 0.9 & 2.3\\
39b & 0.017 & 0.07 & 2.8 & 7.5 & 1.0\\
39r & 0.004 & 0.01 & 0.3 & 7.5 & 1.0\\
40 & 0.019 & 0.08 & 3.5 & - & -\\
}{\multicolumn{6}{l}{Derived properties of the outflows in
the optically thin limit.} \\  
\multicolumn{6}{l}{Typical optical depth corrections for \twelveco 3-2 are
in the range 7-14 \citep{curtis2010}.} \\  
\multicolumn{6}{l}{The correction for velocity confusion is $\gtrsim2$ but
poorly constrained \citep{arce2010}.} \\  
\multicolumn{6}{l}{Finally, an excitation correction in the range 1-20 is
likely required as described in the Appendix.} \\
\multicolumn{6}{l}{The mass and momentum values can be multiplied by these
factors to acquire the corrected values.} \\  
\multicolumn{6}{l}{The energy is weighted more heavily towards high-velocity,
low-optical-depth gas, so the correction factor is likely to be lower.}}
{} {6}
 % tab:outflows
%\Table{ccccccc}{Totals of outflow properties}
{{BGPS source} & {1.1mm mass} & {Intensity} & {Outflow Column} & {Outflow Mass} & {Momentum} & {Energy}\\
 & {\msun} & {(K \kms)} & {(\persc)} & {(\msun)} & {(\msun \kms)} & {(ergs)}\\}
{tab:outflowsums}
{
G136.456+01.268 & 19.81 & 1.657 & 5.45\ee{18} & 0.0507 & 0.357 & 2.76\ee{43}\\
G136.474+01.268 & 15.06 & 0.832 & 2.74\ee{18} & 0.0162 & 0.115 & 8.9\ee{42}\\
G136.500+01.258 & 38.16 & 2.166 & 7.11\ee{18} & 0.03802 & 0.2055 & 1.188\ee{43}\\
G136.512+01.194 & 85.65 & 1.512 & 4.97\ee{18} & 0.02772 & 0.1536 & 9.44\ee{42}\\
G136.536+01.232 & 88.86 & 2.744 & 9.03\ee{18} & 0.04309 & 0.3245 & 2.575\ee{43}\\
G136.671+01.210 & 71.31 & 2.667 & 8.78\ee{18} & 0.0589 & 0.3405 & 2.088\ee{43}\\
G136.719+00.782 & 81.3 & 4.28 & 1.407\ee{19} & 0.0661 & 0.36 & 2.142\ee{43}\\
G136.828+01.064 & 224.1 & 2.51 & 8.25\ee{18} & 0.0361 & 0.222 & 1.569\ee{43}\\
G136.842+00.838 & 21.84 & 0.813 & 2.67\ee{18} & 0.0138 & 0.0851 & 5.67\ee{42}\\
G136.846+01.168 & 67.96 & 2.05 & 6.74\ee{18} & 0.0377 & 0.265 & 2.09\ee{43}\\
G136.849+01.150 & 133.2 & 1.34 & 4.4\ee{18} & 0.01831 & 0.1516 & 1.322\ee{43}\\
G136.948+01.092 & 202.8 & 1.45 & 4.76\ee{18} & 0.0218 & 0.139 & 9.71\ee{42}\\
G137.394+00.610 & 26.68 & 5.38 & 1.768\ee{19} & 0.0998 & 0.783 & 6.77\ee{43}\\
G137.409+00.674 & 41.22 & 1.06 & 3.48\ee{18} & 0.0205 & 0.112 & 6.45\ee{42}\\
G137.479+00.640 & 107.5 & 1.839 & 6.047\ee{18} & 0.02244 & 0.1463 & 1.1647\ee{43}\\
G137.538+01.278 & 87.68 & 0.545 & 1.793\ee{18} & 0.008201 & 0.04483 & 2.473\ee{42}\\
G137.617+01.350 & 128.2 & 1.358 & 4.466\ee{18} & 0.01642 & 0.0656 & 2.989\ee{42}\\
G137.632+01.508 & 48.13 & 2.29 & 7.52\ee{18} & 0.025 & 0.143 & 9.06\ee{42}\\
G137.665+01.526 & 65.13 & 3.16 & 1.04\ee{19} & 0.03374 & 0.2716 & 2.321\ee{43}\\
G137.707+01.490 & 71.2 & 1.763 & 5.78\ee{18} & 0.0481 & 0.364 & 3.04\ee{43}\\
G138.144+01.684 & 201.7 & 3.63 & 1.193\ee{19} & 0.0599 & 0.704 & 8.64\ee{43}\\
G138.295+01.556 & 824.4 & 6.957 & 2.289\ee{19} & 0.09247 & 0.6825 & 5.3999\ee{43}\\
G138.502+01.646 & 361.7 & 4.899 & 1.613\ee{19} & 0.07083 & 1.4379 & 3.47758\ee{44}\\
}{Totals of outflow mass, momentum, and energy.  }
 % tab:outflowsums

%\clearpage
%\Table{cccccccccccc}{Outer Arm CO 3-2 Outflows - Measured Properties}
{{Outflow} & {Latitude} & {Longitude} & {Ellipse} & {Ellipse} & {Ellipse} & {Kinematic} & {$R_G$\tablenotemark{a}} & {Velocity} & {Velocity} &  & {$\int T_A^* dv$}\\
{Number} &  &  & {Major} & {Minor} & {PA} & {Distance} &  & {center} & {min} & {max} & \\
 &  &  & {\arcsec} & {\arcsec} & {\degrees} & {(pc)} & {(pc)} & {(\kms)} &  &  & {(K \kms)}\\}
{tab:outeroutflows}
{
41r & 136.364 & 0.9606 & 25 & 18 & 2 & 5510 & 13000 & -61.8 & -59.2 & -56.5 & 0.5\\
41b & 136.3634 & 0.9568 & 23 & 17 & 353 & 5510 & 13000 & -61.8 & -71.6 & -64.3 & 3.0\\
42r & 136.3522 & 0.9786 & 20 & 14 & 2 & 5500 & 12900 & -62.1 & -59.8 & -57.6 & 0.6\\
42b & 136.3548 & 0.9798 & 20 & 19 & 332 & 5500 & 12900 & -62.1 & -67.8 & -64.4 & 0.5\\
43r & 136.3495 & 0.9612 & 17 & 15 & 63 & 5510 & 13000 & -61.8 & -59.0 & -56.1 & 0.8\\
43b & 136.353 & 0.9621 & 12 & 12 & 333 & 5510 & 13000 & -61.8 & -66.3 & -64.6 & 1.0\\
44r & 136.3554 & 0.9576 & 13 & 13 & 23 & 5500 & 12900 & -61.8 & -59.0 & -55.4 & 2.1\\
44b & 136.3545 & 0.9567 & 14 & 14 & 333 & 5500 & 12900 & -61.8 & -68.0 & -64.5 & 2.0\\
45r & 136.1219 & 2.0816 & 34 & 25 & 297 & 3750 & 11400 & -46.5 & -43.1 & -40.5 & 0.6\\
45b & 136.1233 & 2.0803 & 35 & 25 & 306 & 3750 & 11400 & -46.5 & -57.3 & -50.0 & 1.9\\
46 & 136.1166 & 2.0983 & 26 & 25 & 332 & 3790 & 11400 & -50.2 & -52.6 & -50.2 & 0.5\\
47b & 136.3857 & 2.2687 & 34 & 27 & 332 & 3220 & 11000 & -42.0 & -55.0 & -46.7 & 3.5\\
47r & 136.3861 & 2.267 & 35 & 23 & 304 & 3220 & 11000 & -42.0 & -37.3 & -25.1 & 5.0\\
48b & 136.374 & 2.2628 & 29 & 21 & 332 & 3250 & 11000 & -43.2 & -51.4 & -47.0 & 1.5\\
48r & 136.3736 & 2.2615 & 29 & 22 & 332 & 3250 & 11000 & -43.2 & -39.5 & -22.2 & 8.9\\
49r & 136.4663 & 2.4678 & 29 & 23 & 290 & 3610 & 11300 & -45.7 & -42.2 & -33.0 & 2.2\\
49b & 136.4661 & 2.4693 & 31 & 23 & 292 & 3610 & 11300 & -45.7 & -52.2 & -49.1 & 0.9\\
50b & 136.5087 & 2.5108 & 31 & 25 & 332 & 3380 & 11100 & -43.5 & -48.5 & -46.5 & 0.8\\
50r & 136.5118 & 2.5083 & 28 & 23 & 10 & 3380 & 11100 & -43.5 & -40.6 & -37.5 & 1.0\\
51b & 137.058 & 2.9858 & 28 & 23 & 293 & 4350 & 11900 & -51.8 & -55.5 & -53.0 & 0.8\\
51r & 137.0567 & 2.9864 & 34 & 25 & 8 & 4350 & 11900 & -51.8 & -50.6 & -40.9 & 3.5\\
52r & 137.0662 & 2.9999 & 37 & 26 & 43 & 4390 & 12000 & -52.2 & -49.1 & -41.0 & 7.8\\
52b & 137.0683 & 3.0013 & 38 & 29 & 15 & 4390 & 12000 & -52.2 & -65.8 & -55.2 & 4.1\\
53b & 138.6143 & 1.5611 & 26 & 26 & 330 & 5450 & 13000 & -59.7 & -71.1 & -61.7 & 5.5\\
53r & 138.6158 & 1.563 & 25 & 23 & 330 & 5450 & 13000 & -59.7 & -57.6 & -54.5 & 1.3\\
54r & 136.382 & 0.8392 & 29 & 20 & 343 & 7480 & 14700 & -75.6 & -73.2 & -68.9 & 2.6\\
54b & 136.3824 & 0.838 & 20 & 17 & 332 & 7480 & 14700 & -75.6 & -83.1 & -77.9 & 2.0\\
55b & 136.7623 & 0.4548 & 27 & 16 & 343 & 5230 & 12700 & -60.9 & -65.2 & -62.7 & 5.4\\
55r & 136.7579 & 0.4522 & 24 & 18 & 343 & 5230 & 12700 & -60.9 & -59.0 & -53.2 & 6.0\\
}{\tablenotetext{a}{Galactocentric Radius}}

\Table{cccccc}{Outer Arm CO 3-2 Outflows - Derived Properties}
{{Outflow} & {Mass} & {Momentum} & {Energy} & {Dynamical} & {Momentum}\\
{Number} &  &  &  & {Age} & {Flux}\\
 & {(\msun)} & {(\msun \kms)} & {(10$^{42}$ ergs)} & {(10$^4$ years)} & {$10^{-6}$ \msun \kms yr$^{-1}$}\\}
{tab:outeroutflowsderived}
{
41r & 0.037 & 0.11 & 3.5 & 3.6 & 30.2\\
41b & 0.196 & 0.96 & 54.6 & 3.6 & 30.2\\
42r & 0.029 & 0.06 & 1.3 & 4.3 & 3.9\\
42b & 0.03 & 0.11 & 3.9 & 4.3 & 3.9\\
43r & 0.033 & 0.13 & 5.1 & 7.3 & 2.9\\
43b & 0.024 & 0.08 & 2.8 & 7.3 & 2.9\\
44r & 0.062 & 0.21 & 7.9 & 1.8 & 27.7\\
44b & 0.067 & 0.28 & 12.6 & 1.8 & 27.7\\
45r & 0.037 & 0.18 & 8.6 & 1.2 & 62.3\\
45b & 0.126 & 0.55 & 28.3 & 1.2 & 62.3\\
46 & 0.028 & 0.1 & 3.7 & - & -\\
47b & 0.187 & 1.32 & 101.0 & 0.6 & 553.0\\
47r & 0.232 & 1.84 & 164.0 & 0.6 & 553.0\\
48b & 0.054 & 0.33 & 20.8 & 0.4 & 754.0\\
48r & 0.341 & 2.4 & 229.0 & 0.4 & 754.0\\
49r & 0.106 & 0.77 & 62.7 & 1.4 & 71.1\\
49b & 0.047 & 0.22 & 10.8 & 1.4 & 71.1\\
50b & 0.037 & 0.14 & 5.5 & 3.8 & 7.4\\
50r & 0.038 & 0.14 & 5.5 & 3.8 & 7.4\\
51b & 0.058 & 0.13 & 3.0 & 1.0 & 152.0\\
51r & 0.303 & 1.33 & 72.7 & 1.0 & 152.0\\
52r & 0.829 & 4.3 & 250.0 & 1.4 & 479.0\\
52b & 0.472 & 2.5 & 150.0 & 1.4 & 479.0\\
53b & 0.626 & 3.16 & 194.0 & 4.7 & 73.2\\
53r & 0.124 & 0.31 & 8.5 & 4.7 & 73.2\\
54r & 0.461 & 1.24 & 37.5 & 1.8 & 135.0\\
54b & 0.212 & 1.14 & 64.0 & 1.8 & 135.0\\
55b & 0.411 & 1.66 & 68.0 & 7.5 & 28.3\\
55r & 0.404 & 0.47 & 6.6 & 7.5 & 28.3\\
}

%\include{individual_outflows_preface}

%\include{individual_outflows}
% \appendix
% \onecolumn
\section{W5 Appendix: Optically Thin, LTE dipole molecule}
\label{appendix:dipole}
While many authors have solved the problem of converting CO 1-0 beam
temperatures to \hh\ column densities
\citep{garden1991,bourke1997,Cabrit1990,lada1996}, there are no  examples in
the literature of a full derivation of the LTE, optically thin CO-to-\hh\
conversion process for higher rotational states.  We present the full
derivation here, and quantify the systematic errors generated by various
assumptions.

We begin with the assumption of an optically thin cloud such that the radiative
transfer equation \citep[][eqn 1.9]{rohlfs} simplifies to
\begin{equation}
  \label{eqn:radtrans}
  \frac{dI_\nu}{\ds} = -\kappa_\nu I_\nu 
\end{equation}

The absorption and stimulated emission terms yield 
\begin{equation}
  \label{eqn:kappa}
  \kappa_\nu = \frac{h \nu_{ul} B_{ul} n_u}{c} \varphi(\nu)
              -\frac{h \nu_{ul} B_{lu} n_l}{c} \varphi(\nu)
\end{equation}
where $\varphi(\nu)$ is the line shape function ($\int\varphi(\nu) \dnu \equiv
1$), $n$ is the density in the given state, $\nu$ is the frequency of the transition,
$B$ is the Einstein B coefficient, and $h$ is Planck's constant.

By assuming LTE (the Boltzmann distribution) and using Kirchoff's Law and the definition of 
the Einstein A and B values, we can derive a more useful version of this equation
\begin{equation}
  \kappa_\nu = \frac{c^2}{8 \pi \nu_{ul}^2} n_u A_{ul} \left[\exp\left(\frac{h \nu_{ul} }{k_B T_{ex}}\right) - 1 \right] \varphi(\nu)
\end{equation}
where $k_B$ is Boltzmann's constant.

The observable $T_B$ can be related to the optical depth, which is given by 
\begin{equation}
  \int \tau_\nu \dnu = \frac{c^2}{8 \pi \nu_{ul}^2} A_{ul} \left[\exp\left(\frac{h \nu_{ul} }{k_B T_{ex}}\right) - 1 \right] \int \varphi(\nu) \dnu \int n_u \ds 
\end{equation}

Rearranging and converting from density to column ($\int n \ds = N$) gives an equation for the column density
of the molecule in the upper energy state of the transition:
\begin{equation}
  \label{eqn:nuppertau}
  N_u = \frac{8\pi \nu_{ul}^2}{c^2 A_{ul}} \left[\exp\left(\frac{h \nu_{ul} }{k_B T_{ex}}\right) - 1 \right]^{-1} \int \tau_\nu \dnu
\end{equation}

In order to relate the brightness temperature to the optical depth, at CO transition frequencies the full blackbody
formula must be used and the CMB must also be taken into account.  \citet{rohlfs} equation 15.29 
\begin{equation} 
  \label{eqn:tbrightnesscmb}
  T_B(\nu) = \frac{h \nu}{k_B} \left(\left[e^{h \nu / k_B T_{ex}} - 1\right]^{-1} - \left[e^{h \nu / k_B T_{CMB}} - 1\right]^{-1} \right) (1-e^{-\tau_\nu})
\end{equation}
is rearranged to solve for $\tau_\nu$:
\begin{equation}
  \label{eqn:tau}
  \tau_\nu = -\ln\left[ 1 - \frac{k_B T_B}{h \nu} \left(\left[e^{h \nu / k_B T_{ex}} - 1\right]^{-1} - \left[e^{h \nu / k_B T_{CMB}} - 1\right]^{-1} \right)^{-1} \right]
\end{equation}

We convert from frequency to velocity units with $\dnu = \nu/c \dv$, and plug \eqref{eqn:tau} into \eqref{eqn:nuppertau} to get
\begin{equation}
  \label{eqn:nuppernoapprox}
  N_u = \frac{8\pi \nu_{ul}^3}{c^3 A_{ul}} \left[\exp\left(\frac{h \nu_{ul} }{k_B T_{ex}}\right) - 1 \right]^{-1} \int -\ln\left[ 1 - \frac{k_B T_B}{h \nu_{ul}} \left(\left[e^{h \nu_{ul} / k_B T_{ex}} - 1\right]^{-1} - \left[e^{h \nu_{ul} / k_B T_{CMB}} - 1\right]^{-1} \right)^{-1} \right] \dv
\end{equation}
which is the full LTE upper-level column density with no approximations applied.

The first term of the Taylor expansion is appropriate for $\tau<<1$ ($\ln[1+x]\approx x-\frac{x^2}{2}+\frac{x^3}{3}\ldots$)
\begin{equation}
  N_u = \frac{8\pi \nu_{ul}^3}{c^3 A_{ul}} \left[\exp\left(\frac{h \nu_{ul} }{k_B T_{ex}}\right) - 1 \right]^{-1} \int \frac{k_B T_B}{h \nu_{ul}} \left(\left[e^{h \nu_{ul} / k_B T_{ex}} - 1\right]^{-1} - \left[e^{h \nu_{ul} / k_B T_{CMB}} - 1\right]^{-1} \right)^{-1} \dv
\end{equation}
which simplifies to
\begin{equation}
  \label{eqn:nupper}
  N_u = \frac{8\pi \nu_{ul}^2 k_B}{c^3 A_{ul} h }  \frac{e^{h\nu_{ul}/k_B T_{CMB}} - 1}{e^{h\nu_{ul}/k_B T_{CMB}} - e^{h\nu_{ul}/k_B T_{ex}}} \int T_B  \dv
\end{equation}

This can be converted to use $\mu_e$ \citep[0.1222 for
\twelveco; ][]{Muenter1975}, the electric dipole moment of the molecule, instead
of $A_{ul}$, using \citet{rohlfs} equation 15.20 $\left((A_{ul}=(64\pi^4)/(3 h
c^3)\right)\nu^3 \mu_{e}^2$):
\begin{equation}
  \label{eqn:nuppermuju}
  N_u = \frac{3  }{8 \pi^3 \mu_e^2 } \frac{k_B}{\nu_{ul}} \frac{2 J_u + 1}{J_u} 
    \frac{e^{h\nu_{ul}/k_B T_{cmb}} - 1}{e^{h\nu_{ul}/k_B T_{CMB}} - e^{h\nu_{ul}/k_B T_{ex}}} \int T_B  \dv
\end{equation}

The total column can be derived from the column in the upper state using the partition
function and the Boltzmann distribution
\begin{equation}
  n_{tot}  =        \sum_{J=0}^\infty n_J = n_0 \sum_{J=0}^\infty  (2J+1) \exp\left(-\frac{J(J+1) B_e h}{k_B T_{ex}}\right) \label{eqn:approxpartition}\\
\end{equation}
This equation is frequently approximated using an integral
\citep[e.g.][]{Cabrit1990}, but a more accurate numerical solution using up to
thousands of rotational states is easily computed
\begin{equation}
  n_J = \left[ \sum_{j=0}^{j=j_{max}} (2j+1) \exp\left(-\frac{j(j+1) B_e h}{k_B T_{ex}}\right) \right]^{-1} (2J+1) \exp\left(-\frac{J(J+1) B_e h}{k_B T_{ex}}\right)
\end{equation}
The effects of using the approximation and the full numerical solution are shown in figure \ref{fig:approx}.

%We note that there are a number ($>1$) of different values of $\mu_e$ frequently reported in the literature.
%\citet{Burrus1958} reports a Stark-effect measurement of $\mu_e = 0.112\pm0.005$ Debye.  \citet{Muenter1975}
%report an improvement on this measurement, yielding $\mu_e = 0.1222$.  More recently, \citet{Goorvitch1994}
%report a value for the rotationless dipole momenut $\mu_0 = 0.1101$, which is negligibly different from the 
%\citet{Muenter1975} value...

\Figure{figures_chw5/columnconversion_vs_tex_allapprox}
{The LTE, optically thin conversion factor from $T_B$ (K \kms) to N(\hh)
(\persc) assuming X$_{\twelveco}=10^{-4}$ plotted against $T_{ex}$.  The
dashed line shows the effect of using the integral approximation of the 
partition function \citep[e.g.][]{Cabrit1990}.  It is a better
approximation away from the critical point, and is a better approximation
for higher transitions.  The dotted line shows the effects of removing the 
CMB term from \eqref{eqn:tbrightnesscmb}; the CMB populates the lowest two
excited states, but contributes nearly nothing to the $J=3$ state. Top (blue):
J=1-0, Middle (green): J=2-1, Bottom (red): J=3-2.}
{fig:approx}{1.0}{0}


The CO 3-2 transition is also less likely to be in LTE than the 1-0 transition.
The critical density ($n_{cr}\equiv A_{ul}/C_{ul}$) of \twelveco\ 3-2 is 27
times higher than that for 1-0.  We have run RADEX \citep{VanDerTak2007} LVG
models of CO to examine the impact of sub-thermal excitation on column
derivation.  The results of the RADEX models are shown in Figure
\ref{fig:coradex}.  They illustrate that, while it is quite safe to assume the
CO 1-0 transition is in LTE in most circumstances, a similar assumption is
probably invalid for the CO 3-2 transition in typical molecular cloud
environments.

\Figure{figures_chw5/CO_excitation}
{{\it Top}: The derived N(\hh) as a function of $n_{\hh}$ for $T_{B}=1$ K.
The dashed lines represent the LTE-derived $N(\hh)/T_B$ factor, which has 
no density dependence and, for CO 3-2, only a weak dependence on temperature.
We assume an abundance of \twelveco\ relative to \hh\ $X_{CO} = 10^{-4}$.
{\it Bottom}: The correction factor (N(\hh)$_{RADEX}$ / N(\hh)$_{LTE}$) as
a function of $n_{\hh}$.
For $T_K=20$ K, the ``correction factor'' at $10^3$ \percc\ (typical GMC
mean volume densities) is $\sim15$, while at $10^4$ \percc\ (closer to $n_{crit}$ but
perhaps substantially higher than GMC densities) it becomes negligible.  The
correction factor is also systematically lower for a higher gas kinetic
temperature.
For some densities, the ``correction factor'' dips below 1, particularly for CO
1-0.  This effect is from a slight population inversion due to fast spontaneous
decay rates from the higher levels and has been noted before
\citep[e.g.][]{Goldsmith1972}.
}{fig:coradex}{1.0}{0}

%\bibliography{column_derivation}



%\end{document}
\ifstandalone
\bibliographystyle{apj_w_etal}  % or "siam", or "alpha", or "abbrv"
%\bibliography{thesis}      % bib database file refs.bib
\bibliography{bibdesk}      % bib database file refs.bib
\fi

\end{document}

% %\documentclass[defaultstyle,11pt]{thesis}
%\documentclass[]{report}
%\documentclass[]{article}
%\usepackage{aastex_hack}
%\usepackage{deluxetable}
\documentclass[preprint]{aastex}


%%%%%%%%%%%%%%%%%%%%%%%%%%%%%%%%%%%%%%%%%%%%%%%%%%%%%%%%%%%%%%%%
%%%%%%%%%%%  see documentation for information about  %%%%%%%%%%
%%%%%%%%%%%  the options (11pt, defaultstyle, etc.)   %%%%%%%%%%
%%%%%%%  http://www.colorado.edu/its/docs/latex/thesis/  %%%%%%%
%%%%%%%%%%%%%%%%%%%%%%%%%%%%%%%%%%%%%%%%%%%%%%%%%%%%%%%%%%%%%%%%
%		\documentclass[typewriterstyle]{thesis}
% 		\documentclass[modernstyle]{thesis}
% 		\documentclass[modernstyle,11pt]{thesis}
%	 	\documentclass[modernstyle,12pt]{thesis}

%%%%%%%%%%%%%%%%%%%%%%%%%%%%%%%%%%%%%%%%%%%%%%%%%%%%%%%%%%%%%%%%
%%%%%%%%%%%    load any packages which are needed    %%%%%%%%%%%
%%%%%%%%%%%%%%%%%%%%%%%%%%%%%%%%%%%%%%%%%%%%%%%%%%%%%%%%%%%%%%%%
\usepackage{latexsym}		% to get LASY symbols
\usepackage{graphicx}		% to insert PostScript figures
%\usepackage{deluxetable}
\usepackage{rotating}		% for sideways tables/figures
\usepackage{natbib}  % Requires natbib.sty, available from http://ads.harvard.edu/pubs/bibtex/astronat/
\usepackage{savesym}
\usepackage{amssymb}
%\savesymbol{singlespace}
\savesymbol{doublespace}
%\usepackage{wrapfig}
%\usepackage{setspace}
\usepackage{xspace}
\usepackage{color}
\usepackage{multicol}
\usepackage{mdframed}
\usepackage{url}
\usepackage{subfigure}
%\usepackage{emulateapj}
\usepackage{lscape}
\usepackage{grffile}
\usepackage{standalone}
\standalonetrue
\usepackage{import}
\usepackage[utf8]{inputenc}
\usepackage{longtable}
\usepackage{booktabs}



%%%%%%%%%%%%%%%%%%%%%%%%%%%%%%%%%%%%%%%%%%%%%%%%%%%%%%%%%%%%%%%%
%%%%%%%%%%%%       all the preamble material:       %%%%%%%%%%%%
%%%%%%%%%%%%%%%%%%%%%%%%%%%%%%%%%%%%%%%%%%%%%%%%%%%%%%%%%%%%%%%%

% \title{Star Formation in the Galaxy}
% 
% \author{Adam G.}{Ginsburg}
% 
% \otherdegrees{B.S., Rice University, 2007\\
% 	      M.S., University of Colorado, Boulder, 2009}
% 
% \degree{Doctor of Philosophy}		%  #1 {long descr.}
% 	{Ph.D., Rocket Science (ok, fine, astrophysics)}		%  #2 {short descr.}
% 
% \dept{Department of}			%  #1 {designation}
% 	{Astrophysical and Planetary Sciences}		%  #2 {name}
% 
% \advisor{Prof.}				%  #1 {title}
% 	{John Bally}			%  #2 {name}
% 
% \reader{Prof.~Jeremy Darling}		%  2nd person to sign thesis
% \readerThree{Prof.~Jason Glenn}		%  3rd person to sign thesis
% \readerFour{Prof.~Michael Shull}	%  4rd person to sign thesis
% \readerFour{Prof.~Neal Evans}	%  4rd person to sign thesis
% 
% \abstract{  \OnePageChapter	% one page only ??
% 
%     I discovered dust in space.  
% 
% 	}
% 
% 
% \dedication[Dedication]{	% NEVER use \OnePageChapter here.
% 	To 1, the second number in binary.
% 	}
% 
% \acknowledgements{	\OnePageChapter	% *MUST* BE ONLY ONE PAGE!
% 	All y'all.
% 	}
% 
% \ToCisShort	% a 1-page Table of Contents ??
% 
% \LoFisShort	% a 1-page List of Figures ??
% %	\emptyLoF	% no List of Figures at all ??
% 
% \LoTisShort	% a 1-page List of Tables ??
% %	\emptyLoT	% no List of Tables at all ??
% 
% 
% %%%%%%%%%%%%%%%%%%%%%%%%%%%%%%%%%%%%%%%%%%%%%%%%%%%%%%%%%%%%%%%%%
% %%%%%%%%%%%%%%%       BEGIN DOCUMENT...         %%%%%%%%%%%%%%%%%
% %%%%%%%%%%%%%%%%%%%%%%%%%%%%%%%%%%%%%%%%%%%%%%%%%%%%%%%%%%%%%%%%%
% 
% %%%%  footnote style; default=\arabic  (numbered 1,2,3...)
% %%%%  others:  \roman, \Roman, \alph, \Alph, \fnsymbol
% %	"\fnsymbol" uses asterisk, dagger, double-dagger, etc.
% %	\renewcommand{\thefootnote}{\fnsymbol{footnote}}
% %	\setcounter{footnote}{0}

\input{macros}		% file containing author's macro definitions

\begin{document}
% \input{introduction}
% 
% %\input{ch_iras05358}
% \input{ch_w5}
% \input{ch_h2co}
% \input{ch_h2colarge}
% \input{ch_boundhii}
% 
% %\input ch2.tex			% file with Chapter 2 contents
% 
% %%%%%%%%%%%%%%%%%%%%%%%%%%%%%%%%%%%%%%%%%%%%%%%%%%%%%%%%%%%%%%%%%%%
% %%%%%%%%%%%%%%%%%%%%%%%  Bibliography %%%%%%%%%%%%%%%%%%%%%%%%%%%%%
% %%%%%%%%%%%%%%%%%%%%%%%%%%%%%%%%%%%%%%%%%%%%%%%%%%%%%%%%%%%%%%%%%%%
% 
% \bibliographystyle{plain}	% or "siam", or "alpha", or "abbrv"
% 				% see other styles (.bst files) in
% 				% $TEXHOME/texmf/bibtex/bst
% 
% \nocite{*}		% list all refs in database, cited or not.
% 
% \bibliography{thesis}		% bib database file refs.bib
% 
% %%%%%%%%%%%%%%%%%%%%%%%%%%%%%%%%%%%%%%%%%%%%%%%%%%%%%%%%%%%%%%%%%%%
% %%%%%%%%%%%%%%%%%%%%%%%%  Appendices %%%%%%%%%%%%%%%%%%%%%%%%%%%%%%
% %%%%%%%%%%%%%%%%%%%%%%%%%%%%%%%%%%%%%%%%%%%%%%%%%%%%%%%%%%%%%%%%%%%
% 
% \appendix	% don't forget this line if you have appendices!
% 
% %\input appA.tex			% file with Appendix A contents
% %\input appB.tex			% file with Appendix B contents
% 
% %%%%%%%%%%%%%%%%%%%%%%%%%%%%%%%%%%%%%%%%%%%%%%%%%%%%%%%%%%%%%%%%%%%
% %%%%%%%%%%%%%%%%%%%%%%%%   THE END   %%%%%%%%%%%%%%%%%%%%%%%%%%%%%%
% %%%%%%%%%%%%%%%%%%%%%%%%%%%%%%%%%%%%%%%%%%%%%%%%%%%%%%%%%%%%%%%%%%%
% 
% \end{document}
% 
% 

%%\documentclass[defaultstyle,11pt]{thesis}
%\documentclass[]{report}
%\documentclass[]{article}
%\usepackage{aastex_hack}
%\usepackage{deluxetable}
\documentclass[preprint]{aastex}


%%%%%%%%%%%%%%%%%%%%%%%%%%%%%%%%%%%%%%%%%%%%%%%%%%%%%%%%%%%%%%%%
%%%%%%%%%%%  see documentation for information about  %%%%%%%%%%
%%%%%%%%%%%  the options (11pt, defaultstyle, etc.)   %%%%%%%%%%
%%%%%%%  http://www.colorado.edu/its/docs/latex/thesis/  %%%%%%%
%%%%%%%%%%%%%%%%%%%%%%%%%%%%%%%%%%%%%%%%%%%%%%%%%%%%%%%%%%%%%%%%
%		\documentclass[typewriterstyle]{thesis}
% 		\documentclass[modernstyle]{thesis}
% 		\documentclass[modernstyle,11pt]{thesis}
%	 	\documentclass[modernstyle,12pt]{thesis}

%%%%%%%%%%%%%%%%%%%%%%%%%%%%%%%%%%%%%%%%%%%%%%%%%%%%%%%%%%%%%%%%
%%%%%%%%%%%    load any packages which are needed    %%%%%%%%%%%
%%%%%%%%%%%%%%%%%%%%%%%%%%%%%%%%%%%%%%%%%%%%%%%%%%%%%%%%%%%%%%%%
\usepackage{latexsym}		% to get LASY symbols
\usepackage{graphicx}		% to insert PostScript figures
%\usepackage{deluxetable}
\usepackage{rotating}		% for sideways tables/figures
\usepackage{natbib}  % Requires natbib.sty, available from http://ads.harvard.edu/pubs/bibtex/astronat/
\usepackage{savesym}
\usepackage{amssymb}
%\savesymbol{singlespace}
\savesymbol{doublespace}
%\usepackage{wrapfig}
%\usepackage{setspace}
\usepackage{xspace}
\usepackage{color}
\usepackage{multicol}
\usepackage{mdframed}
\usepackage{url}
\usepackage{subfigure}
%\usepackage{emulateapj}
\usepackage{lscape}
\usepackage{grffile}
\usepackage{standalone}
\standalonetrue
\usepackage{import}
\usepackage[utf8]{inputenc}
\usepackage{longtable}
\usepackage{booktabs}



%%%%%%%%%%%%%%%%%%%%%%%%%%%%%%%%%%%%%%%%%%%%%%%%%%%%%%%%%%%%%%%%
%%%%%%%%%%%%       all the preamble material:       %%%%%%%%%%%%
%%%%%%%%%%%%%%%%%%%%%%%%%%%%%%%%%%%%%%%%%%%%%%%%%%%%%%%%%%%%%%%%

% \title{Star Formation in the Galaxy}
% 
% \author{Adam G.}{Ginsburg}
% 
% \otherdegrees{B.S., Rice University, 2007\\
% 	      M.S., University of Colorado, Boulder, 2009}
% 
% \degree{Doctor of Philosophy}		%  #1 {long descr.}
% 	{Ph.D., Rocket Science (ok, fine, astrophysics)}		%  #2 {short descr.}
% 
% \dept{Department of}			%  #1 {designation}
% 	{Astrophysical and Planetary Sciences}		%  #2 {name}
% 
% \advisor{Prof.}				%  #1 {title}
% 	{John Bally}			%  #2 {name}
% 
% \reader{Prof.~Jeremy Darling}		%  2nd person to sign thesis
% \readerThree{Prof.~Jason Glenn}		%  3rd person to sign thesis
% \readerFour{Prof.~Michael Shull}	%  4rd person to sign thesis
% \readerFour{Prof.~Neal Evans}	%  4rd person to sign thesis
% 
% \abstract{  \OnePageChapter	% one page only ??
% 
%     I discovered dust in space.  
% 
% 	}
% 
% 
% \dedication[Dedication]{	% NEVER use \OnePageChapter here.
% 	To 1, the second number in binary.
% 	}
% 
% \acknowledgements{	\OnePageChapter	% *MUST* BE ONLY ONE PAGE!
% 	All y'all.
% 	}
% 
% \ToCisShort	% a 1-page Table of Contents ??
% 
% \LoFisShort	% a 1-page List of Figures ??
% %	\emptyLoF	% no List of Figures at all ??
% 
% \LoTisShort	% a 1-page List of Tables ??
% %	\emptyLoT	% no List of Tables at all ??
% 
% 
% %%%%%%%%%%%%%%%%%%%%%%%%%%%%%%%%%%%%%%%%%%%%%%%%%%%%%%%%%%%%%%%%%
% %%%%%%%%%%%%%%%       BEGIN DOCUMENT...         %%%%%%%%%%%%%%%%%
% %%%%%%%%%%%%%%%%%%%%%%%%%%%%%%%%%%%%%%%%%%%%%%%%%%%%%%%%%%%%%%%%%
% 
% %%%%  footnote style; default=\arabic  (numbered 1,2,3...)
% %%%%  others:  \roman, \Roman, \alph, \Alph, \fnsymbol
% %	"\fnsymbol" uses asterisk, dagger, double-dagger, etc.
% %	\renewcommand{\thefootnote}{\fnsymbol{footnote}}
% %	\setcounter{footnote}{0}

\input{macros}		% file containing author's macro definitions

\begin{document}
% \input{introduction}
% 
% %\input{ch_iras05358}
% \input{ch_w5}
% \input{ch_h2co}
% \input{ch_h2colarge}
% \input{ch_boundhii}
% 
% %\input ch2.tex			% file with Chapter 2 contents
% 
% %%%%%%%%%%%%%%%%%%%%%%%%%%%%%%%%%%%%%%%%%%%%%%%%%%%%%%%%%%%%%%%%%%%
% %%%%%%%%%%%%%%%%%%%%%%%  Bibliography %%%%%%%%%%%%%%%%%%%%%%%%%%%%%
% %%%%%%%%%%%%%%%%%%%%%%%%%%%%%%%%%%%%%%%%%%%%%%%%%%%%%%%%%%%%%%%%%%%
% 
% \bibliographystyle{plain}	% or "siam", or "alpha", or "abbrv"
% 				% see other styles (.bst files) in
% 				% $TEXHOME/texmf/bibtex/bst
% 
% \nocite{*}		% list all refs in database, cited or not.
% 
% \bibliography{thesis}		% bib database file refs.bib
% 
% %%%%%%%%%%%%%%%%%%%%%%%%%%%%%%%%%%%%%%%%%%%%%%%%%%%%%%%%%%%%%%%%%%%
% %%%%%%%%%%%%%%%%%%%%%%%%  Appendices %%%%%%%%%%%%%%%%%%%%%%%%%%%%%%
% %%%%%%%%%%%%%%%%%%%%%%%%%%%%%%%%%%%%%%%%%%%%%%%%%%%%%%%%%%%%%%%%%%%
% 
% \appendix	% don't forget this line if you have appendices!
% 
% %\input appA.tex			% file with Appendix A contents
% %\input appB.tex			% file with Appendix B contents
% 
% %%%%%%%%%%%%%%%%%%%%%%%%%%%%%%%%%%%%%%%%%%%%%%%%%%%%%%%%%%%%%%%%%%%
% %%%%%%%%%%%%%%%%%%%%%%%%   THE END   %%%%%%%%%%%%%%%%%%%%%%%%%%%%%%
% %%%%%%%%%%%%%%%%%%%%%%%%%%%%%%%%%%%%%%%%%%%%%%%%%%%%%%%%%%%%%%%%%%%
% 
% \end{document}
% 
% 

%\bibliographystyle{apj_w_etal}

\chapter{\formaldehyde observations of BGPS sources previously observed with Arecibo}

\section{Introduction}
%Despite intense study, the process of forming massive stars from Giant
%Molecular Clouds (GMCs) is still poorly understood.  

% Millimeter continuum
% studies have begun to reveal the condensations in molecular clouds that will
% likely form into clusters of stars \citep[e.g.][]{Aguirre2011} and many
% high-resolution studies have revealed disks around massive stars
% \citep{Davies2010,Kraus2010,Zapata2009}.  However, the clumps revealed by
% millimeter continuum studies are much larger and often more massive than single
% stars and are therefore not clearly the direct progenitors of stars.  Some of the
% disks around massive stars are massive enough that they may become
% gravitationally unstable and fragment instead of accreting onto the central
% star \citep[e.g.][]{Keto2010,Peters2010}.  The driving question in massive star
% formation still remains open: how do massive stars acquire their mass?

Massive stars are known to form preferentially in clustered environments
\citep{DeWit2005}.  They therefore likely form from ``clumps,'' collections of
gas and dust more dense and compact than Giant Molecular Clouds (GMCs) but larger and
more diffuse than typical low-mass protostellar cores.  ``Clumps'' have been
observed with masses ranging from $10-10^6$\msun\ (but more typically
$10^2-10^3\msun$) and with beam-averaged densities in the range $10^3 \lesssim n(\hh) \lesssim
10^5$ \percc\ and sizes $\sim1$ pc \citep[e.g., ][]{Rosolowsky2010,Dunham2010}.
While giant molecular clouds  in the Galaxy have been surveyed
\citep[e.g., ][]{Jackson2006}, the process by which these clouds condense into
clumps and cores and the mechanisms by which they are dispersed are not
understood. 

It is still not known what sets the final mass of massive stars, but it
is thought that they must ignite while still accreting
\citep{mckee2007}.  Hot O and B stars emitting strongly in the ultraviolet will
ionize their surroundings, creating density-bounded \ion{H}{2} regions.  They
progress from hypercompact through ultracompact and compact and finally diffuse
\ion{H}{2} region phases, during which they either dissociate or blow out their
surrounding medium \citep{Churchwell2002,keto2007}.  The brightest sources in the
Galactic plane in both the free-free continuum in the cm-wavelength regime and
the dust continum in the sub-mm to mm-wavelength regime generally host \uchii\
regions.

%All massive star formation and most low-mass star formation is expected to occur
%in GMCs \citep{McKee and Ostriker?}.  \citet{Tauber1990} and others have viewed
%GMC structure in terms of clump distributions.... mean density in OMC n~150...
%filling factor 0.0002 at 15 pc, 0.02 at 0.15 pc

% Density gradients observed towards UCHIIs with RRLs (Keto, Zhang, Kurtz 2007)

%XXXX maybe move this into a discussion section; too specific
While the gas within \uchii\ regions is hot and ionized, the surrounding gas is
initially molecular.  At the interface between the molecular cloud and
the ionization front, a photon-dominated or photodissociation region appears
\citep{Roshi2005}.  \citet{Churchwell2010} observed HCO$^+$ towards a
sample of \uchii\ regions and noted both infall and outflow motions in
molecular tracers towards these objects.  It should be possible to determine
whether the \uchii\ regions still have collapsing envelopes (infall signatures)
or only disks (outflow signatures) and thereby determine relative evolutionary
states of the regions.

Two centimeter transitions of formaldehyde, \ortho\ \oneone\ (6 cm) and
\twotwo\ (2 cm)\footnote{All references to \formaldehyde\ in this paper,
except where otherwise noted, are to the ortho \ortho\ population, as no para p-\formaldehyde\ 
lines were observed}, have been used to measure the density of molecular
clouds in massive-star-forming regions \citep[e.g., ][]{Dickel1986,Dickel1987},
high-latitude Galactic clouds \citep[e.g., ][]{Turner1989}, the Galactic Center
\citep[e.g., ][]{Zylka1992} starburst galaxies \citep[e.g., ][]{Mangum2008}, and molecular clouds
in a gravitational lens \citep[e.g., ][]{Zeiger2010}.  Studies similar to our own have
been performed by \citet{Wadiak1988} and \citet{Henkel1983}, in which bright
continuum sources were observed in the same transitions with
(approximately) beam-matched telescopes at $\sim$2\arcmin\ resolution. Our
study delves deeper into the spectral line profiles and systematic uncertainties of
\formaldehyde\ densitometry and is performed at higher spatial resolution than
past work.
%These transitions have been accessible to radio
%telescopes for decades, but they have only been accessible at sub-arcminute
%resolution (with single dishes) since the C-band receiver was installed at
%Arecibo.  
% (only 6 cm) pre-stellar cores \citep{Young2010},

%A sample of 21 Ultracompact \ion{H}{2} (UC\ion{H}{2}) regions were observed in
%the \formaldehyde\ \oneone\ transition with the Arecibo Observatory
%\footnote{The Arecibo Observatory is part of the National Astronomy and Ionosphere
%Center, which is operated by Cornell University under a cooperative agreement
%with the National Science Foundation.}
%by \citet{Araya2002}.  An additional 15 observations of \formaldehyde\ \oneone\
%were observed towards suspected massive star forming regions in
%\citet{Araya2004}.  We present follow-up observation of all 21 of the \uchii\
%regions in \citet{Araya2002} and three of the suspected massive star forming
%regions from \citet{Araya2004} in the \twotwo\ transition with the Green Bank
%Telescope (GBT) \footnote{ The National Radio Astronomy Observatory operates
%the GBT and VLA and is a facility of the National Science Foundation operated
%under cooperative agreement by Associated Universities, Inc.  }.  The pair
%of \formaldehyde\ lines are used together to measure the gas volume density
%towards these targets.

This paper presents a pilot study as a proof-of-concept for a much larger
ongoing survey\footnote{GBT project code GBT10B-019} towards 400 lines of sight
and the methodology applicable to the larger survey.

In section \ref{sec:h2coobservations} we present the new observations and describe
other data sets used in our analysis.  Section \ref{sec:models} describes the
modeling procedure used to derive density from the \formaldehyde\ line
observations.  Section \ref{sec:analysis} presents detailed discussion of the
modeling and derivation of physical parameters and their uncertainties.
Section \ref{sec:results} describes the derived and measured values. 
%Section \ref{sec:indiv} presents a detailed discussion for each individual source.  
Section \ref{sec:discussion} discusses the larger implications of our results.
We conclude with a brief summary of important results.

\section{Observations and Data}
\label{sec:h2coobservations}
\subsection{Source Selection} 
The observed lines-of-sight included 21 sources selected from the \citet{Araya2002} \uchii\
sample and 3 from the \citet{Araya2004} ``massive-star forming candidate''
sample.  The sources were selected primarily on the basis of having been
previously observed with Arecibo\footnote{The Arecibo Observatory is part of
the National Astronomy and Ionosphere Center, which is operated by Cornell
University under a cooperative agreement with the National Science Foundation.
} in the \oneone\ transition of \formaldehyde\
with the intent of demonstrating the densitometry method within the Galaxy
rather than making systematic observations of any source class.  Nonetheless,
the \citet{Araya2002} sample includes the majority of the bright \uchii\
regions accessible to Arecibo.  Additionally, there are many detected GMCs
along the line of sight to these \uchii\ regions.

The \citet{Araya2004} observations included 15 pointings towards infrared dark cloud
(IRDC) candidates and High-Mass Protostellar Object (HMPO) candidates.  The
sources we selected from this sample include two sources classified as IRDCs
based on MSX data and one HMPO candidate.  The selection of these sources was
arbitrary; we were only able to observe 24 lines-of-sight in our 4 hour
observation block.  The remaining sources will be discussed in a later paper.
The observed lines of sight are listed in Table \ref{tab:h2comeasured_a}.

% included in original... \LongTables
% maybe use longtables if needed? http://stackoverflow.com/questions/3685832/long-tables-in-latex
% {tab:h2comeasured_a}
\Table{lcccccccc}{Measured \formaldehyde\ \oneone\ line properties}
{\colhead{Source Name\tablenotemark{a}}&\colhead{$l$}&\colhead{$b$}&\colhead{6cm Continuum}&\colhead{Peak}&\colhead{Center}&\colhead{FWHM }&\colhead{RMS}&\colhead{Channel Width}\\
\colhead{           }&\colhead{\degrees}&\colhead{\degrees}&\colhead{(Jy)}&\colhead{(Jy)}&\colhead{(\kms)}&\colhead{(\kms)}&\colhead{(Jy)}&\colhead{(\kms)}\\ }
{tab:h2comeasured_a}{
       G32.80+0.19 0&              0.1904&             32.7968&         2.18 (0.01)&      -0.393 (0.008)&        15.39 (0.05)&         6.57 (0.06)&              0.0049&              1.1374\\
       G32.80+0.19 1&              0.1904&             32.7968&         2.18 (0.01)&      -0.092 (0.008)&        11.45 (0.26)&        10.25 (0.65)&              0.0049&              1.1374\\
       G32.80+0.19 2&              0.1904&             32.7968&         2.18 (0.01)&      -0.063 (0.008)&        80.63 (0.13)&         2.49 (0.36)&              0.0049&              1.1374\\
       G32.80+0.19 3&              0.1904&             32.7968&         2.18 (0.01)&      -0.254 (0.008)&        84.61 (0.02)&         1.37 (0.06)&              0.0049&              1.1374\\
       G32.80+0.19 4&              0.1904&             32.7968&         2.18 (0.01)&      -0.090 (0.008)&        88.66 (0.09)&         3.21 (0.31)&              0.0049&              1.1374\\
       G33.13-0.09 0&             -0.0949&             33.1297&         0.49 (0.00)&      -0.192 (0.007)&        75.92 (0.05)&         3.80 (0.12)&              0.0045&              1.1374\\
       G33.13-0.09 1&             -0.0949&             33.1297&         0.49 (0.00)&      -0.023 (0.007)&        81.62 (0.35)&         2.49 (0.88)&              0.0045&              1.1374\\
       G33.13-0.09 2&             -0.0949&             33.1297&         0.49 (0.00)&      -0.040 (0.007)&       101.50 (0.40)&        11.30 (0.80)&              0.0045&              1.1374\\
       G33.13-0.09 3&             -0.0949&             33.1297&         0.49 (0.00)&      -0.039 (0.007)&        10.39 (0.08)&         2.04 (0.24)&              0.0045&              1.1374\\
       G33.92+0.11 0&              0.1112&              33.914&         0.83 (0.00)&      -0.081 (0.008)&       107.28 (0.18)&         6.62 (0.34)&               0.005&              1.1374\\
       G33.92+0.11 1&              0.1112&              33.914&         0.83 (0.00)&      -0.079 (0.008)&       106.03 (0.06)&         2.41 (0.23)&               0.005&              1.1374\\
       G33.92+0.11 2&              0.1112&              33.914&         0.83 (0.00)&      -0.160 (0.030)&        57.30 (0.40)&        10.60 (0.80)&               0.005&              1.1374\\
       G34.26+0.15 0&              0.1538&             34.2572&         5.57 (0.01)&      -1.828 (0.015)&        60.24 (0.01)&         3.80 (0.03)&              0.0063&              1.1374\\
       G34.26+0.15 1&              0.1538&             34.2572&         5.57 (0.01)&      -0.160 (0.015)&        26.69 (0.08)&         1.04 (0.22)&              0.0063&              1.1374\\
       G34.26+0.15 2&              0.1538&             34.2572&         5.57 (0.01)&      -0.099 (0.015)&        11.25 (0.19)&         2.01 (0.40)&              0.0063&              1.1374\\
       G34.26+0.15 3&              0.1538&             34.2572&         5.57 (0.01)&      -0.126 (0.015)&        51.70 (2.00)&         4.20 (1.00)&              0.0063&              1.1374\\
       G34.26+0.15 4&              0.1538&             34.2572&         5.57 (0.01)&      -0.047 (0.015)&        48.20 (2.00)&         1.80 (1.00)&              0.0063&              1.1374\\
       G35.20-1.74 0&             -1.7409&             35.1997&         5.17 (0.00)&      -1.018 (0.008)&        43.37 (0.01)&         3.67 (0.02)&              0.0051&              1.1374\\
       G35.20-1.74 1&             -1.7409&             35.1997&         5.17 (0.00)&      -0.147 (0.008)&        36.67 (0.10)&         1.49 (0.27)&              0.0051&              1.1374\\
       G35.20-1.74 2&             -1.7409&             35.1997&         5.17 (0.00)&      -0.324 (0.008)&        14.08 (0.01)&         0.93 (0.03)&              0.0051&              1.1374\\
       G35.20-1.74 3&             -1.7409&             35.1997&         5.17 (0.00)&      -0.039 (0.008)&        50.59 (0.53)&         4.92 (1.31)&              0.0051&              1.1374\\
       G35.57-0.03 0&             -0.0306&             35.5779&         0.47 (0.00)&      -0.064 (0.009)&        52.10 (0.10)&         4.60 (0.30)&              0.0053&              1.1374\\
       G35.57-0.03 1&             -0.0306&             35.5779&         0.47 (0.00)&      -0.021 (0.009)&        45.60 (0.30)&         1.90 (0.60)&              0.0053&              1.1374\\
       G35.57-0.03 2&             -0.0306&             35.5779&         0.47 (0.00)&      -0.019 (0.009)&        57.60 (0.50)&         2.90 (0.97)&              0.0053&              1.1374\\
       G35.57-0.03 3&             -0.0306&             35.5779&         0.47 (0.00)&      -0.031 (0.009)&        12.80 (0.20)&         1.84 (0.41)&              0.0053&              1.1374\\
       G35.57-0.03 4&             -0.0306&             35.5779&         0.47 (0.00)&      -0.031 (0.008)&        29.04 (0.11)&         0.82 (0.25)&              0.0053&              1.1374\\
       G35.58+0.07 0&              0.0657&             35.5801&         0.53 (0.01)&      -0.146 (0.004)&        49.37 (0.21)&         5.33 (0.34)&              0.0048&              1.1374\\
       G35.58+0.07 1&              0.0657&             35.5801&         0.53 (0.01)&      -0.049 (0.013)&        53.13 (0.25)&         2.98 (0.64)&              0.0048&              1.1374\\
       G35.58+0.07 2&              0.0657&             35.5801&         0.53 (0.01)&      -0.025 (0.004)&        58.12 (0.29)&         3.63 (0.74)&              0.0048&              1.1374\\
       G35.58+0.07 3&              0.0657&             35.5801&         0.53 (0.01)&      -0.034 (0.004)&        13.24 (0.17)&         2.80 (0.39)&              0.0048&              1.1374\\
       G37.87-0.40 0&             -0.3993&              37.873&         4.40 (0.01)&      -0.531 (0.006)&        60.23 (0.11)&         8.73 (0.35)&              0.0069&              1.1374\\
       G37.87-0.40 1&             -0.3993&              37.873&         4.40 (0.01)&      -0.124 (0.014)&        53.27 (0.19)&         4.03 (0.46)&              0.0069&              1.1374\\
       G37.87-0.40 2&             -0.3993&              37.873&         4.40 (0.01)&      -0.356 (0.019)&        65.13 (0.04)&         2.74 (0.15)&              0.0069&              1.1374\\
       G37.87-0.40 3&             -0.3993&              37.873&         4.40 (0.01)&      -0.324 (0.045)&        72.18 (0.04)&         1.35 (0.14)&              0.0069&              1.1374\\
       G37.87-0.40 4&             -0.3993&              37.873&         4.40 (0.01)&      -0.424 (0.013)&        73.97 (0.13)&         3.01 (0.22)&              0.0069&              1.1374\\
       G37.87-0.40 5&             -0.3993&              37.873&         4.40 (0.01)&      -0.185 (0.012)&        79.98 (0.06)&         1.80 (0.14)&              0.0069&              1.1374\\
       G37.87-0.40 6&             -0.3993&              37.873&         4.40 (0.01)&      -0.114 (0.015)&        91.96 (0.08)&         1.21 (0.18)&              0.0069&              1.1374\\
       G37.87-0.40 7&             -0.3993&              37.873&         4.40 (0.01)&      -0.175 (0.012)&        14.32 (0.14)&         2.94 (0.20)&              0.0069&              1.1374\\
       G37.87-0.40 8&             -0.3993&              37.873&         4.40 (0.01)&      -0.072 (0.022)&        13.16 (0.10)&         0.87 (0.32)&              0.0069&              1.1374\\
       G37.87-0.40 9&             -0.3993&              37.873&         4.40 (0.01)&      -0.137 (0.012)&        20.54 (0.06)&         1.37 (0.14)&              0.0069&              1.1374\\
       G43.89-0.78 0&             -0.7838&             43.8892&         0.66 (0.00)&      -0.181 (0.004)&        54.86 (0.02)&         2.19 (0.06)&              0.0032&              1.1374\\
       G43.89-0.78 1&             -0.7838&             43.8892&         0.66 (0.00)&      -0.020 (0.002)&        50.55 (0.59)&        15.90 (1.20)&              0.0032&              1.1374\\
       G45.07+0.13 0&              0.1323&             45.0711&         0.47 (0.00)&      -0.056 (0.006)&        57.49 (0.10)&         4.24 (0.23)&              0.0035&              1.1374\\
       G45.07+0.13 1&              0.1323&             45.0711&         0.47 (0.00)&      -0.036 (0.006)&        65.44 (0.15)&         4.09 (0.34)&              0.0035&              1.1374\\
       G45.12+0.13 0&              0.1326&             45.1223&         4.28 (0.01)&      -0.188 (0.006)&        55.70 (0.12)&         3.32 (0.24)&              0.0065&              1.1374\\
       G45.12+0.13 1&              0.1326&             45.1223&         4.28 (0.01)&      -0.154 (0.009)&        59.40 (0.13)&         3.11 (0.33)&              0.0065&              1.1374\\
       G45.12+0.13 2&              0.1326&             45.1223&         4.28 (0.01)&      -0.200 (0.010)&        24.86 (0.03)&         1.68 (0.08)&              0.0065&              1.1374\\
       G45.12+0.13 3&              0.1326&             45.1223&         4.28 (0.01)&      -0.027 (0.004)&        65.53 (0.82)&         7.23 (2.03)&              0.0065&              1.1374\\
       G45.45+0.06 0&              0.0593&             45.4548&         4.77 (0.01)&      -1.347 (0.018)&        59.58 (0.02)&         3.18 (0.05)&              0.0063&              1.1374\\
       G45.45+0.06 1&              0.0593&             45.4548&         4.77 (0.01)&      -0.123 (0.040)&        55.34 (0.38)&         3.15 (0.38)&              0.0063&              1.1374\\
       G45.45+0.06 2&              0.0593&             45.4548&         4.77 (0.01)&      -0.056 (0.005)&        25.02 (0.12)&         2.82 (0.28)&              0.0063&              1.1374\\
       G45.47+0.05 0&              0.0455&             45.4655&         0.75 (0.00)&      -0.274 (0.003)&        60.62 (0.03)&         6.59 (0.07)&              0.0039&              1.1374\\
       G45.47+0.05 1&              0.0455&             45.4655&         0.75 (0.00)&      -0.017 (0.004)&        25.55 (0.23)&         2.18 (0.55)&              0.0039&              1.1374\\
       G48.61+0.02 0&              0.0229&             48.6055&         1.01 (0.00)&      -0.067 (0.003)&        18.08 (0.09)&         4.97 (0.22)&              0.0035&              1.1374\\
       G48.61+0.02 1&              0.0229&             48.6055&         1.01 (0.00)&      -0.024 (0.005)&         6.08 (0.13)&         1.20 (0.31)&              0.0035&              1.1374\\
       G48.61+0.02 2&              0.0229&             48.6055&         1.01 (0.00)&      -0.018 (0.003)&        53.73 (0.33)&         4.72 (0.79)&              0.0035&              1.1374\\
       G50.32+0.68 0&              0.6761&             50.3153&         0.24 (0.00)&      -0.011 (0.003)&        26.28 (0.40)&         3.32 (0.94)&              0.0031&              1.1374\\
       G60.88-0.13 0&             -0.1285&             60.8826&         0.66 (0.01)&      -0.093 (0.009)&        22.60 (0.15)&         3.24 (0.35)&              0.0096&              1.1374\\
       G61.48+0.09 0&              0.0893&             61.4769&         6.16 (0.01)&      -0.531 (0.009)&        21.45 (0.02)&         2.81 (0.06)&              0.0084&              1.1374\\
       G69.54-0.98 0&             -0.9759&             69.5398&         0.28 (0.01)&      -0.280 (0.006)&        10.65 (0.05)&         4.55 (0.11)&              0.0076&              1.1374\\
       G70.29+1.60 0&              1.6006&             70.2927&         4.37 (0.13)&      -0.372 (0.008)&       -21.74 (0.07)&         3.92 (0.15)&              0.0108&              1.1374\\
       G70.29+1.60 1&              1.6006&             70.2927&         4.37 (0.13)&      -0.050 (0.007)&       -27.17 (0.58)&         4.86 (1.33)&              0.0108&              1.1374\\
       G70.33+1.59 0&               1.589&             70.3296&         2.21 (0.01)&      -1.201 (0.007)&       -21.24 (0.01)&         3.65 (0.03)&              0.0115&              1.1374\\
   IRAS 20051+3435 0&              0.2088&             32.4662&         0.00 (0.01)&      -0.019 (0.001)&        10.77 (0.07)&         3.60 (0.18)&             0.00071&              2.2747\\
       G41.74+0.10 0&              0.0975&             41.7415&         0.34 (0.00)&      -0.062 (0.004)&        14.60 (0.09)&         2.56 (0.26)&              0.0033&              1.1374\\
       G41.74+0.10 1&              0.0975&             41.7415&         0.34 (0.00)&      -0.020 (0.004)&        10.99 (0.29)&         2.52 (0.71)&              0.0033&              1.1374\\
       G41.74+0.10 2&              0.0975&             41.7415&         0.34 (0.00)&      -0.066 (0.004)&        34.25 (0.05)&         1.63 (0.13)&              0.0033&              1.1374\\
       G41.74+0.10 3&              0.0975&             41.7415&         0.34 (0.00)&      -0.022 (0.005)&        56.61 (0.13)&         1.15 (0.32)&              0.0033&              1.1374\\
       G41.74+0.10 4&              0.0975&             41.7415&         0.34 (0.00)&      -0.043 (0.005)&        17.57 (0.07)&         1.13 (0.18)&              0.0033&              1.1374\\
      IRDC 1923+13 0&             -0.4972&             48.9325&         0.40 (0.00)&      -0.011 (0.001)&        50.20 (0.08)&         1.83 (0.19)&              0.0008&              0.7582\\
      IRDC 1923+13 1&             -0.4972&             48.9325&         0.40 (0.00)&      -0.009 (0.001)&        57.56 (0.09)&         2.57 (0.22)&              0.0008&              0.7582\\
      IRDC 1923+13 2&             -0.4972&             48.9325&         0.40 (0.00)&      -0.005 (0.001)&        47.32 (0.20)&         2.11 (0.51)&              0.0008&              0.7582\\
      IRDC 1916+11 0&             -0.2923&              45.666&         0.00 (0.01)&      -0.005 (0.001)&        25.94 (0.17)&         2.53 (0.41)&             0.00083&              0.7582\\
      IRDC 1916+11 1&             -0.2923&              45.666&         0.00 (0.01)&      -0.013 (0.001)&        55.91 (0.13)&         6.21 (0.34)&             0.00083&              0.7582\\
      IRDC 1916+11 2&             -0.2923&              45.666&         0.00 (0.01)&      -0.003 (0.001)&        48.85 (0.48)&         3.58 (1.13)&             0.00083&              0.7582\\
}{
\tablenotetext{a}{Sources are labeled by the line-of-sight followed by the
number of the component identified, indexed from zero.  The components do not
follow a particular order, but are uniquely identifiable by their velocity,
width, and amplitude.}}

\Table{lccccc}{Measured \formaldehyde\ \twotwo\ line properties}
{\colhead{Source Name}&\colhead{2cm Continuum}&\colhead{Peak\tablenotemark{a}}&\colhead{Center}&\colhead{FWHM }&\colhead{RMS\tablenotemark{b}}\\
\colhead{           }&\colhead{(Jy)}&\colhead{(Jy)}&\colhead{(\kms)}&\colhead{(\kms)}&\colhead{(Jy)}\\ }
{tab:h2comeasured_b}{
       G32.80+0.19 0&         3.68 (0.02)&      -0.519 (0.032)&        15.65 (0.03)&         5.72 (0.08)&              0.0038\\
       G32.80+0.19 1&         3.68 (0.02)&      -0.076 (0.019)&        11.90 (1.18)&         8.17 (0.98)&              0.0038\\
       G32.80+0.19 2&         3.68 (0.02)&      -0.016 (0.001)&        80.47 (0.14)&         4.35 (0.36)&              0.0038\\
       G32.80+0.19 3&         3.68 (0.02)&      -0.065 (0.002)&        84.96 (0.02)&         1.29 (0.05)&              0.0038\\
       G32.80+0.19 4&         3.68 (0.02)&      -0.026 (0.001)&        88.83 (0.06)&         2.31 (0.14)&              0.0038\\
       G33.13-0.09 0&         0.47 (0.02)&      -0.224 (0.003)&        76.17 (0.02)&         3.31 (0.05)&               0.003\\
       G33.13-0.09 1&         0.47 (0.02)&       0.000 (0.000)&         0.00 (0.00)&         0.00 (0.00)&               0.003\\
       G33.13-0.09 2&         0.47 (0.02)&       0.000 (0.000)&         0.00 (0.00)&         0.00 (0.00)&               0.003\\
       G33.13-0.09 3&         0.47 (0.02)&       0.000 (0.000)&         0.00 (0.00)&         0.00 (0.00)&               0.003\\
       G33.92+0.11 0&         0.87 (0.02)&      -0.086 (0.003)&       106.43 (0.03)&         2.17 (0.09)&              0.0032\\
       G33.92+0.11 1&         0.87 (0.02)&      -0.069 (0.002)&       108.83 (0.11)&         6.82 (0.16)&              0.0032\\
       G33.92+0.11 2&         0.87 (0.02)&       0.000 (0.000)&         0.00 (0.00)&         0.00 (0.00)&              0.0032\\
       G34.26+0.15 0&         5.89 (0.02)&      -1.356 (0.006)&        60.99 (0.01)&         3.96 (0.02)&              0.0051\\
       G34.26+0.15 1&         5.89 (0.02)&      -0.046 (0.003)&        27.11 (0.04)&         1.03 (0.09)&              0.0051\\
       G34.26+0.15 2&         5.89 (0.02)&      -0.018 (0.002)&        11.23 (0.16)&         3.19 (0.38)&              0.0051\\
       G34.26+0.15 3&         5.89 (0.02)&      -0.025 (0.004)&        52.82 (0.58)&         6.34 (1.53)&              0.0051\\
       G34.26+0.15 4&         5.89 (0.02)&      -0.018 (0.007)&        47.05 (0.47)&         2.47 (1.15)&              0.0051\\
       G35.20-1.74 0&         5.98 (0.03)&      -0.482 (0.004)&        43.38 (0.02)&         3.71 (0.04)&              0.0055\\
       G35.20-1.74 1&         5.98 (0.03)&      -0.028 (0.005)&        37.91 (0.32)&         3.46 (0.76)&              0.0055\\
       G35.20-1.74 2&         5.98 (0.03)&      -0.056 (0.003)&        14.18 (0.02)&         1.00 (0.05)&              0.0055\\
       G35.20-1.74 3&         5.98 (0.03)&       0.000 (0.000)&         0.00 (0.00)&         0.00 (0.00)&              0.0055\\
       G35.57-0.03 0&         0.32 (0.15)&      -0.075 (0.003)&        52.14 (0.09)&         4.39 (0.21)&              0.0046\\
       G35.57-0.03 1&         0.32 (0.15)&      -0.015 (0.006)&        47.39 (0.25)&         1.31 (0.60)&              0.0046\\
       G35.57-0.03 2&         0.32 (0.15)&       0.000 (0.000)&         0.00 (0.00)&         0.00 (0.00)&              0.0046\\
       G35.57-0.03 3&         0.32 (0.15)&       0.000 (0.000)&         0.00 (0.00)&         0.00 (0.00)&              0.0046\\
       G35.57-0.03 4&         0.32 (0.15)&      -0.024 (0.008)&        29.25 (0.11)&         0.43 (0.15)&              0.0046\\
       G35.58+0.07 0&         0.23 (0.09)&      -0.106 (0.002)&        49.21 (0.06)&         5.00 (0.14)&              0.0031\\
       G35.58+0.07 1&         0.23 (0.09)&       0.000 (0.004)&         0.00 (0.00)&         0.00 (0.00)&              0.0031\\
       G35.58+0.07 2&         0.23 (0.09)&       0.000 (0.004)&         0.00 (0.00)&         0.00 (0.00)&              0.0031\\
       G35.58+0.07 3&         0.23 (0.09)&       0.000 (0.004)&         0.00 (0.00)&         0.00 (0.00)&              0.0031\\
       G37.87-0.40 0&         3.73 (0.02)&      -0.221 (0.003)&        59.99 (0.12)&         8.53 (0.14)&              0.0048\\
       G37.87-0.40 1&         3.73 (0.02)&      -0.045 (0.007)&        54.55 (0.25)&         5.99 (0.34)&              0.0048\\
       G37.87-0.40 2&         3.73 (0.02)&      -0.036 (0.007)&        65.06 (0.11)&         2.57 (0.45)&              0.0048\\
       G37.87-0.40 3&         3.73 (0.02)&      -0.053 (0.003)&        72.44 (0.05)&         1.37 (0.08)&              0.0048\\
       G37.87-0.40 4&         3.73 (0.02)&      -0.047 (0.002)&        74.25 (0.07)&         2.07 (0.18)&              0.0048\\
       G37.87-0.40 5&         3.73 (0.02)&      -0.016 (0.001)&        80.04 (0.03)&         1.28 (0.07)&              0.0048\\
       G37.87-0.40 6&         3.73 (0.02)&      -0.010 (0.002)&        91.99 (0.12)&         1.60 (0.28)&              0.0048\\
       G37.87-0.40 7&         3.73 (0.02)&      -0.026 (0.002)&        14.89 (0.12)&         1.40 (0.20)&              0.0048\\
       G37.87-0.40 8&         3.73 (0.02)&      -0.017 (0.002)&        13.29 (0.19)&         1.52 (0.34)&              0.0048\\
       G37.87-0.40 9&         3.73 (0.02)&      -0.017 (0.001)&        20.52 (0.10)&         3.09 (0.23)&              0.0048\\
       G43.89-0.78 0&         0.53 (0.02)&      -0.059 (0.004)&        54.61 (0.08)&         2.85 (0.23)&               0.003\\
       G43.89-0.78 1&         0.53 (0.02)&      -0.015 (0.002)&        49.59 (0.94)&        14.49 (1.69)&               0.003\\
       G45.07+0.13 0&         0.79 (0.07)&      -0.073 (0.003)&        57.18 (0.08)&         3.45 (0.18)&              0.0029\\
       G45.07+0.13 1&         0.79 (0.07)&      -0.011 (0.003)&        65.67 (0.42)&         3.46 (0.98)&              0.0029\\
       G45.12+0.13 0&         5.20 (0.20)&      -0.086 (0.002)&        56.21 (0.11)&         5.22 (0.21)&              0.0044\\
       G45.12+0.13 1&         5.20 (0.20)&      -0.059 (0.005)&        59.70 (0.06)&         2.42 (0.16)&              0.0044\\
       G45.12+0.13 2&         5.20 (0.20)&      -0.047 (0.002)&        25.14 (0.04)&         1.55 (0.09)&              0.0044\\
       G45.12+0.13 3&         5.20 (0.20)&      -0.021 (0.001)&        64.68 (0.39)&         8.15 (0.87)&              0.0044\\
       G45.45+0.06 0&         3.16 (0.02)&      -0.260 (0.003)&        59.58 (0.01)&         2.06 (0.03)&              0.0043\\
       G45.45+0.06 1&         3.16 (0.02)&      -0.042 (0.002)&        57.90 (0.14)&         9.40 (0.31)&              0.0043\\
       G45.45+0.06 2&         3.16 (0.02)&       0.000 (0.000)&         0.00 (0.00)&         0.00 (0.00)&              0.0043\\
       G45.47+0.05 0&         0.38 (0.02)&      -0.124 (0.003)&        61.67 (0.07)&         5.85 (0.17)&              0.0049\\
       G45.47+0.05 1&         0.38 (0.02)&      -0.000 (0.007)&         0.00 (0.00)&         0.00 (0.00)&              0.0049\\
       G48.61+0.02 0&         0.41 (0.02)&      -0.022 (0.003)&        18.50 (0.25)&         4.39 (0.59)&              0.0033\\
       G48.61+0.02 1&         0.41 (0.02)&      -0.000 (0.000)&         0.00 (0.00)&         0.00 (0.00)&              0.0033\\
       G48.61+0.02 2&         0.41 (0.02)&      -0.005 (0.002)&        52.50 (1.25)&         7.47 (2.94)&              0.0033\\
       G50.32+0.68 0&         0.16 (0.02)&      -0.011 (0.003)&        26.21 (0.44)&         3.10 (1.03)&              0.0036\\
       G60.88-0.13 0&         0.29 (0.02)&      -0.016 (0.003)&        21.63 (0.21)&         2.47 (0.50)&               0.003\\
       G61.48+0.09 0&         3.42 (0.02)&      -0.300 (0.004)&        21.40 (0.02)&         2.39 (0.04)&              0.0037\\
       G69.54-0.98 0&         0.23 (0.02)&      -0.220 (0.002)&         9.97 (0.03)&         5.81 (0.08)&              0.0031\\
       G70.29+1.60 0&         6.21 (0.02)&      -0.159 (0.003)&       -23.52 (0.06)&         5.36 (0.13)&              0.0046\\
       G70.29+1.60 1&         6.21 (0.02)&      -0.000 (0.000)&        -0.00 (0.00)&         0.00 (0.00)&              0.0046\\
       G70.33+1.59 0&         2.68 (0.02)&      -1.081 (0.005)&       -21.17 (0.01)&         2.95 (0.01)&              0.0038\\
   IRAS 20051+3435 0&         0.00 (0.02)&      -0.016 (0.003)&        11.51 (0.37)&         4.14 (0.88)&              0.0032\\
       G41.74+0.10 0&         0.28 (0.02)&      -0.014 (0.002)&        14.36 (0.34)&         3.80 (0.80)&              0.0032\\
       G41.74+0.10 1&         0.28 (0.02)&       0.000 (0.004)&         0.00 (0.00)&         0.00 (0.00)&              0.0032\\
       G41.74+0.10 2&         0.28 (0.02)&       0.000 (0.004)&         0.00 (0.00)&         0.00 (0.00)&              0.0032\\
       G41.74+0.10 3&         0.28 (0.02)&       0.000 (0.004)&         0.00 (0.00)&         0.00 (0.00)&              0.0032\\
       G41.74+0.10 4&         0.28 (0.02)&       0.000 (0.004)&         0.00 (0.00)&         0.00 (0.00)&              0.0032\\
      IRDC 1923+13 0&         0.00 (0.02)&       0.000 (0.000)&         0.00 (0.00)&         0.00 (0.00)&              0.0032\\
      IRDC 1923+13 1&         0.00 (0.02)&       0.000 (0.000)&         0.00 (0.00)&         0.00 (0.00)&              0.0032\\
      IRDC 1923+13 2&         0.00 (0.02)&       0.000 (0.000)&         0.00 (0.00)&         0.00 (0.00)&              0.0032\\
      IRDC 1916+11 0&         0.00 (0.02)&       0.000 (0.000)&         0.00 (0.00)&         0.00 (0.00)&              0.0048\\
      IRDC 1916+11 1&         0.00 (0.02)&       0.000 (0.000)&         0.00 (0.00)&         0.00 (0.00)&              0.0048\\
      IRDC 1916+11 2&         0.00 (0.02)&       0.000 (0.000)&         0.00 (0.00)&         0.00 (0.00)&              0.0048\\
}{
\tablenotetext{a}{ The Upper Limit Flag is 1 when the measurement indicated is
a $3-\sigma$ upper limit on the \twotwo\ line depth when there is a
corresponding \oneone\ line detection. }
\tablenotetext{b}{RMS in 1.011 \kms\ channels.} 
}

\Table{lccccccc}{Distance, BGPS 1.1 mm, and other properties}
{\colhead{Source Name}&\colhead{Distance}&\colhead{Galactocentric}&\colhead{KDA\tablenotemark{a}}&\colhead{$S_{1.1mm}$}&\colhead{Source}&\colhead{\formaldehyde\ }&\colhead{Scenario\tablenotemark{b}}\\
\colhead{           }&\colhead{        }&\colhead{Distance      }&\colhead{Resolution}&\colhead{           }&\colhead{Type  }&\colhead{Spectrum}&\colhead{}\\  
\colhead{           }&\colhead{(kpc)   }&\colhead{         (kpc)}&\colhead{          }&\colhead{(Jy)       }&\colhead{      }&\colhead{Type    }&\colhead{}\\ }
{tab:other}{
       G32.80+0.19 0&                12.9&                 7.4&                 far&                6.94&               UCHII&        red gradient&                 2+3\\
       G32.80+0.19 1&                13.1&                 7.6&                 far&                6.94&               UCHII&            envelope&                 2+3\\
       G32.80+0.19 2&                 9.4&                 5.1&                 far&                6.94&                 GMC&                   -&                 2+3\\
       G32.80+0.19 3&                 9.2&                 5.0&                 far&                6.94&                 GMC&                   -&                 2+3\\
       G32.80+0.19 4&                 9.0&                 4.9&                 far&                6.94&                 GMC&                   -&                 2+3\\
       G33.13-0.09 0&                 9.6&                 5.2&                 far&                2.26&               UCHII&        red gradient&                   2\\
       G33.13-0.09 1&                 9.3&                 5.1&                 far&                2.26&                 GMC&            envelope&                   2\\
       G33.13-0.09 2&                 7.1&                 4.7&             tangent&                2.26&                 GMC&                   -&                   2\\
       G33.13-0.09 3&                 0.9&                 7.6&                near&                2.26&                 GMC&                   -&                   2\\
       G33.92+0.11 0&                 7.0&                 4.6&             tangent&                3.86&               UCHII&        red gradient&                   2\\
       G33.92+0.11 1&                 7.0&                 4.6&             tangent&                3.86&               UCHII&            envelope&                   2\\
       G33.92+0.11 2&                 3.6&                 5.8&                near&                3.86&                 GMC&                   -&                   2\\
       G34.26+0.15 0&                 3.6&                 5.7&                near&               35.69&               UCHII&        red gradient&                   2\\
       G34.26+0.15 1&                 1.9&                 6.9&                near&               35.69&                 GMC&                   -&                   2\\
       G34.26+0.15 2&                 1.0&                 7.6&                near&               35.69&                 GMC&                   -&                   2\\
       G34.26+0.15 3&                 3.6&                 6.0&                near&               35.69&                 GMC&            envelope&                   2\\
       G34.26+0.15 4&                 3.6&                 6.1&                near&               35.69&                 GMC&                   -&                   2\\
       G35.20-1.74 0&                 2.8&                 6.3&                near&                   -&               UCHII&              single&                   4\\
       G35.20-1.74 1&                 2.5&                 6.5&                near&                   -&                 GMC&                   -&                   4\\
       G35.20-1.74 2&                 1.1&                 7.5&                near&                   -&                 GMC&                   -&                   4\\
       G35.20-1.74 3&                 3.2&                 6.1&                near&                   -&                 GMC&                   -&                   4\\
       G35.57-0.03 0&                10.3&                 6.0&                 far&                2.57&               UCHII&              single&                 2+3\\
       G35.57-0.03 1&                10.7&                 6.2&                 far&                2.57&                 GMC&                   -&                 2+3\\
       G35.57-0.03 2&                 3.6&                 5.9&                near&                2.57&                 GMC&                   -&                 2+3\\
       G35.57-0.03 3&                 1.1&                 7.6&                near&                2.57&                 GMC&                   -&                 2+3\\
       G35.57-0.03 4&                 2.0&                 6.8&                near&                2.57&                 GMC&                   -&                 2+3\\
       G35.58+0.07 0&                10.5&                 6.1&                 far&                1.44&               UCHII&       blue gradient&                   2\\
       G35.58+0.07 1&                10.3&                 6.0&                 far&                1.44&               UCHII&                   -&                   2\\
       G35.58+0.07 2&                 3.6&                 5.8&                near&                1.44&                 GMC&                   -&                   2\\
       G35.58+0.07 3&                 1.1&                 7.5&                near&                1.44&                 GMC&                   -&                   2\\
       G37.87-0.40 0&                 9.4&                 5.9&                 far&                4.14&               UCHII&       blue gradient&                   1\\
       G37.87-0.40 1&                 9.8&                 6.1&                 far&                4.14&               UCHII&       blue gradient&                   1\\
       G37.87-0.40 2&                 9.2&                 5.7&                 far&                4.14&               UCHII&       blue gradient&                   1\\
       G37.87-0.40 3&                 8.7&                 5.6&                 far&                4.14&                 GMC&                   -&                   1\\
       G37.87-0.40 4&                 8.6&                 5.5&                 far&                4.14&                 GMC&                   -&                   1\\
       G37.87-0.40 5&                 8.1&                 5.4&                 far&                4.14&                 GMC&                   -&                   1\\
       G37.87-0.40 6&                 6.6&                 5.1&             tangent&                4.14&                 GMC&                   -&                   1\\
       G37.87-0.40 7&                 1.2&                 7.5&                near&                4.14&                 GMC&                   -&                   1\\
       G37.87-0.40 8&                 1.1&                 7.6&                near&                4.14&                 GMC&                   -&                   1\\
       G37.87-0.40 9&                 1.5&                 7.2&                near&                4.14&                 GMC&                   -&                   1\\
       G43.89-0.78 0&                 8.3&                 6.2&                 far&                   -&               UCHII&       blue gradient&                   3\\
       G43.89-0.78 1&                 8.6&                 6.3&                 far&                   -&                 GMC&            envelope&                   3\\
       G45.07+0.13 0&                 7.6&                 6.2&                 far&                4.26&               UCHII&              single&                   2\\
       G45.07+0.13 1&                 6.5&                 6.0&                 far&                4.26&                 GMC&                   -&                   2\\
       G45.12+0.13 0&                 7.4&                 6.2&                 far&                6.78&               UCHII&               other&                   1\\
       G45.12+0.13 1&                 7.4&                 6.1&                 far&                6.78&               UCHII&            envelope&                   1\\
       G45.12+0.13 2&                 1.9&                 7.2&                near&                6.78&                 GMC&                   -&                   1\\
       G45.12+0.13 3&                 7.4&                 6.0&                 far&                6.78&                 GMC&            envelope&                   1\\
       G45.45+0.06 0&                 7.2&                 6.1&                 far&                3.71&               UCHII&       blue gradient&                   2\\
       G45.45+0.06 1&                 7.6&                 6.2&                 far&                3.71&                 GMC&            envelope&                   2\\
       G45.45+0.06 2&                 1.9&                 7.2&                near&                3.71&                 GMC&                   -&                   2\\
       G45.47+0.05 0&                 7.1&                 6.1&                 far&                3.34&               UCHII&        red gradient&               1+2+3\\
       G45.47+0.05 1&                 1.9&                 7.2&                near&                3.34&                 GMC&                   -&               1+2+3\\
       G48.61+0.02 0&                 9.6&                 7.5&                 far&                2.20&               UCHII&        red gradient&                 2+3\\
       G48.61+0.02 1&                 0.7&                 8.0&                near&                2.20&                 GMC&                   -&                 2+3\\
       G48.61+0.02 2&                 6.5&                 6.4&                 far&                2.20&                 GMC&                   -&                 2+3\\
       G50.32+0.68 0&                 2.1&                 7.2&                near&                   -&               UCHII&                   -&                   1\\
       G60.88-0.13 0&                 2.8&                 7.4&                near&                4.90&               UCHII&               limit&                   2\\
       G61.48+0.09 0&                 5.2&                 7.5&                 far&                7.86&               UCHII&              single&                   4\\
       G69.54-0.98 0&                2.57&                 7.9&             tangent&                   -&               UCHII&               thick&                 4+5\\
       G70.29+1.60 0&                 7.3&                 9.1&                 far&                   -&               UCHII&       blue gradient&                   2\\
       G70.29+1.60 1&                 7.8&                 9.3&                 far&                   -&                 GMC&            envelope&                   2\\
       G70.33+1.59 0&                 7.3&                 9.1&                 far&                   -&               UCHII&              single&                 1+2\\
   IRAS 20051+3435 0&                 2.6&                 7.6&             tangent&                   -&                 GMC&               limit&                  -1\\
       G41.74+0.10 0&                11.3&                 7.6&                 far&                0.56&               UCHII&               limit&                  -1\\
       G41.74+0.10 1&                11.6&                 7.7&                 far&                0.56&               UCHII&                   -&                  -1\\
       G41.74+0.10 2&                 2.4&                 6.8&                near&                0.56&                 GMC&                   -&                  -1\\
       G41.74+0.10 3&                 3.8&                 6.1&                near&                0.56&                 GMC&                   -&                  -1\\
       G41.74+0.10 4&                11.2&                 7.4&                 far&                0.56&               UCHII&                   -&                  -1\\
      IRDC 1923+13 0&                 4.2&                 6.5&                near&                   -&                 GMC&               limit&                  -1\\
      IRDC 1923+13 1&                 5.5&                 6.3&             tangent&                   -&                 GMC&                   -&                  -1\\
      IRDC 1923+13 2&                 3.8&                 6.6&                near&                   -&                 GMC&                   -&                  -1\\
      IRDC 1916+11 0&                 2.0&                 7.2&                near&                   -&                 GMC&               limit&                  -1\\
      IRDC 1916+11 1&                 4.2&                 6.2&                near&                   -&                 GMC&                   -&                  -1\\
      IRDC 1916+11 2&                 3.6&                 6.4&                near&                   -&                 GMC&                   -&                  -1\\
}{
\tablenotetext{a}{The Kinematic Distance Ambiguity described in Section \ref{sec:distances}.}
\tablenotetext{b}{Scenario or scenarios most likely to be consistent with the observed spectrum, as described
in Section \ref{sec:scenarios}.  In some cases, the spectrum was consistent with multiple scenarios or some
blend of multiple scenarios.  In others, the source could not be classified, in which case it is marked with 
-1 in this column. }
}

\Table{lcccccccc}{Inferred \formaldehyde\ line properties}
{\colhead{Source Name}&\colhead{$\tau_{1-1}$}&\colhead{$\tau_{1-1}$ (FFC)}&\colhead{$\tau_{2-2}$}&\colhead{$\tau_{2-2}$ (FFC)}&\colhead{2-2 Upper }&\colhead{2cm Area\tablenotemark{a}}&\colhead{6cm Area  \tablenotemark{a}   }&\colhead{FFC Error}\\
\colhead{           }&\colhead{             }&\colhead{                   }&\colhead{             }&\colhead{                   }&\colhead{Limit Flag}&\colhead{\arcsec$^2$                   }&\colhead{\arcsec$^2$                   }&\colhead{}\\ }
{tab:h2coinferred}{
       G32.80+0.19 0&        0.18 (0.055)&         0.2 (0.059)&        0.12 (0.024)&        0.15 (0.031)&                   0&                88.0&               226.2&                 0.1\\
       G32.80+0.19 1&        0.04 (0.013)&       0.043 (0.013)&      0.016 (0.0051)&      0.021 (0.0065)&                   0&                88.0&               226.2&                 0.1\\
       G32.80+0.19 2&      0.027 (0.0089)&      0.029 (0.0095)&    0.0033 (0.00069)&    0.0042 (0.00088)&                   0&                88.0&               226.2&                 0.1\\
       G32.80+0.19 3&        0.11 (0.035)&        0.12 (0.037)&      0.014 (0.0028)&      0.018 (0.0035)&                   0&                88.0&               226.2&                 0.1\\
       G32.80+0.19 4&       0.039 (0.012)&       0.042 (0.013)&     0.0055 (0.0011)&     0.0071 (0.0014)&                   0&                88.0&               226.2&                 0.1\\
       G33.13-0.09 0&          0.34 (0.1)&         0.49 (0.15)&        0.16 (0.032)&         0.63 (0.12)&                   0&                33.5&                33.5&                 0.2\\
       G33.13-0.09 1&       0.035 (0.015)&        0.047 (0.02)&         0 (0.0059)&         0 (0.0031)&                   1&                33.5&                33.5&                 0.2\\
       G33.13-0.09 2&       0.062 (0.022)&       0.084 (0.029)&         0 (0.0059)&         0 (0.0031)&                   1&                33.5&                33.5&                 0.2\\
       G33.13-0.09 3&       0.061 (0.021)&       0.082 (0.028)&         0 (0.0059)&         0 (0.0031)&                   1&                33.5&                33.5&                 0.2\\
       G33.92+0.11 0&       0.084 (0.027)&         0.1 (0.031)&      0.045 (0.0091)&       0.094 (0.018)&                   0&               214.0&               214.0&                 0.2\\
       G33.92+0.11 1&       0.082 (0.026)&       0.098 (0.031)&      0.036 (0.0072)&       0.075 (0.014)&                   0&               214.0&               214.0&                 0.2\\
       G33.92+0.11 2&        0.17 (0.062)&        0.21 (0.074)&         0 (0.0049)&         0 (0.0031)&                   1&               214.0&               214.0&                 0.2\\
       G34.26+0.15 0&         0.38 (0.12)&          0.4 (0.12)&        0.22 (0.043)&        0.26 (0.052)&                   0&                10.9&                10.9&                 0.2\\
       G34.26+0.15 1&      0.028 (0.0089)&      0.029 (0.0092)&     0.0067 (0.0014)&     0.0079 (0.0017)&                   0&                10.9&                10.9&                 0.2\\
       G34.26+0.15 2&      0.017 (0.0059)&       0.018 (0.006)&    0.0026 (0.00059)&     0.0031 (0.0007)&                   0&                10.9&                10.9&                 0.2\\
       G34.26+0.15 3&      0.022 (0.0072)&      0.023 (0.0074)&    0.0036 (0.00092)&     0.0043 (0.0011)&                   0&                10.9&                10.9&                 0.2\\
       G34.26+0.15 4&     0.0082 (0.0036)&     0.0085 (0.0037)&     0.0026 (0.0011)&      0.003 (0.0013)&                   0&                10.9&                10.9&                 0.2\\
       G35.20-1.74 0&        0.21 (0.063)&        0.22 (0.066)&       0.071 (0.014)&       0.084 (0.017)&                   0&                39.5&                39.5&                 0.2\\
       G35.20-1.74 1&      0.028 (0.0085)&      0.029 (0.0088)&     0.0039 (0.0011)&     0.0046 (0.0013)&                   0&                39.5&                39.5&                 0.2\\
       G35.20-1.74 2&       0.063 (0.019)&       0.065 (0.019)&      0.008 (0.0017)&     0.0095 (0.0019)&                   0&                39.5&                39.5&                 0.2\\
       G35.20-1.74 3&     0.0073 (0.0027)&     0.0075 (0.0028)&         0 (0.0023)&         0 (0.0031)&                   1&                39.5&                39.5&                 0.2\\
       G35.57-0.03 0&        0.11 (0.035)&        0.15 (0.049)&       0.056 (0.011)&        0.26 (0.054)&                   0&                 6.7&                 6.7&                 0.1\\
       G35.57-0.03 1&       0.034 (0.018)&       0.046 (0.024)&      0.011 (0.0047)&        0.047 (0.02)&                   0&                 6.7&                 6.7&                 0.1\\
       G35.57-0.03 2&        0.03 (0.017)&       0.042 (0.023)&         0 (0.0099)&          0 (0.019)&                   1&                 6.7&                 6.7&                 0.1\\
       G35.57-0.03 3&        0.05 (0.021)&       0.069 (0.029)&         0 (0.0099)&          0 (0.019)&                   1&                 6.7&                 6.7&                 0.1\\
       G35.57-0.03 4&        0.051 (0.02)&       0.069 (0.028)&      0.017 (0.0065)&       0.077 (0.029)&                   0&                 6.7&                 6.7&                 0.1\\
       G35.58+0.07 0&        0.24 (0.071)&        0.32 (0.097)&       0.085 (0.017)&         0.61 (0.12)&                   0&                 2.1&                 2.1&                 0.2\\
       G35.58+0.07 1&       0.072 (0.029)&       0.096 (0.038)&         0 (0.0072)&          0 (0.019)&                   1&                 2.1&                 2.1&                 0.2\\
       G35.58+0.07 2&       0.037 (0.012)&       0.049 (0.016)&         0 (0.0072)&          0 (0.019)&                   1&                 2.1&                 2.1&                 0.2\\
       G35.58+0.07 3&        0.05 (0.016)&       0.066 (0.021)&         0 (0.0072)&           0 (0.01)&                   1&                 2.1&                 2.1&                 0.2\\
       G37.87-0.40 0&        0.12 (0.037)&        0.13 (0.038)&      0.047 (0.0095)&       0.061 (0.012)&                   0&                27.5&               170.9&                 0.2\\
       G37.87-0.40 1&      0.028 (0.0089)&      0.029 (0.0092)&     0.0095 (0.0024)&      0.012 (0.0031)&                   0&                27.5&               170.9&                 0.2\\
       G37.87-0.40 2&       0.081 (0.025)&       0.084 (0.026)&     0.0075 (0.0021)&     0.0096 (0.0027)&                   0&                27.5&               170.9&                 0.2\\
       G37.87-0.40 3&       0.074 (0.024)&       0.076 (0.025)&      0.011 (0.0023)&       0.014 (0.003)&                   0&                27.5&               170.9&                 0.2\\
       G37.87-0.40 4&       0.097 (0.029)&          0.1 (0.03)&      0.0098 (0.002)&      0.013 (0.0026)&                   0&                27.5&               170.9&                 0.2\\
       G37.87-0.40 5&       0.041 (0.013)&       0.043 (0.013)&    0.0033 (0.00068)&    0.0043 (0.00087)&                   0&                27.5&               170.9&                 0.2\\
       G37.87-0.40 6&      0.025 (0.0083)&      0.026 (0.0086)&    0.0021 (0.00052)&    0.0026 (0.00066)&                   0&                27.5&               170.9&                 0.2\\
       G37.87-0.40 7&       0.039 (0.012)&        0.04 (0.012)&     0.0054 (0.0012)&     0.0069 (0.0015)&                   0&                27.5&               170.9&                 0.2\\
       G37.87-0.40 8&      0.016 (0.0069)&      0.016 (0.0071)&    0.0035 (0.00081)&      0.0046 (0.001)&                   0&                27.5&               170.9&                 0.2\\
       G37.87-0.40 9&       0.03 (0.0095)&      0.031 (0.0098)&    0.0035 (0.00073)&    0.0045 (0.00094)&                   0&                27.5&               170.9&                 0.2\\
       G43.89-0.78 0&        0.25 (0.074)&        0.32 (0.096)&      0.037 (0.0078)&        0.12 (0.024)&                   0&                13.5&                13.5&                 0.1\\
       G43.89-0.78 1&      0.025 (0.0077)&      0.031 (0.0097)&     0.0097 (0.0022)&      0.029 (0.0067)&                   0&                13.5&                13.5&                 0.1\\
       G45.07+0.13 0&       0.092 (0.029)&         0.13 (0.04)&       0.04 (0.0081)&        0.096 (0.02)&                   0&                 2.5&                 2.5&                 0.2\\
       G45.07+0.13 1&        0.058 (0.02)&        0.08 (0.027)&     0.0061 (0.0019)&      0.014 (0.0045)&                   0&                 2.5&                 2.5&                 0.2\\
       G45.12+0.13 0&       0.043 (0.013)&       0.045 (0.013)&      0.014 (0.0028)&      0.017 (0.0033)&                   0&                15.4&               516.6&                 0.2\\
       G45.12+0.13 1&       0.035 (0.011)&       0.036 (0.011)&      0.0095 (0.002)&      0.011 (0.0025)&                   0&                15.4&               516.6&                 0.2\\
       G45.12+0.13 2&       0.046 (0.014)&       0.048 (0.014)&     0.0075 (0.0015)&      0.009 (0.0019)&                   0&                15.4&               516.6&                 0.2\\
       G45.12+0.13 3&       0.006 (0.002)&     0.0062 (0.0021)&    0.0033 (0.00068)&     0.004 (0.00082)&                   0&                15.4&               516.6&                 0.2\\
       G45.45+0.06 0&        0.32 (0.096)&        0.32 (0.095)&       0.063 (0.013)&       0.069 (0.012)&                   0&              1963.0&              1963.0&                 0.2\\
       G45.45+0.06 1&       0.025 (0.011)&       0.026 (0.011)&       0.01 (0.0021)&      0.011 (0.0019)&                   0&              1963.0&              1963.0&                 0.2\\
       G45.45+0.06 2&      0.011 (0.0036)&      0.012 (0.0035)&         0 (0.0031)&           0 (0.01)&                   1&              1963.0&              1963.0&                 0.2\\
       G45.47+0.05 0&         0.35 (0.11)&         0.45 (0.14)&       0.089 (0.018)&        0.39 (0.079)&                   0&                 3.0&                 3.0&                 0.2\\
       G45.47+0.05 1&      0.018 (0.0068)&      0.023 (0.0084)&           0 (0.01)&           0 (0.01)&                   1&                 3.0&                 3.0&                 0.2\\
       G48.61+0.02 0&       0.058 (0.018)&       0.068 (0.021)&      0.015 (0.0034)&       0.053 (0.012)&                   0&                25.5&                25.5&                 0.2\\
       G48.61+0.02 1&       0.02 (0.0075)&      0.023 (0.0088)&         0 (0.0067)&         0 (0.0026)&                   1&                25.5&                25.5&                 0.2\\
       G48.61+0.02 2&      0.016 (0.0052)&      0.018 (0.0061)&     0.0033 (0.0013)&      0.012 (0.0046)&                   0&                25.5&                25.5&                 0.2\\
       G50.32+0.68 0&        0.027 (0.01)&       0.045 (0.017)&     0.0089 (0.0031)&       0.058 (0.019)&                   0&               108.0&               108.0&                 0.2\\
       G60.88-0.13 0&        0.12 (0.037)&        0.14 (0.043)&      0.011 (0.0031)&      0.031 (0.0071)&                   0&               615.0&               615.0&                 0.2\\
       G61.48+0.09 0&       0.088 (0.026)&        0.09 (0.027)&       0.069 (0.014)&       0.088 (0.017)&                   0&               355.0&               355.0&                 0.2\\
       G69.54-0.98 0&         0.98 (0.29)&           5.7 (1.7)&        0.18 (0.037)&          2.9 (0.57)&                   0&                 0.5&                 0.5&                 0.2\\
       G70.29+1.60 0&       0.086 (0.026)&       0.089 (0.027)&      0.022 (0.0044)&      0.026 (0.0052)&                   0&                52.8&                52.8&                 0.1\\
       G70.29+1.60 1&      0.011 (0.0037)&      0.012 (0.0038)&         0 (0.0019)&         0 (0.0026)&                   1&                52.8&                52.8&                 0.1\\
       G70.33+1.59 0&          0.7 (0.21)&         0.78 (0.24)&        0.34 (0.068)&          0.52 (0.1)&                   0&                16.4&                16.4&                 0.1\\
   IRAS 20051+3435 0&        0.12 (0.036)&        0.13 (0.014)&      0.015 (0.0041)&      0.016 (0.0034)&                   0&             2747.75&             2747.75&                 0.0\\
       G41.74+0.10 0&         0.13 (0.04)&          0.2 (0.06)&       0.01 (0.0027)&       0.045 (0.012)&                   0&                75.2&                75.2&                 0.2\\
       G41.74+0.10 1&        0.04 (0.014)&        0.06 (0.021)&         0 (0.0071)&         0 (0.0026)&                   1&                75.2&                75.2&                 0.2\\
       G41.74+0.10 2&        0.14 (0.043)&        0.21 (0.065)&         0 (0.0071)&         0 (0.0026)&                   1&                75.2&                75.2&                 0.2\\
       G41.74+0.10 3&       0.045 (0.017)&       0.067 (0.025)&         0 (0.0071)&         0 (0.0026)&                   1&                75.2&                75.2&                 0.2\\
       G41.74+0.10 4&       0.089 (0.029)&        0.13 (0.043)&         0 (0.0071)&         0 (0.0028)&                   1&                75.2&                75.2&                 0.2\\
      IRDC 1923+13 0&       0.02 (0.0062)&       0.02 (0.0047)&          0 (0.009)&         0 (0.0028)&                   1&             2747.75&             2747.75&                 0.0\\
      IRDC 1923+13 1&      0.016 (0.0051)&      0.017 (0.0038)&          0 (0.009)&         0 (0.0028)&                   1&             2747.75&             2747.75&                 0.0\\
      IRDC 1923+13 2&     0.0081 (0.0028)&     0.0083 (0.0023)&          0 (0.009)&         0 (0.0028)&                   1&             2747.75&             2747.75&                 0.0\\
      IRDC 1916+11 0&       0.033 (0.011)&      0.036 (0.0062)&          0 (0.014)&          0 (0.042)&                   1&             2747.75&             2747.75&                 0.0\\
      IRDC 1916+11 1&       0.082 (0.025)&      0.089 (0.0095)&          0 (0.014)&          0 (0.042)&                   1&             2747.75&             2747.75&                 0.0\\
      IRDC 1916+11 2&      0.017 (0.0064)&      0.018 (0.0046)&          0 (0.014)&          0 (0.042)&                   1&             2747.75&             2747.75&                 0.0\\
}{\tablenotetext{a}{The beam area is 2747.75\arcsec$^2$, which is used when the CMB is the only background continuum illumination}}

\Table{lccccccc}{Derived physical properties from \formaldehyde\ }
{\colhead{Source Name}&\colhead{N(\formaldehyde)\tablenotemark{a}}&\colhead{N(\formaldehyde) (FFC)\tablenotemark{b}}&\colhead{n(\hh) \tablenotemark{a} }&\colhead{n(\hh) (FFC)\tablenotemark{b}}&\colhead{X$_{\formaldehyde}$\tablenotemark{a}}&\colhead{X$_{\formaldehyde}$ (FFC)\tablenotemark{b}}&\colhead{Flag\tablenotemark{c}}\\
\colhead{           }&\colhead{(\persc)        }&\colhead{(\persc)              }&\colhead{(\percc)}&\colhead{(\percc)    }&\colhead{                    }&\colhead{                          }&\colhead{                      }\\ }
{tab:h2coderived}{
       G32.80+0.19 0&$12.79^{+0.11}_{-0.16}$&$\mathbf{12.94^{+0.16}_{-0.24}}$&$5.10^{+0.25}_{-0.26}$&$\mathbf{5.21^{+0.27}_{-0.29}}$&$-10.79^{+0.15}_{-0.20}$&$\mathbf{-10.75^{+0.15}_{-0.18}}$&                   2\\
       G32.80+0.19 1&$12.05^{+0.12}_{-0.11}$&$\mathbf{12.14^{+0.13}_{-0.13}}$&$4.96^{+0.22}_{-0.28}$&$\mathbf{5.05^{+0.21}_{-0.28}}$&$-11.39^{+0.20}_{-0.23}$&$\mathbf{-11.39^{+0.17}_{-0.20}}$&                   2\\
       G32.80+0.19 2&$11.66^{+0.10}_{-0.10}$&$\mathbf{11.71^{+0.10}_{-0.10}}$&$4.16^{+0.39}_{-0.38}$&$\mathbf{4.33^{+0.31}_{-0.32}}$&$-10.97^{+0.44}_{-0.46}$&$\mathbf{-11.10^{+0.37}_{-0.37}}$&                   2\\
       G32.80+0.19 3&$12.18^{+0.10}_{-0.09}$&$\mathbf{12.23^{+0.09}_{-0.09}}$&$4.07^{+0.38}_{-0.39}$&$\mathbf{4.23^{+0.32}_{-0.32}}$&$-10.37^{+0.44}_{-0.45}$&$\mathbf{-10.48^{+0.36}_{-0.38}}$&                   2\\
       G32.80+0.19 4&$11.82^{+0.10}_{-0.09}$&$\mathbf{11.87^{+0.10}_{-0.09}}$&$4.30^{+0.31}_{-0.32}$&$\mathbf{4.44^{+0.26}_{-0.29}}$&$-10.97^{+0.37}_{-0.37}$&$\mathbf{-11.05^{+0.31}_{-0.32}}$&                   2\\
       G33.13-0.09 0&$>12.80                 $&$\mathbf{>13.56}       $&$>4.54                 $&$\mathbf{>5.10}       $&$>-10.62                 $&$\mathbf{>-11.70}       $&                   8\\
       G33.13-0.09 1&$<11.96                 $&$\mathbf{<11.90}       $&$<4.50                 $&$\mathbf{<3.91}       $&$<-8.44                 $&$\mathbf{<-8.45}       $&                   6\\
       G33.13-0.09 2&$\mathbf{<12.20}       $&$<0.00                 $&$\mathbf{<4.29}       $&$<0.00                 $&$\mathbf{<-8.29}       $&$<0.00                 $&                   5\\
       G33.13-0.09 3&$\mathbf{<12.20}       $&$<0.00                 $&$\mathbf{<4.32}       $&$<0.00                 $&$\mathbf{<-8.29}       $&$<0.00                 $&                   5\\
       G33.92+0.11 0&$>12.35                 $&$\mathbf{>12.64}       $&$>4.86                 $&$\mathbf{>5.16}       $&$>-11.29                 $&$\mathbf{>-12.30}       $&                   8\\
       G33.92+0.11 1&$12.34^{+0.07}_{-0.08}$&$\mathbf{12.65^{+0.11}_{-0.17}}$&$4.97^{+0.22}_{-0.23}$&$\mathbf{5.26^{+0.22}_{-0.24}}$&$-11.11^{+0.19}_{-0.22}$&$\mathbf{-11.09^{+0.13}_{-0.16}}$&                   2\\
       G33.92+0.11 2&                   -&                   -&                   -&                   -&                   -&                   -&                   9\\
       G34.26+0.15 0&$13.01^{+0.10}_{-0.17}$&$\mathbf{13.13^{+0.15}_{-0.23}}$&$4.91^{+0.28}_{-0.29}$&$\mathbf{5.01^{+0.31}_{-0.32}}$&$-10.38^{+0.18}_{-0.23}$&$\mathbf{-10.36^{+0.17}_{-0.23}}$&                   2\\
       G34.26+0.15 1&$11.79^{+0.09}_{-0.08}$&$\mathbf{11.83^{+0.09}_{-0.08}}$&$4.67^{+0.23}_{-0.25}$&$\mathbf{4.75^{+0.21}_{-0.24}}$&$-11.36^{+0.26}_{-0.27}$&$\mathbf{-11.40^{+0.23}_{-0.25}}$&                   2\\
       G34.26+0.15 2&$11.53^{+0.10}_{-0.10}$&$\mathbf{11.56^{+0.10}_{-0.10}}$&$4.38^{+0.30}_{-0.33}$&$\mathbf{4.48^{+0.28}_{-0.30}}$&$-11.33^{+0.36}_{-0.37}$&$\mathbf{-11.40^{+0.32}_{-0.34}}$&                   2\\
       G34.26+0.15 3&$11.63^{+0.11}_{-0.10}$&$\mathbf{11.66^{+0.10}_{-0.10}}$&$4.43^{+0.29}_{-0.32}$&$\mathbf{4.53^{+0.26}_{-0.30}}$&$-11.28^{+0.34}_{-0.35}$&$\mathbf{-11.34^{+0.31}_{-0.32}}$&                   2\\
       G34.26+0.15 4&$11.40^{+0.17}_{-0.14}$&$\mathbf{11.45^{+0.17}_{-0.16}}$&$4.87^{+0.31}_{-0.43}$&$\mathbf{4.94^{+0.30}_{-0.42}}$&$-11.95^{+0.30}_{-0.35}$&$\mathbf{-11.98^{+0.27}_{-0.34}}$&                   2\\
       G35.20-1.74 0&$12.60^{+0.08}_{-0.07}$&$\mathbf{12.65^{+0.08}_{-0.07}}$&$4.72^{+0.25}_{-0.25}$&$\mathbf{4.79^{+0.25}_{-0.26}}$&$-10.61^{+0.23}_{-0.27}$&$\mathbf{-10.62^{+0.22}_{-0.26}}$&                   2\\
       G35.20-1.74 1&$11.69^{+0.11}_{-0.10}$&$\mathbf{11.72^{+0.11}_{-0.10}}$&$4.30^{+0.33}_{-0.37}$&$\mathbf{4.41^{+0.30}_{-0.33}}$&$-11.09^{+0.39}_{-0.39}$&$\mathbf{-11.16^{+0.34}_{-0.36}}$&                   2\\
       G35.20-1.74 2&$11.97^{+0.09}_{-0.09}$&$\mathbf{12.00^{+0.09}_{-0.09}}$&$4.20^{+0.34}_{-0.35}$&$\mathbf{4.31^{+0.30}_{-0.30}}$&$-10.70^{+0.39}_{-0.41}$&$\mathbf{-10.79^{+0.34}_{-0.36}}$&                   2\\
       G35.20-1.74 3&$<11.41                 $&$\mathbf{<11.44}       $&$<4.89                 $&$\mathbf{<5.02}       $&$<-9.24                 $&$\mathbf{<-9.30}       $&                   6\\
       G35.57-0.03 0&$>12.42                 $&$\mathbf{>13.38}       $&$>4.82                 $&$\mathbf{>5.61}       $&$>-11.20                 $&$\mathbf{>-12.02}       $&                   8\\
       G35.57-0.03 1&$>11.72                 $&$\mathbf{>12.25}       $&$>4.51                 $&$\mathbf{>5.13}       $&$>-11.80                 $&$\mathbf{>-12.72}       $&                   8\\
       G35.57-0.03 2&$<11.98                 $&$\mathbf{<12.12}       $&$<4.96                 $&$\mathbf{<5.12}       $&$<-8.71                 $&$\mathbf{<-8.72}       $&                   6\\
       G35.57-0.03 3&$<12.09                 $&$\mathbf{<12.23}       $&$<4.60                 $&$\mathbf{<4.78}       $&$<-8.38                 $&$\mathbf{<-8.37}       $&                   6\\
       G35.57-0.03 4&$>11.93                 $&$\mathbf{>12.47}       $&$>4.58                 $&$\mathbf{>5.20}       $&$>-11.55                 $&$\mathbf{>-12.48}       $&                   8\\
       G35.58+0.07 0&$>12.58                 $&$\mathbf{>14.06}       $&$>4.50                 $&$\mathbf{>5.48}       $&$>-10.79                 $&$\mathbf{>-11.71}       $&                   8\\
       G35.58+0.07 1&$<12.19                 $&$\mathbf{<12.32}       $&$<4.08                 $&$\mathbf{<4.55}       $&$<-8.07                 $&$\mathbf{<-8.14}       $&                   6\\
       G35.58+0.07 2&$<11.96                 $&$\mathbf{<12.10}       $&$<4.57                 $&$\mathbf{<4.95}       $&$<-8.53                 $&$\mathbf{<-8.63}       $&                   6\\
       G35.58+0.07 3&$<12.06                 $&$\mathbf{<12.17}       $&$<4.35                 $&$\mathbf{<4.38}       $&$<-8.32                 $&$\mathbf{<-8.24}       $&                   6\\
       G37.87-0.40 0&$12.44^{+0.07}_{-0.07}$&$\mathbf{12.53^{+0.07}_{-0.09}}$&$4.86^{+0.22}_{-0.23}$&$\mathbf{4.98^{+0.21}_{-0.24}}$&$-10.90^{+0.20}_{-0.24}$&$\mathbf{-10.92^{+0.18}_{-0.21}}$&                   2\\
       G37.87-0.40 1&$11.87^{+0.10}_{-0.09}$&$\mathbf{11.95^{+0.10}_{-0.09}}$&$4.89^{+0.22}_{-0.26}$&$\mathbf{5.00^{+0.21}_{-0.24}}$&$-11.50^{+0.21}_{-0.24}$&$\mathbf{-11.53^{+0.18}_{-0.22}}$&                   2\\
       G37.87-0.40 2&$12.10^{+0.18}_{-0.26}$&$\mathbf{12.09^{+0.14}_{-0.28}}$&$3.16^{+1.15}_{-1.20}$&$\mathbf{3.79^{+1.78}_{-0.71}}$&$-9.54^{+1.32}_{-1.41}$&$\mathbf{-10.18^{+0.77}_{-2.06}}$&                   4\\
       G37.87-0.40 3&$12.05^{+0.10}_{-0.10}$&$\mathbf{12.10^{+0.10}_{-0.09}}$&$4.33^{+0.32}_{-0.33}$&$\mathbf{4.49^{+0.28}_{-0.29}}$&$-10.76^{+0.37}_{-0.39}$&$\mathbf{-10.87^{+0.31}_{-0.34}}$&                   2\\
       G37.87-0.40 4&$12.17^{+0.17}_{-0.25}$&$\mathbf{12.14^{+0.10}_{-0.09}}$&$3.36^{+1.35}_{-0.98}$&$\mathbf{4.13^{+0.35}_{-0.36}}$&$-9.67^{+1.12}_{-1.60}$&$\mathbf{-10.46^{+0.41}_{-0.42}}$&                   2\\
       G37.87-0.40 5&$11.85^{+0.18}_{-0.21}$&$\mathbf{11.88^{+0.18}_{-0.25}}$&$3.05^{+1.04}_{-1.17}$&$\mathbf{3.34^{+1.33}_{-1.03}}$&$-9.68^{+1.32}_{-1.25}$&$\mathbf{-9.94^{+1.17}_{-1.58}}$&                   4\\
       G37.87-0.40 6&$11.67^{+0.19}_{-0.23}$&$\mathbf{11.70^{+0.19}_{-0.26}}$&$3.08^{+1.07}_{-1.21}$&$\mathbf{3.33^{+1.32}_{-1.13}}$&$-9.89^{+1.36}_{-1.30}$&$\mathbf{-10.11^{+1.26}_{-1.58}}$&                   4\\
       G37.87-0.40 7&$11.81^{+0.10}_{-0.09}$&$\mathbf{11.86^{+0.10}_{-0.09}}$&$4.28^{+0.31}_{-0.33}$&$\mathbf{4.45^{+0.27}_{-0.29}}$&$-10.95^{+0.37}_{-0.38}$&$\mathbf{-11.07^{+0.32}_{-0.32}}$&                   2\\
       G37.87-0.40 8&$11.56^{+0.11}_{-0.11}$&$\mathbf{11.63^{+0.11}_{-0.10}}$&$4.67^{+0.30}_{-0.33}$&$\mathbf{4.80^{+0.29}_{-0.31}}$&$-11.59^{+0.34}_{-0.38}$&$\mathbf{-11.65^{+0.30}_{-0.34}}$&                   2\\
       G37.87-0.40 9&$11.72^{+0.13}_{-0.29}$&$\mathbf{11.74^{+0.10}_{-0.09}}$&$3.98^{+1.97}_{-0.52}$&$\mathbf{4.32^{+0.31}_{-0.32}}$&$-10.74^{+0.60}_{-2.26}$&$\mathbf{-11.06^{+0.36}_{-0.37}}$&                   2\\
       G43.89-0.78 0&$12.49^{+0.10}_{-0.09}$&$\mathbf{12.76^{+0.08}_{-0.07}}$&$4.18^{+0.34}_{-0.33}$&$\mathbf{4.68^{+0.28}_{-0.28}}$&$-10.17^{+0.37}_{-0.40}$&$\mathbf{-10.40^{+0.24}_{-0.30}}$&                   2\\
       G43.89-0.78 1&$\mathbf{11.87^{+0.09}_{-0.08}}$&$12.80^{+0.61}_{-1.00}$&$\mathbf{4.95^{+0.20}_{-0.23}}$&$6.16^{+0.96}_{-1.84}$&$\mathbf{-11.56^{+0.19}_{-0.22}}$&$-11.84^{+0.88}_{-0.42}$&                   1\\
       G45.07+0.13 0&$12.38^{+0.08}_{-0.08}$&$\mathbf{12.75^{+0.13}_{-0.20}}$&$4.96^{+0.22}_{-0.24}$&$\mathbf{5.25^{+0.25}_{-0.27}}$&$-11.06^{+0.19}_{-0.22}$&$\mathbf{-10.97^{+0.15}_{-0.18}}$&                   2\\
       G45.07+0.13 1&                   -&                   -&                   -&                   -&                   -&                   -&                   9\\
       G45.12+0.13 0&$12.02^{+0.08}_{-0.07}$&$\mathbf{12.07^{+0.08}_{-0.07}}$&$4.83^{+0.21}_{-0.21}$&$\mathbf{4.92^{+0.19}_{-0.21}}$&$-11.30^{+0.20}_{-0.23}$&$\mathbf{-11.32^{+0.19}_{-0.21}}$&                   2\\
       G45.12+0.13 1&$11.90^{+0.09}_{-0.08}$&$\mathbf{11.95^{+0.08}_{-0.08}}$&$4.74^{+0.22}_{-0.23}$&$\mathbf{4.83^{+0.21}_{-0.23}}$&$-11.32^{+0.23}_{-0.25}$&$\mathbf{-11.36^{+0.21}_{-0.23}}$&                   2\\
       G45.12+0.13 2&$11.90^{+0.10}_{-0.08}$&$\mathbf{11.93^{+0.09}_{-0.09}}$&$4.41^{+0.26}_{-0.28}$&$\mathbf{4.52^{+0.24}_{-0.26}}$&$-11.00^{+0.32}_{-0.32}$&$\mathbf{-11.06^{+0.29}_{-0.30}}$&                   2\\
       G45.12+0.13 3&$11.48^{+0.08}_{-0.09}$&$\mathbf{11.55^{+0.08}_{-0.12}}$&$5.15^{+0.19}_{-0.21}$&$\mathbf{5.23^{+0.19}_{-0.22}}$&$-12.15^{+0.16}_{-0.18}$&$\mathbf{-12.16^{+0.14}_{-0.17}}$&                   2\\
       G45.45+0.06 0&$12.62^{+0.08}_{-0.08}$&$\mathbf{12.64^{+0.07}_{-0.08}}$&$4.33^{+0.29}_{-0.31}$&$\mathbf{4.37^{+0.28}_{-0.28}}$&$-10.19^{+0.33}_{-0.35}$&$\mathbf{-10.21^{+0.30}_{-0.33}}$&                   2\\
       G45.45+0.06 1&$11.89^{+0.08}_{-0.08}$&$\mathbf{11.92^{+0.07}_{-0.07}}$&$5.00^{+0.26}_{-0.28}$&$\mathbf{5.04^{+0.26}_{-0.27}}$&$-11.59^{+0.25}_{-0.29}$&$\mathbf{-11.60^{+0.23}_{-0.28}}$&                   2\\
       G45.45+0.06 2&$<11.55                 $&$\mathbf{<11.66}       $&$<4.77                 $&$\mathbf{<5.39}       $&$<-9.04                 $&$\mathbf{<-9.43}       $&                   6\\
       G45.47+0.05 0&$12.71^{+0.09}_{-0.07}$&$\mathbf{13.48^{+0.32}_{-0.50}}$&$4.46^{+0.28}_{-0.28}$&$\mathbf{5.21^{+0.40}_{-0.34}}$&$-10.23^{+0.28}_{-0.31}$&$\mathbf{-10.21^{+0.21}_{-0.19}}$&                   2\\
       G45.47+0.05 1&$\mathbf{<11.91}       $&$<11.92                 $&$\mathbf{<5.34}       $&$<5.23                 $&$\mathbf{<-8.65}       $&$<-8.57                 $&                   5\\
       G48.61+0.02 0&$12.06^{+0.09}_{-0.09}$&$\mathbf{12.53^{+0.13}_{-0.17}}$&$4.68^{+0.23}_{-0.25}$&$\mathbf{5.29^{+0.22}_{-0.24}}$&$-11.10^{+0.25}_{-0.26}$&$\mathbf{-11.25^{+0.13}_{-0.16}}$&                   2\\
       G48.61+0.02 1&$\mathbf{<11.88}       $&$<11.91                 $&$\mathbf{<5.09}       $&$<4.47                 $&$\mathbf{<-8.61}       $&$<-8.58                 $&                   5\\
       G48.61+0.02 2&$11.54^{+0.14}_{-0.13}$&$\mathbf{11.94^{+0.18}_{-0.23}}$&$4.60^{+0.28}_{-0.39}$&$\mathbf{5.22^{+0.23}_{-0.36}}$&$-11.54^{+0.33}_{-0.33}$&$\mathbf{-11.76^{+0.14}_{-0.19}}$&                   2\\
       G50.32+0.68 0&$>11.71                 $&$\mathbf{>12.41}       $&$>4.61                 $&$\mathbf{>5.31}       $&$>-11.77                 $&$\mathbf{>-12.63}       $&                   8\\
       G60.88-0.13 0&$12.24^{+0.18}_{-0.25}$&$\mathbf{12.35^{+0.09}_{-0.09}}$&$3.20^{+1.19}_{-1.16}$&$\mathbf{4.51^{+0.27}_{-0.28}}$&$-9.44^{+1.29}_{-1.43}$&$\mathbf{-10.64^{+0.28}_{-0.31}}$&                   2\\
       G61.48+0.09 0&$>12.51                 $&$\mathbf{>12.62}       $&$>5.07                 $&$\mathbf{>5.19}       $&$>-11.27                 $&$\mathbf{>-12.33}       $&                   8\\
       G69.54-0.98 0&                   -&                   -&                   -&                   -&                   -&                   -&                  11\\
       G70.29+1.60 0&$12.21^{+0.09}_{-0.08}$&$\mathbf{12.25^{+0.08}_{-0.08}}$&$4.67^{+0.23}_{-0.23}$&$\mathbf{4.74^{+0.23}_{-0.24}}$&$-10.94^{+0.24}_{-0.26}$&$\mathbf{-10.97^{+0.23}_{-0.26}}$&                   2\\
       G70.29+1.60 1&$<11.53                 $&$\mathbf{<11.55}       $&$<4.50                 $&$\mathbf{<4.67}       $&$<-8.92                 $&$\mathbf{<-8.98}       $&                   6\\
       G70.33+1.59 0&$13.16^{+0.09}_{-0.14}$&$\mathbf{13.41^{+0.19}_{-0.35}}$&$4.64^{+0.34}_{-0.32}$&$\mathbf{4.83^{+0.39}_{-0.37}}$&$-9.96^{+0.22}_{-0.31}$&$\mathbf{-9.90^{+0.21}_{-0.26}}$&                   2\\
   IRAS 20051+3435 0&$\mathbf{12.20^{+0.11}_{-0.10}}$&$12.23^{+0.04}_{-0.05}$&$\mathbf{4.12^{+0.39}_{-0.41}}$&$4.11^{+0.21}_{-0.23}$&$\mathbf{-10.40^{+0.45}_{-0.46}}$&$-10.35^{+0.22}_{-0.22}$&                   3\\
       G41.74+0.10 0&$12.25^{+0.17}_{-0.23}$&$\mathbf{12.48^{+0.10}_{-0.09}}$&$2.99^{+0.99}_{-1.18}$&$\mathbf{4.50^{+0.28}_{-0.31}}$&$-9.23^{+1.31}_{-1.22}$&$\mathbf{-10.50^{+0.29}_{-0.32}}$&                   2\\
       G41.74+0.10 1&$\mathbf{<12.12}       $&$<0.00                 $&$\mathbf{<4.72}       $&$<0.00                 $&$\mathbf{<-8.37}       $&$<0.00                 $&                   5\\
       G41.74+0.10 2&$\mathbf{<12.18}       $&$<0.00                 $&$\mathbf{<3.21}       $&$<0.00                 $&$\mathbf{<-8.91}       $&$<0.00                 $&                   5\\
       G41.74+0.10 3&$\mathbf{<12.17}       $&$<0.00                 $&$\mathbf{<4.70}       $&$<0.00                 $&$\mathbf{<-8.32}       $&$<0.00                 $&                   5\\
       G41.74+0.10 4&$\mathbf{<12.27}       $&$<0.00                 $&$\mathbf{<4.11}       $&$<0.00                 $&$\mathbf{<-8.22}       $&$<0.00                 $&                   5\\
      IRDC 1923+13 0&$\mathbf{<11.86}       $&$<11.84                 $&$\mathbf{<5.19}       $&$<4.47                 $&$\mathbf{<-8.63}       $&$<-8.65                 $&                   5\\
      IRDC 1923+13 1&$\mathbf{<11.86}       $&$<11.77                 $&$\mathbf{<5.30}       $&$<4.58                 $&$\mathbf{<-8.70}       $&$<-8.72                 $&                   5\\
      IRDC 1923+13 2&$\mathbf{<13.29}       $&$<11.53                 $&$\mathbf{<8.00}       $&$<5.00                 $&$\mathbf{<-8.95}       $&$<-8.96                 $&                   5\\
      IRDC 1916+11 0&$\mathbf{<12.05}       $&$<12.57                 $&$\mathbf{<5.16}       $&$<5.62                 $&$\mathbf{<-8.44}       $&$<-8.46                 $&                   5\\
      IRDC 1916+11 1&$\mathbf{<12.36}       $&$<12.40                 $&$\mathbf{<4.61}       $&$<5.03                 $&$\mathbf{<-8.13}       $&$<-8.16                 $&                   5\\
      IRDC 1916+11 2&$\mathbf{<12.09}       $&$<13.77                 $&$\mathbf{<5.55}       $&$<8.00                 $&$\mathbf{<-8.68}       $&$<-8.68                 $&                   5\\
}{
\tablenotetext{a}{The values used in this paper are shown in boldface.
Uncorrected values are listed in this column.  The filling-factor corrected
values are shown for comparison in the next column even though they were not used for analysis.}
\tablenotetext{b}{The values used in this paper are shown in boldface.
Filling-factor corrected values are listed in this column.  The uncorrected
values are shown for comparison in the previous column even though they not used for analysis.}
\tablenotetext{c}{Flags:\begin{enumerate}
  \item  No filling factor correction (no FFC) is the most reliable.                                             %1: 
  \item  Filling factor correction (FFC) is the most reliable                                                    %2: 
  \item  There is an ambiguity between low density / high abundance and low abundance / high density (no FFC)    %3: 
  \item  There is an ambiguity between low density / high abundance and low abundance / high density (FFC)       %4: 
  \item  Upper Limit (No FFC)                                                                                    %5: 
  \item  Upper Limit (FFC)                                                                                       %6: 
  \item  Lower Limit (No FFC)                                                                                    %7: 
  \item  Lower Limit (FFC)                                                                                       %8: 
  \item  Unreliable estimate because of continuum / filling factor uncertainty.                                  %9: 
  \item  No limit (S/N)                                                                                         %10:
  \item  Optically Thick                                                                                        %11:
\end{enumerate}
}}



\subsection{Green Bank Telescope}
We observed the \formaldehyde\ \twotwo\ line at 2 cm (14.488789 GHz) with the
Green Bank Telescope (GBT)\footnote{ The National Radio Astronomy Observatory operates
the GBT and VLA and is a facility of the National Science Foundation operated
under cooperative agreement by Associated Universities, Inc.  } dual-beam
Ku-band receiver as part of project GBT09C-049.  The GBT dual-beam Ku-band
receiver was used for 4 hours on January 18th, 2010 in beam-switched nodding
mode.  System temperatures ranged from 27 to 38 K in the \formaldehyde\
band centered on the  \twotwo\ line. A bandwidth of 12.5 MHz
(258.8 \kms) and channel width of 3.052 kHz (0.063 \kms) were used with 9-level
sampling, with receiver temperature $\approx21$K.   Three additional tunings
were acquired simultaneously, centered between the H and He 75$\alpha$,
76$\alpha$, and 77$\alpha$ radio recombination lines (RRLs) with the same channel widths and
bandwidths as above at 14.1315, 14.6930, and 15.2846 GHz.  Each source was
observed for 150 seconds in each receiver for a total on-source integration
time of 300 seconds.  Each observation in the pair was independently inspected
to search for emission/absorption in the off position, which was 5.5\arcmin\
away in azimuth.  When absorption was detected in one of the off positions,
that on/off pair was discarded if one of the detected lines was affected, but
otherwise was noted and ignored.  Pointing and focus observations on the
calibrator source 1822-0938 were taken at the start of and two hours into the
observations.  

The gain was assumed to be 1.91 K/Jy based on previous calibration observations
on point sources in Ku-band; our flux density measurements will therefore be
overestimates for extended sources.  The aperture efficiency was
$\eta_{A}=0.671$, and the main beam efficiency $\eta_{MB}=1.32\eta_{A}=0.886$,
so our main-beam corrected measurements could overestimate extended source flux
densities by at most $13\%$ (ignoring atmospheric absorption).
%(based on?  I did not observe flux calibrators.  Maybe from
%\url{https://safe.nrao.edu/wiki/bin/view/GB/Observing/GainPerformance} or 
%\url{https://safe.nrao.edu/wiki/bin/view/GB/PTCS/TSCAL\_090916\#gcurve}).
The data were calibrated using the normal {\sc getnod} procedure in GBTIDL 
\footnote{ GBTIDL (http://gbtidl.nrao.edu/) is the data reduction package
produced by NRAO and written in the IDL language for the reduction of GBT data.
The National Radio Astronomy Observatory is a facility of the National Science
Foundation operated under cooperative agreement by Associated Universities,
Inc.
} ,
which assumes an atmospheric opacity at zenith $\tau=0.014$ at 14.488 GHz. 
%The
%measured opacity ranged from 0.024-0.032 (0.017-0.023 zenith), which was not corrected,
%resulting in a 1-2\% systematic underestimate of source flux densities in the
%table.  All GBT spectra were converted to Jy in order to compare them directly
%with the Arecibo spectra, which were only available in Jy.  

We assume primary beam $\theta_{FWHM}=51.1\arcsec$ %=740/14.48$ 
per the GBT
observers manual.  We assume a conservative 10\% error in the beam area
$\Omega=7.8\times10^{-8}$ sr, which governs the flux density received from the
CMB over the observed area.  Beam size error should be dominated by small errors
in focus.  By utilizing the 305 m Arecibo telescope at 6 cm and the 100 m GBT
at 2 cm, we acquired beam-matched (FWHM$\sim 50\arcsec$) observations of the
\formaldehyde\ \oneone\ and \twotwo\ lines.  



%Question: Continuum accuracy?  Assumed to be 20\%, justified?

\subsubsection{GBT Data Reduction}
In the 24 lines of sight, 75 independent components were identified from the
\oneone\ spectra.  These were fit with gaussians using GBTIDL's {\sc fitgauss}
routine.  Out of these 75 components, 51 had corresponding \twotwo\ detections.
The fitted gaussian spectral lines are listed by line of sight in Table
\ref{tab:h2comeasured_a}.  The gaussian fits may not be representative of the
true spectral line profile; complex spectral line profiles are discussed in Section
\ref{sec:lineprofiles}.
%Some of these (e.g.  G69.54-0.98) are blended or optically thick, so the
%Gaussian fit may not be representative of the true line profile.

The 2 cm continua were measured by fitting a first-order baseline in each
reduced nodded pair excluding the line and the bandpass edges.  Figure
\ref{fig:specexample} shows the flat baselines achieved in the observations,
though the RRL spectrum shows an example of the artifacts seen at the edges of
the bandpass.  The continuum error listed in the table is the RMS of only the
data included in the baseline fit after the baseline was subtracted from the
spectrum; the systematic error from flux calibration uncertainty is 20\% and
dominant.  

\subsection{Arecibo} 
The Arecibo 4.829660 GHz \formaldehyde\ \oneone\ observations used in this
project were previously presented in \citet{Araya2002} and \citet{Araya2004}
and were kindly provided in reduced form by E. Araya.  All observations were performed
using standard on/off position switching and 5 minute integration times in both the 
on and off positions, resulting in off positions 1.25 degrees away from the pointing center.
We assume a 30\% error
in the continuum \citep[based on measured gains in the range 2.0-2.5
as reported in][]{Araya2002} and an effective diameter of 227m ($\theta_{FWHM}
= 56\arcsec$, $\Omega=9.0\times10^{-8}$ sr ) with
10\% uncertainty\footnote{\url{http://www.naic.edu/$\sim$phil/sysperf/misc/hpbw\_vs\_lambda\_2004.html}}.

The Arecibo spectral lines were re-fit for this paper by converting the Arecibo data
from CLASS\footnote{CLASS is part of the GILDAS software developed by IRAM.} to
GBTIDL's {\sc SDFITS} format\footnote{Code for the CLASS-GBTIDL conversion is available from
\url{http://code.google.com/p/casaradio/wiki/class\_to\_gbt}}
and using GBTIDL's {\sc fitgauss} routine.  The 6 cm continua were taken
directly from \citet{Araya2002} Table 3.

\OneColFigure{figures_chH2CO/G32.80+0.19_both}
{ {\it Top:} The GBT \twotwo\ (red) and Arecibo \oneone\ (black) spectra of G32.80+0.19.  
% don't reference figure too early The
% don't reference figure too early fitted density and column to the deepest component at $\sim15 \kms$ is shown in
% don't reference figure too early Figure \ref{fig:fitexample}.  
{\it Bottom:} The GBT H75$\alpha$ (red) and Arecibo H110$\alpha$ (black) spectra with the GRS
\thirteenco\ spectrum (light blue) overlaid.   The left axis is for the RRLs and the right
axis is for the \thirteenco.  The C and He RRLs are not displayed.}
{fig:specexample}
{0.30}{0}

\subsection{Other Archival Data}

\subsubsection{Very Large Array}
We acquired VLA archival images from the Multi-Array Galactic Plane Imaging
Survey (MAGPIS) 6 cm Epoch 3 data set \citep{Becker2005} and the NRAO VLA
Archive Survey (NVAS)\footnote{The NVAS is run by Lourant Sjouwerman at the
NRAO.  It has not yet been published.}.  MAGPIS has a resolution of
$\sim4\arcsec$ and sensitivity $\sigma\sim 2.5 $mJy/bm.  The NVAS has variable
resolution and sensitivity since it is based on VLA archival data.  The VLA
data was used to estimate source sizes and interferometer-to-single-dish flux
ratios.
%[XXXX Need estimates of beam sizes and errors for NVAS data.  Perhaps may be
%better to include one example demonstrating that it's not necessary to include
%a source size correction than to actually include the corrections.]

\subsubsection{ Bolocam 1.1 mm }
We extract 1.1 mm dust continuum fluxes from the Bolocam Galactic Plane Survey
(BGPS) v1.0 release summing over a 25\arcsec\ radius aperture after subtracting
the median in a 50-200\arcsec\ annulus to remove background
contributions. The aperture size is selected to match the 1.1 mm data to the 
2 and 6 cm data.  We assume a uniform 50\% systematic error in BGPS fluxes from
combined uncertainties in the calibration and background subtraction.
\citet{Aguirre2011} contains a complete discussion of the uncertainties in the
BGPS. 
%the measurement uncertainties are only $\sim50\%$, while other
%systematics such as the temperature and opacity can result in factor of 2-4
%changes in the estimated mass and column density.  
We apply the
\citet{Aguirre2011} recommended flux correction of 1.5 and aperture correction
for a 25\arcsec\ aperture of 1.21.  Additionally, data from the Bolocam catalog
\citep{Rosolowsky2010} was used with the flux correction and an aperture correction
of 1.46 for 20\arcsec\ apertures.

\subsubsection{Boston University / Five College Radio Observatory Galactic Ring Survey}
The BU FCRAO GRS \citep{Jackson2006} is a survey of the Galactic plane in the
\thirteenco\ 1-0 line with $\sim 46\arcsec$ resolution.  We extracted spectra in 25\arcsec\ radius
apertures from the publicly available data for comparison with the \formaldehyde\ spectra.

\subsubsection{GLIMPSE}
The Galactic Legacy Infrared Mid-Plane Survey Extraordinaire
\citep[GLIMPSE]{Benjamin2003} maps were used to examine the morphology of the
objects in our survey in order to determine whether an IRDC was present.

\section{Models and Error Estimation}
\label{sec:models}
A grid of large velocity gradient (LVG) models was run using both the RADEX
\citep{VanDerTak2007} code and a proprietary code by \citet{henkel1980} with a 
gradient of 1 \kmspc.  The
models from the two codes were consistent to within $\sim10\%$ in predicted
optical depth and $T_{ex}$.  Both utilized collision
rates from \citet{Green1991} extracted from the LAMDA
database\footnote{\url{http://www.strw.leidenuniv.nl/$\sim$moldata/}} and multiplied by
the recommended factor of 1.6 to account for collisions with \hh\ being more
efficient than He.  The expected accuracy is $\sim30\%$.
\citet{Zeiger2010} demonstrated that the errors in collision rates lead to
systematic errors $\lesssim50\%$  (0.3 dex) in the measured quantities
(N(\formaldehyde), n$(\hh)$).  When measuring density and column, we used the
RADEX models because of their extensively tested code and documentation.  All of the
models used a kinetic temperature of 40 K and covered a range of 500 densities
$\times$ 500 columns logarithmically sampled over $10^1 < n(\hh) < 10^7$ \percc\
and $10^{11} < N(\ortho) < 10^{16} $ \persc.  The assumption $T_K=40$ K is
reasonable in UC\ion{H}{2} regions, which should be warmer than IRDCs and other
cold molecular clouds.  Dust temperatures measured towards UC\ion{H}{2}
regions are around 40 K \citep{Rivera2010}.  In the foreground clouds, this
assumption is less well supported, but as long as the temperatures are higher
than $\sim20$ K, the models change little with temperature (Figure
\ref{fig:modeltau}).

Because of a collisional selection effect, above its critical density
\citep[$n_{cr}(\formaldehyde\ \oneone)\approx 8\ \percc$, \ensuremath{n_{cr}(\formaldehyde\ 
\twotwo) \approx 76\ \percc}, ][]{Mangum2008}
\formaldehyde\ preferentially overpopulates lower states of the K-doublet
\citep[$\Delta$J =0, $\Delta K_a=0$, $\Delta K_c = \pm1$,][]{henkel1980}.  These
spectral lines are cooled to excitation temperatures lower than the CMB and can
therefore be seen in absorption against it.  The \oneone/\twotwo\
absorption line ratio is sensitive to the density of \hh\ at densities
$\gtrsim10^{3.5}\ $ \percc, allowing measurements of the density to within
$\sim$0.3 dex with little sensitivity to gas kinetic temperature
\citep{Mangum2008}.  When density is `measured' with critical density based
tracers such as CO, CS, HCN, or HCO$^+$, the estimate can be off by as much as
2 orders of magnitude because of radiative trapping effects.  Similarly,
measurements of density assuming spherical symmetry can be very far from the
local values.

The collision rates of \formaldehyde\ with \hh\ have been re-derived with a
claimed accuracy of 10\% by \citet{Troscompt2009}.  \citet{Troscompt2009b}
showed that collisions with para-\hh\ are more efficient at cooling
\formaldehyde\ into absorption against the CMB than He or ortho-\hh\ and that
\formaldehyde\ absorption is therefore sensitive to the Ortho/Para ratio of
\hh.   These improved rates are not used in this paper since they are only
computed over a more limited range of temperatures, but may be used in future
works.
% Note to self: point out later that 30% uncertainty has little  to no effect
% on the results of this paper because the interesting results are comparisons
% between different velocities within a single model
% Additional note: Jeff Mangum says the Troscompt work is kinda crap...

% \OneColFigure{Tline0_old_contours}
% {A plot showing the parameter space in which \formaldehyde\ will be seen in 
% absorption or emission.  The contours show where the modeled line temperature
% $T_{line}=0$ assuming illumination only by the CMB or equivalently where the
% line excitation temperature $T_{ex}=T_{CMB}$.  Values of temperature and
% density higher than the displayed curves will be seen in emission (assuming no
% background illumination besides the CMB), while values below the displayed
% curves will be seen in absorption.  The \oneone\ line is shown in solid lines
% from gas kinetic temperature $T_{K}=5$K (thickest) to 55K (thinnest).  The
% \twotwo\ line is shown by dashed (red) lines over the same range; it shows a
% much weaker variation with $T_{K}$.  The \oneone\ line only shows a strong
% dependence on $T_{K}$ below $T_{K} = 10K$, which is approximately the lowest
% temperature expected in molecular clumps.  
% }
% {fig:tline0}
% {0.30}{0}

\OneColFigure{figures_chH2CO/lineratio_vs_n_X9_legend}
{The predicted optical depth ratio ({\it top}) and optical depth ({\it bottom}) vs. volume density
assuming a fixed abundance  $X_{\ortho}=10^{-9}$ \perkmspc\ shows that the
dependence of the derived density on temperature is weak.  At lower abundances,
these curves shift to the right, providing sensitivity to moderately higher
densities.  Our 5-$\sigma$ detection limit is generally around $\tau\sim0.01$.
}
{fig:modeltau}
{0.30}{0}

\subsection{Turbulence}
\label{sec:turbulence}
Molecular gas is often observed to have spectral line widths consistent with
supersonic turbulence \citep{Kainulainen2009} and therefore a lognormal density
distribution \citep{Kritsuk2007}.  Our LVG models assume constant density per
velocity bin, so the resulting models should be smoothed by the probability
distribution function (PDF) of the density.  For clouds with a narrow density
distribution (logarithmic standard deviation of the density $\sigma_s \equiv
\sigma_{ln(\rho)/ln(\bar{\rho})}\lesssim0.5$)\footnote{We use $\rho$ to
indicate number density in this section in order to be consistent with the
cited literature.  Because the widths are relative to a mean density, the
scaling between mass and number density is unimportant.}, the effect of
smoothing is smaller than other systematic errors, but for more turbulent
clouds the density PDF width can exceed an order of magnitude \citep[e.g.,
][]{Federrath2010} and will substantially change the derived density.  Because
the Mach numbers of the turbulence in the observed clouds are unconstrained, we
cannot correct for this added uncertainty.  The change in measured density is
$|\Delta\log(\rho)|<0.25$ for $\sigma_s \le 0.5$, with a slight bias towards
higher densities at lower optical depth ratio $\tau_{\oneone}/\tau_{\twotwo}$
(Figure \ref{fig:turbtaurat}).  However, for $\sigma_s=1.5$, the bias exceeds
an order of magnitude at some densities.

Additionally, we consider the effects of ``gravoturbulence'', in which a
high-density tail inconsistent with a lognormal distribution is observed.
\citet{Kainulainen2009} report column density distributions derived from 2MASS
extinction measurements that can be used as a proxy for the density
distribution for a wide variety of clouds.  Non-star-forming clouds retain a
lognormal distribution and are consistent with the analysis presented above.
However, evolved star-forming regions develop a high-column density tail.  For
evolved (actively star-forming) regions like Ophiucus, Orion, and Perseus,
the high-column density tail is substantial, and \formaldehyde\ density measurements
will be highly biased towards the highest density gas.  More quiescent regions
like the Pipe and Coalsack nebulae are consistent with a lognormal column
distribution to a degree that the high-column density tail would not affect
\formaldehyde\ density measurements significantly.

\OneColFigure{figures_chH2CO/lognormalsmooth_density_ratio_massweight} 
{
The optical depth ratio as a function of density for turbulent density
distributions with widths specified in the legend.  The optical depth ratio
varies more slowly with density than in the pure LVG model (the solid line is
the same as the black 10 K line in Figure \ref{fig:modeltau}a).}
{fig:turbtaurat}{0.30}{0}

To demonstrate the effects of turbulent distributions, we calculate the optical
depth ratio as a function of the mean density for three turbulent widths in
Figure \ref{fig:turbtaurat}.  We compare the density that would be inferred
from the spectral line ratio assuming no turbulence (just LVG) to the `correct'
density including turbulent effects in Figure \ref{fig:turbcorr}.  We have also
compared the LVG and turbulent densities to ``gravoturbulent'' density
distributions, in which a power law tail of high-density gas begins at about
$10^{-2}$ times the peak density \citep[e.g., ][]{Klessen2000,KimCho2011}, but
because the density distributions in these simulations are relatively narrow,
the effects of the high-density tail on the measured density are negligible
except for the most turbulent cases.

Figure \ref{fig:turbtaurat} is meant to demonstrate the effects of turbulence,
but it is \emph{not} used to derive densities, since the true density
distribution in observed clouds is unknown.  However, future measurements of
the density distribution can be used to apply the `correction' shown in Figure
\ref{fig:turbcorr}.

%In
%the gravoturbulent case, we represent the high-density tail as a power-law
%starting at $n=100n_0$ with a slope $\alpha=1.3$, which is drawn directly from the
%plots in \citet{KimCho2011}; the slope of the measured column distributions in \citet{Kainulainen2009}
%is similar, but starts nearly at $N=N_0$. The plotted correction should be a reasonable
%approximation of a GMC that has survived for a few free-fall times.

\OneColFigure{figures_chH2CO/TurbulenceCorrection}
{ The mean density from a lognormal density distribution plotted against the
density derived assuming a single density per region (i.e., the directly
LVG-derived density).   At low densities, the wider turbulent
distributions  are heavily biased towards ``observing'' higher densities than
the true mean density.  The distributions cut off at the low end where the
optical depth ratio becomes a double-valued function of density; at these low
densities, no detections are expected at our survey's sensitivity.  The cutoff
at the high end is where the optical depth ratio becomes constant.  
}
{fig:turbcorr}{0.30}{0}

\section{Analysis}
\label{sec:analysis}

%\Table{lcccccccc}{Measured \formaldehyde\ \oneone\ line properties}
{\colhead{Source Name\tablenotemark{a}}&\colhead{$l$}&\colhead{$b$}&\colhead{6cm Continuum}&\colhead{Peak}&\colhead{Center}&\colhead{FWHM }&\colhead{RMS}&\colhead{Channel Width}\\
\colhead{           }&\colhead{\degrees}&\colhead{\degrees}&\colhead{(Jy)}&\colhead{(Jy)}&\colhead{(\kms)}&\colhead{(\kms)}&\colhead{(Jy)}&\colhead{(\kms)}\\ }
{tab:h2comeasured_a}{
       G32.80+0.19 0&              0.1904&             32.7968&         2.18 (0.01)&      -0.393 (0.008)&        15.39 (0.05)&         6.57 (0.06)&              0.0049&              1.1374\\
       G32.80+0.19 1&              0.1904&             32.7968&         2.18 (0.01)&      -0.092 (0.008)&        11.45 (0.26)&        10.25 (0.65)&              0.0049&              1.1374\\
       G32.80+0.19 2&              0.1904&             32.7968&         2.18 (0.01)&      -0.063 (0.008)&        80.63 (0.13)&         2.49 (0.36)&              0.0049&              1.1374\\
       G32.80+0.19 3&              0.1904&             32.7968&         2.18 (0.01)&      -0.254 (0.008)&        84.61 (0.02)&         1.37 (0.06)&              0.0049&              1.1374\\
       G32.80+0.19 4&              0.1904&             32.7968&         2.18 (0.01)&      -0.090 (0.008)&        88.66 (0.09)&         3.21 (0.31)&              0.0049&              1.1374\\
       G33.13-0.09 0&             -0.0949&             33.1297&         0.49 (0.00)&      -0.192 (0.007)&        75.92 (0.05)&         3.80 (0.12)&              0.0045&              1.1374\\
       G33.13-0.09 1&             -0.0949&             33.1297&         0.49 (0.00)&      -0.023 (0.007)&        81.62 (0.35)&         2.49 (0.88)&              0.0045&              1.1374\\
       G33.13-0.09 2&             -0.0949&             33.1297&         0.49 (0.00)&      -0.040 (0.007)&       101.50 (0.40)&        11.30 (0.80)&              0.0045&              1.1374\\
       G33.13-0.09 3&             -0.0949&             33.1297&         0.49 (0.00)&      -0.039 (0.007)&        10.39 (0.08)&         2.04 (0.24)&              0.0045&              1.1374\\
       G33.92+0.11 0&              0.1112&              33.914&         0.83 (0.00)&      -0.081 (0.008)&       107.28 (0.18)&         6.62 (0.34)&               0.005&              1.1374\\
       G33.92+0.11 1&              0.1112&              33.914&         0.83 (0.00)&      -0.079 (0.008)&       106.03 (0.06)&         2.41 (0.23)&               0.005&              1.1374\\
       G33.92+0.11 2&              0.1112&              33.914&         0.83 (0.00)&      -0.160 (0.030)&        57.30 (0.40)&        10.60 (0.80)&               0.005&              1.1374\\
       G34.26+0.15 0&              0.1538&             34.2572&         5.57 (0.01)&      -1.828 (0.015)&        60.24 (0.01)&         3.80 (0.03)&              0.0063&              1.1374\\
       G34.26+0.15 1&              0.1538&             34.2572&         5.57 (0.01)&      -0.160 (0.015)&        26.69 (0.08)&         1.04 (0.22)&              0.0063&              1.1374\\
       G34.26+0.15 2&              0.1538&             34.2572&         5.57 (0.01)&      -0.099 (0.015)&        11.25 (0.19)&         2.01 (0.40)&              0.0063&              1.1374\\
       G34.26+0.15 3&              0.1538&             34.2572&         5.57 (0.01)&      -0.126 (0.015)&        51.70 (2.00)&         4.20 (1.00)&              0.0063&              1.1374\\
       G34.26+0.15 4&              0.1538&             34.2572&         5.57 (0.01)&      -0.047 (0.015)&        48.20 (2.00)&         1.80 (1.00)&              0.0063&              1.1374\\
       G35.20-1.74 0&             -1.7409&             35.1997&         5.17 (0.00)&      -1.018 (0.008)&        43.37 (0.01)&         3.67 (0.02)&              0.0051&              1.1374\\
       G35.20-1.74 1&             -1.7409&             35.1997&         5.17 (0.00)&      -0.147 (0.008)&        36.67 (0.10)&         1.49 (0.27)&              0.0051&              1.1374\\
       G35.20-1.74 2&             -1.7409&             35.1997&         5.17 (0.00)&      -0.324 (0.008)&        14.08 (0.01)&         0.93 (0.03)&              0.0051&              1.1374\\
       G35.20-1.74 3&             -1.7409&             35.1997&         5.17 (0.00)&      -0.039 (0.008)&        50.59 (0.53)&         4.92 (1.31)&              0.0051&              1.1374\\
       G35.57-0.03 0&             -0.0306&             35.5779&         0.47 (0.00)&      -0.064 (0.009)&        52.10 (0.10)&         4.60 (0.30)&              0.0053&              1.1374\\
       G35.57-0.03 1&             -0.0306&             35.5779&         0.47 (0.00)&      -0.021 (0.009)&        45.60 (0.30)&         1.90 (0.60)&              0.0053&              1.1374\\
       G35.57-0.03 2&             -0.0306&             35.5779&         0.47 (0.00)&      -0.019 (0.009)&        57.60 (0.50)&         2.90 (0.97)&              0.0053&              1.1374\\
       G35.57-0.03 3&             -0.0306&             35.5779&         0.47 (0.00)&      -0.031 (0.009)&        12.80 (0.20)&         1.84 (0.41)&              0.0053&              1.1374\\
       G35.57-0.03 4&             -0.0306&             35.5779&         0.47 (0.00)&      -0.031 (0.008)&        29.04 (0.11)&         0.82 (0.25)&              0.0053&              1.1374\\
       G35.58+0.07 0&              0.0657&             35.5801&         0.53 (0.01)&      -0.146 (0.004)&        49.37 (0.21)&         5.33 (0.34)&              0.0048&              1.1374\\
       G35.58+0.07 1&              0.0657&             35.5801&         0.53 (0.01)&      -0.049 (0.013)&        53.13 (0.25)&         2.98 (0.64)&              0.0048&              1.1374\\
       G35.58+0.07 2&              0.0657&             35.5801&         0.53 (0.01)&      -0.025 (0.004)&        58.12 (0.29)&         3.63 (0.74)&              0.0048&              1.1374\\
       G35.58+0.07 3&              0.0657&             35.5801&         0.53 (0.01)&      -0.034 (0.004)&        13.24 (0.17)&         2.80 (0.39)&              0.0048&              1.1374\\
       G37.87-0.40 0&             -0.3993&              37.873&         4.40 (0.01)&      -0.531 (0.006)&        60.23 (0.11)&         8.73 (0.35)&              0.0069&              1.1374\\
       G37.87-0.40 1&             -0.3993&              37.873&         4.40 (0.01)&      -0.124 (0.014)&        53.27 (0.19)&         4.03 (0.46)&              0.0069&              1.1374\\
       G37.87-0.40 2&             -0.3993&              37.873&         4.40 (0.01)&      -0.356 (0.019)&        65.13 (0.04)&         2.74 (0.15)&              0.0069&              1.1374\\
       G37.87-0.40 3&             -0.3993&              37.873&         4.40 (0.01)&      -0.324 (0.045)&        72.18 (0.04)&         1.35 (0.14)&              0.0069&              1.1374\\
       G37.87-0.40 4&             -0.3993&              37.873&         4.40 (0.01)&      -0.424 (0.013)&        73.97 (0.13)&         3.01 (0.22)&              0.0069&              1.1374\\
       G37.87-0.40 5&             -0.3993&              37.873&         4.40 (0.01)&      -0.185 (0.012)&        79.98 (0.06)&         1.80 (0.14)&              0.0069&              1.1374\\
       G37.87-0.40 6&             -0.3993&              37.873&         4.40 (0.01)&      -0.114 (0.015)&        91.96 (0.08)&         1.21 (0.18)&              0.0069&              1.1374\\
       G37.87-0.40 7&             -0.3993&              37.873&         4.40 (0.01)&      -0.175 (0.012)&        14.32 (0.14)&         2.94 (0.20)&              0.0069&              1.1374\\
       G37.87-0.40 8&             -0.3993&              37.873&         4.40 (0.01)&      -0.072 (0.022)&        13.16 (0.10)&         0.87 (0.32)&              0.0069&              1.1374\\
       G37.87-0.40 9&             -0.3993&              37.873&         4.40 (0.01)&      -0.137 (0.012)&        20.54 (0.06)&         1.37 (0.14)&              0.0069&              1.1374\\
       G43.89-0.78 0&             -0.7838&             43.8892&         0.66 (0.00)&      -0.181 (0.004)&        54.86 (0.02)&         2.19 (0.06)&              0.0032&              1.1374\\
       G43.89-0.78 1&             -0.7838&             43.8892&         0.66 (0.00)&      -0.020 (0.002)&        50.55 (0.59)&        15.90 (1.20)&              0.0032&              1.1374\\
       G45.07+0.13 0&              0.1323&             45.0711&         0.47 (0.00)&      -0.056 (0.006)&        57.49 (0.10)&         4.24 (0.23)&              0.0035&              1.1374\\
       G45.07+0.13 1&              0.1323&             45.0711&         0.47 (0.00)&      -0.036 (0.006)&        65.44 (0.15)&         4.09 (0.34)&              0.0035&              1.1374\\
       G45.12+0.13 0&              0.1326&             45.1223&         4.28 (0.01)&      -0.188 (0.006)&        55.70 (0.12)&         3.32 (0.24)&              0.0065&              1.1374\\
       G45.12+0.13 1&              0.1326&             45.1223&         4.28 (0.01)&      -0.154 (0.009)&        59.40 (0.13)&         3.11 (0.33)&              0.0065&              1.1374\\
       G45.12+0.13 2&              0.1326&             45.1223&         4.28 (0.01)&      -0.200 (0.010)&        24.86 (0.03)&         1.68 (0.08)&              0.0065&              1.1374\\
       G45.12+0.13 3&              0.1326&             45.1223&         4.28 (0.01)&      -0.027 (0.004)&        65.53 (0.82)&         7.23 (2.03)&              0.0065&              1.1374\\
       G45.45+0.06 0&              0.0593&             45.4548&         4.77 (0.01)&      -1.347 (0.018)&        59.58 (0.02)&         3.18 (0.05)&              0.0063&              1.1374\\
       G45.45+0.06 1&              0.0593&             45.4548&         4.77 (0.01)&      -0.123 (0.040)&        55.34 (0.38)&         3.15 (0.38)&              0.0063&              1.1374\\
       G45.45+0.06 2&              0.0593&             45.4548&         4.77 (0.01)&      -0.056 (0.005)&        25.02 (0.12)&         2.82 (0.28)&              0.0063&              1.1374\\
       G45.47+0.05 0&              0.0455&             45.4655&         0.75 (0.00)&      -0.274 (0.003)&        60.62 (0.03)&         6.59 (0.07)&              0.0039&              1.1374\\
       G45.47+0.05 1&              0.0455&             45.4655&         0.75 (0.00)&      -0.017 (0.004)&        25.55 (0.23)&         2.18 (0.55)&              0.0039&              1.1374\\
       G48.61+0.02 0&              0.0229&             48.6055&         1.01 (0.00)&      -0.067 (0.003)&        18.08 (0.09)&         4.97 (0.22)&              0.0035&              1.1374\\
       G48.61+0.02 1&              0.0229&             48.6055&         1.01 (0.00)&      -0.024 (0.005)&         6.08 (0.13)&         1.20 (0.31)&              0.0035&              1.1374\\
       G48.61+0.02 2&              0.0229&             48.6055&         1.01 (0.00)&      -0.018 (0.003)&        53.73 (0.33)&         4.72 (0.79)&              0.0035&              1.1374\\
       G50.32+0.68 0&              0.6761&             50.3153&         0.24 (0.00)&      -0.011 (0.003)&        26.28 (0.40)&         3.32 (0.94)&              0.0031&              1.1374\\
       G60.88-0.13 0&             -0.1285&             60.8826&         0.66 (0.01)&      -0.093 (0.009)&        22.60 (0.15)&         3.24 (0.35)&              0.0096&              1.1374\\
       G61.48+0.09 0&              0.0893&             61.4769&         6.16 (0.01)&      -0.531 (0.009)&        21.45 (0.02)&         2.81 (0.06)&              0.0084&              1.1374\\
       G69.54-0.98 0&             -0.9759&             69.5398&         0.28 (0.01)&      -0.280 (0.006)&        10.65 (0.05)&         4.55 (0.11)&              0.0076&              1.1374\\
       G70.29+1.60 0&              1.6006&             70.2927&         4.37 (0.13)&      -0.372 (0.008)&       -21.74 (0.07)&         3.92 (0.15)&              0.0108&              1.1374\\
       G70.29+1.60 1&              1.6006&             70.2927&         4.37 (0.13)&      -0.050 (0.007)&       -27.17 (0.58)&         4.86 (1.33)&              0.0108&              1.1374\\
       G70.33+1.59 0&               1.589&             70.3296&         2.21 (0.01)&      -1.201 (0.007)&       -21.24 (0.01)&         3.65 (0.03)&              0.0115&              1.1374\\
   IRAS 20051+3435 0&              0.2088&             32.4662&         0.00 (0.01)&      -0.019 (0.001)&        10.77 (0.07)&         3.60 (0.18)&             0.00071&              2.2747\\
       G41.74+0.10 0&              0.0975&             41.7415&         0.34 (0.00)&      -0.062 (0.004)&        14.60 (0.09)&         2.56 (0.26)&              0.0033&              1.1374\\
       G41.74+0.10 1&              0.0975&             41.7415&         0.34 (0.00)&      -0.020 (0.004)&        10.99 (0.29)&         2.52 (0.71)&              0.0033&              1.1374\\
       G41.74+0.10 2&              0.0975&             41.7415&         0.34 (0.00)&      -0.066 (0.004)&        34.25 (0.05)&         1.63 (0.13)&              0.0033&              1.1374\\
       G41.74+0.10 3&              0.0975&             41.7415&         0.34 (0.00)&      -0.022 (0.005)&        56.61 (0.13)&         1.15 (0.32)&              0.0033&              1.1374\\
       G41.74+0.10 4&              0.0975&             41.7415&         0.34 (0.00)&      -0.043 (0.005)&        17.57 (0.07)&         1.13 (0.18)&              0.0033&              1.1374\\
      IRDC 1923+13 0&             -0.4972&             48.9325&         0.40 (0.00)&      -0.011 (0.001)&        50.20 (0.08)&         1.83 (0.19)&              0.0008&              0.7582\\
      IRDC 1923+13 1&             -0.4972&             48.9325&         0.40 (0.00)&      -0.009 (0.001)&        57.56 (0.09)&         2.57 (0.22)&              0.0008&              0.7582\\
      IRDC 1923+13 2&             -0.4972&             48.9325&         0.40 (0.00)&      -0.005 (0.001)&        47.32 (0.20)&         2.11 (0.51)&              0.0008&              0.7582\\
      IRDC 1916+11 0&             -0.2923&              45.666&         0.00 (0.01)&      -0.005 (0.001)&        25.94 (0.17)&         2.53 (0.41)&             0.00083&              0.7582\\
      IRDC 1916+11 1&             -0.2923&              45.666&         0.00 (0.01)&      -0.013 (0.001)&        55.91 (0.13)&         6.21 (0.34)&             0.00083&              0.7582\\
      IRDC 1916+11 2&             -0.2923&              45.666&         0.00 (0.01)&      -0.003 (0.001)&        48.85 (0.48)&         3.58 (1.13)&             0.00083&              0.7582\\
}{
\tablenotetext{a}{Sources are labeled by the line-of-sight followed by the
number of the component identified, indexed from zero.  The components do not
follow a particular order, but are uniquely identifiable by their velocity,
width, and amplitude.}}

%\Table{lccccc}{Measured \formaldehyde\ \twotwo\ line properties}
{\colhead{Source Name}&\colhead{2cm Continuum}&\colhead{Peak\tablenotemark{a}}&\colhead{Center}&\colhead{FWHM }&\colhead{RMS\tablenotemark{b}}\\
\colhead{           }&\colhead{(Jy)}&\colhead{(Jy)}&\colhead{(\kms)}&\colhead{(\kms)}&\colhead{(Jy)}\\ }
{tab:h2comeasured_b}{
       G32.80+0.19 0&         3.68 (0.02)&      -0.519 (0.032)&        15.65 (0.03)&         5.72 (0.08)&              0.0038\\
       G32.80+0.19 1&         3.68 (0.02)&      -0.076 (0.019)&        11.90 (1.18)&         8.17 (0.98)&              0.0038\\
       G32.80+0.19 2&         3.68 (0.02)&      -0.016 (0.001)&        80.47 (0.14)&         4.35 (0.36)&              0.0038\\
       G32.80+0.19 3&         3.68 (0.02)&      -0.065 (0.002)&        84.96 (0.02)&         1.29 (0.05)&              0.0038\\
       G32.80+0.19 4&         3.68 (0.02)&      -0.026 (0.001)&        88.83 (0.06)&         2.31 (0.14)&              0.0038\\
       G33.13-0.09 0&         0.47 (0.02)&      -0.224 (0.003)&        76.17 (0.02)&         3.31 (0.05)&               0.003\\
       G33.13-0.09 1&         0.47 (0.02)&       0.000 (0.000)&         0.00 (0.00)&         0.00 (0.00)&               0.003\\
       G33.13-0.09 2&         0.47 (0.02)&       0.000 (0.000)&         0.00 (0.00)&         0.00 (0.00)&               0.003\\
       G33.13-0.09 3&         0.47 (0.02)&       0.000 (0.000)&         0.00 (0.00)&         0.00 (0.00)&               0.003\\
       G33.92+0.11 0&         0.87 (0.02)&      -0.086 (0.003)&       106.43 (0.03)&         2.17 (0.09)&              0.0032\\
       G33.92+0.11 1&         0.87 (0.02)&      -0.069 (0.002)&       108.83 (0.11)&         6.82 (0.16)&              0.0032\\
       G33.92+0.11 2&         0.87 (0.02)&       0.000 (0.000)&         0.00 (0.00)&         0.00 (0.00)&              0.0032\\
       G34.26+0.15 0&         5.89 (0.02)&      -1.356 (0.006)&        60.99 (0.01)&         3.96 (0.02)&              0.0051\\
       G34.26+0.15 1&         5.89 (0.02)&      -0.046 (0.003)&        27.11 (0.04)&         1.03 (0.09)&              0.0051\\
       G34.26+0.15 2&         5.89 (0.02)&      -0.018 (0.002)&        11.23 (0.16)&         3.19 (0.38)&              0.0051\\
       G34.26+0.15 3&         5.89 (0.02)&      -0.025 (0.004)&        52.82 (0.58)&         6.34 (1.53)&              0.0051\\
       G34.26+0.15 4&         5.89 (0.02)&      -0.018 (0.007)&        47.05 (0.47)&         2.47 (1.15)&              0.0051\\
       G35.20-1.74 0&         5.98 (0.03)&      -0.482 (0.004)&        43.38 (0.02)&         3.71 (0.04)&              0.0055\\
       G35.20-1.74 1&         5.98 (0.03)&      -0.028 (0.005)&        37.91 (0.32)&         3.46 (0.76)&              0.0055\\
       G35.20-1.74 2&         5.98 (0.03)&      -0.056 (0.003)&        14.18 (0.02)&         1.00 (0.05)&              0.0055\\
       G35.20-1.74 3&         5.98 (0.03)&       0.000 (0.000)&         0.00 (0.00)&         0.00 (0.00)&              0.0055\\
       G35.57-0.03 0&         0.32 (0.15)&      -0.075 (0.003)&        52.14 (0.09)&         4.39 (0.21)&              0.0046\\
       G35.57-0.03 1&         0.32 (0.15)&      -0.015 (0.006)&        47.39 (0.25)&         1.31 (0.60)&              0.0046\\
       G35.57-0.03 2&         0.32 (0.15)&       0.000 (0.000)&         0.00 (0.00)&         0.00 (0.00)&              0.0046\\
       G35.57-0.03 3&         0.32 (0.15)&       0.000 (0.000)&         0.00 (0.00)&         0.00 (0.00)&              0.0046\\
       G35.57-0.03 4&         0.32 (0.15)&      -0.024 (0.008)&        29.25 (0.11)&         0.43 (0.15)&              0.0046\\
       G35.58+0.07 0&         0.23 (0.09)&      -0.106 (0.002)&        49.21 (0.06)&         5.00 (0.14)&              0.0031\\
       G35.58+0.07 1&         0.23 (0.09)&       0.000 (0.004)&         0.00 (0.00)&         0.00 (0.00)&              0.0031\\
       G35.58+0.07 2&         0.23 (0.09)&       0.000 (0.004)&         0.00 (0.00)&         0.00 (0.00)&              0.0031\\
       G35.58+0.07 3&         0.23 (0.09)&       0.000 (0.004)&         0.00 (0.00)&         0.00 (0.00)&              0.0031\\
       G37.87-0.40 0&         3.73 (0.02)&      -0.221 (0.003)&        59.99 (0.12)&         8.53 (0.14)&              0.0048\\
       G37.87-0.40 1&         3.73 (0.02)&      -0.045 (0.007)&        54.55 (0.25)&         5.99 (0.34)&              0.0048\\
       G37.87-0.40 2&         3.73 (0.02)&      -0.036 (0.007)&        65.06 (0.11)&         2.57 (0.45)&              0.0048\\
       G37.87-0.40 3&         3.73 (0.02)&      -0.053 (0.003)&        72.44 (0.05)&         1.37 (0.08)&              0.0048\\
       G37.87-0.40 4&         3.73 (0.02)&      -0.047 (0.002)&        74.25 (0.07)&         2.07 (0.18)&              0.0048\\
       G37.87-0.40 5&         3.73 (0.02)&      -0.016 (0.001)&        80.04 (0.03)&         1.28 (0.07)&              0.0048\\
       G37.87-0.40 6&         3.73 (0.02)&      -0.010 (0.002)&        91.99 (0.12)&         1.60 (0.28)&              0.0048\\
       G37.87-0.40 7&         3.73 (0.02)&      -0.026 (0.002)&        14.89 (0.12)&         1.40 (0.20)&              0.0048\\
       G37.87-0.40 8&         3.73 (0.02)&      -0.017 (0.002)&        13.29 (0.19)&         1.52 (0.34)&              0.0048\\
       G37.87-0.40 9&         3.73 (0.02)&      -0.017 (0.001)&        20.52 (0.10)&         3.09 (0.23)&              0.0048\\
       G43.89-0.78 0&         0.53 (0.02)&      -0.059 (0.004)&        54.61 (0.08)&         2.85 (0.23)&               0.003\\
       G43.89-0.78 1&         0.53 (0.02)&      -0.015 (0.002)&        49.59 (0.94)&        14.49 (1.69)&               0.003\\
       G45.07+0.13 0&         0.79 (0.07)&      -0.073 (0.003)&        57.18 (0.08)&         3.45 (0.18)&              0.0029\\
       G45.07+0.13 1&         0.79 (0.07)&      -0.011 (0.003)&        65.67 (0.42)&         3.46 (0.98)&              0.0029\\
       G45.12+0.13 0&         5.20 (0.20)&      -0.086 (0.002)&        56.21 (0.11)&         5.22 (0.21)&              0.0044\\
       G45.12+0.13 1&         5.20 (0.20)&      -0.059 (0.005)&        59.70 (0.06)&         2.42 (0.16)&              0.0044\\
       G45.12+0.13 2&         5.20 (0.20)&      -0.047 (0.002)&        25.14 (0.04)&         1.55 (0.09)&              0.0044\\
       G45.12+0.13 3&         5.20 (0.20)&      -0.021 (0.001)&        64.68 (0.39)&         8.15 (0.87)&              0.0044\\
       G45.45+0.06 0&         3.16 (0.02)&      -0.260 (0.003)&        59.58 (0.01)&         2.06 (0.03)&              0.0043\\
       G45.45+0.06 1&         3.16 (0.02)&      -0.042 (0.002)&        57.90 (0.14)&         9.40 (0.31)&              0.0043\\
       G45.45+0.06 2&         3.16 (0.02)&       0.000 (0.000)&         0.00 (0.00)&         0.00 (0.00)&              0.0043\\
       G45.47+0.05 0&         0.38 (0.02)&      -0.124 (0.003)&        61.67 (0.07)&         5.85 (0.17)&              0.0049\\
       G45.47+0.05 1&         0.38 (0.02)&      -0.000 (0.007)&         0.00 (0.00)&         0.00 (0.00)&              0.0049\\
       G48.61+0.02 0&         0.41 (0.02)&      -0.022 (0.003)&        18.50 (0.25)&         4.39 (0.59)&              0.0033\\
       G48.61+0.02 1&         0.41 (0.02)&      -0.000 (0.000)&         0.00 (0.00)&         0.00 (0.00)&              0.0033\\
       G48.61+0.02 2&         0.41 (0.02)&      -0.005 (0.002)&        52.50 (1.25)&         7.47 (2.94)&              0.0033\\
       G50.32+0.68 0&         0.16 (0.02)&      -0.011 (0.003)&        26.21 (0.44)&         3.10 (1.03)&              0.0036\\
       G60.88-0.13 0&         0.29 (0.02)&      -0.016 (0.003)&        21.63 (0.21)&         2.47 (0.50)&               0.003\\
       G61.48+0.09 0&         3.42 (0.02)&      -0.300 (0.004)&        21.40 (0.02)&         2.39 (0.04)&              0.0037\\
       G69.54-0.98 0&         0.23 (0.02)&      -0.220 (0.002)&         9.97 (0.03)&         5.81 (0.08)&              0.0031\\
       G70.29+1.60 0&         6.21 (0.02)&      -0.159 (0.003)&       -23.52 (0.06)&         5.36 (0.13)&              0.0046\\
       G70.29+1.60 1&         6.21 (0.02)&      -0.000 (0.000)&        -0.00 (0.00)&         0.00 (0.00)&              0.0046\\
       G70.33+1.59 0&         2.68 (0.02)&      -1.081 (0.005)&       -21.17 (0.01)&         2.95 (0.01)&              0.0038\\
   IRAS 20051+3435 0&         0.00 (0.02)&      -0.016 (0.003)&        11.51 (0.37)&         4.14 (0.88)&              0.0032\\
       G41.74+0.10 0&         0.28 (0.02)&      -0.014 (0.002)&        14.36 (0.34)&         3.80 (0.80)&              0.0032\\
       G41.74+0.10 1&         0.28 (0.02)&       0.000 (0.004)&         0.00 (0.00)&         0.00 (0.00)&              0.0032\\
       G41.74+0.10 2&         0.28 (0.02)&       0.000 (0.004)&         0.00 (0.00)&         0.00 (0.00)&              0.0032\\
       G41.74+0.10 3&         0.28 (0.02)&       0.000 (0.004)&         0.00 (0.00)&         0.00 (0.00)&              0.0032\\
       G41.74+0.10 4&         0.28 (0.02)&       0.000 (0.004)&         0.00 (0.00)&         0.00 (0.00)&              0.0032\\
      IRDC 1923+13 0&         0.00 (0.02)&       0.000 (0.000)&         0.00 (0.00)&         0.00 (0.00)&              0.0032\\
      IRDC 1923+13 1&         0.00 (0.02)&       0.000 (0.000)&         0.00 (0.00)&         0.00 (0.00)&              0.0032\\
      IRDC 1923+13 2&         0.00 (0.02)&       0.000 (0.000)&         0.00 (0.00)&         0.00 (0.00)&              0.0032\\
      IRDC 1916+11 0&         0.00 (0.02)&       0.000 (0.000)&         0.00 (0.00)&         0.00 (0.00)&              0.0048\\
      IRDC 1916+11 1&         0.00 (0.02)&       0.000 (0.000)&         0.00 (0.00)&         0.00 (0.00)&              0.0048\\
      IRDC 1916+11 2&         0.00 (0.02)&       0.000 (0.000)&         0.00 (0.00)&         0.00 (0.00)&              0.0048\\
}{
\tablenotetext{a}{ The Upper Limit Flag is 1 when the measurement indicated is
a $3-\sigma$ upper limit on the \twotwo\ line depth when there is a
corresponding \oneone\ line detection. }
\tablenotetext{b}{RMS in 1.011 \kms\ channels.} 
}

%\Table{lccccccc}{Distance, BGPS 1.1 mm, and other properties}
{\colhead{Source Name}&\colhead{Distance}&\colhead{Galactocentric}&\colhead{KDA\tablenotemark{a}}&\colhead{$S_{1.1mm}$}&\colhead{Source}&\colhead{\formaldehyde\ }&\colhead{Scenario\tablenotemark{b}}\\
\colhead{           }&\colhead{        }&\colhead{Distance      }&\colhead{Resolution}&\colhead{           }&\colhead{Type  }&\colhead{Spectrum}&\colhead{}\\  
\colhead{           }&\colhead{(kpc)   }&\colhead{         (kpc)}&\colhead{          }&\colhead{(Jy)       }&\colhead{      }&\colhead{Type    }&\colhead{}\\ }
{tab:other}{
       G32.80+0.19 0&                12.9&                 7.4&                 far&                6.94&               UCHII&        red gradient&                 2+3\\
       G32.80+0.19 1&                13.1&                 7.6&                 far&                6.94&               UCHII&            envelope&                 2+3\\
       G32.80+0.19 2&                 9.4&                 5.1&                 far&                6.94&                 GMC&                   -&                 2+3\\
       G32.80+0.19 3&                 9.2&                 5.0&                 far&                6.94&                 GMC&                   -&                 2+3\\
       G32.80+0.19 4&                 9.0&                 4.9&                 far&                6.94&                 GMC&                   -&                 2+3\\
       G33.13-0.09 0&                 9.6&                 5.2&                 far&                2.26&               UCHII&        red gradient&                   2\\
       G33.13-0.09 1&                 9.3&                 5.1&                 far&                2.26&                 GMC&            envelope&                   2\\
       G33.13-0.09 2&                 7.1&                 4.7&             tangent&                2.26&                 GMC&                   -&                   2\\
       G33.13-0.09 3&                 0.9&                 7.6&                near&                2.26&                 GMC&                   -&                   2\\
       G33.92+0.11 0&                 7.0&                 4.6&             tangent&                3.86&               UCHII&        red gradient&                   2\\
       G33.92+0.11 1&                 7.0&                 4.6&             tangent&                3.86&               UCHII&            envelope&                   2\\
       G33.92+0.11 2&                 3.6&                 5.8&                near&                3.86&                 GMC&                   -&                   2\\
       G34.26+0.15 0&                 3.6&                 5.7&                near&               35.69&               UCHII&        red gradient&                   2\\
       G34.26+0.15 1&                 1.9&                 6.9&                near&               35.69&                 GMC&                   -&                   2\\
       G34.26+0.15 2&                 1.0&                 7.6&                near&               35.69&                 GMC&                   -&                   2\\
       G34.26+0.15 3&                 3.6&                 6.0&                near&               35.69&                 GMC&            envelope&                   2\\
       G34.26+0.15 4&                 3.6&                 6.1&                near&               35.69&                 GMC&                   -&                   2\\
       G35.20-1.74 0&                 2.8&                 6.3&                near&                   -&               UCHII&              single&                   4\\
       G35.20-1.74 1&                 2.5&                 6.5&                near&                   -&                 GMC&                   -&                   4\\
       G35.20-1.74 2&                 1.1&                 7.5&                near&                   -&                 GMC&                   -&                   4\\
       G35.20-1.74 3&                 3.2&                 6.1&                near&                   -&                 GMC&                   -&                   4\\
       G35.57-0.03 0&                10.3&                 6.0&                 far&                2.57&               UCHII&              single&                 2+3\\
       G35.57-0.03 1&                10.7&                 6.2&                 far&                2.57&                 GMC&                   -&                 2+3\\
       G35.57-0.03 2&                 3.6&                 5.9&                near&                2.57&                 GMC&                   -&                 2+3\\
       G35.57-0.03 3&                 1.1&                 7.6&                near&                2.57&                 GMC&                   -&                 2+3\\
       G35.57-0.03 4&                 2.0&                 6.8&                near&                2.57&                 GMC&                   -&                 2+3\\
       G35.58+0.07 0&                10.5&                 6.1&                 far&                1.44&               UCHII&       blue gradient&                   2\\
       G35.58+0.07 1&                10.3&                 6.0&                 far&                1.44&               UCHII&                   -&                   2\\
       G35.58+0.07 2&                 3.6&                 5.8&                near&                1.44&                 GMC&                   -&                   2\\
       G35.58+0.07 3&                 1.1&                 7.5&                near&                1.44&                 GMC&                   -&                   2\\
       G37.87-0.40 0&                 9.4&                 5.9&                 far&                4.14&               UCHII&       blue gradient&                   1\\
       G37.87-0.40 1&                 9.8&                 6.1&                 far&                4.14&               UCHII&       blue gradient&                   1\\
       G37.87-0.40 2&                 9.2&                 5.7&                 far&                4.14&               UCHII&       blue gradient&                   1\\
       G37.87-0.40 3&                 8.7&                 5.6&                 far&                4.14&                 GMC&                   -&                   1\\
       G37.87-0.40 4&                 8.6&                 5.5&                 far&                4.14&                 GMC&                   -&                   1\\
       G37.87-0.40 5&                 8.1&                 5.4&                 far&                4.14&                 GMC&                   -&                   1\\
       G37.87-0.40 6&                 6.6&                 5.1&             tangent&                4.14&                 GMC&                   -&                   1\\
       G37.87-0.40 7&                 1.2&                 7.5&                near&                4.14&                 GMC&                   -&                   1\\
       G37.87-0.40 8&                 1.1&                 7.6&                near&                4.14&                 GMC&                   -&                   1\\
       G37.87-0.40 9&                 1.5&                 7.2&                near&                4.14&                 GMC&                   -&                   1\\
       G43.89-0.78 0&                 8.3&                 6.2&                 far&                   -&               UCHII&       blue gradient&                   3\\
       G43.89-0.78 1&                 8.6&                 6.3&                 far&                   -&                 GMC&            envelope&                   3\\
       G45.07+0.13 0&                 7.6&                 6.2&                 far&                4.26&               UCHII&              single&                   2\\
       G45.07+0.13 1&                 6.5&                 6.0&                 far&                4.26&                 GMC&                   -&                   2\\
       G45.12+0.13 0&                 7.4&                 6.2&                 far&                6.78&               UCHII&               other&                   1\\
       G45.12+0.13 1&                 7.4&                 6.1&                 far&                6.78&               UCHII&            envelope&                   1\\
       G45.12+0.13 2&                 1.9&                 7.2&                near&                6.78&                 GMC&                   -&                   1\\
       G45.12+0.13 3&                 7.4&                 6.0&                 far&                6.78&                 GMC&            envelope&                   1\\
       G45.45+0.06 0&                 7.2&                 6.1&                 far&                3.71&               UCHII&       blue gradient&                   2\\
       G45.45+0.06 1&                 7.6&                 6.2&                 far&                3.71&                 GMC&            envelope&                   2\\
       G45.45+0.06 2&                 1.9&                 7.2&                near&                3.71&                 GMC&                   -&                   2\\
       G45.47+0.05 0&                 7.1&                 6.1&                 far&                3.34&               UCHII&        red gradient&               1+2+3\\
       G45.47+0.05 1&                 1.9&                 7.2&                near&                3.34&                 GMC&                   -&               1+2+3\\
       G48.61+0.02 0&                 9.6&                 7.5&                 far&                2.20&               UCHII&        red gradient&                 2+3\\
       G48.61+0.02 1&                 0.7&                 8.0&                near&                2.20&                 GMC&                   -&                 2+3\\
       G48.61+0.02 2&                 6.5&                 6.4&                 far&                2.20&                 GMC&                   -&                 2+3\\
       G50.32+0.68 0&                 2.1&                 7.2&                near&                   -&               UCHII&                   -&                   1\\
       G60.88-0.13 0&                 2.8&                 7.4&                near&                4.90&               UCHII&               limit&                   2\\
       G61.48+0.09 0&                 5.2&                 7.5&                 far&                7.86&               UCHII&              single&                   4\\
       G69.54-0.98 0&                2.57&                 7.9&             tangent&                   -&               UCHII&               thick&                 4+5\\
       G70.29+1.60 0&                 7.3&                 9.1&                 far&                   -&               UCHII&       blue gradient&                   2\\
       G70.29+1.60 1&                 7.8&                 9.3&                 far&                   -&                 GMC&            envelope&                   2\\
       G70.33+1.59 0&                 7.3&                 9.1&                 far&                   -&               UCHII&              single&                 1+2\\
   IRAS 20051+3435 0&                 2.6&                 7.6&             tangent&                   -&                 GMC&               limit&                  -1\\
       G41.74+0.10 0&                11.3&                 7.6&                 far&                0.56&               UCHII&               limit&                  -1\\
       G41.74+0.10 1&                11.6&                 7.7&                 far&                0.56&               UCHII&                   -&                  -1\\
       G41.74+0.10 2&                 2.4&                 6.8&                near&                0.56&                 GMC&                   -&                  -1\\
       G41.74+0.10 3&                 3.8&                 6.1&                near&                0.56&                 GMC&                   -&                  -1\\
       G41.74+0.10 4&                11.2&                 7.4&                 far&                0.56&               UCHII&                   -&                  -1\\
      IRDC 1923+13 0&                 4.2&                 6.5&                near&                   -&                 GMC&               limit&                  -1\\
      IRDC 1923+13 1&                 5.5&                 6.3&             tangent&                   -&                 GMC&                   -&                  -1\\
      IRDC 1923+13 2&                 3.8&                 6.6&                near&                   -&                 GMC&                   -&                  -1\\
      IRDC 1916+11 0&                 2.0&                 7.2&                near&                   -&                 GMC&               limit&                  -1\\
      IRDC 1916+11 1&                 4.2&                 6.2&                near&                   -&                 GMC&                   -&                  -1\\
      IRDC 1916+11 2&                 3.6&                 6.4&                near&                   -&                 GMC&                   -&                  -1\\
}{
\tablenotetext{a}{The Kinematic Distance Ambiguity described in Section \ref{sec:distances}.}
\tablenotetext{b}{Scenario or scenarios most likely to be consistent with the observed spectrum, as described
in Section \ref{sec:scenarios}.  In some cases, the spectrum was consistent with multiple scenarios or some
blend of multiple scenarios.  In others, the source could not be classified, in which case it is marked with 
-1 in this column. }
}


\subsection{Measuring Line Optical Depth}
\label{sec:linedepth}
In order to measure physical properties of an absorbing source, measurements
must be obtained of the optical depths of both the \oneone\ and \twotwo\ lines.
These measurements are presented in Tables \ref{tab:h2comeasured_a} and \ref{tab:h2comeasured_a}.
Once an optical depth with errors is determined, the spectral line depths can be matched
to large velocity gradient (LVG) models to determine column and spatial
density.  The spectral line optical depth depends both on the nadir flux density of the
absorption line and the strength of the illuminating background continuum
source.  If the background is the CMB, the `filling factor' of the molecular
cloud is simply its size relative to the beam size.  If there is a continuum
source in addition to the CMB, the size of the continuum source and the
intervening molecular cloud both affect the absorption depth.  Throughout this
paper, we use the term `filling factor' to refer to the fraction of the beam area filled
by the absorbing molecular cloud and `covering factor' to refer to the fraction
of the background continuum source that is covered by the intervening molecular
material.

The VLA archival images were used to estimate the size of the illuminating
background source.  When images at both wavelengths were available, we
separately determined the 2 cm and 6 cm source sizes.  The source size
determination is imprecise because we select a single source size for
non-uniform surface brightness profiles, and in many cases the VLA observation
did not recover the full flux density seen in single-dish measurements.
\citet{Araya2002} estimated the interferometer-to- single-dish flux ratio at
6 cm in this sample and found that the interferometer observations recovered
anywhere from 3\% to 100\% of the single-dish flux.  We repeat these
measurements at 2 cm and find that the typical recovery fraction is higher,
$\sim40\%$ to $100\%$, although sources for which only VLA upper limits could
be measured have recovery fractions $<1\%$. 

%When calculating the line optical depth, we assumed that the absorbing
%molecular cloud uniformly covers the illuminating background source.  We make
%the additional assumption that the molecular absorber covers {\it only} the
%background source and has no additional extent in the beam, such that the
%contribution from the CMB is uniform over the \ion{H}{2} region area and zero
%elsewhere (while the CMB is ubiquitous, with the on-off calibration strategy
%its net contribution is zero unless it is absorbed in the on position).  This
%assumption probably holds for the dominant line associated with the UC\ion{H}{2}
%region, but is likely to be incorrect for diffuse clouds along the line of
%sight.  %We have examined the assumption for clouds at $\ell < 60\degree$ by
%%looking at the GRS data cubes at the same velocity to determine whether the
%%cloud is extended and beam-filling.

\OneColFigure{figures_chH2CO/Derived_DensityVsDensityCorrected_all} 
{The filling factor corrected (FFC) density vs. the derived density with no
filling factor correction.  While there are some cases where the correction
results in an order of magnitude or more increase in the density, most points
show a small correction.  The black line is the one-one line.  Red squares
show where the filling factor corrected point was used, while blue circles show
where the uncorrected point was used.  Magenta left-pointing triangles are
limits where the filling factor correction was used, green downward triangles
are limits where the uncorrected points were used, and orange upward triangles
are lower limits where the filling-factor correction was used.}
{fig:ffc}
{0.30}{0}

The optical depth measurements were ``filling factor corrected'' by assuming
the CMB only contributed flux density over the same area as the \ion{H}{2}
region (i.e., the foreground cloud covers the exact same patch of sky as the
\uchii\ region).  When the \ion{H}{2} region is small (e.g., 10\% of the beam
area or less), the contribution of the CMB to the continuum is negligible, but
in cases of more diffuse \ion{H}{2} regions, the CMB contribution is
significant, particularly at 2 cm.  The inferred optical depths and source areas
are presented in Table \ref{tab:h2coinferred}.  Both ``filling factor corrected'' and
uncorrected densities are presented in Table \ref{tab:h2coderived}.  The effect
of the filling factor correction (FFC) on density measurements is shown in
Figure \ref{fig:ffc}.  In a few cases, no volume density-column density
parameter space in the models (Section \ref{sec:models}) was consistent with
the spectral line ratio after filling factor correction: in these cases, the
filling factor correction was not used.  Similarly, no filling factor
correction was applied to sources without detected continuum.  These exceptions
are noted in Table \ref{tab:h2coderived} in the ``Flag'' column.
%These diffuse molecular clouds are more likely to be beam-filling, and
%therefore the uncorrected (no-FFC) density measurements may be more accurate.
%XXXX How were sources put in "FFC" or "No FFC" categories?

The above definitions are summarized briefly in the following equations:
\begin{eqnarray*}
  S_{\nu,obs} &=& S_{\nu,cont} (1-CF e^{-\tau_{\nu}}) - S_{\nu,CMB} (FF e^{-\tau_{\nu}}) \\
  FF &=& \Omega_{cloud} / \Omega_{beam} \\ 
  CF &=& \Omega_{cloud} / \Omega_{continuum} \\ 
\end{eqnarray*}
in which CF is the ``covering factor'', FF is the ``filling factor'', and there
is no positive contribution from the CMB because it is assumed to be removed by
position-switching.


The systematic uncertainties in the continuum and the filling factor result in
similar errors in the optical depth measurement, and together dominate the
total error budget for our measurements.  A 30\% error in the \oneone\ and 20\%
error in the \twotwo\ continuum levels were assumed because of flux calibration
uncertainty characteristic of the instruments. An additional 10\% error in the beam area, which sets the maximum
coupling to the CMB (assuming a beam-filling source), was included to account 
for focus error.
%The beam
%area error is probably a good estimate of the (systematic) error in the CMB
%measurement, while the continuum errors are conservative.  
A 20\% statistical error in the cloud filling factor was assumed for the majority of the
survey, but it was decreased to 10\% when the ratio of continuum to CMB flux was $>0.5$ and the source
size was small, indicating that the VLA-measured source is indeed the dominant
continuum component in the beam.  The statistical error does not account for systematic
errors in the geometric assumptions.  Note that changes to the filling factor
should have a minimal effect on the derived density unless the source sizes at
2 cm and 6 cm differ substantially, while changes in the filling factor will
always have a large effect on the derived column density (Figure
\ref{fig:ffcdependence}).

\OneColFigure{figures_chH2CO/model_ffcdependence}
{The dependence of derived parameters on the filling factor, assuming an
optical depth ratio $\tau_{\oneone}/\tau_{\twotwo} =$1 (solid), 2 (dash-dot),
or 4 (dashed).  
The X-axis is the ``real'' optical depth, $\tau_{1-1}(real) = \tau_{1-1}(observed) / FF$.
Assuming the same filling factor correction is applied to both
the \oneone\ and \twotwo\ lines, filling factor correction will only move the
measurements along the X-axis of these plots.  A decrease in the filling factor
requires an increase in the true optical depth to maintain a constant apparent
$\tau(observed)$, which in turn drives up the derived abundance and column density while
leaving the volume density unchanged (except at high optical depths,
$\tau\gtrsim0.2$).
}
{fig:ffcdependence}{0.30}{0}


% have I discussed low density + high density clouds?  make a plot
% of density measured / real density vs. relative filling factor...
% or filling factor of the high-density stuff

%The signal-to-noise in the absorption lines is excellent. In order to improve
%the quality of our measurements, we need more accurate continuum measurements,
%which can be accomplished with a more careful observing technique and
%calibrator selection, or by observing sources with only the CMB as
%illumination. Cloud size and covering measurements will also improve the
%accuracy of the density measurements.  These can be obtained with higher
%resolution observations or, as shown in \citet{Zeiger2010}, higher frequency
%transitions of \formaldehyde.
%
%(XXXX it may be possible to correct for this... but.... is it worth it?) An
%additional systematic uncertainty not considered in this survey is the
%potential additional contribution of a diffuse Galactic background as a
%backlight.   Emission that is smooth on $>11\arcmin$ scales, as is known to
%exist throughout the plane, will be subtracted off by nodding but may still be
%absorbed by the intervening \formaldehyde.  This signal should be small
%compared to continuum sources and generally less than the CMB at 2 cm (average
%in the plane is $\lesssim0.1K$), but it may be comparable to the CMB at 6 cm
%because of rising synchrotron emission ($\sim1K$), and it will be important to
%account for it in a larger survey.  It is possible to do this using publicly
%available data sets such as the GPA (Langston XXXX cite) and WMAP foreground
%maps (XXXX).

%\Table{lcccccccc}{Inferred \formaldehyde\ line properties}
{\colhead{Source Name}&\colhead{$\tau_{1-1}$}&\colhead{$\tau_{1-1}$ (FFC)}&\colhead{$\tau_{2-2}$}&\colhead{$\tau_{2-2}$ (FFC)}&\colhead{2-2 Upper }&\colhead{2cm Area\tablenotemark{a}}&\colhead{6cm Area  \tablenotemark{a}   }&\colhead{FFC Error}\\
\colhead{           }&\colhead{             }&\colhead{                   }&\colhead{             }&\colhead{                   }&\colhead{Limit Flag}&\colhead{\arcsec$^2$                   }&\colhead{\arcsec$^2$                   }&\colhead{}\\ }
{tab:h2coinferred}{
       G32.80+0.19 0&        0.18 (0.055)&         0.2 (0.059)&        0.12 (0.024)&        0.15 (0.031)&                   0&                88.0&               226.2&                 0.1\\
       G32.80+0.19 1&        0.04 (0.013)&       0.043 (0.013)&      0.016 (0.0051)&      0.021 (0.0065)&                   0&                88.0&               226.2&                 0.1\\
       G32.80+0.19 2&      0.027 (0.0089)&      0.029 (0.0095)&    0.0033 (0.00069)&    0.0042 (0.00088)&                   0&                88.0&               226.2&                 0.1\\
       G32.80+0.19 3&        0.11 (0.035)&        0.12 (0.037)&      0.014 (0.0028)&      0.018 (0.0035)&                   0&                88.0&               226.2&                 0.1\\
       G32.80+0.19 4&       0.039 (0.012)&       0.042 (0.013)&     0.0055 (0.0011)&     0.0071 (0.0014)&                   0&                88.0&               226.2&                 0.1\\
       G33.13-0.09 0&          0.34 (0.1)&         0.49 (0.15)&        0.16 (0.032)&         0.63 (0.12)&                   0&                33.5&                33.5&                 0.2\\
       G33.13-0.09 1&       0.035 (0.015)&        0.047 (0.02)&         0 (0.0059)&         0 (0.0031)&                   1&                33.5&                33.5&                 0.2\\
       G33.13-0.09 2&       0.062 (0.022)&       0.084 (0.029)&         0 (0.0059)&         0 (0.0031)&                   1&                33.5&                33.5&                 0.2\\
       G33.13-0.09 3&       0.061 (0.021)&       0.082 (0.028)&         0 (0.0059)&         0 (0.0031)&                   1&                33.5&                33.5&                 0.2\\
       G33.92+0.11 0&       0.084 (0.027)&         0.1 (0.031)&      0.045 (0.0091)&       0.094 (0.018)&                   0&               214.0&               214.0&                 0.2\\
       G33.92+0.11 1&       0.082 (0.026)&       0.098 (0.031)&      0.036 (0.0072)&       0.075 (0.014)&                   0&               214.0&               214.0&                 0.2\\
       G33.92+0.11 2&        0.17 (0.062)&        0.21 (0.074)&         0 (0.0049)&         0 (0.0031)&                   1&               214.0&               214.0&                 0.2\\
       G34.26+0.15 0&         0.38 (0.12)&          0.4 (0.12)&        0.22 (0.043)&        0.26 (0.052)&                   0&                10.9&                10.9&                 0.2\\
       G34.26+0.15 1&      0.028 (0.0089)&      0.029 (0.0092)&     0.0067 (0.0014)&     0.0079 (0.0017)&                   0&                10.9&                10.9&                 0.2\\
       G34.26+0.15 2&      0.017 (0.0059)&       0.018 (0.006)&    0.0026 (0.00059)&     0.0031 (0.0007)&                   0&                10.9&                10.9&                 0.2\\
       G34.26+0.15 3&      0.022 (0.0072)&      0.023 (0.0074)&    0.0036 (0.00092)&     0.0043 (0.0011)&                   0&                10.9&                10.9&                 0.2\\
       G34.26+0.15 4&     0.0082 (0.0036)&     0.0085 (0.0037)&     0.0026 (0.0011)&      0.003 (0.0013)&                   0&                10.9&                10.9&                 0.2\\
       G35.20-1.74 0&        0.21 (0.063)&        0.22 (0.066)&       0.071 (0.014)&       0.084 (0.017)&                   0&                39.5&                39.5&                 0.2\\
       G35.20-1.74 1&      0.028 (0.0085)&      0.029 (0.0088)&     0.0039 (0.0011)&     0.0046 (0.0013)&                   0&                39.5&                39.5&                 0.2\\
       G35.20-1.74 2&       0.063 (0.019)&       0.065 (0.019)&      0.008 (0.0017)&     0.0095 (0.0019)&                   0&                39.5&                39.5&                 0.2\\
       G35.20-1.74 3&     0.0073 (0.0027)&     0.0075 (0.0028)&         0 (0.0023)&         0 (0.0031)&                   1&                39.5&                39.5&                 0.2\\
       G35.57-0.03 0&        0.11 (0.035)&        0.15 (0.049)&       0.056 (0.011)&        0.26 (0.054)&                   0&                 6.7&                 6.7&                 0.1\\
       G35.57-0.03 1&       0.034 (0.018)&       0.046 (0.024)&      0.011 (0.0047)&        0.047 (0.02)&                   0&                 6.7&                 6.7&                 0.1\\
       G35.57-0.03 2&        0.03 (0.017)&       0.042 (0.023)&         0 (0.0099)&          0 (0.019)&                   1&                 6.7&                 6.7&                 0.1\\
       G35.57-0.03 3&        0.05 (0.021)&       0.069 (0.029)&         0 (0.0099)&          0 (0.019)&                   1&                 6.7&                 6.7&                 0.1\\
       G35.57-0.03 4&        0.051 (0.02)&       0.069 (0.028)&      0.017 (0.0065)&       0.077 (0.029)&                   0&                 6.7&                 6.7&                 0.1\\
       G35.58+0.07 0&        0.24 (0.071)&        0.32 (0.097)&       0.085 (0.017)&         0.61 (0.12)&                   0&                 2.1&                 2.1&                 0.2\\
       G35.58+0.07 1&       0.072 (0.029)&       0.096 (0.038)&         0 (0.0072)&          0 (0.019)&                   1&                 2.1&                 2.1&                 0.2\\
       G35.58+0.07 2&       0.037 (0.012)&       0.049 (0.016)&         0 (0.0072)&          0 (0.019)&                   1&                 2.1&                 2.1&                 0.2\\
       G35.58+0.07 3&        0.05 (0.016)&       0.066 (0.021)&         0 (0.0072)&           0 (0.01)&                   1&                 2.1&                 2.1&                 0.2\\
       G37.87-0.40 0&        0.12 (0.037)&        0.13 (0.038)&      0.047 (0.0095)&       0.061 (0.012)&                   0&                27.5&               170.9&                 0.2\\
       G37.87-0.40 1&      0.028 (0.0089)&      0.029 (0.0092)&     0.0095 (0.0024)&      0.012 (0.0031)&                   0&                27.5&               170.9&                 0.2\\
       G37.87-0.40 2&       0.081 (0.025)&       0.084 (0.026)&     0.0075 (0.0021)&     0.0096 (0.0027)&                   0&                27.5&               170.9&                 0.2\\
       G37.87-0.40 3&       0.074 (0.024)&       0.076 (0.025)&      0.011 (0.0023)&       0.014 (0.003)&                   0&                27.5&               170.9&                 0.2\\
       G37.87-0.40 4&       0.097 (0.029)&          0.1 (0.03)&      0.0098 (0.002)&      0.013 (0.0026)&                   0&                27.5&               170.9&                 0.2\\
       G37.87-0.40 5&       0.041 (0.013)&       0.043 (0.013)&    0.0033 (0.00068)&    0.0043 (0.00087)&                   0&                27.5&               170.9&                 0.2\\
       G37.87-0.40 6&      0.025 (0.0083)&      0.026 (0.0086)&    0.0021 (0.00052)&    0.0026 (0.00066)&                   0&                27.5&               170.9&                 0.2\\
       G37.87-0.40 7&       0.039 (0.012)&        0.04 (0.012)&     0.0054 (0.0012)&     0.0069 (0.0015)&                   0&                27.5&               170.9&                 0.2\\
       G37.87-0.40 8&      0.016 (0.0069)&      0.016 (0.0071)&    0.0035 (0.00081)&      0.0046 (0.001)&                   0&                27.5&               170.9&                 0.2\\
       G37.87-0.40 9&       0.03 (0.0095)&      0.031 (0.0098)&    0.0035 (0.00073)&    0.0045 (0.00094)&                   0&                27.5&               170.9&                 0.2\\
       G43.89-0.78 0&        0.25 (0.074)&        0.32 (0.096)&      0.037 (0.0078)&        0.12 (0.024)&                   0&                13.5&                13.5&                 0.1\\
       G43.89-0.78 1&      0.025 (0.0077)&      0.031 (0.0097)&     0.0097 (0.0022)&      0.029 (0.0067)&                   0&                13.5&                13.5&                 0.1\\
       G45.07+0.13 0&       0.092 (0.029)&         0.13 (0.04)&       0.04 (0.0081)&        0.096 (0.02)&                   0&                 2.5&                 2.5&                 0.2\\
       G45.07+0.13 1&        0.058 (0.02)&        0.08 (0.027)&     0.0061 (0.0019)&      0.014 (0.0045)&                   0&                 2.5&                 2.5&                 0.2\\
       G45.12+0.13 0&       0.043 (0.013)&       0.045 (0.013)&      0.014 (0.0028)&      0.017 (0.0033)&                   0&                15.4&               516.6&                 0.2\\
       G45.12+0.13 1&       0.035 (0.011)&       0.036 (0.011)&      0.0095 (0.002)&      0.011 (0.0025)&                   0&                15.4&               516.6&                 0.2\\
       G45.12+0.13 2&       0.046 (0.014)&       0.048 (0.014)&     0.0075 (0.0015)&      0.009 (0.0019)&                   0&                15.4&               516.6&                 0.2\\
       G45.12+0.13 3&       0.006 (0.002)&     0.0062 (0.0021)&    0.0033 (0.00068)&     0.004 (0.00082)&                   0&                15.4&               516.6&                 0.2\\
       G45.45+0.06 0&        0.32 (0.096)&        0.32 (0.095)&       0.063 (0.013)&       0.069 (0.012)&                   0&              1963.0&              1963.0&                 0.2\\
       G45.45+0.06 1&       0.025 (0.011)&       0.026 (0.011)&       0.01 (0.0021)&      0.011 (0.0019)&                   0&              1963.0&              1963.0&                 0.2\\
       G45.45+0.06 2&      0.011 (0.0036)&      0.012 (0.0035)&         0 (0.0031)&           0 (0.01)&                   1&              1963.0&              1963.0&                 0.2\\
       G45.47+0.05 0&         0.35 (0.11)&         0.45 (0.14)&       0.089 (0.018)&        0.39 (0.079)&                   0&                 3.0&                 3.0&                 0.2\\
       G45.47+0.05 1&      0.018 (0.0068)&      0.023 (0.0084)&           0 (0.01)&           0 (0.01)&                   1&                 3.0&                 3.0&                 0.2\\
       G48.61+0.02 0&       0.058 (0.018)&       0.068 (0.021)&      0.015 (0.0034)&       0.053 (0.012)&                   0&                25.5&                25.5&                 0.2\\
       G48.61+0.02 1&       0.02 (0.0075)&      0.023 (0.0088)&         0 (0.0067)&         0 (0.0026)&                   1&                25.5&                25.5&                 0.2\\
       G48.61+0.02 2&      0.016 (0.0052)&      0.018 (0.0061)&     0.0033 (0.0013)&      0.012 (0.0046)&                   0&                25.5&                25.5&                 0.2\\
       G50.32+0.68 0&        0.027 (0.01)&       0.045 (0.017)&     0.0089 (0.0031)&       0.058 (0.019)&                   0&               108.0&               108.0&                 0.2\\
       G60.88-0.13 0&        0.12 (0.037)&        0.14 (0.043)&      0.011 (0.0031)&      0.031 (0.0071)&                   0&               615.0&               615.0&                 0.2\\
       G61.48+0.09 0&       0.088 (0.026)&        0.09 (0.027)&       0.069 (0.014)&       0.088 (0.017)&                   0&               355.0&               355.0&                 0.2\\
       G69.54-0.98 0&         0.98 (0.29)&           5.7 (1.7)&        0.18 (0.037)&          2.9 (0.57)&                   0&                 0.5&                 0.5&                 0.2\\
       G70.29+1.60 0&       0.086 (0.026)&       0.089 (0.027)&      0.022 (0.0044)&      0.026 (0.0052)&                   0&                52.8&                52.8&                 0.1\\
       G70.29+1.60 1&      0.011 (0.0037)&      0.012 (0.0038)&         0 (0.0019)&         0 (0.0026)&                   1&                52.8&                52.8&                 0.1\\
       G70.33+1.59 0&          0.7 (0.21)&         0.78 (0.24)&        0.34 (0.068)&          0.52 (0.1)&                   0&                16.4&                16.4&                 0.1\\
   IRAS 20051+3435 0&        0.12 (0.036)&        0.13 (0.014)&      0.015 (0.0041)&      0.016 (0.0034)&                   0&             2747.75&             2747.75&                 0.0\\
       G41.74+0.10 0&         0.13 (0.04)&          0.2 (0.06)&       0.01 (0.0027)&       0.045 (0.012)&                   0&                75.2&                75.2&                 0.2\\
       G41.74+0.10 1&        0.04 (0.014)&        0.06 (0.021)&         0 (0.0071)&         0 (0.0026)&                   1&                75.2&                75.2&                 0.2\\
       G41.74+0.10 2&        0.14 (0.043)&        0.21 (0.065)&         0 (0.0071)&         0 (0.0026)&                   1&                75.2&                75.2&                 0.2\\
       G41.74+0.10 3&       0.045 (0.017)&       0.067 (0.025)&         0 (0.0071)&         0 (0.0026)&                   1&                75.2&                75.2&                 0.2\\
       G41.74+0.10 4&       0.089 (0.029)&        0.13 (0.043)&         0 (0.0071)&         0 (0.0028)&                   1&                75.2&                75.2&                 0.2\\
      IRDC 1923+13 0&       0.02 (0.0062)&       0.02 (0.0047)&          0 (0.009)&         0 (0.0028)&                   1&             2747.75&             2747.75&                 0.0\\
      IRDC 1923+13 1&      0.016 (0.0051)&      0.017 (0.0038)&          0 (0.009)&         0 (0.0028)&                   1&             2747.75&             2747.75&                 0.0\\
      IRDC 1923+13 2&     0.0081 (0.0028)&     0.0083 (0.0023)&          0 (0.009)&         0 (0.0028)&                   1&             2747.75&             2747.75&                 0.0\\
      IRDC 1916+11 0&       0.033 (0.011)&      0.036 (0.0062)&          0 (0.014)&          0 (0.042)&                   1&             2747.75&             2747.75&                 0.0\\
      IRDC 1916+11 1&       0.082 (0.025)&      0.089 (0.0095)&          0 (0.014)&          0 (0.042)&                   1&             2747.75&             2747.75&                 0.0\\
      IRDC 1916+11 2&      0.017 (0.0064)&      0.018 (0.0046)&          0 (0.014)&          0 (0.042)&                   1&             2747.75&             2747.75&                 0.0\\
}{\tablenotetext{a}{The beam area is 2747.75\arcsec$^2$, which is used when the CMB is the only background continuum illumination}}


%The models predict a line optical depth as a function of density and column
%assuming the line FWHM is 1 \kms\ and $dv/dr$ is 1 \kms pc$^{-1}$ .
Measurements of volume and column density were taken by averaging over the regions of
LVG model parameter space consistent with both spectral line optical depth measurements
to within $1\sigma$.  The ``$1\sigma$'' (68\% confidence; errors are
non-gaussian) error bars on the derived parameters ($N,n,X$) were taken to be
the extrema of these regions.  An example of this fitting process is shown in
Figure \ref{fig:fitexample}.  A second example demonstrating a lower-limit on the density
(instead of a direct measurement) is shown in Figure \ref{fig:fitexample2}.  This method is not as robust as $\chi^2$ fitting,
but because there are no free fit parameters, a statistically meaningful
$\chi^2_\nu$ cannot be computed.

In some cases, the ratio of the spectral line optical depths was consistent with low
density ($n\lesssim 100$ \percc) and high abundances ($X(\ortho)>10^{-8}$ \perkmspc), but
these were ruled out based on the prior assumption that extremely high
\formaldehyde\ abundances should not be observed at very low densities, since
it is formed at higher densities and destroyed by hard UV at low columns
\citep[see discussion in][]{Troscompt2009b}.
%Additionally, assuming that all
%of the matter inferred from 1.1 mm dust observations is associated with the
%\formaldehyde\ absorber, the beam-averaged 1.1 mm mass can be used as a lower
%limit on the volume density; however, that assumption is most likely invalid
%for the absorbers not directly associated with the \uchii\ region.  The use of
%prior assumptions to rule out portions of parameter space is discussed in
%detail in \citet{Zeiger2010}.

\OneColFigure{figures_chH2CO/G32.80+0.19_0_ffc_Nvsn_notext.png} 
{An example of the column density - density parameter space available given
measured \oneone\ and \twotwo\ optical depths.  The dashed lines show
abundances $\log_{10}(X(\ortho))$ \perkmspc.  The contours show the
regions allowed by the measurements of optical depth (\oneone: black, \twotwo: grey,
ratio: dotted);
the middle curve is the measured value, while the pair of curves around it are
$\pm 1\sigma$ including systematic error.  The shaded region shows the allowed
parameter space from which the physical parameters are derived. }
{fig:fitexample}
{0.30}{0}

\OneColFigure{figures_chH2CO/G33.13-0.09_0_ffc_Nvsn_notext.png}
{  Same description as Figure
\ref{fig:fitexample} but for the strongest component in G33.13-0.09.  It was
only possible to measure lower limits on the volume and column density for this
line; it is therefore assigned flag 8 in Table \ref{tab:h2coderived}.
}
{fig:fitexample2}
{0.30}{0}


%\Table{lccccccc}{Derived physical properties from \formaldehyde\ }
{\colhead{Source Name}&\colhead{N(\formaldehyde)\tablenotemark{a}}&\colhead{N(\formaldehyde) (FFC)\tablenotemark{b}}&\colhead{n(\hh) \tablenotemark{a} }&\colhead{n(\hh) (FFC)\tablenotemark{b}}&\colhead{X$_{\formaldehyde}$\tablenotemark{a}}&\colhead{X$_{\formaldehyde}$ (FFC)\tablenotemark{b}}&\colhead{Flag\tablenotemark{c}}\\
\colhead{           }&\colhead{(\persc)        }&\colhead{(\persc)              }&\colhead{(\percc)}&\colhead{(\percc)    }&\colhead{                    }&\colhead{                          }&\colhead{                      }\\ }
{tab:h2coderived}{
       G32.80+0.19 0&$12.79^{+0.11}_{-0.16}$&$\mathbf{12.94^{+0.16}_{-0.24}}$&$5.10^{+0.25}_{-0.26}$&$\mathbf{5.21^{+0.27}_{-0.29}}$&$-10.79^{+0.15}_{-0.20}$&$\mathbf{-10.75^{+0.15}_{-0.18}}$&                   2\\
       G32.80+0.19 1&$12.05^{+0.12}_{-0.11}$&$\mathbf{12.14^{+0.13}_{-0.13}}$&$4.96^{+0.22}_{-0.28}$&$\mathbf{5.05^{+0.21}_{-0.28}}$&$-11.39^{+0.20}_{-0.23}$&$\mathbf{-11.39^{+0.17}_{-0.20}}$&                   2\\
       G32.80+0.19 2&$11.66^{+0.10}_{-0.10}$&$\mathbf{11.71^{+0.10}_{-0.10}}$&$4.16^{+0.39}_{-0.38}$&$\mathbf{4.33^{+0.31}_{-0.32}}$&$-10.97^{+0.44}_{-0.46}$&$\mathbf{-11.10^{+0.37}_{-0.37}}$&                   2\\
       G32.80+0.19 3&$12.18^{+0.10}_{-0.09}$&$\mathbf{12.23^{+0.09}_{-0.09}}$&$4.07^{+0.38}_{-0.39}$&$\mathbf{4.23^{+0.32}_{-0.32}}$&$-10.37^{+0.44}_{-0.45}$&$\mathbf{-10.48^{+0.36}_{-0.38}}$&                   2\\
       G32.80+0.19 4&$11.82^{+0.10}_{-0.09}$&$\mathbf{11.87^{+0.10}_{-0.09}}$&$4.30^{+0.31}_{-0.32}$&$\mathbf{4.44^{+0.26}_{-0.29}}$&$-10.97^{+0.37}_{-0.37}$&$\mathbf{-11.05^{+0.31}_{-0.32}}$&                   2\\
       G33.13-0.09 0&$>12.80                 $&$\mathbf{>13.56}       $&$>4.54                 $&$\mathbf{>5.10}       $&$>-10.62                 $&$\mathbf{>-11.70}       $&                   8\\
       G33.13-0.09 1&$<11.96                 $&$\mathbf{<11.90}       $&$<4.50                 $&$\mathbf{<3.91}       $&$<-8.44                 $&$\mathbf{<-8.45}       $&                   6\\
       G33.13-0.09 2&$\mathbf{<12.20}       $&$<0.00                 $&$\mathbf{<4.29}       $&$<0.00                 $&$\mathbf{<-8.29}       $&$<0.00                 $&                   5\\
       G33.13-0.09 3&$\mathbf{<12.20}       $&$<0.00                 $&$\mathbf{<4.32}       $&$<0.00                 $&$\mathbf{<-8.29}       $&$<0.00                 $&                   5\\
       G33.92+0.11 0&$>12.35                 $&$\mathbf{>12.64}       $&$>4.86                 $&$\mathbf{>5.16}       $&$>-11.29                 $&$\mathbf{>-12.30}       $&                   8\\
       G33.92+0.11 1&$12.34^{+0.07}_{-0.08}$&$\mathbf{12.65^{+0.11}_{-0.17}}$&$4.97^{+0.22}_{-0.23}$&$\mathbf{5.26^{+0.22}_{-0.24}}$&$-11.11^{+0.19}_{-0.22}$&$\mathbf{-11.09^{+0.13}_{-0.16}}$&                   2\\
       G33.92+0.11 2&                   -&                   -&                   -&                   -&                   -&                   -&                   9\\
       G34.26+0.15 0&$13.01^{+0.10}_{-0.17}$&$\mathbf{13.13^{+0.15}_{-0.23}}$&$4.91^{+0.28}_{-0.29}$&$\mathbf{5.01^{+0.31}_{-0.32}}$&$-10.38^{+0.18}_{-0.23}$&$\mathbf{-10.36^{+0.17}_{-0.23}}$&                   2\\
       G34.26+0.15 1&$11.79^{+0.09}_{-0.08}$&$\mathbf{11.83^{+0.09}_{-0.08}}$&$4.67^{+0.23}_{-0.25}$&$\mathbf{4.75^{+0.21}_{-0.24}}$&$-11.36^{+0.26}_{-0.27}$&$\mathbf{-11.40^{+0.23}_{-0.25}}$&                   2\\
       G34.26+0.15 2&$11.53^{+0.10}_{-0.10}$&$\mathbf{11.56^{+0.10}_{-0.10}}$&$4.38^{+0.30}_{-0.33}$&$\mathbf{4.48^{+0.28}_{-0.30}}$&$-11.33^{+0.36}_{-0.37}$&$\mathbf{-11.40^{+0.32}_{-0.34}}$&                   2\\
       G34.26+0.15 3&$11.63^{+0.11}_{-0.10}$&$\mathbf{11.66^{+0.10}_{-0.10}}$&$4.43^{+0.29}_{-0.32}$&$\mathbf{4.53^{+0.26}_{-0.30}}$&$-11.28^{+0.34}_{-0.35}$&$\mathbf{-11.34^{+0.31}_{-0.32}}$&                   2\\
       G34.26+0.15 4&$11.40^{+0.17}_{-0.14}$&$\mathbf{11.45^{+0.17}_{-0.16}}$&$4.87^{+0.31}_{-0.43}$&$\mathbf{4.94^{+0.30}_{-0.42}}$&$-11.95^{+0.30}_{-0.35}$&$\mathbf{-11.98^{+0.27}_{-0.34}}$&                   2\\
       G35.20-1.74 0&$12.60^{+0.08}_{-0.07}$&$\mathbf{12.65^{+0.08}_{-0.07}}$&$4.72^{+0.25}_{-0.25}$&$\mathbf{4.79^{+0.25}_{-0.26}}$&$-10.61^{+0.23}_{-0.27}$&$\mathbf{-10.62^{+0.22}_{-0.26}}$&                   2\\
       G35.20-1.74 1&$11.69^{+0.11}_{-0.10}$&$\mathbf{11.72^{+0.11}_{-0.10}}$&$4.30^{+0.33}_{-0.37}$&$\mathbf{4.41^{+0.30}_{-0.33}}$&$-11.09^{+0.39}_{-0.39}$&$\mathbf{-11.16^{+0.34}_{-0.36}}$&                   2\\
       G35.20-1.74 2&$11.97^{+0.09}_{-0.09}$&$\mathbf{12.00^{+0.09}_{-0.09}}$&$4.20^{+0.34}_{-0.35}$&$\mathbf{4.31^{+0.30}_{-0.30}}$&$-10.70^{+0.39}_{-0.41}$&$\mathbf{-10.79^{+0.34}_{-0.36}}$&                   2\\
       G35.20-1.74 3&$<11.41                 $&$\mathbf{<11.44}       $&$<4.89                 $&$\mathbf{<5.02}       $&$<-9.24                 $&$\mathbf{<-9.30}       $&                   6\\
       G35.57-0.03 0&$>12.42                 $&$\mathbf{>13.38}       $&$>4.82                 $&$\mathbf{>5.61}       $&$>-11.20                 $&$\mathbf{>-12.02}       $&                   8\\
       G35.57-0.03 1&$>11.72                 $&$\mathbf{>12.25}       $&$>4.51                 $&$\mathbf{>5.13}       $&$>-11.80                 $&$\mathbf{>-12.72}       $&                   8\\
       G35.57-0.03 2&$<11.98                 $&$\mathbf{<12.12}       $&$<4.96                 $&$\mathbf{<5.12}       $&$<-8.71                 $&$\mathbf{<-8.72}       $&                   6\\
       G35.57-0.03 3&$<12.09                 $&$\mathbf{<12.23}       $&$<4.60                 $&$\mathbf{<4.78}       $&$<-8.38                 $&$\mathbf{<-8.37}       $&                   6\\
       G35.57-0.03 4&$>11.93                 $&$\mathbf{>12.47}       $&$>4.58                 $&$\mathbf{>5.20}       $&$>-11.55                 $&$\mathbf{>-12.48}       $&                   8\\
       G35.58+0.07 0&$>12.58                 $&$\mathbf{>14.06}       $&$>4.50                 $&$\mathbf{>5.48}       $&$>-10.79                 $&$\mathbf{>-11.71}       $&                   8\\
       G35.58+0.07 1&$<12.19                 $&$\mathbf{<12.32}       $&$<4.08                 $&$\mathbf{<4.55}       $&$<-8.07                 $&$\mathbf{<-8.14}       $&                   6\\
       G35.58+0.07 2&$<11.96                 $&$\mathbf{<12.10}       $&$<4.57                 $&$\mathbf{<4.95}       $&$<-8.53                 $&$\mathbf{<-8.63}       $&                   6\\
       G35.58+0.07 3&$<12.06                 $&$\mathbf{<12.17}       $&$<4.35                 $&$\mathbf{<4.38}       $&$<-8.32                 $&$\mathbf{<-8.24}       $&                   6\\
       G37.87-0.40 0&$12.44^{+0.07}_{-0.07}$&$\mathbf{12.53^{+0.07}_{-0.09}}$&$4.86^{+0.22}_{-0.23}$&$\mathbf{4.98^{+0.21}_{-0.24}}$&$-10.90^{+0.20}_{-0.24}$&$\mathbf{-10.92^{+0.18}_{-0.21}}$&                   2\\
       G37.87-0.40 1&$11.87^{+0.10}_{-0.09}$&$\mathbf{11.95^{+0.10}_{-0.09}}$&$4.89^{+0.22}_{-0.26}$&$\mathbf{5.00^{+0.21}_{-0.24}}$&$-11.50^{+0.21}_{-0.24}$&$\mathbf{-11.53^{+0.18}_{-0.22}}$&                   2\\
       G37.87-0.40 2&$12.10^{+0.18}_{-0.26}$&$\mathbf{12.09^{+0.14}_{-0.28}}$&$3.16^{+1.15}_{-1.20}$&$\mathbf{3.79^{+1.78}_{-0.71}}$&$-9.54^{+1.32}_{-1.41}$&$\mathbf{-10.18^{+0.77}_{-2.06}}$&                   4\\
       G37.87-0.40 3&$12.05^{+0.10}_{-0.10}$&$\mathbf{12.10^{+0.10}_{-0.09}}$&$4.33^{+0.32}_{-0.33}$&$\mathbf{4.49^{+0.28}_{-0.29}}$&$-10.76^{+0.37}_{-0.39}$&$\mathbf{-10.87^{+0.31}_{-0.34}}$&                   2\\
       G37.87-0.40 4&$12.17^{+0.17}_{-0.25}$&$\mathbf{12.14^{+0.10}_{-0.09}}$&$3.36^{+1.35}_{-0.98}$&$\mathbf{4.13^{+0.35}_{-0.36}}$&$-9.67^{+1.12}_{-1.60}$&$\mathbf{-10.46^{+0.41}_{-0.42}}$&                   2\\
       G37.87-0.40 5&$11.85^{+0.18}_{-0.21}$&$\mathbf{11.88^{+0.18}_{-0.25}}$&$3.05^{+1.04}_{-1.17}$&$\mathbf{3.34^{+1.33}_{-1.03}}$&$-9.68^{+1.32}_{-1.25}$&$\mathbf{-9.94^{+1.17}_{-1.58}}$&                   4\\
       G37.87-0.40 6&$11.67^{+0.19}_{-0.23}$&$\mathbf{11.70^{+0.19}_{-0.26}}$&$3.08^{+1.07}_{-1.21}$&$\mathbf{3.33^{+1.32}_{-1.13}}$&$-9.89^{+1.36}_{-1.30}$&$\mathbf{-10.11^{+1.26}_{-1.58}}$&                   4\\
       G37.87-0.40 7&$11.81^{+0.10}_{-0.09}$&$\mathbf{11.86^{+0.10}_{-0.09}}$&$4.28^{+0.31}_{-0.33}$&$\mathbf{4.45^{+0.27}_{-0.29}}$&$-10.95^{+0.37}_{-0.38}$&$\mathbf{-11.07^{+0.32}_{-0.32}}$&                   2\\
       G37.87-0.40 8&$11.56^{+0.11}_{-0.11}$&$\mathbf{11.63^{+0.11}_{-0.10}}$&$4.67^{+0.30}_{-0.33}$&$\mathbf{4.80^{+0.29}_{-0.31}}$&$-11.59^{+0.34}_{-0.38}$&$\mathbf{-11.65^{+0.30}_{-0.34}}$&                   2\\
       G37.87-0.40 9&$11.72^{+0.13}_{-0.29}$&$\mathbf{11.74^{+0.10}_{-0.09}}$&$3.98^{+1.97}_{-0.52}$&$\mathbf{4.32^{+0.31}_{-0.32}}$&$-10.74^{+0.60}_{-2.26}$&$\mathbf{-11.06^{+0.36}_{-0.37}}$&                   2\\
       G43.89-0.78 0&$12.49^{+0.10}_{-0.09}$&$\mathbf{12.76^{+0.08}_{-0.07}}$&$4.18^{+0.34}_{-0.33}$&$\mathbf{4.68^{+0.28}_{-0.28}}$&$-10.17^{+0.37}_{-0.40}$&$\mathbf{-10.40^{+0.24}_{-0.30}}$&                   2\\
       G43.89-0.78 1&$\mathbf{11.87^{+0.09}_{-0.08}}$&$12.80^{+0.61}_{-1.00}$&$\mathbf{4.95^{+0.20}_{-0.23}}$&$6.16^{+0.96}_{-1.84}$&$\mathbf{-11.56^{+0.19}_{-0.22}}$&$-11.84^{+0.88}_{-0.42}$&                   1\\
       G45.07+0.13 0&$12.38^{+0.08}_{-0.08}$&$\mathbf{12.75^{+0.13}_{-0.20}}$&$4.96^{+0.22}_{-0.24}$&$\mathbf{5.25^{+0.25}_{-0.27}}$&$-11.06^{+0.19}_{-0.22}$&$\mathbf{-10.97^{+0.15}_{-0.18}}$&                   2\\
       G45.07+0.13 1&                   -&                   -&                   -&                   -&                   -&                   -&                   9\\
       G45.12+0.13 0&$12.02^{+0.08}_{-0.07}$&$\mathbf{12.07^{+0.08}_{-0.07}}$&$4.83^{+0.21}_{-0.21}$&$\mathbf{4.92^{+0.19}_{-0.21}}$&$-11.30^{+0.20}_{-0.23}$&$\mathbf{-11.32^{+0.19}_{-0.21}}$&                   2\\
       G45.12+0.13 1&$11.90^{+0.09}_{-0.08}$&$\mathbf{11.95^{+0.08}_{-0.08}}$&$4.74^{+0.22}_{-0.23}$&$\mathbf{4.83^{+0.21}_{-0.23}}$&$-11.32^{+0.23}_{-0.25}$&$\mathbf{-11.36^{+0.21}_{-0.23}}$&                   2\\
       G45.12+0.13 2&$11.90^{+0.10}_{-0.08}$&$\mathbf{11.93^{+0.09}_{-0.09}}$&$4.41^{+0.26}_{-0.28}$&$\mathbf{4.52^{+0.24}_{-0.26}}$&$-11.00^{+0.32}_{-0.32}$&$\mathbf{-11.06^{+0.29}_{-0.30}}$&                   2\\
       G45.12+0.13 3&$11.48^{+0.08}_{-0.09}$&$\mathbf{11.55^{+0.08}_{-0.12}}$&$5.15^{+0.19}_{-0.21}$&$\mathbf{5.23^{+0.19}_{-0.22}}$&$-12.15^{+0.16}_{-0.18}$&$\mathbf{-12.16^{+0.14}_{-0.17}}$&                   2\\
       G45.45+0.06 0&$12.62^{+0.08}_{-0.08}$&$\mathbf{12.64^{+0.07}_{-0.08}}$&$4.33^{+0.29}_{-0.31}$&$\mathbf{4.37^{+0.28}_{-0.28}}$&$-10.19^{+0.33}_{-0.35}$&$\mathbf{-10.21^{+0.30}_{-0.33}}$&                   2\\
       G45.45+0.06 1&$11.89^{+0.08}_{-0.08}$&$\mathbf{11.92^{+0.07}_{-0.07}}$&$5.00^{+0.26}_{-0.28}$&$\mathbf{5.04^{+0.26}_{-0.27}}$&$-11.59^{+0.25}_{-0.29}$&$\mathbf{-11.60^{+0.23}_{-0.28}}$&                   2\\
       G45.45+0.06 2&$<11.55                 $&$\mathbf{<11.66}       $&$<4.77                 $&$\mathbf{<5.39}       $&$<-9.04                 $&$\mathbf{<-9.43}       $&                   6\\
       G45.47+0.05 0&$12.71^{+0.09}_{-0.07}$&$\mathbf{13.48^{+0.32}_{-0.50}}$&$4.46^{+0.28}_{-0.28}$&$\mathbf{5.21^{+0.40}_{-0.34}}$&$-10.23^{+0.28}_{-0.31}$&$\mathbf{-10.21^{+0.21}_{-0.19}}$&                   2\\
       G45.47+0.05 1&$\mathbf{<11.91}       $&$<11.92                 $&$\mathbf{<5.34}       $&$<5.23                 $&$\mathbf{<-8.65}       $&$<-8.57                 $&                   5\\
       G48.61+0.02 0&$12.06^{+0.09}_{-0.09}$&$\mathbf{12.53^{+0.13}_{-0.17}}$&$4.68^{+0.23}_{-0.25}$&$\mathbf{5.29^{+0.22}_{-0.24}}$&$-11.10^{+0.25}_{-0.26}$&$\mathbf{-11.25^{+0.13}_{-0.16}}$&                   2\\
       G48.61+0.02 1&$\mathbf{<11.88}       $&$<11.91                 $&$\mathbf{<5.09}       $&$<4.47                 $&$\mathbf{<-8.61}       $&$<-8.58                 $&                   5\\
       G48.61+0.02 2&$11.54^{+0.14}_{-0.13}$&$\mathbf{11.94^{+0.18}_{-0.23}}$&$4.60^{+0.28}_{-0.39}$&$\mathbf{5.22^{+0.23}_{-0.36}}$&$-11.54^{+0.33}_{-0.33}$&$\mathbf{-11.76^{+0.14}_{-0.19}}$&                   2\\
       G50.32+0.68 0&$>11.71                 $&$\mathbf{>12.41}       $&$>4.61                 $&$\mathbf{>5.31}       $&$>-11.77                 $&$\mathbf{>-12.63}       $&                   8\\
       G60.88-0.13 0&$12.24^{+0.18}_{-0.25}$&$\mathbf{12.35^{+0.09}_{-0.09}}$&$3.20^{+1.19}_{-1.16}$&$\mathbf{4.51^{+0.27}_{-0.28}}$&$-9.44^{+1.29}_{-1.43}$&$\mathbf{-10.64^{+0.28}_{-0.31}}$&                   2\\
       G61.48+0.09 0&$>12.51                 $&$\mathbf{>12.62}       $&$>5.07                 $&$\mathbf{>5.19}       $&$>-11.27                 $&$\mathbf{>-12.33}       $&                   8\\
       G69.54-0.98 0&                   -&                   -&                   -&                   -&                   -&                   -&                  11\\
       G70.29+1.60 0&$12.21^{+0.09}_{-0.08}$&$\mathbf{12.25^{+0.08}_{-0.08}}$&$4.67^{+0.23}_{-0.23}$&$\mathbf{4.74^{+0.23}_{-0.24}}$&$-10.94^{+0.24}_{-0.26}$&$\mathbf{-10.97^{+0.23}_{-0.26}}$&                   2\\
       G70.29+1.60 1&$<11.53                 $&$\mathbf{<11.55}       $&$<4.50                 $&$\mathbf{<4.67}       $&$<-8.92                 $&$\mathbf{<-8.98}       $&                   6\\
       G70.33+1.59 0&$13.16^{+0.09}_{-0.14}$&$\mathbf{13.41^{+0.19}_{-0.35}}$&$4.64^{+0.34}_{-0.32}$&$\mathbf{4.83^{+0.39}_{-0.37}}$&$-9.96^{+0.22}_{-0.31}$&$\mathbf{-9.90^{+0.21}_{-0.26}}$&                   2\\
   IRAS 20051+3435 0&$\mathbf{12.20^{+0.11}_{-0.10}}$&$12.23^{+0.04}_{-0.05}$&$\mathbf{4.12^{+0.39}_{-0.41}}$&$4.11^{+0.21}_{-0.23}$&$\mathbf{-10.40^{+0.45}_{-0.46}}$&$-10.35^{+0.22}_{-0.22}$&                   3\\
       G41.74+0.10 0&$12.25^{+0.17}_{-0.23}$&$\mathbf{12.48^{+0.10}_{-0.09}}$&$2.99^{+0.99}_{-1.18}$&$\mathbf{4.50^{+0.28}_{-0.31}}$&$-9.23^{+1.31}_{-1.22}$&$\mathbf{-10.50^{+0.29}_{-0.32}}$&                   2\\
       G41.74+0.10 1&$\mathbf{<12.12}       $&$<0.00                 $&$\mathbf{<4.72}       $&$<0.00                 $&$\mathbf{<-8.37}       $&$<0.00                 $&                   5\\
       G41.74+0.10 2&$\mathbf{<12.18}       $&$<0.00                 $&$\mathbf{<3.21}       $&$<0.00                 $&$\mathbf{<-8.91}       $&$<0.00                 $&                   5\\
       G41.74+0.10 3&$\mathbf{<12.17}       $&$<0.00                 $&$\mathbf{<4.70}       $&$<0.00                 $&$\mathbf{<-8.32}       $&$<0.00                 $&                   5\\
       G41.74+0.10 4&$\mathbf{<12.27}       $&$<0.00                 $&$\mathbf{<4.11}       $&$<0.00                 $&$\mathbf{<-8.22}       $&$<0.00                 $&                   5\\
      IRDC 1923+13 0&$\mathbf{<11.86}       $&$<11.84                 $&$\mathbf{<5.19}       $&$<4.47                 $&$\mathbf{<-8.63}       $&$<-8.65                 $&                   5\\
      IRDC 1923+13 1&$\mathbf{<11.86}       $&$<11.77                 $&$\mathbf{<5.30}       $&$<4.58                 $&$\mathbf{<-8.70}       $&$<-8.72                 $&                   5\\
      IRDC 1923+13 2&$\mathbf{<13.29}       $&$<11.53                 $&$\mathbf{<8.00}       $&$<5.00                 $&$\mathbf{<-8.95}       $&$<-8.96                 $&                   5\\
      IRDC 1916+11 0&$\mathbf{<12.05}       $&$<12.57                 $&$\mathbf{<5.16}       $&$<5.62                 $&$\mathbf{<-8.44}       $&$<-8.46                 $&                   5\\
      IRDC 1916+11 1&$\mathbf{<12.36}       $&$<12.40                 $&$\mathbf{<4.61}       $&$<5.03                 $&$\mathbf{<-8.13}       $&$<-8.16                 $&                   5\\
      IRDC 1916+11 2&$\mathbf{<12.09}       $&$<13.77                 $&$\mathbf{<5.55}       $&$<8.00                 $&$\mathbf{<-8.68}       $&$<-8.68                 $&                   5\\
}{
\tablenotetext{a}{The values used in this paper are shown in boldface.
Uncorrected values are listed in this column.  The filling-factor corrected
values are shown for comparison in the next column even though they were not used for analysis.}
\tablenotetext{b}{The values used in this paper are shown in boldface.
Filling-factor corrected values are listed in this column.  The uncorrected
values are shown for comparison in the previous column even though they not used for analysis.}
\tablenotetext{c}{Flags:\begin{enumerate}
  \item  No filling factor correction (no FFC) is the most reliable.                                             %1: 
  \item  Filling factor correction (FFC) is the most reliable                                                    %2: 
  \item  There is an ambiguity between low density / high abundance and low abundance / high density (no FFC)    %3: 
  \item  There is an ambiguity between low density / high abundance and low abundance / high density (FFC)       %4: 
  \item  Upper Limit (No FFC)                                                                                    %5: 
  \item  Upper Limit (FFC)                                                                                       %6: 
  \item  Lower Limit (No FFC)                                                                                    %7: 
  \item  Lower Limit (FFC)                                                                                       %8: 
  \item  Unreliable estimate because of continuum / filling factor uncertainty.                                  %9: 
  \item  No limit (S/N)                                                                                         %10:
  \item  Optically Thick                                                                                        %11:
\end{enumerate}
}}



\subsection{Systematic Errors: Absorption Geometry}
There are potential systematic errors associated with geometric assumptions,
i.e. the filling factor.  There are four geometric configurations possible;
these are outlined in Table \ref{tab:systematics}.  The ``small source''
geometry (3 and 4) is technically impossible given that the CMB is always
present in these observations, but it is equivalent to the scenario in which
the small illuminating compact source (\uchii) is much brighter than the CMB in
the beam.  The second column shows the effects of applying the `true' filling
factor correction for errors 2 and 4.  For error type 3, the optical depth will
only be overestimated if the absorber is ``corrected'' to be smaller than the
background source (i.e., if a correction is applied when none should have
been).

Figure \ref{fig:ffcdependence} shows the effects of incorrect geometric
assumptions.  Type 1 and 3 errors - i.e. filling factor overcorrections -  will
result in measurements of column and abundance that are \emph{greater} than
the real values, while type 2 and 4 errors will result in column and abundance
measurements that are \emph{lower} than the real values.

Additionally, it is possible that an observation will include a beam-filling,
low-density source that will contribute negligibly in \twotwo\ line absorption
but substantially in \oneone\ absorption over most of the beam area.  This type
of error will result in an underestimate of the volume density.

Since these errors are failures of assumptions, they cannot be quantified, but
Figure \ref{fig:ffc} shows the effects of correcting for these errors to the
extent possible with the available data.

\Table{lcc}{\formaldehyde\ Geometric Systematic Errors}
{\colhead{Real Geometry} & \colhead{Assumed filling factor $= 1$} & \colhead{Assumed filling factor $< 1$} \\}
{tab:systematics}
{
1. Beam-filling source, beam-filling absorber & \tablenotemark{a}$\tau_M=\tablenotemark{b}\tau_R$ & $\tau_M> \tau_R$ \\
2. Beam-filling source, small absorber        & $\tau_M<\tau_R$ & $\tau_M= \tau_R$ \\
3. Small source, beam-filling absorber        & $\tau_M=\tau_R$ & $\tau_M>=\tau_R$ \\
4. Small source, smaller absorber             & $\tau_M<\tau_R$ & $\tau_M= \tau_R$ \\
}
{
\tablenotetext{a}{$\tau_M = $ measured optical depth}
\tablenotetext{b}{$\tau_R = $ real optical depth}
}

% \Table{ccc}{Geometric Systematic Errors}
% {\colhead{Geometry} & \colhead{filling factor $= 1$} & \colhead{filling factor $< 1$} \\}
% {tab:systematics}
% {
% 1. Beam-filling source, beam-filling absorber & \tablenotemark{a}$\tau_M=\tablenotemark{b}\tau_R$ & $\tau_M> \tau_R$ \\
% 2. Beam-filling source, small absorber        & $\tau_M<\tau_R$ & $\tau_M= \tau_R$ \\
% 3. Small source, beam-filling absorber        & $\tau_M=\tau_R$ & $\tau_M>=\tau_R$ \\
% 4. Small source, smaller absorber             & $\tau_M<\tau_R$ & $\tau_M= \tau_R$ \\
% }
% {
% \tablenotetext{a}{$\tau_M = $ measured optical depth}
% \tablenotetext{b}{$\tau_R = $ real optical depth}
% }

\subsection{RRLs}
Radio recombination lines are used to measure the velocity of the \uchii\ regions.
The recombination lines 75-77$\alpha$ were independently fitted with gaussians
because the signal-to-noise in each spectrum with a detection was high. Out of our 24
spectra, there were 21 H detections, 13 He detections, and 12 C detections;
Table \ref{tab:rrls76} shows the fitted parameters using the 76$\alpha$ lines
(75$\alpha$ and 77$\alpha$ were also measured but are not reported for
brevity).  For some of the analysis in later sections, we additionally use the
deeper and more careful RRL study by \citet{Roshi2005}, who observed 17 of our
sample in the 89-92$\alpha$ lines.  We attempted to measure carbon RRLs in the
\citet{Araya2002} spectra, who only measured hydrogen RRLs.  We detected one
carbon line in G61.48 and tentatively ($\sim2\sigma$) detected another three in
G32.80, G34.26, and G45.45; we report the low-significance detections in these
sources because of corresponding detections of C75-77$\alpha$.

We compare the central velocities of the H and C $\alpha$ lines to the
velocities of the \formaldehyde\ absorption lines on a case-by-case basis in
Figures \ref{fig:g33pt13spectrum}-\ref{fig:g61.48+0.09spectrum}.  The spectral line
profiles are used to fit the observations into the models discussed in detail
in Sections \ref{sec:lineprofiles} and \ref{sec:scenarios}.

\Table{lccccccccc}{Measured RRL 76 properties}
{
 & \multicolumn{3}{c}{H}&\multicolumn{3}{c}{He}&\multicolumn{3}{c}{C}\\
\cline{3-3} \cline{6-6} \cline{9-9} \\
\colhead{Source}&\colhead{Peak}&\colhead{Center}&\colhead{FWHM}&\colhead{Peak}&\colhead{Center}&\colhead{FWHM}&\colhead{Peak}&\colhead{Center}&\colhead{FWHM}\\
\colhead{Name}&\colhead{H76$\alpha$\tablenotemark{a}}&\colhead{H76$\alpha$}&\colhead{H76$\alpha$}&\colhead{He76$\alpha$}&\colhead{He76$\alpha$}&\colhead{He76$\alpha$}&\colhead{C76$\alpha$}&\colhead{C76$\alpha$}&\colhead{C76$\alpha$}\\
\colhead{           }&\colhead{(Jy)}&\colhead{(\kms)}&\colhead{(\kms)}&\colhead{(Jy)}&\colhead{(\kms)}&\colhead{(\kms)}&\colhead{(Jy)}&\colhead{(\kms)}&\colhead{(\kms)}\\ }
{tab:rrls76}{
         G32.80+0.19&               0.622&               15.69&               12.09&               0.066&               16.49&                9.25&               0.015&               15.40&                8.27\\
                    &             (0.001)&              (0.03)&              (0.03)&             (0.002)&              (0.36)&              (0.38)&             (0.002)&              (1.45)&              (1.64)\\
         G33.13-0.09&               0.067&               73.49&               14.10&                   -&                   -&                   -&                   -&                   -&                   -\\
                    &             (0.001)&              (0.17)&              (0.17)&                    &                    &                    &                    &                    &                    \\
         G33.92+0.11&               0.157&              101.86&               12.16&               0.013&               99.07&               13.60&                   -&                   -&                   -\\
                    &             (0.001)&              (0.07)&              (0.07)&             (0.001)&              (0.87)&              (0.87)&                    &                    &                    \\
         G34.26+0.15&               0.367&               54.68&               10.43&               0.034&               51.98&                6.54&               0.026&               59.54&                5.66\\
                    &             (0.004)&              (0.06)&              (0.09)&             (0.002)&              (0.46)&              (0.49)&             (0.002)&              (0.55)&              (0.56)\\
                    &               0.251&               37.46&               22.76&                   -&                   -&                   -&                   -&                   -&                   -\\
                    &             (0.003)&              (0.29)&              (0.12)&                    &                    &                    &                    &                    &                    \\
         G35.20-1.74&               1.016&               47.94&               10.70&               0.105&               48.26&                8.27&               0.045&               44.18&                4.05\\
                    &             (0.002)&              (0.02)&              (0.02)&             (0.002)&              (0.21)&              (0.21)&             (0.003)&              (0.33)&              (0.33)\\
         G35.57-0.03&               0.036&               52.38&               13.71&                   -&                   -&                   -&                   -&                   -&                   -\\
                    &             (0.001)&              (0.41)&              (0.41)&                    &                    &                    &                    &                    &                    \\
         G35.58+0.07&               0.044&               46.68&               10.55&               0.007&               43.15&                6.30&                   -&                   -&                   -\\
                    &             (0.001)&              (0.20)&              (0.20)&             (0.001)&              (0.94)&              (0.94)&                    &                    &                    \\
         G37.87-0.40&               0.446&               59.99&               15.47&               0.042&               60.16&               11.88&               0.018&               59.27&                7.93\\
                    &             (0.001)&              (0.08)&              (0.07)&             (0.001)&              (0.55)&              (0.55)&             (0.001)&              (0.98)&              (0.89)\\
                    &               0.049&               26.21&               10.49&                   -&                   -&                   -&                   -&                   -&                   -\\
                    &             (0.002)&              (0.52)&              (0.42)&                    &                    &                    &                    &                    &                    \\
         G41.74+0.10&               0.038&               11.46&               13.89&                   -&                   -&                   -&                   -&                   -&                   -\\
                    &             (0.001)&              (0.29)&              (0.29)&                    &                    &                    &                    &                    &                    \\
         G43.89-0.78&               0.103&               54.98&               10.83&               0.010&               54.18&                7.72&               0.007&               54.08&                0.82\\
                    &             (0.001)&              (0.08)&              (0.08)&             (0.001)&              (0.68)&              (0.68)&             (0.002)&              (0.34)&              (0.30)\\
         G45.07+0.13&               0.041&               58.22&               10.05&                   -&                   -&                   -&                   -&                   -&                   -\\
                    &             (0.004)&              (0.41)&              (0.64)&                    &                    &                    &                    &                    &                    \\
                    &               0.043&               41.57&               20.01&                   -&                   -&                   -&                   -&                   -&                   -\\
                    &             (0.003)&              (1.53)&              (0.59)&                    &                    &                    &                    &                    &                    \\
         G45.12+0.13&               0.461&               58.70&               17.42&               0.039&               59.85&               10.70&               0.023&               59.58&               12.37\\
                    &             (0.002)&              (0.08)&              (0.08)&             (0.005)&              (2.50)&              (1.62)&             (0.003)&              (4.92)&              (3.76)\\
         G45.45+0.06&               0.493&               55.38&               11.80&               0.050&               56.41&                8.02&               0.014&               63.40&               10.42\\
                    &             (0.001)&              (0.03)&              (0.03)&             (0.004)&              (0.93)&              (0.56)&             (0.002)&              (4.57)&              (3.05)\\
         G45.47+0.05&               0.040&               64.01&               14.51&                   -&                   -&                   -&                   -&                   -&                   -\\
                    &             (0.001)&              (0.41)&              (0.41)&                    &                    &                    &                    &                    &                    \\
         G48.61+0.02&               0.076&               16.77&               10.53&               0.007&               16.33&                7.73&               0.006&               19.08&                4.71\\
                    &             (0.001)&              (0.14)&              (0.14)&             (0.001)&              (1.30)&              (1.39)&             (0.001)&              (1.26)&              (1.29)\\
         G50.32+0.68&               0.034&               26.94&               10.27&                   -&                   -&                   -&                   -&                   -&                   -\\
                    &             (0.001)&              (0.27)&              (0.27)&                    &                    &                    &                    &                    &                    \\
         G60.88-0.13&               0.067&               18.30&                9.08&                   -&                   -&                   -&               0.023&               21.77&                2.54\\
                    &             (0.001)&              (0.12)&              (0.12)&                    &                    &                    &             (0.002)&              (0.19)&              (0.19)\\
         G69.54-0.98&               0.017&                3.69&               16.24&                   -&                   -&                   -&                   -&                   -&                   -\\
                    &             (0.001)&              (0.64)&              (0.64)&                    &                    &                    &                    &                    &                    \\
         G70.33+1.59&               0.343&              -19.18&               12.59&               0.032&              -20.14&               10.12&               0.025&              -21.67&                3.28\\
                    &             (0.001)&              (0.05)&              (0.05)&             (0.001)&              (0.44)&              (0.46)&             (0.002)&              (0.33)&              (0.33)\\
         G70.29+1.60&               0.545&              -26.97&               17.82&               0.042&              -26.32&               14.53&               0.032&              -24.78&                4.71\\
                    &             (0.001)&              (0.12)&              (0.09)&             (0.001)&              (0.64)&              (0.68)&             (0.002)&              (0.36)&              (0.42)\\
                    &               0.066&              -64.41&               13.12&                   -&                   -&                   -&                   -&                   -&                   -\\
                    &             (0.002)&              (0.70)&              (0.50)&                    &                    &                    &                    &                    &                    \\
         G61.48+0.09&               0.566&               25.96&               11.16&               0.046&               28.80&                7.86&               0.059&               21.27&                2.48\\
                    &             (0.001)&              (0.02)&              (0.02)&             (0.001)&              (0.24)&              (0.24)&             (0.002)&              (0.11)&              (0.11)\\
}{\tablenotetext{a}{Some H lines were fit with two gaussian components, in which case the second fit component is on the second line below.  Errors (1$\sigma$) are indicated by the numbers in parentheses on the line below the measurement.}}


%\begin{landscape}
\Table{lccccccccc}{Measured RRL properties}
{\colhead{Source Name}&\colhead{Peak H75$\alpha$\tablenotemark{a}}&\colhead{Center H75$\alpha$}&\colhead{FWHM  H75$\alpha$}&\colhead{Peak He75$\alpha$}&\colhead{Center He75$\alpha$}&\colhead{FWHM He75$\alpha$}&\colhead{Peak C75$\alpha$}&\colhead{Center C75$\alpha$}&\colhead{FWHM C75$\alpha$}\\
\colhead{           }&\colhead{(Jy)}&\colhead{(\kms)}&\colhead{(\kms)}&\colhead{(Jy)}&\colhead{(\kms)}&\colhead{(\kms)}&\colhead{(Jy)}&\colhead{(\kms)}&\colhead{(\kms)}\\ }
{tab:rrls}{
         G32.80+0.19&       0.663 (0.002)&        15.59 (0.03)&        12.10 (0.03)&       0.072 (0.002)&        15.57 (0.28)&        10.28 (0.31)&       0.021 (0.002)&        13.35 (0.73)&         5.79 (0.76)\\
         G33.13-0.09&       0.070 (0.001)&        73.28 (0.18)&        14.34 (0.18)&                   -&                   -&                   -&                   -&                   -&                   -\\
         G33.92+0.11&       0.162 (0.001)&       101.76 (0.07)&        12.25 (0.07)&       0.015 (0.001)&        95.35 (0.86)&        15.18 (0.86)&                   -&                   -&                   -\\
         G34.26+0.15&       0.528 (0.011)&        54.26 (0.18)&        12.32 (0.11)&       0.045 (0.002)&        51.83 (0.77)&         8.65 (0.71)&       0.037 (0.002)&        56.57 (0.84)&         7.41 (0.74)\\
                    &       0.210 (0.005)&        26.71 (0.94)&        17.27 (0.51)&                   -&                   -&                   -&                   -&                   -&                   -\\
         G35.20-1.74&       1.065 (0.016)&        47.89 (0.19)&        10.86 (0.19)&       0.122 (0.018)&        47.80 (1.45)&         8.38 (1.52)&       0.054 (0.021)&        45.04 (2.77)&         5.98 (2.84)\\
         G35.57-0.03&       0.035 (0.002)&        51.82 (0.88)&        15.45 (0.88)&                   -&                   -&                   -&                   -&                   -&                   -\\
         G35.58+0.07&       0.046 (0.001)&        46.52 (0.21)&        10.16 (0.21)&       0.005 (0.001)&        46.14 (2.25)&        17.01 (2.25)&                   -&                   -&                   -\\
         G37.87-0.40&       0.475 (0.001)&        60.07 (0.09)&        15.33 (0.07)&       0.054 (0.001)&        60.34 (0.76)&        11.28 (0.58)&       0.029 (0.002)&        60.86 (1.28)&        10.11 (0.95)\\
                    &       0.059 (0.002)&        28.39 (0.46)&        10.19 (0.36)&                   -&                   -&                   -&                   -&                   -&                   -\\
         G43.89-0.78&       0.106 (0.001)&        54.56 (0.09)&        10.85 (0.09)&       0.012 (0.001)&        53.92 (0.65)&         7.96 (0.65)&       0.007 (0.001)&        54.65 (0.64)&         2.69 (0.64)\\
         G45.07+0.13&       0.073 (0.006)&        55.30 (1.57)&        13.13 (0.70)&                   -&                   -&                   -&                   -&                   -&                   -\\
                    &       0.030 (0.006)&        29.77 (4.10)&        13.58 (1.88)&                   -&                   -&                   -&                   -&                   -&                   -\\
         G45.12+0.13&       0.499 (0.003)&        58.47 (0.13)&        17.17 (0.13)&       0.056 (0.004)&        56.82 (2.42)&        12.19 (2.01)&       0.039 (0.006)&        56.10 (2.84)&         9.59 (2.25)\\
         G45.45+0.06&       0.511 (0.001)&        55.29 (0.04)&        11.95 (0.04)&       0.058 (0.002)&        55.61 (0.73)&         9.09 (0.56)&       0.018 (0.002)&        57.70 (2.54)&        10.20 (2.16)\\
         G45.47+0.05&       0.040 (0.001)&        66.30 (0.43)&        14.78 (0.43)&                   -&                   -&                   -&                   -&                   -&                   -\\
         G48.61+0.02&       0.076 (0.001)&        16.78 (0.11)&         9.77 (0.11)&       0.007 (0.001)&        16.34 (1.47)&        10.58 (1.63)&       0.007 (0.001)&        17.35 (1.02)&         5.25 (1.02)\\
         G50.32+0.68&       0.034 (0.001)&        26.69 (0.21)&         8.77 (0.21)&                   -&                   -&                   -&                   -&                   -&                   -\\
         G60.88-0.13&       0.070 (0.001)&        18.39 (0.12)&         9.01 (0.12)&                   -&                   -&                   -&       0.022 (0.002)&        22.03 (0.20)&         2.48 (0.20)\\
         G61.48+0.09&       0.586 (0.005)&        25.83 (0.12)&        11.33 (0.12)&       0.050 (0.006)&        27.77 (1.25)&         8.68 (1.25)&       0.054 (0.010)&        21.78 (0.70)&         3.25 (0.70)\\
         G69.54-0.98&       0.021 (0.001)&         3.98 (0.49)&        13.63 (0.49)&                   -&                   -&                   -&                   -&                   -&                   -\\
         G70.29+1.60&       0.592 (0.002)&       -27.27 (0.16)&        17.76 (0.11)&       0.047 (0.002)&       -27.05 (0.65)&        13.56 (0.71)&       0.029 (0.003)&       -24.75 (0.49)&         4.53 (0.55)\\
                    &       0.081 (0.003)&       -63.67 (0.83)&        13.51 (0.55)&                   -&                   -&                   -&                   -&                   -&                   -\\
         G70.33+1.59&       0.358 (0.001)&       -19.32 (0.04)&        12.67 (0.04)&       0.039 (0.001)&       -21.29 (0.37)&        11.01 (0.41)&       0.024 (0.002)&       -21.98 (0.41)&         4.88 (0.41)\\
     IRAS 20051+3435&                   -&                   -&                   -&                   -&                   -&                   -&                   -&                   -&                   -\\
         G41.74+0.10&       0.042 (0.001)&        11.02 (0.29)&        13.27 (0.29)&                   -&                   -&                   -&                   -&                   -&                   -\\
        IRDC 1923+13&                   -&                   -&                   -&                   -&                   -&                   -&                   -&                   -&                   -\\
        IRDC 1916+11&                   -&                   -&                   -&                   -&                   -&                   -&                   -&                   -&                   -\\
}{\tablenotetext{a}{Some H lines were fit with two gaussian components, in which case the second fit component is on the second line below}}
\end{landscape}


% We derived electron
% temperatures using Equation 1 from \citet{Quireza2006},
% \begin{equation}
%     \left(
%     \frac{T_e^∗}{K}
%     \right)
%     = \left[ 7103.3 \left( \frac{\nu_L}{\textrm{GHz}} \right)^{1.1}
%     \left(\frac{T_C}{T_L(\textrm{H+})}\right)
%     \left(\frac{\Delta\nu(\textrm{H+})}{\kms}\right)^{-1}
%     \left(1+\frac{n(^4\textrm{He+})}{n(\textrm{H+})}\right)^{-1}
%     \right]^{0.87}
% \end{equation}
% which assumes local
% thermodynamic equilibrium, negligible pressure broadening, and a simple planar
% structure of the \ion{H}{2} region.  These assumptions are not robust and there is a
% strong dependence of the line-to-continuum ratio on the region geometry
% \citep{Lockman1978}, so the derived electron temperatures are only used for
% comparison within our own sample (see figure \ref{fig:TvT}).  
% 
% \Figure{Derived_TAveVsT110}
% {Comparison of the electron temperature derived from the 75-77$\alpha$ lines
% and the 110$\alpha$ line using the assumptions of \citet{Quireza2006}.  The
% lack of correlation demonstrates the unreliable nature of these assumptions,
% confirming the caveats brought up in \citet{Lockman1978}. [XXXX Is this
% interesting enough to keep?  If not, I should probably eliminate the above
% paragraph that refers to this figure too]}
% {fig:TvT}{0.5}{0}
% 
% The Helium/Hydrogen line ratio ranged from 0.03 to 0.11 with an average $<He/H>
% = 0.074 \pm 0.025$, consistent with previous works
% \citep[e.g.][]{Churchwell1974}.  The Carbon/Hydrogen ratio was much more
% scattered, ranging from 0.005 to 0.095.  Excluding the highest point,
% G60.88-0.13, the average is $<C/H> = 0.014 \pm 0.005$.  However, the carbon
% lines are not expected to come from the same parcels of gas as the hydrogen
% because of carbon's lower ionization energy, and there were many
% non-detections, so the carbon emission is probably not from the \ion{H}{2}
% region but may be from a local photon-dominated region.


\section{Results}
\label{sec:results}
\subsection{Derived Properties}
The average properties of the spectral line components associated with the \uchii\ regions
and the other spectral lines representing molecular clouds are
shown in Table \ref{tab:properties}.  The table includes the mean and median
only of spectral lines with both \oneone\ and \twotwo\ detections that yielded
measurements of density; upper and lower limits are not included.  The full
results are presented in Table \ref{tab:h2coderived}.

\Table{cccccccc}
{Inferred properties}
{
          &  \uchii &&& Other Lines (GMC)&&& \\
\hline
%          &  \hline    &&& \hline     &&& \\ % can't have 2 \hlines not separated by \\
Parameter & Median\tablenotemark{a} & Mean \tablenotemark{a} & RMS \tablenotemark{a} & Median \tablenotemark{b} & 
Mean \tablenotemark{b} & RMS \tablenotemark{b} & KS PTE \\}
{tab:properties}
{
log(\hh~Density) (\percc)          &       4.95 &       4.91 &       0.27 &       4.49 &       4.61 &       0.32 &      0.022 \\
log(\ortho~Column) (\persc)         &      12.59 &      12.59 &       0.44 &      11.86 &      11.83 &       0.20 &    6\ee{-6} \\
$X(\ortho)$                         &     -10.84 &     -10.80 &       0.46 &     -11.16 &     -11.26 &       0.45 &      0.028 \\
% unscaled version log(\hh Density) (\percc)      &       5.19 &       5.17 &       0.30 &       4.69 &       4.81 &       0.33 &     0.014 \\
% unscaled version log(\ortho\ Column) (\persc)   &      12.98 &      12.94 &       0.50 &      12.03 &      12.00 &       0.24 &   2.2e-06 \\
% unscaled version $X(\formaldehyde)$ \perkmspc  &     -10.82 &     -10.71 &       0.51 &     -11.22 &     -11.30 &       0.48 &    0.0039 \\
} {
\tablenotetext{a}{Spectral line components associated with \uchii\ regions}
\tablenotetext{b}{Other spectral lines (associated with line-of-sight molecular clouds)}
}

There is statistical evidence that the deepest spectral line components have higher
\formaldehyde\ column and/or abundance than the other (GMC) components (Table
\ref{tab:properties}).  It is unlikely that this difference could be caused by
underestimates of the optical depths in the GMC components (type 2 and 4
errors, see Table \ref{tab:systematics}) because the filling factor correction
should tend to cancel out these errors.  However, it is possible that, in those
cases where the \ion{H}{2} emission and the CMB emission in the beam are the
same order of magnitude, type 1 errors have occurred: the \ion{H}{2} region
absorber is much larger than the \ion{H}{2} region and a significant fraction
of the spectral line depth comes from absorption against the CMB; this error
should have little effect on the derived density (see Figure
\ref{fig:ffcdependence}) but may lead to overestimates of the derived column
density.  

Each identified Gaussian component was associated with an UC\ion{H}{2} region
if it was within 5 \kms\ of the RRL peak, since RRLs are assumed to be
generated in the UC\ion{H}{2} regions.  Any spectral lines blended with the
\uchii\ \formaldehyde\ lines were also associated with the \uchii\ region.
Other velocity components, including those without corresponding RRL
detections, were assumed to be from GMCs along the line of sight or part of the
larger cloud not directly associated with the UC\ion{H}{2} region; 29
components were associated with UC\ion{H}{2} regions and 46 were associated
with unrelated line-of-sight GMCs (Table \ref{tab:other}).

The density difference between the two populations is significant by a
Kolmogorov-Smirnov (KS) test with $\sim2\%$ probability of being drawn from the
same distribution (the `probability to exceed' or PTE in Table
\ref{tab:properties}). This result is in contradiction to the results of
\citet{Wadiak1988}, who found no significant density difference between ``warm
clouds'' and ``cold clouds'' selected and observed in the same manner (though
with larger beams).  The difference is likely because the larger beam sizes in
their study and a failure to include the continuum contribution of the CMB
(which is more substantial in a larger beam, especially at 2 cm), resulting in
a type 3 error and an underestimate of density for their ``warm clouds'' in
particular.

%The inferred column densities of \hh\ are as high as $N(\hh)=10^{24.1}
%\persc$, and the highest measured lower-limit is $N(\hh)>10^{24.0}$.  The
%densities measured ranged from $10^{4.5}$ to $>10^{6.3}$ \percc: the
%UC\ion{H}{2} regions appear to be associated with moderate to high density gas,
%one to two orders of magnitude greater than the surrounding ``clump''.  In the
%cases in which it was not possible to measure a density because of the line
%properties, it is likely that both the column and volume density are much
%higher.

The measured \hh\ densities do not display any trend with heliocentric distance
over the range 2-14 kpc, contrasing with mm-continuum surveys of star forming
regions that tend to measure lower densities at greater distances
\citep{Reid2010}.  The lack of correlation in Figure \ref{fig:densvsdist}
demonstrates the strength of the \formaldehyde\ densitometry method: the
properties of star-forming gas can be explored throughout the galaxy with
distance bias largely removed.   Similarly, no trend with
Galactocentric distance was readily apparent.

\OneColFigure{figures_chH2CO/Derived_DensityVsDistance}
{Derived density plotted against kinematic distance.  No trend is obvious, demonstrating
that the \formaldehyde\ densitometer is not biased by source distance.
Black squares represent GMCs along the line of sight; red triangles represent
UC\ion{H}{2} regions.}
{fig:densvsdist}{0.30}{0}

Densities were measured within a range $10^4\ \percc\ \lesssim n(\hh)
\lesssim10^6$ \percc\ due to sensitivity cutoffs at low densities and
thermalization of the spectral line ratio (ratio $\rightarrow$ 1) at high densities (see 
Section \ref{sec:strengthsweaknesses} for a discussion of the limitations of the
densitometer).  On the high density end, a lower limit on the density remains
interesting, as densities $n(\hh)\gtrsim10^6$ \percc\ are close to those of low-mass
protostellar cores and are a strong indication of runaway gravitational
collapse, since such high densities are rarely observed in non-star-forming
regions.  On the low density end, it should be possible to detect the \twotwo\
transition with sensitivity improvements $\sim2-10\times$, a consideration that
will govern the allocated time-on-source for future \twotwo\ observations.

\subsection{Free-free Contribution to 1.1 mm Flux Density Measurements}
It is expected that all young star-forming regions should be dust-rich and
therefore bright at 1.1 mm.  We therefore compare the BGPS 1.1 mm, GBT 2 cm,
and Arecibo 6 cm continuum measurements for sources covered by the BGPS in
Figure \ref{fig:MassVsCm}.  For a flat-spectrum \citep[$\alpha \approx -0.1$,
$\tau_{ff}<<1$;][]{rohlfs} free-free continuum source, the 2 cm flux density
should be $1.34\times$ the 1.1 mm flux density.  For an optically thick source,
$S_{1.1 mm} = 330~S_{2 cm}$.

The objects targeted in our survey include 9 of the 13 brightest ($S_{1.1 mm,40\arcsec}>1.5$
Jy) sources in the range $32<\ell<48$, and 11 of 26 with $S_{1.1mm,
40\arcsec}>1.0$ Jy. We use flux density measurements from the 40\arcsec\ apertures in the BGPS catalog
because they are most appropriate for determining peak brightness of point-like
sources \citep{Rosolowsky2010}.  Out of the sample within the BGPS survey area,
6 of 15 sources have free-free fractions of at least 30\%, but potentially much higher if 
the free-free emission is not optically thin.  Since the sample was selected from
well-known \uchii\ regions, these (rather incomplete) statistics are a warning
that most of the brightest 1.1 mm emission sources in the BGPS are likely to be
active \uchii\ regions and therefore may include a significant contribution
from free-free emission to their measured flux densities (Figure
\ref{fig:freefreefraction}).  The same warning applies to other mm-wavelength galactic
plane surveys, though the contamination should be less severe at shorter wavelengths.

\FigureTwo{figures_chH2CO/Derived_BGPSVs2cm_brightest_fit}{figures_chH2CO/Derived_BGPSVs6cm_brightest_fit}
{Bolocam 1.1 millimeter flux density versus the cm continuum flux density at 2
cm (left) and 6 cm (right).  The BGPS 1.1 mm flux density is moderately
correlated with both cm continuum measurements; the legend shows the regression
parameter.  The expectation for optically-thin free-free
emission ( $\alpha = -0.1$, dotted) and for intermediate spectral index emission
($\alpha > 0$, dashed) are shown to illustrate that some sources have
significant free-free contributions at 1.1 mm (the optically thick case is not
shown for either 2 or 6 cm because it does not fit on the plot).
The legend shows the predicted flux densities for a given spectral index
$\alpha$, the regression parameter $r$, and its likelihood $p$.  The brighter
sources are likely to be less optically thick in the free-free continuum than
the faint sources. }
{fig:MassVsCm}{1}

\OneColFigure{figures_chH2CO/Derived_FreeFreeFraction1.1mm_brightest}
{The distribution of free-free contributions to the 1.1 mm flux density
assuming the \uchii\ region is optically thin at 2 cm, $f_{ff} = (S_{2
cm}/1.34)/S_{1.1 mm}$.  While 9 sources are either dust-dominated or optically
thick at 2 cm, 6 sources have free-free contributrions of 30\% or greater.  The
other sources in the sample were missing 1.1 mm flux density measurements
because they are outside the BGPS survey area.
}
{fig:freefreefraction}{0.30}{0}

In order to evaluate the impact of this conclusion on the BGPS, we examine the
flux distribution of 6 cm continuum sources from the MAGPIS survey compared to
the BPGS in the same area, $5<\ell<42$ and $|b|<0.42$, which is the full range
of the MAGPIS survey excluding the galactic center, where the BGPS catalog
follows a different flux distribution \citep{Bally2010}.  

In Figure
\ref{fig:contfluxdistr}, we plot histograms of the MAGPIS 6 cm flux density and
the BGPS 40\arcsec\ aperture flux density along with the best-fit power-law
distribution line \footnote{The power law was fit using the python translation
of the \citet{Clauset2009} power-law fitter provided at
\url{http://code.google.com/p/agpy/wiki/PowerLaw}.  The fitter computes the
maximum likelihood value of the power-law $\alpha$ and the cutoff of the
distribution, below which a power law is no longer valid either because of
incompleteness or a change in the underlying distribution.}.  Since the 6 cm
power-law distribution is
shallower than the 1.1 mm distribution, the 6 cm sources can dominate at high
flux densities, although the power-law fit for the 6 cm sources significantly
overpredicts the highest-flux bins and therefore the power-law is not an acceptable fit
above $S_{6 cm}>1$ Jy \footnote{We have tested the consistency of the two data
sets with a low-cutoff power-law distribution by the Monte-Carlo process
described in \citet{Clauset2009}.  The BGPS 40\arcsec\ aperture flux densities
are consistent with a power-law distribution at the $p=0.64$ level, while the
MAGPIS 6 cm fluxes are inconsistent, with $p<0.001$ (where p measures the
probability that the data are drawn from a low-cutoff power-law distribution) }.
The dashed line in Figure \ref{fig:contfluxdistr} shows the best-fit power-law
distribution of the MAGPIS flux densities scaled down by
0.67, which is the expected decrement for an optically thin free-free source 
from 6 cm to 1.1 mm
(spectral index $\alpha=-0.1$).  

Figure \ref{fig:contfluxdistr}b shows the
ratio of the BGPS to the MAGPIS best-fit power-law distribution, indicating
that the free-free contamination fraction is only large ($\sim10\%$) at values
much greater than the valid range of the 6 cm power law fit, which overpredicts the
number of sources at $S_{6 cm} \approx 1 $ Jy.  However, if any
of these sources are \emph{not} optically thin at 6 cm, this fraction could be
much larger.  Additionally, these numbers only describe the sources in which
\emph{all} of the 1.1 mm flux is free-free emission; the implication remains
that a large number of 1.1 mm sources have a substantial (if not dominant)
free-free contribution.  

Finally, we emphasize that unless a large fraction of 6 cm sources are
optically thick in free-free continuum, the lower flux-density BGPS
dust-continuum sample should be negligibly contaminated by free-free emission
sources, but the brightest BGPS sources may have a significant free-free
contribution.

\FigureTwo{figures_chH2CO/fluxdistribution_6cm_1mm_fits}{figures_chH2CO/fluxdistribution_6cm_1mm_fits_ratio}
{{\it Left:} Histograms of BGPS 1.1 mm 40\arcsec\ aperture flux densities (red)
and the MAGPIS 6 cm flux densities (black), and their respective best-fit
power-law distributions ($\alpha(1.1 mm)=2.41\pm0.03$, $\alpha(6
cm)=1.72\pm0.03$).  The dashed black line shows the MAGPIS best-fit power-law
scaled down to the expected flux density at 1.1 mm assuming all sources are
optically thin.  Both distributions appear to be reasonably well-fit by
power-laws above a cutoff (presumably set by completeness), although the power-law
significantly over-predicts the number of sources with $S_{6 cm}>1$Jy.  The
histograms are binned by 0.1 dex, and while the best-fit $\alpha$ and $x_{min}$
values are independent of the binning scheme, the normalization is not.
{\it Right:} The ratio of the number of MAGPIS 6 cm sources to BGPS 1.1 mm
sources as a function of flux density for the best-fit power laws.  Only 10 1.1
mm sources are detected above 5 Jy (in 40\arcsec\ apertures), so even the
brightest detected 1.1 mm sources are not purely free-free, but they probably 
have a substantial free-free component.}
{fig:contfluxdistr}{1}

\subsection{Distances}
\label{sec:distances}
We measure a kinematic distance to each source using the \citet{Reid2009}
rotation curve.  We resolved the Kinematic Distance Ambiguity (KDA) towards
each line of sight using a variety of methods described below.  The method
in \citet{Sewilo2004} allows a resolution in favor of the far
distance for \uchii\ regions with an intervening molecular absorption line at
more positive velocities in the first Galactic quadrant.  Associations with
infrared dark clouds (IRDCs) can resolve the KDA in favor of the near distance.
We compare our KDA resolutions to \citet{Anderson2009}, with whom we agree on
all common sources except for G33.13-0.09, which we place at the far distance
based on the \citet{Sewilo2004} method.  The derived distances are listed in
Table
\ref{tab:other}.

% Distance determination was necessary to measure total dust mass using the 1.1
% mm continuum from the BGPS.  A correlation between the millimeter-derived mass
% and the 2 cm / 6 cm continuum ratio (alternately, the spectral index $\alpha$
% from the relation $S_\nu \propto \nu^\alpha$) is observed for the UC\ion{H}{2}
% regions (Figure \ref{fig:MassVsAlpha}).  Note that the Y-axis is dependent
% on distance, but the X-axis is not directly dependent on distance.  However, if
% the typical source has both a diffuse, extended \ion{H}{2} component and a
% compact \uchii\ component, the measured 6 cm flux would be expected to increase
% more with distance more than the 2 cm flux.  However, the two highest 2 cm / 6
% cm ratios are observed in very distant objects ($S_2/S_6=1.7$ for G32.80+0.19
% at 12.9 kpc and G45.07+0.13 at 7.6 kpc), suggesting that the spectral index
% measurements are not substantially affected by diffuse \ion{H}{2}
% contamination.

% are these two paragraphs in disagreement?
% The correlation implies that objects with steeper spectral indices (higher
% continuum optical depths at higher frequencies) have higher masses. Since
% ultracompact \ion{H}{2} regions are expected to evolve from high to low optical
% depths as they expand \citep{Churchwell2002}, the data are consistent with a
% scenario in which massive stars blow out their host dust clump as they evolve
% from \uchii\ regions to diffuse \ion{H}{2} regions.

%Curiously, the most dust-dominated sources have the shallowest radio spectral
%indices, which would normally indicate an optically thin free-free spectrum
%(Figure \ref{fig:MassVsAlpha}).  However, the most compact \uchii\ regions are
%expected to be the most optically thick and the youngest.  What's going on
%here? XXXX


%Previous figure shows that 1.1mm mass is probably dominated by free-free emission
% \Figure{figures_chH2CO/Derived_MassVsContinuumRatio_brightest}
% {Bolocam 1.1 millimeter-derived mass assuming T$_{dust}=40$K versus the cm
% continuum flux ratio (or spectral index) for the sources within the BGPS with
% cm continuum detections at both 2 and 6 cm.  The systematic error range for the
% mm mass estimates is $\sim0.5$ dex because of temperature, calibration, and
% distance uncertainty, but the statistical errors are relatively small
% ($\sim0.1$ dex).  An evolutionary sequence from positive to negative spectral
% index and high to low dust mass is consistent with the standard \uchii\
% evolutionary sequence \citep{Churchwell2002}.}
% {fig:MassVsAlpha}{0.5}{0}

\subsubsection{Size Estimates}
\label{sec:sizecomp}
%We compare the mass estimated from the 1.1 mm continuum with a mass inferred
%from the \formaldehyde\ measurement.  The \formaldehyde-derived density can
%be used to infer a mass given a size scale and an assumption about geometry.
%For simplicity, we assume all clumps are spherical, which immediately makes our
%measurements at best order-of-magnitude estimates.

We estimate the source size using two methods.  First, we use the VLA
measurements of \uchii\ region sizes.  As stated in Section
\ref{sec:linedepth}, the VLA size measurements are very uncertain and are
simplifications of an evidently complicated geometry.  We estimate a spherical
radius $r=\sqrt{area/\pi}$.  Second, we assume the gas traced by \formaldehyde\
and the BGPS 1.1 mm images are the same and get a `size scale'
$r=2 N_{mm}(\hh)/n(\hh)$ where $n(\hh)$ is derived from the \formaldehyde\ line
ratio.

The sizes derived from the two methods are plotted against each other in Figure
\ref{fig:sizecomp}.  The sizes estimated from the two different methods are not
well correlated and disagree by around an order of magnitude in most sources.
The disagreement could be because of poor VLA-based size estimates, substantial
1.1 mm emission from low-density gas, or incorrect dust temperature or opacity
estimates.  While additional line-of-sight GMCs could in principle contribute
to the $N/n$ size estimate, the disagreement for sources even without
associated GMCs prevents this hypothesis from fully explaining the
disagreement.  Therefore, any quantities derived from the size - i.e.
mass, which depends on $r^3$ - are even less constrained.  We therefore do not
derive any quantities dependent on the intrinsic source size.
%We therefore only
%compare the 1.1 mm derived clump mass to other derived properties.

\OneColFigure{figures_chH2CO/Derived_SizeUCHIIvsSizeNn_brightest}
{A plot of the two derived sizes discussed in Section \ref{sec:sizecomp}.  The
two size estimates are at best very weakly correlated.  Because of the
substantial disagreement between the two methods, we choose not to explore any
parameters with a strong dependence on the size.  The plotted point size
indicates the number of associated line-of-sight GMCs, which in principle could
lead to an overestimate of the $N/n$ size because of additional mass included in
the 1.1 mm continuum measurement.}
{fig:sizecomp}{0.30}{0}

%The two sources with the smallest clump masses, G41.74+0.10 and G35.58+0.07,
%fall significantly below

% The higher derived masses could also indicate higher
% temperatures at steeper spectral indices (the masses are plotted assuming a
% uniform temperature $T_{dust}=40 K$), but that explanation does not fit with
% expectations that optically thinner \ion{H}{2} regions should heat their
% surroundings more.  

%XXXX The above paragraph is a very shallow discussion of a much more involved
%topic.  However, I would prefer not to expand it much for sake of paper brevity
%(and my own personal lack of knowledge - I would need to ready for ~2 weeks on
%this topic before writing a more useful discussion).  John in particular - do
%you have any recommendations on concise additions that could improve this
%discussion?  Otherwise, anyone, recommendations on how the discussion can be
%improved so it doesn't sound quite so superficial?  I'd rather not remove it
%because it is the most direct use of the BGPS 1.1mm data.

\section{Discussion}
\label{sec:discussion}
\subsection{Comparison to extragalactic observations}
\label{sec:exgal}

%The column densities \perkmspc\ measured in the starburst galaxy sample of
%\citet{Mangum2008} are more consistent with the non-UC\ion{H}{2} sample than
%the \uchii\ sample in our small survey.  The densities measured in some of the
%starburst galaxies are quite high and comparable to the upper end of our
%UC\ion{H}{2} sample (Figure \ref{fig:exgalcolden}).
%% Ben says this is not worth discussing...
%The density measurement of the gravitational lense B0218+357 in
%\citet{Zeiger2010} $2\times10^3 < n(\hh) < 1\times10^4$, is closer to the low
%density line-of-sight clouds than the UC\ion{H}{2} regions in our sample. 
% However,
% the column measurement $2.5\times10^{13} < N_{\formaldehyde} < 8.9\times10^{13}$ is
% towards the high end of both distributions. 
% This discrepancy implies a higher 
% \formaldehyde\ abundance in the molecular clouds seen in B0218+357 than in 
% Galactic molecular clouds, which is surprising.
We compare our measured column and volume densities to a selection of starburst
galaxies from \citet{Mangum2008} in Figure \ref{fig:exgalcolden}.  All of the
extragalactic observations have much lower column densities per \kms\ than we
measure in the main lines of most \uchii\ regions, but similar volume densities.
This discrepancy can be easily explained by a difference in the area filling factor
of molecular clouds in observations of galaxies and \uchii\ regions.  In a
galaxy, the total area filling factor of molecular clouds per \kms\ (which is
similar but not identical to the volume filling factor) is likely
to be $<1$, even in extreme starbursts; although the galaxy may appear to be
uniformly filled with molecular gas in projection, at any given velocity it is
likely to have significant gaps of ionized or neutral atomic gas.  In contrast,
an \uchii\ region should be completely embedded in a molecular cloud that is
much larger than the free-free emitting continuum region, so the covering
factor of molecular gas should be $\sim1$.
 
It is therefore interesting to note that Arp 220, possibly the most extreme nearby
example of a starburst galaxy, has nearly the same column per channel as the
low end of the \uchii\ regions, suggesting that it is analagous to a scaled-up
\uchii\ region to within a factor of a few; the measured density in Arp 220 is
consistent with only the highest-density \uchii\ regions.  M82, on the other
hand, has a bright continuum background analagous to an \uchii\ region, but a
correspondingly low filling factor, implying that it consists of many 
compact but bright sources with a total filling factor 0.001-0.1.
Alternatively, the density and column measurements are consistent with M82
being dominated by quiescent GMCs, but that is unlikely given the starburst
nature of the galaxy. % porosity? 

%Arp 220's density is close to that of two \uchii\ regions, G34.26+0.15 and
%G61.48+0.09.  However, Arp 220, unlike the \uchii\ regions, shows \formaldehyde\ \oneone\
%in emission, which does not happen when observing \formaldehyde\ against a bright 
%background.  It is notable that these two sources both also have strong carbon
%RRLs.  If Arp 220 really is a scaled-up \uchii\ region, it would be reasonable to
%expect 

The gravitational lens source B0218+357 is a different scenario.  Its low
density is consistent with that of a non-star-forming GMC, while its column per
\kms\ is comparable to the Galactic sample.  This source is therefore likely to
be a sightline through a `normal' quiescent molecular cloud in its host galaxy,
similar to the narrow beam of an \uchii\ region through the Galactic disk.
\citet{Zeiger2010} note that there is a range of covering factors cited in the
literature, which can affect the measured density and column, but should not
affect the conclusion that the B0128+357 cloud's density is not consistent with
that of massive-star forming regions.  The low-density gas is detected partly
because the \citet{Zeiger2010} data are 3.5$\times$ more sensitive than ours 
with a background continuum source of similar brightness.

\OneColFigure{figures_chH2CO/Derived_DensityVsColumn_ExgalCompare_all_colored}
{Comparison of the UC\ion{H}{2} sample (blue circles are measurements, blue
triangles are lower limits on volume density with poorly constrained column
densities), the GMC sample (red squares), secondary lines associated with
\uchii\ regions (black stars) and the extragalactic sample of
\citet{Mangum2008} (green squares).  The errorbars on the
Galactic data points are excluded for clarity.  The observed galaxies have
similar densities to the Galactic \uchii\ sample, but significantly lower
column densities, suggesting that the molecular gas in these galaxies has a
filling factor $<<1$.  The lack of direct density measurements of UCHII regions
at high densities is due to the presence of a dominant background source; in Arp 220 a
direct measurement of density was possible because \formaldehyde\ was seen in
emission.}
{fig:exgalcolden}
{0.30}{0}

\subsection{Line Profiles}
\label{sec:lineprofiles}
Despite the many systematics discussed above that can affect \formaldehyde\
absorption measurements with a compact illumination source, it is possible to
directly compare the properties of gas along a given line of sight without most
of these hindering factors.  Since most of our spectra have kinematically
resolved spectral line profiles, it is possible to make many density measurements at
different velocities towards each source.  An example of this type of analysis
is shown in Figure \ref{fig:g3280densspec}.  An example demonstrating the need for
this type of analysis is shown in Figure \ref{fig:G70compare}, in which two lines
well-fit by gaussian profiles nonetheless display a density gradient because the
line centers are significantly offset; the figure also demonstrates that the offset
cannot be accounted for by any instrumental effects.

Of our sample, 18 of the 24 observed lines-of-sight had high enough signal-to-noise spectra 
(S/N$\gtrsim5$ in at least four adjacent 0.4 \kms\ channels in both lines) to
measure the density in many velocity bins.  Of these, 12 have different peak
velocities in the \oneone\ and \twotwo\ lines, indicating density gradients in
the molecular gas with velocity.  Figure \ref{fig:g3280densspec}b is an example
density-velocity plot.%, and others are shown in Section \ref{sec:sources}.  

We have classified each high S/N spectrum as {\it gradient}, {\it envelope}, or
{\it single} based on spectral line morphology.  The {\it gradient} classification was used
for gaussian or nearly gaussian lines in which the \oneone\ and \twotwo\ line
centers were offset, indicating a gradient in the density with velocity; the
color listed in the table indicates the direction of \emph{increasing} density.  The {\it
envelope} classification was used for flat profiles on the wings of deeper
gaussian lines.  The {\it single} classification was used for lines where the
\oneone\ and \twotwo\ velocities matched.  Low S/N spectra were not classified.
Classifications are given in Table \ref{tab:other}.

Of the 12 sources with density gradients, 6 show an increased density towards
the red and 5 towards the blue.  One source, G45.12+0.13, shows a slight
increase towards the red over a broad (8 \kms) velocity range, but a sharp
increase towards the blue over only 1 \kms\ and is therefore classified as {\it
other}.

Figures \ref{fig:g33pt13spectrum}-\ref{fig:g61.48+0.09spectrum} show the `main
line' (associated with the \uchii\ region) profile and the associated density,
column, and abundance velocity profiles.  The density, column, and abundance
measured for each main line via the gaussian fit technique are shown
overplotted on the profiles with blue squares.  In all cases, the gaussian fit
measurement of density is consistent with the individual channels nearby and
the gaussian fit measurements of column and abundance are consistent with the
peak column and abundance.  The consistency of adjacent velocity bins confirms
the validity of associating gaussian components in observations of whole
galaxies \citep[e.g., ][]{Mangum2008} or kpc-scale regions, since on these scales
the \oneone\ and \twotwo\ lines should be blended by kinematics to have the
same shape.

\Figure{figures_chH2CO/G32.80+0.19_densityspectrum.png}
{ % can't use \\ for newline here: results in weird tex errors
Plot of the derived parameters per velocity bin for the main line of
G32.80+0.19; the full spectrum is shown in Figure \ref{fig:specexample}.  The
density peak around 16 \kms\ is slightly redshifted of the H and C RRL velocity
centers, although the C RRLs are blueshifted of the H
RRLs, indicating that the PDR has been accelerated towards us along the line of
sight.  The blueshifted emission tail is suggestive of an outflow.  This source
cannot therefore be easily classified under any of the scenarios in Section
\ref{sec:scenarios}, but is consistent with components of scenarios 2 and 3. 
{\it a.} The spectra of G32.80+0.19.  The GBT \twotwo\ spectrum (red solid) has
been smoothed to a resolution of 0.38 \kms\ to match the Arecibo (black dashed)
spectral resolution.   Labeled vertical bars indicate the measured velocity centers
of H and C RRLs from this work, \citet{Roshi2005}, and \citet{Churchwell2010}. 
{\it b.} The measured densities in each spectral bin.  The Y-scale is in log$_{10}$
units. Error bars include a 10\% systematic uncertainty in the continuum and
therefore errors in adjacent channels are not independent.  Limits are
indicated by triangles.  Bins with no information above the 1-$\sigma$ noise
cutoff are left blank.  The increase of density towards higher velocities led us
to classify this source as a {\it red gradient} in Table \ref{tab:other}. 
{\it c.} The measured column densities per spectral bin.  Because these column
densities are derived from a large velocity gradient code, they are in
\perkmspc\ units.
{\it d.} The measured abundances per spectral bin.  The column and abundance are 
somewhat degenerate, but it is possible in some cases to place tight constraints
on the total \ortho\ column while only placing upper limits on abundance
and density.  The abundance must also be interpreted \perkmspc. 
In plots {\it b} through {\it d}, the blue square with error bars
represents the measured value from Table \ref{tab:h2coderived} using gaussian
fits to the lines.
}
{fig:g3280densspec}
{0.25}{0}

\OneColFigure{figures_chH2CO/compare_G70.3_spectra.png}
{Comparison of G70.29+1.60 (top) and G70.33+1.59 (bottom) spectra as observed
by Arecibo (black) and GBT (red/grey).  Note that in G70.29+1.60, the \twotwo\
line is shifted towards the blue of the \oneone\ line, while in G70.33+1.59 the
line centers match well.}
{fig:G70compare}
{0.3}{0}


\subsection{Comparison of RRLs and \formaldehyde\ lines}
\label{sec:scenarios}
We compare the density spectra with the fitted RRL centroids and attempt to
interpret these spectra in the context of various simple models of \ion{H}{2}
region interaction with molecular clouds.  The simple models described below
may actually be short-lived but recurring stages in the normal life cycle of a
collapsing clump that is forming massive ($M\gtrsim10 \msun$) stars
\citep{Peters2010}.

% INTERPRETATION OF $X_{\formaldehyde}$:
% Assuming a constant \formaldehyde\ abundance, the \formaldehyde\ abundance per
% \kms\ per pc should be higher for slower and higher-density gas.  It should
% therefore be particularly low in outflows.

We consider five simple models of embedded \uchii\ regions.  For each scenario,
we include a brief description of the model and an analysis of the
observational consequences in terms of C and H RRL velocities and
velocity-density structure.  We assume that the carbon RRLs are only detected
if seen in the foreground of a bright source.  This assumption is based on
predictions that C RRLs will be amplified by an order of magnitude even in the
presence of a weak background \citep{Natta1994}.   It is backed by a strong
correlation between the continuum and the C RRL intensity  \citep{Roshi2005}.
We also assume that lower-frequency RRLs will have a stronger stimulated
emission component than higher-frequency RRLs \citep{Lockman1978}.  All
\formaldehyde\ absorption is assumed to be against the \uchii\ region in this
section.  The scenario that describes a given spectrum is listed in the figure
caption for each spectrum and in Table \ref{tab:other}.
%We then include a list of observed objects that may fit the described model.

\OneColFigure{figures_chH2CO/UCHII_stationaryuniform_scenario1_spectra}
{Scenario 1: An \uchii\ region forms and begins expanding spherically
in a uniform density gas cloud.  A cartoon of the geometry seen by the observer
is shown on the left side of the figure, with arrows indicating expansion and
darkness of the gray shading indicating relative density.  The white region
around the central star is the ionized \uchii\ region.  On the right side, a
cartoon of the relative velocity and width of the RRLs and \formaldehyde\ lines
is shown.  The relative heights of the \formaldehyde\ lines is representative
of the observed density;  black is \oneone\ and red is \twotwo.  The narrow
emission line with a ? above it indicates a possible blueshifted carbon RRL;
its height has no physical meaning.  In this scenario, the hydrogen
recombination and \formaldehyde\ lines should occur at the same velocity, and
the \formaldehyde\ lines should show relatively low-density (high
\oneone/\twotwo\ ratio) and modest spectral line widths.  A blueshifted carbon
RRL may form, but is not guaranteed.}
{fig:scenario1}
{0.15}{0}

\OneColFigure{figures_chH2CO/UCHII_insideoutcollapse_scenario2_spectra}
{Scenario 2: An \uchii\ region forms from a gravitationally unstable cloud
undergoing inside-out collapse.  See Figure \ref{fig:scenario1} for a complete
description of the figure.  The highest density should correspond to the
highest-velocity infall, so the \twotwo\ line peak should be redshifted of the
\oneone\ line peak.  The hydrogen recombination line may align with a
low-density cloud but should be blueshifted of the infalling gas.  The carbon
RRL should be redshifted from the hydrogen RRL and blueshifted from the
\formaldehyde\ line.}
{fig:scenario2}
{0.15}{0}

\OneColFigure{figures_chH2CO/UCHII_outflow_scenario3_spectra}
{Scenario 3: An \uchii\ region expanding in a uniform medium ejects a bipolar
outflow.  Presumably the bipolar outflow comes from a disk-accreting source.
See Figure \ref{fig:scenario1} for a complete
description of the figure.
The outflow (indicated by the cones emitting from the central source) should
have lower column density but could have high or low volume density.  It will be 
observed as high-velocity blueshifted absorption in a line wing.  Carbon
recombination line emitting regions may be destroyed by the outflowing
material.  As in the simple scenario 1, the hydrogen recombination line should
be at the same velocity as the molecular cloud.}
{fig:scenario3}
{0.15}{0}

\OneColFigure{figures_chH2CO/UCHII_triggered_scenario4_spectra}
{Scenario 4: An \uchii\ region expanding in a uniform medium sweeps up and
accelerates material that undergoes triggered star formation.  Because the
highest-density material is the swept up material, it should be the most
blueshifted.  See Figure \ref{fig:scenario1} for a complete description of
the figure.  The orange and yellow circles are meant to indicate triggered star
formation.}
{fig:scenario4}
{0.15}{0}

\OneColFigure{figures_chH2CO/UCHII_foreground_scenario5_spectra}
{Scenario 5: An \uchii\ region is seen behind a high-density, turbulent gas
cloud.  The turbulence drives large spectral line widths, while the high density makes
the \oneone\ and \twotwo\ line depths very close.   The RRL velocity could in
principle be at any velocity relative to the foreground turbulent cloud.  See
Figure \ref{fig:scenario1} for a complete description of the figure.  In this case,
the ?'s indicate an uncertain velocity for the hydrogen RRLs; a carbon RRL is not
expected because the \ion{H}{2} region is not necessarily interacting with the
molecular gas.}
{fig:scenario5}
{0.15}{0}


\begin{itemize}
    \item SCENARIO 1: STATIC
        \\* In a uniform medium with no bulk motions (i.e., no collapse), a massive star
      ignites and generates an expanding \ion{H}{2} region.  Figure \ref{fig:scenario1}.

%OBSERVATIONAL CONSEQUENCES:
  \begin{enumerate}
    \item Lower frequency RRLs are blueshifted from higher-frequency RRLs
      because of an increased stimulated emission component \citep{Lockman1978}
    \item A carbon RRL should be seen at the same velocity as or blueshifted from
      the hydrogen RRL line center. 
    \item Molecular gas closest to the \ion{H}{2} region should have the
      highest density because of compression by the expanding \ion{H}{2}
      region.  It will be at a similar velocity or blueshifted from the H RRLs.
%-Near the PDR, the \formaldehyde\ abundance (relative to \hh) should drop
%as \formaldehyde\ is preferentially dissociated
\end{enumerate}
%POSSIBLE EXAMPLES:
%G37.87?,G45.12,G45.47,G50.32


  \item SCENARIO 2: COLLAPSE
      \\* A massive star ignites while spherically accreting from a
    molecular cloud undergoing bulk (inside-out) collapse.  Figure \ref{fig:scenario2}.

%  OBSERVATIONAL CONSEQUENCES:
  \begin{enumerate}
    \item The \formaldehyde-measured density should peak at the velocities
      most redshifted relative to the hydrogen RRLs.  Inside-out collapse
      dictates that the highest densities should be infalling at the highest
      speeds.
    \item The C RRL velocity should be between the \formaldehyde\ and H RRL
      velocity since the PDR will be decelerated by radiation and gas pressure
      from the \ion{H}{2} region
    \item Since the accreting star should be at approximately the rest
      velocity of the cloud, there should be little to no gas blueshifted from
      the RRL velocity
\end{enumerate}
%POSSIBLE EXAMPLES:
%G33.13,G33.92,G34.26,G35.58?,G45.07,G45.45,G60.88

  \item SCENARIO 3: OUTFLOW
      \\* An accreting massive star generates a massive outflow with
    a significant component along the line of sight. Figure \ref{fig:scenario3}.

%OBSERVATIONAL CONSEQUENCES:
  \begin{enumerate}
    \item Substantial low-column, low-abundance \perkms\ gas should be observed
      at velocities blue of the RRL velocities.  Densities can range from low
      to high.  Covering factors may be low.
    \item No carbon RRL is expected from the outflow, though if the flow is
      accelerated by ionization pressure a C RRL should be observed blueshifted of
      the H RRL velocity.
\end{enumerate}
%POSSIBLE EXAMPLES:
%G43.89


\item SCENARIO 4: SWEEPING
    \\* An expanding \ion{H}{2} region pushes on a low-density
  envelope, possibly triggering a new stage of star formation as in the ``collect and
  collapse'' scenario.  This scenario is similar to \#1 but with either a
  higher-density envelope or with more gas swept up (i.e., \#4 may represent
  a more evolved region).  Figure \ref{fig:scenario4}.

%OBSERVATIONAL CONSEQUENCES:
  \begin{enumerate}
    \item The hydrogen RRLs should be red of the dense gas and the carbon
      RRLs.  The expanding \ion{H}{2} region should accelerate the dense gas
      blue along the line of sight.
    \item A low-density envelope should persist at the same velocity as the
      \ion{H}{2} region
\end{enumerate}
%POSSIBLE EXAMPLES:
%G35.20,G61.48

\item SCENARIO 5: FOREGROUND CLUMP
    \\* A high density, highly turbulent or high mass and rotating clump of gas
  is in front of the \uchii\ region or surrounds it.  This physical situation may exist
  in all of the above and provides alternate explanations for any spectral line wings.  Figure \ref{fig:scenario5}.
%OBSERVATIONAL CONSEQUENCES:
  \begin{enumerate}
    \item Moderate density gas from a molecular cloud will result in high
      column but moderate density at the center velocity
    \item Wide wings of high density gas will exist both blue and redshifted of
      the highest-column point
  \end{enumerate}
%  POSSIBLE EXAMPLES: (G45.45)

\end{itemize}


%A few sources remain unclassified after examining these models.  Some of these
%show the signatures of both outflow and infall (scenarios 2 and 3): G32.80,
%G35.57-0.03, and G48.61.  Some sources, G69.54, G70.29, and G70.33, had
%peculiar profiles that did not fit under any of the described categories.
%Others, G41.74, IRAS 20051, IRDC 1923, and IRDC 1916 had inadequate S/N to
%classify their velocity/density profiles.


%Because of the degeneracy of scenario 5 with others, we do not classify any
%sources as exhibiting signatures of that scenario.  However, we emphasize
%that velocity structure in dense clouds could reproduce most of the above scenarios, 
%though some only under XXXX conditions....


% This section could use some expansion - C RRLs are only expected to be seen on
% the front side because they are caused by stimulated emission
% \citet{Roshi2005}.  Why would we see redshifted high density material and no
% CRRL in some cases, but redshifted with CRRL in others?  I could imagine
% infall + radiation -> PDR for the case with the CRRL, but infall + continuum
% but NO PDR?  \emph{What other explanations for these profiles exist?}

\subsection{The Filling Factor of Molecular Clouds}
\label{sec:gmcdensity}
We have measured the density in 19 line-of-sight molecular clouds in addition
to the 18 measurements of densities around \uchii\ regions (we only include
measurements, not limits, in these counts).  The measured density from the
\formaldehyde\ line ratio can be compared to other measures of density, e.g.
the mean molecular cloud density measured by \citet{Roman-Duval2010} from the
BU-FCRAO GRS.  It is clear from Figure \ref{fig:denshist} that the average
density in GMCs is typically $\sim2-3$ orders of magnitude lower than densities
measured in our sample of line-of-sight GMCs.

\citet{Roman-Duval2010} point out that the mean densities they measure are
significantly below the critical density of \thirteenco,
$n_{cr}=2.7\ee{3}$ \percc, indicating that they do not resolve the high-density
clumps that make up the GMCs.  Our data indicate that a typical GMC consists of
$n\sim3\ee{4}$ \percc\ gas (the median of our GMC subsample excluding upper
limits), substantially higher than the critical density of \thirteenco.  Taking
the ratio of the median density in the \citet{Roman-Duval2010} catalog to that
in our sample, we derive a volume filling factor of $5\ee{-3}$ of dense gas in
molecular clouds.  

%Our sample may be highly biased since we have selected sightlines based on the
%presence of \uchii\ regions.  However, the unassociated line-of-sight
%foreground clouds should be effectively blindly selected, since their presence
%along the line of sight has nothing to do with the background \uchii\ region.
%They should therefore represent typical sightlines through GMCs, but this
%assertion will be tested more rigorously with the large-scale galactic plane
%survey.  

\OneColFigure{figures_chH2CO/DensityHistogram}
{Histograms of the GMC and \uchii\ subsamples from our data plotted along with
the GMC-averaged densities from the \thirteenco\ \citet{Roman-Duval2010} GRS
measurements arbitrarily scaled to fit on this plot.  The measured densities in
\uchii\ regions are significantly (by a KS test) higher than densities in GMCs.
The \formaldehyde-measured densities in GMCs are 2-4 orders of magnitude higher
than volume-averaged densities of GMCs from the GRS, suggesting that GMCs
consist of very low volume-filling factor ($\sim5\ee{-3}$) high-density
($n(\hh)\sim3\ee{4}$ \percc) clumps. In Section \ref{sec:gmcdensity}, we argue
that the observed difference is most likely not a selection effect imposed by
the different gas tracers.  The
GMC upper limits shown are $3-\sigma$ upper limits, and all are consistent with
the measured GMC densities.
}
{fig:denshist}
{0.30}{0}

We measured an additional 20 upper limits towards GMCs, all of which are
consistent with high densities ($n(\hh)>10^4$ \percc), but could represent a sample
of lower density ($n(\hh)\sim10^3$ \percc) gas, in which case our `measurement' of
the cloud volume filling factor is biased to be too low.  In order to test for
this bias, we need to acquire more sensitive observations of the upper-limit
systems.  However, we continue analysis below based on the assumption that the
cloud filling factor measurement is realistic, i.e. assuming that the density upper
limit measurements have densities consistent with the other observed GMCs.
%This can be done by observing sightlines in
%which there is no uncertainty about the background source (e.g., sightlines
%with absorption only against the CMB), by resolving the emission and absorption
%(e.g., with the EVLA), or by observing an additional \formaldehyde\ line.

%If we take the volume filling factor at face value, it does a reasonable job of
%explaining the CO ``X-factor'' measurement of cloud mass.  One hypothesis
%(CITE?) for the effectiveness of the X-factor in measuring mass is that it
%effectively counts unresolved, optically thick clumps of gas.  This hypothesis
%is consistent with a very low volume filling factor of dense gas, since
%high-density ($n\gtrsim10^4$ \percc) gas clumps become optically thick in \twelveco\ at column
%densities $N\sim4\ee{20}$ or size scales $l\lesssim0.01$ pc (at these scales and densities,
%the clumps are Jeans-stable even at the CMB temperature).  However, the
%similarity of the GRS \thirteenco\ and \formaldehyde\ \oneone\ spectral line
%profiles in Figures \ref{fig:g33pt13spectrum}-42 \ref{fig:g61.48+0.09spectrum}
%suggests that there must be many (hundreds-thousands) of these clumps within
%any $\sim50\arcsec$ beam.

Can a medium with supersonic turbulence produce the same density measurements
without having to invoke high-density clumping?  Below about
$n(\hh)\approx10^5$ \percc, measurements of density in a turbulent medium are biased
towards higher densities, i.e. the densities we report may be overestimates for
GMCs since they have a median density $n(\hh)=10^{4.49}$ \percc.  For turbulent
density PDFs with logarithmic widths $\sigma_{ln(\rho)/ln(\bar{\rho})} \lesssim
1.5$, the overestimate is no more than 0.4 dex, and therefore can only bring
the filling factor up by a factor $<3$.  As discussed in Section
\ref{sec:turbulence} and Figure \ref{fig:turbcorr}, a high-density tail could
create a larger discrepancy ($\sim 0.5$ dex).  However, at the measured
densities, these are extreme upper limits on the `turbulent correction', and
therefore (gravo)turbulence alone cannot account for the measured densities.

What clumping properties are required to reproduce the observed density?  As
long as the clumps are all optically thin in the \formaldehyde\ absorption
lines, the spectral line optical depths and ratio are independent of clumping.  However,
a large number of low-density ($n(\hh)\approx10^{3.5}$ \percc) clumps optically
thick in the \oneone\ line and thin in the \twotwo\ line would appear to have a
higher density.  This phenomenon could only occur at densities
$\lesssim10^{4.5}$ \percc, where the \oneone\ absorption line is much stronger
than the \twotwo\ line, and column densities $10^{14} \persc \gtrsim
N(\ortho) \gtrsim 10^{13.5} $ \persc\ per clump (at higher columns, both
lines are optically thick; at lower columns, both lines are optically thin).
Assuming a typical \formaldehyde\ abundance (ortho+para) $X_{\formaldehyde}=10^{-9}$, the
required spherical clump radius would be $\sim0.3 $ pc, which would be
Jeans-unstable at the assumed density and temperature (40 K) and is therefore
unlikely to persist for long time periods \footnote{However, the lifetime of
such clumps in a turbulent medium in which small-scall turbulence supports the
clumps against collapse is unconstrained.}. We therefore regard a collection
of optically thick clumps in the \formaldehyde\ \oneone\ line to be 
unlikely; clumps optically thick in both lines are even less likely following
the same line of reasoning.

% Given that the \formaldehyde-derived densities are accurate independent of
% clumping, we can examine typical properties of the clumps using their typical
% observed CO properties.  We first use the observation that GMCs are generally
% optically thick in the \twelveco\ 1-0 line.  At density $n(\hh)=10^{4.5}$ \percc,
% \twelveco\ becomes optically thick at a column density $N(\hh)\approx7\ee{16}
% $ \persc, which corresponds to a size-scale $L\sim0.008$ pc assuming a CO
% abundance $X_{CO}=1\ee{-4}$.  
% %Any clumps larger than about 0.01 pc would therefore be optically thick in \twelveco.  
% The Jeans length in a 50 K, $n=10^4 $ \percc\ molecular medium is $\sim0.3$ pc,
% % (lower temperatures or higher densities would reduce this number),
% which serves as a practical upper size
% limit to the clumps at the measured densities, since the free-fall time under
% these conditions is $\sim2\ee{5}$ years (i.e., short compared to the lifetime
% of GMCs).  Clumps would have to be $\gtrsim0.4$
% pc in diameter to be optically thick in \thirteenco\ (assuming
% $X_{\thirteenco}=X_{\twelveco}/60$), so optically thick \thirteenco\ clumps
% would not be expected to survive long enough to be observed along arbitrary
% sightlines.

% The size scales resolved in our survey span a large range from 0.015 pc to 1 pc
% (Figure \ref{fig:sizecomp} shows the size scale range for the \uchii\
% subsample).  There is a wide range of line-widths even for very small
% size scales; for the larger line widths, we are likely probing GMC-scale objects
% along the whole line of sight.  For the smaller line widths, particularly those
% below $\sigma_{FWHM}=1 \kms$, the probed size scales....
% There is a density-linewidth relation... hmmm
%
% the Goodman 1998 "Transition to Coherence" takes place in this regime...
% Goodman 1998 also has upper size scales of 0.2-0.3 pc for "coherent cores" - 
% could GMCs simply be collections of coherent cores?  
% Also, the "smallest eddy that persists" is 0.008 pc - exactly equal to the
% size scale of optically thick 12CO clumps

% An intriguing result of the densities observed in GMCs is that the largest
% clumps in molecular clouds are constrained to be smaller than 0.3 pc at these
% high densities, implying an upper mass limit of about 200 \msun.  In a
% triggered star formation scenario in which pre-existing clumps are crushed, it
% would therefore be unlikely to form compact clusters or individual stars with
% masses $>200\msun$ from GMCs.

% One compelling example consistent with this explanation is Figure \ref{fig:specexample},
% in which three distinct \formaldehyde\ line components are observed to be associated with
% a single, relatively smooth \thirteenco\ line.  The three \formaldehyde\ components all 
% have densities $n\approx10^{4.5}$ \percc.  The approximate size-scale probed by the \uchii\
% sightline at the distance of these clouds is 0.7 pc, while the beamsize of the \thirteenco\
% observation is 2 pc.

The combination of the observed large spatial scales (and therefore low
volume-averaged density) of GMCs and the high densities measured along
essentially arbitrary sightlines through these GMCs suggests that GMCs are not
consistent with a purely turbulent medium with a lognormal density
distribution.  The observations also require a more substantial high-density
tail than typically seen in gravoturbulent simulations, i.e. they require a
clumpier medium.

Alternatively, it is possible that \formaldehyde\ is chemically enriched in
high-density pockets within a turbulent medium, which would imply that
\formaldehyde\ observations probe different gas than CO.  No such mechanism has
been proposed on theoretical grounds, and the timescales for enhancement would
have to be very short \citep[intermittent density enhancement occurs on
timescales much shorter than the dynamical timescale; ][]{Kritsuk2007}, so we
regard this possibility as unlikely but include it for completeness.

Another alternative is that the \thirteenco\ systematically underestimates the
mass or overestimates the volume of the cloud, resulting in an underestimate of
the cloud density.  Sub-thermal excitation of \thirteenco\ in the low-density
parts of the cloud can lead to an underestimate of the mass \citep[][Section
9.3]{Roman-Duval2010}.  Since the cloud sizes were derived using an assumed
spherical symmetry, but molecular gas is typically observed in filamentary
structures, the densities in \citet{Roman-Duval2010} are likely to be lower
limits on the mean density in the molecular gas.  While both of these factors
bring the \formaldehyde\ and \thirteenco\ densities into closer agreement, it
is difficult to quantify these effects.

%There have been many studies of the structure of molecular clouds in the
%contexts of clumpy clouds, turbulent clouds (Kritsuk, Goodman, others), fractal
%clouds (Elmegreen)... our measurement weakly suggests that GMCs do not follow
%the structure distribution predicted by purely turbulent (i.e., lognormally
%distributed) density.   XXXX This paragraph is essential if I am to include this
%section, but I need to fill it out more...

% \subsection{Feeding massive star formation}
% (remove this section unless it is completed)
% 
% Do massive stars form from `coherent cores' with subsonic turbulence like
% low-mass stars, or do they accrete their mass directly from a supersonically
% turbulent, clumpy medium like a GMC?  \citet{Goodman1998} and
% \citet{Pineda2010} have shown that in low-mass star-forming regions, a coherent
% core in which the nonthermal velocity dispersion is less than the thermal
% dispersion is the most likely predecessor to a protostar.  However, it is not
% known whether massive stars ever go through a coherent core phase.  In
% particular, for the most massive stars known with individual or multiple
% stellar systems massing $\sim100\msun$, it would be impossible to form a
% gravitationally stable coherent core at densities similar to those observed in
% B5 \citep[$n\sim10^4-10^5$]{Pineda2010}.
% 
% The line widths we observe around the \uchii\ regions are all clearly
% supersonic, as is expected since the expanding \uchii\ region should directly
% drive turbulence.  However, there is some remote chance that the \formaldehyde\ 
% densitometer could be used to search for, or rule out, such a transition.
% Because of the rarity of high-mass star-forming regions, it is likely that there
% are no analogs to the coherent core in our local neighborhood (e.g., within 1 kpc), 
% in which only the Orion and Cepheus GMCs have formed massive stars (??).  At
% greater distances, it is extremely difficult to find isolated regions in which to
% perform such a test.  
% 
% We therefore suggest that `coherence' should be sought around hypercompact
% \ion{H}{2} regions, in which a massive star has formed a very small \ion{H}{2}
% region, but has not fed back enough on its birth environment to disrupt any
% coherent structures...
% 
% In our spectra, there are hints of `sharp' density transitions with velocity.
% If similar could be observed in space....
% 
% What would an uchii in a turbulent field look like?  In a coherent field?
% 
% If massive stars form from coherent clumps, an upper mass limit is implied,
% but if not, there is no obvious mass cutoff...

% \subsection{Other Results}
% Three sources were observed with potentially optically thick \formaldehyde\
% spectra: G37.87-0.40, G45.12+0.13, and G69.54-0.98.  These are interesting
% targets for follow-up observations and are discussed in more detail in 
% Section \ref{sec:sources}.

\subsection{Strengths and Weaknesses of the \formaldehyde\ K-doublet Densitometer}
\label{sec:strengthsweaknesses}
The dynamic range of a spectral line as a tracer of a physical quantity is an
important consideration when designing an experiment.  We have demonstrated
only a modest dynamic range in density measurements using the \oneone\ and
\twotwo\ lines in absorption against bright background sources: above
$n(\hh)\approx10^{5.6}$ \percc, we are only able to set lower limits on the
density because the spectral line ratio asymptotes to 1 , and below $n(\hh)\approx10^4$
\percc, the \twotwo\ line optical depth drops to very low levels (Figure
\ref{fig:modeltau}). 

The lower limits on density at $n(\hh)\gtrsim10^{5.6}$ \percc\ can only be modestly
improved upon by using higher K-doublet transitions (e.g.  \threethree) when
observing \formaldehyde\ in absorption against bright continuum sources.
However, when observing anomolous absorption against the CMB, an additional
density diagnostic is the transition from absorption to emission at higher
densities, which expands the sensitivity of the \oneone/\twotwo\ pair to
$n(\hh)\approx10^{6.5}$ \percc.

The low-density end can only be probed by more sensitive observations of
the \twotwo\ line.  Because the \formaldehyde\ line depths become negligible
below $n(\hh)\approx10^3$ \percc, the densitometer is not a useful probe below these
densities.  However, at such low densities, even within a molecular cloud,
it is questionable whether any molecular probes are reliable, as even CO will be
underabundant in these environments \citep[e.g., ][]{Glover2010}.

As noted in \citet{Mangum2008} and \citet{Zeiger2010}, the K-doublet
densitometer has the advantage that line detection only depends on the
brightness of the background source and the gas density.  It can therefore be
used nearly independent of distance when observing clouds against the CMB or
bright illuminating background sources.  The \citet{Zeiger2010} measurements
are more sensitive than any presented in our study because of longer
integration times and the selection of a bright illuminating background source
despite their target being at a distance z=0.68.  Following this line of
reasoning, any bright synchrotron or free-free source can be used to
sensitively probe the density of a line-of-sight molecular cloud in the Galaxy.
The observation will have angular resolution limited only by the size of the
background source as long as it is much brighter than the CMB in the beam.

%Weaknesses:
%Low formaldehyde abundance at low A_V - harder to probe low densities
%Limited dynamic range
%Sensitive to continuum source size, brightness determination


\section{Conclusions}
\label{sec:h2coconclusions}
We have presented a pilot study to measure molecular gas densities in clouds
along 24 lines of sight in the \formaldehyde\ \oneone\ and \twotwo\ transitions
primarily toward \uchii\ regions .  We have shown that the \formaldehyde\
densitometer is robust within reasonable ranges of turbulent density
distributions, most cloud geometries, and different cloud clumping properties.
We have presented the methodology and discussed the errors intrinsic to the
\formaldehyde\ densitometer.

Gas volume densities were measured toward 14 of the 24 sources using the
best-fit gaussian profiles; density limits were measured for the remaning 10.
In 18 sources, it was possible to estimate the density in each 0.4 \kms-wide
channel centered on the main line.  Of these, 12 showed some sign of a density
gradient with velocity, 5 appeared to have a single-valued density (i.e. only a
single spectral line component well-fit by a gaussian), and one, G69.54-0.98,
had a spectral line optical depth that was beyond our ability to model.  
% this is not shown The measured spatial densities are uniformly higher than the
% this is not shown beam-averaged densities measured with the BGPS 1.1 mm data.

Velocity-density gradients have been used to fit 18 sources with simple
models of \uchii\ regions embedded in molecular clouds.  We have found some  
examples consistent with inside-out collapse onto the \uchii\ region,
 \uchii\ region expansion, and bulk outflow.  \formaldehyde\
absorption provides a unique probe of the physical conditions around \uchii\
regions because it is only seen in absorption against the continuum background
(for sources much brighter than the CMB), giving different constraints than mm
and sub-mm spectral lines that are seen both in front of and behind the \uchii\ region.

The measurements of serendipitously detected line-of-sight GMCs revealed
densities $\sim200$ times higher than volume-average densities measured using
\thirteenco.  The high density measured suggests that GMCs consist of many
sparsely distributed high-density clumps and have density distributions
inconsistent with the lognormal distribution predicted by supersonic turbulent
models.  The implied density distribution is also more skewed to high densities
than predicted by typical gravoturbulent simulations.  Alternatively, the 
\thirteenco-based mean densities may be lower than the mean densities within
the molecular gas either because they underestimate the mass or overestimate
the volume of GMCs.

The density measurements show that UC\ion{H}{2}s are associated with
high-density ($n(\hh)>10^{4.5}$ \percc) gas, and UC\ion{H}{2}s are associated with
higher column and volume densities than other line-of-sight molecular clouds,
in contradiction with previous results \citep{Wadiak1988}.  

The 6 cm, 2 cm, and 1.1 mm flux density measurements are strongly correlated
and in most objects in our sample the 1.1 mm flux density has a substantial ($>30\%$) contribution 
free-free continuum emission.  This result implies that the
brightest sources detected in the BGPS have significant free-free emission.
A comparison to the 6 cm MAGPIS survey suggests that the sample of 1.1 mm
sources below about 3 Jy is not significantly contaminated by pure free-free
emission sources.  
%While other surveys (e.g., ATLASGAL, HiGAL, JPS) at shorter wavelengths should be less
%affected by free-free emission because of rising dust emissivity with frequency, the
%free-free contribution in the brightest sources in these surveys may still be
%substantial.

% Observed density gradients include both increasing density toward the red and
% the blue.  These two scenarios require different physical explanations, since
% in both cases the \formaldehyde\ absorber must be in front of the continuum
% source.

% A correlation is observed between 1.1 mm -derived dust mass and 2 cm / 6 cm
% Spectral index. This correlation is consistent with the hypothesis that massive
% Stars clear out their dust envelopes as they evolve from ultracompact to
% Diffuse \ion{H}{2} regions.

Comparison of the density measurements in our sample to the starburst sample of
\citet{Mangum2008} suggest that the molecular gas volume filling factor in most of
these galaxies is small ($\sim 0.01$), but in Arp 220 it is quite high
($\gtrsim 0.1$).  The physical properties measured by \formaldehyde\ in Arp 220
are similar to those in \uchii\ regions.  Although velocity-density gradients
are observed in our sample, we argue that kinematic spectral line blending
should uphold the assumption of a single spectral line profile in galaxies as
robust for radiative transfer purposes.

\section{Acknowledgements}
We thank Jim Braatz for assistance with data acquisition and processing,
Esteban Araya for providing us with reduced data, and our referee Jeff Mangum
for a helpful and timely review.  This work was supported by the National
Science Foundation through NSF grant AST-0708403 to John Bally and AST-0707713
to Jeremy Darling.  This research has made use of the SIMBAD database, operated
at CDS, Strasbourg, France.  This research made use of pyspeckit, an
open-source spectroscopic toolkit hosted at \url{http://pyspeckit.bitbucket.org}.


% {\it Facilities:} \facility{GBT}, \facility{Arecibo}, \facility{VLA},
% \facility{FCRAO}, \facility{CSO}

% BU FCRAO GRS:
% This publication makes use of molecular line data from the Boston
% University-FCRAO Galactic Ring Survey (GRS). The GRS is a joint project of
% Boston University and Five College Radio Astronomy Observatory, funded by the
% National Science Foundation under grants AST-9800334, AST-0098562, &
% AST-0100793.
%
% MAGPIS:
% RHB and DJH acknowledge the support of the National Science Foundation under
% grants AST-05-07663 and AST-05-07598, respectively. RHB's work was supported
% in part under the auspices of the US Department of Energy by Lawrence
% Livermore National Laboratory under contract W-7405-ENG-48. DJH was also
% supported in this work by NASA grant NAG5-13062. RLW acknowledges the support
% of the Space Telescope Science Institute, which is operated by the
% Association of Universities for Research in Astronomy under NASA contract
% NAS5-26555.

%\section{Appendix: Individual Source Discussion}

%\bibliography{h2co_pilot}
% \subimport{/Users/adam/work/h2co/pilot/paper/}{source_discussion}
\ifstandalone
\bibliographystyle{apj_w_etal}  % or "siam", or "alpha", or "abbrv"
%\bibliography{thesis}      % bib database file refs.bib
\bibliography{bibdesk}      % bib database file refs.bib
\fi

\end{document}

% %\documentclass[defaultstyle,11pt]{thesis}
%\documentclass[]{report}
%\documentclass[]{article}
%\usepackage{aastex_hack}
%\usepackage{deluxetable}
\documentclass[preprint]{aastex}


%%%%%%%%%%%%%%%%%%%%%%%%%%%%%%%%%%%%%%%%%%%%%%%%%%%%%%%%%%%%%%%%
%%%%%%%%%%%  see documentation for information about  %%%%%%%%%%
%%%%%%%%%%%  the options (11pt, defaultstyle, etc.)   %%%%%%%%%%
%%%%%%%  http://www.colorado.edu/its/docs/latex/thesis/  %%%%%%%
%%%%%%%%%%%%%%%%%%%%%%%%%%%%%%%%%%%%%%%%%%%%%%%%%%%%%%%%%%%%%%%%
%		\documentclass[typewriterstyle]{thesis}
% 		\documentclass[modernstyle]{thesis}
% 		\documentclass[modernstyle,11pt]{thesis}
%	 	\documentclass[modernstyle,12pt]{thesis}

%%%%%%%%%%%%%%%%%%%%%%%%%%%%%%%%%%%%%%%%%%%%%%%%%%%%%%%%%%%%%%%%
%%%%%%%%%%%    load any packages which are needed    %%%%%%%%%%%
%%%%%%%%%%%%%%%%%%%%%%%%%%%%%%%%%%%%%%%%%%%%%%%%%%%%%%%%%%%%%%%%
\usepackage{latexsym}		% to get LASY symbols
\usepackage{graphicx}		% to insert PostScript figures
%\usepackage{deluxetable}
\usepackage{rotating}		% for sideways tables/figures
\usepackage{natbib}  % Requires natbib.sty, available from http://ads.harvard.edu/pubs/bibtex/astronat/
\usepackage{savesym}
\usepackage{amssymb}
%\savesymbol{singlespace}
\savesymbol{doublespace}
%\usepackage{wrapfig}
%\usepackage{setspace}
\usepackage{xspace}
\usepackage{color}
\usepackage{multicol}
\usepackage{mdframed}
\usepackage{url}
\usepackage{subfigure}
%\usepackage{emulateapj}
\usepackage{lscape}
\usepackage{grffile}
\usepackage{standalone}
\standalonetrue
\usepackage{import}
\usepackage[utf8]{inputenc}
\usepackage{longtable}
\usepackage{booktabs}



%%%%%%%%%%%%%%%%%%%%%%%%%%%%%%%%%%%%%%%%%%%%%%%%%%%%%%%%%%%%%%%%
%%%%%%%%%%%%       all the preamble material:       %%%%%%%%%%%%
%%%%%%%%%%%%%%%%%%%%%%%%%%%%%%%%%%%%%%%%%%%%%%%%%%%%%%%%%%%%%%%%

% \title{Star Formation in the Galaxy}
% 
% \author{Adam G.}{Ginsburg}
% 
% \otherdegrees{B.S., Rice University, 2007\\
% 	      M.S., University of Colorado, Boulder, 2009}
% 
% \degree{Doctor of Philosophy}		%  #1 {long descr.}
% 	{Ph.D., Rocket Science (ok, fine, astrophysics)}		%  #2 {short descr.}
% 
% \dept{Department of}			%  #1 {designation}
% 	{Astrophysical and Planetary Sciences}		%  #2 {name}
% 
% \advisor{Prof.}				%  #1 {title}
% 	{John Bally}			%  #2 {name}
% 
% \reader{Prof.~Jeremy Darling}		%  2nd person to sign thesis
% \readerThree{Prof.~Jason Glenn}		%  3rd person to sign thesis
% \readerFour{Prof.~Michael Shull}	%  4rd person to sign thesis
% \readerFour{Prof.~Neal Evans}	%  4rd person to sign thesis
% 
% \abstract{  \OnePageChapter	% one page only ??
% 
%     I discovered dust in space.  
% 
% 	}
% 
% 
% \dedication[Dedication]{	% NEVER use \OnePageChapter here.
% 	To 1, the second number in binary.
% 	}
% 
% \acknowledgements{	\OnePageChapter	% *MUST* BE ONLY ONE PAGE!
% 	All y'all.
% 	}
% 
% \ToCisShort	% a 1-page Table of Contents ??
% 
% \LoFisShort	% a 1-page List of Figures ??
% %	\emptyLoF	% no List of Figures at all ??
% 
% \LoTisShort	% a 1-page List of Tables ??
% %	\emptyLoT	% no List of Tables at all ??
% 
% 
% %%%%%%%%%%%%%%%%%%%%%%%%%%%%%%%%%%%%%%%%%%%%%%%%%%%%%%%%%%%%%%%%%
% %%%%%%%%%%%%%%%       BEGIN DOCUMENT...         %%%%%%%%%%%%%%%%%
% %%%%%%%%%%%%%%%%%%%%%%%%%%%%%%%%%%%%%%%%%%%%%%%%%%%%%%%%%%%%%%%%%
% 
% %%%%  footnote style; default=\arabic  (numbered 1,2,3...)
% %%%%  others:  \roman, \Roman, \alph, \Alph, \fnsymbol
% %	"\fnsymbol" uses asterisk, dagger, double-dagger, etc.
% %	\renewcommand{\thefootnote}{\fnsymbol{footnote}}
% %	\setcounter{footnote}{0}

\input{macros}		% file containing author's macro definitions

\begin{document}
% \input{introduction}
% 
% %\input{ch_iras05358}
% \input{ch_w5}
% \input{ch_h2co}
% \input{ch_h2colarge}
% \input{ch_boundhii}
% 
% %\input ch2.tex			% file with Chapter 2 contents
% 
% %%%%%%%%%%%%%%%%%%%%%%%%%%%%%%%%%%%%%%%%%%%%%%%%%%%%%%%%%%%%%%%%%%%
% %%%%%%%%%%%%%%%%%%%%%%%  Bibliography %%%%%%%%%%%%%%%%%%%%%%%%%%%%%
% %%%%%%%%%%%%%%%%%%%%%%%%%%%%%%%%%%%%%%%%%%%%%%%%%%%%%%%%%%%%%%%%%%%
% 
% \bibliographystyle{plain}	% or "siam", or "alpha", or "abbrv"
% 				% see other styles (.bst files) in
% 				% $TEXHOME/texmf/bibtex/bst
% 
% \nocite{*}		% list all refs in database, cited or not.
% 
% \bibliography{thesis}		% bib database file refs.bib
% 
% %%%%%%%%%%%%%%%%%%%%%%%%%%%%%%%%%%%%%%%%%%%%%%%%%%%%%%%%%%%%%%%%%%%
% %%%%%%%%%%%%%%%%%%%%%%%%  Appendices %%%%%%%%%%%%%%%%%%%%%%%%%%%%%%
% %%%%%%%%%%%%%%%%%%%%%%%%%%%%%%%%%%%%%%%%%%%%%%%%%%%%%%%%%%%%%%%%%%%
% 
% \appendix	% don't forget this line if you have appendices!
% 
% %\input appA.tex			% file with Appendix A contents
% %\input appB.tex			% file with Appendix B contents
% 
% %%%%%%%%%%%%%%%%%%%%%%%%%%%%%%%%%%%%%%%%%%%%%%%%%%%%%%%%%%%%%%%%%%%
% %%%%%%%%%%%%%%%%%%%%%%%%   THE END   %%%%%%%%%%%%%%%%%%%%%%%%%%%%%%
% %%%%%%%%%%%%%%%%%%%%%%%%%%%%%%%%%%%%%%%%%%%%%%%%%%%%%%%%%%%%%%%%%%%
% 
% \end{document}
% 
% 

\chapter{\formaldehyde observations of BGPS sources not previously observed with Arecibo}
\label{ch:h2colarge}


\section{Preface} 

Given our success with the simple 4-hour GBT observation of $\sim20$ sources,
it was decided that a large-scale survey of BGPS sources accessible to Arecibo
and the GBT would be productive.  We therefore selected 400 pointings in the
Arecibo-accessible declination range, 137 of which are in the outer galaxy
($172<\ell<207$) and the others in the inner galaxy ($31<\ell<65$).

The selected sources included \emph{all} peaks in the outer galaxy regions,
including the newly observed regions from the BGPS \vtwo survey.

As of the thesis defense, all of the data for the large survey has been reduced,
but the analysis and interpretation is ongoing.  We report here only initial
results from the outer galaxy component of the large survey.

Further follow-up projects based on these observations, in particular VLA
observations of the W51 star forming complex, have been approved and are
queued for observation.

The following section describes portions of the \formaldehyde observations focusing
on the outer galaxy in detail.

\subimport{/Users/adam/work/h2co/outergal/paper/}{h2co_outergal.tex}


\section{\formaldehyde Mapping}
We were lucky enough to be awarded \emph{double} the time we asked for on the GBT,
allowing us to observe large areas in mapping mode.  
Naturally, we picked the brightest and best-known regions for mapping studies.

In the inner galaxy, we mapped out an area $\sim50\arcmin\times20\arcmin$ centered
on the W51 massive star forming complex.  This region is ideally suited to study from Arecibo,
the GBT, and the VLA, since it is at declination +14 and is one of the brightest continuum
sources in the Galactic plane.  It also turns out to be the nearest proto-massive cluster
at a VLBI-parallax-measured distance of 5.1 kpc (see Chapter \ref{ch:ympc} for a discussion
of massive proto-clusters).  The simple reduction of this data is nearly complete, but 
analysis has not yet begun.

In the outer galaxy, we targeted two regions: the Sh 255 complex in Gem OB1 and
the Sh 233-IR/IRAS 05358 complex I studied for my Comps II project.  We made
small ($\sim 5\arcmin\times 5\arcmin$) maps of these objects in order to
evaluate the density profiles and determine what systematic biases may have
been present in our single-pointing observations.  These outer galaxy sources
are both at $D<2$ kpc, so our resolution is $\lesssim 0.5 pc$ and we therefore
have some marginal hope of discovering dense cores without diluting their
signal too badly.

\subsection{\formaldehyde maps of S233IR}
For the S233IR region, we were able to create a density map, but found the
surprising and counterintuitive result that the density was smallest at the
peak of the BGPS 1.1 mm emission.  The ``envelope'' is at a nearly constant density
$n\sim10^{3.5}\percc$, but the core is either at a low density (which is effectively
ruled out on other considerations) or is significantly self-absorbed.  It turns out 
that \formaldehyde \twotwo \emph{emission} fills in the absorption.

Figure \ref{fig:s233irmulti} shows the mapping results for the S233 IR region, where
`envelope' densities are measured to be $n\sim10^{3.3}-10^{3.7} \percc$, but the `core'
density is more weakly constrained to be $10^{4.5}\percc < n < 10^{5.5}\percc$ based
on the presence of \formaldehyde \twotwo emission and the absence of \oneone emission.
The lack of a direct measurement makes density profile measurement with the current
observations impossible. 

The moderate densities observed in the envelopes are nonetheless an order of
magnitude higher than typical volume-averaged GMC densities
\citep{Roman-Duval2010} as was previously noted for ordinary GMCs with
\formaldehyde detections in \citet{Ginsburg2011}. 

Perhaps most interesting is the contrast between the two BGPS sources shown 
in Figure \ref{fig:s233irmulti}.  In \citet{Ginsburg2009}, I examined primarily
S233IR, but its neighboring region G173.58+2.45 is also a well-studied proto-cluster.
Unlike S233IR, which has a B1/B2 10 \msun star that is probably still accreting,
G173.58 contains no massive stars and has an upper mass limit $M\lesssim4 \msun$
\citep{Shepherd2002}.  The \formaldehyde observations reveal that the density in this
clump is $\sim10^{3.6} \percc$, substantially less than the expected $n\sim10^5 \percc$
in the massive-star forming S233IR.

The BGPS data show this difference as well, but less strikingly.  In the BGPS
data, the inferred masses of S233IR and G173 are 840 and 190 \msun,
respectively, though lower by a factor of $\sim2$ in each when considering only
their condensed $r\lesssim0.4$ pc cores.  Their densities differ by a smaller
amount using the BGPS data and assuming spherical symmetry, with peak densities
$n\sim10^{4.1}$ in G173 and $n\sim10^{4.8}$ in S233IR.  The density difference
reinforces the conclusion drawn from the \formaldehyde data, but also show that its
density measuring power is greater, since the spherical symmetry assumption is known
to be flawed.

% S233IR:
% peak column: 6.09*2.08e22*(np.exp(13.01/20)-1)  = 1.16e23
% h2co column: "%e" % (6.09*2.08e22*(np.exp(13.01/20)-1) * 1e-9)  = 1.16e14
% total flux = 220 / 23.8 = 9.24
% mass is then 9.2 * 1.8**2 * 14.3 * (np.exp(13.01/20)-1)  = 400 msun (600 by other measures; close enough)
% radius ~30 arsec ~ 0.3pc
% n ~ 6.7e4, 10^4.8
% TOTAL flux/mass, in full aperture, is 471/23.8=19.7 Jy, or 840 msun
%
% G173:
% flux ~62 Jy/bm, /ppbeam = 2.6 Jy ~ 2.6 * 1.8**2 * 14 = 120 msun
% TOTAL 107 Jy/beam /ppbeam=4.5 -> 190 msun
% radius = 40 arcsec = 0.35 pc
% density = 120 * 2e33 / (4/3*np.pi*(0.35*3.08e18)**3) / ( 2.8 * 1.67e-24 ) = 1.3e4

\Figure{figures_chH2CO/S233IR_multipanel}
{The S233IR / IRAS 05358+3543 region and its neighbor G173.58+2.45.
{\it Top left:} The \formaldehyde density map covering densities
$10^2 \percc<n<10^5 \percc$ from grey to green.  The grey areas show
regions of low density ($n<10^3$ \percc), while green show high-density
regions ($n\gtrsim10^3.5$ \percc).  The `hole' at the peak of the contours
is likely very high density, $n>10^5$ \percc.
{\it Top center: } The \formaldehyde \oneone absorption map.
{\it Top right: } The \formaldehyde \twotwo absorption map.
Note the lack of absorption at the contour peak: this is probably \twotwo emission
filling in \twotwo absorption, indicating a high $n\gtrsim10^5$ \percc density.
{\it Bottom left: } CO 3-2 map.
{\it Bottom center: } The BGPS v2.0 1.1 mm emission map, with contours at 0.2, 1.0, and 3.0 Jy.
These contours are shown on all of the other maps for reference.
{\it Bottom right: } SO $5_6-4_5$ map.  This line has a very high critical density $n\sim3.5\ee{6}$ \percc
and an upper level energy $T_U=35$ K.
Its morphology, with a hole at the peak of the dust emission, backs the claim that the density is highest
in the area around the dust peak.}
{fig:s233irmulti}{0.2}{0}

\subsection{W51}
The W51 survey was completed in September 2011.  The data reduction process
presented unique challenges: at C-band, the entire region surveyed contains
continuum emission, so no truly suitable `off' position was found within the
survey data.  Similarly, \formaldehyde is ubiquitous across the region, so it
was necessary to `mask out' the absorption lines when building an off position.
This was done by interpolating across the line-containing region with a
polynomial fit.  

\Figure{figures_chH2CO/a2705.20120915.b0s1g0.00000_offspectra.png}
{An example of the \formaldehyde line masking procedure for building an Off
spectrum.  The line-containing regions for each polarization are shown in cyan
and purple, with the interpolated replacement in red and green.
}{fig:h2comask}{0.4}{0}

The W51 data are converted into ``optical depth'' data cubes by dividing the
integrated \formaldehyde absorption signature by the measured continuum level.
These $\tau$ cubes are then fit with the RADEX models used for other
\formaldehyde fitting.  However, there are multiple velocity components in W51,
so I used a two-component (unconstrained) fit for each pixel, which is
frequently unstable but in the case of W51 looks to have produced reasonable
results.  Note that there was \emph{no} \formaldehyde emission detected anywhere
in the W51 region.

A first interesting note is that a local cloud at $v_{lsr}\sim5 \kms$ is
detected in \formaldehyde \oneone across most of the cloud and not detected at
\twotwo, with $\tau_{\oneone}/\tau_{\twotwo} \gtrsim 3$, implying a
very low column
$N_{\formaldehyde}\sim10^{11.5}$ or $N_{\hh} \sim 10^{20.5}$.  
The cloud is seen in \thirteenco as a very weak, diffuse feature, and in HI absorption
as a very sharp, deep (self)-absorption feature.
% This density measurement
% is consistent with observations from \citet{Ginsburg2011} of high density in
% GMCs.  However, GMCs are generally thought of as being low-density clouds, so
% this result may be surprising.

%FIGURE: mcmc column vs density 

I successfully made density maps of the W51 cloud, though because the velocity
structure is quite complicated, I need to fit two components to most of the
map.  Two-component fits are never particularly stable, so it was necessary to
restrict the parameters being fitted, and even then the results aren't
perfectly reliable.  Despite those caveats, there are some reliable fits,
particularly towards the `core' of W51 Main / W51 IRS 2.  There are two
high-density components with $n\sim10^5-10^5.5$ at different velocities evident
in Figure \ref{fig:w51h2cofits}.  The southern component, centered on W51 Main,
has $v_{LSR}\sim56-59$.  The northern component, a strip going through IRS 2
and towards the west, peaks around $v_{LSR}\sim68-69$.  A 10 \kms difference
between two extremely dense components, both which are necessarily in the
foreground of the HII region, is shocking (probably, anyway, unless the sound
speed is very high).


\FigureTwo{figures_chH2CO/W51_H2CO_2parfit_v1_densityvelocity.png}
{figures_chH2CO/W51_H2CO_2parfit_v2_densityvelocity.png}
{Density and Velocity fits to the W51 Arecibo and GBT \formaldehyde 
data cubes.  The yellow regions in the top panel correspond to \oneone
detections and \twotwo nondetections, indicating upper limits $n<10^{3.8}$
(68\% confidence) or $n<10^{4.3}$ (99.7\% confidence).}
{fig:w51h2cofits}{1}

There is a large area where \oneone was detected, but \twotwo was not.  Our
sensitivity allows us to place a modest upper limit on the gas density, with
$3-\sigma$ upper limits $\lesssim10^{4.3}$ \percc (but the most likely
densities are $10^2 < n < 10^4$ \percc).  Figure \ref{fig:w51MCMCcompare} shows
a particular model for a spectrum that is especially unconstrained.  The
\oneone/\twotwo optical depth in this object is $\sim10-20$, indicating that
the volume density must be low.

\FigureTwo{figures_chH2CO/MCMC_DensColplot_67_64.png}{figures_chH2CO/spec67_64_bestfit_MCMC.png}
{Plots demonstrating upper limit fits.  The left plot shows the allowed
parameter space from MCMC sampling of the data given the RADEX model.  The
right plot shows the `best-fit' model to the optical depth spectra, which is
clearly unconstrained by the relatively insensitive \twotwo\ spectrum.  The
sensitivity in the \oneone line is better in large part because of brighter 6
cm background across the whole W51 region.  Despite the lack of constraint on the
volume density, there is a reasonably strong constraint on the column density.}
{fig:w51MCMCcompare}{1}

The molecular gas is concentrated near, but not exactly on, the bright cm
peaks.  W51 IRS2 has a massive clump of gas at 65 \kms, and W51 e2 has a
similar clump.  However, e2 also seems to have a very dense ($n>10^5 \percc$)
infalling clump.  The spectra, along with multicomponent fits, are shown in
Figure \ref{fig:w51hiispectra}.

\FigureTwo{figures_chH2CO/W51_bestfit_spec53_49_IRS2.png}{figures_chH2CO/W51_bestfit_spec53_49_W51e2.png}
{Plots of the optical depth spectra centered on W51 IRS2 (left) and W51 e2, an
ultracompact HII region (right).  IRS2 shows high-density gas with a slight
hint of infall, but otherwise a somewhat vanilla spectrum.  W51e2 has a large,
high-density red shoulder, indicating high-density gas at the most red velocity
in the system.  Because this is foreground gas, that high-density gas
\emph{must} be moving towards the \uchii region.}
{fig:w51hiispectra}{1}


\ifstandalone
\bibliographystyle{apj_w_etal}  % or "siam", or "alpha", or "abbrv"
%\bibliography{thesis}      % bib database file refs.bib
\bibliography{bibdesk}      % bib database file refs.bib
\fi

\end{document}

% %\documentclass[defaultstyle,11pt]{thesis}
%\documentclass[]{report}
%\documentclass[]{article}
%\usepackage{aastex_hack}
%\usepackage{deluxetable}
\documentclass[preprint]{aastex}


%%%%%%%%%%%%%%%%%%%%%%%%%%%%%%%%%%%%%%%%%%%%%%%%%%%%%%%%%%%%%%%%
%%%%%%%%%%%  see documentation for information about  %%%%%%%%%%
%%%%%%%%%%%  the options (11pt, defaultstyle, etc.)   %%%%%%%%%%
%%%%%%%  http://www.colorado.edu/its/docs/latex/thesis/  %%%%%%%
%%%%%%%%%%%%%%%%%%%%%%%%%%%%%%%%%%%%%%%%%%%%%%%%%%%%%%%%%%%%%%%%
%		\documentclass[typewriterstyle]{thesis}
% 		\documentclass[modernstyle]{thesis}
% 		\documentclass[modernstyle,11pt]{thesis}
%	 	\documentclass[modernstyle,12pt]{thesis}

%%%%%%%%%%%%%%%%%%%%%%%%%%%%%%%%%%%%%%%%%%%%%%%%%%%%%%%%%%%%%%%%
%%%%%%%%%%%    load any packages which are needed    %%%%%%%%%%%
%%%%%%%%%%%%%%%%%%%%%%%%%%%%%%%%%%%%%%%%%%%%%%%%%%%%%%%%%%%%%%%%
\usepackage{latexsym}		% to get LASY symbols
\usepackage{graphicx}		% to insert PostScript figures
%\usepackage{deluxetable}
\usepackage{rotating}		% for sideways tables/figures
\usepackage{natbib}  % Requires natbib.sty, available from http://ads.harvard.edu/pubs/bibtex/astronat/
\usepackage{savesym}
\usepackage{amssymb}
%\savesymbol{singlespace}
\savesymbol{doublespace}
%\usepackage{wrapfig}
%\usepackage{setspace}
\usepackage{xspace}
\usepackage{color}
\usepackage{multicol}
\usepackage{mdframed}
\usepackage{url}
\usepackage{subfigure}
%\usepackage{emulateapj}
\usepackage{lscape}
\usepackage{grffile}
\usepackage{standalone}
\standalonetrue
\usepackage{import}
\usepackage[utf8]{inputenc}
\usepackage{longtable}
\usepackage{booktabs}



%%%%%%%%%%%%%%%%%%%%%%%%%%%%%%%%%%%%%%%%%%%%%%%%%%%%%%%%%%%%%%%%
%%%%%%%%%%%%       all the preamble material:       %%%%%%%%%%%%
%%%%%%%%%%%%%%%%%%%%%%%%%%%%%%%%%%%%%%%%%%%%%%%%%%%%%%%%%%%%%%%%

% \title{Star Formation in the Galaxy}
% 
% \author{Adam G.}{Ginsburg}
% 
% \otherdegrees{B.S., Rice University, 2007\\
% 	      M.S., University of Colorado, Boulder, 2009}
% 
% \degree{Doctor of Philosophy}		%  #1 {long descr.}
% 	{Ph.D., Rocket Science (ok, fine, astrophysics)}		%  #2 {short descr.}
% 
% \dept{Department of}			%  #1 {designation}
% 	{Astrophysical and Planetary Sciences}		%  #2 {name}
% 
% \advisor{Prof.}				%  #1 {title}
% 	{John Bally}			%  #2 {name}
% 
% \reader{Prof.~Jeremy Darling}		%  2nd person to sign thesis
% \readerThree{Prof.~Jason Glenn}		%  3rd person to sign thesis
% \readerFour{Prof.~Michael Shull}	%  4rd person to sign thesis
% \readerFour{Prof.~Neal Evans}	%  4rd person to sign thesis
% 
% \abstract{  \OnePageChapter	% one page only ??
% 
%     I discovered dust in space.  
% 
% 	}
% 
% 
% \dedication[Dedication]{	% NEVER use \OnePageChapter here.
% 	To 1, the second number in binary.
% 	}
% 
% \acknowledgements{	\OnePageChapter	% *MUST* BE ONLY ONE PAGE!
% 	All y'all.
% 	}
% 
% \ToCisShort	% a 1-page Table of Contents ??
% 
% \LoFisShort	% a 1-page List of Figures ??
% %	\emptyLoF	% no List of Figures at all ??
% 
% \LoTisShort	% a 1-page List of Tables ??
% %	\emptyLoT	% no List of Tables at all ??
% 
% 
% %%%%%%%%%%%%%%%%%%%%%%%%%%%%%%%%%%%%%%%%%%%%%%%%%%%%%%%%%%%%%%%%%
% %%%%%%%%%%%%%%%       BEGIN DOCUMENT...         %%%%%%%%%%%%%%%%%
% %%%%%%%%%%%%%%%%%%%%%%%%%%%%%%%%%%%%%%%%%%%%%%%%%%%%%%%%%%%%%%%%%
% 
% %%%%  footnote style; default=\arabic  (numbered 1,2,3...)
% %%%%  others:  \roman, \Roman, \alph, \Alph, \fnsymbol
% %	"\fnsymbol" uses asterisk, dagger, double-dagger, etc.
% %	\renewcommand{\thefootnote}{\fnsymbol{footnote}}
% %	\setcounter{footnote}{0}

\input{macros}		% file containing author's macro definitions

\begin{document}
% \input{introduction}
% 
% %\input{ch_iras05358}
% \input{ch_w5}
% \input{ch_h2co}
% \input{ch_h2colarge}
% \input{ch_boundhii}
% 
% %\input ch2.tex			% file with Chapter 2 contents
% 
% %%%%%%%%%%%%%%%%%%%%%%%%%%%%%%%%%%%%%%%%%%%%%%%%%%%%%%%%%%%%%%%%%%%
% %%%%%%%%%%%%%%%%%%%%%%%  Bibliography %%%%%%%%%%%%%%%%%%%%%%%%%%%%%
% %%%%%%%%%%%%%%%%%%%%%%%%%%%%%%%%%%%%%%%%%%%%%%%%%%%%%%%%%%%%%%%%%%%
% 
% \bibliographystyle{plain}	% or "siam", or "alpha", or "abbrv"
% 				% see other styles (.bst files) in
% 				% $TEXHOME/texmf/bibtex/bst
% 
% \nocite{*}		% list all refs in database, cited or not.
% 
% \bibliography{thesis}		% bib database file refs.bib
% 
% %%%%%%%%%%%%%%%%%%%%%%%%%%%%%%%%%%%%%%%%%%%%%%%%%%%%%%%%%%%%%%%%%%%
% %%%%%%%%%%%%%%%%%%%%%%%%  Appendices %%%%%%%%%%%%%%%%%%%%%%%%%%%%%%
% %%%%%%%%%%%%%%%%%%%%%%%%%%%%%%%%%%%%%%%%%%%%%%%%%%%%%%%%%%%%%%%%%%%
% 
% \appendix	% don't forget this line if you have appendices!
% 
% %\input appA.tex			% file with Appendix A contents
% %\input appB.tex			% file with Appendix B contents
% 
% %%%%%%%%%%%%%%%%%%%%%%%%%%%%%%%%%%%%%%%%%%%%%%%%%%%%%%%%%%%%%%%%%%%
% %%%%%%%%%%%%%%%%%%%%%%%%   THE END   %%%%%%%%%%%%%%%%%%%%%%%%%%%%%%
% %%%%%%%%%%%%%%%%%%%%%%%%%%%%%%%%%%%%%%%%%%%%%%%%%%%%%%%%%%%%%%%%%%%
% 
% \end{document}
% 
% 

%%\documentclass[defaultstyle,11pt]{thesis}
%\documentclass[]{report}
%\documentclass[]{article}
%\usepackage{aastex_hack}
%\usepackage{deluxetable}
\documentclass[preprint]{aastex}


%%%%%%%%%%%%%%%%%%%%%%%%%%%%%%%%%%%%%%%%%%%%%%%%%%%%%%%%%%%%%%%%
%%%%%%%%%%%  see documentation for information about  %%%%%%%%%%
%%%%%%%%%%%  the options (11pt, defaultstyle, etc.)   %%%%%%%%%%
%%%%%%%  http://www.colorado.edu/its/docs/latex/thesis/  %%%%%%%
%%%%%%%%%%%%%%%%%%%%%%%%%%%%%%%%%%%%%%%%%%%%%%%%%%%%%%%%%%%%%%%%
%		\documentclass[typewriterstyle]{thesis}
% 		\documentclass[modernstyle]{thesis}
% 		\documentclass[modernstyle,11pt]{thesis}
%	 	\documentclass[modernstyle,12pt]{thesis}

%%%%%%%%%%%%%%%%%%%%%%%%%%%%%%%%%%%%%%%%%%%%%%%%%%%%%%%%%%%%%%%%
%%%%%%%%%%%    load any packages which are needed    %%%%%%%%%%%
%%%%%%%%%%%%%%%%%%%%%%%%%%%%%%%%%%%%%%%%%%%%%%%%%%%%%%%%%%%%%%%%
\usepackage{latexsym}		% to get LASY symbols
\usepackage{graphicx}		% to insert PostScript figures
%\usepackage{deluxetable}
\usepackage{rotating}		% for sideways tables/figures
\usepackage{natbib}  % Requires natbib.sty, available from http://ads.harvard.edu/pubs/bibtex/astronat/
\usepackage{savesym}
\usepackage{amssymb}
%\savesymbol{singlespace}
\savesymbol{doublespace}
%\usepackage{wrapfig}
%\usepackage{setspace}
\usepackage{xspace}
\usepackage{color}
\usepackage{multicol}
\usepackage{mdframed}
\usepackage{url}
\usepackage{subfigure}
%\usepackage{emulateapj}
\usepackage{lscape}
\usepackage{grffile}
\usepackage{standalone}
\standalonetrue
\usepackage{import}
\usepackage[utf8]{inputenc}
\usepackage{longtable}
\usepackage{booktabs}



%%%%%%%%%%%%%%%%%%%%%%%%%%%%%%%%%%%%%%%%%%%%%%%%%%%%%%%%%%%%%%%%
%%%%%%%%%%%%       all the preamble material:       %%%%%%%%%%%%
%%%%%%%%%%%%%%%%%%%%%%%%%%%%%%%%%%%%%%%%%%%%%%%%%%%%%%%%%%%%%%%%

% \title{Star Formation in the Galaxy}
% 
% \author{Adam G.}{Ginsburg}
% 
% \otherdegrees{B.S., Rice University, 2007\\
% 	      M.S., University of Colorado, Boulder, 2009}
% 
% \degree{Doctor of Philosophy}		%  #1 {long descr.}
% 	{Ph.D., Rocket Science (ok, fine, astrophysics)}		%  #2 {short descr.}
% 
% \dept{Department of}			%  #1 {designation}
% 	{Astrophysical and Planetary Sciences}		%  #2 {name}
% 
% \advisor{Prof.}				%  #1 {title}
% 	{John Bally}			%  #2 {name}
% 
% \reader{Prof.~Jeremy Darling}		%  2nd person to sign thesis
% \readerThree{Prof.~Jason Glenn}		%  3rd person to sign thesis
% \readerFour{Prof.~Michael Shull}	%  4rd person to sign thesis
% \readerFour{Prof.~Neal Evans}	%  4rd person to sign thesis
% 
% \abstract{  \OnePageChapter	% one page only ??
% 
%     I discovered dust in space.  
% 
% 	}
% 
% 
% \dedication[Dedication]{	% NEVER use \OnePageChapter here.
% 	To 1, the second number in binary.
% 	}
% 
% \acknowledgements{	\OnePageChapter	% *MUST* BE ONLY ONE PAGE!
% 	All y'all.
% 	}
% 
% \ToCisShort	% a 1-page Table of Contents ??
% 
% \LoFisShort	% a 1-page List of Figures ??
% %	\emptyLoF	% no List of Figures at all ??
% 
% \LoTisShort	% a 1-page List of Tables ??
% %	\emptyLoT	% no List of Tables at all ??
% 
% 
% %%%%%%%%%%%%%%%%%%%%%%%%%%%%%%%%%%%%%%%%%%%%%%%%%%%%%%%%%%%%%%%%%
% %%%%%%%%%%%%%%%       BEGIN DOCUMENT...         %%%%%%%%%%%%%%%%%
% %%%%%%%%%%%%%%%%%%%%%%%%%%%%%%%%%%%%%%%%%%%%%%%%%%%%%%%%%%%%%%%%%
% 
% %%%%  footnote style; default=\arabic  (numbered 1,2,3...)
% %%%%  others:  \roman, \Roman, \alph, \Alph, \fnsymbol
% %	"\fnsymbol" uses asterisk, dagger, double-dagger, etc.
% %	\renewcommand{\thefootnote}{\fnsymbol{footnote}}
% %	\setcounter{footnote}{0}

\input{macros}		% file containing author's macro definitions

\begin{document}
% \input{introduction}
% 
% %\input{ch_iras05358}
% \input{ch_w5}
% \input{ch_h2co}
% \input{ch_h2colarge}
% \input{ch_boundhii}
% 
% %\input ch2.tex			% file with Chapter 2 contents
% 
% %%%%%%%%%%%%%%%%%%%%%%%%%%%%%%%%%%%%%%%%%%%%%%%%%%%%%%%%%%%%%%%%%%%
% %%%%%%%%%%%%%%%%%%%%%%%  Bibliography %%%%%%%%%%%%%%%%%%%%%%%%%%%%%
% %%%%%%%%%%%%%%%%%%%%%%%%%%%%%%%%%%%%%%%%%%%%%%%%%%%%%%%%%%%%%%%%%%%
% 
% \bibliographystyle{plain}	% or "siam", or "alpha", or "abbrv"
% 				% see other styles (.bst files) in
% 				% $TEXHOME/texmf/bibtex/bst
% 
% \nocite{*}		% list all refs in database, cited or not.
% 
% \bibliography{thesis}		% bib database file refs.bib
% 
% %%%%%%%%%%%%%%%%%%%%%%%%%%%%%%%%%%%%%%%%%%%%%%%%%%%%%%%%%%%%%%%%%%%
% %%%%%%%%%%%%%%%%%%%%%%%%  Appendices %%%%%%%%%%%%%%%%%%%%%%%%%%%%%%
% %%%%%%%%%%%%%%%%%%%%%%%%%%%%%%%%%%%%%%%%%%%%%%%%%%%%%%%%%%%%%%%%%%%
% 
% \appendix	% don't forget this line if you have appendices!
% 
% %\input appA.tex			% file with Appendix A contents
% %\input appB.tex			% file with Appendix B contents
% 
% %%%%%%%%%%%%%%%%%%%%%%%%%%%%%%%%%%%%%%%%%%%%%%%%%%%%%%%%%%%%%%%%%%%
% %%%%%%%%%%%%%%%%%%%%%%%%   THE END   %%%%%%%%%%%%%%%%%%%%%%%%%%%%%%
% %%%%%%%%%%%%%%%%%%%%%%%%%%%%%%%%%%%%%%%%%%%%%%%%%%%%%%%%%%%%%%%%%%%
% 
% \end{document}
% 
% 

\chapter{Bound HII regions and Young Massive Protoclusters}
\label{ch:ympc}



\section{Preface}
During a visit from Eli Bressert, we discussed methods of identifying the
precursors to young massive clusters.  A central idea was that the primary
unbinding energy comes from ionized gas, so that if a region could remain
bound against the pressure provided by ionized gas, it would proceed to
high star formation efficiency.  This notion resulted in two papers: the theory
paper \citep{Bressert2012a} and the observational paper \citep{Ginsburg2012a}.
The observational paper, which summarizes the population of proto-YMCs discovered
in the BGPS, is reproduced here.

\subsection{Abstract}
    
We search the $\lambda=1.1$ mm Bolocam Galactic Plane Survey for clumps
containing sufficient mass to form $\sim10^4~\msun$ star clusters.
%by identifying
%compact ($r\lesssim2.5$ pc) massive ($M_{\rm clump}>10^4$ \msun) dust clumps.  
\ncandidates\ candidate massive proto-clusters  are identified in the first Galactic quadrant outside
of the central kiloparsec.  This
sample is complete to clumps with mass M$_{\rm clump}>\mmin$ and radius
$r\lesssim2.5$ pc.  The overall Galactic massive cluster formation rate is
$CFR({\rm M}_{\rm cluster}>10^4) \lesssim \CFR\  \permyr$, which is in
agreement with the rates inferred from Galactic open clusters and M31 massive
clusters.  We find that all massive proto-clusters in the first quadrant are
actively forming massive stars and place an upper limit of
$\tau_{starless}<\tsuplim$~Myr on the lifetime of the starless phase of massive
cluster formation.  If massive clusters go through a starless phase with all 
of their mass in a single clump, the lifetime of this phase is very short.




\section{Introduction}

The Milky Way contains about 150 Globular clusters (GCs) with masses of $10^4$
to over $10^6$ \msun\ and tens of thousands of open clusters containing from
100 to over $10^4$ stars.  However, young massive clusters containing
$\gtrsim10^4~\msun$ of stars are rare, with only a handful known
\citep{PortegiesZwart2010}. While no GCs have formed in the Milky Way within
the last 5 Gyr, open clusters that survive many crossing times continue to
form.   A few of these clusters have stellar masses greater than $10^4$
M$_{\odot}$ and therefore qualify as young massive clusters
\citep[YMCs;][]{PortegiesZwart2010}.   YMCs must either form from clumps having
masses greater than and sizes comparable to the final cluster  or be formed
from a larger, more diffuse reservoir, in which case massive protocluster
clumps may be rare or nonexistent  \citep{Kennicutt2012}.


%We assume an evolutionary sequence
%from `starless' to `starry' with gas present, after which the gas will be
%ejected, leaving behind a  massive cluster that will live for $\sim100$
%Myr \citep{PortegiesZwart2010}.  We distinguish `massive clusters' from
%lower-mass open clusters with $M<10^4$ \msun\ as they may form by different
%mechanisms.

%\todoeli{Check to make sure this is in agreement with your paper: }
Massive proto-clusters (MPCs) are massive clusters (M$_{\rm cluster}>10^4$ \msun)
in the process of forming from a dense gas cloud.  In \citet{Bressert2012}, we
examine the theoretical properties of MPCs: MPCs are assumed to form from
massive, cold starless clumps analagous to pre-stellar cores
\citep{Williams2000}.  In this paper, we refer to two classes of objects:
starless MPCs, which have very low luminosity and do not contain OB stars, and
MPCs, which are gas-rich but have already formed OB stars.  The only
currently known starless MPC is G0.253+0.016, which lies within the dense
central molecular zone and is subject to greater environmental stresses than
similar objects in the Galactic plane \citep{Longmore2012}.

Because massive clusters contain many massive stars, at some point during their
evolution ionization pressure will prevail over protostellar outflows as the
dominant feedback mechanism.  Other sources of feedback are less than
ionization pressure up until the first supernova explosion
\citep{Bressert2012}.  These proto-clusters must have masses
M$_{\rm clump}>{\rm M}_{*}/SFE$ \footnote{We define a star formation efficiency
$SFE={\rm M}_{\rm *,final} / {\rm M}_{\rm gas,initial}$.}, or about $3\ee{4}$ \msun\ for an assumed
SFE=30\% (an upper limit on the star formation efficiency),
confined in a radius $r\lesssim2.5$ pc, in order to remain bound against
ionization feedback.  These properties motivate our search for proto-clusters
in the Bolocam Galactic Plane Survey \citep[BGPS;][
\url{http://irsa.ipac.caltech.edu/data/BOLOCAM_GPS/}]{Aguirre2011}.


The distinction between relatively short-lived `open clusters' and long-lived
($t\gtrsim1$ Gyr) bound clusters occurs at about $10^4$ \msun
\citep{PortegiesZwart2010}.  Clusters with ${\rm M}_{\rm cluster} < 1\ee{4} \msun$~will
be destroyed by interactions with giant molecular clouds over the course of a
few hundred million years after they have dispersed their gas
\citep{Kruijssen2011}, while clusters with ${\rm M}_{\rm cluster}\gtrsim10^4 \msun$ may
survive $\gtrsim 1$ Gyr.  Closer to the Galactic center, within approximately a
kiloparsec, all clusters will be destroyed on shorter timescales by strong
tidal forces or interactions with molecular clouds.
%Because the lower-mass clusters are destroyed on shorter
%timescales, it is difficult to get a complete census of their population;
%assessing their birth rate may be the best way to determine their overall
%population.

In the Galaxy, there are few known massive clusters.
\citet{PortegiesZwart2010} catalogs a few of them, of which NGC 3603, Trumpler
14, and Westerlund 1 and 2 are the likely descendants of the objects we
investigate.  These clusters have $r_{eff} \lesssim 1$ pc, $M\sim10^4$ \msun,
and ages $t\lesssim4$ Myr.  We present a census of their ancestral analogs.

% Bound open clusters and massive clusters may predominantly form from clumps with
% gravitational escape speeds greater than the sound-speed in photo-ionized gas.
% \S\ 2 uses the BGPS to identify
% candidate dense, massive clumps which may be progenitors to young massive
% clusters. Any clump for which  M$_{\rm clump} \times {\rm SFE}(30\%) > 10^4$
% \msun\ is considered to be a young massive proto-cluster (MPC). \ncandidates\ 
% candidates are sufficiently massive to produce clusters similar to  NGC 3603.
% \S\ 4 discusses these results and \S\ 5 provides concluding remarks.


\section{Observations and Analysis}
\label{sec:ympcobservations}

\subsection{The Bolocam Galactic Plane Survey}
\begin{figure*}
    \includegraphics[width=7in]{{figures_chboundhii/candidates_galacticplot_26.0kpc_10kmsun}.png}
\caption{
\label{fig:galplot}
Plot of the massive proto-cluster (MPC) candidates
overlaid on the Galactic plane.  
%Only sources with $M>1000 \msun$ are included; above
%this cut 
%The symbol size is proportional to the log of the source mass.
The green circle represents the galactic center, and the yellow $\odot$ is the
Sun.
A 15 kpc radius disc centered on the Galactic Center indicates the approximate
extent of Galactic star formation.  The white region indicates the coverage of
the
Bolocam Galactic Plane survey and our source selection limits based on distance
and longitude.  The inner cutoff (light grey) is the nearby incompleteness
limit set by the Bolocam spatial filtering;  the catalog includes sources but
is incomplete in this region.  The red dashed circle traces the solar circle.
Blue filled circles represent initial candidates that passed the mass-cutoff
criterion $M(20K)>\mmin$; red stars are those with $M(20K) > 3\ee{4} \msun$.
In the legend, $M_4$ means mass in units of $10^4 \msun$.  
%Empty squares represent flux-cut candidates that failed the mass cutoff
%criterion.  
%The diamonds are sources from the \citet{Faundez2004} SIMBA survey
%of southern IRAS sources subject to the same cutoffs applied to the Bolocam
%data: filled (green) sources have $M>\mmin$. The central yellow star is the
%Galactic Center. The empty stars are massive clusters
%\citep{PortegiesZwart2010}.
%\todome{Add legend for size-mass}
% Done
}
\end{figure*}

The BGPS is a 1.1 mm survey of the first quadrant of the Galactic plane in the
range $-0.5 < b < 0.5$ with resolution $\sim33\arcsec$ sensitive to a maximum
spatial scale of $\sim120\arcsec$ \citep{Aguirre2011,Ginsburg2012}.  The BGPS `Bolocat' v1.0 catalog
includes sources identified by a watershed decomposition algorithm and flux
measurements within apertures of radius 20\arcsec, 40\arcsec, and 60\arcsec\
\citep{Rosolowsky2010}.

We searched the BGPS for candidate MPCs in the 1st quadrant ($6 < \ell < 90$;
5991 sources).  The inner 6 degrees of the Galaxy are excluded because physical
conditions are significantly different from those in the rest
of the galaxy  \citep{YusefZadeh2009} and the BGPS is confusion-limited in 
that region.

\subsection{Source Selection \& Completeness}
\label{sec:selection}
We identify a flux-limited sample by setting our search criteria to
include all sources with ${\rm M}_{\rm clump}>10^4$ \msun\ in a 20\arcsec\ radius out to 26 
kpc, or a physical radius of 2.5 pc at that distance.  The radius cutoff is
motivated by completeness and physical considerations: the cutoff of 26 kpc includes
the entire star forming disk in our targeted longitudes, and $r=2.5$ pc corresponds
to the radius at which a $3\ee{4}$ \msun\ mass has an escape speed $v_{esc}=10$ \kms, i.e.
ionized gas will be bound. 
The maximum radius and minimum mass imply a minimum mean density
$\bar{n}=6\times10^3~\percc$, which implies a maximum free-fall time $t_{ff}<0.65$~Myr.
%These limitations guide our analysis in Section \ref{sec:discussion}.

Using the Bolocat v1.0 catalog, we first set a flux limit on the sample by assuming
the maximum distance of $d=26$ kpc and imposing a mass cutoff of ${\rm M}_{\rm clump}\geq10^4$ \msun\ 
inside a 20\arcsec\ (2.5 pc) radius aperture.  Following equation 19 in
\citet{Aguirre2011}:
\begin{equation} 
    {\rm M}_{\rm gas}\approx 14.3 \left( e^{13.0/T_d}-1 \right)
        \left({S_\nu\over 1\; {\rm Jy}} \right)
        \left(\frac{D}{{\rm 1~kpc}}\right)^{2} \msun 
\end{equation}   
and assuming $T_{dust}=20$K, the implied flux cutoff is 1.13 Jy \footnote{As
per \citet{Rosolowsky2010}, \citet{Aguirre2011}, and \citet{Ginsburg2012}, a
factor of 1.5 calibration correction and 1.46 aperture correction are required
for the 20\arcsec\ radius aperture fluxes reported in the catalog.  These
factors have been applied to the data. }, above which \nsample\ `flux-cutoff'
candidates were selected in the Bolocat v1.0 catalog.  Cutoffs of 4.3 Jy for
the 40\arcsec\ and 10.2 Jy for the 60\arcsec\ Bolocat v1.0 apertures were used
to select more nearby candidates inside the same physical radius, but no
sources were selected based on these larger apertures.

%By measuring flux within an aperture, we are measuring the mass within a given
%radius, which means that the source may be substantially smaller than we
%assume.  The identified sources may therefore have higher escape velocities
%than the minimum $\sim10~\kms$ required.

% redundant? We applied a $M_{\rm clump} > 10^4$ \msun\ cutoff at
%the maximum distance of 17.5 kpc; we are therefore complete to a progenitor
%mass of $1\times10^4$ \msun.

The BGPS is insensitive to scales larger than 120\arcsec\
\citep[][]{Ginsburg2012}\footnote{\citet{Ginsburg2012} presents v2.0 of the
BGPS}.  As a result, the survey is incomplete below a distance $$D_{min} =
\mindist \left(\frac{r_{cluster}}{2.5 \textrm{pc}}\right) \textrm{kpc} $$ from
the Sun.  Within this radius, alternate methods must be sought to determine the
total mass within $r_{cluster} < \rcluster$ pc.  Although the sample is
incomplete for $D < \mindist$ kpc, sources that have sufficient mass despite
the 120\arcsec\ spatial filtering are included.

%We are
%able to identify some sources within this cutoff distance because they have
%enough mass in a smaller radius, but we are not complete at
%$D<\mindist$ kpc.
%Luckily, a great deal is known
%about nearby proto-clusters, and it is possible to determine the masses of many using
%alternate methods such as near and mid-infrared extinction.  These nearby star-forming
%regions are discussed in Section \ref{sec:nearcand}.

Distances to BGPS-selected candidates were determined primarily via literature
search.  Where distances were unavailable, we used velocity measurements from
\citet{Schlingman2011} and assumed the far distance for source selection.  We
then resolved the kinematic distance ambiguity towards these sources by
searching for associated near-infrared stellar extinction features from the
UKIDSS GPS \citep{Lucas2008}.
%and distances from \citet{EllsworthBowers2012} where
%literature distances were unavailable \footnote{\citet{EllsworthBowers2012} combine radial velocity measurements with 
%a variety of kinematic distance ambiguity resolution methods to measure a distance likelihood function for each source.
%We use the maximum likelihood distance from this method.}. 
Most literature distances were determined using a
rotation curve model and some method of kinematic distance ambiguity
resolution. Because the literature used different rotation curve models, there
is a $\sim10\%$ systematic error in distance resulting in a $\sim20\%$
systematic error in mass. We used the larger
40\arcsec\ radius apertures to determine the flux for sources at
$D<\middistcut$ kpc and 60\arcsec\ radius apertures for sources at $D<\mindist$
kpc (corresponding to $r<\rcluster$ pc).

% In addition, we used the radial velocity measurements from
% \citet{Schlingman2011} and \citet{Shirley2012} to determine a maximum mass for
% each source by assuming the source is at the far kinematic distance. This
% assures our survey's completeness even if the distances acquired from the above
% methods prove incorrect.  Even assuming the far distance for all flux-cutoff
% candidates, there are no additional sources with $M>3\times10^4$ \msun, though 
% there were a handful with $M>10^4 \msun$.


The masses were computed assuming a temperature $T_{dust}=20$K, opacity
$\kappa_{271.1 GHz} = 0.0114~\mathrm{cm}^2 \mathrm{g}^{-1}$, and gas-to-dust
ratio of 100  \citep{Aguirre2011} \footnote{$T_{dust}=20$K is more appropriate
for a typical pre-star-forming clump than an evolved HII-region hosting one
\citep[e.g.]{Dunham2010}. However, because we are interested in cold
progenitors as well as actively forming clusters, the selection is based on
$T_{dust}=20$K, which is more inclusive. }.  The mass estimate drops by a
factor of $2.38$ if the temperature assumed is doubled to $T_{dust}=40$K.  

\citet{Ginsburg2011} notes that significant free-free contamination, as high as
80\%, is possible for some 1.1 mm sources, so the selected candidates may prove
to be more moderate-mass and evolved proto-clusters.  We used the NRAO VLA
Archive Survey \citep[NVAS;][]{Crossley2008} to estimate the free-free
contamination for the sample.  For most sources, the free-free contamination
inferred from the VLA observations is small ($<20\%$), but for a subset the
contamination was $\sim20-35\%$ assuming that the free-free emission is
optically thin.  Corrected masses using the measured free-free contamination
and higher dust temperatures are listed in Table \ref{tab:candidates}; these
are reasonable lower limits on the total mass of these regions.  All of the
contamination estimates are technically lower limits both because of the
assumption that the free-free emission is optically thin and because the VLA
filters out large-scale flux.  However, in most cases, the emission is likely
to be dominated by optically thin emission \citep[evolved HII regions tend to
be optically thin and bright, while compact HII regions are optically thick but
relatively faint;][]{Keto2002} and for most sources VLA C or D-array
observations were used, and at 3.6 and 6 cm the largest angular scale recovered
is 180-300 \arcsec, greater than the largest angular scale in the BGPS.  

Applying a cutoff of M$_{\rm clump} > 10^4$ \msun\ left \ncandidates\
protocluster candidates out of the original \nsample.  The more stringent cut
M$_{\rm clump} > 10^4 / SFE \approx 3\ee{4}$ \msun\ leaves only \nMPC\ MPCs . 
% All of the \nsample\
% `flux-cutoff' candidates with $M>10^3$ \msun\ are shown in \figref{fig:galplot}
% providing context of their location in the Galaxy. 

The final candidate list contains only sources with $M(20K)>10^4 \msun$ (the
completeness limit; see Table \ref{tab:candidates}).  The table lists
their physical properties, their literature distance, their mass (assuming $T_{dust}=20
\textrm{~and~} 40 K$ and a free-free subtracted lower-limit) ,
%\todome{Discuss varying dust opacity?  Martin et al 2011} 
% Ignored.  Unnecessary.
and their inferred escape speed ($v_{esc} = \sqrt{2 G M(20K) / r}$) assuming a
radius equal to the aperture size at that distance.  The table also includes
measurements of the IRAS luminosity in the 60 and 100 \um\ bands within the
source aperture.
%The literature search
%revealed that all candidates are known massive-star-forming regions.

\subsection{Source Separation}
These \ncandidates\ candidates include some overlapping sources.
There are two clumps in W51 separated by about 1.5 pc and 4.5 \kms\ along the
line of sight that are each independently massive enough to be classified as
MPCs, but are only discussed as a single entity because they are likely
to merge if their three-dimensional separation is similar to their projected
distance.  The candidates in W49 are more widely separated, about 4.4 pc and 7
\kms\ along the line of sight, but could still merge.

Additionally, 9 of the \ncandidates\ are within 8.7 kpc, so the mass
estimates are lower limits.  These are promising candidates for follow-up, but
cannot be considered complete for population studies.  If our radius restriction
is dropped to 1.5 pc, the minimum complete distance drops to 5.2 kpc and the
three lowest-mass sources in Table \ref{tab:candidates} no longer qualify, but
otherwise the source list remains unchanged.  Our analysis is therefore robust
to the selection criteria used.
%(except the W51 pair, which meets
%the selection criteria despite its proximity).

% Not really interesting?
% \subsection{Line Widths}
% To back up the claim that these proto-clusters cannot be unbound by ionization
% pressure, we examine the line widths in dense gas tracers.  Using a tracer that
% measures the internal motions  of the gas in the gravitational potential, we
% expect the approximately gaussian line-width to represent the quasi-equilibrium
% state of the gas (i.e., if the line width is changing, it is doing so slowly
% relative to the star formation process).  Because the line widths are much
% larger than the sound speed in the neutral gas, the observed clouds must be
% gravitationally bound, otherwise they would expand and their linewidths would
% drop to the sound speed on a dynamical timescale.
% %\todojohn{This is in keeping with results on GMCs, i.e. that they are in equilibrium.
% %Is there a better way to state it?  Are there other results / theoretical arguments that
% %should be cited?}
% 
% Heterodyne observations of HCO$^+$ 3-2 and N$_2$H$^+$ 3-2 data from
% \citet{Schlingman2011} are presented in Table \ref{tab:candidates}.  With
% critical densities $\gtrsim 2\ee{6}$ \percc, these both trace dense gas and therefore
% are limited to the proto-cluster region.  However, HCO$^+$ has frequently been
% observed in self-absorption, so the N$_2$H$^+$ widths are more reliable.
% We report the FWHM of single-component fits.  In order to mitigate the effects
% of self-absorption on the line fitting, we report HCO$^+$ line widths fitted by
% ignoring the central self-absorbed pixels; the channel selection was done by
% eye.
% All of the candidates selected as proto-massive-cluster candidates have
% $v_{internal} \approx v_{esc} > v_{ionized}$, confirming their candidacy.

%In Table 1 we provide a grading scheme to quantify the quality of the
%candidates. Candidates associated with a grade of {\bf A} will form a $\gtrsim
%10^4$ \msun\ cluster, even if $T_{dust}=40$K and SFE = 30\%, where it's mass is
%reduced from the estimates shown in Table 1 by a factor of 2.38. The two latter
%grades, {\bf B} and {\bf C}, follow the same assumptions and the candidates
%masses will fall down to $\sim 10^4$ \msun\ and $<10^4$ \msun, respectively. 



% \subsection{Nearby Candidates}
% \label{sec:nearcand}
% Because the BGPS is insensitive to large angular scales, we must resort to
% other methods for determining protocluster masses in nearby star-forming
% regions.
% The strongest candidates within 5.8 kpc are M17, NGC 7538, W3, S255, W43, W33, G34.15, and others?  While these
% regions are all known to be forming massive stars and have total gas reservoirs
% with $M>10^5$\msun, their large spatial extents mean that they are all more likely
% to form OB associations than bound clusters.  However, some are likely to be MPs...

%In the 3-6 kpc range, W43, W33

%For example, in the W3 region, no clumps have velocity dispersions
%$\sigma_{FWHM} > 6$ \kms, implying that none can keep ionized gas bound
%\citep{Bieging2011}.

% We searched the BGPS for candidate MPCs in the 1st quadrant ($6^o < \ell <
% 90^o$). Using the Bolocat catalog we marked sources with flux densities in a
% 20\arcsec\ aperture that yield a mass $M_{\rm clump}\geq 3\times 10^{4}$ \msun\
% at a distance of 17.5 kpc or less ($20\arcsec\ = 1.7 $ pc at 17.5 kpc)
% assuming $T_{dust}=20$K. We applied a $M_{\rm clump} \times {\rm SFE}(30\%) >
% 10^4$ \msun\ cutoff at the maximum distance of 17.5 kpc; we are therefore
% complete to a progenitor mass of $3\times10^4$\msun. These criteria led to a
% flux-density cutoff of 3.2 Jy, above which 16 candidates were detected in the
% Bolocat catalog. Distances to these candidates were determined via a
% literature search. The final candidate list is given in Table 1 along with
% their physical properties of measured line widths from N$_2$H+, their
% literature distance, their mass (assuming $T_{dust}=20 K$), and their
% inferred escape speed ($v_{esc} = \sqrt{2 G M / r}$) assuming a radius equal
% to the aperture size at that distance.  The literature search also revealed
% that all of our candidates are known star-forming regions, so our list
% contains no contaminants.  
% 
% With distance determinations to these candidates, we were able to compute
% masses assuming a temperature $T_{dust}=20$K, opacity \citep{Aguirre2010},
% and gas-to-dust ratio of 100 \footnote{$T_{dust}=20$K is more appropriate for
% a typical pre-star-forming clump than an evolved HII-region hosting one
% \citep[e.g.][]{Dunham2009}. However, because we are interested in cold
% progenitors as well as actively forming clusters, we estimate the masses of
% the cluster progenitor candidates using $T_{dust}=20$K or $T_{dust}=40$K to
% reflect whether the environment is cold and quiescent or actively forming
% stars. }
% The mass estimate drops by a factor $2.38$ if the temperature assumed is
% doubled to $T_{dust}=40$K. Additionally, for more nearby sources, larger
% apertures (corresponding to the same physical radius) were used to include
% more source flux. Applying a cutoff of M$_{\rm clump} \times {\rm SFE}(30\%)
% > 10^4$ \msun\ left 2 MPCs out of the 16. All of the 16 flux-cutoff
% candidates are shown in Figure \ref{fig:galplot} providing context on where
% they are in the Galaxy. 
% 
% In Table 1 we provide a grading scheme to quantify the quality of the candidates. Candidates associated with a grade of {\bf A} will form a $\gtrsim 10^4$ \msun\ cluster, even if $T_{dust}=40$K and SFE = 30\%, where it's mass is reduced from the estimates shown in Table 1 by a factor of 2.38. The two latter grades, {\bf B} and {\bf C}, follow the same assumptions and the candidates masses will fall down to $\sim 10^3$ \msun\ and $<10^3$ \msun, respectively. 

%\subsection{IRAS luminosities}
%We can and should derive IRAS luminosities for the candidates (this is trivial)
%and compare them to the IRAS luminosity function in the GP.  We can then
%extrapolate the observations to the southern hemisphere and be TRULY complete.

\section{Results}
% \subsection{Proto-Cluster Mass Function}
% \choppingblock
% In order to measure a mass function, we need better detection statistics than
% are provided by our \nMPC\ MPCs.  We note that our observations are complete to
% $M>5000\msun$ in the range $5.8 < D < 12.4$ kpc \todome{Is the 5000 \msun\
% cutoff used anywhere else?  I don't think so}.  In this range, there are
% \ncomplete\ candidates, which follow a mass function $\alpha=\plaw\pm\plawerr$
% \footnote{Computed using the \citet{Clauset2009} MLE as implemented at
% \url{http://code.google.com/p/agpy/wiki/PowerLaw}}.  If these candidates represent proto-clusters, they should
% follow a Schechter function with a cutoff near $10^4$ \msun\
% \citep{PortegiesZwart2010}.  However, the distribution is more consistent with
% a power law $\alpha=2$ than a Schechter function with a cutoff $M<5\times{10^5}
% \msun$.  Above this cutoff, the Schecter function and power-law are
% indistinguishable for our sample.  \todome{Discuss implications?  Not if
% chopped}



% unnecessary comment 
% The mass function of \emph{all} of our candidates independent of
% mass and distance is consistent with a power-law with $\alpha=1.8\pm0.1$ and
% completeness cutoff 800 \msun, but we don't believe this is a fair description
% of the observations because of the varying sensitivity with distance.

\subsection{Cluster formation rate}
\label{sec:cfr}

The massive clumps in Table \ref{tab:candidates} can be used to constrain the
Galactic formation rate of massive clusters (MCs) above \mmin\ if we assume
that the number of observed proto-clusters is a representative sample. The region
surveyed covers a fraction of the surface area of the Galaxy
$f_{observed}=A_{survey} / A_{Galaxy} \approx \obsfrac\%$ assuming the star
forming disk has a radius of 15 kpc\footnote{The observed fraction of the
galaxy changes to 21\% if we only include the area within the solar
circle as discussed in \S \ref{sec:discussion}.}.
%The fraction observed is also
%about 28\% if we assume the star-forming disk
%truncates at 13.5 instead of 15 kpc \citep{Kennicutt2012}.}.  
The cumulative
cluster formation rate above a cluster mass ${\rm M}_{cl}$ is given by $$CFR(>{\rm M}_{cl})
= \frac{N_{MPC}}{\tau_{SF} f_{observed}}$$ where $ \tau_{SF} \approx 2$\ Myr is
the assumed cluster formation timescale \footnote{$\tau_{SF}$, the time from the start of star formation
until gas expulsion, is a poorly understood
quantity, but is reasonably constrained to be $\gtrsim1$~Myr from the age
spread in the Orion Nebula cluster \citep{Hillenbrand1997} and $\lesssim10$~Myr
because the most massive stars will go supernova by that time.}.
% $v_{esc}$ is the escape speed from radius $R$ and $f_A \sim 0.1$ is the
% projected area filling-factor of dense star-forming gas in the clump.  Dense
% cores may survive for $f_A$ crossing-times before colliding.
%The cumulative
%cluster formation rate is given by
%$$
%CFR (>M_{cl}) = 
%{{f_A N_{MPC}(>M) [SFE ]V_{esc}  } 
%        \over 
% { 2 R f_{observed} }}
%$$
%where $f_A = 0.1$, $V_{esc} $= 10
%\kms , $R = \rcluster$ pc, and  $\tau_{SF} = 2$ Myr.  
With the measured
$N_{MPC}({\rm M}_{\rm cluster}>10^4\msun) = \nMPC $\ proto-clusters, we infer a Galactic formation rate 
$$CFR \lesssim \CFR \left(\frac{\tau_{SF}}{2
~\textrm{Myr}}\right)^{-1} \textrm{~Myr}^{-1}$$  This cluster formation rate is
statistically weak, with Poisson error of about 3.5 
\permyr\ and can be improved with more complete surveys \citep[e.g., Hi-Gal,][]{Molinari2010}.  This
formation rate is an upper limit because all of the estimated
masses are upper limits as discussed in Section \ref{sec:selection}.


\subsection{Comparison to Clusters in Andromeda}
%Comparison to Andromeda or direct measurements should provide a prediction of
%the number of clusters in the largest (two?) mass bins.  Do our observations
%agree with such a prediction?
%
%If YES, the implication is that cluster formation proceeds rapidly and
%forms massive, dense, proto-cluster "cores" before actually forming the
%cluster.

%If NO, cluster formation is SLOW and accretion onto the cluster after the
%initial formation may continue and increase cluster mass by factors $>2$ (less
%than that, it doesn't really matter).  In this case, predicting the cluster
%formation rate from protoclusters requires completeness down to smaller mass - 
%i.e., we need to be able to observe the cluster `seeds' in addition to the dense
%pre-clusters.

We use cluster observations in M31 from \citet{Vansevicius2009} to infer the
massive cluster formation rate in M31.  They observe 2 clusters with
${\rm M}_{\rm cluster}>10^4\msun$ and ages $<10$ Myr in 15\% of the M31 star-forming
disk.  The implied cluster formation rate in Andromeda is $\dot{N_{cl}} =
N_{cl}/0.15 / (10 ~\mathrm{Myr}) \approx 1.3$ \permyr.  Given M31's total star
formation rate $\sim 5\times$ lower than the Galactic rate \citep[Andromeda
$\mdot=0.4$, Milky Way $\mdot=2$ \msun \permyr;][]{Barmby2006,Chomiuk2011}, the
predicted Galactic cluster formation rate is $\dot{N_{cl}}(MW) = 5~
\dot{N_{cl}}(M31) = 6.5$ \permyr \citep[assuming the CFR scales linearly with
the SFR; ][]{Bastian2008}.  
%Given our assumed star formation timescale $\tau_{SF}$, the expected
%present-day number of clusters $N_{cl}(MW) = (6.5 \textrm{~MC~}\permyr) (\tau_{SF}) = 13$
%clusters in the Galaxy.  Using the mass cutoff of $3\times10^4$ \msun, we
%detect \nMPC\ MPCs, implying there are \nMPCtot\ MPCs in the galaxy.  
The scaled-up Andromeda cluster formation rate matches the observed Galactic
cluster formation rate.  The samples are small, but as a sanity check, the
agreement is comforting.

% \choppingblock
% The agreement between the M31-based
% prediction and our observations is (un?) remarkable, considering that Poisson
% statistics alone imply a $>50\%$ uncertainty in each of the cluster counts
% produced above.

% We present a CFR function dependent on the local surface density of gas in a 
% galaxy $\Sigma_{\rm gas}$ (normalized by $\Sigma_0 = Value$), the survival time 
% of the cluster $\tau$, and the initial mass of the cluster, $M_{init}$. We assume 
% that it has a functional form 
% $$ CFR = A \biggr[ {{\tau} \over {t_{10}}} \biggr] ^{\alpha} \biggr[ {{M_{init}} \over {M_3}} \biggr] ^{\beta} \biggr[ {{\Sigma_{\rm gas}} \over {\Sigma_0}} \biggr] ^{\gamma} $$ 
% in units of number of clusters forming per $10^6$ years (= 1 Myr) per square kpc. 
% Here $t_{10}$ is in units fo 10 Myr, and $M_3$ is in units of $10^3$ \msun . From 
% a fit to Galactic and extra-galactic cluster catalogs approximate values are $A = 2$ 
% to $5$ clusters per square kpc per Myr, $\alpha = -1.0$, $\beta = -1.5$, and 
% $\gamma = 1.5$. This implies a one square kpc region around the Sun contains 
% about 20 to 50 short-lived ($<$ 1 Myr) clusters or expanding associations ($\sim3*10^4$ 
% Galaxy wide), and forms only 0.2 to 0.5 open clusters (such as the Pleiades) per kpc$^{-2}$ that last 100 Myr.

% \subsubsection{Comparison to observed Galactic clusters}
% 
% \choppingblock
% 
% % Better to use Piskunov 2008 numbers
% %Given a cluster birth rate of 0.2-0.5 $\permyr \perkpc$ (Battinelli \&
% %Capuzzo-Dolcetta 1991, Piskunov et al. 2006), the expected number of clusters
% %between the CMZ (cut off at a Galactic radius of 1 kpc) and the solar circle at
% %8.5 kpc is 14-36 \permyr. However, the measured clusters in the reported
% %surveys are only $\sim 500 \msun$, while we are interested in more massive ($M
% %> 10^4\msun$) clusters. Assuming a power-law distribution with $\beta=2$ and
% %that the Piskunov sample measures a CFR for clusters in the range $100 < M_C <
% %1000 \msun$, we derive a CFR of $2-5\times10^{-4} \permyr \perkpc$. Assuming
% %these live $\sim 20$ Myr, corresponding to the longest lifetime in Portegies
% %Zwart et al's MC list, the expected number of massive clusters is
% %$\sim0.3-0.8$ within the solar circle. 
% % This number is FAR too low, considering that simply counting MCs gets you at least 7 XXXX 
% 
% 
%  \citet{Piskunov2008} claim a cluster formation rate
% of 0.4 \perkpc \permyr\ integrated over all local ($d<850$ pc) clusters.  Integrated over the Galactic 
% plane, this implies $CFR = 281 \permyr$.  However, the clusters observed in this 
% local sample all have large radii ($r>10$ pc) or small masses ($M<10^3 \msun$), 
% and therefore it is difficult to place constraints on the massive CFR from this data.
% 
% Given our observed cluster counts, there was only a 5\% probability
% of finding an $M>3\times10^4$ \msun\ proto-cluster and 17\% probability of an $M>10^4\msun$
% proto-cluster within the 850 pc completeness zone of \citet{Piskunov2008}.  Multiplying
% by the ratio of a MC / MP lifetime (about an order of magnitude) suggests that
% we were likely to find 1-2 objects with $M>10^4$ \msun\ in the Piskunov sample.
% None were found.  \todome{What are the chances of finding 0 in the local sample given
% the ``measured'' rate?}
% 
% \citet{Piskunov2008} also observe a steepening of the cluster mass function
% with age, from $\alpha\sim1$ to $\alpha\sim2$.  Our observed proto-cluster mass
% function is steep, with $\alpha\sim2$.  This contradiction suggests
% either a selection effect in the \citet{Piskunov2008} sample avoiding low-mass
% young clusters, that our sample \emph{under}estimates the cluster masses particularly
% on the high-mass end, or that low-mass proto-clusters live longer than
% high-mass proto-clusters.  The latter explanation would result in an excess of
% observable low-mass protoclusters compared to the cluster
% mass function (i.e., observable protoclusters should have $\alpha>2$).  It is also expected in theory since, for fixed radius,
% $t_{ff} \propto M^{-1/2}$.

\subsection{Star Formation Activity}

In the sample of potential proto-clusters, all have formed massive stars based
on a literature search and IRAS measurements.  A few of the low mass sources,
G012.209-00.104, G012.627-00.016, G019.474+00.171, and G031.414+00.307 have
relatively low IRAS luminosities ($L_{IRAS} = L_{100}+L_{60} < 10^5 \lsun$) and
little free-free emission.  However, \emph{all} are detected in the radio as
H~II regions (some ultracompact) and have luminosities indicating early-B type
powering stars.
%The lowest IRAS 100 \um\ luminosity in our sample is
%$L_{100}(G19.47)\approx6\times10^3 \lsun$; the rest have $L_{100} >
%2\times10^5 \lsun$.  All of the massive candidates therefore require O-type
%powering stars.

Non-detection of `starless' proto-cluster clumps implies an upper limit on the
starless lifetime. For an assumed $\tau_{sf} \sim 2$~Myr, the $1\sigma$ upper
limit on the starless proto-MC clump is $\tau_{starless} <
(\sqrt{N_{cl}}/N_{cl}) \tau_{sf} = \tsuplim~\mathrm{Myr}$ assuming Poisson
statistics and using all 18 sources.  This limit is consistent with massive
star formation on the clump free-fall timescale ($\tau_{ff}\leq0.65$ Myr).  It
implies that massive stars form rapidly when these large masses are condensed
into cluster-scale regions and hints that massive stars are among the first to
form in massive clusters.
%, also the crossing time for $c_s=10 \kms$ and $r=1.7$ pc).  


%It may indicate that massive stars form simu
%before all of the proto-cluster mass
%has been collected into a compact region, i.e. that collapse from molecular
%cloud to proto-cluster clump proceeds after massive stars have ignited within
%the proto-cluster, although the small number statistics allow for other
%explanations.  

\section{Discussion}
\label{sec:discussion}

Assuming a lower limit 30\% SFE and T$_{dust} = 20 {\rm K}$, \nMPC\ candidates
in Table \ref{tab:candidates}  will become massive clusters like NGC 3603:
G010.472+00.026, W51, and W49 (G043.169+00.01).  Even if  T$_{dust} = 40 {\rm
K}$, W49 is still likely to form a $\sim10^4$ \msun\ MC, although G10.47 would
be too small.  W51, which is within the spatial-filtering incompleteness zone,
passes the cutoff and is likely to form a pair of massive clusters.  However,
if the dust in W51 is warm and the free-free contamination is considered, the
total mass in each of the W51 clumps is below the 3\ee{4} \msun\ cutoff.
% \todocara{Do massive clusters in the galactic center affect this discussion?
% Eli's comment: No, different physics in the GC mean we should not be
% concerned.}
% IGNORED unless the referee says otherwise

The BGPS covers about \obsfrac\% of the Galactic star-forming disk in the range
1 kpc $< R_{gal}<15$ kpc.  We can extrapolate our \nMPC\ detections to predict
that there are $\leq$\nMPCtot\ ($\pm \nMPCtoterr$) proto-clusters in the Galaxy
outside of the Galactic center.
The agreement between the SFR-based prediction
from M31 and our observations implies that we have selected genuine massive
proto-clusters (MPCs).  

These most massive sources have escape speeds greater than the sound speed in
ionized gas, indicating that they can continue to accrete gas even after the
formation of massive stars.  Assuming they are embedded in larger-scale gas
reservoirs, we are measuring lower bounds on the `final' clump plus cluster
mass. 

% Additionally, in the 15\% of the galaxy within the 5.8 kpc
% radius in which the survey is incomplete due to spatial filtering, we predict
% that there should be $2\pm1$ MPCs.  
% it's actually 41%, we probably predict more like a few....

%\subsection{Lifetimes}
%In order to estimate the formation rate from our candidate source counts, we 
%need to include a 

%\begin{figure}
%\includegraphics[width=8.5cm]{figures/mass_vs_omega.pdf}
%\label{fig:m_vs_o}
%\caption{The mass of the young massive proto-clusters (MPCs) versus their respective $\Omega$ value ($V_{esc} / C_{s}$). We present MPC candidates that have $\Omega > 1$ and clump masses greater than $10^3$ \msun. The candidates are graded based on their potential to forming a high mass stellar cluster regarding their assumed temperature and star formation efficiency (SFE). For the grading scheme we assume that $T_{dust}=40$K and a SFE of 30\%, essentially a worst case scenario for cluster forming conditions as the mass of the gas clumps is reduced by 42\%. The candidates are ranked on how much stellar mass they will have after gas dissipation. The red triangles represent grade {\bf A} candidates where their stellar mass will be greater than $10^4$ \msun\. Green squares represent grade {\bf B} candidates where their stellar mass will be greater than $10^3$ \msun. The blue diamonds are grade {\bf C} candidates that will have less the $10^3$ \msun\ in stellar mass.}
%\end{figure}

%this paragraph is somewhat in contradiction to Piskunov2008
% Open clusters generally have masses $<10^4$~\msun\ with a lifetime of $<1$~Gyr
% and their disruption can start early in life from their local environment
% \citep{PortegiesZwart2010}. MCs ($>10^4$~\msun) on the other hand are less
% sensitive to the surrounding environment than open clusters and typically
% remain bound for 1 $<$ t $<$ 10  Gyr \citep{PortegiesZwart2010}.


% look at Piskunov Figure 5 - Cluster Mass Functino

%Assuming that we have indeed acquired a complete sample of MPCs
%over their $\sim2 $ Myr lifetimes, this implies that clusters form within highly
%condensed clumps of gas and dust, rather than slowly accreting from large-scale
%mass reservoirs.  


%What of the locations of proto-clusters?  
All of the young massive proto-clusters candidates observed are within the
solar circle despite our survey covering more area outside of
the solar circle.  
%In other galaxies, e.g. M33, the most massive
%cluster (NGC 604) is found in the outer disk; this situation appears not to
%occur in the Milky Way.  
%Since both shear and density are higher in the inner
%galaxy, it appears that gas density is a more important factor than shear
%forces in determining where massive clusters form within the Galaxy.
%
The outer radius limit for massive cluster formation is consistent with the
observed metallicity shift noted at the same radius by \citet{Lepine2011}.
They identify the solar circle as the corotation radius of pattern speed and
orbits within the Galaxy (within this radius, stars orbit faster than the
spiral pattern).  The fact that this radius also represents a cutoff between
the inner, massive-cluster-forming disk and the outer, massive-cluster-free
disk hints that gas crossing spiral arms may be the triggering mechanism for
massive cluster formation.  However, given the small numbers, the detected
clusters are consistent with a gaussian + exponential disk distribution
following that described by \citet{Wolfire2003}.  
%Outside of 8.5 kpc, the open cluster metallicity is approximately flat out to
%20 kpc.  Inside 8.5 kpc, the metallicity increases.  If gas is prevented from
%mixing at the spiral arm corotation radius, a higher average gas density within
%that radius would lead to increased star and cluster formation.
% Our observations are consistent with \citet{Lepine2011}
%\todojohn{Please help me expand this: Where is NGC 604 in M33 ($R_{gal}$)?
%Are there other (classes of?) galaxies that have clusters in the outer disk?  
%What defines outer disk?}

% EXPLAIN MORE IF YOU INCLUDE (didn't make sense on a second reading)
% % See Portegies-Zwart section 4.4
% The presence of massive clusters exclusively in the inner galaxy ($R_{gal} <
% 8.5 kpc$) is a strong indication that these clusters form from giant molecular
% clouds (although this hypothesis was never really in question) because GMCs
% destroy lower-mass clusters \citep{PortegiesZwart2010}.  The survival time of a
% cluster is proportional to its density and inversely proportional to the GMC
% density.   Statistics on proto-clusters may therefore present a new tool for
% the study of cluster disruption.

%, in which case the BGPS surveys just
%shy of half the potential-cluster-forming region. Then we should see three
%MPCs amongst the candidates we detected. The CFR estimates best agree with the
%conditions that our MPC candidates have T$_{dust} = 20 {\rm K}$ and SFE of
%30\% to 50\% (2 or 4 MPCs). 
%Regarding the lower mass candidates (10$^3$ \msun), we would expect to find $\sim 10$ open cluster progenitors. We have 14 such candidates, which is within a factor of two of the CFR estimate. 


Future work should include a census for MPCs within $D\lesssim5$ kpc using the
Herschel Hi-Gal survey \citep{Molinari2010} and in the Southern plane with
ATLASGAL \citep{Schuller2009}.  Some surveys have already identified
proto-clusters in these regions \citep[e.g.][]{Faundez2004,Battersby2011}, but
they are not complete.  A complete survey of distances will be essential for
continuum surveys to be used.

% Added as per Referee's comments
There are two modes of massive cluster formation consistent with our
observations that can be observationally distinguished.  Either a compact
starless massive proto-cluster phase does occur and is short, or the mass to be
included in the cluster is accumulated from larger volumes over longer
timescales.  Extending our proto-cluster survey to the Southern sky, e.g. using
the ATLASGAL and Hi-Gal surveys, will either discover starless MPCs or
strengthen the arguments that there is no starless MPC phase.  If instead
massive clusters form by large scale ($r>2.5$ pc) accretion, substantial
reservoirs of gas should surround these most massive regions and be flowing
into them.  Signatures of this accretion process should be visible: MPCs should
contain molecular filamentary structures feeding into their centers
\citep[e.g.][]{Correnti2012,Hennemann2012,Liu2012}.  Alternatively, lower mass
clumps may merge to form massive clusters \citep{Fujii2012}, in which case
clusters of clumps - which should be detectable in extant galactic plane
surveys - are the likely precursors to massive clusters.  Finally, massive
clusters may form from the global collapse of structures on scales larger than
we have probed, which could also produce clusters of clumps.


\section{Conclusions}
\label{sec:ympcconclusions}

Using the BGPS, we have performed the first flux-limited census of massive
proto-cluster candidates.  We found \ncandidates\ candidates that will be part
of the next generation of open clusters and \nMPC\ that could form massive
clusters similar to NGC 3603 (${\rm M}_{\rm cluster} > 10^4$ \msun).   We have
measured a Galactic massive cluster formation rate $CFR({\rm M}_{\rm
cluster}>10^4) \lesssim \CFR\  \permyr$\ assuming that clusters are equally
likely to form everywhere within the range 1 kpc $ < R_{gal} < $ 15   kpc. 
%however, we think formation limited to 1<r<8.5; what does this imply?
%(the CFR is the same if we assume, as observed, that all massive cluster formation occurs within the solar circle).  
The observed MPC counts are
consistent with observed cluster counts in Andromeda scaled up by $SFR_{M31} /
SFR_{MW}$ assuming a formation timescale of 2 Myr.  
%A lack of massive clusters
%detected in the local neighborhood \citep[$d_{max}\sim 850$ pc]{Piskunov2008} is also
%consistent with the MPC detection rate and assumed lifetime.

Despite this survey being the first sensitive to pre-star-forming MPC clumps, none
were detected.  This lack of detected pre-star-forming MPCs suggests a
timescale upper limit of about $\tau_{starless}<\tsuplim$ Myr for the pre-massive-star phase of
massive cluster formation, and hints that massive clusters may never form
highly condensed clumps ($\bar{n}\gtrsim10^4~\percc$) prior to forming massive
stars.
It leaves open the possibility that massive clusters form from large-scale
($\gtrsim 10$ pc) accretion onto smaller clumps over a prolonged ($\tau > 2$
Myr) star formation timescale.


Observations are needed to distinguish competing models for MC formation:
Birth from isolated massive proto-cluster clumps, either compact and rapid
or diffuse and slow, or from smaller clumps that
never have a mass as large as the final cluster
mass.  
This sample of the \ncandidates\ most massive proto-cluster clumps in the first
quadrant (where they can be observed by both the VLA and ALMA) presents an ideal
starting point for these observations.

\section{Acknowledgements}
We thank the referee for thorough and very helpful comments that strengthened
this Letter.  This work was supported by NSF grant AST 1009847.


%\bibliography{boundhii}

\begin{table*}
\scriptsize
\begin{center}
\caption{\label{tab:candidates}
Massive Protocluster Candidates detected in the Bolocam Galactic Plane Survey with $M>10^4 \msun$ }
\begin{tabular}{ccccccccccc}
\hline
Name & Common & Distance & M(20K) & M(40K) & $^a$M(min) & Radius & $\bar{n}(H_2)$ & $v_{esc}$ & $^bf_{ff}$ & L(IRAS) \\
 & Name & kpc & $1000 M_{\odot}$ & $1000 M_{\odot}$ & $1000 M_{\odot}$ & pc & $10^4$cm$^{-3}$ & km~s$^{-1}$ &  & $10^5 L_{\odot}$ \\
\hline\hline
G010.472+00.026 & G10.47 & 10.8$^{7}$ & 38 & 16 & 16 & 2.1 & 1.4 & 12.7 & 0.01 & 5.0 \\
G012.209-00.104 & - & 13.5$^{7}$ & 14 & 6 & 5 & 1.3 & 2.3 & 9.9 & 0.05 & 0.61 \\
G012.627-00.016 & - & 12.8$^{9}$ & 10 & 4 & 3 & 2.5 & 0.2 & 5.9 & 0.05 & 0.59 \\
G012.809-00.200 & W33 & 3.6$^{7}$ & 12 & 5 & 3 & 1.0 & 3.8 & 10.2 & 0.32 & 3.0 \\
G019.474+00.171 & - & 14.1$^{12}$ & 11 & 4 & 4 & 1.4 & 1.6 & 8.6 & 0.02 & 0.26 \\
G019.609-00.233 & G19.6 & 12.0$^{7}$ & 26 & 11 & 7 & 2.3 & 0.7 & 10.0 & 0.31 & 6.4 \\
G020.082-00.135 & IR18253 & 12.6$^{10}$ & 13 & 5 & 4 & 2.4 & 0.3 & 6.8 & 0.14 & 2.8 \\
G024.791+00.083 & G24.78 & 7.7$^{11}$ & 14 & 6 & 5 & 2.2 & 0.4 & 7.4 & 0.11 & 1.5 \\
G029.955-00.018 & - & 7.4$^{3}$ & 10 & 4 & 2 & 2.2 & 0.3 & 6.4 & 0.34 & 5.3 \\
G030.704-00.067 & W43b & 5.1$^{6}$ & 11 & 4 & 4 & 1.5 & 1.1 & 8.0 & 0.11 & 1.0 \\
G030.820-00.055 & W43a & 5.1$^{10}$ & 11 & 4 & 4 & 1.5 & 1.2 & 8.1 & 0.13 & 1.9 \\
G031.414+00.307 & G31.41 & 7.9$^{2}$ & 18 & 7 & 7 & 2.3 & 0.5 & 8.3 & 0.05 & 0.8 \\
G032.798+00.193 & G32.80 & 12.9$^{1}$ & 22 & 9 & 7 & 2.5 & 0.5 & 8.9 & 0.27 & 6.9 \\
G034.258+00.154 & G34 & 3.6$^{4}$ & 13 & 5 & 4 & 1.0 & 4 & 10.5 & 0.12 & 2.7 \\
G043.164-00.031 & W49 & 11.4$^{5}$ & 24 & 10 & 6 & 2.2 & 0.7 & 9.7 & 0.38 & 9.9 \\
G043.169+00.009 & W49 & 11.4$^{5}$ & 120 & 52 & 39 & 2.2 & 4 & 22.2 & 0.25 & 16.0 \\
G049.489-00.370 & W51IRS2 & 5.4$^{8}$ & 48 & 20 & 14 & 1.6 & 4.3 & 16.2 & 0.27 & 4.5 \\
G049.489-00.386 & W51MAIN & 5.4$^{8}$ & 52 & 22 & 15 & 1.6 & 4.7 & 17.0 & 0.29 & 4.7 \\
\hline
\end{tabular}
\end{center}{\scriptsize 1: \citet{Araya2002}, 2: \citet{Churchwell1990}, 3: \citet{Fish2003}, 4: \citet{Ginsburg2011}, 5: \citet{Gwinn1992}, 7: \citet{Pandian2008}, 8: \citet{Sato2010}, 9: \citet{Sewilo2004}, 10: \citet{Urquhart2012}, 11: \citet{Vig2008}, 12: \citet{Xu2003}.  6: The distances to G030.704 was determined using the
near kinematic distance from the velocity of the HHT-observed HCO+ line \citep{Schlingman2011}.
$^a$: The minimum likely mass, $M_{min} = (1-f_{ff}) M(40K)$.
$^b$: The fraction of flux from free-free emission (as opposed to dust emission) at $\lambda=1.1$ mm
}
\end{table*}



\ifstandalone
\bibliographystyle{apj_w_etal}  % or "siam", or "alpha", or "abbrv"
%\bibliography{thesis}      % bib database file refs.bib
\bibliography{bibdesk}      % bib database file refs.bib
\fi

\end{document}

% 
% %\input ch2.tex			% file with Chapter 2 contents
% 
% %%%%%%%%%%%%%%%%%%%%%%%%%%%%%%%%%%%%%%%%%%%%%%%%%%%%%%%%%%%%%%%%%%%
% %%%%%%%%%%%%%%%%%%%%%%%  Bibliography %%%%%%%%%%%%%%%%%%%%%%%%%%%%%
% %%%%%%%%%%%%%%%%%%%%%%%%%%%%%%%%%%%%%%%%%%%%%%%%%%%%%%%%%%%%%%%%%%%
% 
% \bibliographystyle{plain}	% or "siam", or "alpha", or "abbrv"
% 				% see other styles (.bst files) in
% 				% $TEXHOME/texmf/bibtex/bst
% 
% \nocite{*}		% list all refs in database, cited or not.
% 
% \bibliography{thesis}		% bib database file refs.bib
% 
% %%%%%%%%%%%%%%%%%%%%%%%%%%%%%%%%%%%%%%%%%%%%%%%%%%%%%%%%%%%%%%%%%%%
% %%%%%%%%%%%%%%%%%%%%%%%%  Appendices %%%%%%%%%%%%%%%%%%%%%%%%%%%%%%
% %%%%%%%%%%%%%%%%%%%%%%%%%%%%%%%%%%%%%%%%%%%%%%%%%%%%%%%%%%%%%%%%%%%
% 
% \appendix	% don't forget this line if you have appendices!
% 
% %\input appA.tex			% file with Appendix A contents
% %\input appB.tex			% file with Appendix B contents
% 
% %%%%%%%%%%%%%%%%%%%%%%%%%%%%%%%%%%%%%%%%%%%%%%%%%%%%%%%%%%%%%%%%%%%
% %%%%%%%%%%%%%%%%%%%%%%%%   THE END   %%%%%%%%%%%%%%%%%%%%%%%%%%%%%%
% %%%%%%%%%%%%%%%%%%%%%%%%%%%%%%%%%%%%%%%%%%%%%%%%%%%%%%%%%%%%%%%%%%%
% 
% \end{document}
% 
% 

%%\documentclass[defaultstyle,11pt]{thesis}
%\documentclass[]{report}
%\documentclass[]{article}
%\usepackage{aastex_hack}
%\usepackage{deluxetable}
\documentclass[preprint]{aastex}


%%%%%%%%%%%%%%%%%%%%%%%%%%%%%%%%%%%%%%%%%%%%%%%%%%%%%%%%%%%%%%%%
%%%%%%%%%%%  see documentation for information about  %%%%%%%%%%
%%%%%%%%%%%  the options (11pt, defaultstyle, etc.)   %%%%%%%%%%
%%%%%%%  http://www.colorado.edu/its/docs/latex/thesis/  %%%%%%%
%%%%%%%%%%%%%%%%%%%%%%%%%%%%%%%%%%%%%%%%%%%%%%%%%%%%%%%%%%%%%%%%
%		\documentclass[typewriterstyle]{thesis}
% 		\documentclass[modernstyle]{thesis}
% 		\documentclass[modernstyle,11pt]{thesis}
%	 	\documentclass[modernstyle,12pt]{thesis}

%%%%%%%%%%%%%%%%%%%%%%%%%%%%%%%%%%%%%%%%%%%%%%%%%%%%%%%%%%%%%%%%
%%%%%%%%%%%    load any packages which are needed    %%%%%%%%%%%
%%%%%%%%%%%%%%%%%%%%%%%%%%%%%%%%%%%%%%%%%%%%%%%%%%%%%%%%%%%%%%%%
\usepackage{latexsym}		% to get LASY symbols
\usepackage{graphicx}		% to insert PostScript figures
%\usepackage{deluxetable}
\usepackage{rotating}		% for sideways tables/figures
\usepackage{natbib}  % Requires natbib.sty, available from http://ads.harvard.edu/pubs/bibtex/astronat/
\usepackage{savesym}
\usepackage{amssymb}
%\savesymbol{singlespace}
\savesymbol{doublespace}
%\usepackage{wrapfig}
%\usepackage{setspace}
\usepackage{xspace}
\usepackage{color}
\usepackage{multicol}
\usepackage{mdframed}
\usepackage{url}
\usepackage{subfigure}
%\usepackage{emulateapj}
\usepackage{lscape}
\usepackage{grffile}
\usepackage{standalone}
\standalonetrue
\usepackage{import}
\usepackage[utf8]{inputenc}
\usepackage{longtable}
\usepackage{booktabs}



%%%%%%%%%%%%%%%%%%%%%%%%%%%%%%%%%%%%%%%%%%%%%%%%%%%%%%%%%%%%%%%%
%%%%%%%%%%%%       all the preamble material:       %%%%%%%%%%%%
%%%%%%%%%%%%%%%%%%%%%%%%%%%%%%%%%%%%%%%%%%%%%%%%%%%%%%%%%%%%%%%%

% \title{Star Formation in the Galaxy}
% 
% \author{Adam G.}{Ginsburg}
% 
% \otherdegrees{B.S., Rice University, 2007\\
% 	      M.S., University of Colorado, Boulder, 2009}
% 
% \degree{Doctor of Philosophy}		%  #1 {long descr.}
% 	{Ph.D., Rocket Science (ok, fine, astrophysics)}		%  #2 {short descr.}
% 
% \dept{Department of}			%  #1 {designation}
% 	{Astrophysical and Planetary Sciences}		%  #2 {name}
% 
% \advisor{Prof.}				%  #1 {title}
% 	{John Bally}			%  #2 {name}
% 
% \reader{Prof.~Jeremy Darling}		%  2nd person to sign thesis
% \readerThree{Prof.~Jason Glenn}		%  3rd person to sign thesis
% \readerFour{Prof.~Michael Shull}	%  4rd person to sign thesis
% \readerFour{Prof.~Neal Evans}	%  4rd person to sign thesis
% 
% \abstract{  \OnePageChapter	% one page only ??
% 
%     I discovered dust in space.  
% 
% 	}
% 
% 
% \dedication[Dedication]{	% NEVER use \OnePageChapter here.
% 	To 1, the second number in binary.
% 	}
% 
% \acknowledgements{	\OnePageChapter	% *MUST* BE ONLY ONE PAGE!
% 	All y'all.
% 	}
% 
% \ToCisShort	% a 1-page Table of Contents ??
% 
% \LoFisShort	% a 1-page List of Figures ??
% %	\emptyLoF	% no List of Figures at all ??
% 
% \LoTisShort	% a 1-page List of Tables ??
% %	\emptyLoT	% no List of Tables at all ??
% 
% 
% %%%%%%%%%%%%%%%%%%%%%%%%%%%%%%%%%%%%%%%%%%%%%%%%%%%%%%%%%%%%%%%%%
% %%%%%%%%%%%%%%%       BEGIN DOCUMENT...         %%%%%%%%%%%%%%%%%
% %%%%%%%%%%%%%%%%%%%%%%%%%%%%%%%%%%%%%%%%%%%%%%%%%%%%%%%%%%%%%%%%%
% 
% %%%%  footnote style; default=\arabic  (numbered 1,2,3...)
% %%%%  others:  \roman, \Roman, \alph, \Alph, \fnsymbol
% %	"\fnsymbol" uses asterisk, dagger, double-dagger, etc.
% %	\renewcommand{\thefootnote}{\fnsymbol{footnote}}
% %	\setcounter{footnote}{0}

\newcommand{\paa}{Pa\ensuremath{\alpha}}
\newcommand{\brg}{Br\ensuremath{\gamma}}
\newcommand{\msun}{\ensuremath{M_{\odot}}}			%  Msun
\newcommand{\mdot}{\ensuremath{\dot{M}}\xspace}
\newcommand{\lsun}{\ensuremath{L_{\odot}}}			%  Lsun
\newcommand{\lbol}{\ensuremath{L_{\mathrm{bol}}}}	%  Lbol
\newcommand{\ks}{K\ensuremath{_{\mathrm{s}}}}		%  Ks
\newcommand{\hh}{\ensuremath{\textrm{H}_{2}}\xspace}			%  H2
\newcommand{\formaldehyde}{\ensuremath{\textrm{H}_2\textrm{CO}}\xspace}
\newcommand{\formaldehydeIso}{\ensuremath{\textrm{H}_2~^{13}\textrm{CO}}\xspace}
\newcommand{\methanol}{\ensuremath{\textrm{CH}_3\textrm{OH}}\xspace}
\newcommand{\ortho}{\ensuremath{\textrm{o-H}_2\textrm{CO}}}
\newcommand{\oneone}{\ensuremath{1_{10}-1_{11}}\xspace}
\newcommand{\twotwo}{\ensuremath{2_{11}-2_{12}}\xspace}
\newcommand{\threethree}{\ensuremath{3_{12}-3_{13}}\xspace}
\newcommand{\threeohthree}{\ensuremath{3_{03}-2_{02}}\xspace}
\newcommand{\threetwotwo}{\ensuremath{3_{22}-2_{21}}\xspace}
\newcommand{\threetwoone}{\ensuremath{3_{21}-2_{20}}\xspace}
\newcommand{\JKaKc}{\ensuremath{J_{K_a K_c}}}
\newcommand{\water}{H$_{2}$O}		%  H2O
\newcommand{\feii}{\ion{Fe}{2}}		%  FeII
\newcommand{\uchii}{UC\ion{H}{2}\xspace}
\newcommand{\UCHII}{UC\ion{H}{2}\xspace}
\newcommand{\hii}{H~{\sc ii}\xspace}
\newcommand{\Hii}{H~{\sc ii}\xspace}
\newcommand{\HII}{H~{\sc ii}\xspace}
\newcommand{\kms}{\textrm{km~s}\ensuremath{^{-1}}\xspace}	%  km s-1
\newcommand{\nsample}{456\xspace}
\newcommand{\CFR}{5\xspace} % nMPC / 0.25 / 2 (6 for W51 once, 8 for W51 twice) REFEDIT: With f_observed=0.3, becomes 3/2./0.3 = 5
\newcommand{\permyr}{\ensuremath{\mathrm{Myr}^{-1}}\xspace}
\newcommand{\tsuplim}{0.5\xspace} % upper limit on starless timescale
\newcommand{\ncandidates}{18\xspace}
\newcommand{\mindist}{8.7\xspace}
\newcommand{\rcluster}{2.5\xspace}
\newcommand{\ncomplete}{13\xspace}
\newcommand{\middistcut}{13.0\xspace}
\newcommand{\nMPC}{3\xspace} % only count W51 once.  W51, W49, G010
\newcommand{\obsfrac}{30}
\newcommand{\nMPCtot}{10\xspace} % = nmpc / obsfrac
\newcommand{\nMPCtoterr}{6\xspace} % = sqrt(nmpc) / obsfrac
\newcommand{\plaw}{2.1\xspace}
\newcommand{\plawerr}{0.3\xspace}
\newcommand{\mmin}{\ensuremath{10^4~\msun}\xspace}
%\newcommand{\perkmspc}{\textrm{per~km~s}\ensuremath{^{-1}}\textrm{pc}\ensuremath{^{-1}}\xspace}	%  km s-1 pc-1
\newcommand{\kmspc}{\textrm{km~s}\ensuremath{^{-1}}\textrm{pc}\ensuremath{^{-1}}\xspace}	%  km s-1 pc-1
\newcommand{\sqcm}{cm$^{2}$\xspace}		%  cm^2
\newcommand{\percc}{\ensuremath{\textrm{cm}^{-3}}\xspace}
\newcommand{\persc}{\ensuremath{\textrm{cm}^{-2}}\xspace}
\newcommand{\persr}{\ensuremath{\textrm{sr}^{-1}}\xspace}
\newcommand{\peryr}{\ensuremath{\textrm{yr}^{-1}}\xspace}
\newcommand{\perkmspc}{\textrm{per~km~s}\ensuremath{^{-1}}\textrm{pc}\ensuremath{^{-1}}\xspace}	%  km s-1 pc-1
\newcommand{\perkms}{\textrm{per~km~s}\ensuremath{^{-1}}\xspace}	%  km s-1 
\newcommand{\um}{\ensuremath{\mu m}\xspace}    % micron
\newcommand{\mum}{$\mu$m}
\newcommand{\htwo}{\ensuremath{\textrm{H}_2}}    % micron
\newcommand{\Htwo}{\ensuremath{\textrm{H}_2}}    % micron
\newcommand{\HtwoO}{\ensuremath{\textrm{H}_2\textrm{O}}}    % micron
\newcommand{\htwoo}{\ensuremath{\textrm{H}_2\textrm{O}}}    % micron
\newcommand{\ha}{\ensuremath{\textrm{H}\alpha}}
\newcommand{\hb}{\ensuremath{\textrm{H}\beta}}
%\newcommand{\so}{ SO~(5~6)-(4~5) }
\newcommand{\regone}{Sh~2-201}
\newcommand{\regtwo}{AFGL~4029}
\newcommand{\regthree}{LW Cas Nebula}
\newcommand{\regfour}{IC 1848}
\newcommand{\regfive}{W5 NW}
\newcommand{\regsix}{SFO 11}
\newcommand{\so}{ SO~\ensuremath{5_6-4_5} }
\newcommand{\SO}{ SO~\ensuremath{1_2-1_1} }
\newcommand{\ammonia}{NH\ensuremath{_3}\xspace}
\newcommand{\twelveco}{\ensuremath{^{12}\textrm{CO}}}
\newcommand{\thirteenco}{\ensuremath{^{13}\textrm{CO}}}
\newcommand{\ceighteeno}{\ensuremath{\textrm{C}^{18}\textrm{O}}}
\def\ee#1{\ensuremath{\times10^{#1}}}
\newcommand{\degrees}{\ensuremath{^{\circ}}}
\newcommand{\lowirac}{800}
\newcommand{\highirac}{8000}
\newcommand{\lowmips}{600}
\newcommand{\highmips}{5000}
\newcommand{\perbeam}{\ensuremath{\textrm{beam}^{-1}}}
\newcommand{\ds}{\ensuremath{\textrm{d}s}}
\newcommand{\dnu}{\ensuremath{\textrm{d}\nu}}
\newcommand{\dv}{\ensuremath{\textrm{d}v}}
\def\secref#1{Section \ref{#1}}
\def\eqref#1{Equation \ref{#1}}
%\newcommand{\arcmin}{'}

\newcommand{\necluster}{Sh~2-233IR~NE}
\newcommand{\swcluster}{Sh~2-233IR~SW}
\newcommand{\region}{IRAS 05358}

\newcommand{\nwfive}{40}
\newcommand{\nouter}{15}

\newcommand{\vone}{{\rm v}1.0\xspace}
\newcommand{\vtwo}{{\rm v}2.0\xspace}
\newcommand\mjysr{\ensuremath{{\rm MJy~sr}^{-1}}}
\newcommand\jybm{\ensuremath{{\rm Jy~bm}^{-1}}}
\newcommand\nbolocat{8552\xspace}
\newcommand\nbolocatnew{548\xspace}
\newcommand\nbolocatnonew{8004\xspace} % = nbolocat-nbolocatnew
\renewcommand\arcdeg{\mbox{$^\circ$}\xspace} 
\renewcommand\arcmin{\mbox{$^\prime$}\xspace} 
\renewcommand\arcsec{\mbox{$^{\prime\prime}$}\xspace} 

\newcommand{\todo}[1]{\textcolor{red}{#1}}
\newcommand{\okinfinal}[1]{{#1}}
\newcommand{\keywords}[1]{}
\newcommand{\email}[1]{}
\newcommand{\affil}[1]{}


%aastex hack
%\newcommand\arcdeg{\mbox{$^\circ$}}%
%\newcommand\arcmin{\mbox{$^\prime$}\xspace}%
%\newcommand\arcsec{\mbox{$^{\prime\prime}$}\xspace}%

%\newcommand\epsscale[1]{\gdef\eps@scaling{#1}}
%
%\newcommand\plotone[1]{%
% \typeout{Plotone included the file #1}
% \centering
% \leavevmode
% \includegraphics[width={\eps@scaling\columnwidth}]{#1}%
%}%
%\newcommand\plottwo[2]{{%
% \typeout{Plottwo included the files #1 #2}
% \centering
% \leavevmode
% \columnwidth=.45\columnwidth
% \includegraphics[width={\eps@scaling\columnwidth}]{#1}%
% \hfil
% \includegraphics[width={\eps@scaling\columnwidth}]{#2}%
%}}%


%\newcommand\farcm{\mbox{$.\mkern-4mu^\prime$}}%
%\let\farcm\farcm
%\newcommand\farcs{\mbox{$.\!\!^{\prime\prime}$}}%
%\let\farcs\farcs
%\newcommand\fp{\mbox{$.\!\!^{\scriptscriptstyle\mathrm p}$}}%
%\newcommand\micron{\mbox{$\mu$m}}%
%\def\farcm{%
% \mbox{.\kern -0.7ex\raisebox{.9ex}{\scriptsize$\prime$}}%
%}%
%\def\farcs{%
% \mbox{%
%  \kern  0.13ex.%
%  \kern -0.95ex\raisebox{.9ex}{\scriptsize$\prime\prime$}%
%  \kern -0.1ex%
% }%
%}%

\def\Figure#1#2#3#4#5{
\begin{figure*}[htp]
\includegraphics[scale=#4,angle=#5]{#1}
\caption{#2}
\label{#3}
\end{figure*}
}

% originally intended to be included in a two-column paper
% this is in includegraphics: ,width=3in
% but, not for thesis
\def\OneColFigure#1#2#3#4#5{
\begin{figure}[htpb]
\epsscale{#4}
\includegraphics[scale=#4,angle=#5]{#1}
\caption{#2}
\label{#3}
\end{figure}
}

\def\SubFigure#1#2#3#4#5{
\begin{figure*}[htp]
\addtocounter{figure}{-1}
\epsscale{#4}
\includegraphics[angle=#5]{#1}
\caption{#2}
\label{#3}
\end{figure*}
}

\def\FigureTwo#1#2#3#4#5{
\begin{figure*}[htp]
\epsscale{#5}
\plottwo{#1}{#2}
\caption{#3}
\label{#4}
\end{figure*}
}

\def\TallFigureTwo#1#2#3#4#5#6{
    \FigureTwo{#1}{#2}{#3}{#4}{#5}
    }

\def\SubFigureTwo#1#2#3#4#5{
\begin{figure*}[htp]
\addtocounter{figure}{-1}
\epsscale{#5}
\plottwo{#1}{#2}
\caption{#3}
\label{#4}
\end{figure*}
}

\def\FigureFour#1#2#3#4#5#6{
\begin{figure*}[htp]
\subfigure[]{ \includegraphics[width=3in,type=png,ext=.png,read=.png]{#1} }
\subfigure[]{ \includegraphics[width=3in,type=png,ext=.png,read=.png]{#2} }
\subfigure[]{ \includegraphics[width=3in,type=png,ext=.png,read=.png]{#3} }
\subfigure[]{ \includegraphics[width=3in,type=png,ext=.png,read=.png]{#4} }
\caption{#5}
\label{#6}
\end{figure*}
}

\def\Table#1#2#3#4#5#6{
\begin{deluxetable}{#1}
\tablewidth{0pt}
\tabletypesize{\footnotesize}
\tablecaption{#2}
\tablehead{#3}
\startdata
\label{#4}
#5
\enddata
\bigskip
#6
\end{deluxetable}
}

		% file containing author's macro definitions

\begin{document}
% %\documentclass[defaultstyle,11pt]{thesis}
%\documentclass[]{report}
%\documentclass[]{article}
%\usepackage{aastex_hack}
%\usepackage{deluxetable}
\documentclass[preprint]{aastex}


%%%%%%%%%%%%%%%%%%%%%%%%%%%%%%%%%%%%%%%%%%%%%%%%%%%%%%%%%%%%%%%%
%%%%%%%%%%%  see documentation for information about  %%%%%%%%%%
%%%%%%%%%%%  the options (11pt, defaultstyle, etc.)   %%%%%%%%%%
%%%%%%%  http://www.colorado.edu/its/docs/latex/thesis/  %%%%%%%
%%%%%%%%%%%%%%%%%%%%%%%%%%%%%%%%%%%%%%%%%%%%%%%%%%%%%%%%%%%%%%%%
%		\documentclass[typewriterstyle]{thesis}
% 		\documentclass[modernstyle]{thesis}
% 		\documentclass[modernstyle,11pt]{thesis}
%	 	\documentclass[modernstyle,12pt]{thesis}

%%%%%%%%%%%%%%%%%%%%%%%%%%%%%%%%%%%%%%%%%%%%%%%%%%%%%%%%%%%%%%%%
%%%%%%%%%%%    load any packages which are needed    %%%%%%%%%%%
%%%%%%%%%%%%%%%%%%%%%%%%%%%%%%%%%%%%%%%%%%%%%%%%%%%%%%%%%%%%%%%%
\usepackage{latexsym}		% to get LASY symbols
\usepackage{graphicx}		% to insert PostScript figures
%\usepackage{deluxetable}
\usepackage{rotating}		% for sideways tables/figures
\usepackage{natbib}  % Requires natbib.sty, available from http://ads.harvard.edu/pubs/bibtex/astronat/
\usepackage{savesym}
\usepackage{amssymb}
%\savesymbol{singlespace}
\savesymbol{doublespace}
%\usepackage{wrapfig}
%\usepackage{setspace}
\usepackage{xspace}
\usepackage{color}
\usepackage{multicol}
\usepackage{mdframed}
\usepackage{url}
\usepackage{subfigure}
%\usepackage{emulateapj}
\usepackage{lscape}
\usepackage{grffile}
\usepackage{standalone}
\standalonetrue
\usepackage{import}
\usepackage[utf8]{inputenc}
\usepackage{longtable}
\usepackage{booktabs}



%%%%%%%%%%%%%%%%%%%%%%%%%%%%%%%%%%%%%%%%%%%%%%%%%%%%%%%%%%%%%%%%
%%%%%%%%%%%%       all the preamble material:       %%%%%%%%%%%%
%%%%%%%%%%%%%%%%%%%%%%%%%%%%%%%%%%%%%%%%%%%%%%%%%%%%%%%%%%%%%%%%

% \title{Star Formation in the Galaxy}
% 
% \author{Adam G.}{Ginsburg}
% 
% \otherdegrees{B.S., Rice University, 2007\\
% 	      M.S., University of Colorado, Boulder, 2009}
% 
% \degree{Doctor of Philosophy}		%  #1 {long descr.}
% 	{Ph.D., Rocket Science (ok, fine, astrophysics)}		%  #2 {short descr.}
% 
% \dept{Department of}			%  #1 {designation}
% 	{Astrophysical and Planetary Sciences}		%  #2 {name}
% 
% \advisor{Prof.}				%  #1 {title}
% 	{John Bally}			%  #2 {name}
% 
% \reader{Prof.~Jeremy Darling}		%  2nd person to sign thesis
% \readerThree{Prof.~Jason Glenn}		%  3rd person to sign thesis
% \readerFour{Prof.~Michael Shull}	%  4rd person to sign thesis
% \readerFour{Prof.~Neal Evans}	%  4rd person to sign thesis
% 
% \abstract{  \OnePageChapter	% one page only ??
% 
%     I discovered dust in space.  
% 
% 	}
% 
% 
% \dedication[Dedication]{	% NEVER use \OnePageChapter here.
% 	To 1, the second number in binary.
% 	}
% 
% \acknowledgements{	\OnePageChapter	% *MUST* BE ONLY ONE PAGE!
% 	All y'all.
% 	}
% 
% \ToCisShort	% a 1-page Table of Contents ??
% 
% \LoFisShort	% a 1-page List of Figures ??
% %	\emptyLoF	% no List of Figures at all ??
% 
% \LoTisShort	% a 1-page List of Tables ??
% %	\emptyLoT	% no List of Tables at all ??
% 
% 
% %%%%%%%%%%%%%%%%%%%%%%%%%%%%%%%%%%%%%%%%%%%%%%%%%%%%%%%%%%%%%%%%%
% %%%%%%%%%%%%%%%       BEGIN DOCUMENT...         %%%%%%%%%%%%%%%%%
% %%%%%%%%%%%%%%%%%%%%%%%%%%%%%%%%%%%%%%%%%%%%%%%%%%%%%%%%%%%%%%%%%
% 
% %%%%  footnote style; default=\arabic  (numbered 1,2,3...)
% %%%%  others:  \roman, \Roman, \alph, \Alph, \fnsymbol
% %	"\fnsymbol" uses asterisk, dagger, double-dagger, etc.
% %	\renewcommand{\thefootnote}{\fnsymbol{footnote}}
% %	\setcounter{footnote}{0}

\input{macros}		% file containing author's macro definitions

\begin{document}
% \input{introduction}
% 
% %\input{ch_iras05358}
% \input{ch_w5}
% \input{ch_h2co}
% \input{ch_h2colarge}
% \input{ch_boundhii}
% 
% %\input ch2.tex			% file with Chapter 2 contents
% 
% %%%%%%%%%%%%%%%%%%%%%%%%%%%%%%%%%%%%%%%%%%%%%%%%%%%%%%%%%%%%%%%%%%%
% %%%%%%%%%%%%%%%%%%%%%%%  Bibliography %%%%%%%%%%%%%%%%%%%%%%%%%%%%%
% %%%%%%%%%%%%%%%%%%%%%%%%%%%%%%%%%%%%%%%%%%%%%%%%%%%%%%%%%%%%%%%%%%%
% 
% \bibliographystyle{plain}	% or "siam", or "alpha", or "abbrv"
% 				% see other styles (.bst files) in
% 				% $TEXHOME/texmf/bibtex/bst
% 
% \nocite{*}		% list all refs in database, cited or not.
% 
% \bibliography{thesis}		% bib database file refs.bib
% 
% %%%%%%%%%%%%%%%%%%%%%%%%%%%%%%%%%%%%%%%%%%%%%%%%%%%%%%%%%%%%%%%%%%%
% %%%%%%%%%%%%%%%%%%%%%%%%  Appendices %%%%%%%%%%%%%%%%%%%%%%%%%%%%%%
% %%%%%%%%%%%%%%%%%%%%%%%%%%%%%%%%%%%%%%%%%%%%%%%%%%%%%%%%%%%%%%%%%%%
% 
% \appendix	% don't forget this line if you have appendices!
% 
% %\input appA.tex			% file with Appendix A contents
% %\input appB.tex			% file with Appendix B contents
% 
% %%%%%%%%%%%%%%%%%%%%%%%%%%%%%%%%%%%%%%%%%%%%%%%%%%%%%%%%%%%%%%%%%%%
% %%%%%%%%%%%%%%%%%%%%%%%%   THE END   %%%%%%%%%%%%%%%%%%%%%%%%%%%%%%
% %%%%%%%%%%%%%%%%%%%%%%%%%%%%%%%%%%%%%%%%%%%%%%%%%%%%%%%%%%%%%%%%%%%
% 
% \end{document}
% 
% 

\chapter{Introduction}
\section{Preface}
This thesis\footnote{Necessarily the first two words of a thesis?} describes
the research I have performed with a wide variety of collaborators, mostly
centered on the Bolocam Galactic Plane Survey team led by John Bally and Jason
Glenn.  The BGPS data reduction process, at the core of this work, was done in 
collaboration with James Aguirre and Erik Rosolowsky.

However, the work proceeded somewhat haphazardly: I came into the BGPS team as
the rare student enthusiastic about data reduction.  I never planned to take
over the BGPS data, but it happened a few years into my time at CU.  This
thesis is therefore somewhat scattered: some of the observations reported here
were taken as `follow-up' to the BGPS before it was completed.

This document primarily consists of a number of published papers centered
around a common theme of radio and millimeter observations of the Galaxy, but
without an obvious common driving question.  I have therefore added
thesis-specific introductions to each section to describe where they fit in to
the bigger picture of this document.  I've also included sections describing
work that is not yet published but (hopefully) soon will be.

\section{Star Formation in the Galaxy}
It has been known for at least half a century that stars form from the
gravitational collapse of clouds of cool material.  The gas that will
eventually form stars is typically observed as dark features obscuring
background stars.  The brighter nebulae, which have been studied for far longer
\citep{Messier1764}, contain hot and diffuse gas.  These nebulae, while
spectacular, are not the construction materials of new stars.  However,
they mark the locations where new stars have formed - nebulae are often
stellar nurseries.

To track down the cool star-forming material, it is necessary to observe at
longer wavelengths.  Infrared observations can pierce through obscuring
material, as dust becomes more transparent at longer wavelengths.  With near-
and mid-infrared observations, such as those enabled by HgCdTe detectors like
those in the NICFPS and TripleSpec instruments at Apache Point Observatory and
the InSb detectors used on the Spitzer Space Telescope, it is possible to
observe obscured young stars.  These objects have just ignited fusion in their
cores and represent the youngest generation of new stars.

But this material has already formed stars.  To see the truly cold stuff, that
which still has potential to form new stars, we need to examine gas that is not
heated at all by stars.  Assuming we want to look for gas that can form a star
like our sun and that the density of the gas to form is $\sim10^4$ particles
per cubic centimeter (an assumption left unjustified for now), the Jeans scale
requires a temperature $T\sim10$ K, which means we need to look at wavelengths
$\lambda \gtrsim 100 \um$ in order to observe this gas.

Gas at these densities turns out to be quite rare.  While there are thousands
of stars within 100 pc of the sun, the closest known star-forming globules are
at distances greater than 100 pc.  While this sparsity is explained in part by
our current position in the Galaxy (we're buzzing along its outskirts at 250
\kms), it reflects the reality that star formation in the present epoch is
dispersed and rare.

Even more rare are the massive stars that end their lives in supernovae.  While there
are hundreds of stellar nurseries within a few hundred parsecs, the nearest
region of massive star formation is the Orion Molecular Cloud at a distance of
400 pc.  Out to 1000 pc, though, there are still only a handful of massive star
forming regions, including Monoceros R2 and Cepheus A.

These massive stars in many ways are the most important to study in order to
understand the evolution of gas and dust in the universe and our own origins.
In their deaths, they produce the heavy elements required to form dust,
planets, and life.  Throughout their lives and deaths, massive stars dump
energy into the interstellar medium and effectively control the motions and
future of the gas around them.

The bigger the star, the shorter it lives, so massive stars are nearly as rare
as their birth regions.  They also tend to be found nearby or within these
birth regions.  Since they can be found near large globs of dust, finding these
globs can help us discover new groups of massive stars.

This thesis summarizes surveys within our Galaxy to discover and examine
regions forming.  The largest body of work described here is the Bolocam
Galactic Plane Survey, the first dust continuum survey of a significant
fraction of the Galactic Plane.

With that broad overview in place, the next sections describe a few of the
specific problems addressed in this thesis in greater detail.

\subsection{Turbulence}
Turbulence is one of the defining features of the interstellar medium.
Turbulence is thought to govern many properties of the ISM, rendering it
scale-free and defining the distribution of velocities, densities,
temperatures, and magnetic fields in the gas between stars.

Turbulence is most easily modeled by a Kolmogorov spectrum, in which $\Delta v
\propto \ell^{1/3}$, i.e. the typical velocity dispersion is greatest at the
largest size scales.  Kolmogorov turbulence strictly only describes
incompressible fluids without magnetic fields, while the ISM is compressible
and threaded by magnetic fields.  Nonetheless, Kolmogorov turbulence is nearly
consistent with some observed properties of the ISM.  The Larson size-linewidth
relation, in particular, is similar to that predicted by Kolmogorov turbulence.

Turbulence is often quoted as a source of \emph{pressure} based on the
Kolmogorov description.  At size scales much smaller than the driving scale of
the turbulence (the ``box size'' in a simulation), turbulence becomes isotropic
and can add support against gravitational collapse.  

However, turbulence decays rapidly.  The turbulent decay timescale
$\tau_{decay}\propto L / v$, where $L$ is the turbulent length scale and $v$ is
the velocity scale.  It therefore increases with size scale as
$\tau_{decay}\propto L^{2/3}$.  Turbulence decays most quickly on the smallest
timescales.

We are therefore left with two conditions: Turbulence must be driven at large
scales for turbulence to provide support against gravity, and it must be
constantly driven to resupply the turbulence that is transferred to heat on the
smallest scales.

Because the ISM is compressible, interacting flows within the turbulent medium
will result in density enhancements and voids.  Many simulation studies have
determined that the resulting density distribution, and correspondingly the
column-density distribution, should be approximately log-normal.  Observational
studies agree that in regions not yet significantly affected by gravity, the 
column-density distribution is log-normal.  In regions where stars are actively
forming, a high-density power-law tail forms.

One theory of star formation holds that the initial mass function of stars is
determined entirely by turbulence.  In this description, the highest
overdensities in the turbulent medium become gravitationally unstable and
separate from the turbulent flow as they collapse into proto-stellar cores.

\subsection{Mass Functions}
Perhaps the most fundamental goal of star formation studies is to determine the
Initial Mass Function (IMF) of stars and what, if anything, causes it to vary.
It is also one of the most challenging statistically and observationally.

The IMF defines the probability distribution function of stellar masses at
birth, and therefore differs greatly from the present-day stellar mass function
that is very strongly affected by stellar death at the highest masses.  In
order to determine the mass function for the most massive stars, it is
necessary to look at their birth places.  These birth places are dusty, dense,
and rare.  

It remains somewhat unclear whether the IMF is a universal function or is sampled
independently for individual clusters.  If it is universal, there is a possibility of
forming massive stars anywhere stars form.  If cluster-dependent, then a massive star
must form with a surrounding cluster.

Some groups now claim that the initial mass function is decided in the gas
phase by the formation of cores.  The Core Mass Function measures the
probability distribution function of core masses, where cores are generally
identified observationally as column-density peaks in millimeter/submillimeter
emission maps.  The CMF has a similar functional form to the IMF, but its mean
is higher by a factor $\sim3$ in local star forming regions, leading to the claim
that star formation proceeds from CMF->IMF with 30\% efficiency.

Gas clouds follow a mass function that extends up to the largest
possible coherent scales, giant molecular clouds with scales $\sim50-100$ pc
that are limited by the scale-height of the ISM in Galactic disks.
Between `cores' and GMCs, intermediate scale blobs are often called `clumps'. 
The mass function of these clumps has yet to be determined.  

The mass function of GMCs was determine from CO emission towards the Galactic
plane and in nearby galaxies (e.g., M33) where they can be resolved.  The CMF
was measured in nearby clouds where 30\arcsec\ beams easily resolve $\sim0.1$
pc cores.  However, clumps are only found in large numbers in the Galactic
plane, where distances are uncertain.  They cannot be resolved in other
galaxies (except by ALMA now).

To understand star formation on a galactic scale, it is necessary to understand
the transition from large-scale giant molecular clouds and proto-stellar cores.
Clouds follow a shallow mass function, with the largest clouds containing most
of the gas.  Cores and stars are both drawn from steep mass functions in which
most of the mass near some peak in the distribution.  Presumably there must be
some intermediate state of the gas that is drawn from an intermediate
distribution, shallower than `cores' but steeper than `clouds'.  

\subsubsection{Clusters}
Clusters are also drawn from a mass function comparable to stars, but
ironically their distribution is better measured than for stars.  Clusters are
easily visible - and resolvable - in other galaxies, and massive clusters are
less likely to be embedded than massive stars.  In normal galaxies, cluster
populations are consistent with a Schechter distribution: a power-law
$\alpha\sim2$ with an exponential cutoff at large masses.

Since clusters are not drawn from the same parent distribution as GMCs (which
have $\alpha\sim1$), it is plausible that their precursors are, instead, the
intermediate-scale `clumps' observed in the millimeter continuum.  However, the
clump mass function has yet to be measured, so even this first step of
determining plausibility is incomplete.

Clusters are an important observational tool in astrophysics.  For stellar
studies, they have been used to select populations of co-eval stars.  In
extragalactic studies, they are frequently the smallest observable individual
units.  However, many recent works have pointed out that clusters may be
short-lived, transient phenomena.  Any study of their populations must take
in to account their dissolution.  The most massive clusters, however, are both
the most easily observed and the longest lived, and therefore provide some of the
most useful tools for understanding stars.

As with massive stars, massive clusters are rare.  Only a handful of young
massive clusters are known within our Galaxy, including the most massive,  NGC
3603, the Arches cluster, and Westerlund 1 \citep{PortegiesZwart2010}.  These
are the only locations in the galaxy known to be forming multiple stars near
the (possible) upper stellar mass limit.  Despite their importance, though,
only a handful of these clusters are known and the population of such clusters
is effectively unconstrained.  The incomplete knowledge of clusters is due to
extinction  and confusion within the plane.

%\subsection{Galactic Plane Surveys}
%The idea to observe the plane of the Galaxy is not new.

\section{Outline}
This thesis includes 5 chapters.
Chapter 2 describes observations of the W5 star-forming region to identify outflows;
this chapter is somewhat tangential to the rest.
Chapter 3 describes the BGPS data reduction process and data pipeline.
Chapter 4 is the pilot study of \formaldehyde towards previously-known UCHII regions.
It includes the methodology and analysis of turbulent properties of Galactic GMCs.
Chapter 5 expands upon Chapter 4, detailing the expansion of the \formaldehyde survey
to BGPS-selected sources.
Chapter 6 is a Letter identifying massive proto-clusters in the BGPS.
Chapter 7 concludes.

\ifstandalone
\bibliographystyle{apj_w_etal}  % or "siam", or "alpha", or "abbrv"
%\bibliography{thesis}      % bib database file refs.bib
\bibliography{bibdesk}      % bib database file refs.bib
\fi

\end{document}

% 
% %%\documentclass[defaultstyle,11pt]{thesis}
%\documentclass[]{report}
%\documentclass[]{article}
%\usepackage{aastex_hack}
%\usepackage{deluxetable}
\documentclass[preprint]{aastex}


%%%%%%%%%%%%%%%%%%%%%%%%%%%%%%%%%%%%%%%%%%%%%%%%%%%%%%%%%%%%%%%%
%%%%%%%%%%%  see documentation for information about  %%%%%%%%%%
%%%%%%%%%%%  the options (11pt, defaultstyle, etc.)   %%%%%%%%%%
%%%%%%%  http://www.colorado.edu/its/docs/latex/thesis/  %%%%%%%
%%%%%%%%%%%%%%%%%%%%%%%%%%%%%%%%%%%%%%%%%%%%%%%%%%%%%%%%%%%%%%%%
%		\documentclass[typewriterstyle]{thesis}
% 		\documentclass[modernstyle]{thesis}
% 		\documentclass[modernstyle,11pt]{thesis}
%	 	\documentclass[modernstyle,12pt]{thesis}

%%%%%%%%%%%%%%%%%%%%%%%%%%%%%%%%%%%%%%%%%%%%%%%%%%%%%%%%%%%%%%%%
%%%%%%%%%%%    load any packages which are needed    %%%%%%%%%%%
%%%%%%%%%%%%%%%%%%%%%%%%%%%%%%%%%%%%%%%%%%%%%%%%%%%%%%%%%%%%%%%%
\usepackage{latexsym}		% to get LASY symbols
\usepackage{graphicx}		% to insert PostScript figures
%\usepackage{deluxetable}
\usepackage{rotating}		% for sideways tables/figures
\usepackage{natbib}  % Requires natbib.sty, available from http://ads.harvard.edu/pubs/bibtex/astronat/
\usepackage{savesym}
\usepackage{amssymb}
%\savesymbol{singlespace}
\savesymbol{doublespace}
%\usepackage{wrapfig}
%\usepackage{setspace}
\usepackage{xspace}
\usepackage{color}
\usepackage{multicol}
\usepackage{mdframed}
\usepackage{url}
\usepackage{subfigure}
%\usepackage{emulateapj}
\usepackage{lscape}
\usepackage{grffile}
\usepackage{standalone}
\standalonetrue
\usepackage{import}
\usepackage[utf8]{inputenc}
\usepackage{longtable}
\usepackage{booktabs}



%%%%%%%%%%%%%%%%%%%%%%%%%%%%%%%%%%%%%%%%%%%%%%%%%%%%%%%%%%%%%%%%
%%%%%%%%%%%%       all the preamble material:       %%%%%%%%%%%%
%%%%%%%%%%%%%%%%%%%%%%%%%%%%%%%%%%%%%%%%%%%%%%%%%%%%%%%%%%%%%%%%

% \title{Star Formation in the Galaxy}
% 
% \author{Adam G.}{Ginsburg}
% 
% \otherdegrees{B.S., Rice University, 2007\\
% 	      M.S., University of Colorado, Boulder, 2009}
% 
% \degree{Doctor of Philosophy}		%  #1 {long descr.}
% 	{Ph.D., Rocket Science (ok, fine, astrophysics)}		%  #2 {short descr.}
% 
% \dept{Department of}			%  #1 {designation}
% 	{Astrophysical and Planetary Sciences}		%  #2 {name}
% 
% \advisor{Prof.}				%  #1 {title}
% 	{John Bally}			%  #2 {name}
% 
% \reader{Prof.~Jeremy Darling}		%  2nd person to sign thesis
% \readerThree{Prof.~Jason Glenn}		%  3rd person to sign thesis
% \readerFour{Prof.~Michael Shull}	%  4rd person to sign thesis
% \readerFour{Prof.~Neal Evans}	%  4rd person to sign thesis
% 
% \abstract{  \OnePageChapter	% one page only ??
% 
%     I discovered dust in space.  
% 
% 	}
% 
% 
% \dedication[Dedication]{	% NEVER use \OnePageChapter here.
% 	To 1, the second number in binary.
% 	}
% 
% \acknowledgements{	\OnePageChapter	% *MUST* BE ONLY ONE PAGE!
% 	All y'all.
% 	}
% 
% \ToCisShort	% a 1-page Table of Contents ??
% 
% \LoFisShort	% a 1-page List of Figures ??
% %	\emptyLoF	% no List of Figures at all ??
% 
% \LoTisShort	% a 1-page List of Tables ??
% %	\emptyLoT	% no List of Tables at all ??
% 
% 
% %%%%%%%%%%%%%%%%%%%%%%%%%%%%%%%%%%%%%%%%%%%%%%%%%%%%%%%%%%%%%%%%%
% %%%%%%%%%%%%%%%       BEGIN DOCUMENT...         %%%%%%%%%%%%%%%%%
% %%%%%%%%%%%%%%%%%%%%%%%%%%%%%%%%%%%%%%%%%%%%%%%%%%%%%%%%%%%%%%%%%
% 
% %%%%  footnote style; default=\arabic  (numbered 1,2,3...)
% %%%%  others:  \roman, \Roman, \alph, \Alph, \fnsymbol
% %	"\fnsymbol" uses asterisk, dagger, double-dagger, etc.
% %	\renewcommand{\thefootnote}{\fnsymbol{footnote}}
% %	\setcounter{footnote}{0}

\input{macros}		% file containing author's macro definitions

\begin{document}
% \input{introduction}
% 
% %\input{ch_iras05358}
% \input{ch_w5}
% \input{ch_h2co}
% \input{ch_h2colarge}
% \input{ch_boundhii}
% 
% %\input ch2.tex			% file with Chapter 2 contents
% 
% %%%%%%%%%%%%%%%%%%%%%%%%%%%%%%%%%%%%%%%%%%%%%%%%%%%%%%%%%%%%%%%%%%%
% %%%%%%%%%%%%%%%%%%%%%%%  Bibliography %%%%%%%%%%%%%%%%%%%%%%%%%%%%%
% %%%%%%%%%%%%%%%%%%%%%%%%%%%%%%%%%%%%%%%%%%%%%%%%%%%%%%%%%%%%%%%%%%%
% 
% \bibliographystyle{plain}	% or "siam", or "alpha", or "abbrv"
% 				% see other styles (.bst files) in
% 				% $TEXHOME/texmf/bibtex/bst
% 
% \nocite{*}		% list all refs in database, cited or not.
% 
% \bibliography{thesis}		% bib database file refs.bib
% 
% %%%%%%%%%%%%%%%%%%%%%%%%%%%%%%%%%%%%%%%%%%%%%%%%%%%%%%%%%%%%%%%%%%%
% %%%%%%%%%%%%%%%%%%%%%%%%  Appendices %%%%%%%%%%%%%%%%%%%%%%%%%%%%%%
% %%%%%%%%%%%%%%%%%%%%%%%%%%%%%%%%%%%%%%%%%%%%%%%%%%%%%%%%%%%%%%%%%%%
% 
% \appendix	% don't forget this line if you have appendices!
% 
% %\input appA.tex			% file with Appendix A contents
% %\input appB.tex			% file with Appendix B contents
% 
% %%%%%%%%%%%%%%%%%%%%%%%%%%%%%%%%%%%%%%%%%%%%%%%%%%%%%%%%%%%%%%%%%%%
% %%%%%%%%%%%%%%%%%%%%%%%%   THE END   %%%%%%%%%%%%%%%%%%%%%%%%%%%%%%
% %%%%%%%%%%%%%%%%%%%%%%%%%%%%%%%%%%%%%%%%%%%%%%%%%%%%%%%%%%%%%%%%%%%
% 
% \end{document}
% 
% 

\chapter{Outflows and proto-OB stars in a small protocluster, IRAS 05358+3543}

\section{Introduction}
Collimated, bipolar outflows accompany the birth of young stars from the
earliest stages of star formation to the end of their accretion phase
\citep[e.g.][]{reipurth2001}.    While the birth of isolated low-mass stars is
becoming well understood, the formation of massive stars ($>10 \msun$) and clusters remains a
topic of intense study.    Observations show that moderate to high-mass stars
tend to form in dense clusters \citep{lada2003}.    In a clustered environment,  the dynamics of
the gas and stars can profoundly impact both accretion and mass-loss processes.
Feedback from these massive clusters may play a significant role in momentum
injection and turbulence driving in the interstellar medium.  

Outflows from massive stars are less studied than those from low mass stars
largely because massive stars accrete most of their mass while deeply embedded.
Therefore, unlike low mass young stars that are accessible in the optical,
massive stellar outflows can only be seen at infrared and longer wavelengths.
Direct evidence for jets from massive young stellar objects (YSOs) from \hh\ or
optical emission is generally lacking
\citep[e.g.][]{alvarez2005,kumar2002,wang2003}, although there is evidence that
massive stars are the sources of collimated molecular outflows from millimeter
observations \citep[e.g.][]{beuther2002b}.  Outflows from massive stars may
allow accretion to continue after their radiation pressure would
otherwise halt accretion in a spherically symmetric system
\citep{krumholz2009}.  They therefore represent a crucial component in
understanding how stars above $\sim$10 \msun\ can form.

\region\ is a double cluster of embedded infrared sources located at a distance
of 1.8 kpc in the Auriga molecular cloud complex \citep{heyer1996} associated
with the HII regions Sh-2 231 through 235 at Galactic coordinates around $l,~b$
= 173.48,+2.45 in the Perseus arm.   \necluster\ is the
collection of highly obscured and mm-bright sources slightly northeast of
\swcluster, which is the location of the IRAS 05358+3543 point source and the
optical emission nebula (see Figure \ref{fig:overview_ha}).  The IRAS source is
probably a blend of the three brightest infrared objects in the MSX A-band and
MIPS 24 \um\ images, which are located at \necluster, IR 41, and IR 6. For the
purpose of this paper,the whole complex including both sources is referred toas
\region, and otherwise refer to individual objects specifically.

Early observations revealed the presence of OH \citep{Wouterloot1993}, \HtwoO\ 
\citep{Scalise1989, Henning1992}, and methanol \citep{Menten1991} masers about
an arcminute northeast of the IRAS source, indicating that massive stars are 
likely present at that location.  Near infrared observations revealed
the presence of two embedded clusters  \citep{porras2000,jiang2001} labeled
\swcluster\ for the southwestern cluster associated with the IRAS source, and \necluster\ 
for the northeastern cluster located near the OH, \HtwoO, and CH$_3$OH masers.
Stars identified in \citet{porras2000} are referred to by the designation
``IR (number)'' corresponding to the catalog number in that paper.
\citet{porras2000} also included scanning Fabry-Perot velocity measurements of
the inner $\sim1$\arcmin.  CO observations revealed broad line wings indicative
of a molecular outflow \citep{casoli1986,shepherd1996}.  \citet{kumar2002} and
\citet{khanzadyan2004} presented narrow band images of 2.12 \um\ \htwo\
emission that reveled the presence of multiple outflows.  Interferometric
imaging of CO and SiO confirmed the presence of at least three flows emerging
from the northeast cluster centered on the masers \citep{beuther2002} having a
total mass of about 20 \msun .  \citet{beuther2002} also presented MAMBO 1.2 mm
maps and a mass estimate of 610 \msun\ for the whole region.
\citet{williams2004} presented SCUBA maps and mass estimates of the clusters of
195/126\msun\ for \necluster\ and 24/12 \msun\ for \swcluster\ (850 \um/450 \um).
\citet{Zinchenko1997} measured the dense gas properties using the \ammonia\
(1,1) and (2,2) lines.  They measure a mean density $n \approx 10^{3.60}$ \percc,
temperature 26.5K, and a mass of 600 \msun .  The total luminosity of the two
clusters is about 6300 \lsun , indicating that the region is giving birth to
massive stars \citep{porras2000}. 

Millimeter wavelength interferometry with arcsecond angular resolution has
revealed a compact cluster of deeply embedded sources centered on the \HtwoO\
and methanol maser position \citep{beuther2002,beuther2007,leurini2007}.
\citet{beuther2002} identified 3 mm continuum cores, labeled mm1-mm3 (shown in
Figure \ref{fig:outflowsh2}).  \citet{beuther2007} resolved these cores into
smaller objects.  Source mm1a is associated with a cm continuum point source
and will be discussed in detail below.

\region\ has previously been observed at low spatial resolution in the J=2-1 and J=3-2
transitions with the Kosma 3m telescope \citep{Mao2004}.  While the general presence
of outflows was recognized and a total mass estimated, the specific outflows were not 
resolved.  \citet{beuther2002} observed the CO J=6-5, J=2-1, and J=1-0
transitions at moderate resolution in the inner few arcminutes.
\citet{Thomas2008} observed C$^{17}$O in the J=2-1 and J=3-2 transitions with a
single pointing using the JCMT.



\section{Observations}

A collection of data acquired by the authors and from publicly
available archives is presented.  An overview of the data is presented in figure
\ref{fig:overview_ha}. The goal was to develop a complete picture of the outflows
in \region\ and their probable sources.  CO data were acquired to estimate the
total outflowing mass and to identify outflowing molecular material
unassociated with \hh\ shocks.  Archival Spitzer IRAC and MIPS 24 \um\ data
were used to identify probable YSOs as candidate outflow sources.
Near-infrared spectra were acquired primarily to determine \hh\ kinematics and
develop a 3D picture of the region.  Optical spectra were acquired to attempt
to identify stellar types in the unobscured \swcluster\ region.  Finally,
archival VLA data were used to acquire better constraints on the position and
physical properties of the known ultracompact HII (UCHII) region, and to detect
or set limits on other UCHIIs.

\subsection{Sub-millimeter Observations}

The 345 GHz J = 3-2 rotational transition of CO was observed with the James
Clerk Maxwell Telescope (JCMT) on 4 January,  2008 with the 16 element (14
functional) HARP-B heterodyne focal plane array.   Two  12\arcmin\ $\times$
10\arcmin\  raster scans in R.A.  and Dec.  were taken with orthogonal
orientations to assure complete coverage in the region of interest; this
resulted in a useable field 11.7\arcmin\ $\times$ 11.3\arcmin\ with higher noise
along the edges.  The beam size at 345 GHz is about 15\arcsec.

Observations were conducted during grade 3 conditions with the 225 GHz zenith
optical depth of the atmosphere $\tau\sim0.1$. A channel width of 488 kHz
corresponding to 0.423 \kms\ was used.    The maps required a total of 1 hour
to acquire and resulted in an effective integration time of 4.6 seconds per
pixel (there are 12,000 $6\times6\arcsec$\ pixels in the final grid), resulting
in a noise per pixel of 0.36 K \kms.

The optical depth and telescope efficiency corrections were applied by the JCMT
pipeline to convert the recorded antenna temperatures to the corrected antenna
%temperature T$_A^*$
%where $T_A^* = T_A / ( \eta_T ~ \tau_{345} ~ \textrm{sec}[z] )$ and
%$\tau_{345} = .05 + 2.5 \tau_{225}$
\footnote{See \\
\url{http://docs.jach.hawaii.edu/JCMT/OVERVIEW/tel\_overview/} for a discussion
of JCMT parameters}.  An additional
main-beam correction has been applied, $$T_{mb}=\frac{T_A*}{\eta_{mb}}$$ where
$\eta_{mb} $ was measured by observing Mars to be $\approx0.60$ at 345
GHz.  Emission in the sidelobes is expected to be small at the outflow velocities.

On September 25 and November 15, 2008 the CO, $^{13}$CO, and C$^{18}$O J=2-1
transitions were observed in the central 3\arcmin\ of \region.  The beamsize
at 220 GHz is about 23\arcsec.  The sideband configuration used also includes
the \linebreak \nolinebreak{\so} and $^{13}$CS 5-4 transitions.  Conditions
during these observations were grade 5 ($\tau \sim 0.24-0.28$) and therefore
too poor to use the HARP instrument, but acceptable for the A3 detector.  

Data reduction used the Starlink package following the standard routines
recommended by the JCMT support scientists \footnote{
\url{http://www.jach.hawaii.edu/JCMT/spectral\_line/data\_reduction/acsisdr/}}.
The CO 3-2 data cube was extracted over a velocity range from --50 to 10 \kms\ LSR and
spectral baselines were fit over the velocity range --50 to --40 and 0 to 10
\kms\ and subtracted.  The data were re-gridded into  6\arcsec\  pixels  and 2
pixel Gaussian smoothing was used to fill in the gaps left by the two bad
detectors in the 4 $\times$4 array.   The data cube was cropped to remove
undersampled edges which have high noise and bad baselines.  The beam efficiency
was 0.68 at 230 GHz.

The A3 data cubes were extracted over the velocity range --60 to 20 \kms\ and
baselines were calculated over --60 to --40 and 0 to 20 \kms.  The data
was gridded into 10\arcsec\ pixels with 2 pixel gaussian smoothing to reduce
sub-resolution noise variations.  

% SCUBA 450 and 850\um\ data were acquired from the JCMT archive via the 
% Canadian Astronomical Data Center.

\subsection{Spitzer}

Spitzer IRAC bands 1 to 4 and MIPS band 1 data were retrieved from the Spitzer
Science Center archive.  \citet{qiu2008} acquired the data as part of a study
of many high-mass star forming regions; they identified YSO candidates based on
IRAC colors.  The version 18 post-BCD data products were used to produce images
and photometric catalog from \citet{qiu2008}, which was made from a more
carefully-reduced data set, was used for SED analysis.

\subsection{Near-IR images}
Near-infrared data were acquired using the Wide-field Infrared Camera (WIRCam) on
the Canada-France-Hawaii Telescope (CFHT) on Mauna Kea. The field of view is 
20\arcmin$\times$20\arcmin\ ~and pixel scale 0.3\arcsec.  Data were acquired on 
November 18, 19 and December 20, 2005.  The seeing was 0.5-0.7\arcsec\ during the
observations.  A 0.032 \um\ wide filter centered at 2.122 \um\ was used to take images
of the \hh\ S(1) 1-0 rovibrational transition.  Each \hh\ exposure was 58
seconds, and dithered images were taken for a total exposure time of 1755
seconds.  The data were reduced with the WIRCam pipeline.

\subsection{Near-IR spectra}
Near-infrared spectra were acquired using the TripleSpec instrument at Apache
Point Observatory.  TripleSpec simultaneously acquires J, H, and K band spectra
over a 42\arcsec\ long slit.  A slit width of 1.1\arcsec\ with an
approximate spectral resolution $\lambda/\Delta\lambda=2700$ was used.  

Observations were taken on the nights of December 2, 2008 and January 7, 2009.
Data on December 2 were taken in an ABBA nod pattern, but because of the need
to observe extended structure across the slit a stare strategy was selected on
January 7.  

The data were reduced using the {\sc twodspec} package in IRAF.  
HD31135, an A0 star, was used as a flux calibrator.  Wavelength calibration was
performed using night sky lines.  Lines filling the slit were subtracted
to remove atmospheric emission lines.  Telluric absorption correction was {\emph
not} performed, but telluric absorption is considered in the analysis.

The transformations from the observed geocentric reference frame to $v_{LSR}$ 
were computed to be 0.78 \kms\ on Dec 2 and 19.74 \kms\ on Jan 8.
% and 24.35 \kms\ on Jan 17.

\subsection{Optical Spectra}
Optical spectra were acquired using the Double Imaging Spectrograph instrument
at APO.  The high-resolution red and blue gratings
were centered at 6564 \AA\ and 5007 \AA\ with a coverage of about 1200 angstroms and
resolution $\lambda/\Delta\lambda \approx 5000$.  Sets of three 900s exposures
and three 200s exposures were acquired on the targets and on the spectrophotometric
calibrator G191-b2b with a 1.5" slit.  Observations were taken on the night
of January 17, 2009 under clear conditions.

Optical spectra were also reduced using the {\sc twodspec} package in IRAF.
Wavelength calibration was done with HeNeAr lamps and night sky lines
in the red band, and HeNeAr lamps in the blue band.  Lines filling the slit
were subtracted to remove atmospheric lines, though some astrophysical lines
also filled the slit and these were measured before background subtraction.
The $v_{LSR}$ correction for this date was 24.4 \kms.

% We convert our observations to the local standard of rest using...
% vLSR = vBSR + 9 cos(l) cos(b) + 12 sin(l) cos(b) + 7 sin(b) 
% 
% So the $v_{LSR}$ corrections @midnight are (towards object?):
% Jan 17 2009: 24.35 \kms
% Jan 8  2009: 19.74
% Dec 2  2009: 0.78 \kms
% 
% Subtract these velocities to get the correct velocity.  < Dec we were moving
% towards the object, after Dec 1ish we were moving away.
% 
% using \url{http://www.jupiterspacestation.org/software/Vlsr.html}

\subsection{Optical imaging}

CCD images images were obtained on the nights of 14 and 15 September 2009
NOAO Mosaic 1 Camera at the f/3.1 prime focus 
of the 4 meter Mayal telescope atthe  Kitt Peak National Observatory (KPNO).  
The Mosaic 1 camera is a 8192$\times$8192 pixel array (consisting of eight 
2048$\times$4096 pixel CCD chips) with a pixel scale of 0.26$''$ pixel$^{-1}$ and 
a field of view 35.4$'$ on a side.  Narrow-band filters centered on 
6569\AA\ and 6730\AA\ both with a FWHM of 80\AA\ were use to obtain 
H$\alpha$ and [SII] images.  An SDSS i' filter which is centered on 
7732\AA\ with a FWHM of 1548\AA was used for continuum imaging.
A set of five dithered 600 second exposures were obtained in 
H$\alpha$ and [SII] using the standard MOSDITHER pattern 
to eliminate cosmic rays and the gaps between the
individual chips in Mosaic.  A dithered set of five 180 second 
exposures were obtained in the in the broad-band SDSS i-band filter to
discriminate between H$\alpha$, [SII], and continuum emission. 
Images were reduced in the standard manner by the NOAO Mosaic reduction
pipeline \citep{valdes2007}.   


\Figure{figures_ch05358/mosaicHA_fullfield_simbadlabels}{The CFHT \hh\ (bold), CO 3-2 HARP
(thin), and CO 2-1 A3 (dashed) fields overlaid on the KPNO \ha\ mosaic
with selected objects identified by their SIMBAD names.  \swcluster\ is coincident
with IRAS 05358+3543. % \necluster\ is coincident with the SIMBAD position of LBN 808.
% WARNING: SIMBAD reports S231 at the position of LBN 808, while the original Sharpless
% catalog reports it to be where it is identified in our map.  Similarly SIMBAD reports
% S233 at LBN 802 while we mark the original Sharpless position.
% IRAS 05358+3543 or S233IR is the point to the southwest of LBN 808.
}{fig:overview_ha}{1.0}{}

% VLA DATA SECTION ( VERY LARGE ARRAY ) 
\subsection{VLA data}
VLA archival data from projects AR482, AR513, AS831, and AM697 were re-reduced
to perform a deeper search for UCHII regions and aquire more data points on the
known UCHII's SED.  Data from AR482 were previously published in
\citet{beuther2007}, the other data are unpublished.  The data were reduced
using the VLA pipeline in AIPS ({\sc vlarun}).  The observations used and
sensitivities and beam sizes achieved are listed in Tables \ref{tab:vlatimes}
and \ref{tab:vla}.  There appeared to be calibration errors in the AR482
observations (the phase calibrator was 2-3 times brighter than in all other
observations) and this data were therefore not used in the final analysis, but
it produced consistent pointing results.  


\Table{cccccccc}
{VLA Observation Program Names, Dates, and Times}
{\colhead{VLA } &  \colhead{Observation} &  \colhead{Time } &  \colhead{Array} &
\colhead{Band} & \colhead{Fluxcal} &  \colhead{Phase cal} & \colhead{Phase cal } \\
Observation & Date &on&&&&& Percent \\
Name &&Source&&&&& Uncertainty \\}
{tab:vlatimes}
{
AR482 & August 2 2001    & 2580s & B & X &3c286 & 0555+398   & 22  \\
AR513 & June 21 2003     & 7770s & A & X &3c286 & 0555+398   & 0.8 \\
AS831 & February 26 2005 & 2640s & B & X &3c286 & 0555+398   & 0.7 \\
AS831 & August 5 2005    & 2660s & C & X &3c286 & 0555+398   & 0.3 \\
AS831 & May 11 2006      & 2610s & A & X &3c286 & 0555+398   & 3.0 \\
AL704 & August 7 2007    & 6423s & A & Q &3c273 & 0555+398   & 18  \\ 
AL704 & September 1 2007 & 6423s & A & Q &3c273 & 0555+398   & 13  \\ 
AM697 & November 26 2001 & 2880s & D & Q &3c286 & 0555+398   & 2.2 \\
AM697 & November 28 2001 & 1530s & D & K &3c286 & 0555+398   & 2.1 \\
AM697 & November 28 2001 & 1530s & D & U &3c286 & 0555+398   & 5.8 \\
}{}


\section{Results}
\subsection{Near Infrared Imaging: Outflows and Stars}
Eleven distinct outflows have been identified in \region\ in the images.
Outflows are identified from a combination of J=3-2 CO data, shock excited
\htwo\ emission, and published interferometric maps \citep{beuther2002}.
Suspected CO outflows were identified by the presence of wings on the CO J=3-2
emission lines that extended beyond the typical velocity range of emission
associated with the line core.    The single dish data were compared to the
interferometric maps of \citet{beuther2002}.  The CFHT \hh\ image was then used
to search for shock-excited emission associated with the outflow lobes.

\begin{figure*}[htpb]
    \center
    \epsscale{0.75}
    \hspace{-1.2in}
    \plotone{figures_ch05358/outflow_overview}
%    \includegraphics[width=2.25in]{outflow_overview}
%    \hspace{-0.1in}
%    \includegraphics[width=2.5in]{COcontours_on_H2}
%    \hspace{-0.5in}
%    \includegraphics[width=2.5in]{so5645_on_H2}
    \caption{The outflows described in section \ref{sec:outflows} overlaid on
    the CFHT \hh\ image.  Numbers followed by {\it r} and {\it b} (red and
    blue), {\it n} and {\it s} (north and south), or {\it e} and {\it w} (east
    and west) are thought to be counterflows.  Red and blue vectors indicate
    red and blue doppler shifts.  Green vectors indicate where the doppler
    shift is ambiguous or cannot be determined.  Magenta circles are Spitzer
    24\um\ sources.  Red squares are \citet{beuther2002} mm sources (from left
    to right, mm1, mm2, mm3).  The blue diamond is a YSO candidate detected
    only in IRAC bands.  The length of the vectors corresponds to the
    approximate length of the outflows.  Source 1 and 6 correspond to
    \citet{porras2000} IR 6 and IR 41 respectively, and they are discussed under
    these names in sections \ref{sec:outflows}.  The bows of Outflow 1n and 4n 
    are detected in \ha\ and [S II] emission and are therefore as identified
    as Herbig-Haro objects HH 993 and 994 respectively.
    \label{fig:outflowsh2}}
\end{figure*}
% \Figure{figures_ch05358/khanzadyan_outflow_labels}{The outflows as identified by \citet{khanzadyan2004}
% overlaid on our \hh\ image}{fig:klabels}{0.75}
% \Figure{figures_ch05358/spitzer_RGB}{A Spitzer color composite image of the region.  Red: 8\um\ 
% Green: 4.5\um\  Blue: 3.6\um.}{fig:spitzer_rgb}{1.0}

\Figure{figures_ch05358/COcontours_on_H2} {CO contours integrated from $v_{LSR}=$ -13 to -4
\kms\ (red) and -34 to -21 \kms\ (blue) at levels of 2,4,6,10,20,30,40,50 K
\kms\ overlaid on the \hh\ image.  Specific outflows are labeled in Figure
\ref{fig:outflowsh2} on the same scale.} {fig:COonH2}{0.75}{}

\begin{figure*}[htpb]
    \hspace{-0.6in}
  \includegraphics[scale=0.40,clip=true]{figures_ch05358/so5645_on_h2}
%    \hspace{-0.6in}
  \includegraphics[scale=0.40,clip=true]{figures_ch05358/SII_outflowvectors}
  \caption{(a) \hh\ image with \so\ peak flux contours at 0.5-1.4 K in intervals of 0.15 K
  overlaid.  With a critical density $\sim3.5\ee{6}$ \citep{leidendb}, this
  transition is a dense gas tracer.  (b) The [S II] image with outflow vectors overlaid.
  Diffuse emission can be seen at the north ends of Outflows 1, 4, and 6 and around the
  reflection nebula near source IR 41.} 
  \label{fig:so_on_h2}
\end{figure*}



%\subsection{\htwo\ results}
\label{sec:outflows}

% fix figure references
% This section refers to three figures.
Figure \ref{fig:outflowsh2} shows the \htwo\ S(1) 1-0 2.1218 \um\ (a rovibrational
transition in the electronic ground state from the $v=1$, $J=3$ to the $v=0$,
$J=1$ state) emission in the vicinity of \region\ with outflows and possible
outflow sources labeled.  The mm cores from \citet{beuther2002} are identified
by red squares.  
% Figure \ref{fig:klabels} shows the same image with labels
% from \citet{khanzadyan2004} superposed.  Figure \ref{fig:spitzer_rgb} shows the
% same region in a  3-color IRAC image.

The flow vectors in figure \ref{fig:outflowsh2} were chosen on the basis of the
\htwo\ bow shock morphologies and orientations of chains of \htwo\ features,
association with arcsecond-scale CO features on the \citet{beuther2002} Figure
8 CO map, and/or association with lobes of Doppler-shifted CO emission in the
CO 3-2 data (see figure \ref{fig:COonH2}).  The color of the vector indicates
the suspected Doppler shift; red and blue correspond to red and blueshifts and
green vectors indicate that the Doppler shift is uncertain.    
 
% no clear red/blue
% probable from mm1a by Beuther2002
 {\it \region\ outflow 1:} The most prominent flow in \htwo\ is associated with
 the bright bow-shocks N1 and N6 \citep{khanzadyan2004} located towards PA $\approx$ 345\arcdeg\
 and 170\arcdeg\ respectively from the sub-mm source mms1b \citep{beuther2002}.
 This flow, \citet{beuther2002} outflow A, is associated with redshifted and
 blueshifted CO emission.  The northern shock is seen in \ha\ and [S II]
 emission (figure \ref{fig:so_on_h2}b and \ref{fig:HA_with_CO}) and is given a Herbig-Haro designation
 HH 993.

 This flow is indicated by oppositely directed green vectors from the vicinity
 of smm1, 2, and 3.   It is listed as ``Jet 1'' in \citet{qiu2008}.  \citet{kumar2002}
 identified the knot immediately behind the bow shock as a Mach disk.  In the
 \citet{beuther2002} interferometric maps, the north flow contains redshifted features
 and the south flow contains primarily blueshifted features.  There are also blueshifted 
 CO features to the west of the \hh\ knots that are probably part of a different flow
 that is not seen in \hh\ emission.
 
 The velocity of the flow as measured from \hh\ emssion is blueshifted as much
 as 80 \kms (LSR), but one component is blueshifted only 14 \kms\, which is
 consistent with the cloud velocity.  A redshifted SiO lobe is present in the
 south counterflow.  The presence of \ha, [S II], and [O III] emission in the
 north shock and corresponding nondetections in the south shock suggest that
 there is substantially greater extinction towards the south knot.  While the
 velocities in three of the four apertures picked along the TripleSpec slit are
 blueshifted, there are also knots with velocities consistent with the cloud
 velocity.  \citet{porras2000} measure the velocity of the counterflow to be
 -17.3 \kms, which is consistent with the cloud velocity. Outflow 1 is
 propagating very nearly in the plane of the sky.

% Our optical [S II] data were used to measure a density in the bow shock $\sim
% 500-700$\percc, which is typical of Herbig-Haro bow shocks (e.g. [CITE SOMEONE -
% BALLY?]).

% sources are not clear...
A line connecting the two bow shocks in Outflow 1 goes directly through
\citet{beuther2007} source mm2a despite the clear association in the
\citet{beuther2002} interferometric CO map (their Figure 8) with mm1a.  The
currently available data do not clarify which is the source of the outflow:
while the bent CO outflow appears to trace Outflow 1 back to mm1a, there are
additional parallel CO outflows towards the confused central region that could
originate from either mm1a or mm2a.

A Spitzer 4.5 \um\ and 24 \um\ source is barely detected in \hh\ 2.5\arcmin\ to
the north of Outflow 1.  It is only apparent when the \hh\ image is smoothed
and would have been dismissed as noise except for the association with
a probably 4.5 \um\ extended source.  It is labeled 24\um\ source 7 in figure
\ref{fig:outflowsh2}.  It appears to be slightly resolved at 4.5\um, and is
therefore likely shocked emission.  The object may be a protostellar source 
with an associated outflow, but its proximity to the projected path of Outflow
1 suggests that it may be an older outflow knot.

% Outflow 2 comes from Minier disk
{\it \region\ Outflow 2:} The second brightest \htwo\ features trace a bipolar
flow emerging from the immediate vicinity of the sub-mm cluster at PA $\approx$
135\arcdeg\ (red lobe) and 315\arcdeg\ (blue lobe).  It is listed as ``Jet 2'' in
Figure 6 of \citet{qiu2008}.  
% The northwest lobe is suspected to be blueshifted
% because of the high brightness of the \htwo\ emission (bow shock N3A, B, and C
% in \citet{khanzadyan2004}, figure \ref{fig:klabels}) and the presence of blueshifted emission
% in the JCMT CO data.  However, none of these lines of evidence are conclusive.
The counterflow probably overlaps in the line of sight with the counterflow
from Outflow 3.  It is shorter on the counterflow side either because it has
already penetrated the cloud and is no longer impacting any ambient gas or,
more likely, it has slowly drilled its way out of the molecular cloud and has
not been able to propagate as quickly as the northwest flow.  
The \hh\ velocities measured for these knots are $\sim$ 30 \kms\ blueshifted, or
marginally blue of the cloud LSR velocity.  

The disk identified in \citet{Minier2000} is approximately perpendicular to the
measured angle of Outflow 2 assuming that mm1a is the source of this flow.  It
is therefore an excellent candidate for the outflow source.  A diagram of the
mm1a region is shown in figure \ref{fig:mm1adiagram}.  See Section
\ref{sec:vlaresults} for detailed discussion.



{\it \region\ outflow 3:} The \citet{beuther2002} CO and SiO maps reveal a
third flow, their outflow B at PA $\approx$ 135\arcdeg\  (red lobe) and
315\arcdeg\ (blue lobe).  A chain of  \htwo\ features, \citet{khanzadyan2004}
features N3D and N3E, are probably shocks in this flow.  It is listed as ``Jet
3'' in \citet{qiu2008}.  The two chains of \htwo\ emission indicate that
outflows 2 and 3 are distinct.  There also appears to be a counterflow at a
shorter distance from the mm cores similar to counterflow 2.  

Outflows 2 and 3 may be associated with either redshifted or blueshifted
features in the \citet{beuther2002} CO and SiO maps.  High velocity flows with both
parities are present near both the northwest (\citet{beuther2002} outflow C)
and southeast flow for these jets, but the resolution of the millimeter
observations is inadequate to determine which flow is in which direction.
\citet{porras2000}
measures $v_{LSR} = -7.5$ \kms\ for their knot 4A, which corresponds to the 
blended southeast counterflow of outflows 2 and 3. Their Figure 7 shows a wide 
line that is probably better represented by two or three blended lines, one
consistent with the cloud velocity and the other(s) redshifted.  Since Outflow 2
has a measured blueshift and outflow 3 is significantly fainter, the redshifted
counterflow emission is probably associated with Outflow 2 and the blueshifted
with outflow 3.

{\it \region\ outflow 4:} The JCMT CO data and \htwo\ images reveal a large
outflow lobe consisting of blue lobes 1 and 4 that form a tongue of blueshifted
emission propagating to the northeast at PA $\approx$  20\arcdeg\ (Figure
\ref{fig:outflowsh2}) from the cluster of sub-mm cores.   A faint chain of
\htwo\ features runs along the axis of the CO tongue and terminates in a bright
\htwo\ bow shock located at the northern edge of \ref{fig:outflowsh2}.   Several
\htwo\ knots lie along the expected counterflow direction, but that portion of
the field contains multiple outflows and is highly confused.  If the
counterflow is symmetric with the northeast knot, it extends 2.1 parsecs on the
sky.  

The bow shock of Outflow 4 is seen in the HII and [S II] images, implying that
the extinction is much lower than in the cluster.  Two apertures placed along
the bow shock reveal that it is blueshifted about 70\kms\ and may be extincted
by as little as $A_V\sim.5$.  It is designated HH 994.
 
{\it \region\ outflow 5:} Figure \ref{fig:outflowsh2} shows a bright chain of
\htwo\ knots and bow shocks starting about 10\arcsec\ west of mm3 and
propagating south at PA $\approx$ 190\arcdeg.  The SiO maps of
\citet{beuther2002} show a tongue of blueshifted emission along this chain
(their Outflow C).   The outflow projects back to H$^{13}$CO$^+$ source 3,
which is also a weak mm source.  A lack of obvious counterflow and the
possibility that the knots identified with Outflow 5 could be associated with a
number of different crossing flows makes this identification very tentative.
Higher spatial resolution observations will be required to determine the
association of this outflow.


% OK.  This is dependent on 24um data and spitzer data where it is clear that
% outflows point back to this source
 {\it \region\ outflow 6:} The fourth brightest source in the Spitzer 24\um\ 
 data is located at J(2000) = 05:39:08.5, +35:46:38 (source 5 in the \region\ 
 section of the \citet{qiu2008} catalog) in the middle of the
 molecular ridge that extends from \region\ towards the northwest (24\um\ 
 object 4 in figure \ref{fig:outflowsh2}).   The star is
 located at the northwest end of the tongue of 1.2mm emission mapped by
 \citep{beuther2002} with the MAMBO instrument on the IRAM telescope.  This
 part of the cloud is also seen in silhouette against brighter surrounding
 emission at 8\um.  At wavelengths below 2\um, it is fainter than 14-th
 magnitude and therefore is not listed in the 2MASS catalog, and it is not
 detected in \citet{yan2009} down to 19th magnitude in K.
 %; it is also not detected in our deeper K-band images with an upper
 %limit of [FIND UPPER LIMIT] or ask Chi-hung.   
 
 Spitzer data indicates very red colors between 3.6 and 70 \um, indicating that
 this object is likely to be a Class I protostar.  The SED is fit using the
 online tool provided by \citet{robitaille2007}.  Unfortunately, a wide variety
 of parameters all achieved equally good fits, so no conclusions are drawn
 about the stellar mass or other very uncertain parameters.  However, the top
 models all had $A_V > 20$ and many in the range 30-50, indicating that the
 line of sight is probably through a thick envelope or disk towards this
 source.  

 % note: Av=(0.56N_H+0.23)*10e21
 
 This source lies at the base of the tongue of blueshifted CO 3-2
 emission that extends northwest of \region\ at PA $\approx$ 345\arcdeg\  and
 has mass $\sim .5\msun$.  A pair of \htwo\ features,
 \citet{khanzadyan2004} N12A and N12B are located 30 and 55\arcsec\ from the
 suspected YSO, forming a chain along the axis of the blueshifted CO
 tongue.    \citet{khanzadyan2004} \htwo\ knot N3F lies along the flow axis in
 the redshifted direction.
 
{\it \region\ outflow 7:} The 20\arcsec\ long chain of \htwo\ knots labeled
\citet{khanzadyan2004} N11 appears to trace part of a jet at PA $\approx$
345\arcdeg\ that propagated parallel to outflow 6 about 20\arcsec\ to the east.
The northwest portion of Outflow C in the \citet{beuther2002} SiO map is in 
approximately the same direction as Outflow 7, and it may represent a redshifted
counterflow to the northwest-pointing \hh\ knots.
The jet axis passes within a few arc-seconds of a faint and red YSO located at
J(2000) = 05 39 10.0, +35 46 27 (blue diamond in figure \ref{fig:outflowsh2}
about 35\arcsec\ south of the southern end of the \htwo\ feature).  It may be a
24\um\ source but is lost in the PSF of the bright source at the center of
\necluster.  This object is also undetected down to 19th magnitude in the
\citet{yan2009} K-band image.
 
{\it \region\ outflow 8:}   A prominent jet-like \htwo\ feature protrudes from
the vicinity of \swcluster\ at PA $\approx$ 335\arcdeg\  and ends in
 bright knot %\citet{khanzadyan2004} \htwo\ knot 
N9.     The feature N5B is is
located just outside the ring of \htwo\ emission that surrounds the IRAS source
at the base of the jet.  Towards the southeast, %\citet{khanzadyan2004} \Htwo\
knot N6 is located opposite knot N9 with respect to the southwest cluster.  IR 41,
the \ha\ emission source, labeled 24\um\ source 6 in figure \ref{fig:outflowsh2},
is probably the source of this outflow.
 
{\it \region\ outflow 9:} In the Spitzer and K$_s$ images, an infrared
reflection nebula opens towards the southwest at PA $\approx$ 245\arcdeg\ and
points towards a blueshifted CO region.  The reflection nebula is also seen in
\ha.  It is likely that the CO emission in CO Region 1 (table \ref{tab:comeas})
traces a fossil cavity whose walls provide the scattering surface of the
reflection nebula.

{\it \region\ outflow 10 and IR 6:}   A bright \htwo\ filament protrudes at PA $\approx$
15\arcdeg\ towards the northeast of IR 6 (24\um\ source 1, \citet{qiu2008}
source 8).  The star is the third brightest 24\um\ source in the \region\
region.   Since it is visible at visual wavelengths, it is not heavily
embedded.     Its  H$\alpha$ emission and association with an outflow lobe and
\htwo\ emission suggest that it is a moderate mass Herbig AeBe star associated
with the \region\ complex.  The optical spectrum confirms this hypothesis: the
star has \ha\ absorption wings on either side of a very bright, asymmetric \ha\
emission profile (see section \ref{sec:dis}). 

IR 6 is seen to be the source of Outflow 10.  Data for this source is available
from $\sim$0.45-24\um, so the \citet{robitaille2007} spectral fitter puts
strong constraints on the star's mass and luminosity.  The measured mass and
luminosity are  $M=4.5\pm0.5$ \msun\ and $L = 10^{2.3\pm.25} L_\odot$, parameters
consistent with a B7V ($\pm 1$ spectral class) main sequence star.  The range
of ages in the models covers $10^4-10^7$ years but favors stars in the range
$10^5-10^6$ years.

While there is a small clump of redshifted CO emission to the northeast of the
object, the \htwo\ spectrum shows that the north flow is blueshifted $v_{LSR}\sim-40
$\kms, and the lack of a visible counterflow suggests that the counterflow may
be masked behind an additional extincting medium.  The counterflow drawn in
figure \ref{fig:outflowsh2} is not seen in emission but is identified as a
probable location for a counterflow because of the confident association of
outflow 10n with source IR 6.

{\it \region\ outflow 11:} A chain of \hh\ knots is seen at 2.12\um\ and in the
Spitzer 4.5\um\ image.  They trace back to either IR 78 or 24\um\ source 4.
There is a tongue of redshifted CO 3-2 emission in the same direction as this
flow that suggests it may be redshifted.



%\subsection{IR 41}
{\it IR 41}: 
There is an arc-like \hh\ emission feature surrounding the \ha\ emission line
star IR 41.  This implies that the star is probably a late B-type star with too
little Lyman continuum emission to generate a photon-dominated region (PDR) but
enough soft UV to excite \hh.  From the measured \ha\ and nondetection of \hb\
at the star's location down to a 5-$\sigma$ limit of 1\ee{-17} erg s$^{-1}$
\persc \AA$^{-1}$, a lower limit on the extinction column $A_V=15$ is derived.
The \citet{robitaille2007} fitter yields a mass estimate of 7.4$\pm 0.6$\msun
and luminosity $L=10^{2.97\pm.16}L_\odot$ among the 222 best fits out of a grid
of 200,000 model SEDs (fits with $\chi^2<5000$).  The luminosity is very well
constrained, varying only modestly to $L=10^{2.99\pm.15}L_\odot$ for the 904
best fits ($\chi^2 < 10000$),  while the mass shifts down to $6.5\pm1.0\msun$.
The mass estimate may be biased by the lower number of high-mass models
computed.  The star's mass is most compatible with a main sequence B4V star,
though its luminosity is closer to a B5V star.  The disk mass is constrained to
be $>10^3 \msun$.  The age is reasonably well constrained to be $T =
10^{5.78\pm.12}$ for the best 904 models, but is essentially unconstrained for
the best 222.  Similarly, the stellar temperature is entirely unconstrained by
the fitting process.

The very high values of $\chi^2$ would normally be worrisome, but the $\chi^2$
statistic only represents statistical error, while the data is dominated by
various systematic errors including calibration offsets in the optical/NIR and
poor resolution in the far-IR.  Therefore, it is not possible to find a perfect
model fit, but still possible to put constraints on the physical properties of
the source.

\Figure{figures_ch05358/HA_COcontours_COscale}{The \ha\ image with CO contours at
redshifted, blueshifted, and middle velocities in red, blue, and green
respectively.  Contours are at 2,4,8,12,20 K \kms\ for the red and blue, 
and 20,25,30,40,50,60,70 K \kms\ for the green.  Red is integrated from
-12 to -4 \kms, Blue from -31 to -21 \kms, and green from -21 to -12 \kms.
}{fig:HA_with_CO}{1.0}{}

% \Figure{figures_ch05358/HA_Spitzercontours_Spitzerscale}{\ha\ image with Spitzer 8\um\ contours
% that trace out the edge of the \ha\ emission}{fig:HA_with_8um}{1.0}


%\subsection{South of \region}
{\it South of \region}: 
There is a symmetric flow with one faint \hh\ knot and a bright central source
about 4\arcmin\ south of \region.  The \hh\ knot is at J(2000) = 05:39:15.63 +35:42:13.2.
The flow has a clear red and blue region as identified in figure
\ref{fig:cofig}; the red flow extends from -9 to -14 \kms\ and the blue from -19
to -23 \kms\ (the outflow is swamped by ambient emission in the intermediate
velocity range).  The outflow is $\sim 2\arcmin$ long, though the probable
source identified is not directly between the two lobes.  The ellipses used are
labeled in table \ref{tab:comeas} as Red S and Blue S.



\subsection{Imaging results: Optical}
Deep [S II] images show that some of the outflows have pierced through the
obscuring dust layers or excited extremely bright sulfur emission.
\citet{khanzadyan2004} knot N1 at the end of Outflow 1 is visible [S II] emission
%in figure \ref{fig:sii}.  
The bow of outflow 4 and the northwest end of outflow 6 are
detected in [S II].  Only the Outflow 1 and 4 bows are detected in \ha\ 
emission, indicating that the emission is most likely from shock heating, not
external photoionizing radiation.  If the shocks were externally irradiated, we
would expect the emission to be dominated by the recombination lines.  Because
they have been detected in the optical, these two flows can be classified
as Herbig-Haro objects.


% \begin{figure*}
%   \epsscale{1.0}
%     \plottwo{s233_SII_on_outflows}{s233_halpha_on_outflows}
%     \caption{a. [S II] and b. \ha\ contours overlaid on the \hh\ image.
%     [S II] emission is seen at the bows of outflows 1, 4, and 6 as labeled in figure
%     \ref{fig:outflowsh2}.  \ha\ emission is only seen from Outflow 1 and the
%     emission nebula near IRAS 05358+3543}
%     \label{fig:sii}
% \end{figure*}


\subsection{CO results}

\region\ is located at the center of the CO 3-2 integrated velocity maps
(Figure \ref{fig:cofig}).  The parent molecular cloud, centered at $v_{LSR} =
-17.5$ \kms ,  extends from the southeast towards the northwest with the
brightest emission coming from the core associated with \swcluster, while the
highest integrated emission is associated with \necluster.  \necluster\ has a central
velocity of $\sim -16.0$ \kms\ from the optically thin \ceighteeno\ 2-1 measurements.
Material that has been swept up and accelerated by jets and outflows can be
seen at velocities $v_{LSR} < -21$ \kms and $v_{LSR} > -12$ \kms\ (Figure
\ref{fig:cofig}).  The integrated CO 3-2 map peaks at J(2000) = 05:39:12.8
+35:45:55, while the highest observed brightness temperature is at J(2000) =
5:39:09.4 +35:45:12.  This offset is discussed in the context of CO isotopologues in
section \ref{sec:co21} and in section \ref{sec:discussion-outflows} and shown in
figures \ref{fig:scuba_co21} and \ref{fig:co21map}.

Regions with line wings relative to the ambient cloud within 5\arcmin\ of the
northeast cluster were assumed to be associated with outflows from the cluster.
Further than 5\arcmin, it is likely that the high velocity wings are accelerated by
neighboring HII regions (see section \ref{sec:surroundings}).  These line
wings were integrated over the velocity range -34 to -21 \kms\ (blue) and -12 to
1 \kms\ (red) to acquire estimates of the outflowing mass under the assumption
that outflowing gas is optically thin.  The extracted regions are displayed in
Figure \ref{fig:cofig}b and measurements in table \ref{tab:comeas}.  The line
wings in the central arcminute and central 5 arcminutes were measured for
comparison with lower resolution data and to compute a total outflow mass in
the central region, though they are most prominent in the inner 12\farcs\ (see
Figure \ref{fig:co21_all3}).
.  %This measurement is displayed in Figure
% \ref{fig:s233bco32}.

The objects in Table \ref{tab:comeas} labeled CO Region 1, 2, and 3 have
uncertain associations with outflows.  CO Region 1 is tentatively associated
with outflow 11. CO region 2 may be associated with Outflow 3 but is in a
highly confused region and may have many contributors.  CO region 3 is probably
associated with outflow 10.  In contrast, the associations with outflows 4 and
6/7 are more certain because they are further from the central region and less
confused.   Outflow 1 is seen at high velocities in \citet{beuther2002}
interferometer maps.  Outflow 9 is selected primarily based on CO emission.


\Figure{figures_ch05358/CO3-2_ellipses}
{
The JCMT HARP CO J=3-2 map integrated over all velocities with significant
emission (-34 \kms\ to -4 \kms) shown in gray log scale from 0 to 150 K \kms.
The elliptical regions over which line wings were integrated are shown with
blue and red circles corresponding to blue and red line wings.  The
measurements are presented in table \ref{tab:comeas}.}{fig:cofig}{1.0}{}



\Figure{figures_ch05358/scuba_with_CO21contours}{SCUBA 850\um\ image in linear grayscale from
-1 to +10 mJy/beam, with a saturated peak of 24 mJy/beam, with \twelveco\ 2-1
(orange solid, contours at 45,60,85,100,115,130,145 K \kms) and \thirteenco\
2-1 (green dashed, contours at 20,30,40,45 K \kms) integrated contours.  The
box shows the region plotted in Figure \ref{fig:co21map}.}{fig:scuba_co21}{1.0}{}

\Figure{figures_ch05358/co21_plotmap}{CO spectra of \necluster\ in \twelveco\ (blue),
\thirteenco\ (green), and \ceighteeno\ (red).  The top-left plot is the pixel
centered at J(2000) = 5:39:13.67 +35:46:26.0 and each pixel is 10\arcsec\ on a side. 
The region mapped here is shown with a box in Figure \ref{fig:scuba_co21}.
Redshifted self-absorption, a possible infall tracer, is evident in the \twelveco\
spectra in the outer pixels.  The inner pixels show self-absorption only at
central velocities: this may be an indication that emission from outflows
dominates any infall signature, or simply that there is no bulk infall towards
\necluster.}{fig:co21map}{0.75}{}

\Figure{figures_ch05358/co21_all3_12as}{CO spectra of inner
12\arcsec\ centered on \necluster\ for all observed CO lines.   The CO 3-2 and
2-1 beams are not matched, but in both cases the area integrated over is 1-2
resolution elements across.  The divisions demarcating the red and blue line
wings are shown with vertical dashed lines at $v_{LSR}=-21$ and -12 \kms.
}{fig:co21_all3}{1.0}{}
{}

\begin{deluxetable}{lccccc}
\tablewidth{0pt}
  \tabletypesize{\footnotesize}
    \centering
    \tablecaption{Measured properties of CO flows
      \label{tab:comeas}
    }
\tablehead{ \colhead{\tablenotemark{a}Region Name}  & \colhead{$\int T_{mb}*$}  & \colhead{$M (M_\odot)$}  & \colhead{p ($M_\odot$ \kms)}  & \colhead{N (\persc)} & \colhead{E ($10^{42}$ erg)}}
% & \colhead{redintegrated} & \colhead{redmass} & \colhead{redmomentum} & \colhead{redenergy} } 
\startdata
% blue1 - keep, blue only: associated with outflow 4
\tablenotemark{b}A. Outflow 4a    &      4.27  &     .022  &     .15  &  1.4\ee{19}  & 11 \\
% blue 4 - keep, associated with outflow 4, blue only
\tablenotemark{b}B. Outflow 4b   &      4.60 & .032 & .21 &   1.5\ee{19}  & 13 \\
% blue 9 - keep, blue only, probably outflow 1
\tablenotemark{b}C. Outflow 1n   &      14.5 & .088 & .71 & 4.8\ee{19} & 66 \\ 
% blue 2 - keep, associated with outflow 6/7, blue only
\tablenotemark{b}D. Outflow 6/7  &       4.45 & .045 & .30 & 1.5\ee{19} & 29 \\
%red7 - keep, no association but same direction as Outflow 2
\tablenotemark{r}E. CO Region 3      &    1.31 & .016 & .112 & 4.3\ee{18} & 8.5 \\
% blue3 - remove, not associated with cluster
% blue3     &       9.525039  &     0.659057  &      3.36178  &  2.12468e+44   &     5.01441  &     0.346957   &     1.80229  &  1.01071e+44  \\
% blue 6 - keep, red and blue, use for central outflow
\tablenotemark{b}F. \necluster\ &      41.8 & .464 & 3.72 & 1.4\ee{20} & 330 \\
\tablenotemark{m}F. \necluster\ &      132.9& 1.47 & -    & 4.4\ee{20} & -\\
\tablenotemark{r}F. \necluster\ &      30.0 & .333 & 2.03 & 9.9\ee{19} & 135 \\
% blue 7 - keep, may be outflow 1 south counterflow
\tablenotemark{b}G. Outflow1s   &      14.6 & .064 & .48 & 4.8\ee{19} & 40 \\
%red5 - clear red excess, no clear association
\tablenotemark{r}H. CO Region 2      &    4.54 & .012 & .074 & 1.5\ee{19} & 5 \\
% blue 5 - keep, associated with outflow 9, blue only
\tablenotemark{b}I. Outflow 9    &      6.33 & .039 & .39 & 2.1\ee{19} & 43\\
% blue 8 - keep, blue only, no clear association
\tablenotemark{b}J. CO Region 1  &     3.61 & .015 & .12 & 1.2\ee{19} & 11\\
\tablenotemark{r}K. Red S   &     5.26 & .051 & .34  & 1.7\ee{19} & 26 \\
\tablenotemark{b}L. Blue S  &     3.66 & .053 & .47  & 1.2\ee{19} & 47 \\
% blue 10 - get rid of it
%blue10    &       30.399808 &      0.669383 &       4.35128 &   3.40323e+44  &      46.7147 &       1.02863  &       6.0041 &   3.86866e+44  \\
% blue 11 - remove, no clear association, blue only
%blue11    &        2.641900 &     0.0865521 &      0.438263 &   3.18589e+43  &     0.984461 &     0.0322523  &     0.233822 &   1.88811e+43  \\
% red 1 - south red, keep
% actually, ignore - don't discuss these South Source 1      &    10.6623   &    0.696333    &    4.21276   &  2.9246e+44  \\
% red 2 - too faint / unclear.  rm
%red2      &      2.822263   &   0.0320534   &    0.221736   &  1.9821e+43    &    9.40664   &    0.106834    &   0.579233   & 3.61028e+43  \\
% red 3 - too ambiguous again, not obviously red
%red3      &      6.267010   &    0.104144   &    0.716924   &  5.6371e+43    &    9.23909   &    0.153533    &   0.733901   & 4.04048e+43  \\
%red4 - rm, already covered
%red4      &     38.240242   &     1.97369   &      11.732   & 8.29297e+44    &    64.5034   &      3.3292    &    20.2909   & 1.37858e+45  \\
%red6 - rm, already covered
%red6      &     16.685583   &     6.29804   &     36.7241   & 2.56084e+45    &    24.2383   &     9.14886    &      53.33   & 3.47148e+45  \\
\tablenotemark{b}1\arcmin\ aperture\tablenotemark{c}& 15.1 & .96 & 7.6 & 5.0\ee{19} & 670\\
\tablenotemark{b}3\arcmin\ aperture& 2.7  & 1.6 & 12 & 9.0\ee{18} & 1000\\
\tablenotemark{b}5\arcmin\ aperture& 1.7  & 2.7 & 20 & 5.6\ee{18} & 1600 \\
\tablenotemark{r}1\arcmin\ aperture & 11.8 & 0.75 & 4.7 & 3.9\ee{19} & 320\\
\tablenotemark{r}3\arcmin\ aperture & 1.9 & 1.1 & 6.8 & 6.2\ee{18} & 460\\
\tablenotemark{r}5\arcmin\ aperture & 0.96 & 1.5 & 10 & 3.2\ee{18} & 640\\
\tablenotemark{b}1\arcmin\ $^{12}$CO 2-1 & 10.4 & .94 & 7.1 & 4.9\ee{19} & 590 \\
\tablenotemark{m}1\arcmin\ $^{12}$CO 2-1 & 97.78 & 8.83 & - & 4.6\ee{20} & - \\
\tablenotemark{r}1\arcmin\ $^{12}$CO 2-1 & 9.17 & 0.83 & 5.52 & 4.3\ee{19} & 430 \\
\tablenotemark{m}1\arcmin\ $^{13}$CO 2-1 & 41.12 & 211  & - & 1.1\ee{22} & - \\
\tablenotemark{m}1\arcmin\ C$^{18}$O 2-1 & 5.31 & 271 & - & 1.4\ee{22} & - \\
\enddata
\tablenotetext{a}{Unless labeled otherwise, regions are extracted from CO 3-2 data as shown in figure \ref{fig:cofig}b }
\tablenotetext{b}{Blue integration over velocity range -34 to -21 \kms}
\tablenotetext{c}{Apertures are centered on J(2000) = 05:39:11.238 +35:45:41.80 in \necluster}
\tablenotetext{r}{Red integration over velocity range -13 to -4 \kms}
\tablenotetext{m}{Middle range integration over -21 \kms\ to -13 \kms.  Assumed not to be outflowing, so no momentum is computed}
\end{deluxetable}


\subsection{Near-infrared spectroscopy: Velocities}
\label{sec:tspecresults}

The slit positions used and apertures extracted from those slits are displayed
in Figure \ref{fig:tspecslits}.  Position-velocity diagrams of the 1-0 S(1)
line are displayed in Figure \ref{fig:outflows_h2_pv}.  Velocity measurements
are presented in Table \ref{tab:OutflowH2} and derived properties in Table
\ref{tab:table4}.
% and the velocity fits to the 1-0 S(1) line are shown in figure \ref{fig:h2fits}.

\Figure{figures_ch05358/tspec_slits_apertures_on_h2}{TripleSpec slits (blue) overlaid on the
\hh\ image.  The red boxes indicate the apertures extracted from those slits to
fit and measure \hh\ properties.  The apertures are also indicated in the
position-velocity diagrams.}{fig:tspecslits}{0.75}{}

The near-IR spectrum of Outflow 1 has the largest signal.  All of the K-band
\hh\ lines except the 2-1 S(0) 2.3556 \um\ (too weak) and 1-0 S(4) 1.8920 \um
(poor atmospheric transmission) lines were detected (see Table
\ref{tab:nirmeas}).  Velocities from gaussian fits to each line are reported.
In the central portion of \necluster, outflowing \hh\ emission at
$v_{LSR}\approx-30$ \kms\ is detected.  This material may be associated  with a
line-of-sight flow, or may originate from the base of the already identified
flows 1-3.  In source IR 58,  Br$\gamma$ and He I 2.05835\um are detected,
indicating that there is an embedded PDR in this source.  There is a hint of a
second, fainter star adjacent to IR 58.  IR 93 is observed to be a double
source in the TripleSpec spectrum, but the spectrum is too weak to do any
identification.  Br$\gamma$ and possibly He I are detected at fainter levels.

% 2.06036 HeI?
% 2.16593 BrG

Table \ref{tab:nirmeas} shows the measured line strengths (when detected) of
all \hh\ lines in each aperture.  The errors listed are statistical errors
that do not include the systematics errors introduced by a failure to correct
for narrow atmospheric absorption lines.

\Figure{figures_ch05358/fivepvdiagrams}{Position-velocity diagrams of the \hh\ 2.1218 \um\ line in Outflows
1,2, 4, IR 6, and IR93/IR58.  The velocity range is from -340 to 190 \kms.}{fig:outflows_h2_pv}{1.0}{}
  
\Table{cccccc}
{TripleSpec fitted \htwo\ outflow velocities}
{Outflow Number & Aperture Number &  \tablenotemark{a}v$_{LSR}$ (\kms) &\tablenotemark{b}v$_{LSR} (\kms)$ }
{tab:OutflowH2}
{
1    & 1 & -33.54 (0.15)    & -31.85 (0.32) \\ %-32 (5)  \\
1    & 2 & -13.60 (0.57)    & -13.56 (0.96) \\ %-14 (5)  \\
1    & 3 & -40.51 (0.41)    & -36.13 (0.81) \\ %-36 (5)  \\
1    & 4 & -88.7  (2.8)     & -83.7  (7.9)  \\ %-83 (10) \\
2    & 1 & -82.6  (7.6)     & -81    (21)   \\ %-81 (25) \\ % width 64 +/- 20 - wide!
2    & 2 & -30.41 (0.57)    & -28.9  (1.4)  \\ %-29 (5)  \\
2    & 3 & -33.89 (0.62)    & -35.2  (3.7)  \\ %-35 (10) \\
4    & 1 & -73.34 (0.48)    & -70.2  (1.1)  \\ %-70 (10) \\
4    & 2 & -64.08 (0.61)    & -67.8  (2.2)  \\ %-67 (10) \\
IR6  & 1 & -39.4  (1.6)     & -39.4  (4.2)  \\ %-39 (10) \\
IR93 & 2 & -26.07 (0.43)    & -26.85 (0.97) \\ %-27 (5)  \\  % wider than other lines
IR93 & 3 & -30.6  (1.5)     & -32.0  (2.5)  \\ %-32 (5)  \\  % wider than other lines
IR93 & 4 & -29.14 (0.77)    & -30.3  (2.1)  \\ %-30 (5)  \\   % wider than other lines
IR93 & 6 & -47.7  (7.9)     & -71    (37)   \\ %-45 (15) \\   % manual fit - can't be reproduced easily
}{
\tablenotetext{a}{Measured from \hh\ 1-0 S(1) 2.1218 \um\ line}% as shown in figure \ref{fig:h2fits}}
\tablenotetext{b}{Measured from all detected \hh\ lines fit with model described in section \ref{sec:tspecresults}}
}                



\clearpage
\begin{deluxetable}{ccccccccc}
    \centering
    \tablewidth{0pt}
  \tabletypesize{\scriptsize}
    \tablecaption{Measured properties of \hh\ flows}
%     \tablehead{
% 	\colhead{Outflow} & \colhead{Center}\tablenotemark{a} & \colhead{PA}\tablenotemark{b} 
% 	& \colhead{Length}\tablenotemark{c} & \colhead{Source} \tablenotemark{d} & \colhead{Pos Length} \tablenotemark{e}
% 	& \colhead{Neg Length} \tablenotemark{e} & \colhead{Time (50\kms)}\tablenotemark{f} & \colhead{LOS} \tablenotemark{g}} 
    \tablehead{
    Outflow  & \tablenotemark{a}Center  & \tablenotemark{b}PA  & \tablenotemark{c}Length  & 
    \tablenotemark{d}Source  & \tablenotemark{e}Flow  & \tablenotemark{e}Counterflow  & \tablenotemark{f}Age  & \tablenotemark{g}LOS  \\
    & & & & & Length & Length & (50 \kms) & Velocity \\
    }
\startdata
1 & 05:39:13.023 +35:45:38.66 & -13.3     &142.3"   &mm2?     &58         &84.2        &1.4e4         &-    \\
2 & 05:39:13.058 +35:45:51.28 & -47.0     &44.6"    &mm1a     &44.6       &-           &6.6e3         &Blue \\
3 & 05:39:12.48  +35:45:54.9  & -62       &44"      &mm3?     &44         &-           &6.5e3         &Red  \\
4 & ambiguous                 & 17.8-21.8 &141-144" &  ?      &141-144    &-           &2.1e4         &Blue \\
5 & 05:39:12     +35:45:51    & 170       &38-48    &mm3?     &38-48      &-           &6.5e3         &Blue \\
6 & 05:39:09.7   +35:45:17    & 14.5      &197      &Q5\tablenotemark{h} &197        &-           &2.9e4         &Blue \\
8 & 05:39:10.002 +35:45:10.87 & -154.6    &105.5"   &IR41?    &54.7       &52.9        &7.9e3         &-    \\         
\enddata
\tablenotetext{a}{Midpoint of bipolar outflow if symmetric, position of jet
source candidate if asymmetric}
\tablenotetext{b}{Position angle uncertainties are $\sim 5\arcdeg$ because they
are not perfectly collimated, causing an ambiguity in their true directions.
The exact angles used to draw vectors in figure \ref{fig:outflowsh2} are listed
for reproducibility.}
\tablenotetext{c}{Total length of outflow on the sky, including counterflow}
\tablenotetext{d}{Candidate jet source object.  Outflows 2 and 6 have clear
associations, the others are weaker candidates.}
\tablenotetext{e}{Flow length is the distance from the CENTER position to the
last \hh\ knot in the position angle direction as listed.  Counterflow length
is the distance from the CENTER position to the opposite far knot.}
\tablenotetext{f}{Timescale of jet assuming it is propagating at 50 \kms, an
effective lower limit to see \hh\ emission.  If two lengths are available, uses
the longer of the two.  These are lower limits to the true timescale
\citep{parker1991}.}
\tablenotetext{g}{The parity of the outflow along the line of sight.  Outflow 1
and 8 have counterflows with parities as indicated in figure
\ref{fig:outflowsh2}}
\tablenotetext{h}{\citet{qiu2008} \region source 5. }
\end{deluxetable}



\subsection{Spectroscopic Results: Optical}
\label{sec:dis}

IR 6 and IR 41 (objects 1 and 6 in Figure \ref{fig:outflowsh2}) both show
\ha\ in emission.  IR 41 is close to the reflection nebula in the southeast
portion of \region\ and is probably the reflected star.   The reflection
nebula's spectrum is very similar to IR 41's spectrum at \ha\ in both width
and brightness (see Figure \ref{fig:outflow10_pv}).
%, and \ref{fig:outflow10_spec}).  

\Figure{figures_ch05358/Outflow10_IR6_41_color}{A position-velocity diagram of IR 6 and 41
including the reflection nebula near IR 41 ($\approx 7.4\msun$).  IR 6 shows a
two-peaked \ha\ emission profile, but is the less massive ($\approx 4.5\msun$) of
the pair.  The separation between the two sources is 55\farcs3, and each pixel
is 0\farcs4.}{fig:outflow10_pv}{0.25}{}

% \FigureTwo{outflow10_redap02}{outflow10_redap01}{Red spectra of the two
% stars in the Outflow 10 slit.  Left: IR 6 spectrum.  \ha\ absorption wings
% are evident around the bright \ha\ emission.  Right: IR 41
% spectrum.  \ha\ is clearly detected.  Measurements are presented in
% Table \ref{tab:optical}. }{fig:outflow10_spec}{1.0}

There are three
components in the \ha\ profile of IR 6: a broad absorption feature seen far
($\sim400\kms$ from the line center) on the wings and two emission peaks.  The
peaks are separated by 190 \kms\ and the blueshifted peak is weaker than the
redshifted (Table \ref{tab:IR6}).  The H$\beta$ profile shows much deeper
absorption and weaker emission but with similar characteristics.   The presence of the \ha\ emission
makes identification of the stellar type from the \ha\ line profile uncertain.
The derived extinction to IR 6 is at least $A_V=7$ from an assumed \ha/\hb\
ratio of 2.87 \citep{agnsquared}.  The \hb\ flux was measured from zero to the
peaks of the emission profile and therefore probably overestimates the
\hb\ flux and underestimates the extinction.

Additional lines detected in the outflows, ambient medium, and stars
are presented in Table 8.  The association of [S II] and [O I] emission
with the star forming region is uncertain and may be foreground emission.

\Table{cccccccc}
{IR 6 Deblended Profiles}
{& \tablenotemark{a} Blue & \tablenotemark{b}Blue & \tablenotemark{a} Red & \tablenotemark{b}Red & Absorption &
Gaussian / & \tablenotemark{b}Absorption \\
&Emission & Wavelength & Emission & Wavelength & & Lorentzian FWHM & Wavelength \\
}
{tab:IR6}
{\ha\ & 4.4\ee{-14} & 6559.79 & 1.3\ee{-13} & 6564.23 & -2.6\ee{-14} & 1.5 / 0.19 & 6563.02 \\
\hb & \tablenotemark{c}2.4\ee{-14} & 4857.68 & \tablenotemark{c}1.8\ee{-14} & 
    4864.28 & \tablenotemark{d} -4.6\ee{-14} & 0.17 / 16.5 & 4861.91 \\}
{
\linebreak
\tablecomments{Measurements are made using a Voigt profile fit in the IRAF {\sc splot} task.}
\tablenotetext{a}{Flux measurements are in units of erg s$^{-1}$ \persc \AA$^{-1}$}
\tablenotetext{b}{Wavelengths are in Geocentric coordinates.  Subtract 0.53\AA\ from \ha\
and 0.39\AA\ from \hb\ to put in LSR coordinates.}
\tablenotetext{c}{\hb\ emission was measured assuming a continuum of zero and therefore represents
an upper limit in the \hb\ emission}
\tablenotetext{d}{\hb\ deblending may contain systematic errors
from a guessed subtraction of the \hb\ emission}
}

\Table{cccccccc}
{Lines observed in the optical spectra}
{Source & \ha\ & \hb\ & [S II] 6716\AA & [S II] 6731\AA & [O I] 6300\AA & [O I] 6363\AA &  [O I] 5577\AA }
{tab:optical}
{ 
Outflow1 ap1                   & 4.3\ee{-16} & - & 5.7\ee{-16} & 6.3\ee{-16} & 5.3\ee{-16} & 1.8\ee{-16} & - \\
                               & 6561.49     & - & 6715.3      & 6729.6      & 6299.7      & 6363.3      & - \\
Outflow1 ap2                   & 4.5\ee{-16} & - & 4.5\ee{-16} & 4.6\ee{-16} & 3.1\ee{-16} & 1.2\ee{-16} & - \\
                               & 6561.22     & - & 6714.9      & 6729.3      & 6299.4      & 6363.2      & - \\
Ambient Medium - slit 1        & 6.7\ee{-17} & 5.3\ee{-18}   & 1.0\ee{-17} & 7.9\ee{-18} & 3.5\ee{-16} & 1.2\ee{-16} & 4.8\ee{-17}   \\ % 459 pixels blue
                               & 6562.87     & 4861.7        & 6716.7      & 6731.2      & 6300.3      & 6363.8      & 5578.0  \\ % 178 pixels red 1, 314 red 2
IR 41 nebula                   & 2.6\ee{-15} & - & - & - & 4.4\ee{-16} & 1.9\ee{-16} & - \\
                               & 6562.85     & - & - & - & 6300.3      & 6363.9      & - \\
IR 41                          & 6.5\ee{-15} & - & - & - & 1.1\ee{-16} & 7\ee{-17}   & - \\
                               & 6562.9      & - & - & - & 6300.0      & 6363.3      & - \\
IR 6                           &  1.76\ee{-13} & \tablenotemark{a} 4.1\ee{-14}    &-&-&-&-   &  \\ %2.9\ee{-16} \\
                               & -             & -                                &-&-&-&-   &  \\ %5578.5      \\
}{
\linebreak
\tablecomments{
Wavelengths listed are in \AA\ and are geocentric.  To convert to LSR
velocities, subtract 24.35 \kms.  The ambient medium fluxes represent averages
across the slit.
Fluxes are in erg s$^{-1} \persc $\AA$^{-1}$. 
}
\tablenotetext{a}{\hb\ measurement in IR 6 is an upper limit}
}


% Use temden & OI (6300+6363)/5577 to get temp...
% 5577: 5577.99, 1.07e-14
% 6300: 6300.34, 6.25e-14
% 6363: 6363.81, 2.09e-14
% exp(-ccm_a_opt(.5577)*.47/1.086) = 0.65345832487876254
% exp(-ccm_a_opt(.6363)*.47/1.086) = 0.6926005572989089
% exp(-ccm_a_opt(.6300)*.47/1.086) = 0.68965907033077245
%
% measurements:
% HBeta = 3.57e-15, 4861.32 = -.617 \kms
% HAlpha = 1.19e-14, 6562.87 = 2.74 \kms
% Ratio = 3.33....
% A_halpha = 

% \FigureTwo{outflow1_redap00}{outflow1_redap01}{Outflow 1 spectra.  Left: Front
% Right: Rear.  The slit orientation and position used were the same as the
% TripleSpec slit position, though longer.  No emission was detected from the
% counterflow. [need figure to define apertures?] }{fig:outflow1_spec}{1.0}



\subsection{Radio Interferometry}
\label{sec:vlaresults}
A point source was detected in the X, U, K and Q band VLA maps with high
significance at the same location as the X-band point source reported in
\citet{beuther2007}.  Seven-parameter gaussians were fit to each image to measure
the beam sizes and positions and flux densities.  The measurements are listed
in Table \ref{tab:vla}. The locations of the point source and the shape of the
beams from the re-reduced X and Q band images are displayed in figure
\ref{fig:mm1adiagram}. 
A Class II 6.7 GHz methanol maser was detected in \region\ by \citet{Menten1991}.
It was observed with the European VLBI Network (EVN) by \citet{Minier2000} and 
seen to consist of a linear string of maser spots that trace a probable disk in
addition to maser spots scattered around a line perpendicular to the proposed
disk (see Figure \ref{fig:mm1adiagram}).
The VLA source is more than a VLA beam away from the VLBI CH$_3$OH maser disk
identified by \citet{Minier2000}.  It is to the southeast in the opposite
direction of Outflow 2.  Outflow 2 is at position angle -47$\degree$, while the
disk is at PA 25\degree.  The 8$\degree$ difference from being perpendicular is
well within the error associated with determining the angle of the outflow in
this confused region, so the VLBI disk is a strong candidate for the source of Outflow 2.


\begin{figure*}[htpb]
  \epsscale{0.75}
  \plotone{figures_ch05358/figure_mm1a_pythondiagram}
  \caption{A diagram of the region surrounding mm1a from \citet{beuther2007}.
    The ellipses are centered at the measured source centers and their sizes
    represent the beam sizes of the Plateau de Bure interferometer at 1.2mm
    \citep[blue,][]{beuther2007}, Gemini MICHELLE at 7.9\um\
    \citep[red,][]{Longmore2006}, the VLA at 3.6cm (green), and the VLA at 7mm
    (orange). The maser disk was measured with the European VLBI Network by
    \citet{Minier2000}, so the size and direction of the disk are very well
    constrained.  The black circle is centered on the pointing center of the
    VLBI observation and represents the absolute pointing uncertainty.  The
    arrow pointing in the direction of Outflow 2 traces clumps along the
    outflow back to the mm emission region.  The vector is not to scale -
    Outflow 2 is about 45\arcsec\ long.
\label{fig:mm1adiagram}}
\end{figure*}

The astrometric uncertainty in VLA measurements are typically
$\lesssim0.1$\arcsec. Different epochs of high-resolution X-band and Q-band
data confirmed that the pointing accuracy is substantially better than
0.1\arcsec\ in this case.  The VLBI absolute pointing uncertainty is reported
to have an upper limit of $\sim 0.03$\arcsec\ \citep{Minier2000}.  The
separation between the VLA Q-band center and the VLBI disk center is
0.22\arcsec, whereas the separation between the combined X and Q band pointing
centers is only 0.027\arcsec, which can be viewed as a characteristic
uncertainty. This is evidence for at least two distinct massive stars in a
binary separated by $\sim$ 400 AU.  While the statistical significance of the
binary separation is quite high using formal errors, the systematic errors
cannot be constrained nearly as well.  This object is a candidate binary system
but is not yet confirmed.

%, and a possible third much younger member of the system in the
% hot core mm1a associated with the \citet{Longmore2006} mid-infrared source.
% However, the possibility of a third source is much less certain because the
% \citet{Longmore2006} observations were referenced to the VLA pointing and the
% uncertainty in the mm source location is enough to encompass the VLA source.

\begin{deluxetable}{lllll}
    \tablewidth{0pt}
    \tablecolumns{7}
    \tabletypesize{\footnotesize} %DO NOT include this in the caption!!!
    \tablecaption{VLA measurements near IRAS 05358 mm1a
        \label{tab:vla}
    }
%    \rotate
%\footnotesize
\tablehead{
\colhead{Frequency } & \colhead{Beam major /} & \colhead{RA (error)}   & 
\colhead{Peak flux }  & \colhead{Map RMS }   \\
\colhead{Observed} & \colhead{minor / PA} & \colhead{Dec (error)} &  \colhead{(error)} & \colhead{(mJy/beam)} \\ }
 % \colhead{Deconvolved source major / minor / PA } &   % all sources are unresolved...
\startdata
43.3 GHz & 0.022\arcsec\  / 0.029\arcsec\ / -10.4 \degree & 05:39:13.065425 (0.000015) &  1.319 (0.027) & 0.179 \\
                                                         &&35:45:51.14732 (0.00031)    &  \\
22.5 GHz &  1.52\arcsec\  / 1.28\arcsec\  / 232 \degree    & 05:39:13.05521 (0.0029)    &  1.26 (0.04)   & 0.091 \\
&&35:45:51.378 (0.046)        &\\
15.0 GHz & 1.58\arcsec\   / 2.00\arcsec\  / 0 \degree      & 05:39:13.062 (0.005)       &  1.274 (0.065) & 0.124 \\
&&35:45:51.4 (0.1)            &\\
8.4 GHz  &  0.107\arcsec\ / 0.122\arcsec\ / 7.9 \degree  & 05:39:13.064548 (0.000036) &  0.506 (0.003) & 0.015 \\
&&35:45:51.170356 (0.000613)  &\\
\enddata
\tablenotetext{.}{Errors reported here are fit errors.  Absolute flux calibration
errors are negligible for the X-band data but are about equivalent to measurement errors
for the K, and U bands and dominant in the Q band}
\end{deluxetable}


\section{Analysis}

\subsection{Near-Infrared Spectroscopic Extinction Measurements} 
Extinction along a line of sight can be calculated using the 1-0 Q(3) / 1-0 S(1) line ratio. 
\begin{equation}
    A_\lambda = 1.09 \left[ -\textrm{ln}
    \frac{S_\nu(S)/S_\nu(Q)}{A_{ul}(S)\lambda_Q/A_{ul}(Q) \lambda_S} \right] \left[
    \left(\frac{\lambda_S}{\lambda_Q}\right)^{-1.8} -1 \right]^{-1}
\end{equation}
Because they are from the same upper state,
their intensity ratio should be set by their Einstein A values times the
relative energies of the transitions.  %We use the lower-signal 1-0 Q(4) / 1-0 S(2) and 1-0 Q(2) /
%1-0 S(0) ratios as `sanity checks' when detected.
However, as shown by
\citet{luhman1998}, narrow atmospheric absorption lines in the long wavelength
portion of the K band, where the Q branch lines lie, can create a significant
bias.  Because the lines have not been corrected for atmospheric absorption,
the Q branch fluxes should actually be lower limits.  Since the 1-0 S(1)
transition at 2.1818 microns is affected very little by atmospheric absorption,
and the exinction measured is proportional to the Q/S line ratio, the measured
extinction should be a lower limit.

%Interesting but requires more work
The [Fe II] 1.6435 and 1.2567 \um\ lines were detected in Outflow 1, allowing for
another direct measurement of the extinction.  The measured ratio $FR$ =
1.26\um/1.64\um\ in Outflow 1 was .8, while the true value is at least 1.24 but
may be as high as 1.49 \citep{Smith2006,luhman1998,Giannini2008}.  The
extinction measured from this ratio ranges from $A_V = 4.1$ ($FR=1.24$) to 5.8
($FR=1.49$).  The S(1)/Q(3) ratio uncorrected for telluric absorption is .91,
which yields an extinction lower limit of $A_V = 18.7$, is inconsistent with
the measurement from [Fe II].  The \ha\ detection and \hb\ upper limit give a
lower limit on the extinction of $A_V = 6.6$, which is consistent with both of
the other methods to within the calibration uncertainty.  

It is possible that the two measurements come from unresolved regions with
different levels of extinction, though a factor of at least 3 change over an
area $\sim 100 AU$ far from the millimeter cores seems unlikely.  A strong IR
radiation field could plausibly change the line ratio from the expected
Einstein A value.  The question is not resolved but may be possible to address
with near-IR observations of nearby bright HH flows with more careful
atmospheric calibration.


\subsection{Optical Spectra}
\subsubsection{Stellar Type}
IR 6 is suspected to be the source of the bright \hh\ finger at PA $\approx$
15\arcdeg.  %It is also an \ha\ emission source, having previously been
%identified in the IPHAS \ha\ survey \citep{Witham2008}.  
IR 6 is also
a 24\um\ source and was detected by MSX (designation G173.4956+02.4218).  We
identify this star as a Herbig Ae/Be star. 

\subsubsection{Density and Extinction Measurements}
The spectrum of knot N1 (the bow of Outflow 1) allowed a measurement of
electron density in the shocks from the [S II] 6716/6731  line ratio
%(figure \ref{fig:outflow1_spec})
.  Densities were determined to be $n=$ 700 \percc\  in the forward lump
and $n=$500 \percc\ in the second lump.  \ha, [N II] 6583, [O I] 6300, and [O
III] 6363 were also detected, but no lines were detected in the blue portion of
the spectrum presumably because of extinction.  The measured velocities from [S
II] are faster than the \hh\ velocity measurements at about $v_{LSR} = -68
\pm 5$ \kms. 

There is also an ambient ionized medium that uniformly fills the slit with a
[S II]-measured density $n_e=120\ \percc$.  Evidently, nearby massive stars are
ionizing the low-density ISM located in front of \region.   This material is
moving at velocity $v_{LSR} = -7 \pm 5$ \kms\ and is extincted by $A_V=1.5$ as
determined from \ha/H$_\beta=2.87$  assuming the gas is at 10$^4$ K.



\subsection{UCHII region measurement}
A uniform density, ideal HII region will have an intensity curve $I = I_0
( 1 - e^{-\tau_\nu})$ where 
\begin{equation} \tau = 8.235\times10^{-2}
\left(\frac{T_e}{K}\right)^{-1.35} \left(\frac{\nu}{\textrm{GHz}}\right)^{-2.1}
\left(\frac{\textrm{EM}}{\textrm{pc~cm}^{-6}}\right) a(\nu,T)
\end{equation}
following \citet{rohlfs2004} equation 9.35, where $a(\nu,T) \approx 1$ is a
correction factor.   By assuming an excitation temperature $T_{ex} = 8500$K, 
blackbody with a turnover to an optically thin thermal source was fit to the
centimeter SED.  The turnover frequency from this fit is $\tau=$15.5 GHz,
corresponding to an  emission measure $EM=7.4 \times 10^8$ pc cm$^{-6}$.  This
turnover frequency is lower than the $\sim35$ GHz reported by
\citet{beuther2007}.  The turnover is clearly visible in the U, K and Q data
points in figure \ref{fig:HIIregionfit}.  

By assuming the X-band emission is optically thick, a source size can be derived.
\begin{equation}
    \label{eqn:uchiirad}
    2 r = \left[\frac{S_\nu}{2 k_B T_{ex}}  \lambda^2  D^2\right]^{1/2}
\end{equation}
where D is the distance to the source.  Assuming a spherical UCHII region and a
distance of 1.8 kpc, the source has radius $r=$30 AU (for comparison, the
Q band beam minor axis is $\sim$90 AU, so the region could in principle be
resolved by the VLA + Pie Town configuration).  

The measured density is $n=(EM / r)^{1/2} = 2.2\times10^6$ \percc, with a
corresponding emitting mass $M = n \mu m_H 4/3 \pi r^3 = 1.0\times 10^{-6}$
\msun\ using $\mu=1.4$.  Using \citet{kurtz1994} equation 1,

\begin{equation}
  N_{Lyc} = (8.04\times10^{46} s^{-1}) T_e^{-0.85} \left(\frac{r}{pc} \right)^3 n_e^2
\end{equation}

the number of Lyman continuum photons per second required to maintain
ionization is estimated to be  $N_{Lyc} = 5.9\times10^{44}$ 
%\footnote{This equation assumes an optically thin HII region, but the source is optically
%thick at 3.6cm}
, a factor of $\sim4$ lower than measured by \citet{beuther2007} and
closer to a B2 ZAMS star ($\sim11\msun$) than B1 using Table 2 of
\citet{panagia1973}.  If the star has not yet reached the main sequence, it
could be significantly more massive \citep{hosokawa2009}, so our stellar mass
estimate is a lower limit.

% 2*6.67e-8*(11*2e33)/(1.5*1.38e-16*8500/1.67e-24) / 1.496e13
The gravitational binding radius of a 11 \msun\ star is $r_g = 2 G M / c_s^2 \approx
190$ AU (the HII region is assumed to be supported entirely by thermal
pressure, which provides an upper limit on the binding radius since turbulent
pressure can exceed thermal pressure).  The UCHII region radius of 30 AU is
much smaller, indicating that, under the assumption of spherical symmetry, the
HII region is bound.

\citet{leurini2007} noted that the CH$_3$CN line profile around this source
could be fit with a binary system with separation $<1100$AU and a total mass
of 7-22 \msun .  This is entirely consistent with our picture of a massive binary
system with a 11 \msun\ star in a UCHII region and another high mass star with
a maser disk.

There are no other sources in the \region\ region to a 5$\sigma$ limit of 0.075
mJy in the X-band, which provides the strictest upper limit.  From equation
\ref{eqn:uchiirad}, this corresponds to an optically thick source size of 24
AU.  The maser disk has a spatial extent of around 140 AU, so it is quite unlikely
that either an undetected UCHII region or the observed UCHII are associated with
the maser disk.

Assuming the same turnover point for undetected sources,  
an upper limit is set on $N_{Lyc}$ for undetected sources:
\begin{equation}
  N_{Lyc} = (8.04\times10^{46} s^{-1}) \left(\frac{S_\nu }{  2 k_B T_{ex}^{1.85} }  \lambda_{cm}^2  D_{pc}^2\right) EM
\end{equation}
Our 5$\sigma$ upper limit is $N_{Lyc} = 1.38\ee{44}$ s$^{-1}$, indicating that
any stars present must be a later class than B3, or lower than about 8 \msun .  
For an emission measure as much as 3 times higher, the corresponding stellar
mass would be less than 10 \msun .  It is likely that no other massive
stars have formed in \necluster.

After independently determining the best-fit UCHII model to the VLA data, we
included the PdBI data points from \citet{beuther2007} and fit a power-law to
both data sets.  If the emission measure was allowed to vary, the derived
parameters were $EM=6.3\ee{8}$ and $\beta=0.7$.  However, doing this visibly
worsened the UCHII region fit without significantly improving the power-law
fit, so the fit was repeated holding a fixed emission measure, yielding
$\beta=0.8$ (plotted in Figure \ref{fig:HIIregionfit}b).  This power-law is
much shallower than the $\beta=1.6$ measured by \citet{beuther2007} without
access to the 44 GHz data point, and suggests that there is a significant
population of large grains in source mm1a.  However, we caution that the fits
were performed only accounting for statistical errors, not the significant and
unknown systematic errors that are likely to be present in mm interferometric
data.  The PdBI beams are much larger than the VLA beams, so the larger beams
could be systematically shifted up by including additional emission, which
would reduce $\beta$.  Nonetheless, the new VLA data constrains the UCHII
emission to contribute no more than 10\% of the 3.1mm flux.

\begin{figure*}[htpb]
\epsscale{0.75}
\plotone{figures_ch05358/HIIregion_fit}
\plotone{figures_ch05358/HIIregion_plusdust_fit}
\caption{(a) The HII region fit to measured X, K, U, and Q band data.  Error bars
represent statistical error in the flux measurement.  The Q band error is
dominated by flux calibration uncertainty (see Table \ref{tab:vlatimes}).  The
measured turnover is at 9.5 GHz. (b) A fit to both the VLA data presented in this paper
and the (sub)mm points from \citet{beuther2007}.  The best fit spectral index for the 
dust emission is $\alpha=2.8$ ($\beta=0.8$), significantly lower than the $\alpha=3.6$
measured by \citet{beuther2007} without the 0.7 mm data point.}
\label{fig:HIIregionfit}
\end{figure*}


\subsection{Mass, Energy, and Momentum estimates from CO}

\subsubsection{Equations}
The column density for CO J=3-2 is estimated using the equation 
\begin{equation}
  \label{eqn:column}
N_{\hh} =
\frac{\hh}{\textrm{CO}}\frac{8\pi\nu^3k_B}{3c^3hB_eA_{ul}}(1-e^{h\nu/k_BT_{ex}})^{-1}
\frac{1}{\eta_{mb}} \int T_A^*(v) dv 
\end{equation}
where $A_{ul}=A_{32}=2.5\times10^{-6}\textrm{s}^{-1}$ and
$A_{21}=6.9\ee{-7}$s$^{-1}$\citep{turner1977}, the rotational constant $B_e =
57.64$, 55.10, and 55.89  GHz for \twelveco, \thirteenco, and \ceighteeno\ respectively,
$\eta_{mb} = .68$, and $T_{ex}$ is assumed to be 20K.  The partition function
is approximated as 
\begin{equation}
    Z=\sum_{J=1}^\infty (2J+1) exp
  \left(\frac{-J(J+1)hB_e}{k_B T_{ex}}\right) \approx \int_0^\infty (2J+1)exp
  \left(\frac{-J(J+1)hB}{k_bT_{ex}}\right) dJ
\end{equation}
which is valid when $T_{ex} >> hB_e/k_B \sim 2.8 $K.
Equation \ref{eqn:column} becomes 
\begin{equation} 
  N_{\hh} = ( 3.27\times10^{18} \persc) \frac{1}{\eta_{mb}} \int T_A^*(v) dv
\end{equation}
where the integrand is in units K \kms.  The mass is then 
\begin{equation}
  M = \mu\ m_{\hh}\ A\ N_{\hh} = 1.42\times10^{-5} A \frac{1}{\eta_{mb}} \int
  T_A^*(v) dv
\end{equation} 
where A is the area in cm$^2$, $\mu=1.4$ is a constant to account for the presence of
helium, and again velocity is in \kms.

\subsubsection{CO J = 2-1 Isotopologue Comparison}
\label{sec:co21}

\citet{Thomas2008} observed C$^{17}$O in the J=2-1 and 3-2 transitions each with a
single pointing using the JCMT centered at J(2000) = 05:39:10.8 +35:45:16 and measured a
column density $N_{\hh}=4.03\times10^{22}$ cm$^{-2}$.  The peak column
density is $1.7\ee{22}$\persc\ in \thirteenco\ and $2.2\ee{22}$ in \ceighteeno\ at J(2000) = 5:39:10.2
+35:45:26, which is reasonably consistent with the C$^{17}$O measurement
considering abundance uncertainties.  The peaks of the integrated
spectra for \ceighteeno\ and \thirteenco\ are coincident, but the \twelveco\
integrated peak is at J(2000) = 5:39:12.6 +35:45:46 (Figure \ref{fig:scuba_co21},
discussed more in section \ref{sec:discussion-outflows}).

Measurements of the column density, mass, momentum, and energy are performed as
in Equation \ref{eqn:column}.  Assuming a \twelveco/\thirteenco\ ratio of 60
\citep{lucas1998} and optically thin \thirteenco, the mean column density
across the region is $N_{\hh} = 4.5\times10^{21}$ \persc.  The resulting total
mass of the central $\sim3$\arcmin\ is about 320 \msun, which is substantially
smaller than the 600 \msun\ measured by \citet{beuther2002} and
\citet{Zinchenko1997}, but it is nearly consistent with 870\um\ and NH$_3$
estimates of 450 and 400 \msun\ from \citet{Mao2004} and is within the
systematic uncertainties of these measurements.  Assuming \ceighteeno\ is
optically thin and the \ceighteeno/\thirteenco\ ratio is 10, the 
column density is 5.2\ee{21} \persc\ and the mass is 360 \msun, which is
consistent with the \thirteenco\ measurements, indicating that optical depth
effects are probably not responsible for the discrepancy with the dust mass
estimate.

% We also measure a total mass in a 1 \arcmin\ radius of \necluster\ of 210
% \msun\ (\thirteenco) and 270 \msun (\ceighteeno), which is consistent with
% the SCUBA measurements of \citet{williams2004}.

% Presumably the 1.2mm and NH$_3$ observations were
% sensitive only to the dense gas, while the 13CO traces parts of the cloud that
% are still supported by turbulent pressure and are therefore at lower densities.

\subsubsection{CO Mass and Energy Measurements for Specific Outflows}

% \citet{beuther2002} measure a total mass of 610 \msun\ and an outflow mass of 
% 20 \msun .  Our CO measurements of the same region give a mass estimate of 4 (CO 3-2) or 2 (CO 2-1)
% \msun, in poor agreement with their results.  See section \ref{sec:discussion-outflows}.

Table \ref{tab:comeas} lists measurements of mass and momentum in
apertures shown in figure \ref{fig:cofig}.  Where red and blue masses are
listed, there is an outflow in the red and blue along the line of sight.  Where
only one is listed, an excess to one side of the cloud rest velocity was
detected and assumed to be accelerated gas from a protostellar outflow.  Blue
velocities are integrated from -33 to -21 \kms.  Red velocities are integrated
from -12 \kms\ to 1 \kms.  All masses are computed assuming CO is optically
thin in the outflow, which leads to a lower bound on the mass; \thirteenco\ 2-1 was
measured to have an optical depth of 0.1 in 7 very high velocity outflows in
\citet{choi1993}, so if a relative abundance \twelveco/\thirteenco =
60 is assumed\citep{lucas1998}, masses increase by a factor of 6.

It is not possible to completely distinguish outflowing matter from the ambient
medium.  While the outflowing matter is generally at higher velocities, the
outflow and ambient line profiles are blended.  A uniform selection of high
velocities was applied across the region but this may include some matter from
the cloud, biasing the mass measurements upward.  Outflows in the plane of the
sky and low-velocity components of outflows will be blended with the cloud
profile, which would lead to underestimates of the outflowing mass.  The
momentum measurements, however, should be more robust because they are weighted
by velocity, and higher velocity material is more certainly outflowing.  The
momentum measurements are referenced to the central velocity of \necluster,
$v_{LSR}=-16.0$ \kms.



\section{Discussion}

\subsection{Outflow Mass and Momentum}
\label{sec:discussion-outflows}
\citet{beuther2002} reported a total outflowing mass of 20 \msun\ in \necluster.  We
measure a significantly lower outflow mass of 2 \msun\ under the assumption
that the gas is optically thin, but this assumption is not valid: a lower limit
can be set from the weak \thirteenco\ 2-1 outflow detection  (lower limit
because not all of the outflowing material is detected) on the outflowing mass
of $\sim 4$ \msun .  \citet{choi1993} measure an optical depth of  \thirteenco\
2-1 $\tau \approx 0.1$ in 7 very high velocity outflows.  
%If \region\
%is assumed to be typical of their sample and the solar abundance of
%\thirteenco\ (\twelveco/\thirteenco=89) is used, our outflow mass is about
%18\msun (the solar abundance is used in order to compare directly to \citet{beuther2002},
%who follow the method of \citet{Cabrit1990}, and to set an upper limit on the mass).  
Our \thirteenco\ data suggests that the optical depth is somewhat lower,
 $\tau\approx0.07$. The abundance \twelveco/\thirteenco = 60 is used  \citep{lucas1998}  
to derive a total outflowing mass estimate $M\approx25$ \msun .  The total outflowing mass is
therefore $\sim 4\%$ of the total cloud mass, though most
of the outflowing material is coming from \necluster, so as much as 13\% of the
material in \necluster\ may be outflowing.

The most prominent outflow in \region, Outflow 1, is primarily along
the plane of the sky, so the high velocity CO is likely associated
with the other outflows that have significant components along the line of
sight.  As pointed out in \citet{beuther2002}, the integrated and peak CO are 
aligned with the main mm core.  High-velocity \hh\ near the mm cores and the
blueshifted outflows 2 and 4 all suggest that there are many distinct outflows
that together are responsible for the high velocity CO gas.

The offset between the integrated \thirteenco\ peak and \twelveco\ peak in the 
J=2-1 integrated maps, which corresponds with an offset in the peak of the
integrated CO 3-2 map and the peak temperature observed in CO 3-2, suggests
that the gas mass is largely associated with \swcluster, but the outflowing gas
is primarily associated with \necluster.  The integrated and maximum brightness
temperatures in \thirteenco\ and \ceighteeno\ are also centered near \swcluster, 
which rules out optical depth as the cause of this offset.  CO may be depleted
in the dense mm cores, which would help account for the lower mass estimate
from CO isotopologues relative to dust mass and NH$_3$.  Alternately, the gas
temperature in \swcluster\ may be significantly higher than in \necluster\ 
except in the outflows, which are probably warm.  In this case, the outflowing
\twelveco\ enhances the integrated intensity because of its high temperature
and reduced effective optical depth, but it does not set the peak brightness
because of the low filling-factor of the high-temperature gas.

Because the outflows are seen in \hh, which requires shock velocities $\sim30$
\kms\ to be excited \citep{bally2007}, and because the association between the
high-velocity CO and the plane-of-the-sky \hh\ is unclear,  a velocity of 30
\kms\ is used when estimating the dynamical age.
Assuming the outflow is about 0.5 pc long in one direction (e.g. Outflow 1), 
the dynamical age is 1.6\ee{4} years. Outflow 4, which is around 1 pc
long, is also seen at a velocity of -70 \kms\ LSR, or about -50 \kms\ with
respect to the cloud, and therefore has a dynamic age 2\ee{4} years, which is
consistent.

\subsection{Energy Injection / Ejection}
Using an assumed outflow lifetime of $5\times10^3$ years for $v=100\ \kms$ as a
lower limit (because the full extent of the flows is not necessarily observed)
and $1\times10^5$ as an upper limit (for the CO velocities $\sim10\ \kms$ and
the longest $\sim1$ pc flows), mechanical luminosities of the outflows $L=E/t$
are derived.  The summed mechanical luminosity of the outflows is compared to the
turbulent decay luminosity within a 12\arcsec, 1\arcmin, and 5\arcmin\ radius
centered on \necluster\ in Table \ref{tab:turb}.  \setlength\tabcolsep{3pt}  

\Table{ccccccc}{Comparison of turbulent decay and outflow injection}
{Radius (pc) & $t_{turb}$\tablenotemark{a} (yr) & $L_{turb} (\lsun)$ & $L_{outflows}$\tablenotemark{b} $(\lsun) $
& Binding Energy (ergs) \tablenotemark{c} & Outflow Energy (ergs) & Turbulent Energy (ergs) \tablenotemark{d}}
{tab:turb}
{
0.10  & 2\ee{4} &  20  & 0.03-0.6    & 3.4 \ee{46} & 3.5 \ee{44} & 5.0\ee{46}\\
0.52  & 1\ee{5} &  12  & 0.6 - 9.4   & 5.9 \ee{46} & 5.9 \ee{45} & 1.5\ee{47}\\
2.62  & 5\ee{5} &  2.3 & 1-22        & 1.2 \ee{46} & 1.4 \ee{46} & 1.5\ee{47}\\
}
{
\tablenotetext{a}{Masses are assumed to be 600 \msun\ for the 1\arcmin\ and 5\arcmin\ 
apertures, and 200\msun\ for the 12\arcsec\ aperture.}
\tablenotetext{b}{Outflow luminosities are given as a range with a lower limit
$L=E_{out} / 10^5 \textrm{yr}$ and upper limit $L=E_{out}/ 5\times10^3 \textrm{yr}$, 
where $E_{out}$ is from Table \ref{tab:comeas} multiplied by 6 to correct
for outflow opacity.  }
\tablenotetext{c}{Binding energy is the order-of-magnitude estimate GM$^2$/R}
\tablenotetext{d}{Turbulent energy is computed using the measured 5 \kms\ line
width as the turbulent velocity.}
}


The rate of turbulent decay can be estimated from the crossing-time of the
region,  $L / v$, where $L$ is the length scale and $v$ is the the typical
turbulent velocity.  On the largest ($\sim$ few pc) scales, the mechanical
luminosity from high-velocity outflowing material is approximately capable of
balancing turbulent decay and upholding the cloud against collapse.  However,
at the size scales of the \necluster\ clump ($\sim 0.1$ pc), turbulent decay
occurs on more than an order of magnitude faster timescales than outflow energy
injection.  On the smallest scales, outflow energy can be lost from the cluster
through collimated outflows, though wide-angle flows and wrapped up magnetic
fields will not propagate outside of the core region. Once collimated flows
impact the local interstellar medium in a bow shock, their energy and momentum
are distributed more isotropically and again contribute to turbulence.  The
imbalance on a small size scale is consistent with the observed infall
signature (Figure \ref{fig:co21_all3}) in the inner 12\arcsec\ around
\necluster\ and the lack of a similar profile elsewhere.

\subsection{Comparison to other clumps}
The classification scheme laid out in \citet{klein2005} is used to identify
\necluster\ as a Protocluster and \swcluster\ as a Young Cluster.  \citet{maury2009}
performed a similar analysis of the Early Protocluster NGC 2264-C.  They also
found that the outflow mechanical luminosity could provide the majority of the
turbulent energy $L_{turb}\sim1.2 \lsun$ within the protocluster in a radius of
0.7 pc with a mass 2300\msun .  \citet{williams2003} performed an outflow
study of the OMC 2/3 region with radius 1.2 pc and mass 1100 \msun, which is also an Early
Protocluster, and concluded that $L_{turb} \sim L_{flow} \sim 1.3 \lsun$.
While all three regions have nearly the same turbulent decay luminosities and
outflow mechanical luminosities, \necluster\ in \region\ is significantly more
compact and lower mass than the Early Clusters, and is the only one of the three
that contains signatures of massive star formation.  

% \Table{ccccc}{Table of other protoclusters}
% {name & mass & radius & most massive star & luminosity}
% {tab:protoclusters}
% {
% IRAS 05358+3543 & 600  & 0.5 & 12 & 6300 \\
% NGC 2264-C      & 2300 & 0.7 & <8 & 2300 \\ }
% {notemarks}

% In the collapsing region, the
% luminosity from gravitational infall $L =  E_{grav} v/r \sim 8 \lsun$ is 2-3
% orders of magnitude larger than the outflow luminosity in the same area.


% The CO mass measured in the blue wing of the Outflow 4 bow shock probably represents
% a lower limit on the ejected mass from this object.  Assuming a constant mass loss
% rate and that the CO we measure was ejected in 1/5 the dynamical time, we derive
% a mass loss rate of 6\ee{-6} \msun/yr.  

% Protostellar feedback is necessary to halt star formation, and it determines the
% star formation efficiency.  Outflows inject momentum into their parent molecular
% clouds, creating turbulence and holding the cloud up against collapse.
% Although the majority of the outflows seen in \region\ have penetrated the
% cloud and dragged only a small fraction of the total cloud mass with them, the
% central outflow from the forming massive stars is carrying $\sim.3-6\%$ of the
% total mass of the \region\ cloud away at low velocities.  

% In \region, it seems likely that feedback from the massive stars forming in
% mm1a will  halt collapse before outflows destroy the region.  Once enough
% material has accreted onto the UCHII region's star, the HII region will expand
% until it is no longer gravitationally bound and an ionization front will
% proceed outward, pushing gas ahead of it at a few \kms.  Isolated star
% formation in portions of the molecular cloud not associated with the \region\
% clump will continue until the ionization front from the mm1a massive star
% pushes away the gas.  

\subsection{Surrounding Regions}
\label{sec:surroundings}
%We briefly discuss other results apparent in the \ha\ and CO 3-2 images.
%The surrounding regions will be discussed in more detail in Yan et al (2009).

About 8\arcmin\ to the southeast of \region\ is another embedded star forming
region, G173.58+2.45.  Interferometric and stellar population studies have been
performed by \citet{shepherd1996} and \citet{shepherd2002}.  The bipolar
outflow detected in their interferometric maps is also cleanly resolved in our
figure \ref{fig:cofig}.  In our wide-field \hh\ maps, there is a complex of
outflows similar to that of \region, but fainter.

The large HII region Sharpless 231 to the northeast can be seen in the \ha\ 
image (figure \ref{fig:overview_ha}).  The expanding HII region is pushing against
the molecular ridge that includes \region\ and accelerating the CO gas in the
blue direction (e.g. the northern blueshifted clumps in figures \ref{fig:HA_with_CO}
and \ref{fig:cofig}).  It can be seen from the IRAC 8\um\ data 
%(contours in figure \ref{fig:HA_with_8um}) 
that UV radiation from the HII region reaches to
the \region\ clusters.  The expanding HII region's pressure on the molecular
ridge may be responsible for triggering the collapse of \region\ and G173.  The
size gradient from S232 ($\sim 30\arcmin$\ across) to S231 ($\sim 10\arcmin$)
to S233 ($\sim 2-3\arcmin$) is suggestive of an age gradient assuming uniform
HII region expansion velocities and a common distance.  Investigation of this
hypothesis will require detailed stellar population studies in the HII regions
with proper regard for eliminating foreground and background sources.

\subsection{Massive Star Binary}
Our identification of a probable massive star binary with an associated outflow
contributes to a very small sample of known maser disks with \hh\ emission
perpendicular to the disk.  \citet{debuizer2003} observed 28 methanol maser
sources with linear distributions of maser spots in the \hh\ 2.12 \um\ line,
and he identified only 2 sources with \hh\ emission perpendicular to the maser
lines.  None of the outflows identified in his survey were as collimated as
Outflow 2, so the methanol disk / outflow combination presented here may be the
most convincing association of a massive protostellar disk with a collimated
outflow.

The association of a massive star with an UCHII region and a methanol maser
disk and the very small size of the UCHII region both suggest that the massive
stellar system is very young.  \citet{walsh1998} suggested that the development
of a UCHII region leads to the destruction of maser emission regions.  Their
conclusion is consistent with our interpretation of mm1a as a binary system. 

\section{Summary \& Conclusion}

We have presented a multiwavelength study of the \region\ star forming region.
\region\ contains an embedded cluster of massive stars and is surrounded by
outflows.  The outflows were linked to probable sources and determined that at
least one outflow is probably associated with a massive ($\sim 10 \msun$) star.
Added kinematic information and a wide field view of the infrared outflows has
been used to develop a more complete picture of the region.

\begin{itemize}
  \item \necluster\ is a Protocluster and \swcluster\ is a Young Cluster
  \item Energy injection on the scales of \region\ can maintain turbulence, but on
    the small scales of the \necluster\ protocluster, is inadequate by $\sim 2$ orders
    of magnitude.  \necluster\ is collapsing.
  \item there are 11 candidate outflows, 7 of which have candidate counterflows, in the 
    \region\ complex
  \item there is a probable massive binary with one member of mass 12 \msun\ in
    mm1a, and the other which is the source of Outflow 2
  \item there are at least two moderate-mass ($\sim$5\msun) young stars in \region\ 
%\item the extinction towards the \hh\ bow shocks is variable, suggesting they are pushing
%  past the molecular cloud
\end{itemize}

We have identified additional middle- and high-mass young stars with outflows, and 
presented a case for a high-mass binary system within the millimeter core mm1a.

% \section{Acknowledgements}
% We thank the anonymous referee for very helpful suggestions particularly
% regarding the mm1a SED analysis.  We thank Vincent Minier for providing us with
% the positions of the VLBI maser spots and Steve Myers and George Moellenbrock
% for their assistance with VLA data reduction.
% 
% We also thank Cara Battersby, Devin Silvia, Mike Shull, and Jeremy Darling for
% helpful comments on early drafts. 
% 
% This work made use of SAOIMAGE DS9 (\url{http://hea-www.harvard.edu/RD/ds9/}),
% IRAF (\url{http://iraf.net/}, scipy (\url{http://www.scipy.org}), and APLpy
% (\url{http://aplpy.sourceforge.net/}).
% 
% J. P. W. thanks the NSF for support through NSF-AST08-08144.


%\bibliographystyle{apj_w_etal}
%\bibliography{iras05358}


\clearpage
\begin{landscape}
\begin{deluxetable}{ccccccccccccc}\setlength\tabcolsep{3pt}
  \scriptsize
  \tabletypesize{\tiny}
  \tablewidth{0pt}
    \centering
    \tablecaption{Measured \hh\ line strengths
      \label{tab:nirmeas}}
    \tablehead{
        &1-0 S(0)&1-0 S(1)&1-0 S(2)&1-0 S(3)&1-0 S(6)&1-0 S(7)&1-0 S(8)&1-0 S(9)&1-0 Q(1)&1-0 Q(2)&1-0 Q(3)&1-0 Q(4)  \\
aperture&2.2233&2.12183&2.03376&1.95756&1.78795&1.74803&1.71466&1.68772&2.40659&2.41344&2.42373&2.43749}
\startdata
outflow1ap1  &   3.60E-15&  9.80E-15&  5.50E-15&  1.20E-14&  4.70E-15&  3.10E-15&  8.60E-16&  1.10E-15&  9.20E-15&  6.10E-15&  1.10E-14&  6.90E-15 \\
             & ( 2.4e-17)&( 3.4e-17)&( 6.8e-17)&(   2e-16)&(   2e-16)&( 2.8e-17)&( 2.8e-17)&( 2.7e-17)&( 1.4e-16)&( 7.4e-17)&( 8.8e-17)&( 7.8e-17) \\
outflow1ap2  &   7.10E-16&  1.80E-15&  9.90E-16&  1.80E-15&         -&         -&         -&         -&  3.00E-15&  1.90E-15&  3.20E-15&  2.00E-15 \\
             & ( 2.1e-17)&( 2.7e-17)&( 6.8e-17)&( 1.7e-16)&         -&         -&         -&         -&( 1.3e-16)&(   4e-17)&( 7.8e-17)&( 3.8e-17) \\
outflow1ap3  &   1.60E-15&  4.10E-15&  2.20E-15&  4.70E-15&         -&  8.30E-16&         -&         -&  5.60E-15&  3.70E-15&  6.60E-15&  4.80E-15 \\
             & ( 2.4e-17)&( 3.4e-17)&( 6.3e-17)&( 1.8e-16)&         -&( 2.8e-17)&         -&         -&( 1.4e-16)&( 5.9e-17)&( 8.2e-17)&( 5.2e-17) \\
outflow1ap4  &          -&  9.00E-16&         -&         -&         -&         -&         -&         -&         -&         -&         -&         - \\
             &          -&(   3e-17)&         -&         -&         -&         -&         -&         -&         -&         -&         -&         - \\
outflow2ap1  &          -&  3.60E-16&         -&         -&         -&         -&         -&         -&         -&         -&         -&         - \\
             &          -&( 1.5e-17)&         -&         -&         -&         -&         -&         -&         -&         -&         -&         - \\
outflow2ap2  &   9.40E-16&  2.40E-15&  1.50E-15&  1.80E-15&         -&  4.00E-16&         -&         -&  3.00E-15&         -&  3.70E-15&         - \\
             & ( 1.7e-17)&( 2.2e-17)&( 5.8e-17)&( 1.1e-16)&         -&( 2.3e-17)&         -&         -&( 4.7e-17)&         -&( 7.9e-17)&         - \\
outflow2ap3  &   2.10E-15&  1.90E-15&  1.80E-15&  2.20E-15&         -&  6.70E-16&         -&         -&  5.70E-15&         -&  7.30E-15&         - \\
             & ( 1.7e-17)&( 2.2e-17)&( 5.1e-17)&( 1.3e-16)&         -&( 2.9e-17)&         -&         -& -8.00E-16&         -& -8.00E-16&         - \\
outflow4ap1  &   5.50E-16&  2.00E-15&  8.50E-16&  2.00E-15&         -&  9.40E-16&  1.90E-16&  3.40E-16&  1.40E-15&         -&  1.40E-15&         - \\
             & (   2e-17)&(   2e-17)&(   5e-17)&( 1.3e-16)&         -&( 2.8e-17)&( 1.8e-17)&( 2.3e-17)& -4.00E-16&         -&( 6.9e-17)&         - \\
outflow4ap2  &   5.60E-16&  2.00E-15&  5.30E-16&  2.10E-15&         -&  5.80E-16&         -&  1.10E-16&         -&         -&  2.00E-15&         - \\
             & (   2e-17)&( 2.2e-17)&( 2.4e-17)&( 1.2e-16)&         -&( 2.3e-17)&         -&( 1.8e-17)&         -&         -& -2.00E-16&         - \\
     IR6ap1  &          -&  1.10E-15&         -&  9.30E-16&         -&  4.30E-16&         -&         -&         -&         -&         -&         - \\
             &          -&(   3e-17)&         -&( 1.4e-16)&         -&( 3.2e-17)&         -&         -&         -&         -&         -&         - \\
    IR93ap1  &          -&  6.60E-15&         -&  2.70E-15&         -&         -&         -&         -&         -&         -&  5.80E-15&         - \\
             &          -&( 3.5e-17)&         -&(   1e-16)&         -&         -&         -&         -&         -&         -&( 7.4e-17)&         - \\
    IR93ap2  &   4.40E-15&  6.60E-15&  3.90E-15&  3.30E-15&         -&  1.10E-15&         -&         -&  7.60E-15&  5.10E-15&  6.90E-15&  5.50E-15 \\
             & ( 3.2e-17)&( 3.7e-17)&( 9.2e-17)&( 1.4e-16)&         -&( 2.7e-17)&         -&         -&(   8e-17)&( 5.2e-17)&( 7.4e-17)&( 6.1e-17) \\
    IR93ap3  &   1.00E-15&  1.70E-15&         -&  9.00E-16&         -&         -&         -&         -&  2.00E-15&  1.70E-15&  1.90E-15&         - \\
             & ( 2.3e-17)&( 3.6e-17)&         -&( 1.2e-16)&         -&         -&         -&         -&(   8e-17)&( 3.8e-17)&( 8.8e-17)&         - \\
    IR93ap4  &   2.60E-15&  3.70E-15&         -&         -&         -&         -&         -&         -&  4.30E-15&  3.50E-15&  4.70E-15&         - \\
             & ( 3.2e-17)&( 3.6e-17)&         -&         -&         -&         -&         -&         -&( 8.5e-17)&( 5.2e-17)&( 7.4e-17)&         - \\
    IR93ap5  &          -&  1.90E-15&         -&  1.00E-15&         -&         -&         -&         -&         -&         -&         -&         - \\
             &          -&(2.4e-17) &         -&(1.00e-16)&         -&         -&         -&         -&         -&         -&         -&         - \\
    IR93ap6  &          -&  4.10E-16&         -&         -&         -&         -&         -&         -&         -&         -&         -&         - \\
             &          -&(   3e-17)&         -&         -&         -&         -&         -&         -&         -&         -&         -&         - \\  
\enddata
   \tablecomments{Fluxes are in units erg s$^{-1} \persc $\AA$^{-1}$.  Errors are listed on
   the second row for each aperture.  Errors of (0) indicate that the line was detected, but that
   the fluxes should not be trusted because the background was probably oversubtracted.}
\end{deluxetable}\addtocounter{table}{-1}
\clearpage
\end{landscape}

\begin{deluxetable}{ccccccccccccccc}
  \scriptsize
  \tabletypesize{\tiny}
  %\rotate
  \tablewidth{0pt}
    \centering
    \tablecaption{Measured \hh\ line strengths (cont'd)
      \label{tab:nirmeas2}}
    \tablehead{
&2-1 S(1)&2-1 S(3)&3-2 S(3)&3-2 S(4)&4-3 S(5) & [Fe II] & [Fe II] \\
&2.24772&2.07351&2.2014&2.12797&2.20095       & 1.6435  & 1.2567  \\ }
\startdata
outflow1ap1  &   2.00E-15&  1.20E-15&  6.20E-16&  2.60E-16&  7.10E-16 & 4.4e-15    &3.5e-15    \\
             & ( 2.5e-17)&( 3.5e-17)&( 2.2e-17)&( 1.6e-17)&( 1.9e-17) & ( 7.9e-17) &(   4e-17) \\
outflow1ap2  &          -&         -&         -&         -&         - & 6.7e-16    &3.1e-16    \\
             &          -&         -&         -&         -&         - & ( 7.8e-17) &( 3.3e-17) \\
outflow1ap3  &   9.90E-16&         -&  5.70E-16&  2.50E-16&  6.40E-16 & 1.3e-15    &5.7e-16    \\
             & ( 2.6e-17)&         -&(       0)&( 1.2e-17)&(       0) & ( 8.9e-17) &(   4e-17) \\
outflow1ap4  &          -&         -&         -&         -&         - & -          & -         \\
             &          -&         -&         -&         -&         - & -          & -         \\
outflow2ap1  &          -&         -&         -&         -&         - & -          & -         \\
             &          -&         -&         -&         -&         - & -          & -         \\
outflow2ap2  &   6.40E-16&         -&         -&         -&         - & -          & -         \\
             & ( 1.9e-17)&         -&         -&         -&         - & -          & -         \\
outflow2ap3  &          -&         -&         -&         -&         - & -          & -         \\
             &          -&         -&         -&         -&         - & -          & -         \\
outflow4ap1  &   4.30E-16&  4.20E-16&         -&         -&  1.60E-16 & -          & -         \\
             & ( 1.9e-17)& (0)      &         -&         -& (0)       & -          & -         \\
outflow4ap2  &          -&         -&         -&         -&         - & -          & -         \\
             &          -&         -&         -&         -&         - & -          & -         \\
     IR6ap1  &          -&         -&         -&         -&         - & -          & -         \\
             &          -&         -&         -&         -&         - & -          & -         \\
    IR93ap1  &          -&         -&         -&         -&         - & -          & -         \\
             &          -&         -&         -&         -&         - & -          & -         \\
    IR93ap2  &   3.80E-15&         -&  3.10E-15&         -&         - & -          & -         \\
             & ( 2.2e-17)&         -& (0)      &         -&         - & -          & -         \\
    IR93ap3  &          -&         -&         -&         -&         - & -          & -         \\
             &          -&         -&         -&         -&         - & -          & -         \\
    IR93ap4  &          -&         -&         -&         -&         - & -          & -         \\
             &          -&         -&         -&         -&         - & -          & -         \\
    IR93ap5  &          -&         -&         -&         -&         - & -          & -         \\
             &          -&         -&         -&         -&         - & -          & -         \\
    IR93ap6  &          -&         -&         -&         -&         - & -          & -         \\
             &          -&         -&         -&         -&         - & -          & -         \\             
   \enddata
   \tablecomments{Fluxes are in units erg s$^{-1} \persc $\AA$^{-1}$.  Errors are listed on
   the second row for each aperture.  Errors of (0) indicate that the line was detected, but that
   the fluxes should not be trusted because the background was probably oversubtracted.}
\end{deluxetable}
\clearpage

\ifstandalone
\bibliographystyle{apj_w_etal}  % or "siam", or "alpha", or "abbrv"
%\bibliography{thesis}      % bib database file refs.bib
\bibliography{bibdesk}      % bib database file refs.bib
\fi

\end{document}

% %\documentclass[defaultstyle,11pt]{thesis}
%\documentclass[]{report}
%\documentclass[]{article}
%\usepackage{aastex_hack}
%\usepackage{deluxetable}
\documentclass[preprint]{aastex}


%%%%%%%%%%%%%%%%%%%%%%%%%%%%%%%%%%%%%%%%%%%%%%%%%%%%%%%%%%%%%%%%
%%%%%%%%%%%  see documentation for information about  %%%%%%%%%%
%%%%%%%%%%%  the options (11pt, defaultstyle, etc.)   %%%%%%%%%%
%%%%%%%  http://www.colorado.edu/its/docs/latex/thesis/  %%%%%%%
%%%%%%%%%%%%%%%%%%%%%%%%%%%%%%%%%%%%%%%%%%%%%%%%%%%%%%%%%%%%%%%%
%		\documentclass[typewriterstyle]{thesis}
% 		\documentclass[modernstyle]{thesis}
% 		\documentclass[modernstyle,11pt]{thesis}
%	 	\documentclass[modernstyle,12pt]{thesis}

%%%%%%%%%%%%%%%%%%%%%%%%%%%%%%%%%%%%%%%%%%%%%%%%%%%%%%%%%%%%%%%%
%%%%%%%%%%%    load any packages which are needed    %%%%%%%%%%%
%%%%%%%%%%%%%%%%%%%%%%%%%%%%%%%%%%%%%%%%%%%%%%%%%%%%%%%%%%%%%%%%
\usepackage{latexsym}		% to get LASY symbols
\usepackage{graphicx}		% to insert PostScript figures
%\usepackage{deluxetable}
\usepackage{rotating}		% for sideways tables/figures
\usepackage{natbib}  % Requires natbib.sty, available from http://ads.harvard.edu/pubs/bibtex/astronat/
\usepackage{savesym}
\usepackage{amssymb}
%\savesymbol{singlespace}
\savesymbol{doublespace}
%\usepackage{wrapfig}
%\usepackage{setspace}
\usepackage{xspace}
\usepackage{color}
\usepackage{multicol}
\usepackage{mdframed}
\usepackage{url}
\usepackage{subfigure}
%\usepackage{emulateapj}
\usepackage{lscape}
\usepackage{grffile}
\usepackage{standalone}
\standalonetrue
\usepackage{import}
\usepackage[utf8]{inputenc}
\usepackage{longtable}
\usepackage{booktabs}



%%%%%%%%%%%%%%%%%%%%%%%%%%%%%%%%%%%%%%%%%%%%%%%%%%%%%%%%%%%%%%%%
%%%%%%%%%%%%       all the preamble material:       %%%%%%%%%%%%
%%%%%%%%%%%%%%%%%%%%%%%%%%%%%%%%%%%%%%%%%%%%%%%%%%%%%%%%%%%%%%%%

% \title{Star Formation in the Galaxy}
% 
% \author{Adam G.}{Ginsburg}
% 
% \otherdegrees{B.S., Rice University, 2007\\
% 	      M.S., University of Colorado, Boulder, 2009}
% 
% \degree{Doctor of Philosophy}		%  #1 {long descr.}
% 	{Ph.D., Rocket Science (ok, fine, astrophysics)}		%  #2 {short descr.}
% 
% \dept{Department of}			%  #1 {designation}
% 	{Astrophysical and Planetary Sciences}		%  #2 {name}
% 
% \advisor{Prof.}				%  #1 {title}
% 	{John Bally}			%  #2 {name}
% 
% \reader{Prof.~Jeremy Darling}		%  2nd person to sign thesis
% \readerThree{Prof.~Jason Glenn}		%  3rd person to sign thesis
% \readerFour{Prof.~Michael Shull}	%  4rd person to sign thesis
% \readerFour{Prof.~Neal Evans}	%  4rd person to sign thesis
% 
% \abstract{  \OnePageChapter	% one page only ??
% 
%     I discovered dust in space.  
% 
% 	}
% 
% 
% \dedication[Dedication]{	% NEVER use \OnePageChapter here.
% 	To 1, the second number in binary.
% 	}
% 
% \acknowledgements{	\OnePageChapter	% *MUST* BE ONLY ONE PAGE!
% 	All y'all.
% 	}
% 
% \ToCisShort	% a 1-page Table of Contents ??
% 
% \LoFisShort	% a 1-page List of Figures ??
% %	\emptyLoF	% no List of Figures at all ??
% 
% \LoTisShort	% a 1-page List of Tables ??
% %	\emptyLoT	% no List of Tables at all ??
% 
% 
% %%%%%%%%%%%%%%%%%%%%%%%%%%%%%%%%%%%%%%%%%%%%%%%%%%%%%%%%%%%%%%%%%
% %%%%%%%%%%%%%%%       BEGIN DOCUMENT...         %%%%%%%%%%%%%%%%%
% %%%%%%%%%%%%%%%%%%%%%%%%%%%%%%%%%%%%%%%%%%%%%%%%%%%%%%%%%%%%%%%%%
% 
% %%%%  footnote style; default=\arabic  (numbered 1,2,3...)
% %%%%  others:  \roman, \Roman, \alph, \Alph, \fnsymbol
% %	"\fnsymbol" uses asterisk, dagger, double-dagger, etc.
% %	\renewcommand{\thefootnote}{\fnsymbol{footnote}}
% %	\setcounter{footnote}{0}

\input{macros}		% file containing author's macro definitions

\begin{document}
% \input{introduction}
% 
% %\input{ch_iras05358}
% \input{ch_w5}
% \input{ch_h2co}
% \input{ch_h2colarge}
% \input{ch_boundhii}
% 
% %\input ch2.tex			% file with Chapter 2 contents
% 
% %%%%%%%%%%%%%%%%%%%%%%%%%%%%%%%%%%%%%%%%%%%%%%%%%%%%%%%%%%%%%%%%%%%
% %%%%%%%%%%%%%%%%%%%%%%%  Bibliography %%%%%%%%%%%%%%%%%%%%%%%%%%%%%
% %%%%%%%%%%%%%%%%%%%%%%%%%%%%%%%%%%%%%%%%%%%%%%%%%%%%%%%%%%%%%%%%%%%
% 
% \bibliographystyle{plain}	% or "siam", or "alpha", or "abbrv"
% 				% see other styles (.bst files) in
% 				% $TEXHOME/texmf/bibtex/bst
% 
% \nocite{*}		% list all refs in database, cited or not.
% 
% \bibliography{thesis}		% bib database file refs.bib
% 
% %%%%%%%%%%%%%%%%%%%%%%%%%%%%%%%%%%%%%%%%%%%%%%%%%%%%%%%%%%%%%%%%%%%
% %%%%%%%%%%%%%%%%%%%%%%%%  Appendices %%%%%%%%%%%%%%%%%%%%%%%%%%%%%%
% %%%%%%%%%%%%%%%%%%%%%%%%%%%%%%%%%%%%%%%%%%%%%%%%%%%%%%%%%%%%%%%%%%%
% 
% \appendix	% don't forget this line if you have appendices!
% 
% %\input appA.tex			% file with Appendix A contents
% %\input appB.tex			% file with Appendix B contents
% 
% %%%%%%%%%%%%%%%%%%%%%%%%%%%%%%%%%%%%%%%%%%%%%%%%%%%%%%%%%%%%%%%%%%%
% %%%%%%%%%%%%%%%%%%%%%%%%   THE END   %%%%%%%%%%%%%%%%%%%%%%%%%%%%%%
% %%%%%%%%%%%%%%%%%%%%%%%%%%%%%%%%%%%%%%%%%%%%%%%%%%%%%%%%%%%%%%%%%%%
% 
% \end{document}
% 
% 

\chapter{Using outflows to track star formation in the W5 HII region complex}
\label{ch:w5}
\section{Preface}
Only a few months after arriving at CU, I was given the opportunity to visit
the peak of Mauna Kea to perform observations with the JCMT.  I spend about 3
weeks at the telescope over the course of two years primarily mapping the W5
complex.  A side-project done during these observations resulted in my Comps II
project on IRAS 05358+3543.  These data were taken using Jonathan Williams'
Hawaii time allocation with the HARP receiver.  The data were taken with
essentially no plan for how they would be used.  The paper may have diminished
our group's overall interest in the W5 region: it turns out that star formation
is probably at its end here, being quenched by massive-star feedback.  However,
there is a largely ignored cloud to the northwest of the well-studied W5 bubbles
that has significant potential to form new stars.  

The W5 study was originally intended to include a Bolocam census of cores, but
the data in this region turned out to be the most problematic and contained
little signal.  We acquired additional data in 2009, but never got around to
performing a joint analysis of the CO and continuum data.  In part, at least,
this is because W5 is so faint at millimeter wavelengths compared to many
Galactic Plane sources.

This work is essentially a very detailed study of a star-forming region with
minimal implications for star forming theories at the moment.

\section{Introduction}


Galactic-scale shocks such as spiral density waves promote the formation of
giant molecular clouds (GMCs) where massive stars, star clusters, and OB
associations form.  The massive stars in such groups can either disrupt the
surrounding medium or promote further star formation.  While ionizing and soft
UV radiation, stellar winds, and eventually supernova explosions destroy clouds
in the immediate vicinity of massive stars, as the resulting bubbles age and
decelerate, they can also trigger further star formation.  In the ``collect and
collapse'' scenario
\citep[e.g.][]{elmegreen:sequential:1977}, gas swept-up by expanding bubbles
can collapse into new star-forming clouds.  In the ``radiation-driven
implosion'' model \citep{bertoldi:cometary:1990,klein:implosion:1983},
pre-existing clouds may be compressed by photo-ablation pressure or by the
increased pressure as they are overrun by an expanding shell.  In some
circumstances, forming stars are simply exposed as low-density gas is removed
by winds and radiation from massive stars.  These processes may play significant roles in
determining the efficiency of star formation in clustered environments
\citep{elmegreen1998}.


% Shortly after igniting, young massive stars may either provide the additional
% pressure needed to collapse neighboring gas clumps into stars, or they may
% simply disperse and destroy their natal environments.  The former phenomenon,
% known as triggered star formation, may play a significant role in determining
% the efficiency of star formation in clustered environments
% \citet{elmegreen1998}.  In addition, triggering from external processes
% asynchronous with star formation by direct collapse (e.g.  supernovae, spiral
% density waves) are thought to be partly responsible for the evenutal collapse
% of parent molecular clouds, but it is not yet known on what scales triggering
% is important, nor how strongly each mechanism contributes.  

Feedback from low mass stars may also control the shape of the stellar initial
mass function in clusters \citep{adams1996,Peters2010}.  Low mass young stars
generate high velocity, collimated outflows that contribute to the turbulent
support of a gas clump, preventing the clump from forming stars long enough
that it is eventually blown away by massive star feedback.  It is therefore
important to understand the strength of low-mass protostellar feedback relative
to other feedback mechanisms.

Outflows are a ubiquitous indicator of the presence of ongoing star formation
\citep{reipurth2001}.  CO outflows are an indicator of ongoing embedded star
formation at a younger stage than optical outflows because shielding from the
interstellar radiation field is required for CO to survive.  Although Herbig-
Haro shocks and \hh\ knots reveal the locations of the highest-velocity
segments of these outflows, CO has typically been thought of as a
``calorimeter'' measuring the majority of the mass and momentum ejected from
protostars or swept up by the ejecta \citep{Bachiller1996}.

%\subsection{W5}

The W5 star forming complex in the outer galaxy is a prime location to study
massive star formation and triggering.  The bright-rimmed clouds in W5 have
been recognized as good candidates for ongoing triggering by a number of groups
\citep{lefloch:cometary:1997,thompson:searching:2004,karr:triggered:2003}.  The
clustering properties were analyzed by \citet{koenig:clustered:2008} using
Spitzer infrared data, and a number of significant clusters were discovered.
The whole W5 complex may be a product of triggering, as it is located on one
side of the W4 chimney thought to be created by multiple supernovae during the
last $\sim$10 MYr \citep[][Figure \ref{fig:color_overview}]
{oey:hierarchical:2005}.  

Following \citet{koenig:clustered:2008}, we adopt a distance to W5 of 2 kpc
based on the water-maser parallax distance to the neighboring W3(OH) region
\citep{Hachisuka2006}.  As with W3, the W5 cloud is substantially
($\approx1.5\times$) closer than its kinematic distance would suggest
($v_{LSR}(-40~\kms)\approx3$ kpc).  Given this distance,
\citet{koenig:clustered:2008} derived a total gas mass of 6.5\ee{4} \msun\ from
a 2 \um\ extinction map.

The W5 complex was mapped in the $^{12}$CO 1-0 emission line by the Five
College Radio Astronomy Observatory (FCRAO) using the SEQUOIA receiver array
\citep{heyer:ogs:1998}.  The same array was used to map W5 in the \thirteenco\
1-0 line (C. Brunt, private communication).  Some early work searched for
outflows in W5 \citep{bretherton:unbiased:2002}, but the low-resolution CO 1-0
data only revealed a few, and only one was published.  The higher resolution
and sensitivity observations presented here reveal many additional outflows.

\begin{figure*}
  % Generated by GIMP
  \includegraphics[angle=90,width=5in]{figures_chw5/w345mos_brightened_labeled}
  \caption{An overview of the W3/4/5 complex (also known as the ``Heart and
  Soul Nebula'') in false color. Orange shows 8 \um\ emission from the Spitzer
  and MSX satellites.  Purple shows 21 cm continuum emission from the DRAO CGPS
  \citep{Taylor2003:CGPS}; the DSS R image was used to set the display opacity
  of the 21 cm continuum as displayed (purely for aesthetic purposes).  The
  green shows JCMT \twelveco\ 3-2 along with FCRAO \twelveco\ 1-0 to fill in 
  gaps that were not observed with the JCMT.  The image spans
  $\sim7\arcdeg$ in galactic longitude.  This overview image shows the
  hypothesized interaction between the W4 superbubble and the W3 and W5
  star-forming regions \citep{oey:hierarchical:2005}.}
  \label{fig:color_overview}
\end{figure*}

%\subsection{Neighboring Regions}
While W5 is thought to be associated with the W3/4/5 complex, there are other
infrared sources in the same part of the sky that are not obviously associated
with W5.  Some of these have been noted to be in the outer arm (several kpc
behind W5) by \citet{Digel1996} and \citet{Snell2002}.

\par
\par In section 2, we present the new and archival data used in our study.  In
section 3, we discuss the outflow detection process and compare outflow
detectability in W5 to that in Perseus.  In section 4, we discuss the physical
properties of the outflows and their implications for star formation in the W5
complex.  In section 5, we briefly describe the outer-arm outflows discovered.

\section{OBSERVATIONS}
\subsection{JCMT HARP CO 3-2}
CO J=3-2 345.79599 GHz data were acquired at the 15 m James Clerk Maxwell
Telescope (JCMT) using the HARP array on a series of observing runs in 2008.
On 2-4 January, 2008, $\sim$ 800 square arcminutes were mapped.  During the
run, $\tau_{225}$, the zenith opacity at 225 GHz measured using the Caltech
Submillimeter Observator tipping radiometer, ranged from 0.1 to 0.4
($0.4<\tau_{345
GHz}<1.6$\footnote{\url{http://docs.jach.hawaii.edu/JCMT/SCD/SN/002.2/node5.html}}).
Additional areas were mapped on 4-7 August, 16-20 and 31 October, and 1 and
12-15 Nov 2008 in similar conditions.  A total of $\sim$ 3 square degrees (12000
arcmin$^2$) in the W5 complex were mapped (a velocity-integrated mosaic is
shown in Figure \ref{fig:outflows_on_co32}).

HARP is a 16 pixel SIS receiver array acting as a front-end to the ACSIS
digital auto-correlation spectrometer.  In January 2008, 14 of the 16 detectors
were functional.  In the 2nd half of 2008, 12 of 16 were functional,
necessitating longer scans to achieve similar S/N.

In January 2008, a single spectral window centered at 345.7959899 with bandwidth 1.0
GHz and channel width 488 kHz (0.42 \kms) was used.  In August 2008 and later, we used 250 MHz
bandwidth and 61 kHz (0.05 \kms) channel width.  At this frequency, the beam
FWHM is 14\arcsec\ (0.14 pc at a distance of 2 kpc)
\footnote{\url{http://docs.jach.hawaii.edu/JCMT/OVERVIEW/tel_overview/}}.

A raster mapping strategy was used.  In 2008, the array was shifted by 1/2 of
an array spacing (58.2\arcsec) between scans.  Data was sampled at a rate of
$0.6 s$ per integration.  Two perpendicular scans were used for each field
observed.  Most fields were 10$\times$10\arcmin\ and took $\sim45$ minutes.
When only 12 receptors were available, 1/4 array stepping (29.1\arcsec) was
used with a sample rate of $0.4 s$ per integration.

Data were reduced using the SMURF package within the STARLINK software distribution
\footnote{\url{http://starlink.jach.hawaii.edu/}}.  The SMURF command {\sc makecube} was used to
generate mosaics of contiguous sub-fields.  The data were gridded on to cubes
with 6\arcsec\ pixels and smoothed with a $\sigma=2$-pixel gaussian, resulting
in a map FWHM resolution of 18\arcsec (0.17 pc).  A linear fit was subtracted from each
spectrum over emission-free velocities (generally -60 to -50 and -20 to -10
\kms) to remove the baseline.  The final map RMS was $\sigma_{T_A^*}\sim
0.06-0.11 K$ in 0.42 \kms\ channels.

The sky reference position (off position) in January 2008 was J2000 2:31:04.069 +62:59:13.81.
In later epochs, off positions closer to the target fields were selected from blank sky regions
identified in January 2008 in order to increase observing efficiency.  A
main-beam efficiency $\eta_{mb}=0.60$ was used as per the JCMT website to
convert measurements to $T_{mb}$, though maps and spectra are presented in the
original $T_A^*$ units.

\subsection{FCRAO Outer Galaxy Survey}
The FCRAO Outer Galaxy Survey (OGS)
observed the W5 complex in \twelveco\  \citep{heyer:ogs:1998} and \thirteenco\
1-0 (C. Brunt, private communication).  The \thirteenco\ data cube achieved a
mean sensitivity of 0.35 K per 0.13 \kms\ channel, or 0.6 K \kms\ integrated.
The \thirteenco\ cube was integrated over all velocities and resampled to match
the BGPS map using the {\sc montage}\footnote{\url{http://montage.ipac.caltech.edu/}}
package.  The FWHM beam size was  $\theta_{B}=$50\farcs (0.48 pc).  The integrated
\twelveco\ data cube, with a sensitivity $\sigma= 1 K$ \kms, is displayed with 
region name identifications in Figure \ref{fig:regionboxes_on_CO}.

\subsection{Spitzer}
Spitzer IRAC and MIPS 24 \um\ images from \citet{koenig:clustered:2008} were
used for morphological comparison.  The reduction and extraction techniques are
detailed in their paper.

\begin{figure*}
  %\epsscale{1.0}
  %CODE: cutouts.py
  \includegraphics[angle=90,width=5in]{figures_chw5/outflows_ostars_on_CO32}
  \caption{A mosaic of the CO 3-2 data cube integrated from -20 to -60 \kms.
  The grayscale is linear from 0 to 150 K \kms.  The red and blue X's mark the
  locations of redshifted and blueshifted outflows.  Dark red and dark blue
  plus symbols mark outflows at outer arm velocities.  Green circles mark the
  location of all known B0 and earlier stars in the W5 region from SIMBAD.}
  \label{fig:outflows_on_co32}
\end{figure*}

\Figure{figures_chw5/regionboxes_on_CO}
{Individual region masks overlaid on the FCRAO \twelveco\ integrated image.
The named regions, S201, AFGL4029, LWCas, W5NW, W5W, W5SE, W5S, and W5SW, were all
selected based on the presence of outflows within the box.
The inactive regions were selected from regions with substantial CO emission
but without outflows.  The `empty' regions have essentially no CO emission within
them and are used to place limits on the molecular gas within the east and west
`bubbles'.  W5NWpc is compared directly to the Perseus molecular cloud in 
Section \ref{sec:percompare}
%Finally, W5All is used in Section \ref{sec:covsbgps} to compute
%global properties of the complex.  
}{fig:regionboxes_on_CO}{0.5}{0}


\section{Analysis}
\subsection{Outflow Detections}
Outflows were identified in the CO data cubes by manually searching through
position-velocity space for line wings using STARLINK's GAIA display software.  Outflow
candidates were identified by high velocity wings inconsistent with the local
cloud velocity distribution, which ranged from a width of 3 \kms\ to  7 \kms.
Once an outflow candidate was identified in the position-velocity diagrams, the
velocity range over which the wing showed emission in the position-velocity
diagram (down to $T_A^*=0$) was integrated over to create a map from which the
approximate outflow size and position was determined (e.g. Figures
\ref{fig:outflow1} and \ref{fig:pv2b}). 
%The integration limits for computing physical
%properties were also set from the position-velocity diagram limits, not the
%averaged spectra: in some cases, emission at higher velocities away from the
%outflow was included in the aperture that is not part of the outflow.

Unlike \citet{curtis2010} and \citet{hatchell2009}, we did not use an
`objective' outflow identification method because of the greater velocity complexity
and poorer spatial resolution of our observations.  The outflow selection
criteria in these papers requires the presence of a sub-mm clump in order to
identify a candidate driving source (and therefore a targeted region in which to search for
outflows), making a similar objective identification impossible for
our survey.
As discussed later in 
Section \ref{sec:subregions}, the regions associated with outflows have wide
lines and many are double-peaked.  Additionally, many smaller areas associated
with outflows have collections of gaussian-profiled clumps that are not connected to
the cloud in position-velocity diagrams but are not outflows.  In particular,
W5 is pockmarked by dozens of small cometary globules that are sometimes
spatially coincident with the clouds but slightly offset in velocity.

While \citet{arce2010} described the benefits of 3D visualization using
isosurface contours, we found that the varying signal-to-noise across
large-scale ($\sim500$ pixel$^2$) regions with significant extent in RA/Dec and
limited velocity dynamic range made this method diffult for W5.  There
were many low-intensity outflows that were detectable by careful searches
through position-velocity space that are not as apparent using isosurface
methods.  Out of the 55 outflows reported here, only 14 \footnote{
Outflows 15, 20, 24, the cluster of outflows 26-32, 47, 48, 52, and 53 could 
all have readily been detected by pointed single-dish measurements.} would be
considered obvious, high-intensity, high-velocity flows from their spectra
alone; the rest could not be unambiguously detected without a search through
position-velocity space.

In the majority of sources, the individual outflow lobes were
unresolved, although some showed hints of position-velocity gradients at low
significance and in many the red and blue flows are spatially separated.    Only
Outflow 1's lobes were clearly resolved (Figure \ref{fig:outflow1}).  Some of
most suggestive gradients occurred where the outflow merged with its host
molecular cloud in position-velocity space, making the gradient difficult to
distinguish (e.g., Outflow 12, Figure \ref{fig:pv12}).  Bipolar pairs were
selected when there were red and blue flows close to one another.  The
classification of a bipolar flow was either `yc' (yes - confident), `yu' (yes -
unconfident), or `n' (no) in Table \ref{tab:outflows}.  This identification is
discussed in the captions for each outflow figure in the online
supplement. The AFGL 4029 region has many red and blue lobes but confusion
prevented pairing.  

\FigureFour{figures_chw5/pvdiagrams/pvdiagram_rotate_of1}{figures_chw5/OutflowOverlay_4.5_1}{figures_chw5/OutflowSpectra1}{figures_chw5/OutflowOverlay_8um_1}
{Position-velocity diagrams (a), spectra (c), and contour overlays of Outflow 1 on Spitzer 4.5 \um\ (b)
and 8 \um\ (d) images.  
This outflow is clearly resolved and bipolar.
{\it (a)}: Position-velocity diagram of the blue flow displayed in arcsinh stretch
from $T_A^*=$0 to 3 K.  Locations of the red and blue flows are indicated by vertical dashed lines.
The location of the position-velocity cut is indicated by the orange dashed line in panels (b) and
(d), although the position-velocity cut is longer than those cut-out images.
{\it (b)} Spitzer 4.5 \um\ image displayed in logarithmic stretch from 30 to 500 MJy \persr.    
{\it (c)}: Spectrum of
the outflow integrated over the outflow aperture and the velocity range
specified with shading.  The velocity center (vertical dashed line) is
determined by fitting a gaussian to the \thirteenco\ spectrum in an aperture
including both outflow lobes.  In the few cases in which \thirteenco\ 1-0 was 
unavailable, a gaussian was fit to the \twelveco\ 3-2 spectrum.
{\it (d)}: Contours of the red and blue outflows superposed on the 
Spitzer 8 \um\ image displayed in logarithmic stretch.  The contours are
generated from a total intensity image integrated over the outflow
velocities indicated in panel (c).  The contours in both panels (b) and (d) are displayed at levels of
0.5,1,1.5,2,3,4,5,6 K \kms\ ($\sigma\approx0.25$ K \kms).
The contour levels and stretches specified in this caption apply to all of the
figures in the supplementary materials except where otherwise noted.
}
{fig:outflow1}


\FigureFour{figures_chw5/pvdiagrams/pvdiagram_rotate_of2}{figures_chw5/OutflowOverlay_4.5_2}{figures_chw5/OutflowSpectra2}{figures_chw5/OutflowOverlay_8um_2}
{Position-velocity diagram, spectra, and contour overlays of Outflow 2 (see
Figure \ref{fig:outflow1} for a complete description).  While the two lobes are
widely separated, there are no nearby lobes that could lead to confusion, so we
regard this pair as a reliable bipolar outflow identification.  
}
{fig:pv2b}

\FigureFour{figures_chw5/pvdiagrams/pvdiagram_rotate_of12}{figures_chw5/OutflowOverlay_4.5_12}{figures_chw5/OutflowSpectra12}{figures_chw5/OutflowOverlay_8um_12}
{Position-velocity diagram, spectra, and contour overlays of Outflow 12.  Much of the red outflow
is lost in the complex velocity profile of the molecular cloud(s), but it is high enough velocity
to still be distinguished. }
{fig:pv12}

In cases where only the red- or blue-shifted lobe was visible, the
surrounding pixels were searched for lower-significance and lower-velocity
counterparts.  For cases in which emission was detected, a
candidate counterflow was identified and incorporated into the catalog.
However, in 12 cases, the counterflow still evaded detection, either because of
confusion or because the counterflow is not present in CO.

The outflow positions are overlaid on the CO 3-2 image in Figure
\ref{fig:outflows_on_co32} to provide an overview of where star formation is
most active.  The figures in Section \ref{sec:sfactivity}  show outflow
locations overlaid on small-scale images.

Because our detection method involved searching for high-velocity outflows
by eye, there should be no false detections.  However, it is possible
that some of these outflows are generated by mechanisms other than
protostellar jets and winds since we have not identified their driving sources.

One possible alternative driving mechanism is a photoevaporation flow,
which could be accelerated up to the sound speed of the ionized medium,
$c_{II} \approx 10$~\kms.
Gas accelerating away from the cloud would not be detected as an outflow
because it would be rapidly ionized.
However, gas driven inward would be accelerated and remain molecular.  It
could exhibit red and / or blue flows depending on the line of
sight orientation.  While there are viable candidates for this form of outflow
impersonator, such flows can only have peak velocities $v\lesssim c_{II}/4
\approx 2.5$~\kms\ in the strong adiabatic shock limit, so that any gas
seen with higher velocity tails are unlikely to be radiation-driven.  

Another plausible outflow impostor is the high-velocity tail in a
turbulent distribution.  However, for a typical molecular cloud, the low
temperatures would require very high mach-number shocks ($\mathcal{M}\gtrsim10$
assuming $T_{cloud}\sim20 $ K and $v_{flow} \sim 3$ \kms) that in idealized
turbulence should be rare and short-lived.  It is not known how frequent such
high-velocity excursions will be in non-ideal turbulence with gravity (A.
Goodman, P. Padoan, private communication).  Finally, it is less likely for
turbulent intermittency to have nearly coincident red and blue lobes, so
intermittency can be morphologically excluded in most cases. 


\subsubsection{Comparison to Perseus CO 3-2 observations}  % may not need subsub...
\label{sec:percompare}
We used the HARP CO 3-2 cubes from \citet{hatchell2007} to evaluate our ability
to identify outflows.  We selected an outflow that was
well-resolved and unconfused, L1448, and evaluated it at both the native
sensitivity of the \citet{hatchell2007} observations and degraded in resolution
and sensitivity to match our own.  We focus on L1448 IRS2,
labeled Outflow 30 in \citet{hatchell2007}. Figure \ref{fig:l1448} shows a
comparison between the original quality and degraded data.  

Integrating over the outflow velocity range, we measure each lobe to be
about $1.6\arcmin\times0.8\arcmin$ ($0.14\times0.07$pc).  Assuming a distance
to Perseus of 250 pc \citep[e.g.][]{Enoch2006},  we smooth by a factor of 8 by convolving
the cube with a FWHM = 111\arcsec\ gaussian, then downsample by the same factor of 8
to achieve 6\arcsec\ square pixels at 2 kpc.  The resulting noise was reduced
because of the spatial and spectral smoothing and was measured to be $\approx 0.05$ K in
0.54 \kms\ channels, which is comparable to the sensitivity in our
survey.  It is still possible to distinguish the outflows
from the cloud in position-velocity space.  Each lobe is individually
unresolved (long axis $\sim12\arcsec$ compared to our beam FWHM of 18\arcsec),
but the two are separated by $\gtrsim 20\arcsec$ and therefore an overall
spatial separation can still be measured.  Because they are just barely
unresolved at this distance, the lobes' surface brightnesses are approximately
the same at 2 kpc as at 250 pc; if this outflow were seen at a greater distance
it would appear fainter.

\citet{hatchell2007} detected 4 outflows within this map, plus an additional
confused candidate.  We note an additional grouping of outflowing material in
the north-middle of the map(centered on coordinate 150$\times$150 in Figure
\ref{fig:l1448}).  In the smoothed version, only three outflows are detected in
the blue and two in the red, making flow-counterflow association difficult.
The north-central blueshifted component appears to be the counterpart of the
red flow when smoothed, although it is clearly the counterpart of the northwest
blue flow in the full-resolution image.

We are therefore able to detect any outflows comparable to L1448 (assuming a
favorable geometry), but are likely to see clustered outflows as single or
possibly extended lobes and will count fewer lobes than would be detected at
higher resolution.  Additionally, it is clear from this example that two
adjacent outflows with opposite polarity are not necessarily associated, and
therefore the outflows' source(s) may not be between the two lobes.  

\Figure{figures_chw5/hatchell_l1448_compare}
{Comparison of L1448 seen at a distance of 250 pc (left) versus 2 kpc (middle) with
sensitivity 0.5 K and 0.05K per 0.5 \kms\ channel respectively.  {\it Far
Left}: Position-velocity diagram (log scale) of the outflow L1448 IRS2 at its
native resolution and velocity.  L1448 IRS2 is the rightmost outflow in the contour 
plots.  The PV diagram is rotated 45\arcdeg\ from RA/Dec axes to go along the outflow
axis.
{\it Middle Left}: Position-velocity diagram (log scale) of the same outflow smoothed
and rebinned to be eight times more distant.
{\it Top Right}: The integrated map is displayed at its native resolution (linear scale).
The red contours are of the same data integrated from 6.5 to 16 \kms\ and the blue from 
-6 to 0 \kms.  Contours are at 1,3, and 5 K \kms\ ($\sim 6, 18, 30 \sigma$).  Axes
are offsets in arcseconds.  Because we are only examining the relative detectability of
outflows at two distances, we are not concerned with absolute coordinates.
{\it Bottom Right}: The same map as it would be observed at eight times greater
distance.  Axes are offsets in arcseconds assuming the greater distance.
Contours are integrated over the same velocity range as above, but are
displayed at levels 0.25,0.50,0.75,1.00 K \kms\ ($\sim 12, 24, 48, 60 \sigma$).
The entire region is detected at high significance, but dominated by confusion.
It is still evident that the red and blue lobes are distinct, but they are each
unresolved. 
}
{fig:l1448}{0.5}{0}

In order to determine overall detectability of outflows compared to Perseus, we
compare to \citet{curtis2010} in Figure \ref{fig:lengthhist}.  Out of 29
outflows in their survey with measured `lobe lengths', 22 (71\%) were smaller
than 128\arcsec\, which would be below our 18\arcsec\ resolution if 
observed at 2 kpc.  Even the largest lobes (HRF26R,
HRF28R, HRF44B) would only extend $\sim60\arcsec$ at 2 kpc.  Each lobe in the
largest outflow in our survey, Outflow 1, is $\sim80\arcsec$ (660\arcsec\ at
250pc), but no other individual outflow lobes in W5 are clearly resolved.
However, as seen in Figure \ref{fig:lengthhist}, many bipolar lobes are
\emph{separated} by more than the telescope resolution, and the overall lobe
separation distribution (as opposed to the lobe length, which is mostly unmeasured
in our sample) in W5 is quite similar to the separation distribution in
Perseus.  The 2-sample KS test gives a 25\% probability that they are drawn
from the same distribution (the null hypothesis that they are drawn from the
same distribution cannot be rejected).  
%However, \citet{Yu1999} discovered a CO 1-0 outflow in Barnard 5 that extended
%$\sim12\arcmin$, substantially larger than the largest detected in our survey 
%even at a distance of 2 kpc.

\Figure{figures_chw5/OutflowLengthHistogram}
{Histogram of the measured outflow lobe separations.  The grey hatched region shows
\citet{curtis2010} values.  The vertical dashed line represents the spatial
resolution of our survey.  The two distributions are similar.}
{fig:lengthhist}{0.5}{0}

On average, the \citet{curtis2010} outflow velocities are similar to ours
(Figure \ref{fig:widthhist}).   We detect lower velocity outflows because we do
not set a strict lower velocity limit criterion.  We do not detect the highest
velocity outflows most likely because of our poorer sensitivity to the faint
high-velocity tips of outflows, although it is also possible that no
high-velocity ($v>20$ \kms) flows exist in the W5 region.  Note that the
histogram compares quantities that are not directly equivalent: the outflows in
\citet{curtis2010} and our own data are measured out to the point at which the
outflow signal is lost, while the `region' velocities are full-width half-max
(FWHM) velocities.  

Finally, we use the detectability of outflows in Perseus to inform our
expectations in W5.  Since it appears that we can detect outflows from low-mass
protostars with sub-stellar to $\sim30L_\odot$ luminosities
at the distance of W5 and these objects should be the most numerous in a
standard initial mass function, the distribution of physical properties in W5
outflows should be similar to those in Perseus.  However, because W5 is a
somewhat more massive cloud ($M_{W5}\approx 5 M_{Perseus}$ \footnote{$M_{W5}$
is estimated from \thirteenco.  We also estimate the total molecular mass in W5 using
the X-factor and acquire $M_{W5}=5.0\ee{4}$ \msun, in agreement with
\citet{karr:triggered:2003}, who estimated a molecular mass of 4.4\ee{4} from
\twelveco\ using the same X-factor.  \citet{koenig:clustered:2008} estimated a
total gas mass of 6.5\ee{4} from a 2MASS extinction map.   The total molecular mass
in Perseus is $M_{Perseus} \sim 10^4$ \citep{bally-perseus2008}}), we might expect the
high-end of the distribution to extend to higher values of outflow mass,
momentum, and energy.  Since we will likely see clustered outflows confused
into a smaller number of distinct lobes, we expect a bias towards higher values
of the derived quantities but a lower detection rate.

\subsubsection{Velocity, Column Density, and Mass Measurements}
\label{sec:measurements}
Throughout this section, we assume that the CO
lines are optically thin and thermally excited.   The measured properties
are presented in Table \ref{tab:outflows}.  These assumptions are
likely to be invalid, so we also discuss the consequences of applying `typical'
optical depth corrections to the derived quantities.   Because we do
not measure optical depths and the optical depth correction for CO 3-2 is less
well quantified than for CO 1-0 \citep{curtis2010,Cabrit1990}\footnote{In
\citet{curtis2010}, this correction factor ranged from 1.8 to 14.3;
\citet{arce2010} did not enumerate the optical depth correction they
used but it is typically around 7 \citep{Cabrit1990}.  }, we only present the
uncorrected measurements in Table \ref{tab:outflowsderived}.

The outflow velocity ranges were measured by examining
both RA-velocity and Dec-velocity diagrams interactively using the STARLINK
GAIA data cube viewing tool.  The velocity limits are set to include
all outflow emission that is distinguishable from the cloud (i.e. the velocity
at which outflow lobes dominate over the gaussian wing of the cloud
emission) down to zero emission.  An outflow size \citep[or lobe size,
following ][]{curtis2010} was determined by integrating over the blue and red
velocity ranges and creating an elliptical aperture to include both peaks; the
position and size therefore have approximately beam-sized ($\approx18\arcsec$)
accuracy.  The integrated outflow maps are shown as red and blue contours in
Figure \ref{fig:pv2b}.  The velocity center was computed by fitting a
gaussian to the FCRAO \thirteenco\ spectrum averaged over the elliptical
aperture.

\Figure{figures_chw5/WidthHistogram}
{Histogram of the outflow line widths. {\it Black lines}: histogram of the measured
outflow widths (half-width zero-intensity, measured from the fitted central
velocity of the cloud to the highest velocity with non-zero emission).  {\it
Blue dashed lines}: outflow half-width zero-intensity (HWZI) for the outer arm (non-W5) sample.
{\it Solid red shaded}: The measured widths (HWHM) of the sub-regions as
tabulated in Table \ref{tab:regionspectra}.   
{\it Gray dotted}: Outflow $v_{max}$ (HWZI) values for Perseus
from \citet{curtis2010}. }
{fig:widthhist}{0.5}{0}


The column density is estimated from \twelveco\ J=3-2  assuming local thermal
equilibrium (LTE) and optically thin emission using the equation 
$ N(\hh) =
5.3\ee{18}\eta_{mb}^{-1} \int T_A^*(v) dv $ for $T_{ex}=20$ K. 
The derivation is given in the Appendix.
The column density in the lobes is likely to be dominated by low-velocity gas
and therefore our dominant uncertainty may be missing low-velocity emission
rather than poor assumptions about the optical depth.

The scalar momentum and energy were computed from
\begin{equation}
      p = M \frac{\sum T_A^*(v) (v-v_{c}) \Delta v}{ \sum T_A^*(v) \Delta v}
\end{equation}
\begin{equation}
      E = \frac{M}{2} \frac{\sum T_A^*(v) (v-v_{c})^2 \Delta v}{ \sum T_A^*(v) \Delta v}
\end{equation}
where $v_c$ is the \thirteenco\ 1-0 centroid velocity.  The same 
assumptions used in determining column density are applied here.

We
estimate an outflow lifetime by taking half the distance between the red and
blue outflow centroids divided by the maximum measured velocity difference
($\Delta v_{max} = (v_{max,red}-v_{max,blue})/2$), $\tau_{flow} = L_{flow} / ( 2 \Delta
v_{max})$, where $L_{flow}$ refers to the length of the flow.  This method
assumes that the outflow inclination is 45\arcdeg; if it is more parallel to
the plane of the sky, we overestimate the age, and vice-versa.  The momentum
flux is then $\dot{P} = p / \tau$.  Similarly, we compute a mass loss rate by
dividing the total outflow mass by the dynamical age, which yields what is
likely a lower limit on the mass loss rate (if the lifetime is underestimated,
the mass loss rate is overestimated, but the outflow mass is always a lower
limit because of optical depth and confusion effects).    

The dynamical ages are highly suspect since the red and blue lobes are often
unresolved or barely resolved, and diffuse emission averaged with the lobe
emission can shift the centroid position.  Additionally, it is not clear what
portion of the outflow corresponds to the centroid: the bow shock or the jet could both potentially
dominate the outflow emission.  \citet{curtis2010} discuss the many ways in
which the dynamical age can be in error.  
Our mass loss rates are similar to those in Perseus \emph{without} correcting our 
measurements for optical depth, while our outflow masses are an order of magnitude lower.
It therefore appears that our dynamical age estimates must be too low, since we have no 
reason to expect protostars in W5 to be undergoing mass loss at a greater rate than those in
Perseus.
However, given more reliable
dynamical age estimates from higher resolution observations of shock tracers,
the mass loss rates could be corrected and compared to other star-forming
regions.

Because the emission was assumed to be optically thin, the mass, column,
energy, and momentum measurements we present are strictly lower limits.  While
some authors have computed correction factors to \twelveco\ 1-0 optical depths
\citep[e.g.][]{Cabrit1990},  the corrections are different for the 3-2
transition \citep[1.8 to 14.3,][]{curtis2010}.  Additionally, CO 3-2 may
require a correction for sub-thermal excitation because of its higher critical
density (the CO 3-2 critical density is 27 times higher than CO 1-0; see
Appendix \ref{appendix:dipole} for modeling of this effect).

Additionally, most of the outflow mass is at the lowest distinguishable
velocities in typical outflows \citep[e.g.][]{arce2010}.  It is therefore
plausible that in the more turbulent W5 region, a greater fraction of the
outflow mass is blended (velocity confused) with the cloud and therefore not
included in mass, momentum, and energy measurements.  This omission could be
greater than the underestimate due to poor opacity assumptions.

The total mass of the W5 outflows is $M_{tot}\approx1.5 \msun$,
substantially lower, even with an optical depth correction of $10\times$, than
the 163 \msun\ reported in Perseus \citep{arce2010}.  \citet{arce2010} also include
a correction factor of 2.5 to account for higher temperatures in outflows and a
factor of 2 to account for emission blended with the cloud.  The temperature
correction is inappropriate for CO 3-2 (see Appendix \ref{appendix:dipole},
Figure \ref{fig:approx}), but the resulting total outflow mass in W5 with an
optical depth correction and a factor of 2 confusion correction is about 30
\msun.  In order to make our measurements consistent with a mass of 160 \msun\ , a density
upper limit in the outflowing gas of $n(\hh) < 10^{3.5} \percc$ is required,
since a lower gas density results in greater mass for a given intensity (see
Appendix \ref{appendix:dipole}, Figure \ref{fig:coradex}).  However, we
expect the total outflow mass in W5 to be greater than in Perseus because 
of the greater cloud mass, implying that the density in the flows must be even
lower, or additional corrections are needed.

The total outflow momentum is $p_{tot}\approx10.9 \msun$ \kms, versus a quoted
517 \msun \kms\ in Perseus \citep{arce2010}.  \citet{arce2010} included
inclination and dissociative shock corrections for the momentum measurements
on top of the correction factors already applied to the mass.  If these
corrections are removed from the Perseus momentum total (except for optical depth,
which is variable in their data and therefore cannot be removed), the uncorrected outflow
momentum in Perseus would be about 74 \msun \kms. The W5 outflow momentum, if
corrected with a `typical' optical depth in the range 7-14, would match or exceed
this value.  If an additional CO 3-2 excitation correction (in the range 1-20)
is applied, the W5 outflow momentum would significantly exceed that in Perseus.

Assuming a turbulent line width $\Delta v \sim 3$ \kms\ (approximately the
smallest FWHM line-width observed), the total turbulent momentum in the ambient
cloud is $p = M_{tot} \Delta v = 1.3\ee{5} \msun$ \kms, which is $\sim10^5$
times the measured outflow momentum - the outflows detected in our
survey cannot be the sole source of the observed turbulent line widths, even if
corrected for optical depth and missing mass.  

Table \ref{tab:regionspectra} presents the turbulent momentum for each
sub-region computed by multiplying the measured velocity width by the
integrated \thirteenco\ mass.  Even if the outflow measurements are 
orders of magnitude low because of optical depth, cloud blending, sub-thermal
excitation, and other missing-mass considerations, outflows contribute
negligibly to the total momentum of high velocity gas in W5.  This result is
unsurprising, as there are many other likely sources of energy in the region
such as stellar wind bubbles and shock fronts between the ionized and molecular
gas.  Additionally, in regions unaffected by feedback from the HII region (e.g.
W5NW), cloud-cloud collisions are a possible source of energy.

Figure \ref{fig:outflowhist} displays the distribution of measured properties
and compares them to those derived in the COMPLETE \citep{arce2010} and
\citet{curtis2010} HARP CO 3-2 surveys of Perseus.  Our derived masses are
substantially lower than those in \citet{arce2010} even if corrected for
optical depth, but our momenta are similar to the CPOC (COMPLETE Perseus
Outflow Candidate) sample and our energies are higher, indicating a bias
towards detecting mass at high velocities.  The bias is more heavily towards
high velocities than the CO 1-0 used in \citet{arce2010}.  The discrepancy
between our values and those of \citet{arce2010} and \citet{curtis2010} can be
partly accounted for by the optical depth correction applied in those works:
\thirteenco\ was used to correct for opacity at low velocities, where most of
the outflow mass is expected.  Those works may also have been less affected by
blending because of the smaller cloud line widths in Perseus.

%\Figure{OutflowHistograms}{Histograms of outflow properties.}{fig:outflowhist}{1.0}
\FigureFour{figures_chw5/OutflowMassHistogram_legend}{figures_chw5/OutflowEnergyHistogram}{figures_chw5/OutflowMomentumHistogram}{figures_chw5/OutflowColumnHistogram}
{Histograms of outflow physical properties.  
%All display a peak at low values in addition
%to a high excess.  None are obviously well-fit by power-law or gaussian distributions.
The solid unfilled lines are the W5 outflows (this paper), the forward-slash
hashed lines show \citet{arce2010} CPOCs , the dark gray
shaded region shows \citet{arce2010} values for known outflows in Perseus, and
the light gray, backslash-hashed regions show \citet{curtis2010} CO 3-2 outflow
properties.  The outflow masses measured in Perseus are systematically higher
partly because both surveys corrected for line optical depth using \thirteenco.
The medians of the distributions are 0.017, 0.044, 0.33, and 0.14 \msun\ for
W5, Curtis, Arce Known, and Arce CPOCs respectively, which implies that an
optical depth and excitation correction factor of 2.5-20 would be required to
make the distributions agree (although W5, being a more massive region, might
be expected to have more massive and powerful outflows).  It is likely that CO
3-2 is sub-thermally excited in outflows, and CO outflows may be destroyed by
UV radiation in the W5 complex while they easily survive in the lower-mass
Perseus region, which are other factors that could push the W5 mass
distribution lower.
}
{fig:outflowhist}

The momentum flux and mass loss rate are compared to the values derived in
Perseus by \citet{hatchell2007} and \citet{curtis2010} in Figures
\ref{fig:outflowPflux} and \ref{fig:outflowmdot}.  Both of our values are
computed using the dynamical timescale $\tau_d$ measured from outflow lobe
separation, while the \citet{hatchell2007} values are derived using a more
direct momentum-flux measurement in which the momentum flux contribution 
of each pixel in the resolved outflow map is considered.  
The derived
momentum fluxes (Figure \ref{fig:outflowPflux}) are approximately consistent
with the \citet{curtis2010} Perseus momentum fluxes; \citet{curtis2010} measure
momentum fluxes in a range $1\ee{-6}<\dot{P}<7\ee{-4}$ \msun \kms \peryr,
higher than our measured $6\ee{-7}<\dot{P}<1\ee{-4}$ \msun \kms \peryr\ by
approximately the opacity correction they applied.  As seen in Figure
\ref{fig:outflowPflux}, the \citet{hatchell2007} momentum flux measurements in
Perseus cover a much lower range $6\ee{-8}<\dot{P}<2\ee{-5}$ \msun \kms \peryr\
and are not consistent with our measurements.  This disagreement is most likely
because of the difference in method.  The W5 outflows exhibit
substantially higher mass-loss rates and momentum fluxes if we assume a factor
of 10 opacity correction, as expected from our bias toward higher-velocity,
higher-mass flows.

\Figure{figures_chw5/OutflowMomentumFluxHistogram}
{Histogram of the measured outflow momentum fluxes.  The black thick line shows
our data, the grey shaded region shows the \citet{hatchell2007} data, and the
hatched region shows \citet{curtis2010} values.  Our measurements peak squarely
between the two Perseus JCMT CO 3-2 data sets, although the \citet{curtis2010}
results include an opacity correction that our data do not, suggesting that our
results are likely consistent with \citet{curtis2010} but inconsistent with the
\citet{hatchell2007} direct measurement method.}
{fig:outflowPflux}{0.5}{0}

\Figure{figures_chw5/OutflowMassLossRateHistogram}
{Histogram of the measured mass loss rate.  The black thick line shows our
data, while the grey shaded region shows the \citet{hatchell2007} data, which
is simply computed by $\dot{M} = \dot{P} \times 10 / 5$ \kms, where the factor
of 10 is a correction for opacity.  Our mass loss rates are very comparable to
those of \citet{hatchell2007}, but different methods were used so the
comparison may not be physically meaningful.  \citet{curtis2010} (hatched) used
a dynamical time method similar to our own and also derived similar mass loss
rates, although their mass measurements have been opacity-corrected using the
\thirteenco\ 3-2 line.  Because our mass loss rates agree reasonably with
Perseus, but our outflow mass measurements are an order of magnitude low, we
believe our dynamical age estimates to be too small.
}
{fig:outflowmdot}{0.5}{0}


\subsection{Structure of the W5 molecular clouds: A thin sheet?}
The W5 complex extends $\sim 1.6\arcdeg \times 0.7 \arcdeg$ within 20\arcdeg\
of parallel with the galactic plane.  At the assumed distance of 2 kpc, it has
a projected length of $\sim60$ pc (Figure \ref{fig:outflows_on_co32}).
%% In the far-infrared (Figure\ref{fig:outflows_on_IRAS100}) and %%
In the 8 \um\ band (Figure \ref{fig:color_overview}), the region appears to
consist of two blown-out bubbles with $\sim 10-15$ pc radii centered on
$\ell=138.1, b=1.4$ and $\ell=137.5,b=0.9$.  While the bubbles are filled in
with low-level far-infrared emission, there is no CO detected down to a
$3-\sigma$ limit of 3.0 K \kms\ (\twelveco\ 1-0), 2.4 K \kms\ (\twelveco\ 3-2,
excepting a few isolated clumps), and 1.5 K \kms\ (\thirteenco\ 1-0).  Using the 
X-factor (the CO-to-\hh\ conversion factor) for \twelveco\, $N(\hh) =
3.6\ee{20} \persc / (\mathrm{K}~\kms)$, we derive an upper limit $N(\hh) <
1.1\ee{21}\persc$, or $A_V \lesssim 0.6$.  Individual `wisps' and `clumps' of
CO can sometimes be seen, particularly towards the cloud edges, but in general
the bubbles are absent of CO gas.

% The 
% strictest column limit comes from the \twelveco\ 3-2 observations (assuming
% LTE, $\tau<<1$, and T$_{ex}=20K$), with a $3-\sigma$ column limit $N_{\hh} <
% 7.8\ee{18}$ within an 18\arcsec\ beam, or $A_V \lesssim 0.1$ magnitudes
% \citep[using the][conversion $A_V=1.9\ee{21} \persc$]{Bohlin1978}\footnote{ The
% \twelveco\ 1-0 line gives a limit of 3.8\ee{19} \persc\ or $A_V=0.02$ in a
% 45\arcsec\ FWHM beam.  This limit may be more restrictive than the CO 3-2 limit
% if CO 3-2 is sub-thermally excited, though at this resolution the CO 3-2 limit
% is $N(\hh) < 3.1\ee{18}$.}.  

Given such low column limits,  the W5 cloud must be much smaller along the line
of sight than its $\sim50$ pc size projected on the sky.  Alternately, along
the line-of-sight, the columns of molecular gas are too low for CO to
self-shield, and it is therefore destroyed by the UV radiation of W5's O-stars.
In either case, there is a significant excess of molecular gas in the plane of
the sky compared to the line of sight, which makes W5 an excellent location to
perform unobscured observations of the star formation process.  The implied
thin geometry of the W5 molecular cloud may therefore be similar to the bubbles
observed by \citet{Beaumont2009}, but on a larger scale.

There is also morphological evidence supporting the face-on
hypothesis.  In the AFGL 4029 region (Section \ref{sec:afgl4029}) and all along
the south of W5, there are ridges with many individual cometary `heads'
pointing towards the O-stars that are unconfused along the line of sight.  This
sort of separation would not be expected if we were looking through the clouds
towards the O-stars.  W5W, however, presents a counterexample in which there
are two clouds along the line of sight that may well be masking a more complex
geometry.

% \subsection{\thirteenco\ 1-0}
% The \thirteenco\ integrated intensity was used to compute the \hh\ column using
% \begin{equation}
%     N_{\hh} = 1.694\ee{20} / (1-e^{(-5.3/T_{ex})}) \int T_{A}^* (v)
%     dv\ \persc,
% \end{equation} 
% where we have assumed the excitation temperature is the gas kinetic temperature
% $T_{ex} = T_K = 20$ K.  We do not include a main-beam efficiency correction
% because the factor was unknown; this omission may introduce a systematic
% underestimate of up to $\sim40\%$ (based on the \twelveco\ 1-0 $\eta_{MB}$).
% The 1-$\sigma$ column sensitivity limit is 4.4\ee{20} \persc\ or 0.16 \msun for
% an assumed 2 kpc distance in 22\farcs5 square pixels.

\section{Sub-regions}
\label{sec:subregions}

Individual regions were selected from the mosaic for comparison.  All regions
with multiple outflows and indicators of star formation activity were named and
included as regions for analysis.  Additionally, three ``inactive'' regions were
selected based on the presence of \thirteenco\ emission but the lack of
outflows in the \twelveco\ 3-2 data.  Finally, two regions devoid of CO emission
were selected as a baseline comparison and to place upper limits on the molecular
gas content of the east and west `bubbles'.  The regions are identified on the
integrated \thirteenco\ image in Figure \ref{fig:regionboxes_on_CO}. 

%Cloud velocity widths were measured in a range 3-7 \kms (full-width
%zero-intensity).  This range is inferred from by-eye inspection of
%position-velocity diagrams and 1d gaussian fits to a few (arbitrary) lines of
%sight. 
Average spectra were taken of each ``region'' within the indicated box.
Gaussians were fit to the spectrum to determine line-widths and centers (Figure
\ref{fig:regionspectra}, Table \ref{tab:regionspectra}).  Gaussian fits were
necessary because in many locations there are at least two velocity components,
so the second moment (the ``intensity-weighted dispersion'') is a poor
estimator of line width.  Widths ranged from $v_{FWHM} = 2.3$ to 6.2 \kms\
(Figure \ref{fig:widthhist}).  


%\Table{cccccccccc}{Gaussian fit parameters of sub-regions}
{{Region} & {Velocity 1} & {Width 1} & {Amplitude 1} & {Velocity 2} & {Width 2} & {Amplitude 2} &  &  & \\
 & {(\kms)} & {(FWHM, \kms)} & {(K)} & {(\kms)} & {(FWHM, \kms)} & {(K)} &  & \\}
{tab:regionspectra}
{
S201 & -38.04 & 3.149 & 2.35 & - & - & -\\
AFGL4029 & -38.91 & 3.3605 & 1.48 & - & - & -\\
LWCas & -38.83 & 3.478 & 2.33 & - & - & -\\
W5W & -41.37 & 3.8775 & 3.07 & -36.16 & 3.8305 & 1.90\\
W5NW & -36.37 & 3.854 & 1.6 & - & - & -\\
W5NWpc & -36.37 & 3.713 & 1.19 & -41.81 & 4.3475 & 0.47\\
W5SW & -42.78 & 4.136 & 0.6 & -36.34 & 4.183 & 0.22\\
W5S & -40.15 & 2.914 & 0.34 & -35.76 & 2.2795 & 0.40\\
Inactive1 & -42.91 & 2.6555 & 0.75 & -39.38 & 4.2065 & 0.42\\
Inactive2 & -38.94 & 3.7365 & 1.2 & - & - & -\\
empty & -37.81 & 5.217 & 0.04 & - & - & -\\
\hline \\
\thirteenco\ fits &&&&&&& \thirteenco      & \thirteenco          & \thirteenco       \\
                   &&&&&&&              mass &              momentum &              energy\\
                   &&&&&&&(\msun)           &(\msun \kms)         &(ergs)              \\
\hline \\
S201 & -37.97 & 2.5615 & 0.56 & - & - & - & 1300 & 3500 & 8.9\ee{46}\\
AFGL4029 & -38.66 & 2.35 & 0.35 & - & - & - & 2600 & 6100 & 1.4\ee{47}\\
LWCas & -38.75 & 2.679 & 0.51 & - & - & - & 3700 & 10000 & 2.7\ee{47}\\
W5W & -41.23 & 2.773 & 1.09 & -36.51 & 3.5485 & 0.47 & 4500 & 13000 & 3.5\ee{47}\\
W5NW & -36.1 & 3.431 & 0.7 & - & - & - & 5300 & 18000 & 6.3\ee{47}\\
W5NWpc & -36.18 & 3.3135 & 0.42 & -41.44 & 3.619 & 0.14 & 15000 & 50000 & 1.6\ee{48}\\
W5SW & -42.6 & 3.807 & 0.1 & -36.15 & 4.2535 & 0.05 & 790 & 3000 & 1.1\ee{47}\\
W5S & -39.9 & 2.444 & 0.07 & -35.48 & 2.209 & 0.08 & 320 & 790 & 1.9\ee{46}\\
Inactive1 & -42.58 & 2.5145 & 0.1 & -38.97 & 2.82 & 0.07 & 1400 & 3500 & 8.7\ee{46}\\
Inactive2 & -38.82 & 3.196 & 0.37 & - & - & - & 3100 & 9900 & 3.2\ee{47}\\
empty & -38.44 & 4.7705 & 0.02 & - & - & - & 340 & 1600 & 7.8\ee{46}\\
}{%empty comment
}
 %tab:regionspectra
%\include{regiontable_mnras} % tab:regiontable

\Figure{figures_chw5/regionspectra_grid}
{Spatially averaged spectra of the individual regions analyzed.  \twelveco\ 3-2
is shown by thick black lines and \thirteenco\ 1-0 is shown by thin red lines.
Gaussian fits are overplotted in blue and green dashed lines, respectively.
The fit properties are given in Table
\ref{tab:regionspectra}.}{fig:regionspectra}{0.5}{0}


\subsection{Sh 2-201}
Sh 2-201 is an HII region and is part of the same molecular cloud as the
bright-rimmed clouds in W5E, but it does not share a cometary shape with these
clouds (Figure \ref{fig:S201}).  Instead, it is internally heated and has its
own ionizing source \citep{Felli1987}.  The AFGL 4029 cloud edge is at a projected
distance of $\sim7$ pc from the nearest exposed O-star, and the closest
illuminated point in the Spitzer 8 and 24 \um\ maps is at a projected distance
of $\sim 5$ pc.  The star forming process must therefore have begun before
radiation driven shocks from the W5 O-stars could have impacted the cloud.  
%There are 4 bipolar CO outflows
%associated with this region.  The outflows come primarily from the 8 \um- faint
%part of the region, perhaps because the 8 \um\ emission traces the edges of a
%blown out gas-poor cavity.

\Figure{figures_chw5/S201_CO32on8UM} %{regionoverlays/S201_BGPSon24UM}
{Small scale map of the Sh 2-201 region plotted with CO 3-2 contours integrated
from -60 to -20 \kms\ at levels 3,7.2,17.3,41.6, and 100 K \kms.  
% (left) The MIPS 24 \um\ image is
% displayed inverted in log scale from \lowmips\ to \highmips\ MJy \persr.
The IRAC 8 \um\ image is displayed in inverted log scale from \lowirac\
to \highirac\ MJy \persr. Contours of the CO 3-2 cube integrated from
-60 to -20 \kms\ are overlaid at logarithmically spaced levels from 3 to 100 K
\kms\ (3.0,7.2,17.3,41.6,100; $\sigma\approx0.7$ K \kms).  The
ellipses represent the individual outflow lobe apertures mentioned in Section
\ref{sec:measurements}.
}{fig:S201}{0.5}{0}

\subsection{AFGL 4029}
\label{sec:afgl4029}
% do I actually add any interesting information about AFGL 4029?
AFGL 4029 is a young cluster embedded in a cometary cloud (Figure
\ref{fig:afgl4029}).  There is one clear bipolar outflow and 6 single-lobed
flows that cannot be unambiguously associated with an opposite direction
counterpart.  The cluster is mostly unresolved in the data presented here and
is clearly the most active CO clump in W5.  It contains a cluster of at least 30 B-stars
\citep{Deharveng1997}.  The outflows from this region have a full width $\Delta
v \approx 30$ \kms, which is entirely inconsistent with a radiation-driven
inflow or outflow since it is greater than the sound speed in the ionized
medium.

The northeast cometary cloud is strongly affected by
the W5 HII region.  It has an outflow in the head of the cloud (Figure
\ref{fig:necomet}), and the cloud shows a velocity gradient with distance from
the HII region.  The polarity of the gradient suggests that the cometary cloud
must be on the far side of the ionizing O-star along the line of sight assuming
that the HII region pressure is responsible for accelerating the cloud edge.

\Figure{figures_chw5/AFGL4029_CO32on8UM}%{regionoverlays/AFGL4029_BGPSon24UM}
{Small scale map of the AFGL 4029 region plotted with CO 3-2 contours integrated
from -60 to -20 \kms\ at levels 3,7.2,17.3,41.6, and 100 K \kms.  
The IRAC 8 \um\ image is displayed in inverted log scale from \lowirac\ to
\highirac\ MJy \persr. Contours of the CO 3-2 cube integrated from -60 to -20
\kms\ are overlaid at logarithmically spaced levels from 3 to 100 K \kms\
(3.0,7.2,17.3,41.6,100; $\sigma\approx0.7$ K \kms).  Outflows 26-32 are ejected
from a forming dense cluster.  A diagram displaying the kinematics of the
northern cometary cloud is shown in Figure \ref{fig:necomet}. }
{fig:afgl4029}{0.5}{0}


\Figure{figures_chw5/NECometContours}
{The northeast cometary cloud.  Contours are shown at 0.5,1,2, and 5 K \kms\
integrated over the ranges -44.0 to -41.9 \kms\ (blue) and -38.1 to -35.6 \kms\
(red).  There is a velocity gradient across the tail, suggesting that the front
edge is being pushed away along the line of sight.}
{fig:necomet}{0.5}{0}

\subsection{W5 Ridge}
\label{sec:w5ridge}
The W5 complex consists of two HII region bubbles separated by a ridge of
molecular gas (Figure \ref{fig:lwcas}).  This ridge contains the LW Cas optical
nebula, a reflection nebula around the variable star LW Cas, on its east side
and an X-shaped nebula on the west.  The east portion of LW Cas Nebula is
bright in both the continuum and CO J=3-2 but lacks outflows (see Figure
\ref{fig:lwcas}).  The east portion also has the highest average peak antenna
temperature, suggesting that the gas temperature in this region is
substantially higher than in the majority of the W5 complex (higher spatial
densities could also increase the observed $T_A$, but the presence of nearby heating
sources make a higher temperature more plausible).  It is possible
that high gas temperatures are suppressing star formation in the cloud.
Alternately, the radiation that is heating the gas may destroy any outflowing
CO, which is more likely assuming the two Class I objects identified in this
region by \citet{koenig:clustered:2008} are genuine protostars.

\Figure{figures_chw5/LWCas_CO32on24UM} %{regionoverlays/LWCas_BGPSon8UM}
{Small scale map of the LW Cas nebula plotted with CO 3-2 contours integrated
from -60 to -20 \kms\ at levels 3,7.2,17.3,41.6, and 100 K \kms.  The feature containing outflows 20 and
21 is the X-shaped ridge referenced in Section \ref{sec:w5ridge}.  This
sub-region is notable for having very few outflows associated with the most
significant patches of CO emission.   The gas
around it is heated on the left side by the O7V star HD 18326 ($D_{proj}=8.5$
pc), suggesting that this gas could be substantially warmer than the other
molecular clouds in W5.
}{fig:lwcas}{1.0}{0}

The ridge is surprisingly faint in HI 21 cm emission compared to the two HII regions
(Figure \ref{fig:HIridge}) considering its 24 \um\ surface brightness.  The
integrated HI intensity from -45 to -35 \kms\ is $\sim800$ K \kms, whereas in
the HII region bubble it is around 1000 K \kms.  The CO-bright regions show
lower levels of emission similar to the ridge at 700-800 K \kms.  However, the
ridge contains no CO gas and very few young stars \citep[Figure 7 in
][]{koenig:clustered:2008}.  It is possible that the ridge contains cool HI but
has very low column-densities along the direction pointing towards the O-stars,
in which case the self-shielding is too little to prevent CO dissociation.
This ridge may therefore be an excellent location to explore the transition
from molecular to atomic gas under the influence of ionizing radiation in
conditions different from high-density photodissociation (photon-dominated)
regions.

\FigureTwo{figures_chw5/HI_IRAS100_overlay}{figures_chw5/24micron_21cmcontOverlay}
{{\it Top:} The DRAO 21 cm HI map integrated from -45 to -35 \kms\ displayed in grayscale
from 700 (black) to 1050 (white) K \kms\ with IRAS 100 \um\ contours (red, 40 MJy sr$^{-1}$) and
\twelveco\ 1-0 contours integrated over the same range (white, 4 K \kms)
overlaid.  The ridge of IRAS 100 \um\ emission at $\ell=138.0$ coincides with a
relative lack of HI emission at these velocities, suggesting either that there
is less or colder gas along the ridge.  {\it Bottom:} The Spitzer 24 \um\ map
with 21 cm continuum contours at 6, 8, and 10 MJy sr$^{-1}$ overlaid.  The IRAS contours
are also overlaid to provide a reference for comparing the two figures and to demonstrate
that the HII region abuts the cold-HI area.  The moderate
excess of continuum emission implies a somewhat higher electron density along the 
line of sight through the ridge.}
{fig:HIridge}{1.0}

We examine Outflow 20 as a possible case for pressure-driven implosion
(radiation, RDI, or gas pressure, PDI) by examining the relative timescales of the outflow
driving source and the HII-region-driven compression front.  A typical
molecular outflow source (Class 0 or I) has a lifetime of $\sim5\ee{5}$ years
\citep{Evans2009}.   Given that there is an active outflow at the head of this
cloud, we use 0.5 MYr as an upper limit.  The approximate distance from this
source to the cloud front behind it is $\sim 3.3$ pc.  If we assume the cloud
front has been pushed at a constant speed $v\leq c_{II} \approx 10 \kms$,  we
derive a lower limit on its age of 0.3 MYr.  While these limits allow for the
protostar to be older than the compression front by up to 0.2 MYr, it is likely
that the compression front moved more slowly (e.g., 3 \kms\ if it was pushed
by a D-type shock front) and that the protostar is not yet at the end of its
lifetime - it is very plausible that this soure was born in a radiation-driven
implosion.


\subsection{Southern Pillars}
\label{sec:pillars}
There are 3 cometary clouds that resemble the ``elephant trunk'' nebula in IC
1396 (Figure \ref{fig:comets}).  Each of these pillars contains evidence of at
least one outflow in the head of the cloud (see the supplementary materials, outflows
16-19 and 38)
%Figures \ref{fig:pv16},\ref{fig:pv17}, \ref{fig:pv18}, \ref{fig:pv19}, and \ref{fig:pv38}).  
These pillars are low-mass and isolated; there is no other outflow activity in
southern W5.  However, because of the bright illumination on their northern
edges and robust star formation tracers, these objects present a reasonable 
case for triggered star formation by the RDI mechanism.

The kinematics of these cometary clouds suggest that they have been pushed in
different directions by the HII region (Figure \ref{fig:comets}).  The central
cometary cloud (Figure \ref{fig:comets}b) has two tails.  The southwest tail
emission peaks around -39.5 \kms\ and the southeast tail peaks at -41.5 \kms,
while the head is peaked at an intermediate -40.5 \kms.  These velocity shifts
suggest that the gas is being accelerated perpendicular to the head-tail axis
and that the southeast tail is on the back side of the cometary head, while the
southwest tail is on the front side.  The expanding HII region is crushing this
head-tail system.

The southeast cometary cloud (Figure \ref{fig:comets}a) peaks at -35.0 \kms.
There are no clearly-separated CO tails as in the central cloud, but there is a
velocity shift across the tail, in which the west (right) side is blueshifted
compared to the east (left) side, which is the opposite sense from the central
cometary cloud.

The southwest cometary cloud (Figure \ref{fig:comets}c) peaks at -40.3 \kms\
and has weakly defined tails similar to the central cloud.  Both of its tails
are at approximately the same velocity (-42.5 \kms).

The kinematics of these tails provide some hints of their 3D structure and
location in the cloud.  Future study to compare the many cometary flows in W5
to physical models and simulations is warranted.  Since these flows are likely
at different locations along the line of sight (as required for their different
velocities), analysis of their ionized edges may allow for more precise
determination of the full 3D structure of the clouds relative to their ionizing
sources.

\Figure{figures_chw5/SouthComet_VelocityContours}
{CO 3-2 contours overlaid on the Spitzer 8 \um\ image of the W5S cometary
clouds described in Section \ref{sec:pillars}.  Contours are color-coded by velocity
and shown for 0.84 \kms\ channels at levels of 1 K (a, b) and 0.5 K (c).  The
velocity ranges plotted are (a) -41.5 to -33.0 \kms (b) -44.7 to -36.7 \kms\
(c) -43.6 to -35.6 \kms.  The labels show the minimum, maxmimum, and middle
velocities to guide the eye.  The grey boxes indicate the regions selected for
CO contours; while there is CO emission associated with the southern 8 \um\
emission, we do not display it here. The velocity gradients are discussed in
Section \ref{sec:pillars}.
}
{fig:comets}{0.5}{0}

%\Figure{regionoverlays/W5S_CO32on24UM}%{regionoverlays/W5S_BGPSon8UM}
%{Small scale map of the W5 S region showing BGPS 1.1 mm contours overlaid on
%the Spitzer 24 \um\ map.  The 1.1 mm sources are significantly brighter in this
%region than in the W5SE region, suggesting that they are either very significantly
%more condensed (reducing the impact of Bolocam's effective filter function) or they
%are hotter.
%}{fig:w5s}{1.0}

\subsection{W5 Southeast}
The region identified as W5SE has very little star formation activity despite
having significant molecular gas (M$_{\thirteenco} \sim 800$\msun).  While
there are two outflows and two Class I objects \citep{koenig:clustered:2008} in the
southeast of the two clumps ($\ell=138.15,b=0.77$, Figure \ref{fig:w5SE}), the main clump
($\ell=138.0,b=0.8$) has no detected outflows.  The CO emission is particularly
clumpy in this region, with many independent, unresolved clumps both in
position and velocity.  In the 8 and 24 micron Spitzer images, it is clear that
these clouds are illuminated from the northwest.  This region represents a case
in which the expanding HII region has impacted molecular gas but has not
triggered additional star formation.  The high clump-to-clump velocity
dispersion observed in this region may be analogous to the W5S cometary clouds
(Section \ref{sec:pillars}) but without condensed clumps around which to form
cometary clouds.

\Figure{figures_chw5/W5SE_CO32on8UM}
{Small scale map of the W5 SE region showing the star-forming clump containing
outflows 39 and 40 and the non-star-forming clump at $\ell=138.0,b=0.8$. 
CO 3-2 contours integrated from -60 to -20 \kms\ are displayed at levels
3,7.2,17.3,41.6, and 100 K \kms.
}{fig:w5SE}{0.5}{0}

\subsection{W5 Southwest}
There is an isolated clump associated with outflows in the southwest part of W5
(Figure \ref{fig:w5SW}) at $v_{LSR} \sim -45~\kms$.  While this clump is likely
to be associated with the W5 region, it shows little evidence of interaction
with the HII region.  If it is eventually impacted by the expanding ionization
front (i.e. if it is within the W5 complex), this clump will be an example of
``revealed'', not triggered, star formation.  

The other source in W5SW is a cometary cloud with a blueshifted head and
redshifted tail (Figure \ref{fig:swcomet}; Outflow 13).  The head contains a
redshifted outflow; no blueshifted counterpart was detected (the velocity
gradient displayed in Figure \ref{fig:swcomet} is smaller than the outflow
velocity and is also redshifted away from the head).  The lack of a blueshifted
counterpart may be because the flow is blowing into ionized gas where the CO is
dissociated.

Because of its evident interaction with the HII region, this source is an
interesting candidate for a non-protostellar outflow impersonator.  However,
because the head is blueshifted relative to the tail, we can infer that the
head has been accelerated towards us by pressure from the HII region, implying
that it is in the foreground of the cloud.  Given this geometry, a
radiation-driven flow would appear blueshifted, not redshifted, as the detected
flow is.  Additionally, the outflow is seen as fast as 7.5 \kms\ redshifted
from the cloud, which is a factor of 2 too fast to be driven by radiation in a
standard D-type shock.  Finally, the outflow velocity is much greater than seen
in a simulation of a cometary cloud by \citet{Gritschneder2010}, while the
head-to-tail velocity gradient is comparable.


\Figure{figures_chw5/W5SW_CO32on8UM}%{regionoverlays/W5SW_BGPSon24UM}
{Small scale map of the W5 SW region plotted with CO 3-2 contours integrated
from -60 to -20 \kms\ at levels 3,7.2,17.3,41.6, and 100 K \kms. Outflow 13 is at the head of a 
cometary cloud (Figure \ref{fig:swcomet}) and therefore has clearly been
affected by the expanding HII region, but the region including bipolar Outflow
10 shows no evidence of interaction with the HII region. 
}{fig:w5SW}{0.5}{0}

\Figure{figures_chw5/SWComet_VelocityContours}
{The cometary cloud in the W5 Southwest region (Outflow 13).  Contours are
shown at 1 K for 0.84 \kms\ wide channels from -37.2 \kms\ (blue) to -30.5
\kms\ (red).  The head is clearly blueshifted relative to the tail and contains
a spatially unresolved redshifted outflow.}
{fig:swcomet}{0.5}{0}


\subsection{W5 West / IC 1848}
\label{sec:i02459}
There is a bright infrared source seen in the center of W5W (IRAS 02459+6029;
Figure \ref{fig:w5w}), but the nearest CO outflow lobe is $\approx1$ pc away.
The nondetection may be due to confusion in this area: there are two layers of
CO gas separated by $\sim$5 \kms, so low-velocity outflow detection is more
difficult. % the minimum detectable outflow velocity in this region is $\sim10$ \kms.
Unlike the rest of the W5 complex, this region appears to have multiple
independent confusing components along the line of sight (Figure
\ref{fig:regionspectra}), and therefore the CO data provide much less
useful physical information (multiple components are also observed in the
\thirteenco\ data, ruling out self-absorption as the cause of the multiple
components).

%The presense of two separate sheets confuses the physical association in this
%region.  A velocity gradient is present in both sheets, but they are less
%separated in velocity on the side nearest the W5 O-stars, so it is unlikely
%that the sheets have been blown away from each other.

%The BGPS source G136.828+01.064 ($M_{1.1 mm}\sim220\msun$) is associated with a
%deeply embedded cluster and 5 Class I objects from
%\citet{koenig:clustered:2008}.  This location marks the worst velocity
%confusion in CO but contains hints that there may be outflows.  This region is a
%good candidate for high-resolution follow-up.


\Figure{figures_chw5/W5W_8um_TwoVelocityOverlay}%{regionoverlays/W5W_BGPSon24UM}
{Small scale map of the W5 W region.  The IRAC 8 \um\ image is displayed in
inverted log scale from \lowirac\ to \highirac\ MJy \persr.  Contours of the CO
3-2 cube integrated from -50 to -38 \kms\ (blue) and -38 to -26 \kms\ (red) are
overlaid at levels 5,10,20,30,40,50,60 K \kms\ $\sigma\approx0.5$ K \kms.  The
lack of outflow detections is partly explained by the two spatially overlapping
clouds that are adjacent in velocity.
}{fig:w5w}{0.5}{0}


\subsection{W5 NW}
The northwest cluster  containing outflows 1-8 is at a
slightly different velocity ($\sim-35~\kms$) than the majority of the W5 cloud
complex ($\sim-38~\kms$; Figure \ref{fig:w5pv}), but it shares contiguous
emission with the neighboring W5W region.  % (see Figure \ref{fig:w5nwpercompare}).     
It contains many outflows and therefore is actively forming stars  (Figure
\ref{fig:w5nw}).  However, this cluster exhibits much lower CO brightness
temperatures and weaker Spitzer 8 \um\ emission than the ``bright-rimmed
clouds'' seen near the W5 O-stars. We therefore conclude that the region has
not been directly impacted by any photoionizing radiation from the W5 O-stars.

The lack of interaction with the W5 O-stars implies that the star formation in
this region, though quite vigorous, has not been directly triggered.  Therefore
not all of the current generation of star formation in W5 has been triggered on
small or intermediate scales (e.g., radiation-driven implosion).  Even the
``collect and collapse'' scenario seems unlikely here, as the region with the
most outflows also displays some of the smoothest morphology (Figures
\ref{fig:outflows_on_co32} and \ref{fig:w5nw}); in ``collect and collapse'' the
expansion of an HII region leads to clumping and fragmentation, and the spaces
between the clumps should be relatively cleared out.

\Figure{figures_chw5/pvdiagrams/w5_12co_latPV_outflows}
{Integrated longitude-velocity diagram of the W5 complex from $b=0.25$ to
$b=2.15$ in \twelveco\ 1-0 from the FCRAO OGS.  The W5NW region is seen at a
distinct average velocity around $\ell=136.5$, $v_{LSR}=-34$ \kms.  The red and
blue triangles mark the longitude-velocity locations of the detected outflows.
In all cases, they mark the low-velocity start of the outflow.}
{fig:w5pv}{0.5}{0}


\Figure{figures_chw5/W5NW_CO32on8UM}%{regionoverlays/W5NW_BGPSon24UM}
{Small scale map of the W5 NW region plotted with CO 3-2 contours integrated
from -60 to -20 \kms\ at levels 3,7.2,17.3,41.6, and 100 K \kms. Despite its
distance from the W5 O-stars, $D_{proj}\approx20$ pc, this cluster is the most
active site of star formation in the complex as measured by outflow activity.}
{fig:w5nw}{0.5}{0}

\section{Discussion}

\subsection{Comparison to other outflows}
\label{sec:comparison}
The outflow properties we derive are similar to those in the B0-star forming
clump IRAS 05358+3543 \citep[$M\approx600\msun$][]{Ginsburg2009}, in which CO
3-2 and 2-1 were used to derive outflow masses in the range 0.01-0.09 \msun.
However, some significantly larger outflows, up to 1.6 pc in one direction were
detected, while the largest resolved outflow in our survey was only 0.8 pc (one
direction).  
%With 54 detected outflows across a star-forming region that
%includes many young B-stars (and forming B-stars, like AFGL 4029) and contains
%an order of magnitude more mass than IRAS 05358+3543, the lack of long outflows
%is somewhat surprising.

As noted in Section \ref{sec:percompare}, 
the total molecular mass in W5 is larger than
Perseus, $M_{W5}\sim 4.5\ee{4} \msun$ while $M_{Perseus}\sim10^4\msun$
\citep{bally-perseus2008}.  The length distribution of outflows (Figure
\ref{fig:lengthhist}) is strikingly similar, while other physical properties
have substantially different mean values with or without correction factors
included.

The W5NW region is more directly comparable to Perseus, with a total mass of
$\sim1.5\ee{4}$ \msun\ (Table \ref{tab:regionspectra}) and a similar size.  In
Figure \ref{fig:regionboxes_on_CO}, we show both the W5NW region, which
contains all of the identified outflows, and the W5NWpc region, which is a
larger area intended to be directly comparable in both mass and spatial scale
to the Perseus molecular cloud.  The W5NWpc region contains more than an order of
magnitude more turbulent energy than the Perseus complex \citep[$E_{turb,Per} =
1.6\ee{46}$ ergs,][]{arce2010} despite its similar mass.  Even the smaller W5NW
region has $\sim5\times$ more turbulent energy than the Perseus complex,
largely because of the greater average line width ($\sigma_{FWHM,W5NW}\approx
3.5$ \kms).  As with the whole of W5, there is far too much turbulent energy in
W5NW to be provided by outflows alone, implying the presence of another driver
of turbulence.

Figure \ref{fig:w5nwpercompare} shows the W5NWpc region and Perseus molecular cloud
on the same scale, though in two different emission lines.  The Perseus cloud
contains many more outflows and candidates (70 in Perseus vs. 13 in W5NWpc)
despite a much larger physical area surveyed in W5.  While it is likely that
many of the W5W outflows will break apart into multiple flows at higher
resolution, it does not seem likely that each would break apart into 5 flows,
as would be required to bring the numbers into agreement.  Since the highest
density of outflows in Perseus is in the NGC 1333 cloud, it may be that there
is no equivalently evolved region in W5NWpc.  The W5W region may be comparably
massive, but it is also confused and strongly interacting with the W5 HII
region - either star formation is suppressed in this region, or outflows are
rendered undetectable.  In the latter case, the most likely mechanisms for
hiding outflows are molecular dissociation by ionizing radiation and velocity
confusion.

Another possibility highlighted in Figure \ref{fig:w5nwpercompare} is that the
W5NW region is interacting with the W4 bubble.  The cloud in the top right of
Figure \ref{fig:w5nwpercompare} is somewhat cometary, has higher peak
brightness temperature, and is at a slightly different velocity (-45 \kms) than
W5NW.  The velocity difference of $\sim8$ \kms could simply be two clouds
physically unassociated along the line of sight, or could indicate the presence
of another expanding bubble pushing two sheets of gas away from each other.
Either way, the northwest portion of the W5NW region is clumpier than the
area in which the outflows were detected, and it includes no outflow detections.

\FigureTwo{figures_chw5/W5NW_perseuscomparison}{figures_chw5/perseuscomparison_arceellipses}
{(a) An integrated CO 3-2 image of the W5W/NW region with ellipses overlaid
displaying the locations and sizes of outflows.  The dark red and blue ellipses
in the lower right are associated with outer-arm outflows.  W5W is the bottom-left,
CO-bright region.  W5NW is the top-center region containing the cluster of outflows.
(b) An integrated CO 1-0 image of the Perseus molecular cloud from the COMPLETE
survey \citep{arce2010}.  Note that the spatial scale is identical to that of
(a) assuming that W5 is 8 times more distant than Perseus.  The green ellipses
represent \citet{arce2010} CPOCs while the orange represent known outflows from
the same paper.}
{fig:w5nwpercompare}{1.0}


% However, the Perseus molecular cloud contains a total of 4800 \msun\
% \citep[from near-IR extinction][]{Evans2009}, while many independent clouds
% within W5 contain a similar mass (Table \ref{tab:regionspetra}), and the total
% gas mass in W5 is $\sim15\times$ larger \citep[also from near-IR extinction
% extinction][]{koenig:clustered:2008}.  Assuming a similar level of star
% formation activity (or star formation efficiency per gas mass), we should see
% substantially more outflows and more outflowing mass than is observed.  The
% simplest explanation for the lower outflowing mass in W5 is that the CO we use
% to trace it is destroyed by UV photons from the O-stars in the region.

\subsection{Star Formation Activity}
\label{sec:sfactivity}
%We have divided the W5 region into distinct sub-regions that all show signs
%of star formation (Section \ref{sec:subregions}).  The AFGL 4029, LW Cas
%Nebula, IC 1848, W5SE, and southern pillar sub-regions are all along the edge
%of the diffuse HII regions that shape W5.  In these sub-regions, the
%radiation-driven ionization front pressure from the O-stars could have led to
%triggered collapse.  However, there are also outflows in Sh 201 and W5 NW,
%which are not adjacent to the HII region and do not show signs of external
%pressure such as illuminated cloud edges in the CO and Spitzer infrared maps.  

CO outflows are an excellent tracer of ongoing embedded star formation
\citep[e.g.][]{Shu1987}.  We use the locations of newly discovered outflows to
qualitatively describe the star formation activity within the W5 complex and
evaluate the hypothesis that star formation has been triggered on small or
intermediate scales.

Class 0/I objects are nearly always associated with outflows in nearby
star-forming regions \citep[e.g. Perseus][]{curtis2010,hatchell2007}.  However,
\citet{koenig:clustered:2008} detected 171 Class I sources in W5 using Spitzer
photometry.  Since our detection threshold for outflow appears to be similar to
that in Perseus (Section \ref{sec:percompare}), the lower number of outflow
detections is surprising, especially considering that some of the detected
outflows are outside the Spitzer-MIPS field (MIPS detections are required for
Class I objects, and flows 1-4 are outside that range) or are in the outer arm
(flows 39-54).  Additionally, we should detect outflows from Class 0
objects that would not be identified by Spitzer colors.

There are a number of explanations for our detection deficiency.  The Class I
objects detected within the HII region ``bubble'' most likely have outflows in
which the CO is dissociated similar to jet systems in Orion \citep[e.g.
HH46/47, a pc-scale flow in which CO is only visible very near the
protostar;][]{Chernin1991,Stanke1999}.  This hypothesis can be tested by
searching for optical and infrared jets associated with these objects, which
presumably have lower mass envelopes and therefore less extinction than
typical Class I objects.  Additionally, there are many outflow systems that are
are likely to be associated with clusters of outflows rather than individual
outflows as demonstrated in Section \ref{sec:percompare}, where we were able to identify
fewer outflows when `observing' the Perseus objects at a greater distance.  There
are 24 sources in the \citet{koenig:clustered:2008} Class I catalog within
15\arcsec\ (one JCMT beam at 345 GHz) of another, and in many cases there are
multiple \citet{koenig:clustered:2008} Class I sources within the contours of a
single outflow system.



\subsection{Evaluating Triggering}
In the previous section, we discussed in detail the relationship between each
sub-region and the HII region.  Some regions are observed to be star-forming
but not interacting with the HII region (W5NW, Sh 2-201), while others are
interacting with the HII region but show no evidence or reduced evidence of
star formation (W5SE, W5W, LW Cas).  At the very least, there is significant
complexity in the triggering mechanisms, and no one mechanism or size scale is
dominant.  If we were to trust outflows as unbiased tracers of star formation,
we might conclude that the majority of star formation in W5 is untriggered
(spontaneous), but such a conclusion is unreliable because both radiatively
triggered star formation and ``revealed'' star formation may not exhibit
molecular outflows (although ionized atomic outflows should still be visible
around young stars formed through these scenarios).

In Section \ref{sec:w5ridge}, we analyzed a particular case in which the RDI
mechanism could plausibly have crushed a cloud to create the observed
protostar.  
It is not possible to determine whether interaction with the HII region was a 
necessary precondition for the star's formation, but it at least accelerated
the process.  The other cometary clouds share this property,
but in total there are only 5 cometary clouds with detected outflows at their
tips, indicating that this mechanism is not the dominant driver of star
formation in W5.

The `collect and collapse' scenario might naively be expected to produce an
excess of young stars at the interaction front between the HII region and the
molecular cloud.  However, because such interactions naturally tend to form
instabilities, this scenario produces cloud morphologies indistinguishable from
those of RDI.  There is not an obvious excess of sources associated with cloud
edges over those deep within the clouds (e.g., Figure
\ref{fig:outflows_on_co32}).  We therefore cannot provide any direct evidence
for this triggering scenario.

The overall picture of W5 is of two concurrent episodes of massive-star
formation that have lead to adjacent blown-out bubbles.  Despite the added
external pressure along the central ridge, it is relatively deficient in both
star formation activity and dense gas, perhaps because of heating by the strong
ionizing radiation field.  The lack of star formation along that central ridge
implies that much of the gas was squeezed and heated, but it was not crushed
into gravitationally unstable fragments.  While some star formation may have
been triggered in W5, there is strong evidence for pre-existing star formation
being at least a comparable, if not the dominant, mechanism of star formation
in the complex.


\section{Outflow systems beyond W5}
Fifteen outflows were detected at velocities inconsistent with the local W5
cloud velocities.  Of these, 8 are consistent with Perseus arm velocities
($v_{LSR} > -55$ \kms) and could be associated with different clouds within the
same spiral arm.  The other 7 have central velocities $v_{LSR} < -55$ \kms\ and
are associated with the outer arm identified in previous surveys
\citep[e.g.][]{Digel1996}.  The properties of these outflows are given in Tables
\ref{tab:outeroutflows} and \ref{tab:outeroutflowsderived}; the distances listed are kinematic distances assuming
$R_0=8.4$ kpc and $v_0=254$~\kms\ \citep{Reid2009}.

Of these outflows, all but one are within 2\arcmin\ of an IRAS point source.
Outflow 54 is the most distant in our survey at a kinematic distance $d=7.5$
kpc ($v_{lsr}=-75.6$ \kms) and galactocentric distance $D_G = 14.7$ kpc.  It
has no known associations in the literature.

Outflows 41 - 44 are associated with a cloud at $v_{LSR}\sim -62$ \kms\ known
in the literature as LDN 1375 and associated with IRAS 02413+6037.  Outflows 53
and 55 are at a similar velocity and associated with IRAS 02598+6008 and IRAS
02425+6000 respectively.  All of these sources lie roughly on the periphery of
the W5 complex.

Outflows 45 - 52 are associated with a string of IRAS sources and HII regions
to the north of W5 and have velocities in the range $-55 < v_{LSR} < -45$.
They therefore could be in the Perseus arm but are clearly unassociated with
the W5 complex.  Outflows 45 and 46 are associated with IRAS 02435+6144 and
they may also be associated with the Sharpless HII region Sh 2-194.  Outflows
47 and 48 are associated with IRAS 02461+6147, also known as AFGL 5085.
Outflows 49 and 50 are nearby but not necessarily associated with IRAS
02475+6156, and may be associated with Sh 2-196.  Outflows 51 and 52 are
associated with IRAS 02541+6208.  

% \subsection{Comparison to other studies}
% 
% XXXX:  Use otherpapers compare.txt.  Discuss what the most powerful outflows in each region
% would look like at 2kpc
 
% \Table{ccccc}{Comparison to other outflow surveys}
% {{Paper} & {Region} & {Tracer} &  {Distance (pc)} & {Area (arcmin$^2$)}  & {Area (pc$^2$)} 
% & {Resolution (\arcsec)} & {Resolution (pc)} & {Outflows detected} \\ }
% {tab:surveys}
% {
% \citet{arce2010}                       & CO 1-0 & Perseus  &    250   &   57600     &   305   & 46 & 0.056  &    96 \\
% \citet{dionatos2010}                   & CO 3-2 & Serpens  &    310   &   29        &   0.24  & 14 & 0.021  &    20 \\
% \citet{davis:jcmt:2010}                & CO 3-2 & Taurus   &    140   &   2705      &   4.5   & 14 & 0.0095 &    16 \\
% \citet{hatchell2007} \tablenotemark{a} & CO 3-2 & Perseus  &    250   &   204       &   1.1   & 14 & 0.017  &    37 \\
% This paper                             & CO 3-2 & W5       &   2000   &   7200      &   2440  & 14 & 0.14   &    38 \\
% }{
% \tablenotetext{a}{This survey targeted 51 mm cores, hence its much higher detection rate per unit area.}
% }

\section{Conclusions}

We have identified \nwfive\ molecular outflow candidates in the W5 star forming
region and an additional \nouter\ outflows spatially coincident but located in
the outer arm of the Galaxy.  

\begin{itemize}
%    \item The majority of the millimeter sources are associated with outflows,
%      though some of the brightest millimeter sources lack outflows.  These
%      sources may consist of warmer dust that has not yet become Jeans
%      unstable.  Millimeter emission is a good tracer of active embedded star
%      formation, while far-infrared brightness is not.
%    \item The majority of the gas seen in \thirteenco\ and 1.1 mm emission is
%      associated with star formation.  There is therefore only a small amount
%      of
    \item The majority of the CO clouds in the W5 complex are forming stars.
      Star formation is not limited to cloud edges around the HII region.
      Because star formation activity is observed outside of the region of
      influence of the W5 O-stars, it is apparent that direct triggering by
      massive star feedback is not responsible for all of the star formation in
      W5.
    \item The W5 complex is seen nearly face-on as evidenced by a strict upper
      limit on the CO column through the center of the HII-region bubbles.  It
      is therefore an excellent region to study massive star feedback and
      revealed and triggered star formation.
    \item Outflows contribute negligibly to the turbulent energy of molecular
      clouds in the W5 complex.  This result is unsurprising near an HII
      region, but supports the idea that massive star forming regions are
      qualitatively different from low-mass star-forming regions in which the
      observed turbulence could be driven by outflow feedback.  Even in regions
      far separated from the O-stars, there is more turbulence and less energy
      injection from outflows than in, e.g., Perseus.
    \item Despite detecting a significant number of powerful outflows, the
      total outflowing mass detected in this survey ($\sim 1.5$ \msun\ without
      optical depth correction, perhaps $10-20$ \msun\ when optical depth is
      considered) was somewhat smaller than in Perseus, a low to intermediate
      mass star forming region with $\sim 1/6$ the molecular mass of W5. 
    \item The low mass measured may be partly because the CO 3-2 line is
      sub-thermally excited in outflows.  Therefore, while CO 3-2 is an
      excellent tracer of outflows for detection, it does not serve as a
      `calorimeter' in the same capacity as CO 1-0.
    \item Even considering excitation and optical depth corrections, it is
      likely that the mass of outflows in W5 is less than would be expected
      from a simple extrapolation from Perseus based on cloud mass. CO is
      likely to be photodissociated in the outflows when they reach the HII
      region, accounting for the deficiency around the HII region edges.
      However, in areas unaffected by the W5 O-stars such as W5NW, the
      deficiency may be because the greater turbulence in the W5 clouds
      suppresses star formation or hides outflows.
    \item Velocity gradients across the tails of many cometary clouds have been
      observed, hinting at their geometry and confirming that the outflows seen
      from their heads must be generated by protostars within.
      %The data presented contain tens of cometary clouds with
      %precise kinematic information about their molecular gas.
    \item Outflows have been detected in the Outer Arm at galactocentric
      distances $\gtrsim12$ kpc.  These represent some of the highest
      galactocentric distance star forming regions discovered to date.
    %\item  Feedback in W5 is primarily destroying rather than
    %  triggering SF.
\end{itemize}

\section{Acknowledgements}
We thank the two anonymous referees for their assistance in refining this
document.  We thank Devin Silvia for a careful proofread of the text. This work
has made use of the APLpy plotting package
(\url{http://aplpy.sourceforge.net}), the pyregion package
(\url{http://leejjoon.github.com/pyregion/}), the agpy code package
(\url{http://code.google.com/p/agpy/}) , IPAC's Montage
(\url{http://montage.ipac.caltech.edu/}), the DS9 visualization tool
(\url{http://hea-www.harvard.edu/RD/ds9/}), the pyspeckit spectrosopic analysis
toolkit (\url{http://pyspeckit.bitbucket.org}), and the STARLINK package
(\url{http://starlink.jach.hawaii.edu/}).  IRAS data was acquired through IRSA
at IPAC (\url{http://irsa.ipac.caltech.edu/}).  DRAO 21 cm data was acquired
from the Canadian Astronomical Data Center
(\url{http://cadcwww.hia.nrc.ca/cgps/}).  The authors are supported by the
National Science Foundation through NSF grant AST-0708403.  This research has
made use of the SIMBAD database, operated at CDS, Strasbourg, France

%{\it Facilities:} JCMT, VLA

%\bibliography{w5outflows}

%{\tiny
%\clearpage
%\onecolumn
\LongTable{ccccccccccc}{CO 3-2 Outflow Measured Properties}
{{Outflow} & {Latitude} & {Longitude} & {Ellipse} & {Ellipse} & {Ellipse} & {Velocity} & {Velocity} &  & {$\int T_A^* dv$} & {Bipolar?  \tablenotemark{a}}\\
{Number} &  &  & {Major} & {Minor} & {PA} & {center} & {min} & {max} &  & \\
 &  &  & {\arcsec} & {\arcsec} & {\degrees} & {(\kms)} & {(\kms)} & {(\kms)} & {(K \kms)} & \\}
{tab:outflows}
{
1b & 136.4437 & 1.2622 & 60 & 27 & 342 & -36.1 & -47.6 & -40.3 & 1.0 & yc\\
1r & 136.4674 & 1.2705 & 49 & 24 & 346 & -36.1 & -31.9 & -23.4 & 1.5 & yc\\
2b & 136.4899 & 1.1904 & 30 & 23 & 299 & -35.7 & -48.0 & -39.7 & 0.7 & yc\\
2r & 136.4743 & 1.2042 & 31 & 28 & 332 & -35.7 & -31.7 & -23.0 & 1.3 & yc\\
3 & 136.475 & 1.2548 & 35 & 25 & 332 & -31.8 & -31.8 & -26.8 & 1.3 & n\\
4b & 136.5038 & 1.2623 & 26 & 22 & 35 & -36.2 & -44.1 & -40.1 & 0.8 & yu\\
4r & 136.5109 & 1.2751 & 25 & 22 & 332 & -36.2 & -32.4 & -28.6 & 0.9 & yu\\
5r & 136.5126 & 1.2453 & 24 & 22 & 10 & -35.3 & -31.4 & -28.8 & 0.8 & yu\\
5b & 136.5236 & 1.2524 & 39 & 22 & 3 & -35.3 & -45.0 & -39.2 & 1.4 & yu\\
6b & 136.532 & 1.228 & 28 & 25 & 332 & -35.3 & -44.8 & -40.0 & 0.4 & yc\\
6r & 136.5327 & 1.2333 & 28 & 20 & 318 & -35.3 & -30.6 & -24.0 & 1.0 & yc\\
7b & 136.5453 & 1.2318 & 24 & 19 & 332 & -34.9 & -47.5 & -39.9 & 1.7 & yc\\
7r & 136.5506 & 1.2383 & 27 & 23 & 314 & -34.9 & -29.9 & -22.7 & 1.3 & yc\\
8b & 136.5799 & 1.2755 & 18 & 14 & 332 & -34.5 & -41.5 & -39.3 & 0.6 & yc\\
8r & 136.581 & 1.2601 & 34 & 30 & 332 & -34.5 & -29.6 & -23.9 & 1.4 & yc\\
9b & 136.67 & 1.2123 & 30 & 27 & 332 & -35.0 & -44.5 & -38.5 & 1.4 & yc\\
9r & 136.6766 & 1.2059 & 40 & 31 & 332 & -35.0 & -31.6 & -26.7 & 0.3 & yc\\
10b & 136.7172 & 0.7859 & 39 & 24 & 353 & -42.8 & -52.6 & -47.5 & 3.3 & yc\\
10r & 136.7271 & 0.7797 & 31 & 26 & 332 & -42.8 & -38.1 & -33.1 & 4.1 & yc\\
11b & 136.8195 & 1.082 & 25 & 24 & 331 & -34.2 & -40.7 & -37.0 & 3.1 & yc\\
11r & 136.8173 & 1.0799 & 24 & 22 & 331 & -34.2 & -31.4 & -20.4 & 1.5 & yc\\
12b & 136.8414 & 1.1512 & 30 & 26 & 332 & -40.4 & -53.3 & -46.2 & 1.5 & yc\\
12r & 136.8479 & 1.1517 & 27 & 25 & 332 & -40.4 & -34.6 & -30.1 & 0.9 & yc\\
13 & 136.8461 & 0.8426 & 28 & 27 & 332 & -31.0 & -31.0 & -23.5 & 1.0 & n\\
14 & 136.8591 & 1.176 & 24 & 23 & 332 & -47.1 & -54.5 & -47.1 & 0.8 & n\\
15 & 136.9443 & 1.0841 & 28 & 18 & 348 & -45.0 & -55.0 & -45.0 & 3.1 & n\\
16b & 137.3929 & 0.5977 & 23 & 18 & 333 & -40.7 & -47.0 & -42.6 & 0.7 & yu\\
16r & 137.3981 & 0.6121 & 22 & 19 & 357 & -40.7 & -38.7 & -35.2 & 1.9 & yu\\
17b & 137.4084 & 0.6762 & 20 & 18 & 293 & -40.3 & -57.9 & -43.0 & 2.3 & yc\\
17r & 137.412 & 0.6775 & 20 & 18 & 308 & -40.3 & -37.6 & -30.4 & 1.1 & yc\\
18b & 137.4925 & 0.6289 & 16 & 15 & 333 & -35.5 & -39.2 & -37.6 & 1.1 & yc\\
18r & 137.4908 & 0.6292 & 18 & 17 & 307 & -35.5 & -33.4 & -31.0 & 2.0 & yc\\
19b & 137.4815 & 0.6409 & 20 & 17 & 1 & -36.0 & -41.9 & -38.9 & 1.3 & yc\\
19r & 137.4798 & 0.6404 & 20 & 16 & 301 & -36.0 & -33.1 & -25.9 & 0.7 & yc\\
20r & 137.5368 & 1.2792 & 24 & 21 & 332 & -37.4 & -33.0 & -22.5 & 5.2 & yc\\
20b & 137.539 & 1.279 & 27 & 23 & 17 & -37.4 & -52.0 & -41.8 & 3.4 & yc\\
21b & 137.6152 & 1.3543 & 31 & 28 & 322 & -39.5 & -52.0 & -43.7 & 4.5 & yc\\
21r & 137.6169 & 1.3585 & 31 & 18 & 4 & -39.5 & -35.2 & -30.0 & 1.2 & yc\\
22 & 137.6213 & 1.506 & 27 & 21 & 293 & -40.3 & -46.0 & -40.3 & 2.1 & n\\
23b & 137.6389 & 1.5251 & 21 & 14 & 331 & -38.5 & -42.5 & -40.5 & 1.6 & yc\\
23r & 137.6449 & 1.5194 & 19 & 12 & 331 & -38.5 & -36.5 & -32.0 & 1.9 & yc\\
24r & 137.7094 & 1.4824 & 20 & 20 & 331 & -38.2 & -33.8 & -25.4 & 4.2 & yc\\
24b & 137.7146 & 1.4809 & 25 & 19 & 292 & -38.2 & -50.0 & -42.7 & 4.4 & yc\\
25b & 138.1398 & 1.6858 & 39 & 26 & 282 & -38.8 & -49.5 & -43.2 & 0.6 & yc\\
25r & 138.142 & 1.6884 & 43 & 35 & 11 & -38.8 & -34.3 & -27.5 & 1.7 & yc\\
26b & 138.2913 & 1.5538 & 29 & 29 & 355 & -38.7 & -52.0 & -47.4 & 1.2 & yc\\
26r & 138.2966 & 1.5564 & 28 & 28 & 330 & -38.7 & -30.0 & -20.0 & 4.2 & yc\\
27 & 138.3017 & 1.5689 & 26 & 25 & 330 & -30.0 & -30.0 & -22.0 & 1.8 & n\\
28 & 138.3042 & 1.5437 & 20 & 19 & 330 & -43.3 & -46.1 & -43.3 & 1.4 & n\\
29 & 138.3053 & 1.5537 & 22 & 20 & 330 & -45.3 & -51.6 & -45.3 & 2.5 & n\\
30 & 138.3115 & 1.5443 & 26 & 26 & 330 & -33.0 & -33.0 & -29.2 & 1.2 & n\\
31 & 138.3184 & 1.5566 & 26 & 25 & 330 & -44.4 & -49.1 & -44.4 & 1.1 & n\\
32 & 138.3213 & 1.5658 & 27 & 27 & 330 & -31.7 & -31.7 & -27.0 & 1.4 & n\\
33b & 138.3618 & 1.5073 & 28 & 26 & 330 & -39.4 & -49.5 & -44.0 & 1.3 & yc\\
33r & 138.3642 & 1.4959 & 29 & 21 & 330 & -39.4 & -34.7 & -25.8 & 2.0 & yc\\
34r & 138.4779 & 1.6137 & 22 & 21 & 330 & -36.9 & -33.1 & -29.1 & 0.5 & yc\\
34b & 138.4768 & 1.6142 & 21 & 20 & 330 & -36.9 & -43.6 & -40.6 & 0.8 & yc\\
35r & 138.4998 & 1.6496 & 22 & 20 & 4 & -37.5 & -31.3 & -24.1 & 1.4 & yc\\
35b & 138.5021 & 1.6458 & 23 & 21 & 330 & -37.5 & -49.5 & -43.6 & 1.3 & yc\\
36b & 138.5034 & 1.6654 & 35 & 26 & 5 & -37.5 & -50.4 & -42.3 & 1.2 & yc\\
36r & 138.5061 & 1.6576 & 22 & 21 & 330 & -37.5 & -32.6 & -26.7 & 1.4 & yc\\
37r & 138.5208 & 1.6618 & 27 & 22 & 330 & -38.5 & -33.5 & -31.4 & 0.6 & yc\\
37b & 138.5241 & 1.6667 & 23 & 23 & 18 & -38.5 & -47.0 & -43.6 & 0.6 & yc\\
38b & 137.4983 & 0.6062 & 16 & 15 & 333 & -36.1 & -39.2 & -38.5 & 0.8 & yc\\
38r & 137.4977 & 0.6055 & 15 & 14 & 307 & -36.1 & -33.7 & -32.5 & 0.5 & yc\\
39b & 138.1506 & 0.7724 & 23 & 16 & 321 & -38.8 & -45.3 & -41.0 & 2.0 & yc\\
39r & 138.1591 & 0.7713 & 17 & 13 & 304 & -38.8 & -36.6 & -34.7 & 0.7 & yc\\
40 & 138.1356 & 0.7634 & 22 & 18 & 4 & -36.0 & -36.0 & -27.6 & 2.2 & n\\
}{\multicolumn{11}{l}{Measured properties of the outflows.} \\ 
\multicolumn{11}{l}{$^a$ Is the outflow part of a bipolar pair?  yc = yes, confident; yu = yes, uncertain; n = no} \\ }
{}
{11}

\clearpage

\LongTable{cccccc}{CO 3-2 Outflow Derived Properties}
{{Outflow} & {Mass} & {Momentum} & {Energy} & {Dynamical} & {Momentum}\\
{Number} &  &  &  & {Age} & {Flux}\\
 & {(\msun)} & {(\msun \kms)} & {(10$^{42}$ ergs)} & {(10$^4$ years)} & {$10^{-6}$ \msun}\\
 &  &  &  &  & {\kms yr$^{-1}$}\\}
{tab:outflowsderived}
{
1b & 0.034 & 0.26 & 21.1 & 7.0 & 7.2\\
1r & 0.04 & 0.24 & 17.3 & 7.0 & 7.2\\
2b & 0.011 & 0.07 & 4.9 & 5.4 & 4.4\\
2r & 0.025 & 0.17 & 13.0 & 5.4 & 4.4\\
3 & 0.025 & 0.12 & 5.8 & - & -\\
4b & 0.01 & 0.06 & 4.0 & 7.2 & 1.5\\
4r & 0.011 & 0.04 & 1.8 & 7.2 & 1.5\\
5r & 0.01 & 0.04 & 2.0 & 4.5 & 4.0\\
5b & 0.025 & 0.14 & 8.0 & 4.5 & 4.0\\
6b & 0.007 & 0.04 & 3.0 & 1.7 & 8.1\\
6r & 0.013 & 0.09 & 6.8 & 1.7 & 8.1\\
7b & 0.017 & 0.13 & 10.5 & 2.4 & 10.9\\
7r & 0.018 & 0.13 & 9.7 & 2.4 & 10.9\\
8b & 0.003 & 0.02 & 0.9 & 4.9 & 4.5\\
8r & 0.032 & 0.2 & 13.2 & 4.9 & 4.5\\
9b & 0.025 & 0.13 & 7.2 & 3.9 & 4.2\\
9r & 0.009 & 0.04 & 1.8 & 3.9 & 4.2\\
10b & 0.068 & 0.41 & 25.7 & 3.9 & 22.0\\
10r & 0.074 & 0.45 & 28.0 & 3.9 & 22.0\\
11b & 0.042 & 0.17 & 7.2 & 0.7 & 35.3\\
11r & 0.017 & 0.09 & 5.9 & 0.7 & 35.3\\
12b & 0.026 & 0.14 & 8.7 & 1.8 & 15.2\\
12r & 0.014 & 0.13 & 12.7 & 1.8 & 15.2\\
13 & 0.016 & 0.1 & 5.8 & - & -\\
14 & 0.01 & 0.06 & 4.1 & - & -\\
15 & 0.036 & 0.24 & 17.3 & - & -\\
16b & 0.006 & 0.03 & 1.2 & 11.1 & 0.7\\
16r & 0.018 & 0.05 & 1.3 & 11.1 & 0.7\\
17b & 0.019 & 0.12 & 9.4 & 1.4 & 10.6\\
17r & 0.009 & 0.03 & 0.7 & 1.4 & 10.6\\
18b & 0.006 & 0.02 & 0.5 & 1.4 & 4.0\\
18r & 0.013 & 0.04 & 1.0 & 1.4 & 4.0\\
19b & 0.011 & 0.05 & 2.4 & 0.7 & 9.6\\
19r & 0.005 & 0.01 & 0.2 & 0.7 & 9.6\\
20r & 0.059 & 0.5 & 46.3 & 0.5 & 156.0\\
20b & 0.047 & 0.33 & 26.6 & 0.5 & 156.0\\
21b & 0.086 & 0.58 & 41.4 & 1.7 & 39.1\\
21r & 0.014 & 0.08 & 4.3 & 1.7 & 39.1\\
22 & 0.027 & 0.1 & 4.3 & - & -\\
23b & 0.011 & 0.03 & 0.9 & 4.5 & 1.3\\
23r & 0.01 & 0.03 & 1.0 & 4.5 & 1.3\\
24r & 0.037 & 0.3 & 26.1 & 1.7 & 34.1\\
24b & 0.047 & 0.28 & 18.3 & 1.7 & 34.1\\
25b & 0.014 & 0.09 & 6.8 & 1.0 & 42.8\\
25r & 0.056 & 0.35 & 23.0 & 1.0 & 42.8\\
26b & 0.023 & 0.24 & 26.1 & 1.1 & 98.3\\
26r & 0.072 & 0.85 & 106.0 & 1.1 & 98.3\\
27 & 0.026 & 0.09 & 4.5 & - & -\\
28 & 0.012 & 0.07 & 4.6 & - & -\\
29 & 0.024 & 0.06 & 2.1 & - & -\\
30 & 0.018 & 0.12 & 8.0 & - & -\\
31 & 0.016 & 0.03 & 0.7 & - & -\\
32 & 0.023 & 0.18 & 14.5 & - & -\\
33b & 0.022 & 0.14 & 10.1 & 3.0 & 11.4\\
33r & 0.026 & 0.2 & 16.1 & 3.0 & 11.4\\
34r & 0.005 & 0.03 & 1.7 & 0.7 & 8.4\\
34b & 0.007 & 0.03 & 1.2 & 0.7 & 8.4\\
35r & 0.013 & 0.12 & 11.6 & 1.3 & 18.7\\
35b & 0.014 & 0.12 & 11.0 & 1.3 & 18.7\\
36b & 0.025 & 0.19 & 15.8 & 2.7 & 10.7\\
36r & 0.014 & 0.1 & 6.8 & 2.7 & 10.7\\
37r & 0.008 & 0.04 & 1.6 & 2.1 & 4.3\\
37b & 0.007 & 0.06 & 4.2 & 2.1 & 4.3\\
38b & 0.005 & 0.01 & 0.4 & 0.9 & 2.3\\
38r & 0.002 & 0.01 & 0.1 & 0.9 & 2.3\\
39b & 0.017 & 0.07 & 2.8 & 7.5 & 1.0\\
39r & 0.004 & 0.01 & 0.3 & 7.5 & 1.0\\
40 & 0.019 & 0.08 & 3.5 & - & -\\
}{\multicolumn{6}{l}{Derived properties of the outflows in
the optically thin limit.} \\  
\multicolumn{6}{l}{Typical optical depth corrections for \twelveco 3-2 are
in the range 7-14 \citep{curtis2010}.} \\  
\multicolumn{6}{l}{The correction for velocity confusion is $\gtrsim2$ but
poorly constrained \citep{arce2010}.} \\  
\multicolumn{6}{l}{Finally, an excitation correction in the range 1-20 is
likely required as described in the Appendix.} \\
\multicolumn{6}{l}{The mass and momentum values can be multiplied by these
factors to acquire the corrected values.} \\  
\multicolumn{6}{l}{The energy is weighted more heavily towards high-velocity,
low-optical-depth gas, so the correction factor is likely to be lower.}}
{} {6}
 % tab:outflows
%\Table{ccccccc}{Totals of outflow properties}
{{BGPS source} & {1.1mm mass} & {Intensity} & {Outflow Column} & {Outflow Mass} & {Momentum} & {Energy}\\
 & {\msun} & {(K \kms)} & {(\persc)} & {(\msun)} & {(\msun \kms)} & {(ergs)}\\}
{tab:outflowsums}
{
G136.456+01.268 & 19.81 & 1.657 & 5.45\ee{18} & 0.0507 & 0.357 & 2.76\ee{43}\\
G136.474+01.268 & 15.06 & 0.832 & 2.74\ee{18} & 0.0162 & 0.115 & 8.9\ee{42}\\
G136.500+01.258 & 38.16 & 2.166 & 7.11\ee{18} & 0.03802 & 0.2055 & 1.188\ee{43}\\
G136.512+01.194 & 85.65 & 1.512 & 4.97\ee{18} & 0.02772 & 0.1536 & 9.44\ee{42}\\
G136.536+01.232 & 88.86 & 2.744 & 9.03\ee{18} & 0.04309 & 0.3245 & 2.575\ee{43}\\
G136.671+01.210 & 71.31 & 2.667 & 8.78\ee{18} & 0.0589 & 0.3405 & 2.088\ee{43}\\
G136.719+00.782 & 81.3 & 4.28 & 1.407\ee{19} & 0.0661 & 0.36 & 2.142\ee{43}\\
G136.828+01.064 & 224.1 & 2.51 & 8.25\ee{18} & 0.0361 & 0.222 & 1.569\ee{43}\\
G136.842+00.838 & 21.84 & 0.813 & 2.67\ee{18} & 0.0138 & 0.0851 & 5.67\ee{42}\\
G136.846+01.168 & 67.96 & 2.05 & 6.74\ee{18} & 0.0377 & 0.265 & 2.09\ee{43}\\
G136.849+01.150 & 133.2 & 1.34 & 4.4\ee{18} & 0.01831 & 0.1516 & 1.322\ee{43}\\
G136.948+01.092 & 202.8 & 1.45 & 4.76\ee{18} & 0.0218 & 0.139 & 9.71\ee{42}\\
G137.394+00.610 & 26.68 & 5.38 & 1.768\ee{19} & 0.0998 & 0.783 & 6.77\ee{43}\\
G137.409+00.674 & 41.22 & 1.06 & 3.48\ee{18} & 0.0205 & 0.112 & 6.45\ee{42}\\
G137.479+00.640 & 107.5 & 1.839 & 6.047\ee{18} & 0.02244 & 0.1463 & 1.1647\ee{43}\\
G137.538+01.278 & 87.68 & 0.545 & 1.793\ee{18} & 0.008201 & 0.04483 & 2.473\ee{42}\\
G137.617+01.350 & 128.2 & 1.358 & 4.466\ee{18} & 0.01642 & 0.0656 & 2.989\ee{42}\\
G137.632+01.508 & 48.13 & 2.29 & 7.52\ee{18} & 0.025 & 0.143 & 9.06\ee{42}\\
G137.665+01.526 & 65.13 & 3.16 & 1.04\ee{19} & 0.03374 & 0.2716 & 2.321\ee{43}\\
G137.707+01.490 & 71.2 & 1.763 & 5.78\ee{18} & 0.0481 & 0.364 & 3.04\ee{43}\\
G138.144+01.684 & 201.7 & 3.63 & 1.193\ee{19} & 0.0599 & 0.704 & 8.64\ee{43}\\
G138.295+01.556 & 824.4 & 6.957 & 2.289\ee{19} & 0.09247 & 0.6825 & 5.3999\ee{43}\\
G138.502+01.646 & 361.7 & 4.899 & 1.613\ee{19} & 0.07083 & 1.4379 & 3.47758\ee{44}\\
}{Totals of outflow mass, momentum, and energy.  }
 % tab:outflowsums

%\clearpage
%\Table{cccccccccccc}{Outer Arm CO 3-2 Outflows - Measured Properties}
{{Outflow} & {Latitude} & {Longitude} & {Ellipse} & {Ellipse} & {Ellipse} & {Kinematic} & {$R_G$\tablenotemark{a}} & {Velocity} & {Velocity} &  & {$\int T_A^* dv$}\\
{Number} &  &  & {Major} & {Minor} & {PA} & {Distance} &  & {center} & {min} & {max} & \\
 &  &  & {\arcsec} & {\arcsec} & {\degrees} & {(pc)} & {(pc)} & {(\kms)} &  &  & {(K \kms)}\\}
{tab:outeroutflows}
{
41r & 136.364 & 0.9606 & 25 & 18 & 2 & 5510 & 13000 & -61.8 & -59.2 & -56.5 & 0.5\\
41b & 136.3634 & 0.9568 & 23 & 17 & 353 & 5510 & 13000 & -61.8 & -71.6 & -64.3 & 3.0\\
42r & 136.3522 & 0.9786 & 20 & 14 & 2 & 5500 & 12900 & -62.1 & -59.8 & -57.6 & 0.6\\
42b & 136.3548 & 0.9798 & 20 & 19 & 332 & 5500 & 12900 & -62.1 & -67.8 & -64.4 & 0.5\\
43r & 136.3495 & 0.9612 & 17 & 15 & 63 & 5510 & 13000 & -61.8 & -59.0 & -56.1 & 0.8\\
43b & 136.353 & 0.9621 & 12 & 12 & 333 & 5510 & 13000 & -61.8 & -66.3 & -64.6 & 1.0\\
44r & 136.3554 & 0.9576 & 13 & 13 & 23 & 5500 & 12900 & -61.8 & -59.0 & -55.4 & 2.1\\
44b & 136.3545 & 0.9567 & 14 & 14 & 333 & 5500 & 12900 & -61.8 & -68.0 & -64.5 & 2.0\\
45r & 136.1219 & 2.0816 & 34 & 25 & 297 & 3750 & 11400 & -46.5 & -43.1 & -40.5 & 0.6\\
45b & 136.1233 & 2.0803 & 35 & 25 & 306 & 3750 & 11400 & -46.5 & -57.3 & -50.0 & 1.9\\
46 & 136.1166 & 2.0983 & 26 & 25 & 332 & 3790 & 11400 & -50.2 & -52.6 & -50.2 & 0.5\\
47b & 136.3857 & 2.2687 & 34 & 27 & 332 & 3220 & 11000 & -42.0 & -55.0 & -46.7 & 3.5\\
47r & 136.3861 & 2.267 & 35 & 23 & 304 & 3220 & 11000 & -42.0 & -37.3 & -25.1 & 5.0\\
48b & 136.374 & 2.2628 & 29 & 21 & 332 & 3250 & 11000 & -43.2 & -51.4 & -47.0 & 1.5\\
48r & 136.3736 & 2.2615 & 29 & 22 & 332 & 3250 & 11000 & -43.2 & -39.5 & -22.2 & 8.9\\
49r & 136.4663 & 2.4678 & 29 & 23 & 290 & 3610 & 11300 & -45.7 & -42.2 & -33.0 & 2.2\\
49b & 136.4661 & 2.4693 & 31 & 23 & 292 & 3610 & 11300 & -45.7 & -52.2 & -49.1 & 0.9\\
50b & 136.5087 & 2.5108 & 31 & 25 & 332 & 3380 & 11100 & -43.5 & -48.5 & -46.5 & 0.8\\
50r & 136.5118 & 2.5083 & 28 & 23 & 10 & 3380 & 11100 & -43.5 & -40.6 & -37.5 & 1.0\\
51b & 137.058 & 2.9858 & 28 & 23 & 293 & 4350 & 11900 & -51.8 & -55.5 & -53.0 & 0.8\\
51r & 137.0567 & 2.9864 & 34 & 25 & 8 & 4350 & 11900 & -51.8 & -50.6 & -40.9 & 3.5\\
52r & 137.0662 & 2.9999 & 37 & 26 & 43 & 4390 & 12000 & -52.2 & -49.1 & -41.0 & 7.8\\
52b & 137.0683 & 3.0013 & 38 & 29 & 15 & 4390 & 12000 & -52.2 & -65.8 & -55.2 & 4.1\\
53b & 138.6143 & 1.5611 & 26 & 26 & 330 & 5450 & 13000 & -59.7 & -71.1 & -61.7 & 5.5\\
53r & 138.6158 & 1.563 & 25 & 23 & 330 & 5450 & 13000 & -59.7 & -57.6 & -54.5 & 1.3\\
54r & 136.382 & 0.8392 & 29 & 20 & 343 & 7480 & 14700 & -75.6 & -73.2 & -68.9 & 2.6\\
54b & 136.3824 & 0.838 & 20 & 17 & 332 & 7480 & 14700 & -75.6 & -83.1 & -77.9 & 2.0\\
55b & 136.7623 & 0.4548 & 27 & 16 & 343 & 5230 & 12700 & -60.9 & -65.2 & -62.7 & 5.4\\
55r & 136.7579 & 0.4522 & 24 & 18 & 343 & 5230 & 12700 & -60.9 & -59.0 & -53.2 & 6.0\\
}{\tablenotetext{a}{Galactocentric Radius}}

\Table{cccccc}{Outer Arm CO 3-2 Outflows - Derived Properties}
{{Outflow} & {Mass} & {Momentum} & {Energy} & {Dynamical} & {Momentum}\\
{Number} &  &  &  & {Age} & {Flux}\\
 & {(\msun)} & {(\msun \kms)} & {(10$^{42}$ ergs)} & {(10$^4$ years)} & {$10^{-6}$ \msun \kms yr$^{-1}$}\\}
{tab:outeroutflowsderived}
{
41r & 0.037 & 0.11 & 3.5 & 3.6 & 30.2\\
41b & 0.196 & 0.96 & 54.6 & 3.6 & 30.2\\
42r & 0.029 & 0.06 & 1.3 & 4.3 & 3.9\\
42b & 0.03 & 0.11 & 3.9 & 4.3 & 3.9\\
43r & 0.033 & 0.13 & 5.1 & 7.3 & 2.9\\
43b & 0.024 & 0.08 & 2.8 & 7.3 & 2.9\\
44r & 0.062 & 0.21 & 7.9 & 1.8 & 27.7\\
44b & 0.067 & 0.28 & 12.6 & 1.8 & 27.7\\
45r & 0.037 & 0.18 & 8.6 & 1.2 & 62.3\\
45b & 0.126 & 0.55 & 28.3 & 1.2 & 62.3\\
46 & 0.028 & 0.1 & 3.7 & - & -\\
47b & 0.187 & 1.32 & 101.0 & 0.6 & 553.0\\
47r & 0.232 & 1.84 & 164.0 & 0.6 & 553.0\\
48b & 0.054 & 0.33 & 20.8 & 0.4 & 754.0\\
48r & 0.341 & 2.4 & 229.0 & 0.4 & 754.0\\
49r & 0.106 & 0.77 & 62.7 & 1.4 & 71.1\\
49b & 0.047 & 0.22 & 10.8 & 1.4 & 71.1\\
50b & 0.037 & 0.14 & 5.5 & 3.8 & 7.4\\
50r & 0.038 & 0.14 & 5.5 & 3.8 & 7.4\\
51b & 0.058 & 0.13 & 3.0 & 1.0 & 152.0\\
51r & 0.303 & 1.33 & 72.7 & 1.0 & 152.0\\
52r & 0.829 & 4.3 & 250.0 & 1.4 & 479.0\\
52b & 0.472 & 2.5 & 150.0 & 1.4 & 479.0\\
53b & 0.626 & 3.16 & 194.0 & 4.7 & 73.2\\
53r & 0.124 & 0.31 & 8.5 & 4.7 & 73.2\\
54r & 0.461 & 1.24 & 37.5 & 1.8 & 135.0\\
54b & 0.212 & 1.14 & 64.0 & 1.8 & 135.0\\
55b & 0.411 & 1.66 & 68.0 & 7.5 & 28.3\\
55r & 0.404 & 0.47 & 6.6 & 7.5 & 28.3\\
}

%\include{individual_outflows_preface}

%\include{individual_outflows}
% \appendix
% \onecolumn
\section{W5 Appendix: Optically Thin, LTE dipole molecule}
\label{appendix:dipole}
While many authors have solved the problem of converting CO 1-0 beam
temperatures to \hh\ column densities
\citep{garden1991,bourke1997,Cabrit1990,lada1996}, there are no  examples in
the literature of a full derivation of the LTE, optically thin CO-to-\hh\
conversion process for higher rotational states.  We present the full
derivation here, and quantify the systematic errors generated by various
assumptions.

We begin with the assumption of an optically thin cloud such that the radiative
transfer equation \citep[][eqn 1.9]{rohlfs} simplifies to
\begin{equation}
  \label{eqn:radtrans}
  \frac{dI_\nu}{\ds} = -\kappa_\nu I_\nu 
\end{equation}

The absorption and stimulated emission terms yield 
\begin{equation}
  \label{eqn:kappa}
  \kappa_\nu = \frac{h \nu_{ul} B_{ul} n_u}{c} \varphi(\nu)
              -\frac{h \nu_{ul} B_{lu} n_l}{c} \varphi(\nu)
\end{equation}
where $\varphi(\nu)$ is the line shape function ($\int\varphi(\nu) \dnu \equiv
1$), $n$ is the density in the given state, $\nu$ is the frequency of the transition,
$B$ is the Einstein B coefficient, and $h$ is Planck's constant.

By assuming LTE (the Boltzmann distribution) and using Kirchoff's Law and the definition of 
the Einstein A and B values, we can derive a more useful version of this equation
\begin{equation}
  \kappa_\nu = \frac{c^2}{8 \pi \nu_{ul}^2} n_u A_{ul} \left[\exp\left(\frac{h \nu_{ul} }{k_B T_{ex}}\right) - 1 \right] \varphi(\nu)
\end{equation}
where $k_B$ is Boltzmann's constant.

The observable $T_B$ can be related to the optical depth, which is given by 
\begin{equation}
  \int \tau_\nu \dnu = \frac{c^2}{8 \pi \nu_{ul}^2} A_{ul} \left[\exp\left(\frac{h \nu_{ul} }{k_B T_{ex}}\right) - 1 \right] \int \varphi(\nu) \dnu \int n_u \ds 
\end{equation}

Rearranging and converting from density to column ($\int n \ds = N$) gives an equation for the column density
of the molecule in the upper energy state of the transition:
\begin{equation}
  \label{eqn:nuppertau}
  N_u = \frac{8\pi \nu_{ul}^2}{c^2 A_{ul}} \left[\exp\left(\frac{h \nu_{ul} }{k_B T_{ex}}\right) - 1 \right]^{-1} \int \tau_\nu \dnu
\end{equation}

In order to relate the brightness temperature to the optical depth, at CO transition frequencies the full blackbody
formula must be used and the CMB must also be taken into account.  \citet{rohlfs} equation 15.29 
\begin{equation} 
  \label{eqn:tbrightnesscmb}
  T_B(\nu) = \frac{h \nu}{k_B} \left(\left[e^{h \nu / k_B T_{ex}} - 1\right]^{-1} - \left[e^{h \nu / k_B T_{CMB}} - 1\right]^{-1} \right) (1-e^{-\tau_\nu})
\end{equation}
is rearranged to solve for $\tau_\nu$:
\begin{equation}
  \label{eqn:tau}
  \tau_\nu = -\ln\left[ 1 - \frac{k_B T_B}{h \nu} \left(\left[e^{h \nu / k_B T_{ex}} - 1\right]^{-1} - \left[e^{h \nu / k_B T_{CMB}} - 1\right]^{-1} \right)^{-1} \right]
\end{equation}

We convert from frequency to velocity units with $\dnu = \nu/c \dv$, and plug \eqref{eqn:tau} into \eqref{eqn:nuppertau} to get
\begin{equation}
  \label{eqn:nuppernoapprox}
  N_u = \frac{8\pi \nu_{ul}^3}{c^3 A_{ul}} \left[\exp\left(\frac{h \nu_{ul} }{k_B T_{ex}}\right) - 1 \right]^{-1} \int -\ln\left[ 1 - \frac{k_B T_B}{h \nu_{ul}} \left(\left[e^{h \nu_{ul} / k_B T_{ex}} - 1\right]^{-1} - \left[e^{h \nu_{ul} / k_B T_{CMB}} - 1\right]^{-1} \right)^{-1} \right] \dv
\end{equation}
which is the full LTE upper-level column density with no approximations applied.

The first term of the Taylor expansion is appropriate for $\tau<<1$ ($\ln[1+x]\approx x-\frac{x^2}{2}+\frac{x^3}{3}\ldots$)
\begin{equation}
  N_u = \frac{8\pi \nu_{ul}^3}{c^3 A_{ul}} \left[\exp\left(\frac{h \nu_{ul} }{k_B T_{ex}}\right) - 1 \right]^{-1} \int \frac{k_B T_B}{h \nu_{ul}} \left(\left[e^{h \nu_{ul} / k_B T_{ex}} - 1\right]^{-1} - \left[e^{h \nu_{ul} / k_B T_{CMB}} - 1\right]^{-1} \right)^{-1} \dv
\end{equation}
which simplifies to
\begin{equation}
  \label{eqn:nupper}
  N_u = \frac{8\pi \nu_{ul}^2 k_B}{c^3 A_{ul} h }  \frac{e^{h\nu_{ul}/k_B T_{CMB}} - 1}{e^{h\nu_{ul}/k_B T_{CMB}} - e^{h\nu_{ul}/k_B T_{ex}}} \int T_B  \dv
\end{equation}

This can be converted to use $\mu_e$ \citep[0.1222 for
\twelveco; ][]{Muenter1975}, the electric dipole moment of the molecule, instead
of $A_{ul}$, using \citet{rohlfs} equation 15.20 $\left((A_{ul}=(64\pi^4)/(3 h
c^3)\right)\nu^3 \mu_{e}^2$):
\begin{equation}
  \label{eqn:nuppermuju}
  N_u = \frac{3  }{8 \pi^3 \mu_e^2 } \frac{k_B}{\nu_{ul}} \frac{2 J_u + 1}{J_u} 
    \frac{e^{h\nu_{ul}/k_B T_{cmb}} - 1}{e^{h\nu_{ul}/k_B T_{CMB}} - e^{h\nu_{ul}/k_B T_{ex}}} \int T_B  \dv
\end{equation}

The total column can be derived from the column in the upper state using the partition
function and the Boltzmann distribution
\begin{equation}
  n_{tot}  =        \sum_{J=0}^\infty n_J = n_0 \sum_{J=0}^\infty  (2J+1) \exp\left(-\frac{J(J+1) B_e h}{k_B T_{ex}}\right) \label{eqn:approxpartition}\\
\end{equation}
This equation is frequently approximated using an integral
\citep[e.g.][]{Cabrit1990}, but a more accurate numerical solution using up to
thousands of rotational states is easily computed
\begin{equation}
  n_J = \left[ \sum_{j=0}^{j=j_{max}} (2j+1) \exp\left(-\frac{j(j+1) B_e h}{k_B T_{ex}}\right) \right]^{-1} (2J+1) \exp\left(-\frac{J(J+1) B_e h}{k_B T_{ex}}\right)
\end{equation}
The effects of using the approximation and the full numerical solution are shown in figure \ref{fig:approx}.

%We note that there are a number ($>1$) of different values of $\mu_e$ frequently reported in the literature.
%\citet{Burrus1958} reports a Stark-effect measurement of $\mu_e = 0.112\pm0.005$ Debye.  \citet{Muenter1975}
%report an improvement on this measurement, yielding $\mu_e = 0.1222$.  More recently, \citet{Goorvitch1994}
%report a value for the rotationless dipole momenut $\mu_0 = 0.1101$, which is negligibly different from the 
%\citet{Muenter1975} value...

\Figure{figures_chw5/columnconversion_vs_tex_allapprox}
{The LTE, optically thin conversion factor from $T_B$ (K \kms) to N(\hh)
(\persc) assuming X$_{\twelveco}=10^{-4}$ plotted against $T_{ex}$.  The
dashed line shows the effect of using the integral approximation of the 
partition function \citep[e.g.][]{Cabrit1990}.  It is a better
approximation away from the critical point, and is a better approximation
for higher transitions.  The dotted line shows the effects of removing the 
CMB term from \eqref{eqn:tbrightnesscmb}; the CMB populates the lowest two
excited states, but contributes nearly nothing to the $J=3$ state. Top (blue):
J=1-0, Middle (green): J=2-1, Bottom (red): J=3-2.}
{fig:approx}{1.0}{0}


The CO 3-2 transition is also less likely to be in LTE than the 1-0 transition.
The critical density ($n_{cr}\equiv A_{ul}/C_{ul}$) of \twelveco\ 3-2 is 27
times higher than that for 1-0.  We have run RADEX \citep{VanDerTak2007} LVG
models of CO to examine the impact of sub-thermal excitation on column
derivation.  The results of the RADEX models are shown in Figure
\ref{fig:coradex}.  They illustrate that, while it is quite safe to assume the
CO 1-0 transition is in LTE in most circumstances, a similar assumption is
probably invalid for the CO 3-2 transition in typical molecular cloud
environments.

\Figure{figures_chw5/CO_excitation}
{{\it Top}: The derived N(\hh) as a function of $n_{\hh}$ for $T_{B}=1$ K.
The dashed lines represent the LTE-derived $N(\hh)/T_B$ factor, which has 
no density dependence and, for CO 3-2, only a weak dependence on temperature.
We assume an abundance of \twelveco\ relative to \hh\ $X_{CO} = 10^{-4}$.
{\it Bottom}: The correction factor (N(\hh)$_{RADEX}$ / N(\hh)$_{LTE}$) as
a function of $n_{\hh}$.
For $T_K=20$ K, the ``correction factor'' at $10^3$ \percc\ (typical GMC
mean volume densities) is $\sim15$, while at $10^4$ \percc\ (closer to $n_{crit}$ but
perhaps substantially higher than GMC densities) it becomes negligible.  The
correction factor is also systematically lower for a higher gas kinetic
temperature.
For some densities, the ``correction factor'' dips below 1, particularly for CO
1-0.  This effect is from a slight population inversion due to fast spontaneous
decay rates from the higher levels and has been noted before
\citep[e.g.][]{Goldsmith1972}.
}{fig:coradex}{1.0}{0}

%\bibliography{column_derivation}



%\end{document}
\ifstandalone
\bibliographystyle{apj_w_etal}  % or "siam", or "alpha", or "abbrv"
%\bibliography{thesis}      % bib database file refs.bib
\bibliography{bibdesk}      % bib database file refs.bib
\fi

\end{document}

% %\documentclass[defaultstyle,11pt]{thesis}
%\documentclass[]{report}
%\documentclass[]{article}
%\usepackage{aastex_hack}
%\usepackage{deluxetable}
\documentclass[preprint]{aastex}


%%%%%%%%%%%%%%%%%%%%%%%%%%%%%%%%%%%%%%%%%%%%%%%%%%%%%%%%%%%%%%%%
%%%%%%%%%%%  see documentation for information about  %%%%%%%%%%
%%%%%%%%%%%  the options (11pt, defaultstyle, etc.)   %%%%%%%%%%
%%%%%%%  http://www.colorado.edu/its/docs/latex/thesis/  %%%%%%%
%%%%%%%%%%%%%%%%%%%%%%%%%%%%%%%%%%%%%%%%%%%%%%%%%%%%%%%%%%%%%%%%
%		\documentclass[typewriterstyle]{thesis}
% 		\documentclass[modernstyle]{thesis}
% 		\documentclass[modernstyle,11pt]{thesis}
%	 	\documentclass[modernstyle,12pt]{thesis}

%%%%%%%%%%%%%%%%%%%%%%%%%%%%%%%%%%%%%%%%%%%%%%%%%%%%%%%%%%%%%%%%
%%%%%%%%%%%    load any packages which are needed    %%%%%%%%%%%
%%%%%%%%%%%%%%%%%%%%%%%%%%%%%%%%%%%%%%%%%%%%%%%%%%%%%%%%%%%%%%%%
\usepackage{latexsym}		% to get LASY symbols
\usepackage{graphicx}		% to insert PostScript figures
%\usepackage{deluxetable}
\usepackage{rotating}		% for sideways tables/figures
\usepackage{natbib}  % Requires natbib.sty, available from http://ads.harvard.edu/pubs/bibtex/astronat/
\usepackage{savesym}
\usepackage{amssymb}
%\savesymbol{singlespace}
\savesymbol{doublespace}
%\usepackage{wrapfig}
%\usepackage{setspace}
\usepackage{xspace}
\usepackage{color}
\usepackage{multicol}
\usepackage{mdframed}
\usepackage{url}
\usepackage{subfigure}
%\usepackage{emulateapj}
\usepackage{lscape}
\usepackage{grffile}
\usepackage{standalone}
\standalonetrue
\usepackage{import}
\usepackage[utf8]{inputenc}
\usepackage{longtable}
\usepackage{booktabs}



%%%%%%%%%%%%%%%%%%%%%%%%%%%%%%%%%%%%%%%%%%%%%%%%%%%%%%%%%%%%%%%%
%%%%%%%%%%%%       all the preamble material:       %%%%%%%%%%%%
%%%%%%%%%%%%%%%%%%%%%%%%%%%%%%%%%%%%%%%%%%%%%%%%%%%%%%%%%%%%%%%%

% \title{Star Formation in the Galaxy}
% 
% \author{Adam G.}{Ginsburg}
% 
% \otherdegrees{B.S., Rice University, 2007\\
% 	      M.S., University of Colorado, Boulder, 2009}
% 
% \degree{Doctor of Philosophy}		%  #1 {long descr.}
% 	{Ph.D., Rocket Science (ok, fine, astrophysics)}		%  #2 {short descr.}
% 
% \dept{Department of}			%  #1 {designation}
% 	{Astrophysical and Planetary Sciences}		%  #2 {name}
% 
% \advisor{Prof.}				%  #1 {title}
% 	{John Bally}			%  #2 {name}
% 
% \reader{Prof.~Jeremy Darling}		%  2nd person to sign thesis
% \readerThree{Prof.~Jason Glenn}		%  3rd person to sign thesis
% \readerFour{Prof.~Michael Shull}	%  4rd person to sign thesis
% \readerFour{Prof.~Neal Evans}	%  4rd person to sign thesis
% 
% \abstract{  \OnePageChapter	% one page only ??
% 
%     I discovered dust in space.  
% 
% 	}
% 
% 
% \dedication[Dedication]{	% NEVER use \OnePageChapter here.
% 	To 1, the second number in binary.
% 	}
% 
% \acknowledgements{	\OnePageChapter	% *MUST* BE ONLY ONE PAGE!
% 	All y'all.
% 	}
% 
% \ToCisShort	% a 1-page Table of Contents ??
% 
% \LoFisShort	% a 1-page List of Figures ??
% %	\emptyLoF	% no List of Figures at all ??
% 
% \LoTisShort	% a 1-page List of Tables ??
% %	\emptyLoT	% no List of Tables at all ??
% 
% 
% %%%%%%%%%%%%%%%%%%%%%%%%%%%%%%%%%%%%%%%%%%%%%%%%%%%%%%%%%%%%%%%%%
% %%%%%%%%%%%%%%%       BEGIN DOCUMENT...         %%%%%%%%%%%%%%%%%
% %%%%%%%%%%%%%%%%%%%%%%%%%%%%%%%%%%%%%%%%%%%%%%%%%%%%%%%%%%%%%%%%%
% 
% %%%%  footnote style; default=\arabic  (numbered 1,2,3...)
% %%%%  others:  \roman, \Roman, \alph, \Alph, \fnsymbol
% %	"\fnsymbol" uses asterisk, dagger, double-dagger, etc.
% %	\renewcommand{\thefootnote}{\fnsymbol{footnote}}
% %	\setcounter{footnote}{0}

\input{macros}		% file containing author's macro definitions

\begin{document}
% \input{introduction}
% 
% %\input{ch_iras05358}
% \input{ch_w5}
% \input{ch_h2co}
% \input{ch_h2colarge}
% \input{ch_boundhii}
% 
% %\input ch2.tex			% file with Chapter 2 contents
% 
% %%%%%%%%%%%%%%%%%%%%%%%%%%%%%%%%%%%%%%%%%%%%%%%%%%%%%%%%%%%%%%%%%%%
% %%%%%%%%%%%%%%%%%%%%%%%  Bibliography %%%%%%%%%%%%%%%%%%%%%%%%%%%%%
% %%%%%%%%%%%%%%%%%%%%%%%%%%%%%%%%%%%%%%%%%%%%%%%%%%%%%%%%%%%%%%%%%%%
% 
% \bibliographystyle{plain}	% or "siam", or "alpha", or "abbrv"
% 				% see other styles (.bst files) in
% 				% $TEXHOME/texmf/bibtex/bst
% 
% \nocite{*}		% list all refs in database, cited or not.
% 
% \bibliography{thesis}		% bib database file refs.bib
% 
% %%%%%%%%%%%%%%%%%%%%%%%%%%%%%%%%%%%%%%%%%%%%%%%%%%%%%%%%%%%%%%%%%%%
% %%%%%%%%%%%%%%%%%%%%%%%%  Appendices %%%%%%%%%%%%%%%%%%%%%%%%%%%%%%
% %%%%%%%%%%%%%%%%%%%%%%%%%%%%%%%%%%%%%%%%%%%%%%%%%%%%%%%%%%%%%%%%%%%
% 
% \appendix	% don't forget this line if you have appendices!
% 
% %\input appA.tex			% file with Appendix A contents
% %\input appB.tex			% file with Appendix B contents
% 
% %%%%%%%%%%%%%%%%%%%%%%%%%%%%%%%%%%%%%%%%%%%%%%%%%%%%%%%%%%%%%%%%%%%
% %%%%%%%%%%%%%%%%%%%%%%%%   THE END   %%%%%%%%%%%%%%%%%%%%%%%%%%%%%%
% %%%%%%%%%%%%%%%%%%%%%%%%%%%%%%%%%%%%%%%%%%%%%%%%%%%%%%%%%%%%%%%%%%%
% 
% \end{document}
% 
% 

%%\documentclass[defaultstyle,11pt]{thesis}
%\documentclass[]{report}
%\documentclass[]{article}
%\usepackage{aastex_hack}
%\usepackage{deluxetable}
\documentclass[preprint]{aastex}


%%%%%%%%%%%%%%%%%%%%%%%%%%%%%%%%%%%%%%%%%%%%%%%%%%%%%%%%%%%%%%%%
%%%%%%%%%%%  see documentation for information about  %%%%%%%%%%
%%%%%%%%%%%  the options (11pt, defaultstyle, etc.)   %%%%%%%%%%
%%%%%%%  http://www.colorado.edu/its/docs/latex/thesis/  %%%%%%%
%%%%%%%%%%%%%%%%%%%%%%%%%%%%%%%%%%%%%%%%%%%%%%%%%%%%%%%%%%%%%%%%
%		\documentclass[typewriterstyle]{thesis}
% 		\documentclass[modernstyle]{thesis}
% 		\documentclass[modernstyle,11pt]{thesis}
%	 	\documentclass[modernstyle,12pt]{thesis}

%%%%%%%%%%%%%%%%%%%%%%%%%%%%%%%%%%%%%%%%%%%%%%%%%%%%%%%%%%%%%%%%
%%%%%%%%%%%    load any packages which are needed    %%%%%%%%%%%
%%%%%%%%%%%%%%%%%%%%%%%%%%%%%%%%%%%%%%%%%%%%%%%%%%%%%%%%%%%%%%%%
\usepackage{latexsym}		% to get LASY symbols
\usepackage{graphicx}		% to insert PostScript figures
%\usepackage{deluxetable}
\usepackage{rotating}		% for sideways tables/figures
\usepackage{natbib}  % Requires natbib.sty, available from http://ads.harvard.edu/pubs/bibtex/astronat/
\usepackage{savesym}
\usepackage{amssymb}
%\savesymbol{singlespace}
\savesymbol{doublespace}
%\usepackage{wrapfig}
%\usepackage{setspace}
\usepackage{xspace}
\usepackage{color}
\usepackage{multicol}
\usepackage{mdframed}
\usepackage{url}
\usepackage{subfigure}
%\usepackage{emulateapj}
\usepackage{lscape}
\usepackage{grffile}
\usepackage{standalone}
\standalonetrue
\usepackage{import}
\usepackage[utf8]{inputenc}
\usepackage{longtable}
\usepackage{booktabs}



%%%%%%%%%%%%%%%%%%%%%%%%%%%%%%%%%%%%%%%%%%%%%%%%%%%%%%%%%%%%%%%%
%%%%%%%%%%%%       all the preamble material:       %%%%%%%%%%%%
%%%%%%%%%%%%%%%%%%%%%%%%%%%%%%%%%%%%%%%%%%%%%%%%%%%%%%%%%%%%%%%%

% \title{Star Formation in the Galaxy}
% 
% \author{Adam G.}{Ginsburg}
% 
% \otherdegrees{B.S., Rice University, 2007\\
% 	      M.S., University of Colorado, Boulder, 2009}
% 
% \degree{Doctor of Philosophy}		%  #1 {long descr.}
% 	{Ph.D., Rocket Science (ok, fine, astrophysics)}		%  #2 {short descr.}
% 
% \dept{Department of}			%  #1 {designation}
% 	{Astrophysical and Planetary Sciences}		%  #2 {name}
% 
% \advisor{Prof.}				%  #1 {title}
% 	{John Bally}			%  #2 {name}
% 
% \reader{Prof.~Jeremy Darling}		%  2nd person to sign thesis
% \readerThree{Prof.~Jason Glenn}		%  3rd person to sign thesis
% \readerFour{Prof.~Michael Shull}	%  4rd person to sign thesis
% \readerFour{Prof.~Neal Evans}	%  4rd person to sign thesis
% 
% \abstract{  \OnePageChapter	% one page only ??
% 
%     I discovered dust in space.  
% 
% 	}
% 
% 
% \dedication[Dedication]{	% NEVER use \OnePageChapter here.
% 	To 1, the second number in binary.
% 	}
% 
% \acknowledgements{	\OnePageChapter	% *MUST* BE ONLY ONE PAGE!
% 	All y'all.
% 	}
% 
% \ToCisShort	% a 1-page Table of Contents ??
% 
% \LoFisShort	% a 1-page List of Figures ??
% %	\emptyLoF	% no List of Figures at all ??
% 
% \LoTisShort	% a 1-page List of Tables ??
% %	\emptyLoT	% no List of Tables at all ??
% 
% 
% %%%%%%%%%%%%%%%%%%%%%%%%%%%%%%%%%%%%%%%%%%%%%%%%%%%%%%%%%%%%%%%%%
% %%%%%%%%%%%%%%%       BEGIN DOCUMENT...         %%%%%%%%%%%%%%%%%
% %%%%%%%%%%%%%%%%%%%%%%%%%%%%%%%%%%%%%%%%%%%%%%%%%%%%%%%%%%%%%%%%%
% 
% %%%%  footnote style; default=\arabic  (numbered 1,2,3...)
% %%%%  others:  \roman, \Roman, \alph, \Alph, \fnsymbol
% %	"\fnsymbol" uses asterisk, dagger, double-dagger, etc.
% %	\renewcommand{\thefootnote}{\fnsymbol{footnote}}
% %	\setcounter{footnote}{0}

\input{macros}		% file containing author's macro definitions

\begin{document}
% \input{introduction}
% 
% %\input{ch_iras05358}
% \input{ch_w5}
% \input{ch_h2co}
% \input{ch_h2colarge}
% \input{ch_boundhii}
% 
% %\input ch2.tex			% file with Chapter 2 contents
% 
% %%%%%%%%%%%%%%%%%%%%%%%%%%%%%%%%%%%%%%%%%%%%%%%%%%%%%%%%%%%%%%%%%%%
% %%%%%%%%%%%%%%%%%%%%%%%  Bibliography %%%%%%%%%%%%%%%%%%%%%%%%%%%%%
% %%%%%%%%%%%%%%%%%%%%%%%%%%%%%%%%%%%%%%%%%%%%%%%%%%%%%%%%%%%%%%%%%%%
% 
% \bibliographystyle{plain}	% or "siam", or "alpha", or "abbrv"
% 				% see other styles (.bst files) in
% 				% $TEXHOME/texmf/bibtex/bst
% 
% \nocite{*}		% list all refs in database, cited or not.
% 
% \bibliography{thesis}		% bib database file refs.bib
% 
% %%%%%%%%%%%%%%%%%%%%%%%%%%%%%%%%%%%%%%%%%%%%%%%%%%%%%%%%%%%%%%%%%%%
% %%%%%%%%%%%%%%%%%%%%%%%%  Appendices %%%%%%%%%%%%%%%%%%%%%%%%%%%%%%
% %%%%%%%%%%%%%%%%%%%%%%%%%%%%%%%%%%%%%%%%%%%%%%%%%%%%%%%%%%%%%%%%%%%
% 
% \appendix	% don't forget this line if you have appendices!
% 
% %\input appA.tex			% file with Appendix A contents
% %\input appB.tex			% file with Appendix B contents
% 
% %%%%%%%%%%%%%%%%%%%%%%%%%%%%%%%%%%%%%%%%%%%%%%%%%%%%%%%%%%%%%%%%%%%
% %%%%%%%%%%%%%%%%%%%%%%%%   THE END   %%%%%%%%%%%%%%%%%%%%%%%%%%%%%%
% %%%%%%%%%%%%%%%%%%%%%%%%%%%%%%%%%%%%%%%%%%%%%%%%%%%%%%%%%%%%%%%%%%%
% 
% \end{document}
% 
% 

%\bibliographystyle{apj_w_etal}

\chapter{\formaldehyde observations of BGPS sources previously observed with Arecibo}

\section{Introduction}
%Despite intense study, the process of forming massive stars from Giant
%Molecular Clouds (GMCs) is still poorly understood.  

% Millimeter continuum
% studies have begun to reveal the condensations in molecular clouds that will
% likely form into clusters of stars \citep[e.g.][]{Aguirre2011} and many
% high-resolution studies have revealed disks around massive stars
% \citep{Davies2010,Kraus2010,Zapata2009}.  However, the clumps revealed by
% millimeter continuum studies are much larger and often more massive than single
% stars and are therefore not clearly the direct progenitors of stars.  Some of the
% disks around massive stars are massive enough that they may become
% gravitationally unstable and fragment instead of accreting onto the central
% star \citep[e.g.][]{Keto2010,Peters2010}.  The driving question in massive star
% formation still remains open: how do massive stars acquire their mass?

Massive stars are known to form preferentially in clustered environments
\citep{DeWit2005}.  They therefore likely form from ``clumps,'' collections of
gas and dust more dense and compact than Giant Molecular Clouds (GMCs) but larger and
more diffuse than typical low-mass protostellar cores.  ``Clumps'' have been
observed with masses ranging from $10-10^6$\msun\ (but more typically
$10^2-10^3\msun$) and with beam-averaged densities in the range $10^3 \lesssim n(\hh) \lesssim
10^5$ \percc\ and sizes $\sim1$ pc \citep[e.g., ][]{Rosolowsky2010,Dunham2010}.
While giant molecular clouds  in the Galaxy have been surveyed
\citep[e.g., ][]{Jackson2006}, the process by which these clouds condense into
clumps and cores and the mechanisms by which they are dispersed are not
understood. 

It is still not known what sets the final mass of massive stars, but it
is thought that they must ignite while still accreting
\citep{mckee2007}.  Hot O and B stars emitting strongly in the ultraviolet will
ionize their surroundings, creating density-bounded \ion{H}{2} regions.  They
progress from hypercompact through ultracompact and compact and finally diffuse
\ion{H}{2} region phases, during which they either dissociate or blow out their
surrounding medium \citep{Churchwell2002,keto2007}.  The brightest sources in the
Galactic plane in both the free-free continuum in the cm-wavelength regime and
the dust continum in the sub-mm to mm-wavelength regime generally host \uchii\
regions.

%All massive star formation and most low-mass star formation is expected to occur
%in GMCs \citep{McKee and Ostriker?}.  \citet{Tauber1990} and others have viewed
%GMC structure in terms of clump distributions.... mean density in OMC n~150...
%filling factor 0.0002 at 15 pc, 0.02 at 0.15 pc

% Density gradients observed towards UCHIIs with RRLs (Keto, Zhang, Kurtz 2007)

%XXXX maybe move this into a discussion section; too specific
While the gas within \uchii\ regions is hot and ionized, the surrounding gas is
initially molecular.  At the interface between the molecular cloud and
the ionization front, a photon-dominated or photodissociation region appears
\citep{Roshi2005}.  \citet{Churchwell2010} observed HCO$^+$ towards a
sample of \uchii\ regions and noted both infall and outflow motions in
molecular tracers towards these objects.  It should be possible to determine
whether the \uchii\ regions still have collapsing envelopes (infall signatures)
or only disks (outflow signatures) and thereby determine relative evolutionary
states of the regions.

Two centimeter transitions of formaldehyde, \ortho\ \oneone\ (6 cm) and
\twotwo\ (2 cm)\footnote{All references to \formaldehyde\ in this paper,
except where otherwise noted, are to the ortho \ortho\ population, as no para p-\formaldehyde\ 
lines were observed}, have been used to measure the density of molecular
clouds in massive-star-forming regions \citep[e.g., ][]{Dickel1986,Dickel1987},
high-latitude Galactic clouds \citep[e.g., ][]{Turner1989}, the Galactic Center
\citep[e.g., ][]{Zylka1992} starburst galaxies \citep[e.g., ][]{Mangum2008}, and molecular clouds
in a gravitational lens \citep[e.g., ][]{Zeiger2010}.  Studies similar to our own have
been performed by \citet{Wadiak1988} and \citet{Henkel1983}, in which bright
continuum sources were observed in the same transitions with
(approximately) beam-matched telescopes at $\sim$2\arcmin\ resolution. Our
study delves deeper into the spectral line profiles and systematic uncertainties of
\formaldehyde\ densitometry and is performed at higher spatial resolution than
past work.
%These transitions have been accessible to radio
%telescopes for decades, but they have only been accessible at sub-arcminute
%resolution (with single dishes) since the C-band receiver was installed at
%Arecibo.  
% (only 6 cm) pre-stellar cores \citep{Young2010},

%A sample of 21 Ultracompact \ion{H}{2} (UC\ion{H}{2}) regions were observed in
%the \formaldehyde\ \oneone\ transition with the Arecibo Observatory
%\footnote{The Arecibo Observatory is part of the National Astronomy and Ionosphere
%Center, which is operated by Cornell University under a cooperative agreement
%with the National Science Foundation.}
%by \citet{Araya2002}.  An additional 15 observations of \formaldehyde\ \oneone\
%were observed towards suspected massive star forming regions in
%\citet{Araya2004}.  We present follow-up observation of all 21 of the \uchii\
%regions in \citet{Araya2002} and three of the suspected massive star forming
%regions from \citet{Araya2004} in the \twotwo\ transition with the Green Bank
%Telescope (GBT) \footnote{ The National Radio Astronomy Observatory operates
%the GBT and VLA and is a facility of the National Science Foundation operated
%under cooperative agreement by Associated Universities, Inc.  }.  The pair
%of \formaldehyde\ lines are used together to measure the gas volume density
%towards these targets.

This paper presents a pilot study as a proof-of-concept for a much larger
ongoing survey\footnote{GBT project code GBT10B-019} towards 400 lines of sight
and the methodology applicable to the larger survey.

In section \ref{sec:h2coobservations} we present the new observations and describe
other data sets used in our analysis.  Section \ref{sec:models} describes the
modeling procedure used to derive density from the \formaldehyde\ line
observations.  Section \ref{sec:analysis} presents detailed discussion of the
modeling and derivation of physical parameters and their uncertainties.
Section \ref{sec:results} describes the derived and measured values. 
%Section \ref{sec:indiv} presents a detailed discussion for each individual source.  
Section \ref{sec:discussion} discusses the larger implications of our results.
We conclude with a brief summary of important results.

\section{Observations and Data}
\label{sec:h2coobservations}
\subsection{Source Selection} 
The observed lines-of-sight included 21 sources selected from the \citet{Araya2002} \uchii\
sample and 3 from the \citet{Araya2004} ``massive-star forming candidate''
sample.  The sources were selected primarily on the basis of having been
previously observed with Arecibo\footnote{The Arecibo Observatory is part of
the National Astronomy and Ionosphere Center, which is operated by Cornell
University under a cooperative agreement with the National Science Foundation.
} in the \oneone\ transition of \formaldehyde\
with the intent of demonstrating the densitometry method within the Galaxy
rather than making systematic observations of any source class.  Nonetheless,
the \citet{Araya2002} sample includes the majority of the bright \uchii\
regions accessible to Arecibo.  Additionally, there are many detected GMCs
along the line of sight to these \uchii\ regions.

The \citet{Araya2004} observations included 15 pointings towards infrared dark cloud
(IRDC) candidates and High-Mass Protostellar Object (HMPO) candidates.  The
sources we selected from this sample include two sources classified as IRDCs
based on MSX data and one HMPO candidate.  The selection of these sources was
arbitrary; we were only able to observe 24 lines-of-sight in our 4 hour
observation block.  The remaining sources will be discussed in a later paper.
The observed lines of sight are listed in Table \ref{tab:h2comeasured_a}.

% included in original... \LongTables
% maybe use longtables if needed? http://stackoverflow.com/questions/3685832/long-tables-in-latex
% {tab:h2comeasured_a}
\Table{lcccccccc}{Measured \formaldehyde\ \oneone\ line properties}
{\colhead{Source Name\tablenotemark{a}}&\colhead{$l$}&\colhead{$b$}&\colhead{6cm Continuum}&\colhead{Peak}&\colhead{Center}&\colhead{FWHM }&\colhead{RMS}&\colhead{Channel Width}\\
\colhead{           }&\colhead{\degrees}&\colhead{\degrees}&\colhead{(Jy)}&\colhead{(Jy)}&\colhead{(\kms)}&\colhead{(\kms)}&\colhead{(Jy)}&\colhead{(\kms)}\\ }
{tab:h2comeasured_a}{
       G32.80+0.19 0&              0.1904&             32.7968&         2.18 (0.01)&      -0.393 (0.008)&        15.39 (0.05)&         6.57 (0.06)&              0.0049&              1.1374\\
       G32.80+0.19 1&              0.1904&             32.7968&         2.18 (0.01)&      -0.092 (0.008)&        11.45 (0.26)&        10.25 (0.65)&              0.0049&              1.1374\\
       G32.80+0.19 2&              0.1904&             32.7968&         2.18 (0.01)&      -0.063 (0.008)&        80.63 (0.13)&         2.49 (0.36)&              0.0049&              1.1374\\
       G32.80+0.19 3&              0.1904&             32.7968&         2.18 (0.01)&      -0.254 (0.008)&        84.61 (0.02)&         1.37 (0.06)&              0.0049&              1.1374\\
       G32.80+0.19 4&              0.1904&             32.7968&         2.18 (0.01)&      -0.090 (0.008)&        88.66 (0.09)&         3.21 (0.31)&              0.0049&              1.1374\\
       G33.13-0.09 0&             -0.0949&             33.1297&         0.49 (0.00)&      -0.192 (0.007)&        75.92 (0.05)&         3.80 (0.12)&              0.0045&              1.1374\\
       G33.13-0.09 1&             -0.0949&             33.1297&         0.49 (0.00)&      -0.023 (0.007)&        81.62 (0.35)&         2.49 (0.88)&              0.0045&              1.1374\\
       G33.13-0.09 2&             -0.0949&             33.1297&         0.49 (0.00)&      -0.040 (0.007)&       101.50 (0.40)&        11.30 (0.80)&              0.0045&              1.1374\\
       G33.13-0.09 3&             -0.0949&             33.1297&         0.49 (0.00)&      -0.039 (0.007)&        10.39 (0.08)&         2.04 (0.24)&              0.0045&              1.1374\\
       G33.92+0.11 0&              0.1112&              33.914&         0.83 (0.00)&      -0.081 (0.008)&       107.28 (0.18)&         6.62 (0.34)&               0.005&              1.1374\\
       G33.92+0.11 1&              0.1112&              33.914&         0.83 (0.00)&      -0.079 (0.008)&       106.03 (0.06)&         2.41 (0.23)&               0.005&              1.1374\\
       G33.92+0.11 2&              0.1112&              33.914&         0.83 (0.00)&      -0.160 (0.030)&        57.30 (0.40)&        10.60 (0.80)&               0.005&              1.1374\\
       G34.26+0.15 0&              0.1538&             34.2572&         5.57 (0.01)&      -1.828 (0.015)&        60.24 (0.01)&         3.80 (0.03)&              0.0063&              1.1374\\
       G34.26+0.15 1&              0.1538&             34.2572&         5.57 (0.01)&      -0.160 (0.015)&        26.69 (0.08)&         1.04 (0.22)&              0.0063&              1.1374\\
       G34.26+0.15 2&              0.1538&             34.2572&         5.57 (0.01)&      -0.099 (0.015)&        11.25 (0.19)&         2.01 (0.40)&              0.0063&              1.1374\\
       G34.26+0.15 3&              0.1538&             34.2572&         5.57 (0.01)&      -0.126 (0.015)&        51.70 (2.00)&         4.20 (1.00)&              0.0063&              1.1374\\
       G34.26+0.15 4&              0.1538&             34.2572&         5.57 (0.01)&      -0.047 (0.015)&        48.20 (2.00)&         1.80 (1.00)&              0.0063&              1.1374\\
       G35.20-1.74 0&             -1.7409&             35.1997&         5.17 (0.00)&      -1.018 (0.008)&        43.37 (0.01)&         3.67 (0.02)&              0.0051&              1.1374\\
       G35.20-1.74 1&             -1.7409&             35.1997&         5.17 (0.00)&      -0.147 (0.008)&        36.67 (0.10)&         1.49 (0.27)&              0.0051&              1.1374\\
       G35.20-1.74 2&             -1.7409&             35.1997&         5.17 (0.00)&      -0.324 (0.008)&        14.08 (0.01)&         0.93 (0.03)&              0.0051&              1.1374\\
       G35.20-1.74 3&             -1.7409&             35.1997&         5.17 (0.00)&      -0.039 (0.008)&        50.59 (0.53)&         4.92 (1.31)&              0.0051&              1.1374\\
       G35.57-0.03 0&             -0.0306&             35.5779&         0.47 (0.00)&      -0.064 (0.009)&        52.10 (0.10)&         4.60 (0.30)&              0.0053&              1.1374\\
       G35.57-0.03 1&             -0.0306&             35.5779&         0.47 (0.00)&      -0.021 (0.009)&        45.60 (0.30)&         1.90 (0.60)&              0.0053&              1.1374\\
       G35.57-0.03 2&             -0.0306&             35.5779&         0.47 (0.00)&      -0.019 (0.009)&        57.60 (0.50)&         2.90 (0.97)&              0.0053&              1.1374\\
       G35.57-0.03 3&             -0.0306&             35.5779&         0.47 (0.00)&      -0.031 (0.009)&        12.80 (0.20)&         1.84 (0.41)&              0.0053&              1.1374\\
       G35.57-0.03 4&             -0.0306&             35.5779&         0.47 (0.00)&      -0.031 (0.008)&        29.04 (0.11)&         0.82 (0.25)&              0.0053&              1.1374\\
       G35.58+0.07 0&              0.0657&             35.5801&         0.53 (0.01)&      -0.146 (0.004)&        49.37 (0.21)&         5.33 (0.34)&              0.0048&              1.1374\\
       G35.58+0.07 1&              0.0657&             35.5801&         0.53 (0.01)&      -0.049 (0.013)&        53.13 (0.25)&         2.98 (0.64)&              0.0048&              1.1374\\
       G35.58+0.07 2&              0.0657&             35.5801&         0.53 (0.01)&      -0.025 (0.004)&        58.12 (0.29)&         3.63 (0.74)&              0.0048&              1.1374\\
       G35.58+0.07 3&              0.0657&             35.5801&         0.53 (0.01)&      -0.034 (0.004)&        13.24 (0.17)&         2.80 (0.39)&              0.0048&              1.1374\\
       G37.87-0.40 0&             -0.3993&              37.873&         4.40 (0.01)&      -0.531 (0.006)&        60.23 (0.11)&         8.73 (0.35)&              0.0069&              1.1374\\
       G37.87-0.40 1&             -0.3993&              37.873&         4.40 (0.01)&      -0.124 (0.014)&        53.27 (0.19)&         4.03 (0.46)&              0.0069&              1.1374\\
       G37.87-0.40 2&             -0.3993&              37.873&         4.40 (0.01)&      -0.356 (0.019)&        65.13 (0.04)&         2.74 (0.15)&              0.0069&              1.1374\\
       G37.87-0.40 3&             -0.3993&              37.873&         4.40 (0.01)&      -0.324 (0.045)&        72.18 (0.04)&         1.35 (0.14)&              0.0069&              1.1374\\
       G37.87-0.40 4&             -0.3993&              37.873&         4.40 (0.01)&      -0.424 (0.013)&        73.97 (0.13)&         3.01 (0.22)&              0.0069&              1.1374\\
       G37.87-0.40 5&             -0.3993&              37.873&         4.40 (0.01)&      -0.185 (0.012)&        79.98 (0.06)&         1.80 (0.14)&              0.0069&              1.1374\\
       G37.87-0.40 6&             -0.3993&              37.873&         4.40 (0.01)&      -0.114 (0.015)&        91.96 (0.08)&         1.21 (0.18)&              0.0069&              1.1374\\
       G37.87-0.40 7&             -0.3993&              37.873&         4.40 (0.01)&      -0.175 (0.012)&        14.32 (0.14)&         2.94 (0.20)&              0.0069&              1.1374\\
       G37.87-0.40 8&             -0.3993&              37.873&         4.40 (0.01)&      -0.072 (0.022)&        13.16 (0.10)&         0.87 (0.32)&              0.0069&              1.1374\\
       G37.87-0.40 9&             -0.3993&              37.873&         4.40 (0.01)&      -0.137 (0.012)&        20.54 (0.06)&         1.37 (0.14)&              0.0069&              1.1374\\
       G43.89-0.78 0&             -0.7838&             43.8892&         0.66 (0.00)&      -0.181 (0.004)&        54.86 (0.02)&         2.19 (0.06)&              0.0032&              1.1374\\
       G43.89-0.78 1&             -0.7838&             43.8892&         0.66 (0.00)&      -0.020 (0.002)&        50.55 (0.59)&        15.90 (1.20)&              0.0032&              1.1374\\
       G45.07+0.13 0&              0.1323&             45.0711&         0.47 (0.00)&      -0.056 (0.006)&        57.49 (0.10)&         4.24 (0.23)&              0.0035&              1.1374\\
       G45.07+0.13 1&              0.1323&             45.0711&         0.47 (0.00)&      -0.036 (0.006)&        65.44 (0.15)&         4.09 (0.34)&              0.0035&              1.1374\\
       G45.12+0.13 0&              0.1326&             45.1223&         4.28 (0.01)&      -0.188 (0.006)&        55.70 (0.12)&         3.32 (0.24)&              0.0065&              1.1374\\
       G45.12+0.13 1&              0.1326&             45.1223&         4.28 (0.01)&      -0.154 (0.009)&        59.40 (0.13)&         3.11 (0.33)&              0.0065&              1.1374\\
       G45.12+0.13 2&              0.1326&             45.1223&         4.28 (0.01)&      -0.200 (0.010)&        24.86 (0.03)&         1.68 (0.08)&              0.0065&              1.1374\\
       G45.12+0.13 3&              0.1326&             45.1223&         4.28 (0.01)&      -0.027 (0.004)&        65.53 (0.82)&         7.23 (2.03)&              0.0065&              1.1374\\
       G45.45+0.06 0&              0.0593&             45.4548&         4.77 (0.01)&      -1.347 (0.018)&        59.58 (0.02)&         3.18 (0.05)&              0.0063&              1.1374\\
       G45.45+0.06 1&              0.0593&             45.4548&         4.77 (0.01)&      -0.123 (0.040)&        55.34 (0.38)&         3.15 (0.38)&              0.0063&              1.1374\\
       G45.45+0.06 2&              0.0593&             45.4548&         4.77 (0.01)&      -0.056 (0.005)&        25.02 (0.12)&         2.82 (0.28)&              0.0063&              1.1374\\
       G45.47+0.05 0&              0.0455&             45.4655&         0.75 (0.00)&      -0.274 (0.003)&        60.62 (0.03)&         6.59 (0.07)&              0.0039&              1.1374\\
       G45.47+0.05 1&              0.0455&             45.4655&         0.75 (0.00)&      -0.017 (0.004)&        25.55 (0.23)&         2.18 (0.55)&              0.0039&              1.1374\\
       G48.61+0.02 0&              0.0229&             48.6055&         1.01 (0.00)&      -0.067 (0.003)&        18.08 (0.09)&         4.97 (0.22)&              0.0035&              1.1374\\
       G48.61+0.02 1&              0.0229&             48.6055&         1.01 (0.00)&      -0.024 (0.005)&         6.08 (0.13)&         1.20 (0.31)&              0.0035&              1.1374\\
       G48.61+0.02 2&              0.0229&             48.6055&         1.01 (0.00)&      -0.018 (0.003)&        53.73 (0.33)&         4.72 (0.79)&              0.0035&              1.1374\\
       G50.32+0.68 0&              0.6761&             50.3153&         0.24 (0.00)&      -0.011 (0.003)&        26.28 (0.40)&         3.32 (0.94)&              0.0031&              1.1374\\
       G60.88-0.13 0&             -0.1285&             60.8826&         0.66 (0.01)&      -0.093 (0.009)&        22.60 (0.15)&         3.24 (0.35)&              0.0096&              1.1374\\
       G61.48+0.09 0&              0.0893&             61.4769&         6.16 (0.01)&      -0.531 (0.009)&        21.45 (0.02)&         2.81 (0.06)&              0.0084&              1.1374\\
       G69.54-0.98 0&             -0.9759&             69.5398&         0.28 (0.01)&      -0.280 (0.006)&        10.65 (0.05)&         4.55 (0.11)&              0.0076&              1.1374\\
       G70.29+1.60 0&              1.6006&             70.2927&         4.37 (0.13)&      -0.372 (0.008)&       -21.74 (0.07)&         3.92 (0.15)&              0.0108&              1.1374\\
       G70.29+1.60 1&              1.6006&             70.2927&         4.37 (0.13)&      -0.050 (0.007)&       -27.17 (0.58)&         4.86 (1.33)&              0.0108&              1.1374\\
       G70.33+1.59 0&               1.589&             70.3296&         2.21 (0.01)&      -1.201 (0.007)&       -21.24 (0.01)&         3.65 (0.03)&              0.0115&              1.1374\\
   IRAS 20051+3435 0&              0.2088&             32.4662&         0.00 (0.01)&      -0.019 (0.001)&        10.77 (0.07)&         3.60 (0.18)&             0.00071&              2.2747\\
       G41.74+0.10 0&              0.0975&             41.7415&         0.34 (0.00)&      -0.062 (0.004)&        14.60 (0.09)&         2.56 (0.26)&              0.0033&              1.1374\\
       G41.74+0.10 1&              0.0975&             41.7415&         0.34 (0.00)&      -0.020 (0.004)&        10.99 (0.29)&         2.52 (0.71)&              0.0033&              1.1374\\
       G41.74+0.10 2&              0.0975&             41.7415&         0.34 (0.00)&      -0.066 (0.004)&        34.25 (0.05)&         1.63 (0.13)&              0.0033&              1.1374\\
       G41.74+0.10 3&              0.0975&             41.7415&         0.34 (0.00)&      -0.022 (0.005)&        56.61 (0.13)&         1.15 (0.32)&              0.0033&              1.1374\\
       G41.74+0.10 4&              0.0975&             41.7415&         0.34 (0.00)&      -0.043 (0.005)&        17.57 (0.07)&         1.13 (0.18)&              0.0033&              1.1374\\
      IRDC 1923+13 0&             -0.4972&             48.9325&         0.40 (0.00)&      -0.011 (0.001)&        50.20 (0.08)&         1.83 (0.19)&              0.0008&              0.7582\\
      IRDC 1923+13 1&             -0.4972&             48.9325&         0.40 (0.00)&      -0.009 (0.001)&        57.56 (0.09)&         2.57 (0.22)&              0.0008&              0.7582\\
      IRDC 1923+13 2&             -0.4972&             48.9325&         0.40 (0.00)&      -0.005 (0.001)&        47.32 (0.20)&         2.11 (0.51)&              0.0008&              0.7582\\
      IRDC 1916+11 0&             -0.2923&              45.666&         0.00 (0.01)&      -0.005 (0.001)&        25.94 (0.17)&         2.53 (0.41)&             0.00083&              0.7582\\
      IRDC 1916+11 1&             -0.2923&              45.666&         0.00 (0.01)&      -0.013 (0.001)&        55.91 (0.13)&         6.21 (0.34)&             0.00083&              0.7582\\
      IRDC 1916+11 2&             -0.2923&              45.666&         0.00 (0.01)&      -0.003 (0.001)&        48.85 (0.48)&         3.58 (1.13)&             0.00083&              0.7582\\
}{
\tablenotetext{a}{Sources are labeled by the line-of-sight followed by the
number of the component identified, indexed from zero.  The components do not
follow a particular order, but are uniquely identifiable by their velocity,
width, and amplitude.}}

\Table{lccccc}{Measured \formaldehyde\ \twotwo\ line properties}
{\colhead{Source Name}&\colhead{2cm Continuum}&\colhead{Peak\tablenotemark{a}}&\colhead{Center}&\colhead{FWHM }&\colhead{RMS\tablenotemark{b}}\\
\colhead{           }&\colhead{(Jy)}&\colhead{(Jy)}&\colhead{(\kms)}&\colhead{(\kms)}&\colhead{(Jy)}\\ }
{tab:h2comeasured_b}{
       G32.80+0.19 0&         3.68 (0.02)&      -0.519 (0.032)&        15.65 (0.03)&         5.72 (0.08)&              0.0038\\
       G32.80+0.19 1&         3.68 (0.02)&      -0.076 (0.019)&        11.90 (1.18)&         8.17 (0.98)&              0.0038\\
       G32.80+0.19 2&         3.68 (0.02)&      -0.016 (0.001)&        80.47 (0.14)&         4.35 (0.36)&              0.0038\\
       G32.80+0.19 3&         3.68 (0.02)&      -0.065 (0.002)&        84.96 (0.02)&         1.29 (0.05)&              0.0038\\
       G32.80+0.19 4&         3.68 (0.02)&      -0.026 (0.001)&        88.83 (0.06)&         2.31 (0.14)&              0.0038\\
       G33.13-0.09 0&         0.47 (0.02)&      -0.224 (0.003)&        76.17 (0.02)&         3.31 (0.05)&               0.003\\
       G33.13-0.09 1&         0.47 (0.02)&       0.000 (0.000)&         0.00 (0.00)&         0.00 (0.00)&               0.003\\
       G33.13-0.09 2&         0.47 (0.02)&       0.000 (0.000)&         0.00 (0.00)&         0.00 (0.00)&               0.003\\
       G33.13-0.09 3&         0.47 (0.02)&       0.000 (0.000)&         0.00 (0.00)&         0.00 (0.00)&               0.003\\
       G33.92+0.11 0&         0.87 (0.02)&      -0.086 (0.003)&       106.43 (0.03)&         2.17 (0.09)&              0.0032\\
       G33.92+0.11 1&         0.87 (0.02)&      -0.069 (0.002)&       108.83 (0.11)&         6.82 (0.16)&              0.0032\\
       G33.92+0.11 2&         0.87 (0.02)&       0.000 (0.000)&         0.00 (0.00)&         0.00 (0.00)&              0.0032\\
       G34.26+0.15 0&         5.89 (0.02)&      -1.356 (0.006)&        60.99 (0.01)&         3.96 (0.02)&              0.0051\\
       G34.26+0.15 1&         5.89 (0.02)&      -0.046 (0.003)&        27.11 (0.04)&         1.03 (0.09)&              0.0051\\
       G34.26+0.15 2&         5.89 (0.02)&      -0.018 (0.002)&        11.23 (0.16)&         3.19 (0.38)&              0.0051\\
       G34.26+0.15 3&         5.89 (0.02)&      -0.025 (0.004)&        52.82 (0.58)&         6.34 (1.53)&              0.0051\\
       G34.26+0.15 4&         5.89 (0.02)&      -0.018 (0.007)&        47.05 (0.47)&         2.47 (1.15)&              0.0051\\
       G35.20-1.74 0&         5.98 (0.03)&      -0.482 (0.004)&        43.38 (0.02)&         3.71 (0.04)&              0.0055\\
       G35.20-1.74 1&         5.98 (0.03)&      -0.028 (0.005)&        37.91 (0.32)&         3.46 (0.76)&              0.0055\\
       G35.20-1.74 2&         5.98 (0.03)&      -0.056 (0.003)&        14.18 (0.02)&         1.00 (0.05)&              0.0055\\
       G35.20-1.74 3&         5.98 (0.03)&       0.000 (0.000)&         0.00 (0.00)&         0.00 (0.00)&              0.0055\\
       G35.57-0.03 0&         0.32 (0.15)&      -0.075 (0.003)&        52.14 (0.09)&         4.39 (0.21)&              0.0046\\
       G35.57-0.03 1&         0.32 (0.15)&      -0.015 (0.006)&        47.39 (0.25)&         1.31 (0.60)&              0.0046\\
       G35.57-0.03 2&         0.32 (0.15)&       0.000 (0.000)&         0.00 (0.00)&         0.00 (0.00)&              0.0046\\
       G35.57-0.03 3&         0.32 (0.15)&       0.000 (0.000)&         0.00 (0.00)&         0.00 (0.00)&              0.0046\\
       G35.57-0.03 4&         0.32 (0.15)&      -0.024 (0.008)&        29.25 (0.11)&         0.43 (0.15)&              0.0046\\
       G35.58+0.07 0&         0.23 (0.09)&      -0.106 (0.002)&        49.21 (0.06)&         5.00 (0.14)&              0.0031\\
       G35.58+0.07 1&         0.23 (0.09)&       0.000 (0.004)&         0.00 (0.00)&         0.00 (0.00)&              0.0031\\
       G35.58+0.07 2&         0.23 (0.09)&       0.000 (0.004)&         0.00 (0.00)&         0.00 (0.00)&              0.0031\\
       G35.58+0.07 3&         0.23 (0.09)&       0.000 (0.004)&         0.00 (0.00)&         0.00 (0.00)&              0.0031\\
       G37.87-0.40 0&         3.73 (0.02)&      -0.221 (0.003)&        59.99 (0.12)&         8.53 (0.14)&              0.0048\\
       G37.87-0.40 1&         3.73 (0.02)&      -0.045 (0.007)&        54.55 (0.25)&         5.99 (0.34)&              0.0048\\
       G37.87-0.40 2&         3.73 (0.02)&      -0.036 (0.007)&        65.06 (0.11)&         2.57 (0.45)&              0.0048\\
       G37.87-0.40 3&         3.73 (0.02)&      -0.053 (0.003)&        72.44 (0.05)&         1.37 (0.08)&              0.0048\\
       G37.87-0.40 4&         3.73 (0.02)&      -0.047 (0.002)&        74.25 (0.07)&         2.07 (0.18)&              0.0048\\
       G37.87-0.40 5&         3.73 (0.02)&      -0.016 (0.001)&        80.04 (0.03)&         1.28 (0.07)&              0.0048\\
       G37.87-0.40 6&         3.73 (0.02)&      -0.010 (0.002)&        91.99 (0.12)&         1.60 (0.28)&              0.0048\\
       G37.87-0.40 7&         3.73 (0.02)&      -0.026 (0.002)&        14.89 (0.12)&         1.40 (0.20)&              0.0048\\
       G37.87-0.40 8&         3.73 (0.02)&      -0.017 (0.002)&        13.29 (0.19)&         1.52 (0.34)&              0.0048\\
       G37.87-0.40 9&         3.73 (0.02)&      -0.017 (0.001)&        20.52 (0.10)&         3.09 (0.23)&              0.0048\\
       G43.89-0.78 0&         0.53 (0.02)&      -0.059 (0.004)&        54.61 (0.08)&         2.85 (0.23)&               0.003\\
       G43.89-0.78 1&         0.53 (0.02)&      -0.015 (0.002)&        49.59 (0.94)&        14.49 (1.69)&               0.003\\
       G45.07+0.13 0&         0.79 (0.07)&      -0.073 (0.003)&        57.18 (0.08)&         3.45 (0.18)&              0.0029\\
       G45.07+0.13 1&         0.79 (0.07)&      -0.011 (0.003)&        65.67 (0.42)&         3.46 (0.98)&              0.0029\\
       G45.12+0.13 0&         5.20 (0.20)&      -0.086 (0.002)&        56.21 (0.11)&         5.22 (0.21)&              0.0044\\
       G45.12+0.13 1&         5.20 (0.20)&      -0.059 (0.005)&        59.70 (0.06)&         2.42 (0.16)&              0.0044\\
       G45.12+0.13 2&         5.20 (0.20)&      -0.047 (0.002)&        25.14 (0.04)&         1.55 (0.09)&              0.0044\\
       G45.12+0.13 3&         5.20 (0.20)&      -0.021 (0.001)&        64.68 (0.39)&         8.15 (0.87)&              0.0044\\
       G45.45+0.06 0&         3.16 (0.02)&      -0.260 (0.003)&        59.58 (0.01)&         2.06 (0.03)&              0.0043\\
       G45.45+0.06 1&         3.16 (0.02)&      -0.042 (0.002)&        57.90 (0.14)&         9.40 (0.31)&              0.0043\\
       G45.45+0.06 2&         3.16 (0.02)&       0.000 (0.000)&         0.00 (0.00)&         0.00 (0.00)&              0.0043\\
       G45.47+0.05 0&         0.38 (0.02)&      -0.124 (0.003)&        61.67 (0.07)&         5.85 (0.17)&              0.0049\\
       G45.47+0.05 1&         0.38 (0.02)&      -0.000 (0.007)&         0.00 (0.00)&         0.00 (0.00)&              0.0049\\
       G48.61+0.02 0&         0.41 (0.02)&      -0.022 (0.003)&        18.50 (0.25)&         4.39 (0.59)&              0.0033\\
       G48.61+0.02 1&         0.41 (0.02)&      -0.000 (0.000)&         0.00 (0.00)&         0.00 (0.00)&              0.0033\\
       G48.61+0.02 2&         0.41 (0.02)&      -0.005 (0.002)&        52.50 (1.25)&         7.47 (2.94)&              0.0033\\
       G50.32+0.68 0&         0.16 (0.02)&      -0.011 (0.003)&        26.21 (0.44)&         3.10 (1.03)&              0.0036\\
       G60.88-0.13 0&         0.29 (0.02)&      -0.016 (0.003)&        21.63 (0.21)&         2.47 (0.50)&               0.003\\
       G61.48+0.09 0&         3.42 (0.02)&      -0.300 (0.004)&        21.40 (0.02)&         2.39 (0.04)&              0.0037\\
       G69.54-0.98 0&         0.23 (0.02)&      -0.220 (0.002)&         9.97 (0.03)&         5.81 (0.08)&              0.0031\\
       G70.29+1.60 0&         6.21 (0.02)&      -0.159 (0.003)&       -23.52 (0.06)&         5.36 (0.13)&              0.0046\\
       G70.29+1.60 1&         6.21 (0.02)&      -0.000 (0.000)&        -0.00 (0.00)&         0.00 (0.00)&              0.0046\\
       G70.33+1.59 0&         2.68 (0.02)&      -1.081 (0.005)&       -21.17 (0.01)&         2.95 (0.01)&              0.0038\\
   IRAS 20051+3435 0&         0.00 (0.02)&      -0.016 (0.003)&        11.51 (0.37)&         4.14 (0.88)&              0.0032\\
       G41.74+0.10 0&         0.28 (0.02)&      -0.014 (0.002)&        14.36 (0.34)&         3.80 (0.80)&              0.0032\\
       G41.74+0.10 1&         0.28 (0.02)&       0.000 (0.004)&         0.00 (0.00)&         0.00 (0.00)&              0.0032\\
       G41.74+0.10 2&         0.28 (0.02)&       0.000 (0.004)&         0.00 (0.00)&         0.00 (0.00)&              0.0032\\
       G41.74+0.10 3&         0.28 (0.02)&       0.000 (0.004)&         0.00 (0.00)&         0.00 (0.00)&              0.0032\\
       G41.74+0.10 4&         0.28 (0.02)&       0.000 (0.004)&         0.00 (0.00)&         0.00 (0.00)&              0.0032\\
      IRDC 1923+13 0&         0.00 (0.02)&       0.000 (0.000)&         0.00 (0.00)&         0.00 (0.00)&              0.0032\\
      IRDC 1923+13 1&         0.00 (0.02)&       0.000 (0.000)&         0.00 (0.00)&         0.00 (0.00)&              0.0032\\
      IRDC 1923+13 2&         0.00 (0.02)&       0.000 (0.000)&         0.00 (0.00)&         0.00 (0.00)&              0.0032\\
      IRDC 1916+11 0&         0.00 (0.02)&       0.000 (0.000)&         0.00 (0.00)&         0.00 (0.00)&              0.0048\\
      IRDC 1916+11 1&         0.00 (0.02)&       0.000 (0.000)&         0.00 (0.00)&         0.00 (0.00)&              0.0048\\
      IRDC 1916+11 2&         0.00 (0.02)&       0.000 (0.000)&         0.00 (0.00)&         0.00 (0.00)&              0.0048\\
}{
\tablenotetext{a}{ The Upper Limit Flag is 1 when the measurement indicated is
a $3-\sigma$ upper limit on the \twotwo\ line depth when there is a
corresponding \oneone\ line detection. }
\tablenotetext{b}{RMS in 1.011 \kms\ channels.} 
}

\Table{lccccccc}{Distance, BGPS 1.1 mm, and other properties}
{\colhead{Source Name}&\colhead{Distance}&\colhead{Galactocentric}&\colhead{KDA\tablenotemark{a}}&\colhead{$S_{1.1mm}$}&\colhead{Source}&\colhead{\formaldehyde\ }&\colhead{Scenario\tablenotemark{b}}\\
\colhead{           }&\colhead{        }&\colhead{Distance      }&\colhead{Resolution}&\colhead{           }&\colhead{Type  }&\colhead{Spectrum}&\colhead{}\\  
\colhead{           }&\colhead{(kpc)   }&\colhead{         (kpc)}&\colhead{          }&\colhead{(Jy)       }&\colhead{      }&\colhead{Type    }&\colhead{}\\ }
{tab:other}{
       G32.80+0.19 0&                12.9&                 7.4&                 far&                6.94&               UCHII&        red gradient&                 2+3\\
       G32.80+0.19 1&                13.1&                 7.6&                 far&                6.94&               UCHII&            envelope&                 2+3\\
       G32.80+0.19 2&                 9.4&                 5.1&                 far&                6.94&                 GMC&                   -&                 2+3\\
       G32.80+0.19 3&                 9.2&                 5.0&                 far&                6.94&                 GMC&                   -&                 2+3\\
       G32.80+0.19 4&                 9.0&                 4.9&                 far&                6.94&                 GMC&                   -&                 2+3\\
       G33.13-0.09 0&                 9.6&                 5.2&                 far&                2.26&               UCHII&        red gradient&                   2\\
       G33.13-0.09 1&                 9.3&                 5.1&                 far&                2.26&                 GMC&            envelope&                   2\\
       G33.13-0.09 2&                 7.1&                 4.7&             tangent&                2.26&                 GMC&                   -&                   2\\
       G33.13-0.09 3&                 0.9&                 7.6&                near&                2.26&                 GMC&                   -&                   2\\
       G33.92+0.11 0&                 7.0&                 4.6&             tangent&                3.86&               UCHII&        red gradient&                   2\\
       G33.92+0.11 1&                 7.0&                 4.6&             tangent&                3.86&               UCHII&            envelope&                   2\\
       G33.92+0.11 2&                 3.6&                 5.8&                near&                3.86&                 GMC&                   -&                   2\\
       G34.26+0.15 0&                 3.6&                 5.7&                near&               35.69&               UCHII&        red gradient&                   2\\
       G34.26+0.15 1&                 1.9&                 6.9&                near&               35.69&                 GMC&                   -&                   2\\
       G34.26+0.15 2&                 1.0&                 7.6&                near&               35.69&                 GMC&                   -&                   2\\
       G34.26+0.15 3&                 3.6&                 6.0&                near&               35.69&                 GMC&            envelope&                   2\\
       G34.26+0.15 4&                 3.6&                 6.1&                near&               35.69&                 GMC&                   -&                   2\\
       G35.20-1.74 0&                 2.8&                 6.3&                near&                   -&               UCHII&              single&                   4\\
       G35.20-1.74 1&                 2.5&                 6.5&                near&                   -&                 GMC&                   -&                   4\\
       G35.20-1.74 2&                 1.1&                 7.5&                near&                   -&                 GMC&                   -&                   4\\
       G35.20-1.74 3&                 3.2&                 6.1&                near&                   -&                 GMC&                   -&                   4\\
       G35.57-0.03 0&                10.3&                 6.0&                 far&                2.57&               UCHII&              single&                 2+3\\
       G35.57-0.03 1&                10.7&                 6.2&                 far&                2.57&                 GMC&                   -&                 2+3\\
       G35.57-0.03 2&                 3.6&                 5.9&                near&                2.57&                 GMC&                   -&                 2+3\\
       G35.57-0.03 3&                 1.1&                 7.6&                near&                2.57&                 GMC&                   -&                 2+3\\
       G35.57-0.03 4&                 2.0&                 6.8&                near&                2.57&                 GMC&                   -&                 2+3\\
       G35.58+0.07 0&                10.5&                 6.1&                 far&                1.44&               UCHII&       blue gradient&                   2\\
       G35.58+0.07 1&                10.3&                 6.0&                 far&                1.44&               UCHII&                   -&                   2\\
       G35.58+0.07 2&                 3.6&                 5.8&                near&                1.44&                 GMC&                   -&                   2\\
       G35.58+0.07 3&                 1.1&                 7.5&                near&                1.44&                 GMC&                   -&                   2\\
       G37.87-0.40 0&                 9.4&                 5.9&                 far&                4.14&               UCHII&       blue gradient&                   1\\
       G37.87-0.40 1&                 9.8&                 6.1&                 far&                4.14&               UCHII&       blue gradient&                   1\\
       G37.87-0.40 2&                 9.2&                 5.7&                 far&                4.14&               UCHII&       blue gradient&                   1\\
       G37.87-0.40 3&                 8.7&                 5.6&                 far&                4.14&                 GMC&                   -&                   1\\
       G37.87-0.40 4&                 8.6&                 5.5&                 far&                4.14&                 GMC&                   -&                   1\\
       G37.87-0.40 5&                 8.1&                 5.4&                 far&                4.14&                 GMC&                   -&                   1\\
       G37.87-0.40 6&                 6.6&                 5.1&             tangent&                4.14&                 GMC&                   -&                   1\\
       G37.87-0.40 7&                 1.2&                 7.5&                near&                4.14&                 GMC&                   -&                   1\\
       G37.87-0.40 8&                 1.1&                 7.6&                near&                4.14&                 GMC&                   -&                   1\\
       G37.87-0.40 9&                 1.5&                 7.2&                near&                4.14&                 GMC&                   -&                   1\\
       G43.89-0.78 0&                 8.3&                 6.2&                 far&                   -&               UCHII&       blue gradient&                   3\\
       G43.89-0.78 1&                 8.6&                 6.3&                 far&                   -&                 GMC&            envelope&                   3\\
       G45.07+0.13 0&                 7.6&                 6.2&                 far&                4.26&               UCHII&              single&                   2\\
       G45.07+0.13 1&                 6.5&                 6.0&                 far&                4.26&                 GMC&                   -&                   2\\
       G45.12+0.13 0&                 7.4&                 6.2&                 far&                6.78&               UCHII&               other&                   1\\
       G45.12+0.13 1&                 7.4&                 6.1&                 far&                6.78&               UCHII&            envelope&                   1\\
       G45.12+0.13 2&                 1.9&                 7.2&                near&                6.78&                 GMC&                   -&                   1\\
       G45.12+0.13 3&                 7.4&                 6.0&                 far&                6.78&                 GMC&            envelope&                   1\\
       G45.45+0.06 0&                 7.2&                 6.1&                 far&                3.71&               UCHII&       blue gradient&                   2\\
       G45.45+0.06 1&                 7.6&                 6.2&                 far&                3.71&                 GMC&            envelope&                   2\\
       G45.45+0.06 2&                 1.9&                 7.2&                near&                3.71&                 GMC&                   -&                   2\\
       G45.47+0.05 0&                 7.1&                 6.1&                 far&                3.34&               UCHII&        red gradient&               1+2+3\\
       G45.47+0.05 1&                 1.9&                 7.2&                near&                3.34&                 GMC&                   -&               1+2+3\\
       G48.61+0.02 0&                 9.6&                 7.5&                 far&                2.20&               UCHII&        red gradient&                 2+3\\
       G48.61+0.02 1&                 0.7&                 8.0&                near&                2.20&                 GMC&                   -&                 2+3\\
       G48.61+0.02 2&                 6.5&                 6.4&                 far&                2.20&                 GMC&                   -&                 2+3\\
       G50.32+0.68 0&                 2.1&                 7.2&                near&                   -&               UCHII&                   -&                   1\\
       G60.88-0.13 0&                 2.8&                 7.4&                near&                4.90&               UCHII&               limit&                   2\\
       G61.48+0.09 0&                 5.2&                 7.5&                 far&                7.86&               UCHII&              single&                   4\\
       G69.54-0.98 0&                2.57&                 7.9&             tangent&                   -&               UCHII&               thick&                 4+5\\
       G70.29+1.60 0&                 7.3&                 9.1&                 far&                   -&               UCHII&       blue gradient&                   2\\
       G70.29+1.60 1&                 7.8&                 9.3&                 far&                   -&                 GMC&            envelope&                   2\\
       G70.33+1.59 0&                 7.3&                 9.1&                 far&                   -&               UCHII&              single&                 1+2\\
   IRAS 20051+3435 0&                 2.6&                 7.6&             tangent&                   -&                 GMC&               limit&                  -1\\
       G41.74+0.10 0&                11.3&                 7.6&                 far&                0.56&               UCHII&               limit&                  -1\\
       G41.74+0.10 1&                11.6&                 7.7&                 far&                0.56&               UCHII&                   -&                  -1\\
       G41.74+0.10 2&                 2.4&                 6.8&                near&                0.56&                 GMC&                   -&                  -1\\
       G41.74+0.10 3&                 3.8&                 6.1&                near&                0.56&                 GMC&                   -&                  -1\\
       G41.74+0.10 4&                11.2&                 7.4&                 far&                0.56&               UCHII&                   -&                  -1\\
      IRDC 1923+13 0&                 4.2&                 6.5&                near&                   -&                 GMC&               limit&                  -1\\
      IRDC 1923+13 1&                 5.5&                 6.3&             tangent&                   -&                 GMC&                   -&                  -1\\
      IRDC 1923+13 2&                 3.8&                 6.6&                near&                   -&                 GMC&                   -&                  -1\\
      IRDC 1916+11 0&                 2.0&                 7.2&                near&                   -&                 GMC&               limit&                  -1\\
      IRDC 1916+11 1&                 4.2&                 6.2&                near&                   -&                 GMC&                   -&                  -1\\
      IRDC 1916+11 2&                 3.6&                 6.4&                near&                   -&                 GMC&                   -&                  -1\\
}{
\tablenotetext{a}{The Kinematic Distance Ambiguity described in Section \ref{sec:distances}.}
\tablenotetext{b}{Scenario or scenarios most likely to be consistent with the observed spectrum, as described
in Section \ref{sec:scenarios}.  In some cases, the spectrum was consistent with multiple scenarios or some
blend of multiple scenarios.  In others, the source could not be classified, in which case it is marked with 
-1 in this column. }
}

\Table{lcccccccc}{Inferred \formaldehyde\ line properties}
{\colhead{Source Name}&\colhead{$\tau_{1-1}$}&\colhead{$\tau_{1-1}$ (FFC)}&\colhead{$\tau_{2-2}$}&\colhead{$\tau_{2-2}$ (FFC)}&\colhead{2-2 Upper }&\colhead{2cm Area\tablenotemark{a}}&\colhead{6cm Area  \tablenotemark{a}   }&\colhead{FFC Error}\\
\colhead{           }&\colhead{             }&\colhead{                   }&\colhead{             }&\colhead{                   }&\colhead{Limit Flag}&\colhead{\arcsec$^2$                   }&\colhead{\arcsec$^2$                   }&\colhead{}\\ }
{tab:h2coinferred}{
       G32.80+0.19 0&        0.18 (0.055)&         0.2 (0.059)&        0.12 (0.024)&        0.15 (0.031)&                   0&                88.0&               226.2&                 0.1\\
       G32.80+0.19 1&        0.04 (0.013)&       0.043 (0.013)&      0.016 (0.0051)&      0.021 (0.0065)&                   0&                88.0&               226.2&                 0.1\\
       G32.80+0.19 2&      0.027 (0.0089)&      0.029 (0.0095)&    0.0033 (0.00069)&    0.0042 (0.00088)&                   0&                88.0&               226.2&                 0.1\\
       G32.80+0.19 3&        0.11 (0.035)&        0.12 (0.037)&      0.014 (0.0028)&      0.018 (0.0035)&                   0&                88.0&               226.2&                 0.1\\
       G32.80+0.19 4&       0.039 (0.012)&       0.042 (0.013)&     0.0055 (0.0011)&     0.0071 (0.0014)&                   0&                88.0&               226.2&                 0.1\\
       G33.13-0.09 0&          0.34 (0.1)&         0.49 (0.15)&        0.16 (0.032)&         0.63 (0.12)&                   0&                33.5&                33.5&                 0.2\\
       G33.13-0.09 1&       0.035 (0.015)&        0.047 (0.02)&         0 (0.0059)&         0 (0.0031)&                   1&                33.5&                33.5&                 0.2\\
       G33.13-0.09 2&       0.062 (0.022)&       0.084 (0.029)&         0 (0.0059)&         0 (0.0031)&                   1&                33.5&                33.5&                 0.2\\
       G33.13-0.09 3&       0.061 (0.021)&       0.082 (0.028)&         0 (0.0059)&         0 (0.0031)&                   1&                33.5&                33.5&                 0.2\\
       G33.92+0.11 0&       0.084 (0.027)&         0.1 (0.031)&      0.045 (0.0091)&       0.094 (0.018)&                   0&               214.0&               214.0&                 0.2\\
       G33.92+0.11 1&       0.082 (0.026)&       0.098 (0.031)&      0.036 (0.0072)&       0.075 (0.014)&                   0&               214.0&               214.0&                 0.2\\
       G33.92+0.11 2&        0.17 (0.062)&        0.21 (0.074)&         0 (0.0049)&         0 (0.0031)&                   1&               214.0&               214.0&                 0.2\\
       G34.26+0.15 0&         0.38 (0.12)&          0.4 (0.12)&        0.22 (0.043)&        0.26 (0.052)&                   0&                10.9&                10.9&                 0.2\\
       G34.26+0.15 1&      0.028 (0.0089)&      0.029 (0.0092)&     0.0067 (0.0014)&     0.0079 (0.0017)&                   0&                10.9&                10.9&                 0.2\\
       G34.26+0.15 2&      0.017 (0.0059)&       0.018 (0.006)&    0.0026 (0.00059)&     0.0031 (0.0007)&                   0&                10.9&                10.9&                 0.2\\
       G34.26+0.15 3&      0.022 (0.0072)&      0.023 (0.0074)&    0.0036 (0.00092)&     0.0043 (0.0011)&                   0&                10.9&                10.9&                 0.2\\
       G34.26+0.15 4&     0.0082 (0.0036)&     0.0085 (0.0037)&     0.0026 (0.0011)&      0.003 (0.0013)&                   0&                10.9&                10.9&                 0.2\\
       G35.20-1.74 0&        0.21 (0.063)&        0.22 (0.066)&       0.071 (0.014)&       0.084 (0.017)&                   0&                39.5&                39.5&                 0.2\\
       G35.20-1.74 1&      0.028 (0.0085)&      0.029 (0.0088)&     0.0039 (0.0011)&     0.0046 (0.0013)&                   0&                39.5&                39.5&                 0.2\\
       G35.20-1.74 2&       0.063 (0.019)&       0.065 (0.019)&      0.008 (0.0017)&     0.0095 (0.0019)&                   0&                39.5&                39.5&                 0.2\\
       G35.20-1.74 3&     0.0073 (0.0027)&     0.0075 (0.0028)&         0 (0.0023)&         0 (0.0031)&                   1&                39.5&                39.5&                 0.2\\
       G35.57-0.03 0&        0.11 (0.035)&        0.15 (0.049)&       0.056 (0.011)&        0.26 (0.054)&                   0&                 6.7&                 6.7&                 0.1\\
       G35.57-0.03 1&       0.034 (0.018)&       0.046 (0.024)&      0.011 (0.0047)&        0.047 (0.02)&                   0&                 6.7&                 6.7&                 0.1\\
       G35.57-0.03 2&        0.03 (0.017)&       0.042 (0.023)&         0 (0.0099)&          0 (0.019)&                   1&                 6.7&                 6.7&                 0.1\\
       G35.57-0.03 3&        0.05 (0.021)&       0.069 (0.029)&         0 (0.0099)&          0 (0.019)&                   1&                 6.7&                 6.7&                 0.1\\
       G35.57-0.03 4&        0.051 (0.02)&       0.069 (0.028)&      0.017 (0.0065)&       0.077 (0.029)&                   0&                 6.7&                 6.7&                 0.1\\
       G35.58+0.07 0&        0.24 (0.071)&        0.32 (0.097)&       0.085 (0.017)&         0.61 (0.12)&                   0&                 2.1&                 2.1&                 0.2\\
       G35.58+0.07 1&       0.072 (0.029)&       0.096 (0.038)&         0 (0.0072)&          0 (0.019)&                   1&                 2.1&                 2.1&                 0.2\\
       G35.58+0.07 2&       0.037 (0.012)&       0.049 (0.016)&         0 (0.0072)&          0 (0.019)&                   1&                 2.1&                 2.1&                 0.2\\
       G35.58+0.07 3&        0.05 (0.016)&       0.066 (0.021)&         0 (0.0072)&           0 (0.01)&                   1&                 2.1&                 2.1&                 0.2\\
       G37.87-0.40 0&        0.12 (0.037)&        0.13 (0.038)&      0.047 (0.0095)&       0.061 (0.012)&                   0&                27.5&               170.9&                 0.2\\
       G37.87-0.40 1&      0.028 (0.0089)&      0.029 (0.0092)&     0.0095 (0.0024)&      0.012 (0.0031)&                   0&                27.5&               170.9&                 0.2\\
       G37.87-0.40 2&       0.081 (0.025)&       0.084 (0.026)&     0.0075 (0.0021)&     0.0096 (0.0027)&                   0&                27.5&               170.9&                 0.2\\
       G37.87-0.40 3&       0.074 (0.024)&       0.076 (0.025)&      0.011 (0.0023)&       0.014 (0.003)&                   0&                27.5&               170.9&                 0.2\\
       G37.87-0.40 4&       0.097 (0.029)&          0.1 (0.03)&      0.0098 (0.002)&      0.013 (0.0026)&                   0&                27.5&               170.9&                 0.2\\
       G37.87-0.40 5&       0.041 (0.013)&       0.043 (0.013)&    0.0033 (0.00068)&    0.0043 (0.00087)&                   0&                27.5&               170.9&                 0.2\\
       G37.87-0.40 6&      0.025 (0.0083)&      0.026 (0.0086)&    0.0021 (0.00052)&    0.0026 (0.00066)&                   0&                27.5&               170.9&                 0.2\\
       G37.87-0.40 7&       0.039 (0.012)&        0.04 (0.012)&     0.0054 (0.0012)&     0.0069 (0.0015)&                   0&                27.5&               170.9&                 0.2\\
       G37.87-0.40 8&      0.016 (0.0069)&      0.016 (0.0071)&    0.0035 (0.00081)&      0.0046 (0.001)&                   0&                27.5&               170.9&                 0.2\\
       G37.87-0.40 9&       0.03 (0.0095)&      0.031 (0.0098)&    0.0035 (0.00073)&    0.0045 (0.00094)&                   0&                27.5&               170.9&                 0.2\\
       G43.89-0.78 0&        0.25 (0.074)&        0.32 (0.096)&      0.037 (0.0078)&        0.12 (0.024)&                   0&                13.5&                13.5&                 0.1\\
       G43.89-0.78 1&      0.025 (0.0077)&      0.031 (0.0097)&     0.0097 (0.0022)&      0.029 (0.0067)&                   0&                13.5&                13.5&                 0.1\\
       G45.07+0.13 0&       0.092 (0.029)&         0.13 (0.04)&       0.04 (0.0081)&        0.096 (0.02)&                   0&                 2.5&                 2.5&                 0.2\\
       G45.07+0.13 1&        0.058 (0.02)&        0.08 (0.027)&     0.0061 (0.0019)&      0.014 (0.0045)&                   0&                 2.5&                 2.5&                 0.2\\
       G45.12+0.13 0&       0.043 (0.013)&       0.045 (0.013)&      0.014 (0.0028)&      0.017 (0.0033)&                   0&                15.4&               516.6&                 0.2\\
       G45.12+0.13 1&       0.035 (0.011)&       0.036 (0.011)&      0.0095 (0.002)&      0.011 (0.0025)&                   0&                15.4&               516.6&                 0.2\\
       G45.12+0.13 2&       0.046 (0.014)&       0.048 (0.014)&     0.0075 (0.0015)&      0.009 (0.0019)&                   0&                15.4&               516.6&                 0.2\\
       G45.12+0.13 3&       0.006 (0.002)&     0.0062 (0.0021)&    0.0033 (0.00068)&     0.004 (0.00082)&                   0&                15.4&               516.6&                 0.2\\
       G45.45+0.06 0&        0.32 (0.096)&        0.32 (0.095)&       0.063 (0.013)&       0.069 (0.012)&                   0&              1963.0&              1963.0&                 0.2\\
       G45.45+0.06 1&       0.025 (0.011)&       0.026 (0.011)&       0.01 (0.0021)&      0.011 (0.0019)&                   0&              1963.0&              1963.0&                 0.2\\
       G45.45+0.06 2&      0.011 (0.0036)&      0.012 (0.0035)&         0 (0.0031)&           0 (0.01)&                   1&              1963.0&              1963.0&                 0.2\\
       G45.47+0.05 0&         0.35 (0.11)&         0.45 (0.14)&       0.089 (0.018)&        0.39 (0.079)&                   0&                 3.0&                 3.0&                 0.2\\
       G45.47+0.05 1&      0.018 (0.0068)&      0.023 (0.0084)&           0 (0.01)&           0 (0.01)&                   1&                 3.0&                 3.0&                 0.2\\
       G48.61+0.02 0&       0.058 (0.018)&       0.068 (0.021)&      0.015 (0.0034)&       0.053 (0.012)&                   0&                25.5&                25.5&                 0.2\\
       G48.61+0.02 1&       0.02 (0.0075)&      0.023 (0.0088)&         0 (0.0067)&         0 (0.0026)&                   1&                25.5&                25.5&                 0.2\\
       G48.61+0.02 2&      0.016 (0.0052)&      0.018 (0.0061)&     0.0033 (0.0013)&      0.012 (0.0046)&                   0&                25.5&                25.5&                 0.2\\
       G50.32+0.68 0&        0.027 (0.01)&       0.045 (0.017)&     0.0089 (0.0031)&       0.058 (0.019)&                   0&               108.0&               108.0&                 0.2\\
       G60.88-0.13 0&        0.12 (0.037)&        0.14 (0.043)&      0.011 (0.0031)&      0.031 (0.0071)&                   0&               615.0&               615.0&                 0.2\\
       G61.48+0.09 0&       0.088 (0.026)&        0.09 (0.027)&       0.069 (0.014)&       0.088 (0.017)&                   0&               355.0&               355.0&                 0.2\\
       G69.54-0.98 0&         0.98 (0.29)&           5.7 (1.7)&        0.18 (0.037)&          2.9 (0.57)&                   0&                 0.5&                 0.5&                 0.2\\
       G70.29+1.60 0&       0.086 (0.026)&       0.089 (0.027)&      0.022 (0.0044)&      0.026 (0.0052)&                   0&                52.8&                52.8&                 0.1\\
       G70.29+1.60 1&      0.011 (0.0037)&      0.012 (0.0038)&         0 (0.0019)&         0 (0.0026)&                   1&                52.8&                52.8&                 0.1\\
       G70.33+1.59 0&          0.7 (0.21)&         0.78 (0.24)&        0.34 (0.068)&          0.52 (0.1)&                   0&                16.4&                16.4&                 0.1\\
   IRAS 20051+3435 0&        0.12 (0.036)&        0.13 (0.014)&      0.015 (0.0041)&      0.016 (0.0034)&                   0&             2747.75&             2747.75&                 0.0\\
       G41.74+0.10 0&         0.13 (0.04)&          0.2 (0.06)&       0.01 (0.0027)&       0.045 (0.012)&                   0&                75.2&                75.2&                 0.2\\
       G41.74+0.10 1&        0.04 (0.014)&        0.06 (0.021)&         0 (0.0071)&         0 (0.0026)&                   1&                75.2&                75.2&                 0.2\\
       G41.74+0.10 2&        0.14 (0.043)&        0.21 (0.065)&         0 (0.0071)&         0 (0.0026)&                   1&                75.2&                75.2&                 0.2\\
       G41.74+0.10 3&       0.045 (0.017)&       0.067 (0.025)&         0 (0.0071)&         0 (0.0026)&                   1&                75.2&                75.2&                 0.2\\
       G41.74+0.10 4&       0.089 (0.029)&        0.13 (0.043)&         0 (0.0071)&         0 (0.0028)&                   1&                75.2&                75.2&                 0.2\\
      IRDC 1923+13 0&       0.02 (0.0062)&       0.02 (0.0047)&          0 (0.009)&         0 (0.0028)&                   1&             2747.75&             2747.75&                 0.0\\
      IRDC 1923+13 1&      0.016 (0.0051)&      0.017 (0.0038)&          0 (0.009)&         0 (0.0028)&                   1&             2747.75&             2747.75&                 0.0\\
      IRDC 1923+13 2&     0.0081 (0.0028)&     0.0083 (0.0023)&          0 (0.009)&         0 (0.0028)&                   1&             2747.75&             2747.75&                 0.0\\
      IRDC 1916+11 0&       0.033 (0.011)&      0.036 (0.0062)&          0 (0.014)&          0 (0.042)&                   1&             2747.75&             2747.75&                 0.0\\
      IRDC 1916+11 1&       0.082 (0.025)&      0.089 (0.0095)&          0 (0.014)&          0 (0.042)&                   1&             2747.75&             2747.75&                 0.0\\
      IRDC 1916+11 2&      0.017 (0.0064)&      0.018 (0.0046)&          0 (0.014)&          0 (0.042)&                   1&             2747.75&             2747.75&                 0.0\\
}{\tablenotetext{a}{The beam area is 2747.75\arcsec$^2$, which is used when the CMB is the only background continuum illumination}}

\Table{lccccccc}{Derived physical properties from \formaldehyde\ }
{\colhead{Source Name}&\colhead{N(\formaldehyde)\tablenotemark{a}}&\colhead{N(\formaldehyde) (FFC)\tablenotemark{b}}&\colhead{n(\hh) \tablenotemark{a} }&\colhead{n(\hh) (FFC)\tablenotemark{b}}&\colhead{X$_{\formaldehyde}$\tablenotemark{a}}&\colhead{X$_{\formaldehyde}$ (FFC)\tablenotemark{b}}&\colhead{Flag\tablenotemark{c}}\\
\colhead{           }&\colhead{(\persc)        }&\colhead{(\persc)              }&\colhead{(\percc)}&\colhead{(\percc)    }&\colhead{                    }&\colhead{                          }&\colhead{                      }\\ }
{tab:h2coderived}{
       G32.80+0.19 0&$12.79^{+0.11}_{-0.16}$&$\mathbf{12.94^{+0.16}_{-0.24}}$&$5.10^{+0.25}_{-0.26}$&$\mathbf{5.21^{+0.27}_{-0.29}}$&$-10.79^{+0.15}_{-0.20}$&$\mathbf{-10.75^{+0.15}_{-0.18}}$&                   2\\
       G32.80+0.19 1&$12.05^{+0.12}_{-0.11}$&$\mathbf{12.14^{+0.13}_{-0.13}}$&$4.96^{+0.22}_{-0.28}$&$\mathbf{5.05^{+0.21}_{-0.28}}$&$-11.39^{+0.20}_{-0.23}$&$\mathbf{-11.39^{+0.17}_{-0.20}}$&                   2\\
       G32.80+0.19 2&$11.66^{+0.10}_{-0.10}$&$\mathbf{11.71^{+0.10}_{-0.10}}$&$4.16^{+0.39}_{-0.38}$&$\mathbf{4.33^{+0.31}_{-0.32}}$&$-10.97^{+0.44}_{-0.46}$&$\mathbf{-11.10^{+0.37}_{-0.37}}$&                   2\\
       G32.80+0.19 3&$12.18^{+0.10}_{-0.09}$&$\mathbf{12.23^{+0.09}_{-0.09}}$&$4.07^{+0.38}_{-0.39}$&$\mathbf{4.23^{+0.32}_{-0.32}}$&$-10.37^{+0.44}_{-0.45}$&$\mathbf{-10.48^{+0.36}_{-0.38}}$&                   2\\
       G32.80+0.19 4&$11.82^{+0.10}_{-0.09}$&$\mathbf{11.87^{+0.10}_{-0.09}}$&$4.30^{+0.31}_{-0.32}$&$\mathbf{4.44^{+0.26}_{-0.29}}$&$-10.97^{+0.37}_{-0.37}$&$\mathbf{-11.05^{+0.31}_{-0.32}}$&                   2\\
       G33.13-0.09 0&$>12.80                 $&$\mathbf{>13.56}       $&$>4.54                 $&$\mathbf{>5.10}       $&$>-10.62                 $&$\mathbf{>-11.70}       $&                   8\\
       G33.13-0.09 1&$<11.96                 $&$\mathbf{<11.90}       $&$<4.50                 $&$\mathbf{<3.91}       $&$<-8.44                 $&$\mathbf{<-8.45}       $&                   6\\
       G33.13-0.09 2&$\mathbf{<12.20}       $&$<0.00                 $&$\mathbf{<4.29}       $&$<0.00                 $&$\mathbf{<-8.29}       $&$<0.00                 $&                   5\\
       G33.13-0.09 3&$\mathbf{<12.20}       $&$<0.00                 $&$\mathbf{<4.32}       $&$<0.00                 $&$\mathbf{<-8.29}       $&$<0.00                 $&                   5\\
       G33.92+0.11 0&$>12.35                 $&$\mathbf{>12.64}       $&$>4.86                 $&$\mathbf{>5.16}       $&$>-11.29                 $&$\mathbf{>-12.30}       $&                   8\\
       G33.92+0.11 1&$12.34^{+0.07}_{-0.08}$&$\mathbf{12.65^{+0.11}_{-0.17}}$&$4.97^{+0.22}_{-0.23}$&$\mathbf{5.26^{+0.22}_{-0.24}}$&$-11.11^{+0.19}_{-0.22}$&$\mathbf{-11.09^{+0.13}_{-0.16}}$&                   2\\
       G33.92+0.11 2&                   -&                   -&                   -&                   -&                   -&                   -&                   9\\
       G34.26+0.15 0&$13.01^{+0.10}_{-0.17}$&$\mathbf{13.13^{+0.15}_{-0.23}}$&$4.91^{+0.28}_{-0.29}$&$\mathbf{5.01^{+0.31}_{-0.32}}$&$-10.38^{+0.18}_{-0.23}$&$\mathbf{-10.36^{+0.17}_{-0.23}}$&                   2\\
       G34.26+0.15 1&$11.79^{+0.09}_{-0.08}$&$\mathbf{11.83^{+0.09}_{-0.08}}$&$4.67^{+0.23}_{-0.25}$&$\mathbf{4.75^{+0.21}_{-0.24}}$&$-11.36^{+0.26}_{-0.27}$&$\mathbf{-11.40^{+0.23}_{-0.25}}$&                   2\\
       G34.26+0.15 2&$11.53^{+0.10}_{-0.10}$&$\mathbf{11.56^{+0.10}_{-0.10}}$&$4.38^{+0.30}_{-0.33}$&$\mathbf{4.48^{+0.28}_{-0.30}}$&$-11.33^{+0.36}_{-0.37}$&$\mathbf{-11.40^{+0.32}_{-0.34}}$&                   2\\
       G34.26+0.15 3&$11.63^{+0.11}_{-0.10}$&$\mathbf{11.66^{+0.10}_{-0.10}}$&$4.43^{+0.29}_{-0.32}$&$\mathbf{4.53^{+0.26}_{-0.30}}$&$-11.28^{+0.34}_{-0.35}$&$\mathbf{-11.34^{+0.31}_{-0.32}}$&                   2\\
       G34.26+0.15 4&$11.40^{+0.17}_{-0.14}$&$\mathbf{11.45^{+0.17}_{-0.16}}$&$4.87^{+0.31}_{-0.43}$&$\mathbf{4.94^{+0.30}_{-0.42}}$&$-11.95^{+0.30}_{-0.35}$&$\mathbf{-11.98^{+0.27}_{-0.34}}$&                   2\\
       G35.20-1.74 0&$12.60^{+0.08}_{-0.07}$&$\mathbf{12.65^{+0.08}_{-0.07}}$&$4.72^{+0.25}_{-0.25}$&$\mathbf{4.79^{+0.25}_{-0.26}}$&$-10.61^{+0.23}_{-0.27}$&$\mathbf{-10.62^{+0.22}_{-0.26}}$&                   2\\
       G35.20-1.74 1&$11.69^{+0.11}_{-0.10}$&$\mathbf{11.72^{+0.11}_{-0.10}}$&$4.30^{+0.33}_{-0.37}$&$\mathbf{4.41^{+0.30}_{-0.33}}$&$-11.09^{+0.39}_{-0.39}$&$\mathbf{-11.16^{+0.34}_{-0.36}}$&                   2\\
       G35.20-1.74 2&$11.97^{+0.09}_{-0.09}$&$\mathbf{12.00^{+0.09}_{-0.09}}$&$4.20^{+0.34}_{-0.35}$&$\mathbf{4.31^{+0.30}_{-0.30}}$&$-10.70^{+0.39}_{-0.41}$&$\mathbf{-10.79^{+0.34}_{-0.36}}$&                   2\\
       G35.20-1.74 3&$<11.41                 $&$\mathbf{<11.44}       $&$<4.89                 $&$\mathbf{<5.02}       $&$<-9.24                 $&$\mathbf{<-9.30}       $&                   6\\
       G35.57-0.03 0&$>12.42                 $&$\mathbf{>13.38}       $&$>4.82                 $&$\mathbf{>5.61}       $&$>-11.20                 $&$\mathbf{>-12.02}       $&                   8\\
       G35.57-0.03 1&$>11.72                 $&$\mathbf{>12.25}       $&$>4.51                 $&$\mathbf{>5.13}       $&$>-11.80                 $&$\mathbf{>-12.72}       $&                   8\\
       G35.57-0.03 2&$<11.98                 $&$\mathbf{<12.12}       $&$<4.96                 $&$\mathbf{<5.12}       $&$<-8.71                 $&$\mathbf{<-8.72}       $&                   6\\
       G35.57-0.03 3&$<12.09                 $&$\mathbf{<12.23}       $&$<4.60                 $&$\mathbf{<4.78}       $&$<-8.38                 $&$\mathbf{<-8.37}       $&                   6\\
       G35.57-0.03 4&$>11.93                 $&$\mathbf{>12.47}       $&$>4.58                 $&$\mathbf{>5.20}       $&$>-11.55                 $&$\mathbf{>-12.48}       $&                   8\\
       G35.58+0.07 0&$>12.58                 $&$\mathbf{>14.06}       $&$>4.50                 $&$\mathbf{>5.48}       $&$>-10.79                 $&$\mathbf{>-11.71}       $&                   8\\
       G35.58+0.07 1&$<12.19                 $&$\mathbf{<12.32}       $&$<4.08                 $&$\mathbf{<4.55}       $&$<-8.07                 $&$\mathbf{<-8.14}       $&                   6\\
       G35.58+0.07 2&$<11.96                 $&$\mathbf{<12.10}       $&$<4.57                 $&$\mathbf{<4.95}       $&$<-8.53                 $&$\mathbf{<-8.63}       $&                   6\\
       G35.58+0.07 3&$<12.06                 $&$\mathbf{<12.17}       $&$<4.35                 $&$\mathbf{<4.38}       $&$<-8.32                 $&$\mathbf{<-8.24}       $&                   6\\
       G37.87-0.40 0&$12.44^{+0.07}_{-0.07}$&$\mathbf{12.53^{+0.07}_{-0.09}}$&$4.86^{+0.22}_{-0.23}$&$\mathbf{4.98^{+0.21}_{-0.24}}$&$-10.90^{+0.20}_{-0.24}$&$\mathbf{-10.92^{+0.18}_{-0.21}}$&                   2\\
       G37.87-0.40 1&$11.87^{+0.10}_{-0.09}$&$\mathbf{11.95^{+0.10}_{-0.09}}$&$4.89^{+0.22}_{-0.26}$&$\mathbf{5.00^{+0.21}_{-0.24}}$&$-11.50^{+0.21}_{-0.24}$&$\mathbf{-11.53^{+0.18}_{-0.22}}$&                   2\\
       G37.87-0.40 2&$12.10^{+0.18}_{-0.26}$&$\mathbf{12.09^{+0.14}_{-0.28}}$&$3.16^{+1.15}_{-1.20}$&$\mathbf{3.79^{+1.78}_{-0.71}}$&$-9.54^{+1.32}_{-1.41}$&$\mathbf{-10.18^{+0.77}_{-2.06}}$&                   4\\
       G37.87-0.40 3&$12.05^{+0.10}_{-0.10}$&$\mathbf{12.10^{+0.10}_{-0.09}}$&$4.33^{+0.32}_{-0.33}$&$\mathbf{4.49^{+0.28}_{-0.29}}$&$-10.76^{+0.37}_{-0.39}$&$\mathbf{-10.87^{+0.31}_{-0.34}}$&                   2\\
       G37.87-0.40 4&$12.17^{+0.17}_{-0.25}$&$\mathbf{12.14^{+0.10}_{-0.09}}$&$3.36^{+1.35}_{-0.98}$&$\mathbf{4.13^{+0.35}_{-0.36}}$&$-9.67^{+1.12}_{-1.60}$&$\mathbf{-10.46^{+0.41}_{-0.42}}$&                   2\\
       G37.87-0.40 5&$11.85^{+0.18}_{-0.21}$&$\mathbf{11.88^{+0.18}_{-0.25}}$&$3.05^{+1.04}_{-1.17}$&$\mathbf{3.34^{+1.33}_{-1.03}}$&$-9.68^{+1.32}_{-1.25}$&$\mathbf{-9.94^{+1.17}_{-1.58}}$&                   4\\
       G37.87-0.40 6&$11.67^{+0.19}_{-0.23}$&$\mathbf{11.70^{+0.19}_{-0.26}}$&$3.08^{+1.07}_{-1.21}$&$\mathbf{3.33^{+1.32}_{-1.13}}$&$-9.89^{+1.36}_{-1.30}$&$\mathbf{-10.11^{+1.26}_{-1.58}}$&                   4\\
       G37.87-0.40 7&$11.81^{+0.10}_{-0.09}$&$\mathbf{11.86^{+0.10}_{-0.09}}$&$4.28^{+0.31}_{-0.33}$&$\mathbf{4.45^{+0.27}_{-0.29}}$&$-10.95^{+0.37}_{-0.38}$&$\mathbf{-11.07^{+0.32}_{-0.32}}$&                   2\\
       G37.87-0.40 8&$11.56^{+0.11}_{-0.11}$&$\mathbf{11.63^{+0.11}_{-0.10}}$&$4.67^{+0.30}_{-0.33}$&$\mathbf{4.80^{+0.29}_{-0.31}}$&$-11.59^{+0.34}_{-0.38}$&$\mathbf{-11.65^{+0.30}_{-0.34}}$&                   2\\
       G37.87-0.40 9&$11.72^{+0.13}_{-0.29}$&$\mathbf{11.74^{+0.10}_{-0.09}}$&$3.98^{+1.97}_{-0.52}$&$\mathbf{4.32^{+0.31}_{-0.32}}$&$-10.74^{+0.60}_{-2.26}$&$\mathbf{-11.06^{+0.36}_{-0.37}}$&                   2\\
       G43.89-0.78 0&$12.49^{+0.10}_{-0.09}$&$\mathbf{12.76^{+0.08}_{-0.07}}$&$4.18^{+0.34}_{-0.33}$&$\mathbf{4.68^{+0.28}_{-0.28}}$&$-10.17^{+0.37}_{-0.40}$&$\mathbf{-10.40^{+0.24}_{-0.30}}$&                   2\\
       G43.89-0.78 1&$\mathbf{11.87^{+0.09}_{-0.08}}$&$12.80^{+0.61}_{-1.00}$&$\mathbf{4.95^{+0.20}_{-0.23}}$&$6.16^{+0.96}_{-1.84}$&$\mathbf{-11.56^{+0.19}_{-0.22}}$&$-11.84^{+0.88}_{-0.42}$&                   1\\
       G45.07+0.13 0&$12.38^{+0.08}_{-0.08}$&$\mathbf{12.75^{+0.13}_{-0.20}}$&$4.96^{+0.22}_{-0.24}$&$\mathbf{5.25^{+0.25}_{-0.27}}$&$-11.06^{+0.19}_{-0.22}$&$\mathbf{-10.97^{+0.15}_{-0.18}}$&                   2\\
       G45.07+0.13 1&                   -&                   -&                   -&                   -&                   -&                   -&                   9\\
       G45.12+0.13 0&$12.02^{+0.08}_{-0.07}$&$\mathbf{12.07^{+0.08}_{-0.07}}$&$4.83^{+0.21}_{-0.21}$&$\mathbf{4.92^{+0.19}_{-0.21}}$&$-11.30^{+0.20}_{-0.23}$&$\mathbf{-11.32^{+0.19}_{-0.21}}$&                   2\\
       G45.12+0.13 1&$11.90^{+0.09}_{-0.08}$&$\mathbf{11.95^{+0.08}_{-0.08}}$&$4.74^{+0.22}_{-0.23}$&$\mathbf{4.83^{+0.21}_{-0.23}}$&$-11.32^{+0.23}_{-0.25}$&$\mathbf{-11.36^{+0.21}_{-0.23}}$&                   2\\
       G45.12+0.13 2&$11.90^{+0.10}_{-0.08}$&$\mathbf{11.93^{+0.09}_{-0.09}}$&$4.41^{+0.26}_{-0.28}$&$\mathbf{4.52^{+0.24}_{-0.26}}$&$-11.00^{+0.32}_{-0.32}$&$\mathbf{-11.06^{+0.29}_{-0.30}}$&                   2\\
       G45.12+0.13 3&$11.48^{+0.08}_{-0.09}$&$\mathbf{11.55^{+0.08}_{-0.12}}$&$5.15^{+0.19}_{-0.21}$&$\mathbf{5.23^{+0.19}_{-0.22}}$&$-12.15^{+0.16}_{-0.18}$&$\mathbf{-12.16^{+0.14}_{-0.17}}$&                   2\\
       G45.45+0.06 0&$12.62^{+0.08}_{-0.08}$&$\mathbf{12.64^{+0.07}_{-0.08}}$&$4.33^{+0.29}_{-0.31}$&$\mathbf{4.37^{+0.28}_{-0.28}}$&$-10.19^{+0.33}_{-0.35}$&$\mathbf{-10.21^{+0.30}_{-0.33}}$&                   2\\
       G45.45+0.06 1&$11.89^{+0.08}_{-0.08}$&$\mathbf{11.92^{+0.07}_{-0.07}}$&$5.00^{+0.26}_{-0.28}$&$\mathbf{5.04^{+0.26}_{-0.27}}$&$-11.59^{+0.25}_{-0.29}$&$\mathbf{-11.60^{+0.23}_{-0.28}}$&                   2\\
       G45.45+0.06 2&$<11.55                 $&$\mathbf{<11.66}       $&$<4.77                 $&$\mathbf{<5.39}       $&$<-9.04                 $&$\mathbf{<-9.43}       $&                   6\\
       G45.47+0.05 0&$12.71^{+0.09}_{-0.07}$&$\mathbf{13.48^{+0.32}_{-0.50}}$&$4.46^{+0.28}_{-0.28}$&$\mathbf{5.21^{+0.40}_{-0.34}}$&$-10.23^{+0.28}_{-0.31}$&$\mathbf{-10.21^{+0.21}_{-0.19}}$&                   2\\
       G45.47+0.05 1&$\mathbf{<11.91}       $&$<11.92                 $&$\mathbf{<5.34}       $&$<5.23                 $&$\mathbf{<-8.65}       $&$<-8.57                 $&                   5\\
       G48.61+0.02 0&$12.06^{+0.09}_{-0.09}$&$\mathbf{12.53^{+0.13}_{-0.17}}$&$4.68^{+0.23}_{-0.25}$&$\mathbf{5.29^{+0.22}_{-0.24}}$&$-11.10^{+0.25}_{-0.26}$&$\mathbf{-11.25^{+0.13}_{-0.16}}$&                   2\\
       G48.61+0.02 1&$\mathbf{<11.88}       $&$<11.91                 $&$\mathbf{<5.09}       $&$<4.47                 $&$\mathbf{<-8.61}       $&$<-8.58                 $&                   5\\
       G48.61+0.02 2&$11.54^{+0.14}_{-0.13}$&$\mathbf{11.94^{+0.18}_{-0.23}}$&$4.60^{+0.28}_{-0.39}$&$\mathbf{5.22^{+0.23}_{-0.36}}$&$-11.54^{+0.33}_{-0.33}$&$\mathbf{-11.76^{+0.14}_{-0.19}}$&                   2\\
       G50.32+0.68 0&$>11.71                 $&$\mathbf{>12.41}       $&$>4.61                 $&$\mathbf{>5.31}       $&$>-11.77                 $&$\mathbf{>-12.63}       $&                   8\\
       G60.88-0.13 0&$12.24^{+0.18}_{-0.25}$&$\mathbf{12.35^{+0.09}_{-0.09}}$&$3.20^{+1.19}_{-1.16}$&$\mathbf{4.51^{+0.27}_{-0.28}}$&$-9.44^{+1.29}_{-1.43}$&$\mathbf{-10.64^{+0.28}_{-0.31}}$&                   2\\
       G61.48+0.09 0&$>12.51                 $&$\mathbf{>12.62}       $&$>5.07                 $&$\mathbf{>5.19}       $&$>-11.27                 $&$\mathbf{>-12.33}       $&                   8\\
       G69.54-0.98 0&                   -&                   -&                   -&                   -&                   -&                   -&                  11\\
       G70.29+1.60 0&$12.21^{+0.09}_{-0.08}$&$\mathbf{12.25^{+0.08}_{-0.08}}$&$4.67^{+0.23}_{-0.23}$&$\mathbf{4.74^{+0.23}_{-0.24}}$&$-10.94^{+0.24}_{-0.26}$&$\mathbf{-10.97^{+0.23}_{-0.26}}$&                   2\\
       G70.29+1.60 1&$<11.53                 $&$\mathbf{<11.55}       $&$<4.50                 $&$\mathbf{<4.67}       $&$<-8.92                 $&$\mathbf{<-8.98}       $&                   6\\
       G70.33+1.59 0&$13.16^{+0.09}_{-0.14}$&$\mathbf{13.41^{+0.19}_{-0.35}}$&$4.64^{+0.34}_{-0.32}$&$\mathbf{4.83^{+0.39}_{-0.37}}$&$-9.96^{+0.22}_{-0.31}$&$\mathbf{-9.90^{+0.21}_{-0.26}}$&                   2\\
   IRAS 20051+3435 0&$\mathbf{12.20^{+0.11}_{-0.10}}$&$12.23^{+0.04}_{-0.05}$&$\mathbf{4.12^{+0.39}_{-0.41}}$&$4.11^{+0.21}_{-0.23}$&$\mathbf{-10.40^{+0.45}_{-0.46}}$&$-10.35^{+0.22}_{-0.22}$&                   3\\
       G41.74+0.10 0&$12.25^{+0.17}_{-0.23}$&$\mathbf{12.48^{+0.10}_{-0.09}}$&$2.99^{+0.99}_{-1.18}$&$\mathbf{4.50^{+0.28}_{-0.31}}$&$-9.23^{+1.31}_{-1.22}$&$\mathbf{-10.50^{+0.29}_{-0.32}}$&                   2\\
       G41.74+0.10 1&$\mathbf{<12.12}       $&$<0.00                 $&$\mathbf{<4.72}       $&$<0.00                 $&$\mathbf{<-8.37}       $&$<0.00                 $&                   5\\
       G41.74+0.10 2&$\mathbf{<12.18}       $&$<0.00                 $&$\mathbf{<3.21}       $&$<0.00                 $&$\mathbf{<-8.91}       $&$<0.00                 $&                   5\\
       G41.74+0.10 3&$\mathbf{<12.17}       $&$<0.00                 $&$\mathbf{<4.70}       $&$<0.00                 $&$\mathbf{<-8.32}       $&$<0.00                 $&                   5\\
       G41.74+0.10 4&$\mathbf{<12.27}       $&$<0.00                 $&$\mathbf{<4.11}       $&$<0.00                 $&$\mathbf{<-8.22}       $&$<0.00                 $&                   5\\
      IRDC 1923+13 0&$\mathbf{<11.86}       $&$<11.84                 $&$\mathbf{<5.19}       $&$<4.47                 $&$\mathbf{<-8.63}       $&$<-8.65                 $&                   5\\
      IRDC 1923+13 1&$\mathbf{<11.86}       $&$<11.77                 $&$\mathbf{<5.30}       $&$<4.58                 $&$\mathbf{<-8.70}       $&$<-8.72                 $&                   5\\
      IRDC 1923+13 2&$\mathbf{<13.29}       $&$<11.53                 $&$\mathbf{<8.00}       $&$<5.00                 $&$\mathbf{<-8.95}       $&$<-8.96                 $&                   5\\
      IRDC 1916+11 0&$\mathbf{<12.05}       $&$<12.57                 $&$\mathbf{<5.16}       $&$<5.62                 $&$\mathbf{<-8.44}       $&$<-8.46                 $&                   5\\
      IRDC 1916+11 1&$\mathbf{<12.36}       $&$<12.40                 $&$\mathbf{<4.61}       $&$<5.03                 $&$\mathbf{<-8.13}       $&$<-8.16                 $&                   5\\
      IRDC 1916+11 2&$\mathbf{<12.09}       $&$<13.77                 $&$\mathbf{<5.55}       $&$<8.00                 $&$\mathbf{<-8.68}       $&$<-8.68                 $&                   5\\
}{
\tablenotetext{a}{The values used in this paper are shown in boldface.
Uncorrected values are listed in this column.  The filling-factor corrected
values are shown for comparison in the next column even though they were not used for analysis.}
\tablenotetext{b}{The values used in this paper are shown in boldface.
Filling-factor corrected values are listed in this column.  The uncorrected
values are shown for comparison in the previous column even though they not used for analysis.}
\tablenotetext{c}{Flags:\begin{enumerate}
  \item  No filling factor correction (no FFC) is the most reliable.                                             %1: 
  \item  Filling factor correction (FFC) is the most reliable                                                    %2: 
  \item  There is an ambiguity between low density / high abundance and low abundance / high density (no FFC)    %3: 
  \item  There is an ambiguity between low density / high abundance and low abundance / high density (FFC)       %4: 
  \item  Upper Limit (No FFC)                                                                                    %5: 
  \item  Upper Limit (FFC)                                                                                       %6: 
  \item  Lower Limit (No FFC)                                                                                    %7: 
  \item  Lower Limit (FFC)                                                                                       %8: 
  \item  Unreliable estimate because of continuum / filling factor uncertainty.                                  %9: 
  \item  No limit (S/N)                                                                                         %10:
  \item  Optically Thick                                                                                        %11:
\end{enumerate}
}}



\subsection{Green Bank Telescope}
We observed the \formaldehyde\ \twotwo\ line at 2 cm (14.488789 GHz) with the
Green Bank Telescope (GBT)\footnote{ The National Radio Astronomy Observatory operates
the GBT and VLA and is a facility of the National Science Foundation operated
under cooperative agreement by Associated Universities, Inc.  } dual-beam
Ku-band receiver as part of project GBT09C-049.  The GBT dual-beam Ku-band
receiver was used for 4 hours on January 18th, 2010 in beam-switched nodding
mode.  System temperatures ranged from 27 to 38 K in the \formaldehyde\
band centered on the  \twotwo\ line. A bandwidth of 12.5 MHz
(258.8 \kms) and channel width of 3.052 kHz (0.063 \kms) were used with 9-level
sampling, with receiver temperature $\approx21$K.   Three additional tunings
were acquired simultaneously, centered between the H and He 75$\alpha$,
76$\alpha$, and 77$\alpha$ radio recombination lines (RRLs) with the same channel widths and
bandwidths as above at 14.1315, 14.6930, and 15.2846 GHz.  Each source was
observed for 150 seconds in each receiver for a total on-source integration
time of 300 seconds.  Each observation in the pair was independently inspected
to search for emission/absorption in the off position, which was 5.5\arcmin\
away in azimuth.  When absorption was detected in one of the off positions,
that on/off pair was discarded if one of the detected lines was affected, but
otherwise was noted and ignored.  Pointing and focus observations on the
calibrator source 1822-0938 were taken at the start of and two hours into the
observations.  

The gain was assumed to be 1.91 K/Jy based on previous calibration observations
on point sources in Ku-band; our flux density measurements will therefore be
overestimates for extended sources.  The aperture efficiency was
$\eta_{A}=0.671$, and the main beam efficiency $\eta_{MB}=1.32\eta_{A}=0.886$,
so our main-beam corrected measurements could overestimate extended source flux
densities by at most $13\%$ (ignoring atmospheric absorption).
%(based on?  I did not observe flux calibrators.  Maybe from
%\url{https://safe.nrao.edu/wiki/bin/view/GB/Observing/GainPerformance} or 
%\url{https://safe.nrao.edu/wiki/bin/view/GB/PTCS/TSCAL\_090916\#gcurve}).
The data were calibrated using the normal {\sc getnod} procedure in GBTIDL 
\footnote{ GBTIDL (http://gbtidl.nrao.edu/) is the data reduction package
produced by NRAO and written in the IDL language for the reduction of GBT data.
The National Radio Astronomy Observatory is a facility of the National Science
Foundation operated under cooperative agreement by Associated Universities,
Inc.
} ,
which assumes an atmospheric opacity at zenith $\tau=0.014$ at 14.488 GHz. 
%The
%measured opacity ranged from 0.024-0.032 (0.017-0.023 zenith), which was not corrected,
%resulting in a 1-2\% systematic underestimate of source flux densities in the
%table.  All GBT spectra were converted to Jy in order to compare them directly
%with the Arecibo spectra, which were only available in Jy.  

We assume primary beam $\theta_{FWHM}=51.1\arcsec$ %=740/14.48$ 
per the GBT
observers manual.  We assume a conservative 10\% error in the beam area
$\Omega=7.8\times10^{-8}$ sr, which governs the flux density received from the
CMB over the observed area.  Beam size error should be dominated by small errors
in focus.  By utilizing the 305 m Arecibo telescope at 6 cm and the 100 m GBT
at 2 cm, we acquired beam-matched (FWHM$\sim 50\arcsec$) observations of the
\formaldehyde\ \oneone\ and \twotwo\ lines.  



%Question: Continuum accuracy?  Assumed to be 20\%, justified?

\subsubsection{GBT Data Reduction}
In the 24 lines of sight, 75 independent components were identified from the
\oneone\ spectra.  These were fit with gaussians using GBTIDL's {\sc fitgauss}
routine.  Out of these 75 components, 51 had corresponding \twotwo\ detections.
The fitted gaussian spectral lines are listed by line of sight in Table
\ref{tab:h2comeasured_a}.  The gaussian fits may not be representative of the
true spectral line profile; complex spectral line profiles are discussed in Section
\ref{sec:lineprofiles}.
%Some of these (e.g.  G69.54-0.98) are blended or optically thick, so the
%Gaussian fit may not be representative of the true line profile.

The 2 cm continua were measured by fitting a first-order baseline in each
reduced nodded pair excluding the line and the bandpass edges.  Figure
\ref{fig:specexample} shows the flat baselines achieved in the observations,
though the RRL spectrum shows an example of the artifacts seen at the edges of
the bandpass.  The continuum error listed in the table is the RMS of only the
data included in the baseline fit after the baseline was subtracted from the
spectrum; the systematic error from flux calibration uncertainty is 20\% and
dominant.  

\subsection{Arecibo} 
The Arecibo 4.829660 GHz \formaldehyde\ \oneone\ observations used in this
project were previously presented in \citet{Araya2002} and \citet{Araya2004}
and were kindly provided in reduced form by E. Araya.  All observations were performed
using standard on/off position switching and 5 minute integration times in both the 
on and off positions, resulting in off positions 1.25 degrees away from the pointing center.
We assume a 30\% error
in the continuum \citep[based on measured gains in the range 2.0-2.5
as reported in][]{Araya2002} and an effective diameter of 227m ($\theta_{FWHM}
= 56\arcsec$, $\Omega=9.0\times10^{-8}$ sr ) with
10\% uncertainty\footnote{\url{http://www.naic.edu/$\sim$phil/sysperf/misc/hpbw\_vs\_lambda\_2004.html}}.

The Arecibo spectral lines were re-fit for this paper by converting the Arecibo data
from CLASS\footnote{CLASS is part of the GILDAS software developed by IRAM.} to
GBTIDL's {\sc SDFITS} format\footnote{Code for the CLASS-GBTIDL conversion is available from
\url{http://code.google.com/p/casaradio/wiki/class\_to\_gbt}}
and using GBTIDL's {\sc fitgauss} routine.  The 6 cm continua were taken
directly from \citet{Araya2002} Table 3.

\OneColFigure{figures_chH2CO/G32.80+0.19_both}
{ {\it Top:} The GBT \twotwo\ (red) and Arecibo \oneone\ (black) spectra of G32.80+0.19.  
% don't reference figure too early The
% don't reference figure too early fitted density and column to the deepest component at $\sim15 \kms$ is shown in
% don't reference figure too early Figure \ref{fig:fitexample}.  
{\it Bottom:} The GBT H75$\alpha$ (red) and Arecibo H110$\alpha$ (black) spectra with the GRS
\thirteenco\ spectrum (light blue) overlaid.   The left axis is for the RRLs and the right
axis is for the \thirteenco.  The C and He RRLs are not displayed.}
{fig:specexample}
{0.30}{0}

\subsection{Other Archival Data}

\subsubsection{Very Large Array}
We acquired VLA archival images from the Multi-Array Galactic Plane Imaging
Survey (MAGPIS) 6 cm Epoch 3 data set \citep{Becker2005} and the NRAO VLA
Archive Survey (NVAS)\footnote{The NVAS is run by Lourant Sjouwerman at the
NRAO.  It has not yet been published.}.  MAGPIS has a resolution of
$\sim4\arcsec$ and sensitivity $\sigma\sim 2.5 $mJy/bm.  The NVAS has variable
resolution and sensitivity since it is based on VLA archival data.  The VLA
data was used to estimate source sizes and interferometer-to-single-dish flux
ratios.
%[XXXX Need estimates of beam sizes and errors for NVAS data.  Perhaps may be
%better to include one example demonstrating that it's not necessary to include
%a source size correction than to actually include the corrections.]

\subsubsection{ Bolocam 1.1 mm }
We extract 1.1 mm dust continuum fluxes from the Bolocam Galactic Plane Survey
(BGPS) v1.0 release summing over a 25\arcsec\ radius aperture after subtracting
the median in a 50-200\arcsec\ annulus to remove background
contributions. The aperture size is selected to match the 1.1 mm data to the 
2 and 6 cm data.  We assume a uniform 50\% systematic error in BGPS fluxes from
combined uncertainties in the calibration and background subtraction.
\citet{Aguirre2011} contains a complete discussion of the uncertainties in the
BGPS. 
%the measurement uncertainties are only $\sim50\%$, while other
%systematics such as the temperature and opacity can result in factor of 2-4
%changes in the estimated mass and column density.  
We apply the
\citet{Aguirre2011} recommended flux correction of 1.5 and aperture correction
for a 25\arcsec\ aperture of 1.21.  Additionally, data from the Bolocam catalog
\citep{Rosolowsky2010} was used with the flux correction and an aperture correction
of 1.46 for 20\arcsec\ apertures.

\subsubsection{Boston University / Five College Radio Observatory Galactic Ring Survey}
The BU FCRAO GRS \citep{Jackson2006} is a survey of the Galactic plane in the
\thirteenco\ 1-0 line with $\sim 46\arcsec$ resolution.  We extracted spectra in 25\arcsec\ radius
apertures from the publicly available data for comparison with the \formaldehyde\ spectra.

\subsubsection{GLIMPSE}
The Galactic Legacy Infrared Mid-Plane Survey Extraordinaire
\citep[GLIMPSE]{Benjamin2003} maps were used to examine the morphology of the
objects in our survey in order to determine whether an IRDC was present.

\section{Models and Error Estimation}
\label{sec:models}
A grid of large velocity gradient (LVG) models was run using both the RADEX
\citep{VanDerTak2007} code and a proprietary code by \citet{henkel1980} with a 
gradient of 1 \kmspc.  The
models from the two codes were consistent to within $\sim10\%$ in predicted
optical depth and $T_{ex}$.  Both utilized collision
rates from \citet{Green1991} extracted from the LAMDA
database\footnote{\url{http://www.strw.leidenuniv.nl/$\sim$moldata/}} and multiplied by
the recommended factor of 1.6 to account for collisions with \hh\ being more
efficient than He.  The expected accuracy is $\sim30\%$.
\citet{Zeiger2010} demonstrated that the errors in collision rates lead to
systematic errors $\lesssim50\%$  (0.3 dex) in the measured quantities
(N(\formaldehyde), n$(\hh)$).  When measuring density and column, we used the
RADEX models because of their extensively tested code and documentation.  All of the
models used a kinetic temperature of 40 K and covered a range of 500 densities
$\times$ 500 columns logarithmically sampled over $10^1 < n(\hh) < 10^7$ \percc\
and $10^{11} < N(\ortho) < 10^{16} $ \persc.  The assumption $T_K=40$ K is
reasonable in UC\ion{H}{2} regions, which should be warmer than IRDCs and other
cold molecular clouds.  Dust temperatures measured towards UC\ion{H}{2}
regions are around 40 K \citep{Rivera2010}.  In the foreground clouds, this
assumption is less well supported, but as long as the temperatures are higher
than $\sim20$ K, the models change little with temperature (Figure
\ref{fig:modeltau}).

Because of a collisional selection effect, above its critical density
\citep[$n_{cr}(\formaldehyde\ \oneone)\approx 8\ \percc$, \ensuremath{n_{cr}(\formaldehyde\ 
\twotwo) \approx 76\ \percc}, ][]{Mangum2008}
\formaldehyde\ preferentially overpopulates lower states of the K-doublet
\citep[$\Delta$J =0, $\Delta K_a=0$, $\Delta K_c = \pm1$,][]{henkel1980}.  These
spectral lines are cooled to excitation temperatures lower than the CMB and can
therefore be seen in absorption against it.  The \oneone/\twotwo\
absorption line ratio is sensitive to the density of \hh\ at densities
$\gtrsim10^{3.5}\ $ \percc, allowing measurements of the density to within
$\sim$0.3 dex with little sensitivity to gas kinetic temperature
\citep{Mangum2008}.  When density is `measured' with critical density based
tracers such as CO, CS, HCN, or HCO$^+$, the estimate can be off by as much as
2 orders of magnitude because of radiative trapping effects.  Similarly,
measurements of density assuming spherical symmetry can be very far from the
local values.

The collision rates of \formaldehyde\ with \hh\ have been re-derived with a
claimed accuracy of 10\% by \citet{Troscompt2009}.  \citet{Troscompt2009b}
showed that collisions with para-\hh\ are more efficient at cooling
\formaldehyde\ into absorption against the CMB than He or ortho-\hh\ and that
\formaldehyde\ absorption is therefore sensitive to the Ortho/Para ratio of
\hh.   These improved rates are not used in this paper since they are only
computed over a more limited range of temperatures, but may be used in future
works.
% Note to self: point out later that 30% uncertainty has little  to no effect
% on the results of this paper because the interesting results are comparisons
% between different velocities within a single model
% Additional note: Jeff Mangum says the Troscompt work is kinda crap...

% \OneColFigure{Tline0_old_contours}
% {A plot showing the parameter space in which \formaldehyde\ will be seen in 
% absorption or emission.  The contours show where the modeled line temperature
% $T_{line}=0$ assuming illumination only by the CMB or equivalently where the
% line excitation temperature $T_{ex}=T_{CMB}$.  Values of temperature and
% density higher than the displayed curves will be seen in emission (assuming no
% background illumination besides the CMB), while values below the displayed
% curves will be seen in absorption.  The \oneone\ line is shown in solid lines
% from gas kinetic temperature $T_{K}=5$K (thickest) to 55K (thinnest).  The
% \twotwo\ line is shown by dashed (red) lines over the same range; it shows a
% much weaker variation with $T_{K}$.  The \oneone\ line only shows a strong
% dependence on $T_{K}$ below $T_{K} = 10K$, which is approximately the lowest
% temperature expected in molecular clumps.  
% }
% {fig:tline0}
% {0.30}{0}

\OneColFigure{figures_chH2CO/lineratio_vs_n_X9_legend}
{The predicted optical depth ratio ({\it top}) and optical depth ({\it bottom}) vs. volume density
assuming a fixed abundance  $X_{\ortho}=10^{-9}$ \perkmspc\ shows that the
dependence of the derived density on temperature is weak.  At lower abundances,
these curves shift to the right, providing sensitivity to moderately higher
densities.  Our 5-$\sigma$ detection limit is generally around $\tau\sim0.01$.
}
{fig:modeltau}
{0.30}{0}

\subsection{Turbulence}
\label{sec:turbulence}
Molecular gas is often observed to have spectral line widths consistent with
supersonic turbulence \citep{Kainulainen2009} and therefore a lognormal density
distribution \citep{Kritsuk2007}.  Our LVG models assume constant density per
velocity bin, so the resulting models should be smoothed by the probability
distribution function (PDF) of the density.  For clouds with a narrow density
distribution (logarithmic standard deviation of the density $\sigma_s \equiv
\sigma_{ln(\rho)/ln(\bar{\rho})}\lesssim0.5$)\footnote{We use $\rho$ to
indicate number density in this section in order to be consistent with the
cited literature.  Because the widths are relative to a mean density, the
scaling between mass and number density is unimportant.}, the effect of
smoothing is smaller than other systematic errors, but for more turbulent
clouds the density PDF width can exceed an order of magnitude \citep[e.g.,
][]{Federrath2010} and will substantially change the derived density.  Because
the Mach numbers of the turbulence in the observed clouds are unconstrained, we
cannot correct for this added uncertainty.  The change in measured density is
$|\Delta\log(\rho)|<0.25$ for $\sigma_s \le 0.5$, with a slight bias towards
higher densities at lower optical depth ratio $\tau_{\oneone}/\tau_{\twotwo}$
(Figure \ref{fig:turbtaurat}).  However, for $\sigma_s=1.5$, the bias exceeds
an order of magnitude at some densities.

Additionally, we consider the effects of ``gravoturbulence'', in which a
high-density tail inconsistent with a lognormal distribution is observed.
\citet{Kainulainen2009} report column density distributions derived from 2MASS
extinction measurements that can be used as a proxy for the density
distribution for a wide variety of clouds.  Non-star-forming clouds retain a
lognormal distribution and are consistent with the analysis presented above.
However, evolved star-forming regions develop a high-column density tail.  For
evolved (actively star-forming) regions like Ophiucus, Orion, and Perseus,
the high-column density tail is substantial, and \formaldehyde\ density measurements
will be highly biased towards the highest density gas.  More quiescent regions
like the Pipe and Coalsack nebulae are consistent with a lognormal column
distribution to a degree that the high-column density tail would not affect
\formaldehyde\ density measurements significantly.

\OneColFigure{figures_chH2CO/lognormalsmooth_density_ratio_massweight} 
{
The optical depth ratio as a function of density for turbulent density
distributions with widths specified in the legend.  The optical depth ratio
varies more slowly with density than in the pure LVG model (the solid line is
the same as the black 10 K line in Figure \ref{fig:modeltau}a).}
{fig:turbtaurat}{0.30}{0}

To demonstrate the effects of turbulent distributions, we calculate the optical
depth ratio as a function of the mean density for three turbulent widths in
Figure \ref{fig:turbtaurat}.  We compare the density that would be inferred
from the spectral line ratio assuming no turbulence (just LVG) to the `correct'
density including turbulent effects in Figure \ref{fig:turbcorr}.  We have also
compared the LVG and turbulent densities to ``gravoturbulent'' density
distributions, in which a power law tail of high-density gas begins at about
$10^{-2}$ times the peak density \citep[e.g., ][]{Klessen2000,KimCho2011}, but
because the density distributions in these simulations are relatively narrow,
the effects of the high-density tail on the measured density are negligible
except for the most turbulent cases.

Figure \ref{fig:turbtaurat} is meant to demonstrate the effects of turbulence,
but it is \emph{not} used to derive densities, since the true density
distribution in observed clouds is unknown.  However, future measurements of
the density distribution can be used to apply the `correction' shown in Figure
\ref{fig:turbcorr}.

%In
%the gravoturbulent case, we represent the high-density tail as a power-law
%starting at $n=100n_0$ with a slope $\alpha=1.3$, which is drawn directly from the
%plots in \citet{KimCho2011}; the slope of the measured column distributions in \citet{Kainulainen2009}
%is similar, but starts nearly at $N=N_0$. The plotted correction should be a reasonable
%approximation of a GMC that has survived for a few free-fall times.

\OneColFigure{figures_chH2CO/TurbulenceCorrection}
{ The mean density from a lognormal density distribution plotted against the
density derived assuming a single density per region (i.e., the directly
LVG-derived density).   At low densities, the wider turbulent
distributions  are heavily biased towards ``observing'' higher densities than
the true mean density.  The distributions cut off at the low end where the
optical depth ratio becomes a double-valued function of density; at these low
densities, no detections are expected at our survey's sensitivity.  The cutoff
at the high end is where the optical depth ratio becomes constant.  
}
{fig:turbcorr}{0.30}{0}

\section{Analysis}
\label{sec:analysis}

%\Table{lcccccccc}{Measured \formaldehyde\ \oneone\ line properties}
{\colhead{Source Name\tablenotemark{a}}&\colhead{$l$}&\colhead{$b$}&\colhead{6cm Continuum}&\colhead{Peak}&\colhead{Center}&\colhead{FWHM }&\colhead{RMS}&\colhead{Channel Width}\\
\colhead{           }&\colhead{\degrees}&\colhead{\degrees}&\colhead{(Jy)}&\colhead{(Jy)}&\colhead{(\kms)}&\colhead{(\kms)}&\colhead{(Jy)}&\colhead{(\kms)}\\ }
{tab:h2comeasured_a}{
       G32.80+0.19 0&              0.1904&             32.7968&         2.18 (0.01)&      -0.393 (0.008)&        15.39 (0.05)&         6.57 (0.06)&              0.0049&              1.1374\\
       G32.80+0.19 1&              0.1904&             32.7968&         2.18 (0.01)&      -0.092 (0.008)&        11.45 (0.26)&        10.25 (0.65)&              0.0049&              1.1374\\
       G32.80+0.19 2&              0.1904&             32.7968&         2.18 (0.01)&      -0.063 (0.008)&        80.63 (0.13)&         2.49 (0.36)&              0.0049&              1.1374\\
       G32.80+0.19 3&              0.1904&             32.7968&         2.18 (0.01)&      -0.254 (0.008)&        84.61 (0.02)&         1.37 (0.06)&              0.0049&              1.1374\\
       G32.80+0.19 4&              0.1904&             32.7968&         2.18 (0.01)&      -0.090 (0.008)&        88.66 (0.09)&         3.21 (0.31)&              0.0049&              1.1374\\
       G33.13-0.09 0&             -0.0949&             33.1297&         0.49 (0.00)&      -0.192 (0.007)&        75.92 (0.05)&         3.80 (0.12)&              0.0045&              1.1374\\
       G33.13-0.09 1&             -0.0949&             33.1297&         0.49 (0.00)&      -0.023 (0.007)&        81.62 (0.35)&         2.49 (0.88)&              0.0045&              1.1374\\
       G33.13-0.09 2&             -0.0949&             33.1297&         0.49 (0.00)&      -0.040 (0.007)&       101.50 (0.40)&        11.30 (0.80)&              0.0045&              1.1374\\
       G33.13-0.09 3&             -0.0949&             33.1297&         0.49 (0.00)&      -0.039 (0.007)&        10.39 (0.08)&         2.04 (0.24)&              0.0045&              1.1374\\
       G33.92+0.11 0&              0.1112&              33.914&         0.83 (0.00)&      -0.081 (0.008)&       107.28 (0.18)&         6.62 (0.34)&               0.005&              1.1374\\
       G33.92+0.11 1&              0.1112&              33.914&         0.83 (0.00)&      -0.079 (0.008)&       106.03 (0.06)&         2.41 (0.23)&               0.005&              1.1374\\
       G33.92+0.11 2&              0.1112&              33.914&         0.83 (0.00)&      -0.160 (0.030)&        57.30 (0.40)&        10.60 (0.80)&               0.005&              1.1374\\
       G34.26+0.15 0&              0.1538&             34.2572&         5.57 (0.01)&      -1.828 (0.015)&        60.24 (0.01)&         3.80 (0.03)&              0.0063&              1.1374\\
       G34.26+0.15 1&              0.1538&             34.2572&         5.57 (0.01)&      -0.160 (0.015)&        26.69 (0.08)&         1.04 (0.22)&              0.0063&              1.1374\\
       G34.26+0.15 2&              0.1538&             34.2572&         5.57 (0.01)&      -0.099 (0.015)&        11.25 (0.19)&         2.01 (0.40)&              0.0063&              1.1374\\
       G34.26+0.15 3&              0.1538&             34.2572&         5.57 (0.01)&      -0.126 (0.015)&        51.70 (2.00)&         4.20 (1.00)&              0.0063&              1.1374\\
       G34.26+0.15 4&              0.1538&             34.2572&         5.57 (0.01)&      -0.047 (0.015)&        48.20 (2.00)&         1.80 (1.00)&              0.0063&              1.1374\\
       G35.20-1.74 0&             -1.7409&             35.1997&         5.17 (0.00)&      -1.018 (0.008)&        43.37 (0.01)&         3.67 (0.02)&              0.0051&              1.1374\\
       G35.20-1.74 1&             -1.7409&             35.1997&         5.17 (0.00)&      -0.147 (0.008)&        36.67 (0.10)&         1.49 (0.27)&              0.0051&              1.1374\\
       G35.20-1.74 2&             -1.7409&             35.1997&         5.17 (0.00)&      -0.324 (0.008)&        14.08 (0.01)&         0.93 (0.03)&              0.0051&              1.1374\\
       G35.20-1.74 3&             -1.7409&             35.1997&         5.17 (0.00)&      -0.039 (0.008)&        50.59 (0.53)&         4.92 (1.31)&              0.0051&              1.1374\\
       G35.57-0.03 0&             -0.0306&             35.5779&         0.47 (0.00)&      -0.064 (0.009)&        52.10 (0.10)&         4.60 (0.30)&              0.0053&              1.1374\\
       G35.57-0.03 1&             -0.0306&             35.5779&         0.47 (0.00)&      -0.021 (0.009)&        45.60 (0.30)&         1.90 (0.60)&              0.0053&              1.1374\\
       G35.57-0.03 2&             -0.0306&             35.5779&         0.47 (0.00)&      -0.019 (0.009)&        57.60 (0.50)&         2.90 (0.97)&              0.0053&              1.1374\\
       G35.57-0.03 3&             -0.0306&             35.5779&         0.47 (0.00)&      -0.031 (0.009)&        12.80 (0.20)&         1.84 (0.41)&              0.0053&              1.1374\\
       G35.57-0.03 4&             -0.0306&             35.5779&         0.47 (0.00)&      -0.031 (0.008)&        29.04 (0.11)&         0.82 (0.25)&              0.0053&              1.1374\\
       G35.58+0.07 0&              0.0657&             35.5801&         0.53 (0.01)&      -0.146 (0.004)&        49.37 (0.21)&         5.33 (0.34)&              0.0048&              1.1374\\
       G35.58+0.07 1&              0.0657&             35.5801&         0.53 (0.01)&      -0.049 (0.013)&        53.13 (0.25)&         2.98 (0.64)&              0.0048&              1.1374\\
       G35.58+0.07 2&              0.0657&             35.5801&         0.53 (0.01)&      -0.025 (0.004)&        58.12 (0.29)&         3.63 (0.74)&              0.0048&              1.1374\\
       G35.58+0.07 3&              0.0657&             35.5801&         0.53 (0.01)&      -0.034 (0.004)&        13.24 (0.17)&         2.80 (0.39)&              0.0048&              1.1374\\
       G37.87-0.40 0&             -0.3993&              37.873&         4.40 (0.01)&      -0.531 (0.006)&        60.23 (0.11)&         8.73 (0.35)&              0.0069&              1.1374\\
       G37.87-0.40 1&             -0.3993&              37.873&         4.40 (0.01)&      -0.124 (0.014)&        53.27 (0.19)&         4.03 (0.46)&              0.0069&              1.1374\\
       G37.87-0.40 2&             -0.3993&              37.873&         4.40 (0.01)&      -0.356 (0.019)&        65.13 (0.04)&         2.74 (0.15)&              0.0069&              1.1374\\
       G37.87-0.40 3&             -0.3993&              37.873&         4.40 (0.01)&      -0.324 (0.045)&        72.18 (0.04)&         1.35 (0.14)&              0.0069&              1.1374\\
       G37.87-0.40 4&             -0.3993&              37.873&         4.40 (0.01)&      -0.424 (0.013)&        73.97 (0.13)&         3.01 (0.22)&              0.0069&              1.1374\\
       G37.87-0.40 5&             -0.3993&              37.873&         4.40 (0.01)&      -0.185 (0.012)&        79.98 (0.06)&         1.80 (0.14)&              0.0069&              1.1374\\
       G37.87-0.40 6&             -0.3993&              37.873&         4.40 (0.01)&      -0.114 (0.015)&        91.96 (0.08)&         1.21 (0.18)&              0.0069&              1.1374\\
       G37.87-0.40 7&             -0.3993&              37.873&         4.40 (0.01)&      -0.175 (0.012)&        14.32 (0.14)&         2.94 (0.20)&              0.0069&              1.1374\\
       G37.87-0.40 8&             -0.3993&              37.873&         4.40 (0.01)&      -0.072 (0.022)&        13.16 (0.10)&         0.87 (0.32)&              0.0069&              1.1374\\
       G37.87-0.40 9&             -0.3993&              37.873&         4.40 (0.01)&      -0.137 (0.012)&        20.54 (0.06)&         1.37 (0.14)&              0.0069&              1.1374\\
       G43.89-0.78 0&             -0.7838&             43.8892&         0.66 (0.00)&      -0.181 (0.004)&        54.86 (0.02)&         2.19 (0.06)&              0.0032&              1.1374\\
       G43.89-0.78 1&             -0.7838&             43.8892&         0.66 (0.00)&      -0.020 (0.002)&        50.55 (0.59)&        15.90 (1.20)&              0.0032&              1.1374\\
       G45.07+0.13 0&              0.1323&             45.0711&         0.47 (0.00)&      -0.056 (0.006)&        57.49 (0.10)&         4.24 (0.23)&              0.0035&              1.1374\\
       G45.07+0.13 1&              0.1323&             45.0711&         0.47 (0.00)&      -0.036 (0.006)&        65.44 (0.15)&         4.09 (0.34)&              0.0035&              1.1374\\
       G45.12+0.13 0&              0.1326&             45.1223&         4.28 (0.01)&      -0.188 (0.006)&        55.70 (0.12)&         3.32 (0.24)&              0.0065&              1.1374\\
       G45.12+0.13 1&              0.1326&             45.1223&         4.28 (0.01)&      -0.154 (0.009)&        59.40 (0.13)&         3.11 (0.33)&              0.0065&              1.1374\\
       G45.12+0.13 2&              0.1326&             45.1223&         4.28 (0.01)&      -0.200 (0.010)&        24.86 (0.03)&         1.68 (0.08)&              0.0065&              1.1374\\
       G45.12+0.13 3&              0.1326&             45.1223&         4.28 (0.01)&      -0.027 (0.004)&        65.53 (0.82)&         7.23 (2.03)&              0.0065&              1.1374\\
       G45.45+0.06 0&              0.0593&             45.4548&         4.77 (0.01)&      -1.347 (0.018)&        59.58 (0.02)&         3.18 (0.05)&              0.0063&              1.1374\\
       G45.45+0.06 1&              0.0593&             45.4548&         4.77 (0.01)&      -0.123 (0.040)&        55.34 (0.38)&         3.15 (0.38)&              0.0063&              1.1374\\
       G45.45+0.06 2&              0.0593&             45.4548&         4.77 (0.01)&      -0.056 (0.005)&        25.02 (0.12)&         2.82 (0.28)&              0.0063&              1.1374\\
       G45.47+0.05 0&              0.0455&             45.4655&         0.75 (0.00)&      -0.274 (0.003)&        60.62 (0.03)&         6.59 (0.07)&              0.0039&              1.1374\\
       G45.47+0.05 1&              0.0455&             45.4655&         0.75 (0.00)&      -0.017 (0.004)&        25.55 (0.23)&         2.18 (0.55)&              0.0039&              1.1374\\
       G48.61+0.02 0&              0.0229&             48.6055&         1.01 (0.00)&      -0.067 (0.003)&        18.08 (0.09)&         4.97 (0.22)&              0.0035&              1.1374\\
       G48.61+0.02 1&              0.0229&             48.6055&         1.01 (0.00)&      -0.024 (0.005)&         6.08 (0.13)&         1.20 (0.31)&              0.0035&              1.1374\\
       G48.61+0.02 2&              0.0229&             48.6055&         1.01 (0.00)&      -0.018 (0.003)&        53.73 (0.33)&         4.72 (0.79)&              0.0035&              1.1374\\
       G50.32+0.68 0&              0.6761&             50.3153&         0.24 (0.00)&      -0.011 (0.003)&        26.28 (0.40)&         3.32 (0.94)&              0.0031&              1.1374\\
       G60.88-0.13 0&             -0.1285&             60.8826&         0.66 (0.01)&      -0.093 (0.009)&        22.60 (0.15)&         3.24 (0.35)&              0.0096&              1.1374\\
       G61.48+0.09 0&              0.0893&             61.4769&         6.16 (0.01)&      -0.531 (0.009)&        21.45 (0.02)&         2.81 (0.06)&              0.0084&              1.1374\\
       G69.54-0.98 0&             -0.9759&             69.5398&         0.28 (0.01)&      -0.280 (0.006)&        10.65 (0.05)&         4.55 (0.11)&              0.0076&              1.1374\\
       G70.29+1.60 0&              1.6006&             70.2927&         4.37 (0.13)&      -0.372 (0.008)&       -21.74 (0.07)&         3.92 (0.15)&              0.0108&              1.1374\\
       G70.29+1.60 1&              1.6006&             70.2927&         4.37 (0.13)&      -0.050 (0.007)&       -27.17 (0.58)&         4.86 (1.33)&              0.0108&              1.1374\\
       G70.33+1.59 0&               1.589&             70.3296&         2.21 (0.01)&      -1.201 (0.007)&       -21.24 (0.01)&         3.65 (0.03)&              0.0115&              1.1374\\
   IRAS 20051+3435 0&              0.2088&             32.4662&         0.00 (0.01)&      -0.019 (0.001)&        10.77 (0.07)&         3.60 (0.18)&             0.00071&              2.2747\\
       G41.74+0.10 0&              0.0975&             41.7415&         0.34 (0.00)&      -0.062 (0.004)&        14.60 (0.09)&         2.56 (0.26)&              0.0033&              1.1374\\
       G41.74+0.10 1&              0.0975&             41.7415&         0.34 (0.00)&      -0.020 (0.004)&        10.99 (0.29)&         2.52 (0.71)&              0.0033&              1.1374\\
       G41.74+0.10 2&              0.0975&             41.7415&         0.34 (0.00)&      -0.066 (0.004)&        34.25 (0.05)&         1.63 (0.13)&              0.0033&              1.1374\\
       G41.74+0.10 3&              0.0975&             41.7415&         0.34 (0.00)&      -0.022 (0.005)&        56.61 (0.13)&         1.15 (0.32)&              0.0033&              1.1374\\
       G41.74+0.10 4&              0.0975&             41.7415&         0.34 (0.00)&      -0.043 (0.005)&        17.57 (0.07)&         1.13 (0.18)&              0.0033&              1.1374\\
      IRDC 1923+13 0&             -0.4972&             48.9325&         0.40 (0.00)&      -0.011 (0.001)&        50.20 (0.08)&         1.83 (0.19)&              0.0008&              0.7582\\
      IRDC 1923+13 1&             -0.4972&             48.9325&         0.40 (0.00)&      -0.009 (0.001)&        57.56 (0.09)&         2.57 (0.22)&              0.0008&              0.7582\\
      IRDC 1923+13 2&             -0.4972&             48.9325&         0.40 (0.00)&      -0.005 (0.001)&        47.32 (0.20)&         2.11 (0.51)&              0.0008&              0.7582\\
      IRDC 1916+11 0&             -0.2923&              45.666&         0.00 (0.01)&      -0.005 (0.001)&        25.94 (0.17)&         2.53 (0.41)&             0.00083&              0.7582\\
      IRDC 1916+11 1&             -0.2923&              45.666&         0.00 (0.01)&      -0.013 (0.001)&        55.91 (0.13)&         6.21 (0.34)&             0.00083&              0.7582\\
      IRDC 1916+11 2&             -0.2923&              45.666&         0.00 (0.01)&      -0.003 (0.001)&        48.85 (0.48)&         3.58 (1.13)&             0.00083&              0.7582\\
}{
\tablenotetext{a}{Sources are labeled by the line-of-sight followed by the
number of the component identified, indexed from zero.  The components do not
follow a particular order, but are uniquely identifiable by their velocity,
width, and amplitude.}}

%\Table{lccccc}{Measured \formaldehyde\ \twotwo\ line properties}
{\colhead{Source Name}&\colhead{2cm Continuum}&\colhead{Peak\tablenotemark{a}}&\colhead{Center}&\colhead{FWHM }&\colhead{RMS\tablenotemark{b}}\\
\colhead{           }&\colhead{(Jy)}&\colhead{(Jy)}&\colhead{(\kms)}&\colhead{(\kms)}&\colhead{(Jy)}\\ }
{tab:h2comeasured_b}{
       G32.80+0.19 0&         3.68 (0.02)&      -0.519 (0.032)&        15.65 (0.03)&         5.72 (0.08)&              0.0038\\
       G32.80+0.19 1&         3.68 (0.02)&      -0.076 (0.019)&        11.90 (1.18)&         8.17 (0.98)&              0.0038\\
       G32.80+0.19 2&         3.68 (0.02)&      -0.016 (0.001)&        80.47 (0.14)&         4.35 (0.36)&              0.0038\\
       G32.80+0.19 3&         3.68 (0.02)&      -0.065 (0.002)&        84.96 (0.02)&         1.29 (0.05)&              0.0038\\
       G32.80+0.19 4&         3.68 (0.02)&      -0.026 (0.001)&        88.83 (0.06)&         2.31 (0.14)&              0.0038\\
       G33.13-0.09 0&         0.47 (0.02)&      -0.224 (0.003)&        76.17 (0.02)&         3.31 (0.05)&               0.003\\
       G33.13-0.09 1&         0.47 (0.02)&       0.000 (0.000)&         0.00 (0.00)&         0.00 (0.00)&               0.003\\
       G33.13-0.09 2&         0.47 (0.02)&       0.000 (0.000)&         0.00 (0.00)&         0.00 (0.00)&               0.003\\
       G33.13-0.09 3&         0.47 (0.02)&       0.000 (0.000)&         0.00 (0.00)&         0.00 (0.00)&               0.003\\
       G33.92+0.11 0&         0.87 (0.02)&      -0.086 (0.003)&       106.43 (0.03)&         2.17 (0.09)&              0.0032\\
       G33.92+0.11 1&         0.87 (0.02)&      -0.069 (0.002)&       108.83 (0.11)&         6.82 (0.16)&              0.0032\\
       G33.92+0.11 2&         0.87 (0.02)&       0.000 (0.000)&         0.00 (0.00)&         0.00 (0.00)&              0.0032\\
       G34.26+0.15 0&         5.89 (0.02)&      -1.356 (0.006)&        60.99 (0.01)&         3.96 (0.02)&              0.0051\\
       G34.26+0.15 1&         5.89 (0.02)&      -0.046 (0.003)&        27.11 (0.04)&         1.03 (0.09)&              0.0051\\
       G34.26+0.15 2&         5.89 (0.02)&      -0.018 (0.002)&        11.23 (0.16)&         3.19 (0.38)&              0.0051\\
       G34.26+0.15 3&         5.89 (0.02)&      -0.025 (0.004)&        52.82 (0.58)&         6.34 (1.53)&              0.0051\\
       G34.26+0.15 4&         5.89 (0.02)&      -0.018 (0.007)&        47.05 (0.47)&         2.47 (1.15)&              0.0051\\
       G35.20-1.74 0&         5.98 (0.03)&      -0.482 (0.004)&        43.38 (0.02)&         3.71 (0.04)&              0.0055\\
       G35.20-1.74 1&         5.98 (0.03)&      -0.028 (0.005)&        37.91 (0.32)&         3.46 (0.76)&              0.0055\\
       G35.20-1.74 2&         5.98 (0.03)&      -0.056 (0.003)&        14.18 (0.02)&         1.00 (0.05)&              0.0055\\
       G35.20-1.74 3&         5.98 (0.03)&       0.000 (0.000)&         0.00 (0.00)&         0.00 (0.00)&              0.0055\\
       G35.57-0.03 0&         0.32 (0.15)&      -0.075 (0.003)&        52.14 (0.09)&         4.39 (0.21)&              0.0046\\
       G35.57-0.03 1&         0.32 (0.15)&      -0.015 (0.006)&        47.39 (0.25)&         1.31 (0.60)&              0.0046\\
       G35.57-0.03 2&         0.32 (0.15)&       0.000 (0.000)&         0.00 (0.00)&         0.00 (0.00)&              0.0046\\
       G35.57-0.03 3&         0.32 (0.15)&       0.000 (0.000)&         0.00 (0.00)&         0.00 (0.00)&              0.0046\\
       G35.57-0.03 4&         0.32 (0.15)&      -0.024 (0.008)&        29.25 (0.11)&         0.43 (0.15)&              0.0046\\
       G35.58+0.07 0&         0.23 (0.09)&      -0.106 (0.002)&        49.21 (0.06)&         5.00 (0.14)&              0.0031\\
       G35.58+0.07 1&         0.23 (0.09)&       0.000 (0.004)&         0.00 (0.00)&         0.00 (0.00)&              0.0031\\
       G35.58+0.07 2&         0.23 (0.09)&       0.000 (0.004)&         0.00 (0.00)&         0.00 (0.00)&              0.0031\\
       G35.58+0.07 3&         0.23 (0.09)&       0.000 (0.004)&         0.00 (0.00)&         0.00 (0.00)&              0.0031\\
       G37.87-0.40 0&         3.73 (0.02)&      -0.221 (0.003)&        59.99 (0.12)&         8.53 (0.14)&              0.0048\\
       G37.87-0.40 1&         3.73 (0.02)&      -0.045 (0.007)&        54.55 (0.25)&         5.99 (0.34)&              0.0048\\
       G37.87-0.40 2&         3.73 (0.02)&      -0.036 (0.007)&        65.06 (0.11)&         2.57 (0.45)&              0.0048\\
       G37.87-0.40 3&         3.73 (0.02)&      -0.053 (0.003)&        72.44 (0.05)&         1.37 (0.08)&              0.0048\\
       G37.87-0.40 4&         3.73 (0.02)&      -0.047 (0.002)&        74.25 (0.07)&         2.07 (0.18)&              0.0048\\
       G37.87-0.40 5&         3.73 (0.02)&      -0.016 (0.001)&        80.04 (0.03)&         1.28 (0.07)&              0.0048\\
       G37.87-0.40 6&         3.73 (0.02)&      -0.010 (0.002)&        91.99 (0.12)&         1.60 (0.28)&              0.0048\\
       G37.87-0.40 7&         3.73 (0.02)&      -0.026 (0.002)&        14.89 (0.12)&         1.40 (0.20)&              0.0048\\
       G37.87-0.40 8&         3.73 (0.02)&      -0.017 (0.002)&        13.29 (0.19)&         1.52 (0.34)&              0.0048\\
       G37.87-0.40 9&         3.73 (0.02)&      -0.017 (0.001)&        20.52 (0.10)&         3.09 (0.23)&              0.0048\\
       G43.89-0.78 0&         0.53 (0.02)&      -0.059 (0.004)&        54.61 (0.08)&         2.85 (0.23)&               0.003\\
       G43.89-0.78 1&         0.53 (0.02)&      -0.015 (0.002)&        49.59 (0.94)&        14.49 (1.69)&               0.003\\
       G45.07+0.13 0&         0.79 (0.07)&      -0.073 (0.003)&        57.18 (0.08)&         3.45 (0.18)&              0.0029\\
       G45.07+0.13 1&         0.79 (0.07)&      -0.011 (0.003)&        65.67 (0.42)&         3.46 (0.98)&              0.0029\\
       G45.12+0.13 0&         5.20 (0.20)&      -0.086 (0.002)&        56.21 (0.11)&         5.22 (0.21)&              0.0044\\
       G45.12+0.13 1&         5.20 (0.20)&      -0.059 (0.005)&        59.70 (0.06)&         2.42 (0.16)&              0.0044\\
       G45.12+0.13 2&         5.20 (0.20)&      -0.047 (0.002)&        25.14 (0.04)&         1.55 (0.09)&              0.0044\\
       G45.12+0.13 3&         5.20 (0.20)&      -0.021 (0.001)&        64.68 (0.39)&         8.15 (0.87)&              0.0044\\
       G45.45+0.06 0&         3.16 (0.02)&      -0.260 (0.003)&        59.58 (0.01)&         2.06 (0.03)&              0.0043\\
       G45.45+0.06 1&         3.16 (0.02)&      -0.042 (0.002)&        57.90 (0.14)&         9.40 (0.31)&              0.0043\\
       G45.45+0.06 2&         3.16 (0.02)&       0.000 (0.000)&         0.00 (0.00)&         0.00 (0.00)&              0.0043\\
       G45.47+0.05 0&         0.38 (0.02)&      -0.124 (0.003)&        61.67 (0.07)&         5.85 (0.17)&              0.0049\\
       G45.47+0.05 1&         0.38 (0.02)&      -0.000 (0.007)&         0.00 (0.00)&         0.00 (0.00)&              0.0049\\
       G48.61+0.02 0&         0.41 (0.02)&      -0.022 (0.003)&        18.50 (0.25)&         4.39 (0.59)&              0.0033\\
       G48.61+0.02 1&         0.41 (0.02)&      -0.000 (0.000)&         0.00 (0.00)&         0.00 (0.00)&              0.0033\\
       G48.61+0.02 2&         0.41 (0.02)&      -0.005 (0.002)&        52.50 (1.25)&         7.47 (2.94)&              0.0033\\
       G50.32+0.68 0&         0.16 (0.02)&      -0.011 (0.003)&        26.21 (0.44)&         3.10 (1.03)&              0.0036\\
       G60.88-0.13 0&         0.29 (0.02)&      -0.016 (0.003)&        21.63 (0.21)&         2.47 (0.50)&               0.003\\
       G61.48+0.09 0&         3.42 (0.02)&      -0.300 (0.004)&        21.40 (0.02)&         2.39 (0.04)&              0.0037\\
       G69.54-0.98 0&         0.23 (0.02)&      -0.220 (0.002)&         9.97 (0.03)&         5.81 (0.08)&              0.0031\\
       G70.29+1.60 0&         6.21 (0.02)&      -0.159 (0.003)&       -23.52 (0.06)&         5.36 (0.13)&              0.0046\\
       G70.29+1.60 1&         6.21 (0.02)&      -0.000 (0.000)&        -0.00 (0.00)&         0.00 (0.00)&              0.0046\\
       G70.33+1.59 0&         2.68 (0.02)&      -1.081 (0.005)&       -21.17 (0.01)&         2.95 (0.01)&              0.0038\\
   IRAS 20051+3435 0&         0.00 (0.02)&      -0.016 (0.003)&        11.51 (0.37)&         4.14 (0.88)&              0.0032\\
       G41.74+0.10 0&         0.28 (0.02)&      -0.014 (0.002)&        14.36 (0.34)&         3.80 (0.80)&              0.0032\\
       G41.74+0.10 1&         0.28 (0.02)&       0.000 (0.004)&         0.00 (0.00)&         0.00 (0.00)&              0.0032\\
       G41.74+0.10 2&         0.28 (0.02)&       0.000 (0.004)&         0.00 (0.00)&         0.00 (0.00)&              0.0032\\
       G41.74+0.10 3&         0.28 (0.02)&       0.000 (0.004)&         0.00 (0.00)&         0.00 (0.00)&              0.0032\\
       G41.74+0.10 4&         0.28 (0.02)&       0.000 (0.004)&         0.00 (0.00)&         0.00 (0.00)&              0.0032\\
      IRDC 1923+13 0&         0.00 (0.02)&       0.000 (0.000)&         0.00 (0.00)&         0.00 (0.00)&              0.0032\\
      IRDC 1923+13 1&         0.00 (0.02)&       0.000 (0.000)&         0.00 (0.00)&         0.00 (0.00)&              0.0032\\
      IRDC 1923+13 2&         0.00 (0.02)&       0.000 (0.000)&         0.00 (0.00)&         0.00 (0.00)&              0.0032\\
      IRDC 1916+11 0&         0.00 (0.02)&       0.000 (0.000)&         0.00 (0.00)&         0.00 (0.00)&              0.0048\\
      IRDC 1916+11 1&         0.00 (0.02)&       0.000 (0.000)&         0.00 (0.00)&         0.00 (0.00)&              0.0048\\
      IRDC 1916+11 2&         0.00 (0.02)&       0.000 (0.000)&         0.00 (0.00)&         0.00 (0.00)&              0.0048\\
}{
\tablenotetext{a}{ The Upper Limit Flag is 1 when the measurement indicated is
a $3-\sigma$ upper limit on the \twotwo\ line depth when there is a
corresponding \oneone\ line detection. }
\tablenotetext{b}{RMS in 1.011 \kms\ channels.} 
}

%\Table{lccccccc}{Distance, BGPS 1.1 mm, and other properties}
{\colhead{Source Name}&\colhead{Distance}&\colhead{Galactocentric}&\colhead{KDA\tablenotemark{a}}&\colhead{$S_{1.1mm}$}&\colhead{Source}&\colhead{\formaldehyde\ }&\colhead{Scenario\tablenotemark{b}}\\
\colhead{           }&\colhead{        }&\colhead{Distance      }&\colhead{Resolution}&\colhead{           }&\colhead{Type  }&\colhead{Spectrum}&\colhead{}\\  
\colhead{           }&\colhead{(kpc)   }&\colhead{         (kpc)}&\colhead{          }&\colhead{(Jy)       }&\colhead{      }&\colhead{Type    }&\colhead{}\\ }
{tab:other}{
       G32.80+0.19 0&                12.9&                 7.4&                 far&                6.94&               UCHII&        red gradient&                 2+3\\
       G32.80+0.19 1&                13.1&                 7.6&                 far&                6.94&               UCHII&            envelope&                 2+3\\
       G32.80+0.19 2&                 9.4&                 5.1&                 far&                6.94&                 GMC&                   -&                 2+3\\
       G32.80+0.19 3&                 9.2&                 5.0&                 far&                6.94&                 GMC&                   -&                 2+3\\
       G32.80+0.19 4&                 9.0&                 4.9&                 far&                6.94&                 GMC&                   -&                 2+3\\
       G33.13-0.09 0&                 9.6&                 5.2&                 far&                2.26&               UCHII&        red gradient&                   2\\
       G33.13-0.09 1&                 9.3&                 5.1&                 far&                2.26&                 GMC&            envelope&                   2\\
       G33.13-0.09 2&                 7.1&                 4.7&             tangent&                2.26&                 GMC&                   -&                   2\\
       G33.13-0.09 3&                 0.9&                 7.6&                near&                2.26&                 GMC&                   -&                   2\\
       G33.92+0.11 0&                 7.0&                 4.6&             tangent&                3.86&               UCHII&        red gradient&                   2\\
       G33.92+0.11 1&                 7.0&                 4.6&             tangent&                3.86&               UCHII&            envelope&                   2\\
       G33.92+0.11 2&                 3.6&                 5.8&                near&                3.86&                 GMC&                   -&                   2\\
       G34.26+0.15 0&                 3.6&                 5.7&                near&               35.69&               UCHII&        red gradient&                   2\\
       G34.26+0.15 1&                 1.9&                 6.9&                near&               35.69&                 GMC&                   -&                   2\\
       G34.26+0.15 2&                 1.0&                 7.6&                near&               35.69&                 GMC&                   -&                   2\\
       G34.26+0.15 3&                 3.6&                 6.0&                near&               35.69&                 GMC&            envelope&                   2\\
       G34.26+0.15 4&                 3.6&                 6.1&                near&               35.69&                 GMC&                   -&                   2\\
       G35.20-1.74 0&                 2.8&                 6.3&                near&                   -&               UCHII&              single&                   4\\
       G35.20-1.74 1&                 2.5&                 6.5&                near&                   -&                 GMC&                   -&                   4\\
       G35.20-1.74 2&                 1.1&                 7.5&                near&                   -&                 GMC&                   -&                   4\\
       G35.20-1.74 3&                 3.2&                 6.1&                near&                   -&                 GMC&                   -&                   4\\
       G35.57-0.03 0&                10.3&                 6.0&                 far&                2.57&               UCHII&              single&                 2+3\\
       G35.57-0.03 1&                10.7&                 6.2&                 far&                2.57&                 GMC&                   -&                 2+3\\
       G35.57-0.03 2&                 3.6&                 5.9&                near&                2.57&                 GMC&                   -&                 2+3\\
       G35.57-0.03 3&                 1.1&                 7.6&                near&                2.57&                 GMC&                   -&                 2+3\\
       G35.57-0.03 4&                 2.0&                 6.8&                near&                2.57&                 GMC&                   -&                 2+3\\
       G35.58+0.07 0&                10.5&                 6.1&                 far&                1.44&               UCHII&       blue gradient&                   2\\
       G35.58+0.07 1&                10.3&                 6.0&                 far&                1.44&               UCHII&                   -&                   2\\
       G35.58+0.07 2&                 3.6&                 5.8&                near&                1.44&                 GMC&                   -&                   2\\
       G35.58+0.07 3&                 1.1&                 7.5&                near&                1.44&                 GMC&                   -&                   2\\
       G37.87-0.40 0&                 9.4&                 5.9&                 far&                4.14&               UCHII&       blue gradient&                   1\\
       G37.87-0.40 1&                 9.8&                 6.1&                 far&                4.14&               UCHII&       blue gradient&                   1\\
       G37.87-0.40 2&                 9.2&                 5.7&                 far&                4.14&               UCHII&       blue gradient&                   1\\
       G37.87-0.40 3&                 8.7&                 5.6&                 far&                4.14&                 GMC&                   -&                   1\\
       G37.87-0.40 4&                 8.6&                 5.5&                 far&                4.14&                 GMC&                   -&                   1\\
       G37.87-0.40 5&                 8.1&                 5.4&                 far&                4.14&                 GMC&                   -&                   1\\
       G37.87-0.40 6&                 6.6&                 5.1&             tangent&                4.14&                 GMC&                   -&                   1\\
       G37.87-0.40 7&                 1.2&                 7.5&                near&                4.14&                 GMC&                   -&                   1\\
       G37.87-0.40 8&                 1.1&                 7.6&                near&                4.14&                 GMC&                   -&                   1\\
       G37.87-0.40 9&                 1.5&                 7.2&                near&                4.14&                 GMC&                   -&                   1\\
       G43.89-0.78 0&                 8.3&                 6.2&                 far&                   -&               UCHII&       blue gradient&                   3\\
       G43.89-0.78 1&                 8.6&                 6.3&                 far&                   -&                 GMC&            envelope&                   3\\
       G45.07+0.13 0&                 7.6&                 6.2&                 far&                4.26&               UCHII&              single&                   2\\
       G45.07+0.13 1&                 6.5&                 6.0&                 far&                4.26&                 GMC&                   -&                   2\\
       G45.12+0.13 0&                 7.4&                 6.2&                 far&                6.78&               UCHII&               other&                   1\\
       G45.12+0.13 1&                 7.4&                 6.1&                 far&                6.78&               UCHII&            envelope&                   1\\
       G45.12+0.13 2&                 1.9&                 7.2&                near&                6.78&                 GMC&                   -&                   1\\
       G45.12+0.13 3&                 7.4&                 6.0&                 far&                6.78&                 GMC&            envelope&                   1\\
       G45.45+0.06 0&                 7.2&                 6.1&                 far&                3.71&               UCHII&       blue gradient&                   2\\
       G45.45+0.06 1&                 7.6&                 6.2&                 far&                3.71&                 GMC&            envelope&                   2\\
       G45.45+0.06 2&                 1.9&                 7.2&                near&                3.71&                 GMC&                   -&                   2\\
       G45.47+0.05 0&                 7.1&                 6.1&                 far&                3.34&               UCHII&        red gradient&               1+2+3\\
       G45.47+0.05 1&                 1.9&                 7.2&                near&                3.34&                 GMC&                   -&               1+2+3\\
       G48.61+0.02 0&                 9.6&                 7.5&                 far&                2.20&               UCHII&        red gradient&                 2+3\\
       G48.61+0.02 1&                 0.7&                 8.0&                near&                2.20&                 GMC&                   -&                 2+3\\
       G48.61+0.02 2&                 6.5&                 6.4&                 far&                2.20&                 GMC&                   -&                 2+3\\
       G50.32+0.68 0&                 2.1&                 7.2&                near&                   -&               UCHII&                   -&                   1\\
       G60.88-0.13 0&                 2.8&                 7.4&                near&                4.90&               UCHII&               limit&                   2\\
       G61.48+0.09 0&                 5.2&                 7.5&                 far&                7.86&               UCHII&              single&                   4\\
       G69.54-0.98 0&                2.57&                 7.9&             tangent&                   -&               UCHII&               thick&                 4+5\\
       G70.29+1.60 0&                 7.3&                 9.1&                 far&                   -&               UCHII&       blue gradient&                   2\\
       G70.29+1.60 1&                 7.8&                 9.3&                 far&                   -&                 GMC&            envelope&                   2\\
       G70.33+1.59 0&                 7.3&                 9.1&                 far&                   -&               UCHII&              single&                 1+2\\
   IRAS 20051+3435 0&                 2.6&                 7.6&             tangent&                   -&                 GMC&               limit&                  -1\\
       G41.74+0.10 0&                11.3&                 7.6&                 far&                0.56&               UCHII&               limit&                  -1\\
       G41.74+0.10 1&                11.6&                 7.7&                 far&                0.56&               UCHII&                   -&                  -1\\
       G41.74+0.10 2&                 2.4&                 6.8&                near&                0.56&                 GMC&                   -&                  -1\\
       G41.74+0.10 3&                 3.8&                 6.1&                near&                0.56&                 GMC&                   -&                  -1\\
       G41.74+0.10 4&                11.2&                 7.4&                 far&                0.56&               UCHII&                   -&                  -1\\
      IRDC 1923+13 0&                 4.2&                 6.5&                near&                   -&                 GMC&               limit&                  -1\\
      IRDC 1923+13 1&                 5.5&                 6.3&             tangent&                   -&                 GMC&                   -&                  -1\\
      IRDC 1923+13 2&                 3.8&                 6.6&                near&                   -&                 GMC&                   -&                  -1\\
      IRDC 1916+11 0&                 2.0&                 7.2&                near&                   -&                 GMC&               limit&                  -1\\
      IRDC 1916+11 1&                 4.2&                 6.2&                near&                   -&                 GMC&                   -&                  -1\\
      IRDC 1916+11 2&                 3.6&                 6.4&                near&                   -&                 GMC&                   -&                  -1\\
}{
\tablenotetext{a}{The Kinematic Distance Ambiguity described in Section \ref{sec:distances}.}
\tablenotetext{b}{Scenario or scenarios most likely to be consistent with the observed spectrum, as described
in Section \ref{sec:scenarios}.  In some cases, the spectrum was consistent with multiple scenarios or some
blend of multiple scenarios.  In others, the source could not be classified, in which case it is marked with 
-1 in this column. }
}


\subsection{Measuring Line Optical Depth}
\label{sec:linedepth}
In order to measure physical properties of an absorbing source, measurements
must be obtained of the optical depths of both the \oneone\ and \twotwo\ lines.
These measurements are presented in Tables \ref{tab:h2comeasured_a} and \ref{tab:h2comeasured_a}.
Once an optical depth with errors is determined, the spectral line depths can be matched
to large velocity gradient (LVG) models to determine column and spatial
density.  The spectral line optical depth depends both on the nadir flux density of the
absorption line and the strength of the illuminating background continuum
source.  If the background is the CMB, the `filling factor' of the molecular
cloud is simply its size relative to the beam size.  If there is a continuum
source in addition to the CMB, the size of the continuum source and the
intervening molecular cloud both affect the absorption depth.  Throughout this
paper, we use the term `filling factor' to refer to the fraction of the beam area filled
by the absorbing molecular cloud and `covering factor' to refer to the fraction
of the background continuum source that is covered by the intervening molecular
material.

The VLA archival images were used to estimate the size of the illuminating
background source.  When images at both wavelengths were available, we
separately determined the 2 cm and 6 cm source sizes.  The source size
determination is imprecise because we select a single source size for
non-uniform surface brightness profiles, and in many cases the VLA observation
did not recover the full flux density seen in single-dish measurements.
\citet{Araya2002} estimated the interferometer-to- single-dish flux ratio at
6 cm in this sample and found that the interferometer observations recovered
anywhere from 3\% to 100\% of the single-dish flux.  We repeat these
measurements at 2 cm and find that the typical recovery fraction is higher,
$\sim40\%$ to $100\%$, although sources for which only VLA upper limits could
be measured have recovery fractions $<1\%$. 

%When calculating the line optical depth, we assumed that the absorbing
%molecular cloud uniformly covers the illuminating background source.  We make
%the additional assumption that the molecular absorber covers {\it only} the
%background source and has no additional extent in the beam, such that the
%contribution from the CMB is uniform over the \ion{H}{2} region area and zero
%elsewhere (while the CMB is ubiquitous, with the on-off calibration strategy
%its net contribution is zero unless it is absorbed in the on position).  This
%assumption probably holds for the dominant line associated with the UC\ion{H}{2}
%region, but is likely to be incorrect for diffuse clouds along the line of
%sight.  %We have examined the assumption for clouds at $\ell < 60\degree$ by
%%looking at the GRS data cubes at the same velocity to determine whether the
%%cloud is extended and beam-filling.

\OneColFigure{figures_chH2CO/Derived_DensityVsDensityCorrected_all} 
{The filling factor corrected (FFC) density vs. the derived density with no
filling factor correction.  While there are some cases where the correction
results in an order of magnitude or more increase in the density, most points
show a small correction.  The black line is the one-one line.  Red squares
show where the filling factor corrected point was used, while blue circles show
where the uncorrected point was used.  Magenta left-pointing triangles are
limits where the filling factor correction was used, green downward triangles
are limits where the uncorrected points were used, and orange upward triangles
are lower limits where the filling-factor correction was used.}
{fig:ffc}
{0.30}{0}

The optical depth measurements were ``filling factor corrected'' by assuming
the CMB only contributed flux density over the same area as the \ion{H}{2}
region (i.e., the foreground cloud covers the exact same patch of sky as the
\uchii\ region).  When the \ion{H}{2} region is small (e.g., 10\% of the beam
area or less), the contribution of the CMB to the continuum is negligible, but
in cases of more diffuse \ion{H}{2} regions, the CMB contribution is
significant, particularly at 2 cm.  The inferred optical depths and source areas
are presented in Table \ref{tab:h2coinferred}.  Both ``filling factor corrected'' and
uncorrected densities are presented in Table \ref{tab:h2coderived}.  The effect
of the filling factor correction (FFC) on density measurements is shown in
Figure \ref{fig:ffc}.  In a few cases, no volume density-column density
parameter space in the models (Section \ref{sec:models}) was consistent with
the spectral line ratio after filling factor correction: in these cases, the
filling factor correction was not used.  Similarly, no filling factor
correction was applied to sources without detected continuum.  These exceptions
are noted in Table \ref{tab:h2coderived} in the ``Flag'' column.
%These diffuse molecular clouds are more likely to be beam-filling, and
%therefore the uncorrected (no-FFC) density measurements may be more accurate.
%XXXX How were sources put in "FFC" or "No FFC" categories?

The above definitions are summarized briefly in the following equations:
\begin{eqnarray*}
  S_{\nu,obs} &=& S_{\nu,cont} (1-CF e^{-\tau_{\nu}}) - S_{\nu,CMB} (FF e^{-\tau_{\nu}}) \\
  FF &=& \Omega_{cloud} / \Omega_{beam} \\ 
  CF &=& \Omega_{cloud} / \Omega_{continuum} \\ 
\end{eqnarray*}
in which CF is the ``covering factor'', FF is the ``filling factor'', and there
is no positive contribution from the CMB because it is assumed to be removed by
position-switching.


The systematic uncertainties in the continuum and the filling factor result in
similar errors in the optical depth measurement, and together dominate the
total error budget for our measurements.  A 30\% error in the \oneone\ and 20\%
error in the \twotwo\ continuum levels were assumed because of flux calibration
uncertainty characteristic of the instruments. An additional 10\% error in the beam area, which sets the maximum
coupling to the CMB (assuming a beam-filling source), was included to account 
for focus error.
%The beam
%area error is probably a good estimate of the (systematic) error in the CMB
%measurement, while the continuum errors are conservative.  
A 20\% statistical error in the cloud filling factor was assumed for the majority of the
survey, but it was decreased to 10\% when the ratio of continuum to CMB flux was $>0.5$ and the source
size was small, indicating that the VLA-measured source is indeed the dominant
continuum component in the beam.  The statistical error does not account for systematic
errors in the geometric assumptions.  Note that changes to the filling factor
should have a minimal effect on the derived density unless the source sizes at
2 cm and 6 cm differ substantially, while changes in the filling factor will
always have a large effect on the derived column density (Figure
\ref{fig:ffcdependence}).

\OneColFigure{figures_chH2CO/model_ffcdependence}
{The dependence of derived parameters on the filling factor, assuming an
optical depth ratio $\tau_{\oneone}/\tau_{\twotwo} =$1 (solid), 2 (dash-dot),
or 4 (dashed).  
The X-axis is the ``real'' optical depth, $\tau_{1-1}(real) = \tau_{1-1}(observed) / FF$.
Assuming the same filling factor correction is applied to both
the \oneone\ and \twotwo\ lines, filling factor correction will only move the
measurements along the X-axis of these plots.  A decrease in the filling factor
requires an increase in the true optical depth to maintain a constant apparent
$\tau(observed)$, which in turn drives up the derived abundance and column density while
leaving the volume density unchanged (except at high optical depths,
$\tau\gtrsim0.2$).
}
{fig:ffcdependence}{0.30}{0}


% have I discussed low density + high density clouds?  make a plot
% of density measured / real density vs. relative filling factor...
% or filling factor of the high-density stuff

%The signal-to-noise in the absorption lines is excellent. In order to improve
%the quality of our measurements, we need more accurate continuum measurements,
%which can be accomplished with a more careful observing technique and
%calibrator selection, or by observing sources with only the CMB as
%illumination. Cloud size and covering measurements will also improve the
%accuracy of the density measurements.  These can be obtained with higher
%resolution observations or, as shown in \citet{Zeiger2010}, higher frequency
%transitions of \formaldehyde.
%
%(XXXX it may be possible to correct for this... but.... is it worth it?) An
%additional systematic uncertainty not considered in this survey is the
%potential additional contribution of a diffuse Galactic background as a
%backlight.   Emission that is smooth on $>11\arcmin$ scales, as is known to
%exist throughout the plane, will be subtracted off by nodding but may still be
%absorbed by the intervening \formaldehyde.  This signal should be small
%compared to continuum sources and generally less than the CMB at 2 cm (average
%in the plane is $\lesssim0.1K$), but it may be comparable to the CMB at 6 cm
%because of rising synchrotron emission ($\sim1K$), and it will be important to
%account for it in a larger survey.  It is possible to do this using publicly
%available data sets such as the GPA (Langston XXXX cite) and WMAP foreground
%maps (XXXX).

%\Table{lcccccccc}{Inferred \formaldehyde\ line properties}
{\colhead{Source Name}&\colhead{$\tau_{1-1}$}&\colhead{$\tau_{1-1}$ (FFC)}&\colhead{$\tau_{2-2}$}&\colhead{$\tau_{2-2}$ (FFC)}&\colhead{2-2 Upper }&\colhead{2cm Area\tablenotemark{a}}&\colhead{6cm Area  \tablenotemark{a}   }&\colhead{FFC Error}\\
\colhead{           }&\colhead{             }&\colhead{                   }&\colhead{             }&\colhead{                   }&\colhead{Limit Flag}&\colhead{\arcsec$^2$                   }&\colhead{\arcsec$^2$                   }&\colhead{}\\ }
{tab:h2coinferred}{
       G32.80+0.19 0&        0.18 (0.055)&         0.2 (0.059)&        0.12 (0.024)&        0.15 (0.031)&                   0&                88.0&               226.2&                 0.1\\
       G32.80+0.19 1&        0.04 (0.013)&       0.043 (0.013)&      0.016 (0.0051)&      0.021 (0.0065)&                   0&                88.0&               226.2&                 0.1\\
       G32.80+0.19 2&      0.027 (0.0089)&      0.029 (0.0095)&    0.0033 (0.00069)&    0.0042 (0.00088)&                   0&                88.0&               226.2&                 0.1\\
       G32.80+0.19 3&        0.11 (0.035)&        0.12 (0.037)&      0.014 (0.0028)&      0.018 (0.0035)&                   0&                88.0&               226.2&                 0.1\\
       G32.80+0.19 4&       0.039 (0.012)&       0.042 (0.013)&     0.0055 (0.0011)&     0.0071 (0.0014)&                   0&                88.0&               226.2&                 0.1\\
       G33.13-0.09 0&          0.34 (0.1)&         0.49 (0.15)&        0.16 (0.032)&         0.63 (0.12)&                   0&                33.5&                33.5&                 0.2\\
       G33.13-0.09 1&       0.035 (0.015)&        0.047 (0.02)&         0 (0.0059)&         0 (0.0031)&                   1&                33.5&                33.5&                 0.2\\
       G33.13-0.09 2&       0.062 (0.022)&       0.084 (0.029)&         0 (0.0059)&         0 (0.0031)&                   1&                33.5&                33.5&                 0.2\\
       G33.13-0.09 3&       0.061 (0.021)&       0.082 (0.028)&         0 (0.0059)&         0 (0.0031)&                   1&                33.5&                33.5&                 0.2\\
       G33.92+0.11 0&       0.084 (0.027)&         0.1 (0.031)&      0.045 (0.0091)&       0.094 (0.018)&                   0&               214.0&               214.0&                 0.2\\
       G33.92+0.11 1&       0.082 (0.026)&       0.098 (0.031)&      0.036 (0.0072)&       0.075 (0.014)&                   0&               214.0&               214.0&                 0.2\\
       G33.92+0.11 2&        0.17 (0.062)&        0.21 (0.074)&         0 (0.0049)&         0 (0.0031)&                   1&               214.0&               214.0&                 0.2\\
       G34.26+0.15 0&         0.38 (0.12)&          0.4 (0.12)&        0.22 (0.043)&        0.26 (0.052)&                   0&                10.9&                10.9&                 0.2\\
       G34.26+0.15 1&      0.028 (0.0089)&      0.029 (0.0092)&     0.0067 (0.0014)&     0.0079 (0.0017)&                   0&                10.9&                10.9&                 0.2\\
       G34.26+0.15 2&      0.017 (0.0059)&       0.018 (0.006)&    0.0026 (0.00059)&     0.0031 (0.0007)&                   0&                10.9&                10.9&                 0.2\\
       G34.26+0.15 3&      0.022 (0.0072)&      0.023 (0.0074)&    0.0036 (0.00092)&     0.0043 (0.0011)&                   0&                10.9&                10.9&                 0.2\\
       G34.26+0.15 4&     0.0082 (0.0036)&     0.0085 (0.0037)&     0.0026 (0.0011)&      0.003 (0.0013)&                   0&                10.9&                10.9&                 0.2\\
       G35.20-1.74 0&        0.21 (0.063)&        0.22 (0.066)&       0.071 (0.014)&       0.084 (0.017)&                   0&                39.5&                39.5&                 0.2\\
       G35.20-1.74 1&      0.028 (0.0085)&      0.029 (0.0088)&     0.0039 (0.0011)&     0.0046 (0.0013)&                   0&                39.5&                39.5&                 0.2\\
       G35.20-1.74 2&       0.063 (0.019)&       0.065 (0.019)&      0.008 (0.0017)&     0.0095 (0.0019)&                   0&                39.5&                39.5&                 0.2\\
       G35.20-1.74 3&     0.0073 (0.0027)&     0.0075 (0.0028)&         0 (0.0023)&         0 (0.0031)&                   1&                39.5&                39.5&                 0.2\\
       G35.57-0.03 0&        0.11 (0.035)&        0.15 (0.049)&       0.056 (0.011)&        0.26 (0.054)&                   0&                 6.7&                 6.7&                 0.1\\
       G35.57-0.03 1&       0.034 (0.018)&       0.046 (0.024)&      0.011 (0.0047)&        0.047 (0.02)&                   0&                 6.7&                 6.7&                 0.1\\
       G35.57-0.03 2&        0.03 (0.017)&       0.042 (0.023)&         0 (0.0099)&          0 (0.019)&                   1&                 6.7&                 6.7&                 0.1\\
       G35.57-0.03 3&        0.05 (0.021)&       0.069 (0.029)&         0 (0.0099)&          0 (0.019)&                   1&                 6.7&                 6.7&                 0.1\\
       G35.57-0.03 4&        0.051 (0.02)&       0.069 (0.028)&      0.017 (0.0065)&       0.077 (0.029)&                   0&                 6.7&                 6.7&                 0.1\\
       G35.58+0.07 0&        0.24 (0.071)&        0.32 (0.097)&       0.085 (0.017)&         0.61 (0.12)&                   0&                 2.1&                 2.1&                 0.2\\
       G35.58+0.07 1&       0.072 (0.029)&       0.096 (0.038)&         0 (0.0072)&          0 (0.019)&                   1&                 2.1&                 2.1&                 0.2\\
       G35.58+0.07 2&       0.037 (0.012)&       0.049 (0.016)&         0 (0.0072)&          0 (0.019)&                   1&                 2.1&                 2.1&                 0.2\\
       G35.58+0.07 3&        0.05 (0.016)&       0.066 (0.021)&         0 (0.0072)&           0 (0.01)&                   1&                 2.1&                 2.1&                 0.2\\
       G37.87-0.40 0&        0.12 (0.037)&        0.13 (0.038)&      0.047 (0.0095)&       0.061 (0.012)&                   0&                27.5&               170.9&                 0.2\\
       G37.87-0.40 1&      0.028 (0.0089)&      0.029 (0.0092)&     0.0095 (0.0024)&      0.012 (0.0031)&                   0&                27.5&               170.9&                 0.2\\
       G37.87-0.40 2&       0.081 (0.025)&       0.084 (0.026)&     0.0075 (0.0021)&     0.0096 (0.0027)&                   0&                27.5&               170.9&                 0.2\\
       G37.87-0.40 3&       0.074 (0.024)&       0.076 (0.025)&      0.011 (0.0023)&       0.014 (0.003)&                   0&                27.5&               170.9&                 0.2\\
       G37.87-0.40 4&       0.097 (0.029)&          0.1 (0.03)&      0.0098 (0.002)&      0.013 (0.0026)&                   0&                27.5&               170.9&                 0.2\\
       G37.87-0.40 5&       0.041 (0.013)&       0.043 (0.013)&    0.0033 (0.00068)&    0.0043 (0.00087)&                   0&                27.5&               170.9&                 0.2\\
       G37.87-0.40 6&      0.025 (0.0083)&      0.026 (0.0086)&    0.0021 (0.00052)&    0.0026 (0.00066)&                   0&                27.5&               170.9&                 0.2\\
       G37.87-0.40 7&       0.039 (0.012)&        0.04 (0.012)&     0.0054 (0.0012)&     0.0069 (0.0015)&                   0&                27.5&               170.9&                 0.2\\
       G37.87-0.40 8&      0.016 (0.0069)&      0.016 (0.0071)&    0.0035 (0.00081)&      0.0046 (0.001)&                   0&                27.5&               170.9&                 0.2\\
       G37.87-0.40 9&       0.03 (0.0095)&      0.031 (0.0098)&    0.0035 (0.00073)&    0.0045 (0.00094)&                   0&                27.5&               170.9&                 0.2\\
       G43.89-0.78 0&        0.25 (0.074)&        0.32 (0.096)&      0.037 (0.0078)&        0.12 (0.024)&                   0&                13.5&                13.5&                 0.1\\
       G43.89-0.78 1&      0.025 (0.0077)&      0.031 (0.0097)&     0.0097 (0.0022)&      0.029 (0.0067)&                   0&                13.5&                13.5&                 0.1\\
       G45.07+0.13 0&       0.092 (0.029)&         0.13 (0.04)&       0.04 (0.0081)&        0.096 (0.02)&                   0&                 2.5&                 2.5&                 0.2\\
       G45.07+0.13 1&        0.058 (0.02)&        0.08 (0.027)&     0.0061 (0.0019)&      0.014 (0.0045)&                   0&                 2.5&                 2.5&                 0.2\\
       G45.12+0.13 0&       0.043 (0.013)&       0.045 (0.013)&      0.014 (0.0028)&      0.017 (0.0033)&                   0&                15.4&               516.6&                 0.2\\
       G45.12+0.13 1&       0.035 (0.011)&       0.036 (0.011)&      0.0095 (0.002)&      0.011 (0.0025)&                   0&                15.4&               516.6&                 0.2\\
       G45.12+0.13 2&       0.046 (0.014)&       0.048 (0.014)&     0.0075 (0.0015)&      0.009 (0.0019)&                   0&                15.4&               516.6&                 0.2\\
       G45.12+0.13 3&       0.006 (0.002)&     0.0062 (0.0021)&    0.0033 (0.00068)&     0.004 (0.00082)&                   0&                15.4&               516.6&                 0.2\\
       G45.45+0.06 0&        0.32 (0.096)&        0.32 (0.095)&       0.063 (0.013)&       0.069 (0.012)&                   0&              1963.0&              1963.0&                 0.2\\
       G45.45+0.06 1&       0.025 (0.011)&       0.026 (0.011)&       0.01 (0.0021)&      0.011 (0.0019)&                   0&              1963.0&              1963.0&                 0.2\\
       G45.45+0.06 2&      0.011 (0.0036)&      0.012 (0.0035)&         0 (0.0031)&           0 (0.01)&                   1&              1963.0&              1963.0&                 0.2\\
       G45.47+0.05 0&         0.35 (0.11)&         0.45 (0.14)&       0.089 (0.018)&        0.39 (0.079)&                   0&                 3.0&                 3.0&                 0.2\\
       G45.47+0.05 1&      0.018 (0.0068)&      0.023 (0.0084)&           0 (0.01)&           0 (0.01)&                   1&                 3.0&                 3.0&                 0.2\\
       G48.61+0.02 0&       0.058 (0.018)&       0.068 (0.021)&      0.015 (0.0034)&       0.053 (0.012)&                   0&                25.5&                25.5&                 0.2\\
       G48.61+0.02 1&       0.02 (0.0075)&      0.023 (0.0088)&         0 (0.0067)&         0 (0.0026)&                   1&                25.5&                25.5&                 0.2\\
       G48.61+0.02 2&      0.016 (0.0052)&      0.018 (0.0061)&     0.0033 (0.0013)&      0.012 (0.0046)&                   0&                25.5&                25.5&                 0.2\\
       G50.32+0.68 0&        0.027 (0.01)&       0.045 (0.017)&     0.0089 (0.0031)&       0.058 (0.019)&                   0&               108.0&               108.0&                 0.2\\
       G60.88-0.13 0&        0.12 (0.037)&        0.14 (0.043)&      0.011 (0.0031)&      0.031 (0.0071)&                   0&               615.0&               615.0&                 0.2\\
       G61.48+0.09 0&       0.088 (0.026)&        0.09 (0.027)&       0.069 (0.014)&       0.088 (0.017)&                   0&               355.0&               355.0&                 0.2\\
       G69.54-0.98 0&         0.98 (0.29)&           5.7 (1.7)&        0.18 (0.037)&          2.9 (0.57)&                   0&                 0.5&                 0.5&                 0.2\\
       G70.29+1.60 0&       0.086 (0.026)&       0.089 (0.027)&      0.022 (0.0044)&      0.026 (0.0052)&                   0&                52.8&                52.8&                 0.1\\
       G70.29+1.60 1&      0.011 (0.0037)&      0.012 (0.0038)&         0 (0.0019)&         0 (0.0026)&                   1&                52.8&                52.8&                 0.1\\
       G70.33+1.59 0&          0.7 (0.21)&         0.78 (0.24)&        0.34 (0.068)&          0.52 (0.1)&                   0&                16.4&                16.4&                 0.1\\
   IRAS 20051+3435 0&        0.12 (0.036)&        0.13 (0.014)&      0.015 (0.0041)&      0.016 (0.0034)&                   0&             2747.75&             2747.75&                 0.0\\
       G41.74+0.10 0&         0.13 (0.04)&          0.2 (0.06)&       0.01 (0.0027)&       0.045 (0.012)&                   0&                75.2&                75.2&                 0.2\\
       G41.74+0.10 1&        0.04 (0.014)&        0.06 (0.021)&         0 (0.0071)&         0 (0.0026)&                   1&                75.2&                75.2&                 0.2\\
       G41.74+0.10 2&        0.14 (0.043)&        0.21 (0.065)&         0 (0.0071)&         0 (0.0026)&                   1&                75.2&                75.2&                 0.2\\
       G41.74+0.10 3&       0.045 (0.017)&       0.067 (0.025)&         0 (0.0071)&         0 (0.0026)&                   1&                75.2&                75.2&                 0.2\\
       G41.74+0.10 4&       0.089 (0.029)&        0.13 (0.043)&         0 (0.0071)&         0 (0.0028)&                   1&                75.2&                75.2&                 0.2\\
      IRDC 1923+13 0&       0.02 (0.0062)&       0.02 (0.0047)&          0 (0.009)&         0 (0.0028)&                   1&             2747.75&             2747.75&                 0.0\\
      IRDC 1923+13 1&      0.016 (0.0051)&      0.017 (0.0038)&          0 (0.009)&         0 (0.0028)&                   1&             2747.75&             2747.75&                 0.0\\
      IRDC 1923+13 2&     0.0081 (0.0028)&     0.0083 (0.0023)&          0 (0.009)&         0 (0.0028)&                   1&             2747.75&             2747.75&                 0.0\\
      IRDC 1916+11 0&       0.033 (0.011)&      0.036 (0.0062)&          0 (0.014)&          0 (0.042)&                   1&             2747.75&             2747.75&                 0.0\\
      IRDC 1916+11 1&       0.082 (0.025)&      0.089 (0.0095)&          0 (0.014)&          0 (0.042)&                   1&             2747.75&             2747.75&                 0.0\\
      IRDC 1916+11 2&      0.017 (0.0064)&      0.018 (0.0046)&          0 (0.014)&          0 (0.042)&                   1&             2747.75&             2747.75&                 0.0\\
}{\tablenotetext{a}{The beam area is 2747.75\arcsec$^2$, which is used when the CMB is the only background continuum illumination}}


%The models predict a line optical depth as a function of density and column
%assuming the line FWHM is 1 \kms\ and $dv/dr$ is 1 \kms pc$^{-1}$ .
Measurements of volume and column density were taken by averaging over the regions of
LVG model parameter space consistent with both spectral line optical depth measurements
to within $1\sigma$.  The ``$1\sigma$'' (68\% confidence; errors are
non-gaussian) error bars on the derived parameters ($N,n,X$) were taken to be
the extrema of these regions.  An example of this fitting process is shown in
Figure \ref{fig:fitexample}.  A second example demonstrating a lower-limit on the density
(instead of a direct measurement) is shown in Figure \ref{fig:fitexample2}.  This method is not as robust as $\chi^2$ fitting,
but because there are no free fit parameters, a statistically meaningful
$\chi^2_\nu$ cannot be computed.

In some cases, the ratio of the spectral line optical depths was consistent with low
density ($n\lesssim 100$ \percc) and high abundances ($X(\ortho)>10^{-8}$ \perkmspc), but
these were ruled out based on the prior assumption that extremely high
\formaldehyde\ abundances should not be observed at very low densities, since
it is formed at higher densities and destroyed by hard UV at low columns
\citep[see discussion in][]{Troscompt2009b}.
%Additionally, assuming that all
%of the matter inferred from 1.1 mm dust observations is associated with the
%\formaldehyde\ absorber, the beam-averaged 1.1 mm mass can be used as a lower
%limit on the volume density; however, that assumption is most likely invalid
%for the absorbers not directly associated with the \uchii\ region.  The use of
%prior assumptions to rule out portions of parameter space is discussed in
%detail in \citet{Zeiger2010}.

\OneColFigure{figures_chH2CO/G32.80+0.19_0_ffc_Nvsn_notext.png} 
{An example of the column density - density parameter space available given
measured \oneone\ and \twotwo\ optical depths.  The dashed lines show
abundances $\log_{10}(X(\ortho))$ \perkmspc.  The contours show the
regions allowed by the measurements of optical depth (\oneone: black, \twotwo: grey,
ratio: dotted);
the middle curve is the measured value, while the pair of curves around it are
$\pm 1\sigma$ including systematic error.  The shaded region shows the allowed
parameter space from which the physical parameters are derived. }
{fig:fitexample}
{0.30}{0}

\OneColFigure{figures_chH2CO/G33.13-0.09_0_ffc_Nvsn_notext.png}
{  Same description as Figure
\ref{fig:fitexample} but for the strongest component in G33.13-0.09.  It was
only possible to measure lower limits on the volume and column density for this
line; it is therefore assigned flag 8 in Table \ref{tab:h2coderived}.
}
{fig:fitexample2}
{0.30}{0}


%\Table{lccccccc}{Derived physical properties from \formaldehyde\ }
{\colhead{Source Name}&\colhead{N(\formaldehyde)\tablenotemark{a}}&\colhead{N(\formaldehyde) (FFC)\tablenotemark{b}}&\colhead{n(\hh) \tablenotemark{a} }&\colhead{n(\hh) (FFC)\tablenotemark{b}}&\colhead{X$_{\formaldehyde}$\tablenotemark{a}}&\colhead{X$_{\formaldehyde}$ (FFC)\tablenotemark{b}}&\colhead{Flag\tablenotemark{c}}\\
\colhead{           }&\colhead{(\persc)        }&\colhead{(\persc)              }&\colhead{(\percc)}&\colhead{(\percc)    }&\colhead{                    }&\colhead{                          }&\colhead{                      }\\ }
{tab:h2coderived}{
       G32.80+0.19 0&$12.79^{+0.11}_{-0.16}$&$\mathbf{12.94^{+0.16}_{-0.24}}$&$5.10^{+0.25}_{-0.26}$&$\mathbf{5.21^{+0.27}_{-0.29}}$&$-10.79^{+0.15}_{-0.20}$&$\mathbf{-10.75^{+0.15}_{-0.18}}$&                   2\\
       G32.80+0.19 1&$12.05^{+0.12}_{-0.11}$&$\mathbf{12.14^{+0.13}_{-0.13}}$&$4.96^{+0.22}_{-0.28}$&$\mathbf{5.05^{+0.21}_{-0.28}}$&$-11.39^{+0.20}_{-0.23}$&$\mathbf{-11.39^{+0.17}_{-0.20}}$&                   2\\
       G32.80+0.19 2&$11.66^{+0.10}_{-0.10}$&$\mathbf{11.71^{+0.10}_{-0.10}}$&$4.16^{+0.39}_{-0.38}$&$\mathbf{4.33^{+0.31}_{-0.32}}$&$-10.97^{+0.44}_{-0.46}$&$\mathbf{-11.10^{+0.37}_{-0.37}}$&                   2\\
       G32.80+0.19 3&$12.18^{+0.10}_{-0.09}$&$\mathbf{12.23^{+0.09}_{-0.09}}$&$4.07^{+0.38}_{-0.39}$&$\mathbf{4.23^{+0.32}_{-0.32}}$&$-10.37^{+0.44}_{-0.45}$&$\mathbf{-10.48^{+0.36}_{-0.38}}$&                   2\\
       G32.80+0.19 4&$11.82^{+0.10}_{-0.09}$&$\mathbf{11.87^{+0.10}_{-0.09}}$&$4.30^{+0.31}_{-0.32}$&$\mathbf{4.44^{+0.26}_{-0.29}}$&$-10.97^{+0.37}_{-0.37}$&$\mathbf{-11.05^{+0.31}_{-0.32}}$&                   2\\
       G33.13-0.09 0&$>12.80                 $&$\mathbf{>13.56}       $&$>4.54                 $&$\mathbf{>5.10}       $&$>-10.62                 $&$\mathbf{>-11.70}       $&                   8\\
       G33.13-0.09 1&$<11.96                 $&$\mathbf{<11.90}       $&$<4.50                 $&$\mathbf{<3.91}       $&$<-8.44                 $&$\mathbf{<-8.45}       $&                   6\\
       G33.13-0.09 2&$\mathbf{<12.20}       $&$<0.00                 $&$\mathbf{<4.29}       $&$<0.00                 $&$\mathbf{<-8.29}       $&$<0.00                 $&                   5\\
       G33.13-0.09 3&$\mathbf{<12.20}       $&$<0.00                 $&$\mathbf{<4.32}       $&$<0.00                 $&$\mathbf{<-8.29}       $&$<0.00                 $&                   5\\
       G33.92+0.11 0&$>12.35                 $&$\mathbf{>12.64}       $&$>4.86                 $&$\mathbf{>5.16}       $&$>-11.29                 $&$\mathbf{>-12.30}       $&                   8\\
       G33.92+0.11 1&$12.34^{+0.07}_{-0.08}$&$\mathbf{12.65^{+0.11}_{-0.17}}$&$4.97^{+0.22}_{-0.23}$&$\mathbf{5.26^{+0.22}_{-0.24}}$&$-11.11^{+0.19}_{-0.22}$&$\mathbf{-11.09^{+0.13}_{-0.16}}$&                   2\\
       G33.92+0.11 2&                   -&                   -&                   -&                   -&                   -&                   -&                   9\\
       G34.26+0.15 0&$13.01^{+0.10}_{-0.17}$&$\mathbf{13.13^{+0.15}_{-0.23}}$&$4.91^{+0.28}_{-0.29}$&$\mathbf{5.01^{+0.31}_{-0.32}}$&$-10.38^{+0.18}_{-0.23}$&$\mathbf{-10.36^{+0.17}_{-0.23}}$&                   2\\
       G34.26+0.15 1&$11.79^{+0.09}_{-0.08}$&$\mathbf{11.83^{+0.09}_{-0.08}}$&$4.67^{+0.23}_{-0.25}$&$\mathbf{4.75^{+0.21}_{-0.24}}$&$-11.36^{+0.26}_{-0.27}$&$\mathbf{-11.40^{+0.23}_{-0.25}}$&                   2\\
       G34.26+0.15 2&$11.53^{+0.10}_{-0.10}$&$\mathbf{11.56^{+0.10}_{-0.10}}$&$4.38^{+0.30}_{-0.33}$&$\mathbf{4.48^{+0.28}_{-0.30}}$&$-11.33^{+0.36}_{-0.37}$&$\mathbf{-11.40^{+0.32}_{-0.34}}$&                   2\\
       G34.26+0.15 3&$11.63^{+0.11}_{-0.10}$&$\mathbf{11.66^{+0.10}_{-0.10}}$&$4.43^{+0.29}_{-0.32}$&$\mathbf{4.53^{+0.26}_{-0.30}}$&$-11.28^{+0.34}_{-0.35}$&$\mathbf{-11.34^{+0.31}_{-0.32}}$&                   2\\
       G34.26+0.15 4&$11.40^{+0.17}_{-0.14}$&$\mathbf{11.45^{+0.17}_{-0.16}}$&$4.87^{+0.31}_{-0.43}$&$\mathbf{4.94^{+0.30}_{-0.42}}$&$-11.95^{+0.30}_{-0.35}$&$\mathbf{-11.98^{+0.27}_{-0.34}}$&                   2\\
       G35.20-1.74 0&$12.60^{+0.08}_{-0.07}$&$\mathbf{12.65^{+0.08}_{-0.07}}$&$4.72^{+0.25}_{-0.25}$&$\mathbf{4.79^{+0.25}_{-0.26}}$&$-10.61^{+0.23}_{-0.27}$&$\mathbf{-10.62^{+0.22}_{-0.26}}$&                   2\\
       G35.20-1.74 1&$11.69^{+0.11}_{-0.10}$&$\mathbf{11.72^{+0.11}_{-0.10}}$&$4.30^{+0.33}_{-0.37}$&$\mathbf{4.41^{+0.30}_{-0.33}}$&$-11.09^{+0.39}_{-0.39}$&$\mathbf{-11.16^{+0.34}_{-0.36}}$&                   2\\
       G35.20-1.74 2&$11.97^{+0.09}_{-0.09}$&$\mathbf{12.00^{+0.09}_{-0.09}}$&$4.20^{+0.34}_{-0.35}$&$\mathbf{4.31^{+0.30}_{-0.30}}$&$-10.70^{+0.39}_{-0.41}$&$\mathbf{-10.79^{+0.34}_{-0.36}}$&                   2\\
       G35.20-1.74 3&$<11.41                 $&$\mathbf{<11.44}       $&$<4.89                 $&$\mathbf{<5.02}       $&$<-9.24                 $&$\mathbf{<-9.30}       $&                   6\\
       G35.57-0.03 0&$>12.42                 $&$\mathbf{>13.38}       $&$>4.82                 $&$\mathbf{>5.61}       $&$>-11.20                 $&$\mathbf{>-12.02}       $&                   8\\
       G35.57-0.03 1&$>11.72                 $&$\mathbf{>12.25}       $&$>4.51                 $&$\mathbf{>5.13}       $&$>-11.80                 $&$\mathbf{>-12.72}       $&                   8\\
       G35.57-0.03 2&$<11.98                 $&$\mathbf{<12.12}       $&$<4.96                 $&$\mathbf{<5.12}       $&$<-8.71                 $&$\mathbf{<-8.72}       $&                   6\\
       G35.57-0.03 3&$<12.09                 $&$\mathbf{<12.23}       $&$<4.60                 $&$\mathbf{<4.78}       $&$<-8.38                 $&$\mathbf{<-8.37}       $&                   6\\
       G35.57-0.03 4&$>11.93                 $&$\mathbf{>12.47}       $&$>4.58                 $&$\mathbf{>5.20}       $&$>-11.55                 $&$\mathbf{>-12.48}       $&                   8\\
       G35.58+0.07 0&$>12.58                 $&$\mathbf{>14.06}       $&$>4.50                 $&$\mathbf{>5.48}       $&$>-10.79                 $&$\mathbf{>-11.71}       $&                   8\\
       G35.58+0.07 1&$<12.19                 $&$\mathbf{<12.32}       $&$<4.08                 $&$\mathbf{<4.55}       $&$<-8.07                 $&$\mathbf{<-8.14}       $&                   6\\
       G35.58+0.07 2&$<11.96                 $&$\mathbf{<12.10}       $&$<4.57                 $&$\mathbf{<4.95}       $&$<-8.53                 $&$\mathbf{<-8.63}       $&                   6\\
       G35.58+0.07 3&$<12.06                 $&$\mathbf{<12.17}       $&$<4.35                 $&$\mathbf{<4.38}       $&$<-8.32                 $&$\mathbf{<-8.24}       $&                   6\\
       G37.87-0.40 0&$12.44^{+0.07}_{-0.07}$&$\mathbf{12.53^{+0.07}_{-0.09}}$&$4.86^{+0.22}_{-0.23}$&$\mathbf{4.98^{+0.21}_{-0.24}}$&$-10.90^{+0.20}_{-0.24}$&$\mathbf{-10.92^{+0.18}_{-0.21}}$&                   2\\
       G37.87-0.40 1&$11.87^{+0.10}_{-0.09}$&$\mathbf{11.95^{+0.10}_{-0.09}}$&$4.89^{+0.22}_{-0.26}$&$\mathbf{5.00^{+0.21}_{-0.24}}$&$-11.50^{+0.21}_{-0.24}$&$\mathbf{-11.53^{+0.18}_{-0.22}}$&                   2\\
       G37.87-0.40 2&$12.10^{+0.18}_{-0.26}$&$\mathbf{12.09^{+0.14}_{-0.28}}$&$3.16^{+1.15}_{-1.20}$&$\mathbf{3.79^{+1.78}_{-0.71}}$&$-9.54^{+1.32}_{-1.41}$&$\mathbf{-10.18^{+0.77}_{-2.06}}$&                   4\\
       G37.87-0.40 3&$12.05^{+0.10}_{-0.10}$&$\mathbf{12.10^{+0.10}_{-0.09}}$&$4.33^{+0.32}_{-0.33}$&$\mathbf{4.49^{+0.28}_{-0.29}}$&$-10.76^{+0.37}_{-0.39}$&$\mathbf{-10.87^{+0.31}_{-0.34}}$&                   2\\
       G37.87-0.40 4&$12.17^{+0.17}_{-0.25}$&$\mathbf{12.14^{+0.10}_{-0.09}}$&$3.36^{+1.35}_{-0.98}$&$\mathbf{4.13^{+0.35}_{-0.36}}$&$-9.67^{+1.12}_{-1.60}$&$\mathbf{-10.46^{+0.41}_{-0.42}}$&                   2\\
       G37.87-0.40 5&$11.85^{+0.18}_{-0.21}$&$\mathbf{11.88^{+0.18}_{-0.25}}$&$3.05^{+1.04}_{-1.17}$&$\mathbf{3.34^{+1.33}_{-1.03}}$&$-9.68^{+1.32}_{-1.25}$&$\mathbf{-9.94^{+1.17}_{-1.58}}$&                   4\\
       G37.87-0.40 6&$11.67^{+0.19}_{-0.23}$&$\mathbf{11.70^{+0.19}_{-0.26}}$&$3.08^{+1.07}_{-1.21}$&$\mathbf{3.33^{+1.32}_{-1.13}}$&$-9.89^{+1.36}_{-1.30}$&$\mathbf{-10.11^{+1.26}_{-1.58}}$&                   4\\
       G37.87-0.40 7&$11.81^{+0.10}_{-0.09}$&$\mathbf{11.86^{+0.10}_{-0.09}}$&$4.28^{+0.31}_{-0.33}$&$\mathbf{4.45^{+0.27}_{-0.29}}$&$-10.95^{+0.37}_{-0.38}$&$\mathbf{-11.07^{+0.32}_{-0.32}}$&                   2\\
       G37.87-0.40 8&$11.56^{+0.11}_{-0.11}$&$\mathbf{11.63^{+0.11}_{-0.10}}$&$4.67^{+0.30}_{-0.33}$&$\mathbf{4.80^{+0.29}_{-0.31}}$&$-11.59^{+0.34}_{-0.38}$&$\mathbf{-11.65^{+0.30}_{-0.34}}$&                   2\\
       G37.87-0.40 9&$11.72^{+0.13}_{-0.29}$&$\mathbf{11.74^{+0.10}_{-0.09}}$&$3.98^{+1.97}_{-0.52}$&$\mathbf{4.32^{+0.31}_{-0.32}}$&$-10.74^{+0.60}_{-2.26}$&$\mathbf{-11.06^{+0.36}_{-0.37}}$&                   2\\
       G43.89-0.78 0&$12.49^{+0.10}_{-0.09}$&$\mathbf{12.76^{+0.08}_{-0.07}}$&$4.18^{+0.34}_{-0.33}$&$\mathbf{4.68^{+0.28}_{-0.28}}$&$-10.17^{+0.37}_{-0.40}$&$\mathbf{-10.40^{+0.24}_{-0.30}}$&                   2\\
       G43.89-0.78 1&$\mathbf{11.87^{+0.09}_{-0.08}}$&$12.80^{+0.61}_{-1.00}$&$\mathbf{4.95^{+0.20}_{-0.23}}$&$6.16^{+0.96}_{-1.84}$&$\mathbf{-11.56^{+0.19}_{-0.22}}$&$-11.84^{+0.88}_{-0.42}$&                   1\\
       G45.07+0.13 0&$12.38^{+0.08}_{-0.08}$&$\mathbf{12.75^{+0.13}_{-0.20}}$&$4.96^{+0.22}_{-0.24}$&$\mathbf{5.25^{+0.25}_{-0.27}}$&$-11.06^{+0.19}_{-0.22}$&$\mathbf{-10.97^{+0.15}_{-0.18}}$&                   2\\
       G45.07+0.13 1&                   -&                   -&                   -&                   -&                   -&                   -&                   9\\
       G45.12+0.13 0&$12.02^{+0.08}_{-0.07}$&$\mathbf{12.07^{+0.08}_{-0.07}}$&$4.83^{+0.21}_{-0.21}$&$\mathbf{4.92^{+0.19}_{-0.21}}$&$-11.30^{+0.20}_{-0.23}$&$\mathbf{-11.32^{+0.19}_{-0.21}}$&                   2\\
       G45.12+0.13 1&$11.90^{+0.09}_{-0.08}$&$\mathbf{11.95^{+0.08}_{-0.08}}$&$4.74^{+0.22}_{-0.23}$&$\mathbf{4.83^{+0.21}_{-0.23}}$&$-11.32^{+0.23}_{-0.25}$&$\mathbf{-11.36^{+0.21}_{-0.23}}$&                   2\\
       G45.12+0.13 2&$11.90^{+0.10}_{-0.08}$&$\mathbf{11.93^{+0.09}_{-0.09}}$&$4.41^{+0.26}_{-0.28}$&$\mathbf{4.52^{+0.24}_{-0.26}}$&$-11.00^{+0.32}_{-0.32}$&$\mathbf{-11.06^{+0.29}_{-0.30}}$&                   2\\
       G45.12+0.13 3&$11.48^{+0.08}_{-0.09}$&$\mathbf{11.55^{+0.08}_{-0.12}}$&$5.15^{+0.19}_{-0.21}$&$\mathbf{5.23^{+0.19}_{-0.22}}$&$-12.15^{+0.16}_{-0.18}$&$\mathbf{-12.16^{+0.14}_{-0.17}}$&                   2\\
       G45.45+0.06 0&$12.62^{+0.08}_{-0.08}$&$\mathbf{12.64^{+0.07}_{-0.08}}$&$4.33^{+0.29}_{-0.31}$&$\mathbf{4.37^{+0.28}_{-0.28}}$&$-10.19^{+0.33}_{-0.35}$&$\mathbf{-10.21^{+0.30}_{-0.33}}$&                   2\\
       G45.45+0.06 1&$11.89^{+0.08}_{-0.08}$&$\mathbf{11.92^{+0.07}_{-0.07}}$&$5.00^{+0.26}_{-0.28}$&$\mathbf{5.04^{+0.26}_{-0.27}}$&$-11.59^{+0.25}_{-0.29}$&$\mathbf{-11.60^{+0.23}_{-0.28}}$&                   2\\
       G45.45+0.06 2&$<11.55                 $&$\mathbf{<11.66}       $&$<4.77                 $&$\mathbf{<5.39}       $&$<-9.04                 $&$\mathbf{<-9.43}       $&                   6\\
       G45.47+0.05 0&$12.71^{+0.09}_{-0.07}$&$\mathbf{13.48^{+0.32}_{-0.50}}$&$4.46^{+0.28}_{-0.28}$&$\mathbf{5.21^{+0.40}_{-0.34}}$&$-10.23^{+0.28}_{-0.31}$&$\mathbf{-10.21^{+0.21}_{-0.19}}$&                   2\\
       G45.47+0.05 1&$\mathbf{<11.91}       $&$<11.92                 $&$\mathbf{<5.34}       $&$<5.23                 $&$\mathbf{<-8.65}       $&$<-8.57                 $&                   5\\
       G48.61+0.02 0&$12.06^{+0.09}_{-0.09}$&$\mathbf{12.53^{+0.13}_{-0.17}}$&$4.68^{+0.23}_{-0.25}$&$\mathbf{5.29^{+0.22}_{-0.24}}$&$-11.10^{+0.25}_{-0.26}$&$\mathbf{-11.25^{+0.13}_{-0.16}}$&                   2\\
       G48.61+0.02 1&$\mathbf{<11.88}       $&$<11.91                 $&$\mathbf{<5.09}       $&$<4.47                 $&$\mathbf{<-8.61}       $&$<-8.58                 $&                   5\\
       G48.61+0.02 2&$11.54^{+0.14}_{-0.13}$&$\mathbf{11.94^{+0.18}_{-0.23}}$&$4.60^{+0.28}_{-0.39}$&$\mathbf{5.22^{+0.23}_{-0.36}}$&$-11.54^{+0.33}_{-0.33}$&$\mathbf{-11.76^{+0.14}_{-0.19}}$&                   2\\
       G50.32+0.68 0&$>11.71                 $&$\mathbf{>12.41}       $&$>4.61                 $&$\mathbf{>5.31}       $&$>-11.77                 $&$\mathbf{>-12.63}       $&                   8\\
       G60.88-0.13 0&$12.24^{+0.18}_{-0.25}$&$\mathbf{12.35^{+0.09}_{-0.09}}$&$3.20^{+1.19}_{-1.16}$&$\mathbf{4.51^{+0.27}_{-0.28}}$&$-9.44^{+1.29}_{-1.43}$&$\mathbf{-10.64^{+0.28}_{-0.31}}$&                   2\\
       G61.48+0.09 0&$>12.51                 $&$\mathbf{>12.62}       $&$>5.07                 $&$\mathbf{>5.19}       $&$>-11.27                 $&$\mathbf{>-12.33}       $&                   8\\
       G69.54-0.98 0&                   -&                   -&                   -&                   -&                   -&                   -&                  11\\
       G70.29+1.60 0&$12.21^{+0.09}_{-0.08}$&$\mathbf{12.25^{+0.08}_{-0.08}}$&$4.67^{+0.23}_{-0.23}$&$\mathbf{4.74^{+0.23}_{-0.24}}$&$-10.94^{+0.24}_{-0.26}$&$\mathbf{-10.97^{+0.23}_{-0.26}}$&                   2\\
       G70.29+1.60 1&$<11.53                 $&$\mathbf{<11.55}       $&$<4.50                 $&$\mathbf{<4.67}       $&$<-8.92                 $&$\mathbf{<-8.98}       $&                   6\\
       G70.33+1.59 0&$13.16^{+0.09}_{-0.14}$&$\mathbf{13.41^{+0.19}_{-0.35}}$&$4.64^{+0.34}_{-0.32}$&$\mathbf{4.83^{+0.39}_{-0.37}}$&$-9.96^{+0.22}_{-0.31}$&$\mathbf{-9.90^{+0.21}_{-0.26}}$&                   2\\
   IRAS 20051+3435 0&$\mathbf{12.20^{+0.11}_{-0.10}}$&$12.23^{+0.04}_{-0.05}$&$\mathbf{4.12^{+0.39}_{-0.41}}$&$4.11^{+0.21}_{-0.23}$&$\mathbf{-10.40^{+0.45}_{-0.46}}$&$-10.35^{+0.22}_{-0.22}$&                   3\\
       G41.74+0.10 0&$12.25^{+0.17}_{-0.23}$&$\mathbf{12.48^{+0.10}_{-0.09}}$&$2.99^{+0.99}_{-1.18}$&$\mathbf{4.50^{+0.28}_{-0.31}}$&$-9.23^{+1.31}_{-1.22}$&$\mathbf{-10.50^{+0.29}_{-0.32}}$&                   2\\
       G41.74+0.10 1&$\mathbf{<12.12}       $&$<0.00                 $&$\mathbf{<4.72}       $&$<0.00                 $&$\mathbf{<-8.37}       $&$<0.00                 $&                   5\\
       G41.74+0.10 2&$\mathbf{<12.18}       $&$<0.00                 $&$\mathbf{<3.21}       $&$<0.00                 $&$\mathbf{<-8.91}       $&$<0.00                 $&                   5\\
       G41.74+0.10 3&$\mathbf{<12.17}       $&$<0.00                 $&$\mathbf{<4.70}       $&$<0.00                 $&$\mathbf{<-8.32}       $&$<0.00                 $&                   5\\
       G41.74+0.10 4&$\mathbf{<12.27}       $&$<0.00                 $&$\mathbf{<4.11}       $&$<0.00                 $&$\mathbf{<-8.22}       $&$<0.00                 $&                   5\\
      IRDC 1923+13 0&$\mathbf{<11.86}       $&$<11.84                 $&$\mathbf{<5.19}       $&$<4.47                 $&$\mathbf{<-8.63}       $&$<-8.65                 $&                   5\\
      IRDC 1923+13 1&$\mathbf{<11.86}       $&$<11.77                 $&$\mathbf{<5.30}       $&$<4.58                 $&$\mathbf{<-8.70}       $&$<-8.72                 $&                   5\\
      IRDC 1923+13 2&$\mathbf{<13.29}       $&$<11.53                 $&$\mathbf{<8.00}       $&$<5.00                 $&$\mathbf{<-8.95}       $&$<-8.96                 $&                   5\\
      IRDC 1916+11 0&$\mathbf{<12.05}       $&$<12.57                 $&$\mathbf{<5.16}       $&$<5.62                 $&$\mathbf{<-8.44}       $&$<-8.46                 $&                   5\\
      IRDC 1916+11 1&$\mathbf{<12.36}       $&$<12.40                 $&$\mathbf{<4.61}       $&$<5.03                 $&$\mathbf{<-8.13}       $&$<-8.16                 $&                   5\\
      IRDC 1916+11 2&$\mathbf{<12.09}       $&$<13.77                 $&$\mathbf{<5.55}       $&$<8.00                 $&$\mathbf{<-8.68}       $&$<-8.68                 $&                   5\\
}{
\tablenotetext{a}{The values used in this paper are shown in boldface.
Uncorrected values are listed in this column.  The filling-factor corrected
values are shown for comparison in the next column even though they were not used for analysis.}
\tablenotetext{b}{The values used in this paper are shown in boldface.
Filling-factor corrected values are listed in this column.  The uncorrected
values are shown for comparison in the previous column even though they not used for analysis.}
\tablenotetext{c}{Flags:\begin{enumerate}
  \item  No filling factor correction (no FFC) is the most reliable.                                             %1: 
  \item  Filling factor correction (FFC) is the most reliable                                                    %2: 
  \item  There is an ambiguity between low density / high abundance and low abundance / high density (no FFC)    %3: 
  \item  There is an ambiguity between low density / high abundance and low abundance / high density (FFC)       %4: 
  \item  Upper Limit (No FFC)                                                                                    %5: 
  \item  Upper Limit (FFC)                                                                                       %6: 
  \item  Lower Limit (No FFC)                                                                                    %7: 
  \item  Lower Limit (FFC)                                                                                       %8: 
  \item  Unreliable estimate because of continuum / filling factor uncertainty.                                  %9: 
  \item  No limit (S/N)                                                                                         %10:
  \item  Optically Thick                                                                                        %11:
\end{enumerate}
}}



\subsection{Systematic Errors: Absorption Geometry}
There are potential systematic errors associated with geometric assumptions,
i.e. the filling factor.  There are four geometric configurations possible;
these are outlined in Table \ref{tab:systematics}.  The ``small source''
geometry (3 and 4) is technically impossible given that the CMB is always
present in these observations, but it is equivalent to the scenario in which
the small illuminating compact source (\uchii) is much brighter than the CMB in
the beam.  The second column shows the effects of applying the `true' filling
factor correction for errors 2 and 4.  For error type 3, the optical depth will
only be overestimated if the absorber is ``corrected'' to be smaller than the
background source (i.e., if a correction is applied when none should have
been).

Figure \ref{fig:ffcdependence} shows the effects of incorrect geometric
assumptions.  Type 1 and 3 errors - i.e. filling factor overcorrections -  will
result in measurements of column and abundance that are \emph{greater} than
the real values, while type 2 and 4 errors will result in column and abundance
measurements that are \emph{lower} than the real values.

Additionally, it is possible that an observation will include a beam-filling,
low-density source that will contribute negligibly in \twotwo\ line absorption
but substantially in \oneone\ absorption over most of the beam area.  This type
of error will result in an underestimate of the volume density.

Since these errors are failures of assumptions, they cannot be quantified, but
Figure \ref{fig:ffc} shows the effects of correcting for these errors to the
extent possible with the available data.

\Table{lcc}{\formaldehyde\ Geometric Systematic Errors}
{\colhead{Real Geometry} & \colhead{Assumed filling factor $= 1$} & \colhead{Assumed filling factor $< 1$} \\}
{tab:systematics}
{
1. Beam-filling source, beam-filling absorber & \tablenotemark{a}$\tau_M=\tablenotemark{b}\tau_R$ & $\tau_M> \tau_R$ \\
2. Beam-filling source, small absorber        & $\tau_M<\tau_R$ & $\tau_M= \tau_R$ \\
3. Small source, beam-filling absorber        & $\tau_M=\tau_R$ & $\tau_M>=\tau_R$ \\
4. Small source, smaller absorber             & $\tau_M<\tau_R$ & $\tau_M= \tau_R$ \\
}
{
\tablenotetext{a}{$\tau_M = $ measured optical depth}
\tablenotetext{b}{$\tau_R = $ real optical depth}
}

% \Table{ccc}{Geometric Systematic Errors}
% {\colhead{Geometry} & \colhead{filling factor $= 1$} & \colhead{filling factor $< 1$} \\}
% {tab:systematics}
% {
% 1. Beam-filling source, beam-filling absorber & \tablenotemark{a}$\tau_M=\tablenotemark{b}\tau_R$ & $\tau_M> \tau_R$ \\
% 2. Beam-filling source, small absorber        & $\tau_M<\tau_R$ & $\tau_M= \tau_R$ \\
% 3. Small source, beam-filling absorber        & $\tau_M=\tau_R$ & $\tau_M>=\tau_R$ \\
% 4. Small source, smaller absorber             & $\tau_M<\tau_R$ & $\tau_M= \tau_R$ \\
% }
% {
% \tablenotetext{a}{$\tau_M = $ measured optical depth}
% \tablenotetext{b}{$\tau_R = $ real optical depth}
% }

\subsection{RRLs}
Radio recombination lines are used to measure the velocity of the \uchii\ regions.
The recombination lines 75-77$\alpha$ were independently fitted with gaussians
because the signal-to-noise in each spectrum with a detection was high. Out of our 24
spectra, there were 21 H detections, 13 He detections, and 12 C detections;
Table \ref{tab:rrls76} shows the fitted parameters using the 76$\alpha$ lines
(75$\alpha$ and 77$\alpha$ were also measured but are not reported for
brevity).  For some of the analysis in later sections, we additionally use the
deeper and more careful RRL study by \citet{Roshi2005}, who observed 17 of our
sample in the 89-92$\alpha$ lines.  We attempted to measure carbon RRLs in the
\citet{Araya2002} spectra, who only measured hydrogen RRLs.  We detected one
carbon line in G61.48 and tentatively ($\sim2\sigma$) detected another three in
G32.80, G34.26, and G45.45; we report the low-significance detections in these
sources because of corresponding detections of C75-77$\alpha$.

We compare the central velocities of the H and C $\alpha$ lines to the
velocities of the \formaldehyde\ absorption lines on a case-by-case basis in
Figures \ref{fig:g33pt13spectrum}-\ref{fig:g61.48+0.09spectrum}.  The spectral line
profiles are used to fit the observations into the models discussed in detail
in Sections \ref{sec:lineprofiles} and \ref{sec:scenarios}.

\Table{lccccccccc}{Measured RRL 76 properties}
{
 & \multicolumn{3}{c}{H}&\multicolumn{3}{c}{He}&\multicolumn{3}{c}{C}\\
\cline{3-3} \cline{6-6} \cline{9-9} \\
\colhead{Source}&\colhead{Peak}&\colhead{Center}&\colhead{FWHM}&\colhead{Peak}&\colhead{Center}&\colhead{FWHM}&\colhead{Peak}&\colhead{Center}&\colhead{FWHM}\\
\colhead{Name}&\colhead{H76$\alpha$\tablenotemark{a}}&\colhead{H76$\alpha$}&\colhead{H76$\alpha$}&\colhead{He76$\alpha$}&\colhead{He76$\alpha$}&\colhead{He76$\alpha$}&\colhead{C76$\alpha$}&\colhead{C76$\alpha$}&\colhead{C76$\alpha$}\\
\colhead{           }&\colhead{(Jy)}&\colhead{(\kms)}&\colhead{(\kms)}&\colhead{(Jy)}&\colhead{(\kms)}&\colhead{(\kms)}&\colhead{(Jy)}&\colhead{(\kms)}&\colhead{(\kms)}\\ }
{tab:rrls76}{
         G32.80+0.19&               0.622&               15.69&               12.09&               0.066&               16.49&                9.25&               0.015&               15.40&                8.27\\
                    &             (0.001)&              (0.03)&              (0.03)&             (0.002)&              (0.36)&              (0.38)&             (0.002)&              (1.45)&              (1.64)\\
         G33.13-0.09&               0.067&               73.49&               14.10&                   -&                   -&                   -&                   -&                   -&                   -\\
                    &             (0.001)&              (0.17)&              (0.17)&                    &                    &                    &                    &                    &                    \\
         G33.92+0.11&               0.157&              101.86&               12.16&               0.013&               99.07&               13.60&                   -&                   -&                   -\\
                    &             (0.001)&              (0.07)&              (0.07)&             (0.001)&              (0.87)&              (0.87)&                    &                    &                    \\
         G34.26+0.15&               0.367&               54.68&               10.43&               0.034&               51.98&                6.54&               0.026&               59.54&                5.66\\
                    &             (0.004)&              (0.06)&              (0.09)&             (0.002)&              (0.46)&              (0.49)&             (0.002)&              (0.55)&              (0.56)\\
                    &               0.251&               37.46&               22.76&                   -&                   -&                   -&                   -&                   -&                   -\\
                    &             (0.003)&              (0.29)&              (0.12)&                    &                    &                    &                    &                    &                    \\
         G35.20-1.74&               1.016&               47.94&               10.70&               0.105&               48.26&                8.27&               0.045&               44.18&                4.05\\
                    &             (0.002)&              (0.02)&              (0.02)&             (0.002)&              (0.21)&              (0.21)&             (0.003)&              (0.33)&              (0.33)\\
         G35.57-0.03&               0.036&               52.38&               13.71&                   -&                   -&                   -&                   -&                   -&                   -\\
                    &             (0.001)&              (0.41)&              (0.41)&                    &                    &                    &                    &                    &                    \\
         G35.58+0.07&               0.044&               46.68&               10.55&               0.007&               43.15&                6.30&                   -&                   -&                   -\\
                    &             (0.001)&              (0.20)&              (0.20)&             (0.001)&              (0.94)&              (0.94)&                    &                    &                    \\
         G37.87-0.40&               0.446&               59.99&               15.47&               0.042&               60.16&               11.88&               0.018&               59.27&                7.93\\
                    &             (0.001)&              (0.08)&              (0.07)&             (0.001)&              (0.55)&              (0.55)&             (0.001)&              (0.98)&              (0.89)\\
                    &               0.049&               26.21&               10.49&                   -&                   -&                   -&                   -&                   -&                   -\\
                    &             (0.002)&              (0.52)&              (0.42)&                    &                    &                    &                    &                    &                    \\
         G41.74+0.10&               0.038&               11.46&               13.89&                   -&                   -&                   -&                   -&                   -&                   -\\
                    &             (0.001)&              (0.29)&              (0.29)&                    &                    &                    &                    &                    &                    \\
         G43.89-0.78&               0.103&               54.98&               10.83&               0.010&               54.18&                7.72&               0.007&               54.08&                0.82\\
                    &             (0.001)&              (0.08)&              (0.08)&             (0.001)&              (0.68)&              (0.68)&             (0.002)&              (0.34)&              (0.30)\\
         G45.07+0.13&               0.041&               58.22&               10.05&                   -&                   -&                   -&                   -&                   -&                   -\\
                    &             (0.004)&              (0.41)&              (0.64)&                    &                    &                    &                    &                    &                    \\
                    &               0.043&               41.57&               20.01&                   -&                   -&                   -&                   -&                   -&                   -\\
                    &             (0.003)&              (1.53)&              (0.59)&                    &                    &                    &                    &                    &                    \\
         G45.12+0.13&               0.461&               58.70&               17.42&               0.039&               59.85&               10.70&               0.023&               59.58&               12.37\\
                    &             (0.002)&              (0.08)&              (0.08)&             (0.005)&              (2.50)&              (1.62)&             (0.003)&              (4.92)&              (3.76)\\
         G45.45+0.06&               0.493&               55.38&               11.80&               0.050&               56.41&                8.02&               0.014&               63.40&               10.42\\
                    &             (0.001)&              (0.03)&              (0.03)&             (0.004)&              (0.93)&              (0.56)&             (0.002)&              (4.57)&              (3.05)\\
         G45.47+0.05&               0.040&               64.01&               14.51&                   -&                   -&                   -&                   -&                   -&                   -\\
                    &             (0.001)&              (0.41)&              (0.41)&                    &                    &                    &                    &                    &                    \\
         G48.61+0.02&               0.076&               16.77&               10.53&               0.007&               16.33&                7.73&               0.006&               19.08&                4.71\\
                    &             (0.001)&              (0.14)&              (0.14)&             (0.001)&              (1.30)&              (1.39)&             (0.001)&              (1.26)&              (1.29)\\
         G50.32+0.68&               0.034&               26.94&               10.27&                   -&                   -&                   -&                   -&                   -&                   -\\
                    &             (0.001)&              (0.27)&              (0.27)&                    &                    &                    &                    &                    &                    \\
         G60.88-0.13&               0.067&               18.30&                9.08&                   -&                   -&                   -&               0.023&               21.77&                2.54\\
                    &             (0.001)&              (0.12)&              (0.12)&                    &                    &                    &             (0.002)&              (0.19)&              (0.19)\\
         G69.54-0.98&               0.017&                3.69&               16.24&                   -&                   -&                   -&                   -&                   -&                   -\\
                    &             (0.001)&              (0.64)&              (0.64)&                    &                    &                    &                    &                    &                    \\
         G70.33+1.59&               0.343&              -19.18&               12.59&               0.032&              -20.14&               10.12&               0.025&              -21.67&                3.28\\
                    &             (0.001)&              (0.05)&              (0.05)&             (0.001)&              (0.44)&              (0.46)&             (0.002)&              (0.33)&              (0.33)\\
         G70.29+1.60&               0.545&              -26.97&               17.82&               0.042&              -26.32&               14.53&               0.032&              -24.78&                4.71\\
                    &             (0.001)&              (0.12)&              (0.09)&             (0.001)&              (0.64)&              (0.68)&             (0.002)&              (0.36)&              (0.42)\\
                    &               0.066&              -64.41&               13.12&                   -&                   -&                   -&                   -&                   -&                   -\\
                    &             (0.002)&              (0.70)&              (0.50)&                    &                    &                    &                    &                    &                    \\
         G61.48+0.09&               0.566&               25.96&               11.16&               0.046&               28.80&                7.86&               0.059&               21.27&                2.48\\
                    &             (0.001)&              (0.02)&              (0.02)&             (0.001)&              (0.24)&              (0.24)&             (0.002)&              (0.11)&              (0.11)\\
}{\tablenotetext{a}{Some H lines were fit with two gaussian components, in which case the second fit component is on the second line below.  Errors (1$\sigma$) are indicated by the numbers in parentheses on the line below the measurement.}}


%\begin{landscape}
\Table{lccccccccc}{Measured RRL properties}
{\colhead{Source Name}&\colhead{Peak H75$\alpha$\tablenotemark{a}}&\colhead{Center H75$\alpha$}&\colhead{FWHM  H75$\alpha$}&\colhead{Peak He75$\alpha$}&\colhead{Center He75$\alpha$}&\colhead{FWHM He75$\alpha$}&\colhead{Peak C75$\alpha$}&\colhead{Center C75$\alpha$}&\colhead{FWHM C75$\alpha$}\\
\colhead{           }&\colhead{(Jy)}&\colhead{(\kms)}&\colhead{(\kms)}&\colhead{(Jy)}&\colhead{(\kms)}&\colhead{(\kms)}&\colhead{(Jy)}&\colhead{(\kms)}&\colhead{(\kms)}\\ }
{tab:rrls}{
         G32.80+0.19&       0.663 (0.002)&        15.59 (0.03)&        12.10 (0.03)&       0.072 (0.002)&        15.57 (0.28)&        10.28 (0.31)&       0.021 (0.002)&        13.35 (0.73)&         5.79 (0.76)\\
         G33.13-0.09&       0.070 (0.001)&        73.28 (0.18)&        14.34 (0.18)&                   -&                   -&                   -&                   -&                   -&                   -\\
         G33.92+0.11&       0.162 (0.001)&       101.76 (0.07)&        12.25 (0.07)&       0.015 (0.001)&        95.35 (0.86)&        15.18 (0.86)&                   -&                   -&                   -\\
         G34.26+0.15&       0.528 (0.011)&        54.26 (0.18)&        12.32 (0.11)&       0.045 (0.002)&        51.83 (0.77)&         8.65 (0.71)&       0.037 (0.002)&        56.57 (0.84)&         7.41 (0.74)\\
                    &       0.210 (0.005)&        26.71 (0.94)&        17.27 (0.51)&                   -&                   -&                   -&                   -&                   -&                   -\\
         G35.20-1.74&       1.065 (0.016)&        47.89 (0.19)&        10.86 (0.19)&       0.122 (0.018)&        47.80 (1.45)&         8.38 (1.52)&       0.054 (0.021)&        45.04 (2.77)&         5.98 (2.84)\\
         G35.57-0.03&       0.035 (0.002)&        51.82 (0.88)&        15.45 (0.88)&                   -&                   -&                   -&                   -&                   -&                   -\\
         G35.58+0.07&       0.046 (0.001)&        46.52 (0.21)&        10.16 (0.21)&       0.005 (0.001)&        46.14 (2.25)&        17.01 (2.25)&                   -&                   -&                   -\\
         G37.87-0.40&       0.475 (0.001)&        60.07 (0.09)&        15.33 (0.07)&       0.054 (0.001)&        60.34 (0.76)&        11.28 (0.58)&       0.029 (0.002)&        60.86 (1.28)&        10.11 (0.95)\\
                    &       0.059 (0.002)&        28.39 (0.46)&        10.19 (0.36)&                   -&                   -&                   -&                   -&                   -&                   -\\
         G43.89-0.78&       0.106 (0.001)&        54.56 (0.09)&        10.85 (0.09)&       0.012 (0.001)&        53.92 (0.65)&         7.96 (0.65)&       0.007 (0.001)&        54.65 (0.64)&         2.69 (0.64)\\
         G45.07+0.13&       0.073 (0.006)&        55.30 (1.57)&        13.13 (0.70)&                   -&                   -&                   -&                   -&                   -&                   -\\
                    &       0.030 (0.006)&        29.77 (4.10)&        13.58 (1.88)&                   -&                   -&                   -&                   -&                   -&                   -\\
         G45.12+0.13&       0.499 (0.003)&        58.47 (0.13)&        17.17 (0.13)&       0.056 (0.004)&        56.82 (2.42)&        12.19 (2.01)&       0.039 (0.006)&        56.10 (2.84)&         9.59 (2.25)\\
         G45.45+0.06&       0.511 (0.001)&        55.29 (0.04)&        11.95 (0.04)&       0.058 (0.002)&        55.61 (0.73)&         9.09 (0.56)&       0.018 (0.002)&        57.70 (2.54)&        10.20 (2.16)\\
         G45.47+0.05&       0.040 (0.001)&        66.30 (0.43)&        14.78 (0.43)&                   -&                   -&                   -&                   -&                   -&                   -\\
         G48.61+0.02&       0.076 (0.001)&        16.78 (0.11)&         9.77 (0.11)&       0.007 (0.001)&        16.34 (1.47)&        10.58 (1.63)&       0.007 (0.001)&        17.35 (1.02)&         5.25 (1.02)\\
         G50.32+0.68&       0.034 (0.001)&        26.69 (0.21)&         8.77 (0.21)&                   -&                   -&                   -&                   -&                   -&                   -\\
         G60.88-0.13&       0.070 (0.001)&        18.39 (0.12)&         9.01 (0.12)&                   -&                   -&                   -&       0.022 (0.002)&        22.03 (0.20)&         2.48 (0.20)\\
         G61.48+0.09&       0.586 (0.005)&        25.83 (0.12)&        11.33 (0.12)&       0.050 (0.006)&        27.77 (1.25)&         8.68 (1.25)&       0.054 (0.010)&        21.78 (0.70)&         3.25 (0.70)\\
         G69.54-0.98&       0.021 (0.001)&         3.98 (0.49)&        13.63 (0.49)&                   -&                   -&                   -&                   -&                   -&                   -\\
         G70.29+1.60&       0.592 (0.002)&       -27.27 (0.16)&        17.76 (0.11)&       0.047 (0.002)&       -27.05 (0.65)&        13.56 (0.71)&       0.029 (0.003)&       -24.75 (0.49)&         4.53 (0.55)\\
                    &       0.081 (0.003)&       -63.67 (0.83)&        13.51 (0.55)&                   -&                   -&                   -&                   -&                   -&                   -\\
         G70.33+1.59&       0.358 (0.001)&       -19.32 (0.04)&        12.67 (0.04)&       0.039 (0.001)&       -21.29 (0.37)&        11.01 (0.41)&       0.024 (0.002)&       -21.98 (0.41)&         4.88 (0.41)\\
     IRAS 20051+3435&                   -&                   -&                   -&                   -&                   -&                   -&                   -&                   -&                   -\\
         G41.74+0.10&       0.042 (0.001)&        11.02 (0.29)&        13.27 (0.29)&                   -&                   -&                   -&                   -&                   -&                   -\\
        IRDC 1923+13&                   -&                   -&                   -&                   -&                   -&                   -&                   -&                   -&                   -\\
        IRDC 1916+11&                   -&                   -&                   -&                   -&                   -&                   -&                   -&                   -&                   -\\
}{\tablenotetext{a}{Some H lines were fit with two gaussian components, in which case the second fit component is on the second line below}}
\end{landscape}


% We derived electron
% temperatures using Equation 1 from \citet{Quireza2006},
% \begin{equation}
%     \left(
%     \frac{T_e^∗}{K}
%     \right)
%     = \left[ 7103.3 \left( \frac{\nu_L}{\textrm{GHz}} \right)^{1.1}
%     \left(\frac{T_C}{T_L(\textrm{H+})}\right)
%     \left(\frac{\Delta\nu(\textrm{H+})}{\kms}\right)^{-1}
%     \left(1+\frac{n(^4\textrm{He+})}{n(\textrm{H+})}\right)^{-1}
%     \right]^{0.87}
% \end{equation}
% which assumes local
% thermodynamic equilibrium, negligible pressure broadening, and a simple planar
% structure of the \ion{H}{2} region.  These assumptions are not robust and there is a
% strong dependence of the line-to-continuum ratio on the region geometry
% \citep{Lockman1978}, so the derived electron temperatures are only used for
% comparison within our own sample (see figure \ref{fig:TvT}).  
% 
% \Figure{Derived_TAveVsT110}
% {Comparison of the electron temperature derived from the 75-77$\alpha$ lines
% and the 110$\alpha$ line using the assumptions of \citet{Quireza2006}.  The
% lack of correlation demonstrates the unreliable nature of these assumptions,
% confirming the caveats brought up in \citet{Lockman1978}. [XXXX Is this
% interesting enough to keep?  If not, I should probably eliminate the above
% paragraph that refers to this figure too]}
% {fig:TvT}{0.5}{0}
% 
% The Helium/Hydrogen line ratio ranged from 0.03 to 0.11 with an average $<He/H>
% = 0.074 \pm 0.025$, consistent with previous works
% \citep[e.g.][]{Churchwell1974}.  The Carbon/Hydrogen ratio was much more
% scattered, ranging from 0.005 to 0.095.  Excluding the highest point,
% G60.88-0.13, the average is $<C/H> = 0.014 \pm 0.005$.  However, the carbon
% lines are not expected to come from the same parcels of gas as the hydrogen
% because of carbon's lower ionization energy, and there were many
% non-detections, so the carbon emission is probably not from the \ion{H}{2}
% region but may be from a local photon-dominated region.


\section{Results}
\label{sec:results}
\subsection{Derived Properties}
The average properties of the spectral line components associated with the \uchii\ regions
and the other spectral lines representing molecular clouds are
shown in Table \ref{tab:properties}.  The table includes the mean and median
only of spectral lines with both \oneone\ and \twotwo\ detections that yielded
measurements of density; upper and lower limits are not included.  The full
results are presented in Table \ref{tab:h2coderived}.

\Table{cccccccc}
{Inferred properties}
{
          &  \uchii &&& Other Lines (GMC)&&& \\
\hline
%          &  \hline    &&& \hline     &&& \\ % can't have 2 \hlines not separated by \\
Parameter & Median\tablenotemark{a} & Mean \tablenotemark{a} & RMS \tablenotemark{a} & Median \tablenotemark{b} & 
Mean \tablenotemark{b} & RMS \tablenotemark{b} & KS PTE \\}
{tab:properties}
{
log(\hh~Density) (\percc)          &       4.95 &       4.91 &       0.27 &       4.49 &       4.61 &       0.32 &      0.022 \\
log(\ortho~Column) (\persc)         &      12.59 &      12.59 &       0.44 &      11.86 &      11.83 &       0.20 &    6\ee{-6} \\
$X(\ortho)$                         &     -10.84 &     -10.80 &       0.46 &     -11.16 &     -11.26 &       0.45 &      0.028 \\
% unscaled version log(\hh Density) (\percc)      &       5.19 &       5.17 &       0.30 &       4.69 &       4.81 &       0.33 &     0.014 \\
% unscaled version log(\ortho\ Column) (\persc)   &      12.98 &      12.94 &       0.50 &      12.03 &      12.00 &       0.24 &   2.2e-06 \\
% unscaled version $X(\formaldehyde)$ \perkmspc  &     -10.82 &     -10.71 &       0.51 &     -11.22 &     -11.30 &       0.48 &    0.0039 \\
} {
\tablenotetext{a}{Spectral line components associated with \uchii\ regions}
\tablenotetext{b}{Other spectral lines (associated with line-of-sight molecular clouds)}
}

There is statistical evidence that the deepest spectral line components have higher
\formaldehyde\ column and/or abundance than the other (GMC) components (Table
\ref{tab:properties}).  It is unlikely that this difference could be caused by
underestimates of the optical depths in the GMC components (type 2 and 4
errors, see Table \ref{tab:systematics}) because the filling factor correction
should tend to cancel out these errors.  However, it is possible that, in those
cases where the \ion{H}{2} emission and the CMB emission in the beam are the
same order of magnitude, type 1 errors have occurred: the \ion{H}{2} region
absorber is much larger than the \ion{H}{2} region and a significant fraction
of the spectral line depth comes from absorption against the CMB; this error
should have little effect on the derived density (see Figure
\ref{fig:ffcdependence}) but may lead to overestimates of the derived column
density.  

Each identified Gaussian component was associated with an UC\ion{H}{2} region
if it was within 5 \kms\ of the RRL peak, since RRLs are assumed to be
generated in the UC\ion{H}{2} regions.  Any spectral lines blended with the
\uchii\ \formaldehyde\ lines were also associated with the \uchii\ region.
Other velocity components, including those without corresponding RRL
detections, were assumed to be from GMCs along the line of sight or part of the
larger cloud not directly associated with the UC\ion{H}{2} region; 29
components were associated with UC\ion{H}{2} regions and 46 were associated
with unrelated line-of-sight GMCs (Table \ref{tab:other}).

The density difference between the two populations is significant by a
Kolmogorov-Smirnov (KS) test with $\sim2\%$ probability of being drawn from the
same distribution (the `probability to exceed' or PTE in Table
\ref{tab:properties}). This result is in contradiction to the results of
\citet{Wadiak1988}, who found no significant density difference between ``warm
clouds'' and ``cold clouds'' selected and observed in the same manner (though
with larger beams).  The difference is likely because the larger beam sizes in
their study and a failure to include the continuum contribution of the CMB
(which is more substantial in a larger beam, especially at 2 cm), resulting in
a type 3 error and an underestimate of density for their ``warm clouds'' in
particular.

%The inferred column densities of \hh\ are as high as $N(\hh)=10^{24.1}
%\persc$, and the highest measured lower-limit is $N(\hh)>10^{24.0}$.  The
%densities measured ranged from $10^{4.5}$ to $>10^{6.3}$ \percc: the
%UC\ion{H}{2} regions appear to be associated with moderate to high density gas,
%one to two orders of magnitude greater than the surrounding ``clump''.  In the
%cases in which it was not possible to measure a density because of the line
%properties, it is likely that both the column and volume density are much
%higher.

The measured \hh\ densities do not display any trend with heliocentric distance
over the range 2-14 kpc, contrasing with mm-continuum surveys of star forming
regions that tend to measure lower densities at greater distances
\citep{Reid2010}.  The lack of correlation in Figure \ref{fig:densvsdist}
demonstrates the strength of the \formaldehyde\ densitometry method: the
properties of star-forming gas can be explored throughout the galaxy with
distance bias largely removed.   Similarly, no trend with
Galactocentric distance was readily apparent.

\OneColFigure{figures_chH2CO/Derived_DensityVsDistance}
{Derived density plotted against kinematic distance.  No trend is obvious, demonstrating
that the \formaldehyde\ densitometer is not biased by source distance.
Black squares represent GMCs along the line of sight; red triangles represent
UC\ion{H}{2} regions.}
{fig:densvsdist}{0.30}{0}

Densities were measured within a range $10^4\ \percc\ \lesssim n(\hh)
\lesssim10^6$ \percc\ due to sensitivity cutoffs at low densities and
thermalization of the spectral line ratio (ratio $\rightarrow$ 1) at high densities (see 
Section \ref{sec:strengthsweaknesses} for a discussion of the limitations of the
densitometer).  On the high density end, a lower limit on the density remains
interesting, as densities $n(\hh)\gtrsim10^6$ \percc\ are close to those of low-mass
protostellar cores and are a strong indication of runaway gravitational
collapse, since such high densities are rarely observed in non-star-forming
regions.  On the low density end, it should be possible to detect the \twotwo\
transition with sensitivity improvements $\sim2-10\times$, a consideration that
will govern the allocated time-on-source for future \twotwo\ observations.

\subsection{Free-free Contribution to 1.1 mm Flux Density Measurements}
It is expected that all young star-forming regions should be dust-rich and
therefore bright at 1.1 mm.  We therefore compare the BGPS 1.1 mm, GBT 2 cm,
and Arecibo 6 cm continuum measurements for sources covered by the BGPS in
Figure \ref{fig:MassVsCm}.  For a flat-spectrum \citep[$\alpha \approx -0.1$,
$\tau_{ff}<<1$;][]{rohlfs} free-free continuum source, the 2 cm flux density
should be $1.34\times$ the 1.1 mm flux density.  For an optically thick source,
$S_{1.1 mm} = 330~S_{2 cm}$.

The objects targeted in our survey include 9 of the 13 brightest ($S_{1.1 mm,40\arcsec}>1.5$
Jy) sources in the range $32<\ell<48$, and 11 of 26 with $S_{1.1mm,
40\arcsec}>1.0$ Jy. We use flux density measurements from the 40\arcsec\ apertures in the BGPS catalog
because they are most appropriate for determining peak brightness of point-like
sources \citep{Rosolowsky2010}.  Out of the sample within the BGPS survey area,
6 of 15 sources have free-free fractions of at least 30\%, but potentially much higher if 
the free-free emission is not optically thin.  Since the sample was selected from
well-known \uchii\ regions, these (rather incomplete) statistics are a warning
that most of the brightest 1.1 mm emission sources in the BGPS are likely to be
active \uchii\ regions and therefore may include a significant contribution
from free-free emission to their measured flux densities (Figure
\ref{fig:freefreefraction}).  The same warning applies to other mm-wavelength galactic
plane surveys, though the contamination should be less severe at shorter wavelengths.

\FigureTwo{figures_chH2CO/Derived_BGPSVs2cm_brightest_fit}{figures_chH2CO/Derived_BGPSVs6cm_brightest_fit}
{Bolocam 1.1 millimeter flux density versus the cm continuum flux density at 2
cm (left) and 6 cm (right).  The BGPS 1.1 mm flux density is moderately
correlated with both cm continuum measurements; the legend shows the regression
parameter.  The expectation for optically-thin free-free
emission ( $\alpha = -0.1$, dotted) and for intermediate spectral index emission
($\alpha > 0$, dashed) are shown to illustrate that some sources have
significant free-free contributions at 1.1 mm (the optically thick case is not
shown for either 2 or 6 cm because it does not fit on the plot).
The legend shows the predicted flux densities for a given spectral index
$\alpha$, the regression parameter $r$, and its likelihood $p$.  The brighter
sources are likely to be less optically thick in the free-free continuum than
the faint sources. }
{fig:MassVsCm}{1}

\OneColFigure{figures_chH2CO/Derived_FreeFreeFraction1.1mm_brightest}
{The distribution of free-free contributions to the 1.1 mm flux density
assuming the \uchii\ region is optically thin at 2 cm, $f_{ff} = (S_{2
cm}/1.34)/S_{1.1 mm}$.  While 9 sources are either dust-dominated or optically
thick at 2 cm, 6 sources have free-free contributrions of 30\% or greater.  The
other sources in the sample were missing 1.1 mm flux density measurements
because they are outside the BGPS survey area.
}
{fig:freefreefraction}{0.30}{0}

In order to evaluate the impact of this conclusion on the BGPS, we examine the
flux distribution of 6 cm continuum sources from the MAGPIS survey compared to
the BPGS in the same area, $5<\ell<42$ and $|b|<0.42$, which is the full range
of the MAGPIS survey excluding the galactic center, where the BGPS catalog
follows a different flux distribution \citep{Bally2010}.  

In Figure
\ref{fig:contfluxdistr}, we plot histograms of the MAGPIS 6 cm flux density and
the BGPS 40\arcsec\ aperture flux density along with the best-fit power-law
distribution line \footnote{The power law was fit using the python translation
of the \citet{Clauset2009} power-law fitter provided at
\url{http://code.google.com/p/agpy/wiki/PowerLaw}.  The fitter computes the
maximum likelihood value of the power-law $\alpha$ and the cutoff of the
distribution, below which a power law is no longer valid either because of
incompleteness or a change in the underlying distribution.}.  Since the 6 cm
power-law distribution is
shallower than the 1.1 mm distribution, the 6 cm sources can dominate at high
flux densities, although the power-law fit for the 6 cm sources significantly
overpredicts the highest-flux bins and therefore the power-law is not an acceptable fit
above $S_{6 cm}>1$ Jy \footnote{We have tested the consistency of the two data
sets with a low-cutoff power-law distribution by the Monte-Carlo process
described in \citet{Clauset2009}.  The BGPS 40\arcsec\ aperture flux densities
are consistent with a power-law distribution at the $p=0.64$ level, while the
MAGPIS 6 cm fluxes are inconsistent, with $p<0.001$ (where p measures the
probability that the data are drawn from a low-cutoff power-law distribution) }.
The dashed line in Figure \ref{fig:contfluxdistr} shows the best-fit power-law
distribution of the MAGPIS flux densities scaled down by
0.67, which is the expected decrement for an optically thin free-free source 
from 6 cm to 1.1 mm
(spectral index $\alpha=-0.1$).  

Figure \ref{fig:contfluxdistr}b shows the
ratio of the BGPS to the MAGPIS best-fit power-law distribution, indicating
that the free-free contamination fraction is only large ($\sim10\%$) at values
much greater than the valid range of the 6 cm power law fit, which overpredicts the
number of sources at $S_{6 cm} \approx 1 $ Jy.  However, if any
of these sources are \emph{not} optically thin at 6 cm, this fraction could be
much larger.  Additionally, these numbers only describe the sources in which
\emph{all} of the 1.1 mm flux is free-free emission; the implication remains
that a large number of 1.1 mm sources have a substantial (if not dominant)
free-free contribution.  

Finally, we emphasize that unless a large fraction of 6 cm sources are
optically thick in free-free continuum, the lower flux-density BGPS
dust-continuum sample should be negligibly contaminated by free-free emission
sources, but the brightest BGPS sources may have a significant free-free
contribution.

\FigureTwo{figures_chH2CO/fluxdistribution_6cm_1mm_fits}{figures_chH2CO/fluxdistribution_6cm_1mm_fits_ratio}
{{\it Left:} Histograms of BGPS 1.1 mm 40\arcsec\ aperture flux densities (red)
and the MAGPIS 6 cm flux densities (black), and their respective best-fit
power-law distributions ($\alpha(1.1 mm)=2.41\pm0.03$, $\alpha(6
cm)=1.72\pm0.03$).  The dashed black line shows the MAGPIS best-fit power-law
scaled down to the expected flux density at 1.1 mm assuming all sources are
optically thin.  Both distributions appear to be reasonably well-fit by
power-laws above a cutoff (presumably set by completeness), although the power-law
significantly over-predicts the number of sources with $S_{6 cm}>1$Jy.  The
histograms are binned by 0.1 dex, and while the best-fit $\alpha$ and $x_{min}$
values are independent of the binning scheme, the normalization is not.
{\it Right:} The ratio of the number of MAGPIS 6 cm sources to BGPS 1.1 mm
sources as a function of flux density for the best-fit power laws.  Only 10 1.1
mm sources are detected above 5 Jy (in 40\arcsec\ apertures), so even the
brightest detected 1.1 mm sources are not purely free-free, but they probably 
have a substantial free-free component.}
{fig:contfluxdistr}{1}

\subsection{Distances}
\label{sec:distances}
We measure a kinematic distance to each source using the \citet{Reid2009}
rotation curve.  We resolved the Kinematic Distance Ambiguity (KDA) towards
each line of sight using a variety of methods described below.  The method
in \citet{Sewilo2004} allows a resolution in favor of the far
distance for \uchii\ regions with an intervening molecular absorption line at
more positive velocities in the first Galactic quadrant.  Associations with
infrared dark clouds (IRDCs) can resolve the KDA in favor of the near distance.
We compare our KDA resolutions to \citet{Anderson2009}, with whom we agree on
all common sources except for G33.13-0.09, which we place at the far distance
based on the \citet{Sewilo2004} method.  The derived distances are listed in
Table
\ref{tab:other}.

% Distance determination was necessary to measure total dust mass using the 1.1
% mm continuum from the BGPS.  A correlation between the millimeter-derived mass
% and the 2 cm / 6 cm continuum ratio (alternately, the spectral index $\alpha$
% from the relation $S_\nu \propto \nu^\alpha$) is observed for the UC\ion{H}{2}
% regions (Figure \ref{fig:MassVsAlpha}).  Note that the Y-axis is dependent
% on distance, but the X-axis is not directly dependent on distance.  However, if
% the typical source has both a diffuse, extended \ion{H}{2} component and a
% compact \uchii\ component, the measured 6 cm flux would be expected to increase
% more with distance more than the 2 cm flux.  However, the two highest 2 cm / 6
% cm ratios are observed in very distant objects ($S_2/S_6=1.7$ for G32.80+0.19
% at 12.9 kpc and G45.07+0.13 at 7.6 kpc), suggesting that the spectral index
% measurements are not substantially affected by diffuse \ion{H}{2}
% contamination.

% are these two paragraphs in disagreement?
% The correlation implies that objects with steeper spectral indices (higher
% continuum optical depths at higher frequencies) have higher masses. Since
% ultracompact \ion{H}{2} regions are expected to evolve from high to low optical
% depths as they expand \citep{Churchwell2002}, the data are consistent with a
% scenario in which massive stars blow out their host dust clump as they evolve
% from \uchii\ regions to diffuse \ion{H}{2} regions.

%Curiously, the most dust-dominated sources have the shallowest radio spectral
%indices, which would normally indicate an optically thin free-free spectrum
%(Figure \ref{fig:MassVsAlpha}).  However, the most compact \uchii\ regions are
%expected to be the most optically thick and the youngest.  What's going on
%here? XXXX


%Previous figure shows that 1.1mm mass is probably dominated by free-free emission
% \Figure{figures_chH2CO/Derived_MassVsContinuumRatio_brightest}
% {Bolocam 1.1 millimeter-derived mass assuming T$_{dust}=40$K versus the cm
% continuum flux ratio (or spectral index) for the sources within the BGPS with
% cm continuum detections at both 2 and 6 cm.  The systematic error range for the
% mm mass estimates is $\sim0.5$ dex because of temperature, calibration, and
% distance uncertainty, but the statistical errors are relatively small
% ($\sim0.1$ dex).  An evolutionary sequence from positive to negative spectral
% index and high to low dust mass is consistent with the standard \uchii\
% evolutionary sequence \citep{Churchwell2002}.}
% {fig:MassVsAlpha}{0.5}{0}

\subsubsection{Size Estimates}
\label{sec:sizecomp}
%We compare the mass estimated from the 1.1 mm continuum with a mass inferred
%from the \formaldehyde\ measurement.  The \formaldehyde-derived density can
%be used to infer a mass given a size scale and an assumption about geometry.
%For simplicity, we assume all clumps are spherical, which immediately makes our
%measurements at best order-of-magnitude estimates.

We estimate the source size using two methods.  First, we use the VLA
measurements of \uchii\ region sizes.  As stated in Section
\ref{sec:linedepth}, the VLA size measurements are very uncertain and are
simplifications of an evidently complicated geometry.  We estimate a spherical
radius $r=\sqrt{area/\pi}$.  Second, we assume the gas traced by \formaldehyde\
and the BGPS 1.1 mm images are the same and get a `size scale'
$r=2 N_{mm}(\hh)/n(\hh)$ where $n(\hh)$ is derived from the \formaldehyde\ line
ratio.

The sizes derived from the two methods are plotted against each other in Figure
\ref{fig:sizecomp}.  The sizes estimated from the two different methods are not
well correlated and disagree by around an order of magnitude in most sources.
The disagreement could be because of poor VLA-based size estimates, substantial
1.1 mm emission from low-density gas, or incorrect dust temperature or opacity
estimates.  While additional line-of-sight GMCs could in principle contribute
to the $N/n$ size estimate, the disagreement for sources even without
associated GMCs prevents this hypothesis from fully explaining the
disagreement.  Therefore, any quantities derived from the size - i.e.
mass, which depends on $r^3$ - are even less constrained.  We therefore do not
derive any quantities dependent on the intrinsic source size.
%We therefore only
%compare the 1.1 mm derived clump mass to other derived properties.

\OneColFigure{figures_chH2CO/Derived_SizeUCHIIvsSizeNn_brightest}
{A plot of the two derived sizes discussed in Section \ref{sec:sizecomp}.  The
two size estimates are at best very weakly correlated.  Because of the
substantial disagreement between the two methods, we choose not to explore any
parameters with a strong dependence on the size.  The plotted point size
indicates the number of associated line-of-sight GMCs, which in principle could
lead to an overestimate of the $N/n$ size because of additional mass included in
the 1.1 mm continuum measurement.}
{fig:sizecomp}{0.30}{0}

%The two sources with the smallest clump masses, G41.74+0.10 and G35.58+0.07,
%fall significantly below

% The higher derived masses could also indicate higher
% temperatures at steeper spectral indices (the masses are plotted assuming a
% uniform temperature $T_{dust}=40 K$), but that explanation does not fit with
% expectations that optically thinner \ion{H}{2} regions should heat their
% surroundings more.  

%XXXX The above paragraph is a very shallow discussion of a much more involved
%topic.  However, I would prefer not to expand it much for sake of paper brevity
%(and my own personal lack of knowledge - I would need to ready for ~2 weeks on
%this topic before writing a more useful discussion).  John in particular - do
%you have any recommendations on concise additions that could improve this
%discussion?  Otherwise, anyone, recommendations on how the discussion can be
%improved so it doesn't sound quite so superficial?  I'd rather not remove it
%because it is the most direct use of the BGPS 1.1mm data.

\section{Discussion}
\label{sec:discussion}
\subsection{Comparison to extragalactic observations}
\label{sec:exgal}

%The column densities \perkmspc\ measured in the starburst galaxy sample of
%\citet{Mangum2008} are more consistent with the non-UC\ion{H}{2} sample than
%the \uchii\ sample in our small survey.  The densities measured in some of the
%starburst galaxies are quite high and comparable to the upper end of our
%UC\ion{H}{2} sample (Figure \ref{fig:exgalcolden}).
%% Ben says this is not worth discussing...
%The density measurement of the gravitational lense B0218+357 in
%\citet{Zeiger2010} $2\times10^3 < n(\hh) < 1\times10^4$, is closer to the low
%density line-of-sight clouds than the UC\ion{H}{2} regions in our sample. 
% However,
% the column measurement $2.5\times10^{13} < N_{\formaldehyde} < 8.9\times10^{13}$ is
% towards the high end of both distributions. 
% This discrepancy implies a higher 
% \formaldehyde\ abundance in the molecular clouds seen in B0218+357 than in 
% Galactic molecular clouds, which is surprising.
We compare our measured column and volume densities to a selection of starburst
galaxies from \citet{Mangum2008} in Figure \ref{fig:exgalcolden}.  All of the
extragalactic observations have much lower column densities per \kms\ than we
measure in the main lines of most \uchii\ regions, but similar volume densities.
This discrepancy can be easily explained by a difference in the area filling factor
of molecular clouds in observations of galaxies and \uchii\ regions.  In a
galaxy, the total area filling factor of molecular clouds per \kms\ (which is
similar but not identical to the volume filling factor) is likely
to be $<1$, even in extreme starbursts; although the galaxy may appear to be
uniformly filled with molecular gas in projection, at any given velocity it is
likely to have significant gaps of ionized or neutral atomic gas.  In contrast,
an \uchii\ region should be completely embedded in a molecular cloud that is
much larger than the free-free emitting continuum region, so the covering
factor of molecular gas should be $\sim1$.
 
It is therefore interesting to note that Arp 220, possibly the most extreme nearby
example of a starburst galaxy, has nearly the same column per channel as the
low end of the \uchii\ regions, suggesting that it is analagous to a scaled-up
\uchii\ region to within a factor of a few; the measured density in Arp 220 is
consistent with only the highest-density \uchii\ regions.  M82, on the other
hand, has a bright continuum background analagous to an \uchii\ region, but a
correspondingly low filling factor, implying that it consists of many 
compact but bright sources with a total filling factor 0.001-0.1.
Alternatively, the density and column measurements are consistent with M82
being dominated by quiescent GMCs, but that is unlikely given the starburst
nature of the galaxy. % porosity? 

%Arp 220's density is close to that of two \uchii\ regions, G34.26+0.15 and
%G61.48+0.09.  However, Arp 220, unlike the \uchii\ regions, shows \formaldehyde\ \oneone\
%in emission, which does not happen when observing \formaldehyde\ against a bright 
%background.  It is notable that these two sources both also have strong carbon
%RRLs.  If Arp 220 really is a scaled-up \uchii\ region, it would be reasonable to
%expect 

The gravitational lens source B0218+357 is a different scenario.  Its low
density is consistent with that of a non-star-forming GMC, while its column per
\kms\ is comparable to the Galactic sample.  This source is therefore likely to
be a sightline through a `normal' quiescent molecular cloud in its host galaxy,
similar to the narrow beam of an \uchii\ region through the Galactic disk.
\citet{Zeiger2010} note that there is a range of covering factors cited in the
literature, which can affect the measured density and column, but should not
affect the conclusion that the B0128+357 cloud's density is not consistent with
that of massive-star forming regions.  The low-density gas is detected partly
because the \citet{Zeiger2010} data are 3.5$\times$ more sensitive than ours 
with a background continuum source of similar brightness.

\OneColFigure{figures_chH2CO/Derived_DensityVsColumn_ExgalCompare_all_colored}
{Comparison of the UC\ion{H}{2} sample (blue circles are measurements, blue
triangles are lower limits on volume density with poorly constrained column
densities), the GMC sample (red squares), secondary lines associated with
\uchii\ regions (black stars) and the extragalactic sample of
\citet{Mangum2008} (green squares).  The errorbars on the
Galactic data points are excluded for clarity.  The observed galaxies have
similar densities to the Galactic \uchii\ sample, but significantly lower
column densities, suggesting that the molecular gas in these galaxies has a
filling factor $<<1$.  The lack of direct density measurements of UCHII regions
at high densities is due to the presence of a dominant background source; in Arp 220 a
direct measurement of density was possible because \formaldehyde\ was seen in
emission.}
{fig:exgalcolden}
{0.30}{0}

\subsection{Line Profiles}
\label{sec:lineprofiles}
Despite the many systematics discussed above that can affect \formaldehyde\
absorption measurements with a compact illumination source, it is possible to
directly compare the properties of gas along a given line of sight without most
of these hindering factors.  Since most of our spectra have kinematically
resolved spectral line profiles, it is possible to make many density measurements at
different velocities towards each source.  An example of this type of analysis
is shown in Figure \ref{fig:g3280densspec}.  An example demonstrating the need for
this type of analysis is shown in Figure \ref{fig:G70compare}, in which two lines
well-fit by gaussian profiles nonetheless display a density gradient because the
line centers are significantly offset; the figure also demonstrates that the offset
cannot be accounted for by any instrumental effects.

Of our sample, 18 of the 24 observed lines-of-sight had high enough signal-to-noise spectra 
(S/N$\gtrsim5$ in at least four adjacent 0.4 \kms\ channels in both lines) to
measure the density in many velocity bins.  Of these, 12 have different peak
velocities in the \oneone\ and \twotwo\ lines, indicating density gradients in
the molecular gas with velocity.  Figure \ref{fig:g3280densspec}b is an example
density-velocity plot.%, and others are shown in Section \ref{sec:sources}.  

We have classified each high S/N spectrum as {\it gradient}, {\it envelope}, or
{\it single} based on spectral line morphology.  The {\it gradient} classification was used
for gaussian or nearly gaussian lines in which the \oneone\ and \twotwo\ line
centers were offset, indicating a gradient in the density with velocity; the
color listed in the table indicates the direction of \emph{increasing} density.  The {\it
envelope} classification was used for flat profiles on the wings of deeper
gaussian lines.  The {\it single} classification was used for lines where the
\oneone\ and \twotwo\ velocities matched.  Low S/N spectra were not classified.
Classifications are given in Table \ref{tab:other}.

Of the 12 sources with density gradients, 6 show an increased density towards
the red and 5 towards the blue.  One source, G45.12+0.13, shows a slight
increase towards the red over a broad (8 \kms) velocity range, but a sharp
increase towards the blue over only 1 \kms\ and is therefore classified as {\it
other}.

Figures \ref{fig:g33pt13spectrum}-\ref{fig:g61.48+0.09spectrum} show the `main
line' (associated with the \uchii\ region) profile and the associated density,
column, and abundance velocity profiles.  The density, column, and abundance
measured for each main line via the gaussian fit technique are shown
overplotted on the profiles with blue squares.  In all cases, the gaussian fit
measurement of density is consistent with the individual channels nearby and
the gaussian fit measurements of column and abundance are consistent with the
peak column and abundance.  The consistency of adjacent velocity bins confirms
the validity of associating gaussian components in observations of whole
galaxies \citep[e.g., ][]{Mangum2008} or kpc-scale regions, since on these scales
the \oneone\ and \twotwo\ lines should be blended by kinematics to have the
same shape.

\Figure{figures_chH2CO/G32.80+0.19_densityspectrum.png}
{ % can't use \\ for newline here: results in weird tex errors
Plot of the derived parameters per velocity bin for the main line of
G32.80+0.19; the full spectrum is shown in Figure \ref{fig:specexample}.  The
density peak around 16 \kms\ is slightly redshifted of the H and C RRL velocity
centers, although the C RRLs are blueshifted of the H
RRLs, indicating that the PDR has been accelerated towards us along the line of
sight.  The blueshifted emission tail is suggestive of an outflow.  This source
cannot therefore be easily classified under any of the scenarios in Section
\ref{sec:scenarios}, but is consistent with components of scenarios 2 and 3. 
{\it a.} The spectra of G32.80+0.19.  The GBT \twotwo\ spectrum (red solid) has
been smoothed to a resolution of 0.38 \kms\ to match the Arecibo (black dashed)
spectral resolution.   Labeled vertical bars indicate the measured velocity centers
of H and C RRLs from this work, \citet{Roshi2005}, and \citet{Churchwell2010}. 
{\it b.} The measured densities in each spectral bin.  The Y-scale is in log$_{10}$
units. Error bars include a 10\% systematic uncertainty in the continuum and
therefore errors in adjacent channels are not independent.  Limits are
indicated by triangles.  Bins with no information above the 1-$\sigma$ noise
cutoff are left blank.  The increase of density towards higher velocities led us
to classify this source as a {\it red gradient} in Table \ref{tab:other}. 
{\it c.} The measured column densities per spectral bin.  Because these column
densities are derived from a large velocity gradient code, they are in
\perkmspc\ units.
{\it d.} The measured abundances per spectral bin.  The column and abundance are 
somewhat degenerate, but it is possible in some cases to place tight constraints
on the total \ortho\ column while only placing upper limits on abundance
and density.  The abundance must also be interpreted \perkmspc. 
In plots {\it b} through {\it d}, the blue square with error bars
represents the measured value from Table \ref{tab:h2coderived} using gaussian
fits to the lines.
}
{fig:g3280densspec}
{0.25}{0}

\OneColFigure{figures_chH2CO/compare_G70.3_spectra.png}
{Comparison of G70.29+1.60 (top) and G70.33+1.59 (bottom) spectra as observed
by Arecibo (black) and GBT (red/grey).  Note that in G70.29+1.60, the \twotwo\
line is shifted towards the blue of the \oneone\ line, while in G70.33+1.59 the
line centers match well.}
{fig:G70compare}
{0.3}{0}


\subsection{Comparison of RRLs and \formaldehyde\ lines}
\label{sec:scenarios}
We compare the density spectra with the fitted RRL centroids and attempt to
interpret these spectra in the context of various simple models of \ion{H}{2}
region interaction with molecular clouds.  The simple models described below
may actually be short-lived but recurring stages in the normal life cycle of a
collapsing clump that is forming massive ($M\gtrsim10 \msun$) stars
\citep{Peters2010}.

% INTERPRETATION OF $X_{\formaldehyde}$:
% Assuming a constant \formaldehyde\ abundance, the \formaldehyde\ abundance per
% \kms\ per pc should be higher for slower and higher-density gas.  It should
% therefore be particularly low in outflows.

We consider five simple models of embedded \uchii\ regions.  For each scenario,
we include a brief description of the model and an analysis of the
observational consequences in terms of C and H RRL velocities and
velocity-density structure.  We assume that the carbon RRLs are only detected
if seen in the foreground of a bright source.  This assumption is based on
predictions that C RRLs will be amplified by an order of magnitude even in the
presence of a weak background \citep{Natta1994}.   It is backed by a strong
correlation between the continuum and the C RRL intensity  \citep{Roshi2005}.
We also assume that lower-frequency RRLs will have a stronger stimulated
emission component than higher-frequency RRLs \citep{Lockman1978}.  All
\formaldehyde\ absorption is assumed to be against the \uchii\ region in this
section.  The scenario that describes a given spectrum is listed in the figure
caption for each spectrum and in Table \ref{tab:other}.
%We then include a list of observed objects that may fit the described model.

\OneColFigure{figures_chH2CO/UCHII_stationaryuniform_scenario1_spectra}
{Scenario 1: An \uchii\ region forms and begins expanding spherically
in a uniform density gas cloud.  A cartoon of the geometry seen by the observer
is shown on the left side of the figure, with arrows indicating expansion and
darkness of the gray shading indicating relative density.  The white region
around the central star is the ionized \uchii\ region.  On the right side, a
cartoon of the relative velocity and width of the RRLs and \formaldehyde\ lines
is shown.  The relative heights of the \formaldehyde\ lines is representative
of the observed density;  black is \oneone\ and red is \twotwo.  The narrow
emission line with a ? above it indicates a possible blueshifted carbon RRL;
its height has no physical meaning.  In this scenario, the hydrogen
recombination and \formaldehyde\ lines should occur at the same velocity, and
the \formaldehyde\ lines should show relatively low-density (high
\oneone/\twotwo\ ratio) and modest spectral line widths.  A blueshifted carbon
RRL may form, but is not guaranteed.}
{fig:scenario1}
{0.15}{0}

\OneColFigure{figures_chH2CO/UCHII_insideoutcollapse_scenario2_spectra}
{Scenario 2: An \uchii\ region forms from a gravitationally unstable cloud
undergoing inside-out collapse.  See Figure \ref{fig:scenario1} for a complete
description of the figure.  The highest density should correspond to the
highest-velocity infall, so the \twotwo\ line peak should be redshifted of the
\oneone\ line peak.  The hydrogen recombination line may align with a
low-density cloud but should be blueshifted of the infalling gas.  The carbon
RRL should be redshifted from the hydrogen RRL and blueshifted from the
\formaldehyde\ line.}
{fig:scenario2}
{0.15}{0}

\OneColFigure{figures_chH2CO/UCHII_outflow_scenario3_spectra}
{Scenario 3: An \uchii\ region expanding in a uniform medium ejects a bipolar
outflow.  Presumably the bipolar outflow comes from a disk-accreting source.
See Figure \ref{fig:scenario1} for a complete
description of the figure.
The outflow (indicated by the cones emitting from the central source) should
have lower column density but could have high or low volume density.  It will be 
observed as high-velocity blueshifted absorption in a line wing.  Carbon
recombination line emitting regions may be destroyed by the outflowing
material.  As in the simple scenario 1, the hydrogen recombination line should
be at the same velocity as the molecular cloud.}
{fig:scenario3}
{0.15}{0}

\OneColFigure{figures_chH2CO/UCHII_triggered_scenario4_spectra}
{Scenario 4: An \uchii\ region expanding in a uniform medium sweeps up and
accelerates material that undergoes triggered star formation.  Because the
highest-density material is the swept up material, it should be the most
blueshifted.  See Figure \ref{fig:scenario1} for a complete description of
the figure.  The orange and yellow circles are meant to indicate triggered star
formation.}
{fig:scenario4}
{0.15}{0}

\OneColFigure{figures_chH2CO/UCHII_foreground_scenario5_spectra}
{Scenario 5: An \uchii\ region is seen behind a high-density, turbulent gas
cloud.  The turbulence drives large spectral line widths, while the high density makes
the \oneone\ and \twotwo\ line depths very close.   The RRL velocity could in
principle be at any velocity relative to the foreground turbulent cloud.  See
Figure \ref{fig:scenario1} for a complete description of the figure.  In this case,
the ?'s indicate an uncertain velocity for the hydrogen RRLs; a carbon RRL is not
expected because the \ion{H}{2} region is not necessarily interacting with the
molecular gas.}
{fig:scenario5}
{0.15}{0}


\begin{itemize}
    \item SCENARIO 1: STATIC
        \\* In a uniform medium with no bulk motions (i.e., no collapse), a massive star
      ignites and generates an expanding \ion{H}{2} region.  Figure \ref{fig:scenario1}.

%OBSERVATIONAL CONSEQUENCES:
  \begin{enumerate}
    \item Lower frequency RRLs are blueshifted from higher-frequency RRLs
      because of an increased stimulated emission component \citep{Lockman1978}
    \item A carbon RRL should be seen at the same velocity as or blueshifted from
      the hydrogen RRL line center. 
    \item Molecular gas closest to the \ion{H}{2} region should have the
      highest density because of compression by the expanding \ion{H}{2}
      region.  It will be at a similar velocity or blueshifted from the H RRLs.
%-Near the PDR, the \formaldehyde\ abundance (relative to \hh) should drop
%as \formaldehyde\ is preferentially dissociated
\end{enumerate}
%POSSIBLE EXAMPLES:
%G37.87?,G45.12,G45.47,G50.32


  \item SCENARIO 2: COLLAPSE
      \\* A massive star ignites while spherically accreting from a
    molecular cloud undergoing bulk (inside-out) collapse.  Figure \ref{fig:scenario2}.

%  OBSERVATIONAL CONSEQUENCES:
  \begin{enumerate}
    \item The \formaldehyde-measured density should peak at the velocities
      most redshifted relative to the hydrogen RRLs.  Inside-out collapse
      dictates that the highest densities should be infalling at the highest
      speeds.
    \item The C RRL velocity should be between the \formaldehyde\ and H RRL
      velocity since the PDR will be decelerated by radiation and gas pressure
      from the \ion{H}{2} region
    \item Since the accreting star should be at approximately the rest
      velocity of the cloud, there should be little to no gas blueshifted from
      the RRL velocity
\end{enumerate}
%POSSIBLE EXAMPLES:
%G33.13,G33.92,G34.26,G35.58?,G45.07,G45.45,G60.88

  \item SCENARIO 3: OUTFLOW
      \\* An accreting massive star generates a massive outflow with
    a significant component along the line of sight. Figure \ref{fig:scenario3}.

%OBSERVATIONAL CONSEQUENCES:
  \begin{enumerate}
    \item Substantial low-column, low-abundance \perkms\ gas should be observed
      at velocities blue of the RRL velocities.  Densities can range from low
      to high.  Covering factors may be low.
    \item No carbon RRL is expected from the outflow, though if the flow is
      accelerated by ionization pressure a C RRL should be observed blueshifted of
      the H RRL velocity.
\end{enumerate}
%POSSIBLE EXAMPLES:
%G43.89


\item SCENARIO 4: SWEEPING
    \\* An expanding \ion{H}{2} region pushes on a low-density
  envelope, possibly triggering a new stage of star formation as in the ``collect and
  collapse'' scenario.  This scenario is similar to \#1 but with either a
  higher-density envelope or with more gas swept up (i.e., \#4 may represent
  a more evolved region).  Figure \ref{fig:scenario4}.

%OBSERVATIONAL CONSEQUENCES:
  \begin{enumerate}
    \item The hydrogen RRLs should be red of the dense gas and the carbon
      RRLs.  The expanding \ion{H}{2} region should accelerate the dense gas
      blue along the line of sight.
    \item A low-density envelope should persist at the same velocity as the
      \ion{H}{2} region
\end{enumerate}
%POSSIBLE EXAMPLES:
%G35.20,G61.48

\item SCENARIO 5: FOREGROUND CLUMP
    \\* A high density, highly turbulent or high mass and rotating clump of gas
  is in front of the \uchii\ region or surrounds it.  This physical situation may exist
  in all of the above and provides alternate explanations for any spectral line wings.  Figure \ref{fig:scenario5}.
%OBSERVATIONAL CONSEQUENCES:
  \begin{enumerate}
    \item Moderate density gas from a molecular cloud will result in high
      column but moderate density at the center velocity
    \item Wide wings of high density gas will exist both blue and redshifted of
      the highest-column point
  \end{enumerate}
%  POSSIBLE EXAMPLES: (G45.45)

\end{itemize}


%A few sources remain unclassified after examining these models.  Some of these
%show the signatures of both outflow and infall (scenarios 2 and 3): G32.80,
%G35.57-0.03, and G48.61.  Some sources, G69.54, G70.29, and G70.33, had
%peculiar profiles that did not fit under any of the described categories.
%Others, G41.74, IRAS 20051, IRDC 1923, and IRDC 1916 had inadequate S/N to
%classify their velocity/density profiles.


%Because of the degeneracy of scenario 5 with others, we do not classify any
%sources as exhibiting signatures of that scenario.  However, we emphasize
%that velocity structure in dense clouds could reproduce most of the above scenarios, 
%though some only under XXXX conditions....


% This section could use some expansion - C RRLs are only expected to be seen on
% the front side because they are caused by stimulated emission
% \citet{Roshi2005}.  Why would we see redshifted high density material and no
% CRRL in some cases, but redshifted with CRRL in others?  I could imagine
% infall + radiation -> PDR for the case with the CRRL, but infall + continuum
% but NO PDR?  \emph{What other explanations for these profiles exist?}

\subsection{The Filling Factor of Molecular Clouds}
\label{sec:gmcdensity}
We have measured the density in 19 line-of-sight molecular clouds in addition
to the 18 measurements of densities around \uchii\ regions (we only include
measurements, not limits, in these counts).  The measured density from the
\formaldehyde\ line ratio can be compared to other measures of density, e.g.
the mean molecular cloud density measured by \citet{Roman-Duval2010} from the
BU-FCRAO GRS.  It is clear from Figure \ref{fig:denshist} that the average
density in GMCs is typically $\sim2-3$ orders of magnitude lower than densities
measured in our sample of line-of-sight GMCs.

\citet{Roman-Duval2010} point out that the mean densities they measure are
significantly below the critical density of \thirteenco,
$n_{cr}=2.7\ee{3}$ \percc, indicating that they do not resolve the high-density
clumps that make up the GMCs.  Our data indicate that a typical GMC consists of
$n\sim3\ee{4}$ \percc\ gas (the median of our GMC subsample excluding upper
limits), substantially higher than the critical density of \thirteenco.  Taking
the ratio of the median density in the \citet{Roman-Duval2010} catalog to that
in our sample, we derive a volume filling factor of $5\ee{-3}$ of dense gas in
molecular clouds.  

%Our sample may be highly biased since we have selected sightlines based on the
%presence of \uchii\ regions.  However, the unassociated line-of-sight
%foreground clouds should be effectively blindly selected, since their presence
%along the line of sight has nothing to do with the background \uchii\ region.
%They should therefore represent typical sightlines through GMCs, but this
%assertion will be tested more rigorously with the large-scale galactic plane
%survey.  

\OneColFigure{figures_chH2CO/DensityHistogram}
{Histograms of the GMC and \uchii\ subsamples from our data plotted along with
the GMC-averaged densities from the \thirteenco\ \citet{Roman-Duval2010} GRS
measurements arbitrarily scaled to fit on this plot.  The measured densities in
\uchii\ regions are significantly (by a KS test) higher than densities in GMCs.
The \formaldehyde-measured densities in GMCs are 2-4 orders of magnitude higher
than volume-averaged densities of GMCs from the GRS, suggesting that GMCs
consist of very low volume-filling factor ($\sim5\ee{-3}$) high-density
($n(\hh)\sim3\ee{4}$ \percc) clumps. In Section \ref{sec:gmcdensity}, we argue
that the observed difference is most likely not a selection effect imposed by
the different gas tracers.  The
GMC upper limits shown are $3-\sigma$ upper limits, and all are consistent with
the measured GMC densities.
}
{fig:denshist}
{0.30}{0}

We measured an additional 20 upper limits towards GMCs, all of which are
consistent with high densities ($n(\hh)>10^4$ \percc), but could represent a sample
of lower density ($n(\hh)\sim10^3$ \percc) gas, in which case our `measurement' of
the cloud volume filling factor is biased to be too low.  In order to test for
this bias, we need to acquire more sensitive observations of the upper-limit
systems.  However, we continue analysis below based on the assumption that the
cloud filling factor measurement is realistic, i.e. assuming that the density upper
limit measurements have densities consistent with the other observed GMCs.
%This can be done by observing sightlines in
%which there is no uncertainty about the background source (e.g., sightlines
%with absorption only against the CMB), by resolving the emission and absorption
%(e.g., with the EVLA), or by observing an additional \formaldehyde\ line.

%If we take the volume filling factor at face value, it does a reasonable job of
%explaining the CO ``X-factor'' measurement of cloud mass.  One hypothesis
%(CITE?) for the effectiveness of the X-factor in measuring mass is that it
%effectively counts unresolved, optically thick clumps of gas.  This hypothesis
%is consistent with a very low volume filling factor of dense gas, since
%high-density ($n\gtrsim10^4$ \percc) gas clumps become optically thick in \twelveco\ at column
%densities $N\sim4\ee{20}$ or size scales $l\lesssim0.01$ pc (at these scales and densities,
%the clumps are Jeans-stable even at the CMB temperature).  However, the
%similarity of the GRS \thirteenco\ and \formaldehyde\ \oneone\ spectral line
%profiles in Figures \ref{fig:g33pt13spectrum}-42 \ref{fig:g61.48+0.09spectrum}
%suggests that there must be many (hundreds-thousands) of these clumps within
%any $\sim50\arcsec$ beam.

Can a medium with supersonic turbulence produce the same density measurements
without having to invoke high-density clumping?  Below about
$n(\hh)\approx10^5$ \percc, measurements of density in a turbulent medium are biased
towards higher densities, i.e. the densities we report may be overestimates for
GMCs since they have a median density $n(\hh)=10^{4.49}$ \percc.  For turbulent
density PDFs with logarithmic widths $\sigma_{ln(\rho)/ln(\bar{\rho})} \lesssim
1.5$, the overestimate is no more than 0.4 dex, and therefore can only bring
the filling factor up by a factor $<3$.  As discussed in Section
\ref{sec:turbulence} and Figure \ref{fig:turbcorr}, a high-density tail could
create a larger discrepancy ($\sim 0.5$ dex).  However, at the measured
densities, these are extreme upper limits on the `turbulent correction', and
therefore (gravo)turbulence alone cannot account for the measured densities.

What clumping properties are required to reproduce the observed density?  As
long as the clumps are all optically thin in the \formaldehyde\ absorption
lines, the spectral line optical depths and ratio are independent of clumping.  However,
a large number of low-density ($n(\hh)\approx10^{3.5}$ \percc) clumps optically
thick in the \oneone\ line and thin in the \twotwo\ line would appear to have a
higher density.  This phenomenon could only occur at densities
$\lesssim10^{4.5}$ \percc, where the \oneone\ absorption line is much stronger
than the \twotwo\ line, and column densities $10^{14} \persc \gtrsim
N(\ortho) \gtrsim 10^{13.5} $ \persc\ per clump (at higher columns, both
lines are optically thick; at lower columns, both lines are optically thin).
Assuming a typical \formaldehyde\ abundance (ortho+para) $X_{\formaldehyde}=10^{-9}$, the
required spherical clump radius would be $\sim0.3 $ pc, which would be
Jeans-unstable at the assumed density and temperature (40 K) and is therefore
unlikely to persist for long time periods \footnote{However, the lifetime of
such clumps in a turbulent medium in which small-scall turbulence supports the
clumps against collapse is unconstrained.}. We therefore regard a collection
of optically thick clumps in the \formaldehyde\ \oneone\ line to be 
unlikely; clumps optically thick in both lines are even less likely following
the same line of reasoning.

% Given that the \formaldehyde-derived densities are accurate independent of
% clumping, we can examine typical properties of the clumps using their typical
% observed CO properties.  We first use the observation that GMCs are generally
% optically thick in the \twelveco\ 1-0 line.  At density $n(\hh)=10^{4.5}$ \percc,
% \twelveco\ becomes optically thick at a column density $N(\hh)\approx7\ee{16}
% $ \persc, which corresponds to a size-scale $L\sim0.008$ pc assuming a CO
% abundance $X_{CO}=1\ee{-4}$.  
% %Any clumps larger than about 0.01 pc would therefore be optically thick in \twelveco.  
% The Jeans length in a 50 K, $n=10^4 $ \percc\ molecular medium is $\sim0.3$ pc,
% % (lower temperatures or higher densities would reduce this number),
% which serves as a practical upper size
% limit to the clumps at the measured densities, since the free-fall time under
% these conditions is $\sim2\ee{5}$ years (i.e., short compared to the lifetime
% of GMCs).  Clumps would have to be $\gtrsim0.4$
% pc in diameter to be optically thick in \thirteenco\ (assuming
% $X_{\thirteenco}=X_{\twelveco}/60$), so optically thick \thirteenco\ clumps
% would not be expected to survive long enough to be observed along arbitrary
% sightlines.

% The size scales resolved in our survey span a large range from 0.015 pc to 1 pc
% (Figure \ref{fig:sizecomp} shows the size scale range for the \uchii\
% subsample).  There is a wide range of line-widths even for very small
% size scales; for the larger line widths, we are likely probing GMC-scale objects
% along the whole line of sight.  For the smaller line widths, particularly those
% below $\sigma_{FWHM}=1 \kms$, the probed size scales....
% There is a density-linewidth relation... hmmm
%
% the Goodman 1998 "Transition to Coherence" takes place in this regime...
% Goodman 1998 also has upper size scales of 0.2-0.3 pc for "coherent cores" - 
% could GMCs simply be collections of coherent cores?  
% Also, the "smallest eddy that persists" is 0.008 pc - exactly equal to the
% size scale of optically thick 12CO clumps

% An intriguing result of the densities observed in GMCs is that the largest
% clumps in molecular clouds are constrained to be smaller than 0.3 pc at these
% high densities, implying an upper mass limit of about 200 \msun.  In a
% triggered star formation scenario in which pre-existing clumps are crushed, it
% would therefore be unlikely to form compact clusters or individual stars with
% masses $>200\msun$ from GMCs.

% One compelling example consistent with this explanation is Figure \ref{fig:specexample},
% in which three distinct \formaldehyde\ line components are observed to be associated with
% a single, relatively smooth \thirteenco\ line.  The three \formaldehyde\ components all 
% have densities $n\approx10^{4.5}$ \percc.  The approximate size-scale probed by the \uchii\
% sightline at the distance of these clouds is 0.7 pc, while the beamsize of the \thirteenco\
% observation is 2 pc.

The combination of the observed large spatial scales (and therefore low
volume-averaged density) of GMCs and the high densities measured along
essentially arbitrary sightlines through these GMCs suggests that GMCs are not
consistent with a purely turbulent medium with a lognormal density
distribution.  The observations also require a more substantial high-density
tail than typically seen in gravoturbulent simulations, i.e. they require a
clumpier medium.

Alternatively, it is possible that \formaldehyde\ is chemically enriched in
high-density pockets within a turbulent medium, which would imply that
\formaldehyde\ observations probe different gas than CO.  No such mechanism has
been proposed on theoretical grounds, and the timescales for enhancement would
have to be very short \citep[intermittent density enhancement occurs on
timescales much shorter than the dynamical timescale; ][]{Kritsuk2007}, so we
regard this possibility as unlikely but include it for completeness.

Another alternative is that the \thirteenco\ systematically underestimates the
mass or overestimates the volume of the cloud, resulting in an underestimate of
the cloud density.  Sub-thermal excitation of \thirteenco\ in the low-density
parts of the cloud can lead to an underestimate of the mass \citep[][Section
9.3]{Roman-Duval2010}.  Since the cloud sizes were derived using an assumed
spherical symmetry, but molecular gas is typically observed in filamentary
structures, the densities in \citet{Roman-Duval2010} are likely to be lower
limits on the mean density in the molecular gas.  While both of these factors
bring the \formaldehyde\ and \thirteenco\ densities into closer agreement, it
is difficult to quantify these effects.

%There have been many studies of the structure of molecular clouds in the
%contexts of clumpy clouds, turbulent clouds (Kritsuk, Goodman, others), fractal
%clouds (Elmegreen)... our measurement weakly suggests that GMCs do not follow
%the structure distribution predicted by purely turbulent (i.e., lognormally
%distributed) density.   XXXX This paragraph is essential if I am to include this
%section, but I need to fill it out more...

% \subsection{Feeding massive star formation}
% (remove this section unless it is completed)
% 
% Do massive stars form from `coherent cores' with subsonic turbulence like
% low-mass stars, or do they accrete their mass directly from a supersonically
% turbulent, clumpy medium like a GMC?  \citet{Goodman1998} and
% \citet{Pineda2010} have shown that in low-mass star-forming regions, a coherent
% core in which the nonthermal velocity dispersion is less than the thermal
% dispersion is the most likely predecessor to a protostar.  However, it is not
% known whether massive stars ever go through a coherent core phase.  In
% particular, for the most massive stars known with individual or multiple
% stellar systems massing $\sim100\msun$, it would be impossible to form a
% gravitationally stable coherent core at densities similar to those observed in
% B5 \citep[$n\sim10^4-10^5$]{Pineda2010}.
% 
% The line widths we observe around the \uchii\ regions are all clearly
% supersonic, as is expected since the expanding \uchii\ region should directly
% drive turbulence.  However, there is some remote chance that the \formaldehyde\ 
% densitometer could be used to search for, or rule out, such a transition.
% Because of the rarity of high-mass star-forming regions, it is likely that there
% are no analogs to the coherent core in our local neighborhood (e.g., within 1 kpc), 
% in which only the Orion and Cepheus GMCs have formed massive stars (??).  At
% greater distances, it is extremely difficult to find isolated regions in which to
% perform such a test.  
% 
% We therefore suggest that `coherence' should be sought around hypercompact
% \ion{H}{2} regions, in which a massive star has formed a very small \ion{H}{2}
% region, but has not fed back enough on its birth environment to disrupt any
% coherent structures...
% 
% In our spectra, there are hints of `sharp' density transitions with velocity.
% If similar could be observed in space....
% 
% What would an uchii in a turbulent field look like?  In a coherent field?
% 
% If massive stars form from coherent clumps, an upper mass limit is implied,
% but if not, there is no obvious mass cutoff...

% \subsection{Other Results}
% Three sources were observed with potentially optically thick \formaldehyde\
% spectra: G37.87-0.40, G45.12+0.13, and G69.54-0.98.  These are interesting
% targets for follow-up observations and are discussed in more detail in 
% Section \ref{sec:sources}.

\subsection{Strengths and Weaknesses of the \formaldehyde\ K-doublet Densitometer}
\label{sec:strengthsweaknesses}
The dynamic range of a spectral line as a tracer of a physical quantity is an
important consideration when designing an experiment.  We have demonstrated
only a modest dynamic range in density measurements using the \oneone\ and
\twotwo\ lines in absorption against bright background sources: above
$n(\hh)\approx10^{5.6}$ \percc, we are only able to set lower limits on the
density because the spectral line ratio asymptotes to 1 , and below $n(\hh)\approx10^4$
\percc, the \twotwo\ line optical depth drops to very low levels (Figure
\ref{fig:modeltau}). 

The lower limits on density at $n(\hh)\gtrsim10^{5.6}$ \percc\ can only be modestly
improved upon by using higher K-doublet transitions (e.g.  \threethree) when
observing \formaldehyde\ in absorption against bright continuum sources.
However, when observing anomolous absorption against the CMB, an additional
density diagnostic is the transition from absorption to emission at higher
densities, which expands the sensitivity of the \oneone/\twotwo\ pair to
$n(\hh)\approx10^{6.5}$ \percc.

The low-density end can only be probed by more sensitive observations of
the \twotwo\ line.  Because the \formaldehyde\ line depths become negligible
below $n(\hh)\approx10^3$ \percc, the densitometer is not a useful probe below these
densities.  However, at such low densities, even within a molecular cloud,
it is questionable whether any molecular probes are reliable, as even CO will be
underabundant in these environments \citep[e.g., ][]{Glover2010}.

As noted in \citet{Mangum2008} and \citet{Zeiger2010}, the K-doublet
densitometer has the advantage that line detection only depends on the
brightness of the background source and the gas density.  It can therefore be
used nearly independent of distance when observing clouds against the CMB or
bright illuminating background sources.  The \citet{Zeiger2010} measurements
are more sensitive than any presented in our study because of longer
integration times and the selection of a bright illuminating background source
despite their target being at a distance z=0.68.  Following this line of
reasoning, any bright synchrotron or free-free source can be used to
sensitively probe the density of a line-of-sight molecular cloud in the Galaxy.
The observation will have angular resolution limited only by the size of the
background source as long as it is much brighter than the CMB in the beam.

%Weaknesses:
%Low formaldehyde abundance at low A_V - harder to probe low densities
%Limited dynamic range
%Sensitive to continuum source size, brightness determination


\section{Conclusions}
\label{sec:h2coconclusions}
We have presented a pilot study to measure molecular gas densities in clouds
along 24 lines of sight in the \formaldehyde\ \oneone\ and \twotwo\ transitions
primarily toward \uchii\ regions .  We have shown that the \formaldehyde\
densitometer is robust within reasonable ranges of turbulent density
distributions, most cloud geometries, and different cloud clumping properties.
We have presented the methodology and discussed the errors intrinsic to the
\formaldehyde\ densitometer.

Gas volume densities were measured toward 14 of the 24 sources using the
best-fit gaussian profiles; density limits were measured for the remaning 10.
In 18 sources, it was possible to estimate the density in each 0.4 \kms-wide
channel centered on the main line.  Of these, 12 showed some sign of a density
gradient with velocity, 5 appeared to have a single-valued density (i.e. only a
single spectral line component well-fit by a gaussian), and one, G69.54-0.98,
had a spectral line optical depth that was beyond our ability to model.  
% this is not shown The measured spatial densities are uniformly higher than the
% this is not shown beam-averaged densities measured with the BGPS 1.1 mm data.

Velocity-density gradients have been used to fit 18 sources with simple
models of \uchii\ regions embedded in molecular clouds.  We have found some  
examples consistent with inside-out collapse onto the \uchii\ region,
 \uchii\ region expansion, and bulk outflow.  \formaldehyde\
absorption provides a unique probe of the physical conditions around \uchii\
regions because it is only seen in absorption against the continuum background
(for sources much brighter than the CMB), giving different constraints than mm
and sub-mm spectral lines that are seen both in front of and behind the \uchii\ region.

The measurements of serendipitously detected line-of-sight GMCs revealed
densities $\sim200$ times higher than volume-average densities measured using
\thirteenco.  The high density measured suggests that GMCs consist of many
sparsely distributed high-density clumps and have density distributions
inconsistent with the lognormal distribution predicted by supersonic turbulent
models.  The implied density distribution is also more skewed to high densities
than predicted by typical gravoturbulent simulations.  Alternatively, the 
\thirteenco-based mean densities may be lower than the mean densities within
the molecular gas either because they underestimate the mass or overestimate
the volume of GMCs.

The density measurements show that UC\ion{H}{2}s are associated with
high-density ($n(\hh)>10^{4.5}$ \percc) gas, and UC\ion{H}{2}s are associated with
higher column and volume densities than other line-of-sight molecular clouds,
in contradiction with previous results \citep{Wadiak1988}.  

The 6 cm, 2 cm, and 1.1 mm flux density measurements are strongly correlated
and in most objects in our sample the 1.1 mm flux density has a substantial ($>30\%$) contribution 
free-free continuum emission.  This result implies that the
brightest sources detected in the BGPS have significant free-free emission.
A comparison to the 6 cm MAGPIS survey suggests that the sample of 1.1 mm
sources below about 3 Jy is not significantly contaminated by pure free-free
emission sources.  
%While other surveys (e.g., ATLASGAL, HiGAL, JPS) at shorter wavelengths should be less
%affected by free-free emission because of rising dust emissivity with frequency, the
%free-free contribution in the brightest sources in these surveys may still be
%substantial.

% Observed density gradients include both increasing density toward the red and
% the blue.  These two scenarios require different physical explanations, since
% in both cases the \formaldehyde\ absorber must be in front of the continuum
% source.

% A correlation is observed between 1.1 mm -derived dust mass and 2 cm / 6 cm
% Spectral index. This correlation is consistent with the hypothesis that massive
% Stars clear out their dust envelopes as they evolve from ultracompact to
% Diffuse \ion{H}{2} regions.

Comparison of the density measurements in our sample to the starburst sample of
\citet{Mangum2008} suggest that the molecular gas volume filling factor in most of
these galaxies is small ($\sim 0.01$), but in Arp 220 it is quite high
($\gtrsim 0.1$).  The physical properties measured by \formaldehyde\ in Arp 220
are similar to those in \uchii\ regions.  Although velocity-density gradients
are observed in our sample, we argue that kinematic spectral line blending
should uphold the assumption of a single spectral line profile in galaxies as
robust for radiative transfer purposes.

\section{Acknowledgements}
We thank Jim Braatz for assistance with data acquisition and processing,
Esteban Araya for providing us with reduced data, and our referee Jeff Mangum
for a helpful and timely review.  This work was supported by the National
Science Foundation through NSF grant AST-0708403 to John Bally and AST-0707713
to Jeremy Darling.  This research has made use of the SIMBAD database, operated
at CDS, Strasbourg, France.  This research made use of pyspeckit, an
open-source spectroscopic toolkit hosted at \url{http://pyspeckit.bitbucket.org}.


% {\it Facilities:} \facility{GBT}, \facility{Arecibo}, \facility{VLA},
% \facility{FCRAO}, \facility{CSO}

% BU FCRAO GRS:
% This publication makes use of molecular line data from the Boston
% University-FCRAO Galactic Ring Survey (GRS). The GRS is a joint project of
% Boston University and Five College Radio Astronomy Observatory, funded by the
% National Science Foundation under grants AST-9800334, AST-0098562, &
% AST-0100793.
%
% MAGPIS:
% RHB and DJH acknowledge the support of the National Science Foundation under
% grants AST-05-07663 and AST-05-07598, respectively. RHB's work was supported
% in part under the auspices of the US Department of Energy by Lawrence
% Livermore National Laboratory under contract W-7405-ENG-48. DJH was also
% supported in this work by NASA grant NAG5-13062. RLW acknowledges the support
% of the Space Telescope Science Institute, which is operated by the
% Association of Universities for Research in Astronomy under NASA contract
% NAS5-26555.

%\section{Appendix: Individual Source Discussion}

%\bibliography{h2co_pilot}
% \subimport{/Users/adam/work/h2co/pilot/paper/}{source_discussion}
\ifstandalone
\bibliographystyle{apj_w_etal}  % or "siam", or "alpha", or "abbrv"
%\bibliography{thesis}      % bib database file refs.bib
\bibliography{bibdesk}      % bib database file refs.bib
\fi

\end{document}

% %\documentclass[defaultstyle,11pt]{thesis}
%\documentclass[]{report}
%\documentclass[]{article}
%\usepackage{aastex_hack}
%\usepackage{deluxetable}
\documentclass[preprint]{aastex}


%%%%%%%%%%%%%%%%%%%%%%%%%%%%%%%%%%%%%%%%%%%%%%%%%%%%%%%%%%%%%%%%
%%%%%%%%%%%  see documentation for information about  %%%%%%%%%%
%%%%%%%%%%%  the options (11pt, defaultstyle, etc.)   %%%%%%%%%%
%%%%%%%  http://www.colorado.edu/its/docs/latex/thesis/  %%%%%%%
%%%%%%%%%%%%%%%%%%%%%%%%%%%%%%%%%%%%%%%%%%%%%%%%%%%%%%%%%%%%%%%%
%		\documentclass[typewriterstyle]{thesis}
% 		\documentclass[modernstyle]{thesis}
% 		\documentclass[modernstyle,11pt]{thesis}
%	 	\documentclass[modernstyle,12pt]{thesis}

%%%%%%%%%%%%%%%%%%%%%%%%%%%%%%%%%%%%%%%%%%%%%%%%%%%%%%%%%%%%%%%%
%%%%%%%%%%%    load any packages which are needed    %%%%%%%%%%%
%%%%%%%%%%%%%%%%%%%%%%%%%%%%%%%%%%%%%%%%%%%%%%%%%%%%%%%%%%%%%%%%
\usepackage{latexsym}		% to get LASY symbols
\usepackage{graphicx}		% to insert PostScript figures
%\usepackage{deluxetable}
\usepackage{rotating}		% for sideways tables/figures
\usepackage{natbib}  % Requires natbib.sty, available from http://ads.harvard.edu/pubs/bibtex/astronat/
\usepackage{savesym}
\usepackage{amssymb}
%\savesymbol{singlespace}
\savesymbol{doublespace}
%\usepackage{wrapfig}
%\usepackage{setspace}
\usepackage{xspace}
\usepackage{color}
\usepackage{multicol}
\usepackage{mdframed}
\usepackage{url}
\usepackage{subfigure}
%\usepackage{emulateapj}
\usepackage{lscape}
\usepackage{grffile}
\usepackage{standalone}
\standalonetrue
\usepackage{import}
\usepackage[utf8]{inputenc}
\usepackage{longtable}
\usepackage{booktabs}



%%%%%%%%%%%%%%%%%%%%%%%%%%%%%%%%%%%%%%%%%%%%%%%%%%%%%%%%%%%%%%%%
%%%%%%%%%%%%       all the preamble material:       %%%%%%%%%%%%
%%%%%%%%%%%%%%%%%%%%%%%%%%%%%%%%%%%%%%%%%%%%%%%%%%%%%%%%%%%%%%%%

% \title{Star Formation in the Galaxy}
% 
% \author{Adam G.}{Ginsburg}
% 
% \otherdegrees{B.S., Rice University, 2007\\
% 	      M.S., University of Colorado, Boulder, 2009}
% 
% \degree{Doctor of Philosophy}		%  #1 {long descr.}
% 	{Ph.D., Rocket Science (ok, fine, astrophysics)}		%  #2 {short descr.}
% 
% \dept{Department of}			%  #1 {designation}
% 	{Astrophysical and Planetary Sciences}		%  #2 {name}
% 
% \advisor{Prof.}				%  #1 {title}
% 	{John Bally}			%  #2 {name}
% 
% \reader{Prof.~Jeremy Darling}		%  2nd person to sign thesis
% \readerThree{Prof.~Jason Glenn}		%  3rd person to sign thesis
% \readerFour{Prof.~Michael Shull}	%  4rd person to sign thesis
% \readerFour{Prof.~Neal Evans}	%  4rd person to sign thesis
% 
% \abstract{  \OnePageChapter	% one page only ??
% 
%     I discovered dust in space.  
% 
% 	}
% 
% 
% \dedication[Dedication]{	% NEVER use \OnePageChapter here.
% 	To 1, the second number in binary.
% 	}
% 
% \acknowledgements{	\OnePageChapter	% *MUST* BE ONLY ONE PAGE!
% 	All y'all.
% 	}
% 
% \ToCisShort	% a 1-page Table of Contents ??
% 
% \LoFisShort	% a 1-page List of Figures ??
% %	\emptyLoF	% no List of Figures at all ??
% 
% \LoTisShort	% a 1-page List of Tables ??
% %	\emptyLoT	% no List of Tables at all ??
% 
% 
% %%%%%%%%%%%%%%%%%%%%%%%%%%%%%%%%%%%%%%%%%%%%%%%%%%%%%%%%%%%%%%%%%
% %%%%%%%%%%%%%%%       BEGIN DOCUMENT...         %%%%%%%%%%%%%%%%%
% %%%%%%%%%%%%%%%%%%%%%%%%%%%%%%%%%%%%%%%%%%%%%%%%%%%%%%%%%%%%%%%%%
% 
% %%%%  footnote style; default=\arabic  (numbered 1,2,3...)
% %%%%  others:  \roman, \Roman, \alph, \Alph, \fnsymbol
% %	"\fnsymbol" uses asterisk, dagger, double-dagger, etc.
% %	\renewcommand{\thefootnote}{\fnsymbol{footnote}}
% %	\setcounter{footnote}{0}

\input{macros}		% file containing author's macro definitions

\begin{document}
% \input{introduction}
% 
% %\input{ch_iras05358}
% \input{ch_w5}
% \input{ch_h2co}
% \input{ch_h2colarge}
% \input{ch_boundhii}
% 
% %\input ch2.tex			% file with Chapter 2 contents
% 
% %%%%%%%%%%%%%%%%%%%%%%%%%%%%%%%%%%%%%%%%%%%%%%%%%%%%%%%%%%%%%%%%%%%
% %%%%%%%%%%%%%%%%%%%%%%%  Bibliography %%%%%%%%%%%%%%%%%%%%%%%%%%%%%
% %%%%%%%%%%%%%%%%%%%%%%%%%%%%%%%%%%%%%%%%%%%%%%%%%%%%%%%%%%%%%%%%%%%
% 
% \bibliographystyle{plain}	% or "siam", or "alpha", or "abbrv"
% 				% see other styles (.bst files) in
% 				% $TEXHOME/texmf/bibtex/bst
% 
% \nocite{*}		% list all refs in database, cited or not.
% 
% \bibliography{thesis}		% bib database file refs.bib
% 
% %%%%%%%%%%%%%%%%%%%%%%%%%%%%%%%%%%%%%%%%%%%%%%%%%%%%%%%%%%%%%%%%%%%
% %%%%%%%%%%%%%%%%%%%%%%%%  Appendices %%%%%%%%%%%%%%%%%%%%%%%%%%%%%%
% %%%%%%%%%%%%%%%%%%%%%%%%%%%%%%%%%%%%%%%%%%%%%%%%%%%%%%%%%%%%%%%%%%%
% 
% \appendix	% don't forget this line if you have appendices!
% 
% %\input appA.tex			% file with Appendix A contents
% %\input appB.tex			% file with Appendix B contents
% 
% %%%%%%%%%%%%%%%%%%%%%%%%%%%%%%%%%%%%%%%%%%%%%%%%%%%%%%%%%%%%%%%%%%%
% %%%%%%%%%%%%%%%%%%%%%%%%   THE END   %%%%%%%%%%%%%%%%%%%%%%%%%%%%%%
% %%%%%%%%%%%%%%%%%%%%%%%%%%%%%%%%%%%%%%%%%%%%%%%%%%%%%%%%%%%%%%%%%%%
% 
% \end{document}
% 
% 

\chapter{\formaldehyde observations of BGPS sources not previously observed with Arecibo}
\label{ch:h2colarge}


\section{Preface} 

Given our success with the simple 4-hour GBT observation of $\sim20$ sources,
it was decided that a large-scale survey of BGPS sources accessible to Arecibo
and the GBT would be productive.  We therefore selected 400 pointings in the
Arecibo-accessible declination range, 137 of which are in the outer galaxy
($172<\ell<207$) and the others in the inner galaxy ($31<\ell<65$).

The selected sources included \emph{all} peaks in the outer galaxy regions,
including the newly observed regions from the BGPS \vtwo survey.

As of the thesis defense, all of the data for the large survey has been reduced,
but the analysis and interpretation is ongoing.  We report here only initial
results from the outer galaxy component of the large survey.

Further follow-up projects based on these observations, in particular VLA
observations of the W51 star forming complex, have been approved and are
queued for observation.

The following section describes portions of the \formaldehyde observations focusing
on the outer galaxy in detail.

\subimport{/Users/adam/work/h2co/outergal/paper/}{h2co_outergal.tex}


\section{\formaldehyde Mapping}
We were lucky enough to be awarded \emph{double} the time we asked for on the GBT,
allowing us to observe large areas in mapping mode.  
Naturally, we picked the brightest and best-known regions for mapping studies.

In the inner galaxy, we mapped out an area $\sim50\arcmin\times20\arcmin$ centered
on the W51 massive star forming complex.  This region is ideally suited to study from Arecibo,
the GBT, and the VLA, since it is at declination +14 and is one of the brightest continuum
sources in the Galactic plane.  It also turns out to be the nearest proto-massive cluster
at a VLBI-parallax-measured distance of 5.1 kpc (see Chapter \ref{ch:ympc} for a discussion
of massive proto-clusters).  The simple reduction of this data is nearly complete, but 
analysis has not yet begun.

In the outer galaxy, we targeted two regions: the Sh 255 complex in Gem OB1 and
the Sh 233-IR/IRAS 05358 complex I studied for my Comps II project.  We made
small ($\sim 5\arcmin\times 5\arcmin$) maps of these objects in order to
evaluate the density profiles and determine what systematic biases may have
been present in our single-pointing observations.  These outer galaxy sources
are both at $D<2$ kpc, so our resolution is $\lesssim 0.5 pc$ and we therefore
have some marginal hope of discovering dense cores without diluting their
signal too badly.

\subsection{\formaldehyde maps of S233IR}
For the S233IR region, we were able to create a density map, but found the
surprising and counterintuitive result that the density was smallest at the
peak of the BGPS 1.1 mm emission.  The ``envelope'' is at a nearly constant density
$n\sim10^{3.5}\percc$, but the core is either at a low density (which is effectively
ruled out on other considerations) or is significantly self-absorbed.  It turns out 
that \formaldehyde \twotwo \emph{emission} fills in the absorption.

Figure \ref{fig:s233irmulti} shows the mapping results for the S233 IR region, where
`envelope' densities are measured to be $n\sim10^{3.3}-10^{3.7} \percc$, but the `core'
density is more weakly constrained to be $10^{4.5}\percc < n < 10^{5.5}\percc$ based
on the presence of \formaldehyde \twotwo emission and the absence of \oneone emission.
The lack of a direct measurement makes density profile measurement with the current
observations impossible. 

The moderate densities observed in the envelopes are nonetheless an order of
magnitude higher than typical volume-averaged GMC densities
\citep{Roman-Duval2010} as was previously noted for ordinary GMCs with
\formaldehyde detections in \citet{Ginsburg2011}. 

Perhaps most interesting is the contrast between the two BGPS sources shown 
in Figure \ref{fig:s233irmulti}.  In \citet{Ginsburg2009}, I examined primarily
S233IR, but its neighboring region G173.58+2.45 is also a well-studied proto-cluster.
Unlike S233IR, which has a B1/B2 10 \msun star that is probably still accreting,
G173.58 contains no massive stars and has an upper mass limit $M\lesssim4 \msun$
\citep{Shepherd2002}.  The \formaldehyde observations reveal that the density in this
clump is $\sim10^{3.6} \percc$, substantially less than the expected $n\sim10^5 \percc$
in the massive-star forming S233IR.

The BGPS data show this difference as well, but less strikingly.  In the BGPS
data, the inferred masses of S233IR and G173 are 840 and 190 \msun,
respectively, though lower by a factor of $\sim2$ in each when considering only
their condensed $r\lesssim0.4$ pc cores.  Their densities differ by a smaller
amount using the BGPS data and assuming spherical symmetry, with peak densities
$n\sim10^{4.1}$ in G173 and $n\sim10^{4.8}$ in S233IR.  The density difference
reinforces the conclusion drawn from the \formaldehyde data, but also show that its
density measuring power is greater, since the spherical symmetry assumption is known
to be flawed.

% S233IR:
% peak column: 6.09*2.08e22*(np.exp(13.01/20)-1)  = 1.16e23
% h2co column: "%e" % (6.09*2.08e22*(np.exp(13.01/20)-1) * 1e-9)  = 1.16e14
% total flux = 220 / 23.8 = 9.24
% mass is then 9.2 * 1.8**2 * 14.3 * (np.exp(13.01/20)-1)  = 400 msun (600 by other measures; close enough)
% radius ~30 arsec ~ 0.3pc
% n ~ 6.7e4, 10^4.8
% TOTAL flux/mass, in full aperture, is 471/23.8=19.7 Jy, or 840 msun
%
% G173:
% flux ~62 Jy/bm, /ppbeam = 2.6 Jy ~ 2.6 * 1.8**2 * 14 = 120 msun
% TOTAL 107 Jy/beam /ppbeam=4.5 -> 190 msun
% radius = 40 arcsec = 0.35 pc
% density = 120 * 2e33 / (4/3*np.pi*(0.35*3.08e18)**3) / ( 2.8 * 1.67e-24 ) = 1.3e4

\Figure{figures_chH2CO/S233IR_multipanel}
{The S233IR / IRAS 05358+3543 region and its neighbor G173.58+2.45.
{\it Top left:} The \formaldehyde density map covering densities
$10^2 \percc<n<10^5 \percc$ from grey to green.  The grey areas show
regions of low density ($n<10^3$ \percc), while green show high-density
regions ($n\gtrsim10^3.5$ \percc).  The `hole' at the peak of the contours
is likely very high density, $n>10^5$ \percc.
{\it Top center: } The \formaldehyde \oneone absorption map.
{\it Top right: } The \formaldehyde \twotwo absorption map.
Note the lack of absorption at the contour peak: this is probably \twotwo emission
filling in \twotwo absorption, indicating a high $n\gtrsim10^5$ \percc density.
{\it Bottom left: } CO 3-2 map.
{\it Bottom center: } The BGPS v2.0 1.1 mm emission map, with contours at 0.2, 1.0, and 3.0 Jy.
These contours are shown on all of the other maps for reference.
{\it Bottom right: } SO $5_6-4_5$ map.  This line has a very high critical density $n\sim3.5\ee{6}$ \percc
and an upper level energy $T_U=35$ K.
Its morphology, with a hole at the peak of the dust emission, backs the claim that the density is highest
in the area around the dust peak.}
{fig:s233irmulti}{0.2}{0}

\subsection{W51}
The W51 survey was completed in September 2011.  The data reduction process
presented unique challenges: at C-band, the entire region surveyed contains
continuum emission, so no truly suitable `off' position was found within the
survey data.  Similarly, \formaldehyde is ubiquitous across the region, so it
was necessary to `mask out' the absorption lines when building an off position.
This was done by interpolating across the line-containing region with a
polynomial fit.  

\Figure{figures_chH2CO/a2705.20120915.b0s1g0.00000_offspectra.png}
{An example of the \formaldehyde line masking procedure for building an Off
spectrum.  The line-containing regions for each polarization are shown in cyan
and purple, with the interpolated replacement in red and green.
}{fig:h2comask}{0.4}{0}

The W51 data are converted into ``optical depth'' data cubes by dividing the
integrated \formaldehyde absorption signature by the measured continuum level.
These $\tau$ cubes are then fit with the RADEX models used for other
\formaldehyde fitting.  However, there are multiple velocity components in W51,
so I used a two-component (unconstrained) fit for each pixel, which is
frequently unstable but in the case of W51 looks to have produced reasonable
results.  Note that there was \emph{no} \formaldehyde emission detected anywhere
in the W51 region.

A first interesting note is that a local cloud at $v_{lsr}\sim5 \kms$ is
detected in \formaldehyde \oneone across most of the cloud and not detected at
\twotwo, with $\tau_{\oneone}/\tau_{\twotwo} \gtrsim 3$, implying a
very low column
$N_{\formaldehyde}\sim10^{11.5}$ or $N_{\hh} \sim 10^{20.5}$.  
The cloud is seen in \thirteenco as a very weak, diffuse feature, and in HI absorption
as a very sharp, deep (self)-absorption feature.
% This density measurement
% is consistent with observations from \citet{Ginsburg2011} of high density in
% GMCs.  However, GMCs are generally thought of as being low-density clouds, so
% this result may be surprising.

%FIGURE: mcmc column vs density 

I successfully made density maps of the W51 cloud, though because the velocity
structure is quite complicated, I need to fit two components to most of the
map.  Two-component fits are never particularly stable, so it was necessary to
restrict the parameters being fitted, and even then the results aren't
perfectly reliable.  Despite those caveats, there are some reliable fits,
particularly towards the `core' of W51 Main / W51 IRS 2.  There are two
high-density components with $n\sim10^5-10^5.5$ at different velocities evident
in Figure \ref{fig:w51h2cofits}.  The southern component, centered on W51 Main,
has $v_{LSR}\sim56-59$.  The northern component, a strip going through IRS 2
and towards the west, peaks around $v_{LSR}\sim68-69$.  A 10 \kms difference
between two extremely dense components, both which are necessarily in the
foreground of the HII region, is shocking (probably, anyway, unless the sound
speed is very high).


\FigureTwo{figures_chH2CO/W51_H2CO_2parfit_v1_densityvelocity.png}
{figures_chH2CO/W51_H2CO_2parfit_v2_densityvelocity.png}
{Density and Velocity fits to the W51 Arecibo and GBT \formaldehyde 
data cubes.  The yellow regions in the top panel correspond to \oneone
detections and \twotwo nondetections, indicating upper limits $n<10^{3.8}$
(68\% confidence) or $n<10^{4.3}$ (99.7\% confidence).}
{fig:w51h2cofits}{1}

There is a large area where \oneone was detected, but \twotwo was not.  Our
sensitivity allows us to place a modest upper limit on the gas density, with
$3-\sigma$ upper limits $\lesssim10^{4.3}$ \percc (but the most likely
densities are $10^2 < n < 10^4$ \percc).  Figure \ref{fig:w51MCMCcompare} shows
a particular model for a spectrum that is especially unconstrained.  The
\oneone/\twotwo optical depth in this object is $\sim10-20$, indicating that
the volume density must be low.

\FigureTwo{figures_chH2CO/MCMC_DensColplot_67_64.png}{figures_chH2CO/spec67_64_bestfit_MCMC.png}
{Plots demonstrating upper limit fits.  The left plot shows the allowed
parameter space from MCMC sampling of the data given the RADEX model.  The
right plot shows the `best-fit' model to the optical depth spectra, which is
clearly unconstrained by the relatively insensitive \twotwo\ spectrum.  The
sensitivity in the \oneone line is better in large part because of brighter 6
cm background across the whole W51 region.  Despite the lack of constraint on the
volume density, there is a reasonably strong constraint on the column density.}
{fig:w51MCMCcompare}{1}

The molecular gas is concentrated near, but not exactly on, the bright cm
peaks.  W51 IRS2 has a massive clump of gas at 65 \kms, and W51 e2 has a
similar clump.  However, e2 also seems to have a very dense ($n>10^5 \percc$)
infalling clump.  The spectra, along with multicomponent fits, are shown in
Figure \ref{fig:w51hiispectra}.

\FigureTwo{figures_chH2CO/W51_bestfit_spec53_49_IRS2.png}{figures_chH2CO/W51_bestfit_spec53_49_W51e2.png}
{Plots of the optical depth spectra centered on W51 IRS2 (left) and W51 e2, an
ultracompact HII region (right).  IRS2 shows high-density gas with a slight
hint of infall, but otherwise a somewhat vanilla spectrum.  W51e2 has a large,
high-density red shoulder, indicating high-density gas at the most red velocity
in the system.  Because this is foreground gas, that high-density gas
\emph{must} be moving towards the \uchii region.}
{fig:w51hiispectra}{1}


\ifstandalone
\bibliographystyle{apj_w_etal}  % or "siam", or "alpha", or "abbrv"
%\bibliography{thesis}      % bib database file refs.bib
\bibliography{bibdesk}      % bib database file refs.bib
\fi

\end{document}

% %\documentclass[defaultstyle,11pt]{thesis}
%\documentclass[]{report}
%\documentclass[]{article}
%\usepackage{aastex_hack}
%\usepackage{deluxetable}
\documentclass[preprint]{aastex}


%%%%%%%%%%%%%%%%%%%%%%%%%%%%%%%%%%%%%%%%%%%%%%%%%%%%%%%%%%%%%%%%
%%%%%%%%%%%  see documentation for information about  %%%%%%%%%%
%%%%%%%%%%%  the options (11pt, defaultstyle, etc.)   %%%%%%%%%%
%%%%%%%  http://www.colorado.edu/its/docs/latex/thesis/  %%%%%%%
%%%%%%%%%%%%%%%%%%%%%%%%%%%%%%%%%%%%%%%%%%%%%%%%%%%%%%%%%%%%%%%%
%		\documentclass[typewriterstyle]{thesis}
% 		\documentclass[modernstyle]{thesis}
% 		\documentclass[modernstyle,11pt]{thesis}
%	 	\documentclass[modernstyle,12pt]{thesis}

%%%%%%%%%%%%%%%%%%%%%%%%%%%%%%%%%%%%%%%%%%%%%%%%%%%%%%%%%%%%%%%%
%%%%%%%%%%%    load any packages which are needed    %%%%%%%%%%%
%%%%%%%%%%%%%%%%%%%%%%%%%%%%%%%%%%%%%%%%%%%%%%%%%%%%%%%%%%%%%%%%
\usepackage{latexsym}		% to get LASY symbols
\usepackage{graphicx}		% to insert PostScript figures
%\usepackage{deluxetable}
\usepackage{rotating}		% for sideways tables/figures
\usepackage{natbib}  % Requires natbib.sty, available from http://ads.harvard.edu/pubs/bibtex/astronat/
\usepackage{savesym}
\usepackage{amssymb}
%\savesymbol{singlespace}
\savesymbol{doublespace}
%\usepackage{wrapfig}
%\usepackage{setspace}
\usepackage{xspace}
\usepackage{color}
\usepackage{multicol}
\usepackage{mdframed}
\usepackage{url}
\usepackage{subfigure}
%\usepackage{emulateapj}
\usepackage{lscape}
\usepackage{grffile}
\usepackage{standalone}
\standalonetrue
\usepackage{import}
\usepackage[utf8]{inputenc}
\usepackage{longtable}
\usepackage{booktabs}



%%%%%%%%%%%%%%%%%%%%%%%%%%%%%%%%%%%%%%%%%%%%%%%%%%%%%%%%%%%%%%%%
%%%%%%%%%%%%       all the preamble material:       %%%%%%%%%%%%
%%%%%%%%%%%%%%%%%%%%%%%%%%%%%%%%%%%%%%%%%%%%%%%%%%%%%%%%%%%%%%%%

% \title{Star Formation in the Galaxy}
% 
% \author{Adam G.}{Ginsburg}
% 
% \otherdegrees{B.S., Rice University, 2007\\
% 	      M.S., University of Colorado, Boulder, 2009}
% 
% \degree{Doctor of Philosophy}		%  #1 {long descr.}
% 	{Ph.D., Rocket Science (ok, fine, astrophysics)}		%  #2 {short descr.}
% 
% \dept{Department of}			%  #1 {designation}
% 	{Astrophysical and Planetary Sciences}		%  #2 {name}
% 
% \advisor{Prof.}				%  #1 {title}
% 	{John Bally}			%  #2 {name}
% 
% \reader{Prof.~Jeremy Darling}		%  2nd person to sign thesis
% \readerThree{Prof.~Jason Glenn}		%  3rd person to sign thesis
% \readerFour{Prof.~Michael Shull}	%  4rd person to sign thesis
% \readerFour{Prof.~Neal Evans}	%  4rd person to sign thesis
% 
% \abstract{  \OnePageChapter	% one page only ??
% 
%     I discovered dust in space.  
% 
% 	}
% 
% 
% \dedication[Dedication]{	% NEVER use \OnePageChapter here.
% 	To 1, the second number in binary.
% 	}
% 
% \acknowledgements{	\OnePageChapter	% *MUST* BE ONLY ONE PAGE!
% 	All y'all.
% 	}
% 
% \ToCisShort	% a 1-page Table of Contents ??
% 
% \LoFisShort	% a 1-page List of Figures ??
% %	\emptyLoF	% no List of Figures at all ??
% 
% \LoTisShort	% a 1-page List of Tables ??
% %	\emptyLoT	% no List of Tables at all ??
% 
% 
% %%%%%%%%%%%%%%%%%%%%%%%%%%%%%%%%%%%%%%%%%%%%%%%%%%%%%%%%%%%%%%%%%
% %%%%%%%%%%%%%%%       BEGIN DOCUMENT...         %%%%%%%%%%%%%%%%%
% %%%%%%%%%%%%%%%%%%%%%%%%%%%%%%%%%%%%%%%%%%%%%%%%%%%%%%%%%%%%%%%%%
% 
% %%%%  footnote style; default=\arabic  (numbered 1,2,3...)
% %%%%  others:  \roman, \Roman, \alph, \Alph, \fnsymbol
% %	"\fnsymbol" uses asterisk, dagger, double-dagger, etc.
% %	\renewcommand{\thefootnote}{\fnsymbol{footnote}}
% %	\setcounter{footnote}{0}

\input{macros}		% file containing author's macro definitions

\begin{document}
% \input{introduction}
% 
% %\input{ch_iras05358}
% \input{ch_w5}
% \input{ch_h2co}
% \input{ch_h2colarge}
% \input{ch_boundhii}
% 
% %\input ch2.tex			% file with Chapter 2 contents
% 
% %%%%%%%%%%%%%%%%%%%%%%%%%%%%%%%%%%%%%%%%%%%%%%%%%%%%%%%%%%%%%%%%%%%
% %%%%%%%%%%%%%%%%%%%%%%%  Bibliography %%%%%%%%%%%%%%%%%%%%%%%%%%%%%
% %%%%%%%%%%%%%%%%%%%%%%%%%%%%%%%%%%%%%%%%%%%%%%%%%%%%%%%%%%%%%%%%%%%
% 
% \bibliographystyle{plain}	% or "siam", or "alpha", or "abbrv"
% 				% see other styles (.bst files) in
% 				% $TEXHOME/texmf/bibtex/bst
% 
% \nocite{*}		% list all refs in database, cited or not.
% 
% \bibliography{thesis}		% bib database file refs.bib
% 
% %%%%%%%%%%%%%%%%%%%%%%%%%%%%%%%%%%%%%%%%%%%%%%%%%%%%%%%%%%%%%%%%%%%
% %%%%%%%%%%%%%%%%%%%%%%%%  Appendices %%%%%%%%%%%%%%%%%%%%%%%%%%%%%%
% %%%%%%%%%%%%%%%%%%%%%%%%%%%%%%%%%%%%%%%%%%%%%%%%%%%%%%%%%%%%%%%%%%%
% 
% \appendix	% don't forget this line if you have appendices!
% 
% %\input appA.tex			% file with Appendix A contents
% %\input appB.tex			% file with Appendix B contents
% 
% %%%%%%%%%%%%%%%%%%%%%%%%%%%%%%%%%%%%%%%%%%%%%%%%%%%%%%%%%%%%%%%%%%%
% %%%%%%%%%%%%%%%%%%%%%%%%   THE END   %%%%%%%%%%%%%%%%%%%%%%%%%%%%%%
% %%%%%%%%%%%%%%%%%%%%%%%%%%%%%%%%%%%%%%%%%%%%%%%%%%%%%%%%%%%%%%%%%%%
% 
% \end{document}
% 
% 

%%\documentclass[defaultstyle,11pt]{thesis}
%\documentclass[]{report}
%\documentclass[]{article}
%\usepackage{aastex_hack}
%\usepackage{deluxetable}
\documentclass[preprint]{aastex}


%%%%%%%%%%%%%%%%%%%%%%%%%%%%%%%%%%%%%%%%%%%%%%%%%%%%%%%%%%%%%%%%
%%%%%%%%%%%  see documentation for information about  %%%%%%%%%%
%%%%%%%%%%%  the options (11pt, defaultstyle, etc.)   %%%%%%%%%%
%%%%%%%  http://www.colorado.edu/its/docs/latex/thesis/  %%%%%%%
%%%%%%%%%%%%%%%%%%%%%%%%%%%%%%%%%%%%%%%%%%%%%%%%%%%%%%%%%%%%%%%%
%		\documentclass[typewriterstyle]{thesis}
% 		\documentclass[modernstyle]{thesis}
% 		\documentclass[modernstyle,11pt]{thesis}
%	 	\documentclass[modernstyle,12pt]{thesis}

%%%%%%%%%%%%%%%%%%%%%%%%%%%%%%%%%%%%%%%%%%%%%%%%%%%%%%%%%%%%%%%%
%%%%%%%%%%%    load any packages which are needed    %%%%%%%%%%%
%%%%%%%%%%%%%%%%%%%%%%%%%%%%%%%%%%%%%%%%%%%%%%%%%%%%%%%%%%%%%%%%
\usepackage{latexsym}		% to get LASY symbols
\usepackage{graphicx}		% to insert PostScript figures
%\usepackage{deluxetable}
\usepackage{rotating}		% for sideways tables/figures
\usepackage{natbib}  % Requires natbib.sty, available from http://ads.harvard.edu/pubs/bibtex/astronat/
\usepackage{savesym}
\usepackage{amssymb}
%\savesymbol{singlespace}
\savesymbol{doublespace}
%\usepackage{wrapfig}
%\usepackage{setspace}
\usepackage{xspace}
\usepackage{color}
\usepackage{multicol}
\usepackage{mdframed}
\usepackage{url}
\usepackage{subfigure}
%\usepackage{emulateapj}
\usepackage{lscape}
\usepackage{grffile}
\usepackage{standalone}
\standalonetrue
\usepackage{import}
\usepackage[utf8]{inputenc}
\usepackage{longtable}
\usepackage{booktabs}



%%%%%%%%%%%%%%%%%%%%%%%%%%%%%%%%%%%%%%%%%%%%%%%%%%%%%%%%%%%%%%%%
%%%%%%%%%%%%       all the preamble material:       %%%%%%%%%%%%
%%%%%%%%%%%%%%%%%%%%%%%%%%%%%%%%%%%%%%%%%%%%%%%%%%%%%%%%%%%%%%%%

% \title{Star Formation in the Galaxy}
% 
% \author{Adam G.}{Ginsburg}
% 
% \otherdegrees{B.S., Rice University, 2007\\
% 	      M.S., University of Colorado, Boulder, 2009}
% 
% \degree{Doctor of Philosophy}		%  #1 {long descr.}
% 	{Ph.D., Rocket Science (ok, fine, astrophysics)}		%  #2 {short descr.}
% 
% \dept{Department of}			%  #1 {designation}
% 	{Astrophysical and Planetary Sciences}		%  #2 {name}
% 
% \advisor{Prof.}				%  #1 {title}
% 	{John Bally}			%  #2 {name}
% 
% \reader{Prof.~Jeremy Darling}		%  2nd person to sign thesis
% \readerThree{Prof.~Jason Glenn}		%  3rd person to sign thesis
% \readerFour{Prof.~Michael Shull}	%  4rd person to sign thesis
% \readerFour{Prof.~Neal Evans}	%  4rd person to sign thesis
% 
% \abstract{  \OnePageChapter	% one page only ??
% 
%     I discovered dust in space.  
% 
% 	}
% 
% 
% \dedication[Dedication]{	% NEVER use \OnePageChapter here.
% 	To 1, the second number in binary.
% 	}
% 
% \acknowledgements{	\OnePageChapter	% *MUST* BE ONLY ONE PAGE!
% 	All y'all.
% 	}
% 
% \ToCisShort	% a 1-page Table of Contents ??
% 
% \LoFisShort	% a 1-page List of Figures ??
% %	\emptyLoF	% no List of Figures at all ??
% 
% \LoTisShort	% a 1-page List of Tables ??
% %	\emptyLoT	% no List of Tables at all ??
% 
% 
% %%%%%%%%%%%%%%%%%%%%%%%%%%%%%%%%%%%%%%%%%%%%%%%%%%%%%%%%%%%%%%%%%
% %%%%%%%%%%%%%%%       BEGIN DOCUMENT...         %%%%%%%%%%%%%%%%%
% %%%%%%%%%%%%%%%%%%%%%%%%%%%%%%%%%%%%%%%%%%%%%%%%%%%%%%%%%%%%%%%%%
% 
% %%%%  footnote style; default=\arabic  (numbered 1,2,3...)
% %%%%  others:  \roman, \Roman, \alph, \Alph, \fnsymbol
% %	"\fnsymbol" uses asterisk, dagger, double-dagger, etc.
% %	\renewcommand{\thefootnote}{\fnsymbol{footnote}}
% %	\setcounter{footnote}{0}

\input{macros}		% file containing author's macro definitions

\begin{document}
% \input{introduction}
% 
% %\input{ch_iras05358}
% \input{ch_w5}
% \input{ch_h2co}
% \input{ch_h2colarge}
% \input{ch_boundhii}
% 
% %\input ch2.tex			% file with Chapter 2 contents
% 
% %%%%%%%%%%%%%%%%%%%%%%%%%%%%%%%%%%%%%%%%%%%%%%%%%%%%%%%%%%%%%%%%%%%
% %%%%%%%%%%%%%%%%%%%%%%%  Bibliography %%%%%%%%%%%%%%%%%%%%%%%%%%%%%
% %%%%%%%%%%%%%%%%%%%%%%%%%%%%%%%%%%%%%%%%%%%%%%%%%%%%%%%%%%%%%%%%%%%
% 
% \bibliographystyle{plain}	% or "siam", or "alpha", or "abbrv"
% 				% see other styles (.bst files) in
% 				% $TEXHOME/texmf/bibtex/bst
% 
% \nocite{*}		% list all refs in database, cited or not.
% 
% \bibliography{thesis}		% bib database file refs.bib
% 
% %%%%%%%%%%%%%%%%%%%%%%%%%%%%%%%%%%%%%%%%%%%%%%%%%%%%%%%%%%%%%%%%%%%
% %%%%%%%%%%%%%%%%%%%%%%%%  Appendices %%%%%%%%%%%%%%%%%%%%%%%%%%%%%%
% %%%%%%%%%%%%%%%%%%%%%%%%%%%%%%%%%%%%%%%%%%%%%%%%%%%%%%%%%%%%%%%%%%%
% 
% \appendix	% don't forget this line if you have appendices!
% 
% %\input appA.tex			% file with Appendix A contents
% %\input appB.tex			% file with Appendix B contents
% 
% %%%%%%%%%%%%%%%%%%%%%%%%%%%%%%%%%%%%%%%%%%%%%%%%%%%%%%%%%%%%%%%%%%%
% %%%%%%%%%%%%%%%%%%%%%%%%   THE END   %%%%%%%%%%%%%%%%%%%%%%%%%%%%%%
% %%%%%%%%%%%%%%%%%%%%%%%%%%%%%%%%%%%%%%%%%%%%%%%%%%%%%%%%%%%%%%%%%%%
% 
% \end{document}
% 
% 

\chapter{Bound HII regions and Young Massive Protoclusters}
\label{ch:ympc}



\section{Preface}
During a visit from Eli Bressert, we discussed methods of identifying the
precursors to young massive clusters.  A central idea was that the primary
unbinding energy comes from ionized gas, so that if a region could remain
bound against the pressure provided by ionized gas, it would proceed to
high star formation efficiency.  This notion resulted in two papers: the theory
paper \citep{Bressert2012a} and the observational paper \citep{Ginsburg2012a}.
The observational paper, which summarizes the population of proto-YMCs discovered
in the BGPS, is reproduced here.

\subsection{Abstract}
    
We search the $\lambda=1.1$ mm Bolocam Galactic Plane Survey for clumps
containing sufficient mass to form $\sim10^4~\msun$ star clusters.
%by identifying
%compact ($r\lesssim2.5$ pc) massive ($M_{\rm clump}>10^4$ \msun) dust clumps.  
\ncandidates\ candidate massive proto-clusters  are identified in the first Galactic quadrant outside
of the central kiloparsec.  This
sample is complete to clumps with mass M$_{\rm clump}>\mmin$ and radius
$r\lesssim2.5$ pc.  The overall Galactic massive cluster formation rate is
$CFR({\rm M}_{\rm cluster}>10^4) \lesssim \CFR\  \permyr$, which is in
agreement with the rates inferred from Galactic open clusters and M31 massive
clusters.  We find that all massive proto-clusters in the first quadrant are
actively forming massive stars and place an upper limit of
$\tau_{starless}<\tsuplim$~Myr on the lifetime of the starless phase of massive
cluster formation.  If massive clusters go through a starless phase with all 
of their mass in a single clump, the lifetime of this phase is very short.




\section{Introduction}

The Milky Way contains about 150 Globular clusters (GCs) with masses of $10^4$
to over $10^6$ \msun\ and tens of thousands of open clusters containing from
100 to over $10^4$ stars.  However, young massive clusters containing
$\gtrsim10^4~\msun$ of stars are rare, with only a handful known
\citep{PortegiesZwart2010}. While no GCs have formed in the Milky Way within
the last 5 Gyr, open clusters that survive many crossing times continue to
form.   A few of these clusters have stellar masses greater than $10^4$
M$_{\odot}$ and therefore qualify as young massive clusters
\citep[YMCs;][]{PortegiesZwart2010}.   YMCs must either form from clumps having
masses greater than and sizes comparable to the final cluster  or be formed
from a larger, more diffuse reservoir, in which case massive protocluster
clumps may be rare or nonexistent  \citep{Kennicutt2012}.


%We assume an evolutionary sequence
%from `starless' to `starry' with gas present, after which the gas will be
%ejected, leaving behind a  massive cluster that will live for $\sim100$
%Myr \citep{PortegiesZwart2010}.  We distinguish `massive clusters' from
%lower-mass open clusters with $M<10^4$ \msun\ as they may form by different
%mechanisms.

%\todoeli{Check to make sure this is in agreement with your paper: }
Massive proto-clusters (MPCs) are massive clusters (M$_{\rm cluster}>10^4$ \msun)
in the process of forming from a dense gas cloud.  In \citet{Bressert2012}, we
examine the theoretical properties of MPCs: MPCs are assumed to form from
massive, cold starless clumps analagous to pre-stellar cores
\citep{Williams2000}.  In this paper, we refer to two classes of objects:
starless MPCs, which have very low luminosity and do not contain OB stars, and
MPCs, which are gas-rich but have already formed OB stars.  The only
currently known starless MPC is G0.253+0.016, which lies within the dense
central molecular zone and is subject to greater environmental stresses than
similar objects in the Galactic plane \citep{Longmore2012}.

Because massive clusters contain many massive stars, at some point during their
evolution ionization pressure will prevail over protostellar outflows as the
dominant feedback mechanism.  Other sources of feedback are less than
ionization pressure up until the first supernova explosion
\citep{Bressert2012}.  These proto-clusters must have masses
M$_{\rm clump}>{\rm M}_{*}/SFE$ \footnote{We define a star formation efficiency
$SFE={\rm M}_{\rm *,final} / {\rm M}_{\rm gas,initial}$.}, or about $3\ee{4}$ \msun\ for an assumed
SFE=30\% (an upper limit on the star formation efficiency),
confined in a radius $r\lesssim2.5$ pc, in order to remain bound against
ionization feedback.  These properties motivate our search for proto-clusters
in the Bolocam Galactic Plane Survey \citep[BGPS;][
\url{http://irsa.ipac.caltech.edu/data/BOLOCAM_GPS/}]{Aguirre2011}.


The distinction between relatively short-lived `open clusters' and long-lived
($t\gtrsim1$ Gyr) bound clusters occurs at about $10^4$ \msun
\citep{PortegiesZwart2010}.  Clusters with ${\rm M}_{\rm cluster} < 1\ee{4} \msun$~will
be destroyed by interactions with giant molecular clouds over the course of a
few hundred million years after they have dispersed their gas
\citep{Kruijssen2011}, while clusters with ${\rm M}_{\rm cluster}\gtrsim10^4 \msun$ may
survive $\gtrsim 1$ Gyr.  Closer to the Galactic center, within approximately a
kiloparsec, all clusters will be destroyed on shorter timescales by strong
tidal forces or interactions with molecular clouds.
%Because the lower-mass clusters are destroyed on shorter
%timescales, it is difficult to get a complete census of their population;
%assessing their birth rate may be the best way to determine their overall
%population.

In the Galaxy, there are few known massive clusters.
\citet{PortegiesZwart2010} catalogs a few of them, of which NGC 3603, Trumpler
14, and Westerlund 1 and 2 are the likely descendants of the objects we
investigate.  These clusters have $r_{eff} \lesssim 1$ pc, $M\sim10^4$ \msun,
and ages $t\lesssim4$ Myr.  We present a census of their ancestral analogs.

% Bound open clusters and massive clusters may predominantly form from clumps with
% gravitational escape speeds greater than the sound-speed in photo-ionized gas.
% \S\ 2 uses the BGPS to identify
% candidate dense, massive clumps which may be progenitors to young massive
% clusters. Any clump for which  M$_{\rm clump} \times {\rm SFE}(30\%) > 10^4$
% \msun\ is considered to be a young massive proto-cluster (MPC). \ncandidates\ 
% candidates are sufficiently massive to produce clusters similar to  NGC 3603.
% \S\ 4 discusses these results and \S\ 5 provides concluding remarks.


\section{Observations and Analysis}
\label{sec:ympcobservations}

\subsection{The Bolocam Galactic Plane Survey}
\begin{figure*}
    \includegraphics[width=7in]{{figures_chboundhii/candidates_galacticplot_26.0kpc_10kmsun}.png}
\caption{
\label{fig:galplot}
Plot of the massive proto-cluster (MPC) candidates
overlaid on the Galactic plane.  
%Only sources with $M>1000 \msun$ are included; above
%this cut 
%The symbol size is proportional to the log of the source mass.
The green circle represents the galactic center, and the yellow $\odot$ is the
Sun.
A 15 kpc radius disc centered on the Galactic Center indicates the approximate
extent of Galactic star formation.  The white region indicates the coverage of
the
Bolocam Galactic Plane survey and our source selection limits based on distance
and longitude.  The inner cutoff (light grey) is the nearby incompleteness
limit set by the Bolocam spatial filtering;  the catalog includes sources but
is incomplete in this region.  The red dashed circle traces the solar circle.
Blue filled circles represent initial candidates that passed the mass-cutoff
criterion $M(20K)>\mmin$; red stars are those with $M(20K) > 3\ee{4} \msun$.
In the legend, $M_4$ means mass in units of $10^4 \msun$.  
%Empty squares represent flux-cut candidates that failed the mass cutoff
%criterion.  
%The diamonds are sources from the \citet{Faundez2004} SIMBA survey
%of southern IRAS sources subject to the same cutoffs applied to the Bolocam
%data: filled (green) sources have $M>\mmin$. The central yellow star is the
%Galactic Center. The empty stars are massive clusters
%\citep{PortegiesZwart2010}.
%\todome{Add legend for size-mass}
% Done
}
\end{figure*}

The BGPS is a 1.1 mm survey of the first quadrant of the Galactic plane in the
range $-0.5 < b < 0.5$ with resolution $\sim33\arcsec$ sensitive to a maximum
spatial scale of $\sim120\arcsec$ \citep{Aguirre2011,Ginsburg2012}.  The BGPS `Bolocat' v1.0 catalog
includes sources identified by a watershed decomposition algorithm and flux
measurements within apertures of radius 20\arcsec, 40\arcsec, and 60\arcsec\
\citep{Rosolowsky2010}.

We searched the BGPS for candidate MPCs in the 1st quadrant ($6 < \ell < 90$;
5991 sources).  The inner 6 degrees of the Galaxy are excluded because physical
conditions are significantly different from those in the rest
of the galaxy  \citep{YusefZadeh2009} and the BGPS is confusion-limited in 
that region.

\subsection{Source Selection \& Completeness}
\label{sec:selection}
We identify a flux-limited sample by setting our search criteria to
include all sources with ${\rm M}_{\rm clump}>10^4$ \msun\ in a 20\arcsec\ radius out to 26 
kpc, or a physical radius of 2.5 pc at that distance.  The radius cutoff is
motivated by completeness and physical considerations: the cutoff of 26 kpc includes
the entire star forming disk in our targeted longitudes, and $r=2.5$ pc corresponds
to the radius at which a $3\ee{4}$ \msun\ mass has an escape speed $v_{esc}=10$ \kms, i.e.
ionized gas will be bound. 
The maximum radius and minimum mass imply a minimum mean density
$\bar{n}=6\times10^3~\percc$, which implies a maximum free-fall time $t_{ff}<0.65$~Myr.
%These limitations guide our analysis in Section \ref{sec:discussion}.

Using the Bolocat v1.0 catalog, we first set a flux limit on the sample by assuming
the maximum distance of $d=26$ kpc and imposing a mass cutoff of ${\rm M}_{\rm clump}\geq10^4$ \msun\ 
inside a 20\arcsec\ (2.5 pc) radius aperture.  Following equation 19 in
\citet{Aguirre2011}:
\begin{equation} 
    {\rm M}_{\rm gas}\approx 14.3 \left( e^{13.0/T_d}-1 \right)
        \left({S_\nu\over 1\; {\rm Jy}} \right)
        \left(\frac{D}{{\rm 1~kpc}}\right)^{2} \msun 
\end{equation}   
and assuming $T_{dust}=20$K, the implied flux cutoff is 1.13 Jy \footnote{As
per \citet{Rosolowsky2010}, \citet{Aguirre2011}, and \citet{Ginsburg2012}, a
factor of 1.5 calibration correction and 1.46 aperture correction are required
for the 20\arcsec\ radius aperture fluxes reported in the catalog.  These
factors have been applied to the data. }, above which \nsample\ `flux-cutoff'
candidates were selected in the Bolocat v1.0 catalog.  Cutoffs of 4.3 Jy for
the 40\arcsec\ and 10.2 Jy for the 60\arcsec\ Bolocat v1.0 apertures were used
to select more nearby candidates inside the same physical radius, but no
sources were selected based on these larger apertures.

%By measuring flux within an aperture, we are measuring the mass within a given
%radius, which means that the source may be substantially smaller than we
%assume.  The identified sources may therefore have higher escape velocities
%than the minimum $\sim10~\kms$ required.

% redundant? We applied a $M_{\rm clump} > 10^4$ \msun\ cutoff at
%the maximum distance of 17.5 kpc; we are therefore complete to a progenitor
%mass of $1\times10^4$ \msun.

The BGPS is insensitive to scales larger than 120\arcsec\
\citep[][]{Ginsburg2012}\footnote{\citet{Ginsburg2012} presents v2.0 of the
BGPS}.  As a result, the survey is incomplete below a distance $$D_{min} =
\mindist \left(\frac{r_{cluster}}{2.5 \textrm{pc}}\right) \textrm{kpc} $$ from
the Sun.  Within this radius, alternate methods must be sought to determine the
total mass within $r_{cluster} < \rcluster$ pc.  Although the sample is
incomplete for $D < \mindist$ kpc, sources that have sufficient mass despite
the 120\arcsec\ spatial filtering are included.

%We are
%able to identify some sources within this cutoff distance because they have
%enough mass in a smaller radius, but we are not complete at
%$D<\mindist$ kpc.
%Luckily, a great deal is known
%about nearby proto-clusters, and it is possible to determine the masses of many using
%alternate methods such as near and mid-infrared extinction.  These nearby star-forming
%regions are discussed in Section \ref{sec:nearcand}.

Distances to BGPS-selected candidates were determined primarily via literature
search.  Where distances were unavailable, we used velocity measurements from
\citet{Schlingman2011} and assumed the far distance for source selection.  We
then resolved the kinematic distance ambiguity towards these sources by
searching for associated near-infrared stellar extinction features from the
UKIDSS GPS \citep{Lucas2008}.
%and distances from \citet{EllsworthBowers2012} where
%literature distances were unavailable \footnote{\citet{EllsworthBowers2012} combine radial velocity measurements with 
%a variety of kinematic distance ambiguity resolution methods to measure a distance likelihood function for each source.
%We use the maximum likelihood distance from this method.}. 
Most literature distances were determined using a
rotation curve model and some method of kinematic distance ambiguity
resolution. Because the literature used different rotation curve models, there
is a $\sim10\%$ systematic error in distance resulting in a $\sim20\%$
systematic error in mass. We used the larger
40\arcsec\ radius apertures to determine the flux for sources at
$D<\middistcut$ kpc and 60\arcsec\ radius apertures for sources at $D<\mindist$
kpc (corresponding to $r<\rcluster$ pc).

% In addition, we used the radial velocity measurements from
% \citet{Schlingman2011} and \citet{Shirley2012} to determine a maximum mass for
% each source by assuming the source is at the far kinematic distance. This
% assures our survey's completeness even if the distances acquired from the above
% methods prove incorrect.  Even assuming the far distance for all flux-cutoff
% candidates, there are no additional sources with $M>3\times10^4$ \msun, though 
% there were a handful with $M>10^4 \msun$.


The masses were computed assuming a temperature $T_{dust}=20$K, opacity
$\kappa_{271.1 GHz} = 0.0114~\mathrm{cm}^2 \mathrm{g}^{-1}$, and gas-to-dust
ratio of 100  \citep{Aguirre2011} \footnote{$T_{dust}=20$K is more appropriate
for a typical pre-star-forming clump than an evolved HII-region hosting one
\citep[e.g.]{Dunham2010}. However, because we are interested in cold
progenitors as well as actively forming clusters, the selection is based on
$T_{dust}=20$K, which is more inclusive. }.  The mass estimate drops by a
factor of $2.38$ if the temperature assumed is doubled to $T_{dust}=40$K.  

\citet{Ginsburg2011} notes that significant free-free contamination, as high as
80\%, is possible for some 1.1 mm sources, so the selected candidates may prove
to be more moderate-mass and evolved proto-clusters.  We used the NRAO VLA
Archive Survey \citep[NVAS;][]{Crossley2008} to estimate the free-free
contamination for the sample.  For most sources, the free-free contamination
inferred from the VLA observations is small ($<20\%$), but for a subset the
contamination was $\sim20-35\%$ assuming that the free-free emission is
optically thin.  Corrected masses using the measured free-free contamination
and higher dust temperatures are listed in Table \ref{tab:candidates}; these
are reasonable lower limits on the total mass of these regions.  All of the
contamination estimates are technically lower limits both because of the
assumption that the free-free emission is optically thin and because the VLA
filters out large-scale flux.  However, in most cases, the emission is likely
to be dominated by optically thin emission \citep[evolved HII regions tend to
be optically thin and bright, while compact HII regions are optically thick but
relatively faint;][]{Keto2002} and for most sources VLA C or D-array
observations were used, and at 3.6 and 6 cm the largest angular scale recovered
is 180-300 \arcsec, greater than the largest angular scale in the BGPS.  

Applying a cutoff of M$_{\rm clump} > 10^4$ \msun\ left \ncandidates\
protocluster candidates out of the original \nsample.  The more stringent cut
M$_{\rm clump} > 10^4 / SFE \approx 3\ee{4}$ \msun\ leaves only \nMPC\ MPCs . 
% All of the \nsample\
% `flux-cutoff' candidates with $M>10^3$ \msun\ are shown in \figref{fig:galplot}
% providing context of their location in the Galaxy. 

The final candidate list contains only sources with $M(20K)>10^4 \msun$ (the
completeness limit; see Table \ref{tab:candidates}).  The table lists
their physical properties, their literature distance, their mass (assuming $T_{dust}=20
\textrm{~and~} 40 K$ and a free-free subtracted lower-limit) ,
%\todome{Discuss varying dust opacity?  Martin et al 2011} 
% Ignored.  Unnecessary.
and their inferred escape speed ($v_{esc} = \sqrt{2 G M(20K) / r}$) assuming a
radius equal to the aperture size at that distance.  The table also includes
measurements of the IRAS luminosity in the 60 and 100 \um\ bands within the
source aperture.
%The literature search
%revealed that all candidates are known massive-star-forming regions.

\subsection{Source Separation}
These \ncandidates\ candidates include some overlapping sources.
There are two clumps in W51 separated by about 1.5 pc and 4.5 \kms\ along the
line of sight that are each independently massive enough to be classified as
MPCs, but are only discussed as a single entity because they are likely
to merge if their three-dimensional separation is similar to their projected
distance.  The candidates in W49 are more widely separated, about 4.4 pc and 7
\kms\ along the line of sight, but could still merge.

Additionally, 9 of the \ncandidates\ are within 8.7 kpc, so the mass
estimates are lower limits.  These are promising candidates for follow-up, but
cannot be considered complete for population studies.  If our radius restriction
is dropped to 1.5 pc, the minimum complete distance drops to 5.2 kpc and the
three lowest-mass sources in Table \ref{tab:candidates} no longer qualify, but
otherwise the source list remains unchanged.  Our analysis is therefore robust
to the selection criteria used.
%(except the W51 pair, which meets
%the selection criteria despite its proximity).

% Not really interesting?
% \subsection{Line Widths}
% To back up the claim that these proto-clusters cannot be unbound by ionization
% pressure, we examine the line widths in dense gas tracers.  Using a tracer that
% measures the internal motions  of the gas in the gravitational potential, we
% expect the approximately gaussian line-width to represent the quasi-equilibrium
% state of the gas (i.e., if the line width is changing, it is doing so slowly
% relative to the star formation process).  Because the line widths are much
% larger than the sound speed in the neutral gas, the observed clouds must be
% gravitationally bound, otherwise they would expand and their linewidths would
% drop to the sound speed on a dynamical timescale.
% %\todojohn{This is in keeping with results on GMCs, i.e. that they are in equilibrium.
% %Is there a better way to state it?  Are there other results / theoretical arguments that
% %should be cited?}
% 
% Heterodyne observations of HCO$^+$ 3-2 and N$_2$H$^+$ 3-2 data from
% \citet{Schlingman2011} are presented in Table \ref{tab:candidates}.  With
% critical densities $\gtrsim 2\ee{6}$ \percc, these both trace dense gas and therefore
% are limited to the proto-cluster region.  However, HCO$^+$ has frequently been
% observed in self-absorption, so the N$_2$H$^+$ widths are more reliable.
% We report the FWHM of single-component fits.  In order to mitigate the effects
% of self-absorption on the line fitting, we report HCO$^+$ line widths fitted by
% ignoring the central self-absorbed pixels; the channel selection was done by
% eye.
% All of the candidates selected as proto-massive-cluster candidates have
% $v_{internal} \approx v_{esc} > v_{ionized}$, confirming their candidacy.

%In Table 1 we provide a grading scheme to quantify the quality of the
%candidates. Candidates associated with a grade of {\bf A} will form a $\gtrsim
%10^4$ \msun\ cluster, even if $T_{dust}=40$K and SFE = 30\%, where it's mass is
%reduced from the estimates shown in Table 1 by a factor of 2.38. The two latter
%grades, {\bf B} and {\bf C}, follow the same assumptions and the candidates
%masses will fall down to $\sim 10^4$ \msun\ and $<10^4$ \msun, respectively. 



% \subsection{Nearby Candidates}
% \label{sec:nearcand}
% Because the BGPS is insensitive to large angular scales, we must resort to
% other methods for determining protocluster masses in nearby star-forming
% regions.
% The strongest candidates within 5.8 kpc are M17, NGC 7538, W3, S255, W43, W33, G34.15, and others?  While these
% regions are all known to be forming massive stars and have total gas reservoirs
% with $M>10^5$\msun, their large spatial extents mean that they are all more likely
% to form OB associations than bound clusters.  However, some are likely to be MPs...

%In the 3-6 kpc range, W43, W33

%For example, in the W3 region, no clumps have velocity dispersions
%$\sigma_{FWHM} > 6$ \kms, implying that none can keep ionized gas bound
%\citep{Bieging2011}.

% We searched the BGPS for candidate MPCs in the 1st quadrant ($6^o < \ell <
% 90^o$). Using the Bolocat catalog we marked sources with flux densities in a
% 20\arcsec\ aperture that yield a mass $M_{\rm clump}\geq 3\times 10^{4}$ \msun\
% at a distance of 17.5 kpc or less ($20\arcsec\ = 1.7 $ pc at 17.5 kpc)
% assuming $T_{dust}=20$K. We applied a $M_{\rm clump} \times {\rm SFE}(30\%) >
% 10^4$ \msun\ cutoff at the maximum distance of 17.5 kpc; we are therefore
% complete to a progenitor mass of $3\times10^4$\msun. These criteria led to a
% flux-density cutoff of 3.2 Jy, above which 16 candidates were detected in the
% Bolocat catalog. Distances to these candidates were determined via a
% literature search. The final candidate list is given in Table 1 along with
% their physical properties of measured line widths from N$_2$H+, their
% literature distance, their mass (assuming $T_{dust}=20 K$), and their
% inferred escape speed ($v_{esc} = \sqrt{2 G M / r}$) assuming a radius equal
% to the aperture size at that distance.  The literature search also revealed
% that all of our candidates are known star-forming regions, so our list
% contains no contaminants.  
% 
% With distance determinations to these candidates, we were able to compute
% masses assuming a temperature $T_{dust}=20$K, opacity \citep{Aguirre2010},
% and gas-to-dust ratio of 100 \footnote{$T_{dust}=20$K is more appropriate for
% a typical pre-star-forming clump than an evolved HII-region hosting one
% \citep[e.g.][]{Dunham2009}. However, because we are interested in cold
% progenitors as well as actively forming clusters, we estimate the masses of
% the cluster progenitor candidates using $T_{dust}=20$K or $T_{dust}=40$K to
% reflect whether the environment is cold and quiescent or actively forming
% stars. }
% The mass estimate drops by a factor $2.38$ if the temperature assumed is
% doubled to $T_{dust}=40$K. Additionally, for more nearby sources, larger
% apertures (corresponding to the same physical radius) were used to include
% more source flux. Applying a cutoff of M$_{\rm clump} \times {\rm SFE}(30\%)
% > 10^4$ \msun\ left 2 MPCs out of the 16. All of the 16 flux-cutoff
% candidates are shown in Figure \ref{fig:galplot} providing context on where
% they are in the Galaxy. 
% 
% In Table 1 we provide a grading scheme to quantify the quality of the candidates. Candidates associated with a grade of {\bf A} will form a $\gtrsim 10^4$ \msun\ cluster, even if $T_{dust}=40$K and SFE = 30\%, where it's mass is reduced from the estimates shown in Table 1 by a factor of 2.38. The two latter grades, {\bf B} and {\bf C}, follow the same assumptions and the candidates masses will fall down to $\sim 10^3$ \msun\ and $<10^3$ \msun, respectively. 

%\subsection{IRAS luminosities}
%We can and should derive IRAS luminosities for the candidates (this is trivial)
%and compare them to the IRAS luminosity function in the GP.  We can then
%extrapolate the observations to the southern hemisphere and be TRULY complete.

\section{Results}
% \subsection{Proto-Cluster Mass Function}
% \choppingblock
% In order to measure a mass function, we need better detection statistics than
% are provided by our \nMPC\ MPCs.  We note that our observations are complete to
% $M>5000\msun$ in the range $5.8 < D < 12.4$ kpc \todome{Is the 5000 \msun\
% cutoff used anywhere else?  I don't think so}.  In this range, there are
% \ncomplete\ candidates, which follow a mass function $\alpha=\plaw\pm\plawerr$
% \footnote{Computed using the \citet{Clauset2009} MLE as implemented at
% \url{http://code.google.com/p/agpy/wiki/PowerLaw}}.  If these candidates represent proto-clusters, they should
% follow a Schechter function with a cutoff near $10^4$ \msun\
% \citep{PortegiesZwart2010}.  However, the distribution is more consistent with
% a power law $\alpha=2$ than a Schechter function with a cutoff $M<5\times{10^5}
% \msun$.  Above this cutoff, the Schecter function and power-law are
% indistinguishable for our sample.  \todome{Discuss implications?  Not if
% chopped}



% unnecessary comment 
% The mass function of \emph{all} of our candidates independent of
% mass and distance is consistent with a power-law with $\alpha=1.8\pm0.1$ and
% completeness cutoff 800 \msun, but we don't believe this is a fair description
% of the observations because of the varying sensitivity with distance.

\subsection{Cluster formation rate}
\label{sec:cfr}

The massive clumps in Table \ref{tab:candidates} can be used to constrain the
Galactic formation rate of massive clusters (MCs) above \mmin\ if we assume
that the number of observed proto-clusters is a representative sample. The region
surveyed covers a fraction of the surface area of the Galaxy
$f_{observed}=A_{survey} / A_{Galaxy} \approx \obsfrac\%$ assuming the star
forming disk has a radius of 15 kpc\footnote{The observed fraction of the
galaxy changes to 21\% if we only include the area within the solar
circle as discussed in \S \ref{sec:discussion}.}.
%The fraction observed is also
%about 28\% if we assume the star-forming disk
%truncates at 13.5 instead of 15 kpc \citep{Kennicutt2012}.}.  
The cumulative
cluster formation rate above a cluster mass ${\rm M}_{cl}$ is given by $$CFR(>{\rm M}_{cl})
= \frac{N_{MPC}}{\tau_{SF} f_{observed}}$$ where $ \tau_{SF} \approx 2$\ Myr is
the assumed cluster formation timescale \footnote{$\tau_{SF}$, the time from the start of star formation
until gas expulsion, is a poorly understood
quantity, but is reasonably constrained to be $\gtrsim1$~Myr from the age
spread in the Orion Nebula cluster \citep{Hillenbrand1997} and $\lesssim10$~Myr
because the most massive stars will go supernova by that time.}.
% $v_{esc}$ is the escape speed from radius $R$ and $f_A \sim 0.1$ is the
% projected area filling-factor of dense star-forming gas in the clump.  Dense
% cores may survive for $f_A$ crossing-times before colliding.
%The cumulative
%cluster formation rate is given by
%$$
%CFR (>M_{cl}) = 
%{{f_A N_{MPC}(>M) [SFE ]V_{esc}  } 
%        \over 
% { 2 R f_{observed} }}
%$$
%where $f_A = 0.1$, $V_{esc} $= 10
%\kms , $R = \rcluster$ pc, and  $\tau_{SF} = 2$ Myr.  
With the measured
$N_{MPC}({\rm M}_{\rm cluster}>10^4\msun) = \nMPC $\ proto-clusters, we infer a Galactic formation rate 
$$CFR \lesssim \CFR \left(\frac{\tau_{SF}}{2
~\textrm{Myr}}\right)^{-1} \textrm{~Myr}^{-1}$$  This cluster formation rate is
statistically weak, with Poisson error of about 3.5 
\permyr\ and can be improved with more complete surveys \citep[e.g., Hi-Gal,][]{Molinari2010}.  This
formation rate is an upper limit because all of the estimated
masses are upper limits as discussed in Section \ref{sec:selection}.


\subsection{Comparison to Clusters in Andromeda}
%Comparison to Andromeda or direct measurements should provide a prediction of
%the number of clusters in the largest (two?) mass bins.  Do our observations
%agree with such a prediction?
%
%If YES, the implication is that cluster formation proceeds rapidly and
%forms massive, dense, proto-cluster "cores" before actually forming the
%cluster.

%If NO, cluster formation is SLOW and accretion onto the cluster after the
%initial formation may continue and increase cluster mass by factors $>2$ (less
%than that, it doesn't really matter).  In this case, predicting the cluster
%formation rate from protoclusters requires completeness down to smaller mass - 
%i.e., we need to be able to observe the cluster `seeds' in addition to the dense
%pre-clusters.

We use cluster observations in M31 from \citet{Vansevicius2009} to infer the
massive cluster formation rate in M31.  They observe 2 clusters with
${\rm M}_{\rm cluster}>10^4\msun$ and ages $<10$ Myr in 15\% of the M31 star-forming
disk.  The implied cluster formation rate in Andromeda is $\dot{N_{cl}} =
N_{cl}/0.15 / (10 ~\mathrm{Myr}) \approx 1.3$ \permyr.  Given M31's total star
formation rate $\sim 5\times$ lower than the Galactic rate \citep[Andromeda
$\mdot=0.4$, Milky Way $\mdot=2$ \msun \permyr;][]{Barmby2006,Chomiuk2011}, the
predicted Galactic cluster formation rate is $\dot{N_{cl}}(MW) = 5~
\dot{N_{cl}}(M31) = 6.5$ \permyr \citep[assuming the CFR scales linearly with
the SFR; ][]{Bastian2008}.  
%Given our assumed star formation timescale $\tau_{SF}$, the expected
%present-day number of clusters $N_{cl}(MW) = (6.5 \textrm{~MC~}\permyr) (\tau_{SF}) = 13$
%clusters in the Galaxy.  Using the mass cutoff of $3\times10^4$ \msun, we
%detect \nMPC\ MPCs, implying there are \nMPCtot\ MPCs in the galaxy.  
The scaled-up Andromeda cluster formation rate matches the observed Galactic
cluster formation rate.  The samples are small, but as a sanity check, the
agreement is comforting.

% \choppingblock
% The agreement between the M31-based
% prediction and our observations is (un?) remarkable, considering that Poisson
% statistics alone imply a $>50\%$ uncertainty in each of the cluster counts
% produced above.

% We present a CFR function dependent on the local surface density of gas in a 
% galaxy $\Sigma_{\rm gas}$ (normalized by $\Sigma_0 = Value$), the survival time 
% of the cluster $\tau$, and the initial mass of the cluster, $M_{init}$. We assume 
% that it has a functional form 
% $$ CFR = A \biggr[ {{\tau} \over {t_{10}}} \biggr] ^{\alpha} \biggr[ {{M_{init}} \over {M_3}} \biggr] ^{\beta} \biggr[ {{\Sigma_{\rm gas}} \over {\Sigma_0}} \biggr] ^{\gamma} $$ 
% in units of number of clusters forming per $10^6$ years (= 1 Myr) per square kpc. 
% Here $t_{10}$ is in units fo 10 Myr, and $M_3$ is in units of $10^3$ \msun . From 
% a fit to Galactic and extra-galactic cluster catalogs approximate values are $A = 2$ 
% to $5$ clusters per square kpc per Myr, $\alpha = -1.0$, $\beta = -1.5$, and 
% $\gamma = 1.5$. This implies a one square kpc region around the Sun contains 
% about 20 to 50 short-lived ($<$ 1 Myr) clusters or expanding associations ($\sim3*10^4$ 
% Galaxy wide), and forms only 0.2 to 0.5 open clusters (such as the Pleiades) per kpc$^{-2}$ that last 100 Myr.

% \subsubsection{Comparison to observed Galactic clusters}
% 
% \choppingblock
% 
% % Better to use Piskunov 2008 numbers
% %Given a cluster birth rate of 0.2-0.5 $\permyr \perkpc$ (Battinelli \&
% %Capuzzo-Dolcetta 1991, Piskunov et al. 2006), the expected number of clusters
% %between the CMZ (cut off at a Galactic radius of 1 kpc) and the solar circle at
% %8.5 kpc is 14-36 \permyr. However, the measured clusters in the reported
% %surveys are only $\sim 500 \msun$, while we are interested in more massive ($M
% %> 10^4\msun$) clusters. Assuming a power-law distribution with $\beta=2$ and
% %that the Piskunov sample measures a CFR for clusters in the range $100 < M_C <
% %1000 \msun$, we derive a CFR of $2-5\times10^{-4} \permyr \perkpc$. Assuming
% %these live $\sim 20$ Myr, corresponding to the longest lifetime in Portegies
% %Zwart et al's MC list, the expected number of massive clusters is
% %$\sim0.3-0.8$ within the solar circle. 
% % This number is FAR too low, considering that simply counting MCs gets you at least 7 XXXX 
% 
% 
%  \citet{Piskunov2008} claim a cluster formation rate
% of 0.4 \perkpc \permyr\ integrated over all local ($d<850$ pc) clusters.  Integrated over the Galactic 
% plane, this implies $CFR = 281 \permyr$.  However, the clusters observed in this 
% local sample all have large radii ($r>10$ pc) or small masses ($M<10^3 \msun$), 
% and therefore it is difficult to place constraints on the massive CFR from this data.
% 
% Given our observed cluster counts, there was only a 5\% probability
% of finding an $M>3\times10^4$ \msun\ proto-cluster and 17\% probability of an $M>10^4\msun$
% proto-cluster within the 850 pc completeness zone of \citet{Piskunov2008}.  Multiplying
% by the ratio of a MC / MP lifetime (about an order of magnitude) suggests that
% we were likely to find 1-2 objects with $M>10^4$ \msun\ in the Piskunov sample.
% None were found.  \todome{What are the chances of finding 0 in the local sample given
% the ``measured'' rate?}
% 
% \citet{Piskunov2008} also observe a steepening of the cluster mass function
% with age, from $\alpha\sim1$ to $\alpha\sim2$.  Our observed proto-cluster mass
% function is steep, with $\alpha\sim2$.  This contradiction suggests
% either a selection effect in the \citet{Piskunov2008} sample avoiding low-mass
% young clusters, that our sample \emph{under}estimates the cluster masses particularly
% on the high-mass end, or that low-mass proto-clusters live longer than
% high-mass proto-clusters.  The latter explanation would result in an excess of
% observable low-mass protoclusters compared to the cluster
% mass function (i.e., observable protoclusters should have $\alpha>2$).  It is also expected in theory since, for fixed radius,
% $t_{ff} \propto M^{-1/2}$.

\subsection{Star Formation Activity}

In the sample of potential proto-clusters, all have formed massive stars based
on a literature search and IRAS measurements.  A few of the low mass sources,
G012.209-00.104, G012.627-00.016, G019.474+00.171, and G031.414+00.307 have
relatively low IRAS luminosities ($L_{IRAS} = L_{100}+L_{60} < 10^5 \lsun$) and
little free-free emission.  However, \emph{all} are detected in the radio as
H~II regions (some ultracompact) and have luminosities indicating early-B type
powering stars.
%The lowest IRAS 100 \um\ luminosity in our sample is
%$L_{100}(G19.47)\approx6\times10^3 \lsun$; the rest have $L_{100} >
%2\times10^5 \lsun$.  All of the massive candidates therefore require O-type
%powering stars.

Non-detection of `starless' proto-cluster clumps implies an upper limit on the
starless lifetime. For an assumed $\tau_{sf} \sim 2$~Myr, the $1\sigma$ upper
limit on the starless proto-MC clump is $\tau_{starless} <
(\sqrt{N_{cl}}/N_{cl}) \tau_{sf} = \tsuplim~\mathrm{Myr}$ assuming Poisson
statistics and using all 18 sources.  This limit is consistent with massive
star formation on the clump free-fall timescale ($\tau_{ff}\leq0.65$ Myr).  It
implies that massive stars form rapidly when these large masses are condensed
into cluster-scale regions and hints that massive stars are among the first to
form in massive clusters.
%, also the crossing time for $c_s=10 \kms$ and $r=1.7$ pc).  


%It may indicate that massive stars form simu
%before all of the proto-cluster mass
%has been collected into a compact region, i.e. that collapse from molecular
%cloud to proto-cluster clump proceeds after massive stars have ignited within
%the proto-cluster, although the small number statistics allow for other
%explanations.  

\section{Discussion}
\label{sec:discussion}

Assuming a lower limit 30\% SFE and T$_{dust} = 20 {\rm K}$, \nMPC\ candidates
in Table \ref{tab:candidates}  will become massive clusters like NGC 3603:
G010.472+00.026, W51, and W49 (G043.169+00.01).  Even if  T$_{dust} = 40 {\rm
K}$, W49 is still likely to form a $\sim10^4$ \msun\ MC, although G10.47 would
be too small.  W51, which is within the spatial-filtering incompleteness zone,
passes the cutoff and is likely to form a pair of massive clusters.  However,
if the dust in W51 is warm and the free-free contamination is considered, the
total mass in each of the W51 clumps is below the 3\ee{4} \msun\ cutoff.
% \todocara{Do massive clusters in the galactic center affect this discussion?
% Eli's comment: No, different physics in the GC mean we should not be
% concerned.}
% IGNORED unless the referee says otherwise

The BGPS covers about \obsfrac\% of the Galactic star-forming disk in the range
1 kpc $< R_{gal}<15$ kpc.  We can extrapolate our \nMPC\ detections to predict
that there are $\leq$\nMPCtot\ ($\pm \nMPCtoterr$) proto-clusters in the Galaxy
outside of the Galactic center.
The agreement between the SFR-based prediction
from M31 and our observations implies that we have selected genuine massive
proto-clusters (MPCs).  

These most massive sources have escape speeds greater than the sound speed in
ionized gas, indicating that they can continue to accrete gas even after the
formation of massive stars.  Assuming they are embedded in larger-scale gas
reservoirs, we are measuring lower bounds on the `final' clump plus cluster
mass. 

% Additionally, in the 15\% of the galaxy within the 5.8 kpc
% radius in which the survey is incomplete due to spatial filtering, we predict
% that there should be $2\pm1$ MPCs.  
% it's actually 41%, we probably predict more like a few....

%\subsection{Lifetimes}
%In order to estimate the formation rate from our candidate source counts, we 
%need to include a 

%\begin{figure}
%\includegraphics[width=8.5cm]{figures/mass_vs_omega.pdf}
%\label{fig:m_vs_o}
%\caption{The mass of the young massive proto-clusters (MPCs) versus their respective $\Omega$ value ($V_{esc} / C_{s}$). We present MPC candidates that have $\Omega > 1$ and clump masses greater than $10^3$ \msun. The candidates are graded based on their potential to forming a high mass stellar cluster regarding their assumed temperature and star formation efficiency (SFE). For the grading scheme we assume that $T_{dust}=40$K and a SFE of 30\%, essentially a worst case scenario for cluster forming conditions as the mass of the gas clumps is reduced by 42\%. The candidates are ranked on how much stellar mass they will have after gas dissipation. The red triangles represent grade {\bf A} candidates where their stellar mass will be greater than $10^4$ \msun\. Green squares represent grade {\bf B} candidates where their stellar mass will be greater than $10^3$ \msun. The blue diamonds are grade {\bf C} candidates that will have less the $10^3$ \msun\ in stellar mass.}
%\end{figure}

%this paragraph is somewhat in contradiction to Piskunov2008
% Open clusters generally have masses $<10^4$~\msun\ with a lifetime of $<1$~Gyr
% and their disruption can start early in life from their local environment
% \citep{PortegiesZwart2010}. MCs ($>10^4$~\msun) on the other hand are less
% sensitive to the surrounding environment than open clusters and typically
% remain bound for 1 $<$ t $<$ 10  Gyr \citep{PortegiesZwart2010}.


% look at Piskunov Figure 5 - Cluster Mass Functino

%Assuming that we have indeed acquired a complete sample of MPCs
%over their $\sim2 $ Myr lifetimes, this implies that clusters form within highly
%condensed clumps of gas and dust, rather than slowly accreting from large-scale
%mass reservoirs.  


%What of the locations of proto-clusters?  
All of the young massive proto-clusters candidates observed are within the
solar circle despite our survey covering more area outside of
the solar circle.  
%In other galaxies, e.g. M33, the most massive
%cluster (NGC 604) is found in the outer disk; this situation appears not to
%occur in the Milky Way.  
%Since both shear and density are higher in the inner
%galaxy, it appears that gas density is a more important factor than shear
%forces in determining where massive clusters form within the Galaxy.
%
The outer radius limit for massive cluster formation is consistent with the
observed metallicity shift noted at the same radius by \citet{Lepine2011}.
They identify the solar circle as the corotation radius of pattern speed and
orbits within the Galaxy (within this radius, stars orbit faster than the
spiral pattern).  The fact that this radius also represents a cutoff between
the inner, massive-cluster-forming disk and the outer, massive-cluster-free
disk hints that gas crossing spiral arms may be the triggering mechanism for
massive cluster formation.  However, given the small numbers, the detected
clusters are consistent with a gaussian + exponential disk distribution
following that described by \citet{Wolfire2003}.  
%Outside of 8.5 kpc, the open cluster metallicity is approximately flat out to
%20 kpc.  Inside 8.5 kpc, the metallicity increases.  If gas is prevented from
%mixing at the spiral arm corotation radius, a higher average gas density within
%that radius would lead to increased star and cluster formation.
% Our observations are consistent with \citet{Lepine2011}
%\todojohn{Please help me expand this: Where is NGC 604 in M33 ($R_{gal}$)?
%Are there other (classes of?) galaxies that have clusters in the outer disk?  
%What defines outer disk?}

% EXPLAIN MORE IF YOU INCLUDE (didn't make sense on a second reading)
% % See Portegies-Zwart section 4.4
% The presence of massive clusters exclusively in the inner galaxy ($R_{gal} <
% 8.5 kpc$) is a strong indication that these clusters form from giant molecular
% clouds (although this hypothesis was never really in question) because GMCs
% destroy lower-mass clusters \citep{PortegiesZwart2010}.  The survival time of a
% cluster is proportional to its density and inversely proportional to the GMC
% density.   Statistics on proto-clusters may therefore present a new tool for
% the study of cluster disruption.

%, in which case the BGPS surveys just
%shy of half the potential-cluster-forming region. Then we should see three
%MPCs amongst the candidates we detected. The CFR estimates best agree with the
%conditions that our MPC candidates have T$_{dust} = 20 {\rm K}$ and SFE of
%30\% to 50\% (2 or 4 MPCs). 
%Regarding the lower mass candidates (10$^3$ \msun), we would expect to find $\sim 10$ open cluster progenitors. We have 14 such candidates, which is within a factor of two of the CFR estimate. 


Future work should include a census for MPCs within $D\lesssim5$ kpc using the
Herschel Hi-Gal survey \citep{Molinari2010} and in the Southern plane with
ATLASGAL \citep{Schuller2009}.  Some surveys have already identified
proto-clusters in these regions \citep[e.g.][]{Faundez2004,Battersby2011}, but
they are not complete.  A complete survey of distances will be essential for
continuum surveys to be used.

% Added as per Referee's comments
There are two modes of massive cluster formation consistent with our
observations that can be observationally distinguished.  Either a compact
starless massive proto-cluster phase does occur and is short, or the mass to be
included in the cluster is accumulated from larger volumes over longer
timescales.  Extending our proto-cluster survey to the Southern sky, e.g. using
the ATLASGAL and Hi-Gal surveys, will either discover starless MPCs or
strengthen the arguments that there is no starless MPC phase.  If instead
massive clusters form by large scale ($r>2.5$ pc) accretion, substantial
reservoirs of gas should surround these most massive regions and be flowing
into them.  Signatures of this accretion process should be visible: MPCs should
contain molecular filamentary structures feeding into their centers
\citep[e.g.][]{Correnti2012,Hennemann2012,Liu2012}.  Alternatively, lower mass
clumps may merge to form massive clusters \citep{Fujii2012}, in which case
clusters of clumps - which should be detectable in extant galactic plane
surveys - are the likely precursors to massive clusters.  Finally, massive
clusters may form from the global collapse of structures on scales larger than
we have probed, which could also produce clusters of clumps.


\section{Conclusions}
\label{sec:ympcconclusions}

Using the BGPS, we have performed the first flux-limited census of massive
proto-cluster candidates.  We found \ncandidates\ candidates that will be part
of the next generation of open clusters and \nMPC\ that could form massive
clusters similar to NGC 3603 (${\rm M}_{\rm cluster} > 10^4$ \msun).   We have
measured a Galactic massive cluster formation rate $CFR({\rm M}_{\rm
cluster}>10^4) \lesssim \CFR\  \permyr$\ assuming that clusters are equally
likely to form everywhere within the range 1 kpc $ < R_{gal} < $ 15   kpc. 
%however, we think formation limited to 1<r<8.5; what does this imply?
%(the CFR is the same if we assume, as observed, that all massive cluster formation occurs within the solar circle).  
The observed MPC counts are
consistent with observed cluster counts in Andromeda scaled up by $SFR_{M31} /
SFR_{MW}$ assuming a formation timescale of 2 Myr.  
%A lack of massive clusters
%detected in the local neighborhood \citep[$d_{max}\sim 850$ pc]{Piskunov2008} is also
%consistent with the MPC detection rate and assumed lifetime.

Despite this survey being the first sensitive to pre-star-forming MPC clumps, none
were detected.  This lack of detected pre-star-forming MPCs suggests a
timescale upper limit of about $\tau_{starless}<\tsuplim$ Myr for the pre-massive-star phase of
massive cluster formation, and hints that massive clusters may never form
highly condensed clumps ($\bar{n}\gtrsim10^4~\percc$) prior to forming massive
stars.
It leaves open the possibility that massive clusters form from large-scale
($\gtrsim 10$ pc) accretion onto smaller clumps over a prolonged ($\tau > 2$
Myr) star formation timescale.


Observations are needed to distinguish competing models for MC formation:
Birth from isolated massive proto-cluster clumps, either compact and rapid
or diffuse and slow, or from smaller clumps that
never have a mass as large as the final cluster
mass.  
This sample of the \ncandidates\ most massive proto-cluster clumps in the first
quadrant (where they can be observed by both the VLA and ALMA) presents an ideal
starting point for these observations.

\section{Acknowledgements}
We thank the referee for thorough and very helpful comments that strengthened
this Letter.  This work was supported by NSF grant AST 1009847.


%\bibliography{boundhii}

\begin{table*}
\scriptsize
\begin{center}
\caption{\label{tab:candidates}
Massive Protocluster Candidates detected in the Bolocam Galactic Plane Survey with $M>10^4 \msun$ }
\begin{tabular}{ccccccccccc}
\hline
Name & Common & Distance & M(20K) & M(40K) & $^a$M(min) & Radius & $\bar{n}(H_2)$ & $v_{esc}$ & $^bf_{ff}$ & L(IRAS) \\
 & Name & kpc & $1000 M_{\odot}$ & $1000 M_{\odot}$ & $1000 M_{\odot}$ & pc & $10^4$cm$^{-3}$ & km~s$^{-1}$ &  & $10^5 L_{\odot}$ \\
\hline\hline
G010.472+00.026 & G10.47 & 10.8$^{7}$ & 38 & 16 & 16 & 2.1 & 1.4 & 12.7 & 0.01 & 5.0 \\
G012.209-00.104 & - & 13.5$^{7}$ & 14 & 6 & 5 & 1.3 & 2.3 & 9.9 & 0.05 & 0.61 \\
G012.627-00.016 & - & 12.8$^{9}$ & 10 & 4 & 3 & 2.5 & 0.2 & 5.9 & 0.05 & 0.59 \\
G012.809-00.200 & W33 & 3.6$^{7}$ & 12 & 5 & 3 & 1.0 & 3.8 & 10.2 & 0.32 & 3.0 \\
G019.474+00.171 & - & 14.1$^{12}$ & 11 & 4 & 4 & 1.4 & 1.6 & 8.6 & 0.02 & 0.26 \\
G019.609-00.233 & G19.6 & 12.0$^{7}$ & 26 & 11 & 7 & 2.3 & 0.7 & 10.0 & 0.31 & 6.4 \\
G020.082-00.135 & IR18253 & 12.6$^{10}$ & 13 & 5 & 4 & 2.4 & 0.3 & 6.8 & 0.14 & 2.8 \\
G024.791+00.083 & G24.78 & 7.7$^{11}$ & 14 & 6 & 5 & 2.2 & 0.4 & 7.4 & 0.11 & 1.5 \\
G029.955-00.018 & - & 7.4$^{3}$ & 10 & 4 & 2 & 2.2 & 0.3 & 6.4 & 0.34 & 5.3 \\
G030.704-00.067 & W43b & 5.1$^{6}$ & 11 & 4 & 4 & 1.5 & 1.1 & 8.0 & 0.11 & 1.0 \\
G030.820-00.055 & W43a & 5.1$^{10}$ & 11 & 4 & 4 & 1.5 & 1.2 & 8.1 & 0.13 & 1.9 \\
G031.414+00.307 & G31.41 & 7.9$^{2}$ & 18 & 7 & 7 & 2.3 & 0.5 & 8.3 & 0.05 & 0.8 \\
G032.798+00.193 & G32.80 & 12.9$^{1}$ & 22 & 9 & 7 & 2.5 & 0.5 & 8.9 & 0.27 & 6.9 \\
G034.258+00.154 & G34 & 3.6$^{4}$ & 13 & 5 & 4 & 1.0 & 4 & 10.5 & 0.12 & 2.7 \\
G043.164-00.031 & W49 & 11.4$^{5}$ & 24 & 10 & 6 & 2.2 & 0.7 & 9.7 & 0.38 & 9.9 \\
G043.169+00.009 & W49 & 11.4$^{5}$ & 120 & 52 & 39 & 2.2 & 4 & 22.2 & 0.25 & 16.0 \\
G049.489-00.370 & W51IRS2 & 5.4$^{8}$ & 48 & 20 & 14 & 1.6 & 4.3 & 16.2 & 0.27 & 4.5 \\
G049.489-00.386 & W51MAIN & 5.4$^{8}$ & 52 & 22 & 15 & 1.6 & 4.7 & 17.0 & 0.29 & 4.7 \\
\hline
\end{tabular}
\end{center}{\scriptsize 1: \citet{Araya2002}, 2: \citet{Churchwell1990}, 3: \citet{Fish2003}, 4: \citet{Ginsburg2011}, 5: \citet{Gwinn1992}, 7: \citet{Pandian2008}, 8: \citet{Sato2010}, 9: \citet{Sewilo2004}, 10: \citet{Urquhart2012}, 11: \citet{Vig2008}, 12: \citet{Xu2003}.  6: The distances to G030.704 was determined using the
near kinematic distance from the velocity of the HHT-observed HCO+ line \citep{Schlingman2011}.
$^a$: The minimum likely mass, $M_{min} = (1-f_{ff}) M(40K)$.
$^b$: The fraction of flux from free-free emission (as opposed to dust emission) at $\lambda=1.1$ mm
}
\end{table*}



\ifstandalone
\bibliographystyle{apj_w_etal}  % or "siam", or "alpha", or "abbrv"
%\bibliography{thesis}      % bib database file refs.bib
\bibliography{bibdesk}      % bib database file refs.bib
\fi

\end{document}

% 
% %\input ch2.tex			% file with Chapter 2 contents
% 
% %%%%%%%%%%%%%%%%%%%%%%%%%%%%%%%%%%%%%%%%%%%%%%%%%%%%%%%%%%%%%%%%%%%
% %%%%%%%%%%%%%%%%%%%%%%%  Bibliography %%%%%%%%%%%%%%%%%%%%%%%%%%%%%
% %%%%%%%%%%%%%%%%%%%%%%%%%%%%%%%%%%%%%%%%%%%%%%%%%%%%%%%%%%%%%%%%%%%
% 
% \bibliographystyle{plain}	% or "siam", or "alpha", or "abbrv"
% 				% see other styles (.bst files) in
% 				% $TEXHOME/texmf/bibtex/bst
% 
% \nocite{*}		% list all refs in database, cited or not.
% 
% \bibliography{thesis}		% bib database file refs.bib
% 
% %%%%%%%%%%%%%%%%%%%%%%%%%%%%%%%%%%%%%%%%%%%%%%%%%%%%%%%%%%%%%%%%%%%
% %%%%%%%%%%%%%%%%%%%%%%%%  Appendices %%%%%%%%%%%%%%%%%%%%%%%%%%%%%%
% %%%%%%%%%%%%%%%%%%%%%%%%%%%%%%%%%%%%%%%%%%%%%%%%%%%%%%%%%%%%%%%%%%%
% 
% \appendix	% don't forget this line if you have appendices!
% 
% %\input appA.tex			% file with Appendix A contents
% %\input appB.tex			% file with Appendix B contents
% 
% %%%%%%%%%%%%%%%%%%%%%%%%%%%%%%%%%%%%%%%%%%%%%%%%%%%%%%%%%%%%%%%%%%%
% %%%%%%%%%%%%%%%%%%%%%%%%   THE END   %%%%%%%%%%%%%%%%%%%%%%%%%%%%%%
% %%%%%%%%%%%%%%%%%%%%%%%%%%%%%%%%%%%%%%%%%%%%%%%%%%%%%%%%%%%%%%%%%%%
% 
% \end{document}
% 
% 

\chapter{Bound HII regions and Young Massive Protoclusters}
\label{ch:ympc}



\section{Preface}
During a visit from Eli Bressert, we discussed methods of identifying the
precursors to young massive clusters.  A central idea was that the primary
unbinding energy comes from ionized gas, so that if a region could remain
bound against the pressure provided by ionized gas, it would proceed to
high star formation efficiency.  This notion resulted in two papers: the theory
paper \citep{Bressert2012} and the observational paper \citep{Ginsburg2012a}.
The observational paper, which summarizes the population of proto-YMCs discovered
in the BGPS, is reproduced here.

Since this paper is a Letter, a great deal of the work that went in to this
chapter is hidden in a few short phrases.  In particular, the search for
distances to the candidate source occupied an enormous amount of time and will
be the limiting factor in future searches for candidate proto-clusters.

Since its publication, parts of the proposed follow-up work were carried out by
another group.  At the end of this chapter, I incorporate their new data to
enhance our results and re-measure the Cluster Formation Rate $1 < CFR < 3$
\permyr ($1-\sigma$) more accurately.

\subsection{Abstract}
    
We search the $\lambda=1.1$ mm Bolocam Galactic Plane Survey for clumps
containing sufficient mass to form $\sim10^4~\msun$ star clusters.
%by identifying
%compact ($r\lesssim2.5$ pc) massive ($M_{\rm clump}>10^4$ \msun) dust clumps.  
\ncandidates\ candidate massive proto-clusters  are identified in the first Galactic quadrant outside
of the central kiloparsec.  This
sample is complete to clumps with mass M$_{\rm clump}>\mmin$ and radius
$r\lesssim2.5$ pc.  The overall Galactic massive cluster formation rate is
$CFR({\rm M}_{\rm cluster}>10^4) \lesssim \CFR\  \permyr$, which is in
agreement with the rates inferred from Galactic open clusters and M31 massive
clusters.  We find that all massive proto-clusters in the first quadrant are
actively forming massive stars and place an upper limit of
$\tau_{starless}<\tsuplim$~Myr on the lifetime of the starless phase of massive
cluster formation.  If massive clusters go through a starless phase with all 
of their mass in a single clump, the lifetime of this phase is very short.




\section{Introduction}

The Milky Way contains about 150 Globular clusters (GCs) with masses of $10^4$
to over $10^6$ \msun\ and tens of thousands of open clusters containing from
100 to over $10^4$ stars.  However, young massive clusters containing
$\gtrsim10^4~\msun$ of stars are rare, with only a handful known
\citep{PortegiesZwart2010}. While no GCs have formed in the Milky Way within
the last 5 Gyr, open clusters that survive many crossing times continue to
form.   A few of these clusters have stellar masses greater than $10^4$
M$_{\odot}$ and therefore qualify as young massive clusters
\citep[YMCs;][]{PortegiesZwart2010}.   YMCs must either form from clumps having
masses greater than and sizes comparable to the final cluster  or be formed
from a larger, more diffuse reservoir, in which case massive protocluster
clumps may be rare or nonexistent  \citep{Kennicutt2012}.


%We assume an evolutionary sequence
%from `starless' to `starry' with gas present, after which the gas will be
%ejected, leaving behind a  massive cluster that will live for $\sim100$
%Myr \citep{PortegiesZwart2010}.  We distinguish `massive clusters' from
%lower-mass open clusters with $M<10^4$ \msun\ as they may form by different
%mechanisms.

%\todoeli{Check to make sure this is in agreement with your paper: }
Massive proto-clusters (MPCs) are massive clusters (M$_{\rm cluster}>10^4$ \msun)
in the process of forming from a dense gas cloud.  In \citet{Bressert2012}, we
examine the theoretical properties of MPCs: MPCs are assumed to form from
massive, cold starless clumps analagous to pre-stellar cores
\citep{Williams2000}.  In this paper, we refer to two classes of objects:
starless MPCs, which have very low luminosity and do not contain OB stars, and
MPCs, which are gas-rich but have already formed OB stars.  The only
currently known starless MPC is G0.253+0.016, which lies within the dense
central molecular zone and is subject to greater environmental stresses than
similar objects in the Galactic plane \citep{Longmore2012}.

Because massive clusters contain many massive stars, at some point during their
evolution ionization pressure will prevail over protostellar outflows as the
dominant feedback mechanism.  Other sources of feedback are less than
ionization pressure up until the first supernova explosion
\citep{Bressert2012}.  These proto-clusters must have masses
M$_{\rm clump}>{\rm M}_{*}/SFE$ \footnote{We define a star formation efficiency
$SFE={\rm M}_{\rm *,final} / {\rm M}_{\rm gas,initial}$.}, or about $3\ee{4}$ \msun\ for an assumed
SFE=30\% (an upper limit on the star formation efficiency),
confined in a radius $r\lesssim2.5$ pc, in order to remain bound against
ionization feedback.  These properties motivate our search for proto-clusters
in the Bolocam Galactic Plane Survey \citep[BGPS;][
\url{http://irsa.ipac.caltech.edu/data/BOLOCAM_GPS/}]{Aguirre2011}.


The distinction between relatively short-lived `open clusters' and long-lived
($t\gtrsim1$ Gyr) bound clusters occurs at about $10^4$ \msun
\citep{PortegiesZwart2010}.  Clusters with ${\rm M}_{\rm cluster} < 1\ee{4} \msun$~will
be destroyed by interactions with giant molecular clouds over the course of a
few hundred million years after they have dispersed their gas
\citep{Kruijssen2011}, while clusters with ${\rm M}_{\rm cluster}\gtrsim10^4 \msun$ may
survive $\gtrsim 1$ Gyr.  Closer to the Galactic center, within approximately a
kiloparsec, all clusters will be destroyed on shorter timescales by strong
tidal forces or interactions with molecular clouds.
%Because the lower-mass clusters are destroyed on shorter
%timescales, it is difficult to get a complete census of their population;
%assessing their birth rate may be the best way to determine their overall
%population.

In the Galaxy, there are few known massive clusters.
\citet{PortegiesZwart2010} catalogs a few of them, of which NGC 3603, Trumpler
14, and Westerlund 1 and 2 are the likely descendants of the objects we
investigate.  These clusters have $r_{eff} \lesssim 1$ pc, $M\sim10^4$ \msun,
and ages $t\lesssim4$ Myr.  We present a census of their ancestral analogs.

% Bound open clusters and massive clusters may predominantly form from clumps with
% gravitational escape speeds greater than the sound-speed in photo-ionized gas.
% \S\ 2 uses the BGPS to identify
% candidate dense, massive clumps which may be progenitors to young massive
% clusters. Any clump for which  M$_{\rm clump} \times {\rm SFE}(30\%) > 10^4$
% \msun\ is considered to be a young massive proto-cluster (MPC). \ncandidates\ 
% candidates are sufficiently massive to produce clusters similar to  NGC 3603.
% \S\ 4 discusses these results and \S\ 5 provides concluding remarks.


\section{Observations and Analysis}
\label{sec:ympcobservations}

\subsection{The Bolocam Galactic Plane Survey}
\begin{figure*}
    \includegraphics[width=7in]{{figures_chboundhii/candidates_galacticplot_26.0kpc_10kmsun}.png}
\caption{
\label{fig:galplot}
Plot of the massive proto-cluster (MPC) candidates
overlaid on the Galactic plane.  
%Only sources with $M>1000 \msun$ are included; above
%this cut 
%The symbol size is proportional to the log of the source mass.
The green circle represents the galactic center, and the yellow $\odot$ is the
Sun.
A 15 kpc radius disc centered on the Galactic Center indicates the approximate
extent of Galactic star formation.  The white region indicates the coverage of
the
Bolocam Galactic Plane survey and our source selection limits based on distance
and longitude.  The inner cutoff (light grey) is the nearby incompleteness
limit set by the Bolocam spatial filtering;  the catalog includes sources but
is incomplete in this region.  The red dashed circle traces the solar circle.
Blue filled circles represent initial candidates that passed the mass-cutoff
criterion $M(20K)>\mmin$; red stars are those with $M(20K) > 3\ee{4} \msun$.
In the legend, $M_4$ means mass in units of $10^4 \msun$.  
%Empty squares represent flux-cut candidates that failed the mass cutoff
%criterion.  
%The diamonds are sources from the \citet{Faundez2004} SIMBA survey
%of southern IRAS sources subject to the same cutoffs applied to the Bolocam
%data: filled (green) sources have $M>\mmin$. The central yellow star is the
%Galactic Center. The empty stars are massive clusters
%\citep{PortegiesZwart2010}.
%\todome{Add legend for size-mass}
% Done
}
\end{figure*}

The BGPS is a 1.1 mm survey of the first quadrant of the Galactic plane in the
range $-0.5 < b < 0.5$ with resolution $\sim33\arcsec$ sensitive to a maximum
spatial scale of $\sim120\arcsec$ \citep{Aguirre2011,Ginsburg2012}.  The BGPS `Bolocat' v1.0 catalog
includes sources identified by a watershed decomposition algorithm and flux
measurements within apertures of radius 20\arcsec, 40\arcsec, and 60\arcsec\
\citep{Rosolowsky2010}.

We searched the BGPS for candidate MPCs in the 1st quadrant ($6 < \ell < 90$;
5991 sources).  The inner 6 degrees of the Galaxy are excluded because physical
conditions are significantly different from those in the rest
of the galaxy  \citep{YusefZadeh2009} and the BGPS is confusion-limited in 
that region.

\subsection{Source Selection \& Completeness}
\label{sec:selection}
We identify a flux-limited sample by setting our search criteria to
include all sources with ${\rm M}_{\rm clump}>10^4$ \msun\ in a 20\arcsec\ radius out to 26 
kpc, or a physical radius of 2.5 pc at that distance.  The radius cutoff is
motivated by completeness and physical considerations: the cutoff of 26 kpc includes
the entire star forming disk in our targeted longitudes, and $r=2.5$ pc corresponds
to the radius at which a $3\ee{4}$ \msun\ mass has an escape speed $v_{esc}=10$ \kms, i.e.
ionized gas will be bound. 
The maximum radius and minimum mass imply a minimum mean density
$\bar{n}=6\times10^3~\percc$, which implies a maximum free-fall time $t_{ff}<0.65$~Myr.
%These limitations guide our analysis in Section \ref{sec:discussion}.

Using the Bolocat v1.0 catalog, we first set a flux limit on the sample by assuming
the maximum distance of $d=26$ kpc and imposing a mass cutoff of ${\rm M}_{\rm clump}\geq10^4$ \msun\ 
inside a 20\arcsec\ (2.5 pc) radius aperture.  Following equation 19 in
\citet{Aguirre2011}:
\begin{equation} 
    {\rm M}_{\rm gas}\approx 14.3 \left( e^{13.0/T_d}-1 \right)
        \left({S_\nu\over 1\; {\rm Jy}} \right)
        \left(\frac{D}{{\rm 1~kpc}}\right)^{2} \msun 
\end{equation}   
and assuming $T_{dust}=20$K, the implied flux cutoff is 1.13 Jy \footnote{As
per \citet{Rosolowsky2010}, \citet{Aguirre2011}, and \citet{Ginsburg2012}, a
factor of 1.5 calibration correction and 1.46 aperture correction are required
for the 20\arcsec\ radius aperture fluxes reported in the catalog.  These
factors have been applied to the data. }, above which \nsample\ `flux-cutoff'
candidates were selected in the Bolocat v1.0 catalog.  Cutoffs of 4.3 Jy for
the 40\arcsec\ and 10.2 Jy for the 60\arcsec\ Bolocat v1.0 apertures were used
to select more nearby candidates inside the same physical radius, but no
sources were selected based on these larger apertures.

%By measuring flux within an aperture, we are measuring the mass within a given
%radius, which means that the source may be substantially smaller than we
%assume.  The identified sources may therefore have higher escape velocities
%than the minimum $\sim10~\kms$ required.

% redundant? We applied a $M_{\rm clump} > 10^4$ \msun\ cutoff at
%the maximum distance of 17.5 kpc; we are therefore complete to a progenitor
%mass of $1\times10^4$ \msun.

The BGPS is insensitive to scales larger than 120\arcsec\
\citep[][]{Ginsburg2012}\footnote{\citet{Ginsburg2012} presents v2.0 of the
BGPS}.  As a result, the survey is incomplete below a distance $$D_{min} =
\mindist \left(\frac{r_{cluster}}{2.5 \textrm{pc}}\right) \textrm{kpc} $$ from
the Sun.  Within this radius, alternate methods must be sought to determine the
total mass within $r_{cluster} < \rcluster$ pc.  Although the sample is
incomplete for $D < \mindist$ kpc, sources that have sufficient mass despite
the 120\arcsec\ spatial filtering are included.

%We are
%able to identify some sources within this cutoff distance because they have
%enough mass in a smaller radius, but we are not complete at
%$D<\mindist$ kpc.
%Luckily, a great deal is known
%about nearby proto-clusters, and it is possible to determine the masses of many using
%alternate methods such as near and mid-infrared extinction.  These nearby star-forming
%regions are discussed in Section \ref{sec:nearcand}.

Distances to BGPS-selected candidates were determined primarily via literature
search.  Where distances were unavailable, we used velocity measurements from
\citet{Schlingman2011} and assumed the far distance for source selection.  We
then resolved the kinematic distance ambiguity towards these sources by
searching for associated near-infrared stellar extinction features from the
UKIDSS GPS \citep{Lucas2008}.
%and distances from \citet{EllsworthBowers2012} where
%literature distances were unavailable \footnote{\citet{EllsworthBowers2012} combine radial velocity measurements with 
%a variety of kinematic distance ambiguity resolution methods to measure a distance likelihood function for each source.
%We use the maximum likelihood distance from this method.}. 
Most literature distances were determined using a
rotation curve model and some method of kinematic distance ambiguity
resolution. Because the literature used different rotation curve models, there
is a $\sim10\%$ systematic error in distance resulting in a $\sim20\%$
systematic error in mass. We used the larger
40\arcsec\ radius apertures to determine the flux for sources at
$D<\middistcut$ kpc and 60\arcsec\ radius apertures for sources at $D<\mindist$
kpc (corresponding to $r<\rcluster$ pc).

% In addition, we used the radial velocity measurements from
% \citet{Schlingman2011} and \citet{Shirley2012} to determine a maximum mass for
% each source by assuming the source is at the far kinematic distance. This
% assures our survey's completeness even if the distances acquired from the above
% methods prove incorrect.  Even assuming the far distance for all flux-cutoff
% candidates, there are no additional sources with $M>3\times10^4$ \msun, though 
% there were a handful with $M>10^4 \msun$.


The masses were computed assuming a temperature $T_{dust}=20$K, opacity
$\kappa_{271.1 GHz} = 0.0114~\mathrm{cm}^2 \mathrm{g}^{-1}$, and gas-to-dust
ratio of 100  \citep{Aguirre2011} \footnote{$T_{dust}=20$K is more appropriate
for a typical pre-star-forming clump than an evolved HII-region hosting one
\citep[e.g.]{Dunham2010}. However, because we are interested in cold
progenitors as well as actively forming clusters, the selection is based on
$T_{dust}=20$K, which is more inclusive. }.  The mass estimate drops by a
factor of $2.38$ if the temperature assumed is doubled to $T_{dust}=40$K.  

\citet{Ginsburg2011} notes that significant free-free contamination, as high as
80\%, is possible for some 1.1 mm sources, so the selected candidates may prove
to be more moderate-mass and evolved proto-clusters.  We used the NRAO VLA
Archive Survey \citep[NVAS;][]{Crossley2008} to estimate the free-free
contamination for the sample.  For most sources, the free-free contamination
inferred from the VLA observations is small ($<20\%$), but for a subset the
contamination was $\sim20-35\%$ assuming that the free-free emission is
optically thin.  Corrected masses using the measured free-free contamination
and higher dust temperatures are listed in Table \ref{tab:candidates}; these
are reasonable lower limits on the total mass of these regions.  All of the
contamination estimates are technically lower limits both because of the
assumption that the free-free emission is optically thin and because the VLA
filters out large-scale flux.  However, in most cases, the emission is likely
to be dominated by optically thin emission \citep[evolved HII regions tend to
be optically thin and bright, while compact HII regions are optically thick but
relatively faint;][]{Keto2002} and for most sources VLA C or D-array
observations were used, and at 3.6 and 6 cm the largest angular scale recovered
is 180-300 \arcsec, greater than the largest angular scale in the BGPS.  

Applying a cutoff of M$_{\rm clump} > 10^4$ \msun\ left \ncandidates\
protocluster candidates out of the original \nsample.  The more stringent cut
M$_{\rm clump} > 10^4 / SFE \approx 3\ee{4}$ \msun\ leaves only \nMPC\ MPCs . 
% All of the \nsample\
% `flux-cutoff' candidates with $M>10^3$ \msun\ are shown in \figref{fig:galplot}
% providing context of their location in the Galaxy. 

The final candidate list contains only sources with $M(20K)>10^4 \msun$ (the
completeness limit; see Table \ref{tab:candidates}).  The table lists
their physical properties, their literature distance, their mass (assuming $T_{dust}=20
\textrm{~and~} 40 K$ and a free-free subtracted lower-limit) ,
%\todome{Discuss varying dust opacity?  Martin et al 2011} 
% Ignored.  Unnecessary.
and their inferred escape speed ($v_{esc} = \sqrt{2 G M(20K) / r}$) assuming a
radius equal to the aperture size at that distance.  The table also includes
measurements of the IRAS luminosity in the 60 and 100 \um\ bands within the
source aperture.
%The literature search
%revealed that all candidates are known massive-star-forming regions.

\subsection{Source Separation}
These \ncandidates\ candidates include some overlapping sources.
There are two clumps in W51 separated by about 1.5 pc and 4.5 \kms\ along the
line of sight that are each independently massive enough to be classified as
MPCs, but are only discussed as a single entity because they are likely
to merge if their three-dimensional separation is similar to their projected
distance.  The candidates in W49 are more widely separated, about 4.4 pc and 7
\kms\ along the line of sight, but could still merge.

Additionally, 9 of the \ncandidates\ are within 8.7 kpc, so the mass
estimates are lower limits.  These are promising candidates for follow-up, but
cannot be considered complete for population studies.  If our radius restriction
is dropped to 1.5 pc, the minimum complete distance drops to 5.2 kpc and the
three lowest-mass sources in Table \ref{tab:candidates} no longer qualify, but
otherwise the source list remains unchanged.  Our analysis is therefore robust
to the selection criteria used.
%(except the W51 pair, which meets
%the selection criteria despite its proximity).

% Not really interesting?
% \subsection{Line Widths}
% To back up the claim that these proto-clusters cannot be unbound by ionization
% pressure, we examine the line widths in dense gas tracers.  Using a tracer that
% measures the internal motions  of the gas in the gravitational potential, we
% expect the approximately gaussian line-width to represent the quasi-equilibrium
% state of the gas (i.e., if the line width is changing, it is doing so slowly
% relative to the star formation process).  Because the line widths are much
% larger than the sound speed in the neutral gas, the observed clouds must be
% gravitationally bound, otherwise they would expand and their linewidths would
% drop to the sound speed on a dynamical timescale.
% %\todojohn{This is in keeping with results on GMCs, i.e. that they are in equilibrium.
% %Is there a better way to state it?  Are there other results / theoretical arguments that
% %should be cited?}
% 
% Heterodyne observations of HCO$^+$ 3-2 and N$_2$H$^+$ 3-2 data from
% \citet{Schlingman2011} are presented in Table \ref{tab:candidates}.  With
% critical densities $\gtrsim 2\ee{6}$ \percc, these both trace dense gas and therefore
% are limited to the proto-cluster region.  However, HCO$^+$ has frequently been
% observed in self-absorption, so the N$_2$H$^+$ widths are more reliable.
% We report the FWHM of single-component fits.  In order to mitigate the effects
% of self-absorption on the line fitting, we report HCO$^+$ line widths fitted by
% ignoring the central self-absorbed pixels; the channel selection was done by
% eye.
% All of the candidates selected as proto-massive-cluster candidates have
% $v_{internal} \approx v_{esc} > v_{ionized}$, confirming their candidacy.

%In Table 1 we provide a grading scheme to quantify the quality of the
%candidates. Candidates associated with a grade of {\bf A} will form a $\gtrsim
%10^4$ \msun\ cluster, even if $T_{dust}=40$K and SFE = 30\%, where it's mass is
%reduced from the estimates shown in Table 1 by a factor of 2.38. The two latter
%grades, {\bf B} and {\bf C}, follow the same assumptions and the candidates
%masses will fall down to $\sim 10^4$ \msun\ and $<10^4$ \msun, respectively. 



% \subsection{Nearby Candidates}
% \label{sec:nearcand}
% Because the BGPS is insensitive to large angular scales, we must resort to
% other methods for determining protocluster masses in nearby star-forming
% regions.
% The strongest candidates within 5.8 kpc are M17, NGC 7538, W3, S255, W43, W33, G34.15, and others?  While these
% regions are all known to be forming massive stars and have total gas reservoirs
% with $M>10^5$\msun, their large spatial extents mean that they are all more likely
% to form OB associations than bound clusters.  However, some are likely to be MPs...

%In the 3-6 kpc range, W43, W33

%For example, in the W3 region, no clumps have velocity dispersions
%$\sigma_{FWHM} > 6$ \kms, implying that none can keep ionized gas bound
%\citep{Bieging2011}.

% We searched the BGPS for candidate MPCs in the 1st quadrant ($6^o < \ell <
% 90^o$). Using the Bolocat catalog we marked sources with flux densities in a
% 20\arcsec\ aperture that yield a mass $M_{\rm clump}\geq 3\times 10^{4}$ \msun\
% at a distance of 17.5 kpc or less ($20\arcsec\ = 1.7 $ pc at 17.5 kpc)
% assuming $T_{dust}=20$K. We applied a $M_{\rm clump} \times {\rm SFE}(30\%) >
% 10^4$ \msun\ cutoff at the maximum distance of 17.5 kpc; we are therefore
% complete to a progenitor mass of $3\times10^4$\msun. These criteria led to a
% flux-density cutoff of 3.2 Jy, above which 16 candidates were detected in the
% Bolocat catalog. Distances to these candidates were determined via a
% literature search. The final candidate list is given in Table 1 along with
% their physical properties of measured line widths from N$_2$H+, their
% literature distance, their mass (assuming $T_{dust}=20 K$), and their
% inferred escape speed ($v_{esc} = \sqrt{2 G M / r}$) assuming a radius equal
% to the aperture size at that distance.  The literature search also revealed
% that all of our candidates are known star-forming regions, so our list
% contains no contaminants.  
% 
% With distance determinations to these candidates, we were able to compute
% masses assuming a temperature $T_{dust}=20$K, opacity \citep{Aguirre2010},
% and gas-to-dust ratio of 100 \footnote{$T_{dust}=20$K is more appropriate for
% a typical pre-star-forming clump than an evolved HII-region hosting one
% \citep[e.g.][]{Dunham2009}. However, because we are interested in cold
% progenitors as well as actively forming clusters, we estimate the masses of
% the cluster progenitor candidates using $T_{dust}=20$K or $T_{dust}=40$K to
% reflect whether the environment is cold and quiescent or actively forming
% stars. }
% The mass estimate drops by a factor $2.38$ if the temperature assumed is
% doubled to $T_{dust}=40$K. Additionally, for more nearby sources, larger
% apertures (corresponding to the same physical radius) were used to include
% more source flux. Applying a cutoff of M$_{\rm clump} \times {\rm SFE}(30\%)
% > 10^4$ \msun\ left 2 MPCs out of the 16. All of the 16 flux-cutoff
% candidates are shown in Figure \ref{fig:galplot} providing context on where
% they are in the Galaxy. 
% 
% In Table 1 we provide a grading scheme to quantify the quality of the candidates. Candidates associated with a grade of {\bf A} will form a $\gtrsim 10^4$ \msun\ cluster, even if $T_{dust}=40$K and SFE = 30\%, where it's mass is reduced from the estimates shown in Table 1 by a factor of 2.38. The two latter grades, {\bf B} and {\bf C}, follow the same assumptions and the candidates masses will fall down to $\sim 10^3$ \msun\ and $<10^3$ \msun, respectively. 

%\subsection{IRAS luminosities}
%We can and should derive IRAS luminosities for the candidates (this is trivial)
%and compare them to the IRAS luminosity function in the GP.  We can then
%extrapolate the observations to the southern hemisphere and be TRULY complete.

\section{Results}
% \subsection{Proto-Cluster Mass Function}
% \choppingblock
% In order to measure a mass function, we need better detection statistics than
% are provided by our \nMPC\ MPCs.  We note that our observations are complete to
% $M>5000\msun$ in the range $5.8 < D < 12.4$ kpc \todome{Is the 5000 \msun\
% cutoff used anywhere else?  I don't think so}.  In this range, there are
% \ncomplete\ candidates, which follow a mass function $\alpha=\plaw\pm\plawerr$
% \footnote{Computed using the \citet{Clauset2009} MLE as implemented at
% \url{http://code.google.com/p/agpy/wiki/PowerLaw}}.  If these candidates represent proto-clusters, they should
% follow a Schechter function with a cutoff near $10^4$ \msun\
% \citep{PortegiesZwart2010}.  However, the distribution is more consistent with
% a power law $\alpha=2$ than a Schechter function with a cutoff $M<5\times{10^5}
% \msun$.  Above this cutoff, the Schecter function and power-law are
% indistinguishable for our sample.  \todome{Discuss implications?  Not if
% chopped}



% unnecessary comment 
% The mass function of \emph{all} of our candidates independent of
% mass and distance is consistent with a power-law with $\alpha=1.8\pm0.1$ and
% completeness cutoff 800 \msun, but we don't believe this is a fair description
% of the observations because of the varying sensitivity with distance.

\subsection{Cluster formation rate}
\label{sec:cfr}

The massive clumps in Table \ref{tab:candidates} can be used to constrain the
Galactic formation rate of massive clusters (MCs) above \mmin\ if we assume
that the number of observed proto-clusters is a representative sample. The region
surveyed covers a fraction of the surface area of the Galaxy
$f_{observed}=A_{survey} / A_{Galaxy} \approx \obsfrac\%$ assuming the star
forming disk has a radius of 15 kpc\footnote{The observed fraction of the
galaxy changes to 21\% if we only include the area within the solar
circle as discussed in \S \ref{sec:discussion}.}.
%The fraction observed is also
%about 28\% if we assume the star-forming disk
%truncates at 13.5 instead of 15 kpc \citep{Kennicutt2012}.}.  
The cumulative
cluster formation rate above a cluster mass ${\rm M}_{cl}$ is given by $$CFR(>{\rm M}_{cl})
= \frac{N_{MPC}}{\tau_{SF} f_{observed}}$$ where $ \tau_{SF} \approx 2$\ Myr is
the assumed cluster formation timescale \footnote{$\tau_{SF}$, the time from the start of star formation
until gas expulsion, is a poorly understood
quantity, but is reasonably constrained to be $\gtrsim1$~Myr from the age
spread in the Orion Nebula cluster \citep{Hillenbrand1997} and $\lesssim10$~Myr
because the most massive stars will go supernova by that time.}.
% $v_{esc}$ is the escape speed from radius $R$ and $f_A \sim 0.1$ is the
% projected area filling-factor of dense star-forming gas in the clump.  Dense
% cores may survive for $f_A$ crossing-times before colliding.
%The cumulative
%cluster formation rate is given by
%$$
%CFR (>M_{cl}) = 
%{{f_A N_{MPC}(>M) [SFE ]V_{esc}  } 
%        \over 
% { 2 R f_{observed} }}
%$$
%where $f_A = 0.1$, $V_{esc} $= 10
%\kms , $R = \rcluster$ pc, and  $\tau_{SF} = 2$ Myr.  
With the measured
$N_{MPC}({\rm M}_{\rm cluster}>10^4\msun) = \nMPC $\ proto-clusters, we infer a Galactic formation rate 
$$CFR \lesssim \CFR \left(\frac{\tau_{SF}}{2
~\textrm{Myr}}\right)^{-1} \textrm{~Myr}^{-1}$$  This cluster formation rate is
statistically weak, with Poisson error of about 3.5 
\permyr\ and can be improved with more complete surveys \citep[e.g., Hi-Gal,][]{Molinari2010}.  This
formation rate is an upper limit because all of the estimated
masses are upper limits as discussed in Section \ref{sec:selection}.


\subsection{Comparison to Clusters in Andromeda}
%Comparison to Andromeda or direct measurements should provide a prediction of
%the number of clusters in the largest (two?) mass bins.  Do our observations
%agree with such a prediction?
%
%If YES, the implication is that cluster formation proceeds rapidly and
%forms massive, dense, proto-cluster "cores" before actually forming the
%cluster.

%If NO, cluster formation is SLOW and accretion onto the cluster after the
%initial formation may continue and increase cluster mass by factors $>2$ (less
%than that, it doesn't really matter).  In this case, predicting the cluster
%formation rate from protoclusters requires completeness down to smaller mass - 
%i.e., we need to be able to observe the cluster `seeds' in addition to the dense
%pre-clusters.

We use cluster observations in M31 from \citet{Vansevicius2009} to infer the
massive cluster formation rate in M31.  They observe 2 clusters with
${\rm M}_{\rm cluster}>10^4\msun$ and ages $<10$ Myr in 15\% of the M31 star-forming
disk.  The implied cluster formation rate in Andromeda is $\dot{N_{cl}} =
N_{cl}/0.15 / (10 ~\mathrm{Myr}) \approx 1.3$ \permyr.  Given M31's total star
formation rate $\sim 5\times$ lower than the Galactic rate \citep[Andromeda
$\mdot=0.4$, Milky Way $\mdot=2$ \msun \permyr;][]{Barmby2006,Chomiuk2011}, the
predicted Galactic cluster formation rate is $\dot{N_{cl}}(MW) = 5~
\dot{N_{cl}}(M31) = 6.5$ \permyr \citep[assuming the CFR scales linearly with
the SFR; ][]{Bastian2008}.  
%Given our assumed star formation timescale $\tau_{SF}$, the expected
%present-day number of clusters $N_{cl}(MW) = (6.5 \textrm{~MC~}\permyr) (\tau_{SF}) = 13$
%clusters in the Galaxy.  Using the mass cutoff of $3\times10^4$ \msun, we
%detect \nMPC\ MPCs, implying there are \nMPCtot\ MPCs in the galaxy.  
The scaled-up Andromeda cluster formation rate matches the observed Galactic
cluster formation rate.  The samples are small, but as a sanity check, the
agreement is comforting.

% \choppingblock
% The agreement between the M31-based
% prediction and our observations is (un?) remarkable, considering that Poisson
% statistics alone imply a $>50\%$ uncertainty in each of the cluster counts
% produced above.

% We present a CFR function dependent on the local surface density of gas in a 
% galaxy $\Sigma_{\rm gas}$ (normalized by $\Sigma_0 = Value$), the survival time 
% of the cluster $\tau$, and the initial mass of the cluster, $M_{init}$. We assume 
% that it has a functional form 
% $$ CFR = A \biggr[ {{\tau} \over {t_{10}}} \biggr] ^{\alpha} \biggr[ {{M_{init}} \over {M_3}} \biggr] ^{\beta} \biggr[ {{\Sigma_{\rm gas}} \over {\Sigma_0}} \biggr] ^{\gamma} $$ 
% in units of number of clusters forming per $10^6$ years (= 1 Myr) per square kpc. 
% Here $t_{10}$ is in units fo 10 Myr, and $M_3$ is in units of $10^3$ \msun . From 
% a fit to Galactic and extra-galactic cluster catalogs approximate values are $A = 2$ 
% to $5$ clusters per square kpc per Myr, $\alpha = -1.0$, $\beta = -1.5$, and 
% $\gamma = 1.5$. This implies a one square kpc region around the Sun contains 
% about 20 to 50 short-lived ($<$ 1 Myr) clusters or expanding associations ($\sim3*10^4$ 
% Galaxy wide), and forms only 0.2 to 0.5 open clusters (such as the Pleiades) per kpc$^{-2}$ that last 100 Myr.

% \subsubsection{Comparison to observed Galactic clusters}
% 
% \choppingblock
% 
% % Better to use Piskunov 2008 numbers
% %Given a cluster birth rate of 0.2-0.5 $\permyr \perkpc$ (Battinelli \&
% %Capuzzo-Dolcetta 1991, Piskunov et al. 2006), the expected number of clusters
% %between the CMZ (cut off at a Galactic radius of 1 kpc) and the solar circle at
% %8.5 kpc is 14-36 \permyr. However, the measured clusters in the reported
% %surveys are only $\sim 500 \msun$, while we are interested in more massive ($M
% %> 10^4\msun$) clusters. Assuming a power-law distribution with $\beta=2$ and
% %that the Piskunov sample measures a CFR for clusters in the range $100 < M_C <
% %1000 \msun$, we derive a CFR of $2-5\times10^{-4} \permyr \perkpc$. Assuming
% %these live $\sim 20$ Myr, corresponding to the longest lifetime in Portegies
% %Zwart et al's MC list, the expected number of massive clusters is
% %$\sim0.3-0.8$ within the solar circle. 
% % This number is FAR too low, considering that simply counting MCs gets you at least 7 XXXX 
% 
% 
%  \citet{Piskunov2008} claim a cluster formation rate
% of 0.4 \perkpc \permyr\ integrated over all local ($d<850$ pc) clusters.  Integrated over the Galactic 
% plane, this implies $CFR = 281 \permyr$.  However, the clusters observed in this 
% local sample all have large radii ($r>10$ pc) or small masses ($M<10^3 \msun$), 
% and therefore it is difficult to place constraints on the massive CFR from this data.
% 
% Given our observed cluster counts, there was only a 5\% probability
% of finding an $M>3\times10^4$ \msun\ proto-cluster and 17\% probability of an $M>10^4\msun$
% proto-cluster within the 850 pc completeness zone of \citet{Piskunov2008}.  Multiplying
% by the ratio of a MC / MP lifetime (about an order of magnitude) suggests that
% we were likely to find 1-2 objects with $M>10^4$ \msun\ in the Piskunov sample.
% None were found.  \todome{What are the chances of finding 0 in the local sample given
% the ``measured'' rate?}
% 
% \citet{Piskunov2008} also observe a steepening of the cluster mass function
% with age, from $\alpha\sim1$ to $\alpha\sim2$.  Our observed proto-cluster mass
% function is steep, with $\alpha\sim2$.  This contradiction suggests
% either a selection effect in the \citet{Piskunov2008} sample avoiding low-mass
% young clusters, that our sample \emph{under}estimates the cluster masses particularly
% on the high-mass end, or that low-mass proto-clusters live longer than
% high-mass proto-clusters.  The latter explanation would result in an excess of
% observable low-mass protoclusters compared to the cluster
% mass function (i.e., observable protoclusters should have $\alpha>2$).  It is also expected in theory since, for fixed radius,
% $t_{ff} \propto M^{-1/2}$.

\subsection{Star Formation Activity}

In the sample of potential proto-clusters, all have formed massive stars based
on a literature search and IRAS measurements.  A few of the low mass sources,
G012.209-00.104, G012.627-00.016, G019.474+00.171, and G031.414+00.307 have
relatively low IRAS luminosities ($L_{IRAS} = L_{100}+L_{60} < 10^5 \lsun$) and
little free-free emission.  However, \emph{all} are detected in the radio as
H~II regions (some ultracompact) and have luminosities indicating early-B type
powering stars.
%The lowest IRAS 100 \um\ luminosity in our sample is
%$L_{100}(G19.47)\approx6\times10^3 \lsun$; the rest have $L_{100} >
%2\times10^5 \lsun$.  All of the massive candidates therefore require O-type
%powering stars.

Non-detection of `starless' proto-cluster clumps implies an upper limit on the
starless lifetime. For an assumed $\tau_{sf} \sim 2$~Myr, the $1\sigma$ upper
limit on the starless proto-MC clump is $\tau_{starless} <
(\sqrt{N_{cl}}/N_{cl}) \tau_{sf} = \tsuplim~\mathrm{Myr}$ assuming Poisson
statistics and using all 18 sources.  This limit is consistent with massive
star formation on the clump free-fall timescale ($\tau_{ff}\leq0.65$ Myr).  It
implies that massive stars form rapidly when these large masses are condensed
into cluster-scale regions and hints that massive stars are among the first to
form in massive clusters.
%, also the crossing time for $c_s=10 \kms$ and $r=1.7$ pc).  


%It may indicate that massive stars form simu
%before all of the proto-cluster mass
%has been collected into a compact region, i.e. that collapse from molecular
%cloud to proto-cluster clump proceeds after massive stars have ignited within
%the proto-cluster, although the small number statistics allow for other
%explanations.  

\section{Discussion}
\label{sec:discussion}

Assuming a lower limit 30\% SFE and T$_{dust} = 20 {\rm K}$, \nMPC\ candidates
in Table \ref{tab:candidates}  will become massive clusters like NGC 3603:
G010.472+00.026, W51, and W49 (G043.169+00.01).  Even if  T$_{dust} = 40 {\rm
K}$, W49 is still likely to form a $\sim10^4$ \msun\ MC, although G10.47 would
be too small.  W51, which is within the spatial-filtering incompleteness zone,
passes the cutoff and is likely to form a pair of massive clusters.  However,
if the dust in W51 is warm and the free-free contamination is considered, the
total mass in each of the W51 clumps is below the 3\ee{4} \msun\ cutoff.
% \todocara{Do massive clusters in the galactic center affect this discussion?
% Eli's comment: No, different physics in the GC mean we should not be
% concerned.}
% IGNORED unless the referee says otherwise

The BGPS covers about \obsfrac\% of the Galactic star-forming disk in the range
1 kpc $< R_{gal}<15$ kpc.  We can extrapolate our \nMPC\ detections to predict
that there are $\leq$\nMPCtot\ ($\pm \nMPCtoterr$) proto-clusters in the Galaxy
outside of the Galactic center.
The agreement between the SFR-based prediction
from M31 and our observations implies that we have selected genuine massive
proto-clusters (MPCs).  

These most massive sources have escape speeds greater than the sound speed in
ionized gas, indicating that they can continue to accrete gas even after the
formation of massive stars.  Assuming they are embedded in larger-scale gas
reservoirs, we are measuring lower bounds on the `final' clump plus cluster
mass. 

% Additionally, in the 15\% of the galaxy within the 5.8 kpc
% radius in which the survey is incomplete due to spatial filtering, we predict
% that there should be $2\pm1$ MPCs.  
% it's actually 41%, we probably predict more like a few....

%\subsection{Lifetimes}
%In order to estimate the formation rate from our candidate source counts, we 
%need to include a 

%\begin{figure}
%\includegraphics[width=8.5cm]{figures/mass_vs_omega.pdf}
%\label{fig:m_vs_o}
%\caption{The mass of the young massive proto-clusters (MPCs) versus their respective $\Omega$ value ($V_{esc} / C_{s}$). We present MPC candidates that have $\Omega > 1$ and clump masses greater than $10^3$ \msun. The candidates are graded based on their potential to forming a high mass stellar cluster regarding their assumed temperature and star formation efficiency (SFE). For the grading scheme we assume that $T_{dust}=40$K and a SFE of 30\%, essentially a worst case scenario for cluster forming conditions as the mass of the gas clumps is reduced by 42\%. The candidates are ranked on how much stellar mass they will have after gas dissipation. The red triangles represent grade {\bf A} candidates where their stellar mass will be greater than $10^4$ \msun\. Green squares represent grade {\bf B} candidates where their stellar mass will be greater than $10^3$ \msun. The blue diamonds are grade {\bf C} candidates that will have less the $10^3$ \msun\ in stellar mass.}
%\end{figure}

%this paragraph is somewhat in contradiction to Piskunov2008
% Open clusters generally have masses $<10^4$~\msun\ with a lifetime of $<1$~Gyr
% and their disruption can start early in life from their local environment
% \citep{PortegiesZwart2010}. MCs ($>10^4$~\msun) on the other hand are less
% sensitive to the surrounding environment than open clusters and typically
% remain bound for 1 $<$ t $<$ 10  Gyr \citep{PortegiesZwart2010}.


% look at Piskunov Figure 5 - Cluster Mass Functino

%Assuming that we have indeed acquired a complete sample of MPCs
%over their $\sim2 $ Myr lifetimes, this implies that clusters form within highly
%condensed clumps of gas and dust, rather than slowly accreting from large-scale
%mass reservoirs.  


%What of the locations of proto-clusters?  
All of the young massive proto-clusters candidates observed are within the
solar circle despite our survey covering more area outside of
the solar circle.  
%In other galaxies, e.g. M33, the most massive
%cluster (NGC 604) is found in the outer disk; this situation appears not to
%occur in the Milky Way.  
%Since both shear and density are higher in the inner
%galaxy, it appears that gas density is a more important factor than shear
%forces in determining where massive clusters form within the Galaxy.
%
The outer radius limit for massive cluster formation is consistent with the
observed metallicity shift noted at the same radius by \citet{Lepine2011}.
They identify the solar circle as the corotation radius of pattern speed and
orbits within the Galaxy (within this radius, stars orbit faster than the
spiral pattern).  The fact that this radius also represents a cutoff between
the inner, massive-cluster-forming disk and the outer, massive-cluster-free
disk hints that gas crossing spiral arms may be the triggering mechanism for
massive cluster formation.  However, given the small numbers, the detected
clusters are consistent with a gaussian + exponential disk distribution
following that described by \citet{Wolfire2003}.  
%Outside of 8.5 kpc, the open cluster metallicity is approximately flat out to
%20 kpc.  Inside 8.5 kpc, the metallicity increases.  If gas is prevented from
%mixing at the spiral arm corotation radius, a higher average gas density within
%that radius would lead to increased star and cluster formation.
% Our observations are consistent with \citet{Lepine2011}
%\todojohn{Please help me expand this: Where is NGC 604 in M33 ($R_{gal}$)?
%Are there other (classes of?) galaxies that have clusters in the outer disk?  
%What defines outer disk?}

% EXPLAIN MORE IF YOU INCLUDE (didn't make sense on a second reading)
% % See Portegies-Zwart section 4.4
% The presence of massive clusters exclusively in the inner galaxy ($R_{gal} <
% 8.5 kpc$) is a strong indication that these clusters form from giant molecular
% clouds (although this hypothesis was never really in question) because GMCs
% destroy lower-mass clusters \citep{PortegiesZwart2010}.  The survival time of a
% cluster is proportional to its density and inversely proportional to the GMC
% density.   Statistics on proto-clusters may therefore present a new tool for
% the study of cluster disruption.

%, in which case the BGPS surveys just
%shy of half the potential-cluster-forming region. Then we should see three
%MPCs amongst the candidates we detected. The CFR estimates best agree with the
%conditions that our MPC candidates have T$_{dust} = 20 {\rm K}$ and SFE of
%30\% to 50\% (2 or 4 MPCs). 
%Regarding the lower mass candidates (10$^3$ \msun), we would expect to find $\sim 10$ open cluster progenitors. We have 14 such candidates, which is within a factor of two of the CFR estimate. 


Future work should include a census for MPCs within $D\lesssim5$ kpc using the
Herschel Hi-Gal survey \citep{Molinari2010} and in the Southern plane with
ATLASGAL \citep{Schuller2009}.  Some surveys have already identified
proto-clusters in these regions \citep[e.g.][]{Faundez2004,Battersby2011}, but
they are not complete.  A complete survey of distances will be essential for
continuum surveys to be used.

% Added as per Referee's comments
There are two modes of massive cluster formation consistent with our
observations that can be observationally distinguished.  Either a compact
starless massive proto-cluster phase does occur and is short, or the mass to be
included in the cluster is accumulated from larger volumes over longer
timescales.  Extending our proto-cluster survey to the Southern sky, e.g. using
the ATLASGAL and Hi-Gal surveys, will either discover starless MPCs or
strengthen the arguments that there is no starless MPC phase.  If instead
massive clusters form by large scale ($r>2.5$ pc) accretion, substantial
reservoirs of gas should surround these most massive regions and be flowing
into them.  Signatures of this accretion process should be visible: MPCs should
contain molecular filamentary structures feeding into their centers
\citep[e.g.][]{Correnti2012,Hennemann2012,Liu2012}.  Alternatively, lower mass
clumps may merge to form massive clusters \citep{Fujii2012}, in which case
clusters of clumps - which should be detectable in extant galactic plane
surveys - are the likely precursors to massive clusters.  Finally, massive
clusters may form from the global collapse of structures on scales larger than
we have probed, which could also produce clusters of clumps.


\section{Conclusions}
\label{sec:ympcconclusions}

Using the BGPS, we have performed the first flux-limited census of massive
proto-cluster candidates.  We found \ncandidates\ candidates that will be part
of the next generation of open clusters and \nMPC\ that could form massive
clusters similar to NGC 3603 (${\rm M}_{\rm cluster} > 10^4$ \msun).   We have
measured a Galactic massive cluster formation rate $CFR({\rm M}_{\rm
cluster}>10^4) \lesssim \CFR\  \permyr$\ assuming that clusters are equally
likely to form everywhere within the range 1 kpc $ < R_{gal} < $ 15   kpc. 
%however, we think formation limited to 1<r<8.5; what does this imply?
%(the CFR is the same if we assume, as observed, that all massive cluster formation occurs within the solar circle).  
The observed MPC counts are
consistent with observed cluster counts in Andromeda scaled up by $SFR_{M31} /
SFR_{MW}$ assuming a formation timescale of 2 Myr.  
%A lack of massive clusters
%detected in the local neighborhood \citep[$d_{max}\sim 850$ pc]{Piskunov2008} is also
%consistent with the MPC detection rate and assumed lifetime.

Despite this survey being the first sensitive to pre-star-forming MPC clumps, none
were detected.  This lack of detected pre-star-forming MPCs suggests a
timescale upper limit of about $\tau_{starless}<\tsuplim$ Myr for the pre-massive-star phase of
massive cluster formation, and hints that massive clusters may never form
highly condensed clumps ($\bar{n}\gtrsim10^4~\percc$) prior to forming massive
stars.
It leaves open the possibility that massive clusters form from large-scale
($\gtrsim 10$ pc) accretion onto smaller clumps over a prolonged ($\tau > 2$
Myr) star formation timescale.


Observations are needed to distinguish competing models for MC formation:
Birth from isolated massive proto-cluster clumps, either compact and rapid
or diffuse and slow, or from smaller clumps that
never have a mass as large as the final cluster
mass.  
This sample of the \ncandidates\ most massive proto-cluster clumps in the first
quadrant (where they can be observed by both the VLA and ALMA) presents an ideal
starting point for these observations.

% \section{Acknowledgements}
% We thank the referee for thorough and very helpful comments that strengthened
% this Letter.  This work was supported by NSF grant AST 1009847.


%\bibliography{boundhii}

\begin{table*}
\scriptsize
\begin{center}
\caption{\label{tab:candidates}
Massive Protocluster Candidates detected in the Bolocam Galactic Plane Survey with $M>10^4 \msun$ }
\begin{tabular}{ccccccccccc}
\hline
Name & Common & Distance & M(20K) & M(40K) & $^a$M(min) & Radius & $\bar{n}(H_2)$ & $v_{esc}$ & $^bf_{ff}$ & L(IRAS) \\
 & Name & kpc & $1000 M_{\odot}$ & $1000 M_{\odot}$ & $1000 M_{\odot}$ & pc & $10^4$cm$^{-3}$ & km~s$^{-1}$ &  & $10^5 L_{\odot}$ \\
\hline\hline
G010.472+00.026 & G10.47 & 10.8$^{7}$ & 38 & 16 & 16 & 2.1 & 1.4 & 12.7 & 0.01 & 5.0 \\
G012.209-00.104 & - & 13.5$^{7}$ & 14 & 6 & 5 & 1.3 & 2.3 & 9.9 & 0.05 & 0.61 \\
G012.627-00.016 & - & 12.8$^{9}$ & 10 & 4 & 3 & 2.5 & 0.2 & 5.9 & 0.05 & 0.59 \\
G012.809-00.200 & W33 & 3.6$^{7}$ & 12 & 5 & 3 & 1.0 & 3.8 & 10.2 & 0.32 & 3.0 \\
G019.474+00.171 & - & 14.1$^{12}$ & 11 & 4 & 4 & 1.4 & 1.6 & 8.6 & 0.02 & 0.26 \\
G019.609-00.233 & G19.6 & 12.0$^{7}$ & 26 & 11 & 7 & 2.3 & 0.7 & 10.0 & 0.31 & 6.4 \\
G020.082-00.135 & IR18253 & 12.6$^{10}$ & 13 & 5 & 4 & 2.4 & 0.3 & 6.8 & 0.14 & 2.8 \\
G024.791+00.083 & G24.78 & 7.7$^{11}$ & 14 & 6 & 5 & 2.2 & 0.4 & 7.4 & 0.11 & 1.5 \\
G029.955-00.018 & - & 7.4$^{3}$ & 10 & 4 & 2 & 2.2 & 0.3 & 6.4 & 0.34 & 5.3 \\
G030.704-00.067 & W43b & 5.1$^{6}$ & 11 & 4 & 4 & 1.5 & 1.1 & 8.0 & 0.11 & 1.0 \\
G030.820-00.055 & W43a & 5.1$^{10}$ & 11 & 4 & 4 & 1.5 & 1.2 & 8.1 & 0.13 & 1.9 \\
G031.414+00.307 & G31.41 & 7.9$^{2}$ & 18 & 7 & 7 & 2.3 & 0.5 & 8.3 & 0.05 & 0.8 \\
G032.798+00.193 & G32.80 & 12.9$^{1}$ & 22 & 9 & 7 & 2.5 & 0.5 & 8.9 & 0.27 & 6.9 \\
G034.258+00.154 & G34 & 3.6$^{4}$ & 13 & 5 & 4 & 1.0 & 4 & 10.5 & 0.12 & 2.7 \\
G043.164-00.031 & W49 & 11.4$^{5}$ & 24 & 10 & 6 & 2.2 & 0.7 & 9.7 & 0.38 & 9.9 \\
G043.169+00.009 & W49 & 11.4$^{5}$ & 120 & 52 & 39 & 2.2 & 4 & 22.2 & 0.25 & 16.0 \\
G049.489-00.370 & W51IRS2 & 5.4$^{8}$ & 48 & 20 & 14 & 1.6 & 4.3 & 16.2 & 0.27 & 4.5 \\
G049.489-00.386 & W51MAIN & 5.4$^{8}$ & 52 & 22 & 15 & 1.6 & 4.7 & 17.0 & 0.29 & 4.7 \\
\hline
\end{tabular}
\end{center}{\scriptsize 1: \citet{Araya2002}, 2: \citet{Churchwell1990}, 3: \citet{Fish2003}, 4: \citet{Ginsburg2011}, 5: \citet{Gwinn1992}, 7: \citet{Pandian2008}, 8: \citet{Sato2010}, 9: \citet{Sewilo2004}, 10: \citet{Urquhart2012}, 11: \citet{Vig2008}, 12: \citet{Xu2003}.  6: The distances to G030.704 was determined using the
near kinematic distance from the velocity of the HHT-observed HCO+ line \citep{Schlingman2011}.
$^a$: The minimum likely mass, $M_{min} = (1-f_{ff}) M(40K)$.
$^b$: The fraction of flux from free-free emission (as opposed to dust emission) at $\lambda=1.1$ mm
}
\end{table*}



\section{Follow-up work}
In order to get a complete census of massive proto-clusters in the Galactic
plane, it is necessary to examine the southern plane as well.
\citet{Urquhart2013a} began this examination using ATLASGAL data and identified
6 new candidates in the southern sky.  In principle, this indicates either some
incompleteness in the BGPS or a genuinely higher cluster formation rate in the
southern sky (which, with such small numbers, is easily consistent with uniform
sampling from a disk distribution).

However, we note that two of the candidates in \citet{Urquhart2013a} are
assigned the wrong kinematic distance - they are placed at the far distance
when strong evidence puts them at the near.

\subimport{/Users/adam/work/BoundHII/distance_massiveclumps/}{standalone}

\ifstandalone
\bibliographystyle{apj_w_etal}  % or "siam", or "alpha", or "abbrv"
%\bibliography{thesis}      % bib database file refs.bib
\bibliography{bibdesk}      % bib database file refs.bib
\fi

\end{document}
