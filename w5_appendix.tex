% \appendix
% \onecolumn
\section{W5 Appendix: Optically Thin, LTE dipole molecule}
\label{appendix:dipole}
While many authors have solved the problem of converting CO 1-0 beam
temperatures to \hh\ column densities
\citep{garden1991,bourke1997,Cabrit1990,lada1996}, there are no  examples in
the literature of a full derivation of the LTE, optically thin CO-to-\hh\
conversion process for higher rotational states.  We present the full
derivation here, and quantify the systematic errors generated by various
assumptions.

We begin with the assumption of an optically thin cloud such that the radiative
transfer equation \citep[][eqn 1.9]{rohlfs} simplifies to
\begin{equation}
  \label{eqn:radtrans}
  \frac{dI_\nu}{\ds} = -\kappa_\nu I_\nu 
\end{equation}

The absorption and stimulated emission terms yield 
\begin{equation}
  \label{eqn:kappa}
  \kappa_\nu = \frac{h \nu_{ul} B_{ul} n_u}{c} \varphi(\nu)
              -\frac{h \nu_{ul} B_{lu} n_l}{c} \varphi(\nu)
\end{equation}
where $\varphi(\nu)$ is the line shape function ($\int\varphi(\nu) \dnu \equiv
1$), $n$ is the density in the given state, $\nu$ is the frequency of the transition,
$B$ is the Einstein B coefficient, and $h$ is Planck's constant.

By assuming LTE (the Boltzmann distribution) and using Kirchoff's Law and the definition of 
the Einstein A and B values, we can derive a more useful version of this equation
\begin{equation}
  \kappa_\nu = \frac{c^2}{8 \pi \nu_{ul}^2} n_u A_{ul} \left[\exp\left(\frac{h \nu_{ul} }{k_B T_{ex}}\right) - 1 \right] \varphi(\nu)
\end{equation}
where $k_B$ is Boltzmann's constant.

The observable $T_B$ can be related to the optical depth, which is given by 
\begin{equation}
  \int \tau_\nu \dnu = \frac{c^2}{8 \pi \nu_{ul}^2} A_{ul} \left[\exp\left(\frac{h \nu_{ul} }{k_B T_{ex}}\right) - 1 \right] \int \varphi(\nu) \dnu \int n_u \ds 
\end{equation}

Rearranging and converting from density to column ($\int n \ds = N$) gives an equation for the column density
of the molecule in the upper energy state of the transition:
\begin{equation}
  \label{eqn:nuppertau}
  N_u = \frac{8\pi \nu_{ul}^2}{c^2 A_{ul}} \left[\exp\left(\frac{h \nu_{ul} }{k_B T_{ex}}\right) - 1 \right]^{-1} \int \tau_\nu \dnu
\end{equation}

In order to relate the brightness temperature to the optical depth, at CO transition frequencies the full blackbody
formula must be used and the CMB must also be taken into account.  \citet{rohlfs} equation 15.29 
\begin{equation} 
  \label{eqn:tbrightnesscmb}
  T_B(\nu) = \frac{h \nu}{k_B} \left(\left[e^{h \nu / k_B T_{ex}} - 1\right]^{-1} - \left[e^{h \nu / k_B T_{CMB}} - 1\right]^{-1} \right) (1-e^{-\tau_\nu})
\end{equation}
is rearranged to solve for $\tau_\nu$:
\begin{equation}
  \label{eqn:tau}
  \tau_\nu = -\ln\left[ 1 - \frac{k_B T_B}{h \nu} \left(\left[e^{h \nu / k_B T_{ex}} - 1\right]^{-1} - \left[e^{h \nu / k_B T_{CMB}} - 1\right]^{-1} \right)^{-1} \right]
\end{equation}

We convert from frequency to velocity units with $\dnu = \nu/c \dv$, and plug \eqref{eqn:tau} into \eqref{eqn:nuppertau} to get
\begin{equation}
  \label{eqn:nuppernoapprox}
  N_u = \frac{8\pi \nu_{ul}^3}{c^3 A_{ul}} \left[\exp\left(\frac{h \nu_{ul} }{k_B T_{ex}}\right) - 1 \right]^{-1} \int -\ln\left[ 1 - \frac{k_B T_B}{h \nu_{ul}} \left(\left[e^{h \nu_{ul} / k_B T_{ex}} - 1\right]^{-1} - \left[e^{h \nu_{ul} / k_B T_{CMB}} - 1\right]^{-1} \right)^{-1} \right] \dv
\end{equation}
which is the full LTE upper-level column density with no approximations applied.

The first term of the Taylor expansion is appropriate for $\tau<<1$ ($\ln[1+x]\approx x-\frac{x^2}{2}+\frac{x^3}{3}\ldots$)
\begin{equation}
  N_u = \frac{8\pi \nu_{ul}^3}{c^3 A_{ul}} \left[\exp\left(\frac{h \nu_{ul} }{k_B T_{ex}}\right) - 1 \right]^{-1} \int \frac{k_B T_B}{h \nu_{ul}} \left(\left[e^{h \nu_{ul} / k_B T_{ex}} - 1\right]^{-1} - \left[e^{h \nu_{ul} / k_B T_{CMB}} - 1\right]^{-1} \right)^{-1} \dv
\end{equation}
which simplifies to
\begin{equation}
  \label{eqn:nupper}
  N_u = \frac{8\pi \nu_{ul}^2 k_B}{c^3 A_{ul} h }  \frac{e^{h\nu_{ul}/k_B T_{CMB}} - 1}{e^{h\nu_{ul}/k_B T_{CMB}} - e^{h\nu_{ul}/k_B T_{ex}}} \int T_B  \dv
\end{equation}

This can be converted to use $\mu_e$ \citep[0.1222 for
\twelveco; ][]{Muenter1975}, the electric dipole moment of the molecule, instead
of $A_{ul}$, using \citet{rohlfs} equation 15.20 $\left((A_{ul}=(64\pi^4)/(3 h
c^3)\right)\nu^3 \mu_{e}^2$):
\begin{equation}
  \label{eqn:nuppermuju}
  N_u = \frac{3  }{8 \pi^3 \mu_e^2 } \frac{k_B}{\nu_{ul}} \frac{2 J_u + 1}{J_u} 
    \frac{e^{h\nu_{ul}/k_B T_{cmb}} - 1}{e^{h\nu_{ul}/k_B T_{CMB}} - e^{h\nu_{ul}/k_B T_{ex}}} \int T_B  \dv
\end{equation}

The total column can be derived from the column in the upper state using the partition
function and the Boltzmann distribution
\begin{equation}
  n_{tot}  =        \sum_{J=0}^\infty n_J = n_0 \sum_{J=0}^\infty  (2J+1) \exp\left(-\frac{J(J+1) B_e h}{k_B T_{ex}}\right) \label{eqn:approxpartition}\\
\end{equation}
This equation is frequently approximated using an integral
\citep[e.g.][]{Cabrit1990}, but a more accurate numerical solution using up to
thousands of rotational states is easily computed
\begin{equation}
  n_J = \left[ \sum_{j=0}^{j=j_{max}} (2j+1) \exp\left(-\frac{j(j+1) B_e h}{k_B T_{ex}}\right) \right]^{-1} (2J+1) \exp\left(-\frac{J(J+1) B_e h}{k_B T_{ex}}\right)
\end{equation}
The effects of using the approximation and the full numerical solution are shown in figure \ref{fig:approx}.

%We note that there are a number ($>1$) of different values of $\mu_e$ frequently reported in the literature.
%\citet{Burrus1958} reports a Stark-effect measurement of $\mu_e = 0.112\pm0.005$ Debye.  \citet{Muenter1975}
%report an improvement on this measurement, yielding $\mu_e = 0.1222$.  More recently, \citet{Goorvitch1994}
%report a value for the rotationless dipole momenut $\mu_0 = 0.1101$, which is negligibly different from the 
%\citet{Muenter1975} value...

\Figure{figures_chw5/columnconversion_vs_tex_allapprox}
{The LTE, optically thin conversion factor from $T_B$ (K \kms) to N(\hh)
(\persc) assuming X$_{\twelveco}=10^{-4}$ plotted against $T_{ex}$.  The
dashed line shows the effect of using the integral approximation of the 
partition function \citep[e.g.][]{Cabrit1990}.  It is a better
approximation away from the critical point, and is a better approximation
for higher transitions.  The dotted line shows the effects of removing the 
CMB term from \eqref{eqn:tbrightnesscmb}; the CMB populates the lowest two
excited states, but contributes nearly nothing to the $J=3$ state. Top (blue):
J=1-0, Middle (green): J=2-1, Bottom (red): J=3-2.}
{fig:approx}{0.5}{0}


The CO 3-2 transition is also less likely to be in LTE than the 1-0 transition.
The critical density ($n_{cr}\equiv A_{ul}/C_{ul}$) of \twelveco\ 3-2 is 27
times higher than that for 1-0.  We have run RADEX \citep{VanDerTak2007} LVG
models of CO to examine the impact of sub-thermal excitation on column
derivation.  The results of the RADEX models are shown in Figure
\ref{fig:coradex}.  They illustrate that, while it is quite safe to assume the
CO 1-0 transition is in LTE in most circumstances, a similar assumption is
probably invalid for the CO 3-2 transition in typical molecular cloud
environments.

\Figure{figures_chw5/CO_excitation}
{{\it Top}: The derived N(\hh) as a function of $n_{\hh}$ for $T_{B}=1$ K.
The dashed lines represent the LTE-derived $N(\hh)/T_B$ factor, which has 
no density dependence and, for CO 3-2, only a weak dependence on temperature.
We assume an abundance of \twelveco\ relative to \hh\ $X_{CO} = 10^{-4}$.
{\it Bottom}: The correction factor (N(\hh)$_{RADEX}$ / N(\hh)$_{LTE}$) as
a function of $n_{\hh}$.
For $T_K=20$ K, the ``correction factor'' at $10^3$ \percc\ (typical GMC
mean volume densities) is $\sim15$, while at $10^4$ \percc\ (closer to $n_{crit}$ but
perhaps substantially higher than GMC densities) it becomes negligible.  The
correction factor is also systematically lower for a higher gas kinetic
temperature.
For some densities, the ``correction factor'' dips below 1, particularly for CO
1-0.  This effect is from a slight population inversion due to fast spontaneous
decay rates from the higher levels and has been noted before
\citep[e.g.][]{Goldsmith1972}.
}{fig:coradex}{0.5}{0}

%\bibliography{column_derivation}
